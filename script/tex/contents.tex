\chapter*{《思無邪匯寶》總序}
\addcontentsline{toc}{chapter}{《思無邪匯寶》總序}
\markboth{《思無邪匯寶》總序}{《思無邪匯寶》總序}

很多中國古籍,因戰亂、政治或其他原因,再加上自然的淘汰,消失在歷史的長河裡,其中尤以被視爲俗文學的小說、戲劇、唱本等,流失最多。傳統漢文化輕視這些作品,公私書目中極少著錄,公私藏書亦鮮蒐集,且在元明淸三代,屢遭禁止,毀版焚書,時有所聞。在各類書中,豔情小說,受害最烈。除了官方明令禁毀外,社會輿論、各階層的社會組織,都極力制止此類書籍之流通。但另一方面,社會的需要又使這些書籍不斷出版,地下流行。不過,在強力壓制下,很多書籍都消失了;遺留下來的亦往往因不能公開發行的關係,印刷極不講究,錯漏甚多。硏究中國古代小說的人,往往從禁毀書目或其他批判資料中,知道這類書籍的存在而無從閱讀,偶然接觸到坊間流行本,卻又質量太差,不僅難以窺見原作的眞面目,也難於作爲硏究的對象。

好在自本世紀以來,西方文學思想東漸,使學術界注意到小說的價値,蒐集資料,編寫書目,從事硏究的人愈來愈多。豔情小說作爲小說資料,漸漸被發現、介紹。另一方面,不少在本土消失的書籍,因早年流傳到海外,獲得保存,其中最重要的,就是俗文學書籍,當然有些是豔情小說。但無論是在海內外,這些資料均非一般硏究者所能知,即偶有所知也難得一讀,更不要說作整體硏究了。因此,直到目前還是中國古典小說硏究中一個薄弱環節。

中國小說資料本國保存的不全,閱讀條件亦差,因而硏究者往往需要飄洋過海,去域外讀這些書。前輩學人如鄭振鐸、孫楷第、王重民、劉修業、王古魯等,當代學人如李田意、柳存仁、吳曉鈴、馬幼垣、胡萬川諸位,皆曾專門去東西洋圖書館訪讀中國古代小說。本人多年來從事中國古典小說硏究,有緣至世界各地讀書,深知其間艱苦,故發願將中外所藏明淸善本小說,匯爲一編,影印出版,供硏究者使用。而身在國外,需覓國內合作者。數年前與中國社會科學院文學硏究所劉世德、石昌渝兩位合作,編纂《古本小說叢刊》,原計畫將明淸善本通俗小說盡數收入。然限於出版環境,豔情小說未能廁身其中。因此,將這些小說另行刊印,是必要的。但此類書籍原來印刷多極草率,且以保存未精,常有缺文殘頁;且不少書文字已漶漫,即原書亦不易辨識,影印自難卒讀;若干書版本繁多,亦勢難全部印行。故當務之急,乃出校刊排印本;或據別本校勘,務求通讀;有不同版本者,又需一一校勘,作出定本,方能爲硏究者使用。將來有條件時再擇其中有價値而淸晰者影印。

本叢書計收書五十種,採用版本超過百種,另又收若干附錄。除國內蒐集到部分資料外,大部分資料採自日本、俄羅斯、英國、法國、荷蘭、美國等諸國圖書館及私人藏書。其中如全本《姑妄言》、《海陵佚史》及《龍陽逸史》等,皆爲首次面世者。《海陵佚史》爲《醒世恆言》第二十三回〈金海陵縱慾亡身〉的前身,而比此回篇幅大一倍以上,主要是豔情描寫部分,這對我們了解《三言》的來歷大有幫助。《姑妄言》爲雍正庚戌序抄本,長達百萬言,是繼《金瓶梅》後的一部豔情巨著。此書未見記錄,抄本是從莫斯科的俄羅斯國立圖書館引渡回來的。刊本《僧尼孽海》亦初次應用於校刊中。《肉蒲團》爲常見之書,然流行者多爲日本刊本或此刊本之排印本;偶有用淸刊本者,亦未經詳細校勘。本叢書所收之《肉蒲團》,根據有代表性之刊本抄本六種校勘,除錄入各版本書影外,又收各種揷圖達百幅。各書之出版說明,旣述故事梗槪,說明所用版本情況及校勘各類技術問題外,關於作者及該書硏究之成果,亦加簡述。此叢書所收諸書中,曾於本世紀排印出版者,不超過三分之一。這大槪是歷史上第一次中國豔情小說的大結集,所收的超過現存已知的百分之九十以上。大陸近年來有豔情小說熱,出版了一批所謂明淸豔情小說的書,其中大部分是言情小說,少數算得上豔情小說,但只採用一般坊本,整理粗糙,又隨意刪削,可稱劣本。禁錮不能抑制人性的需要,只能產生粗製濫造的劣品來。淸代豔情小說屢禁不止,只弄得質量每下愈況,目前的情況亦復如此。

豔情小說有文言有白話,有長篇,亦有中篇、短篇。這批資料爲我們展示了明淸豔情小說的全景,亦讓我們看到文言小說發展到白話人情小說的若干線索。大致上文言小說到宋元,已漸衰落。明初瞿佑的《剪燈新話》,以及紛紛出現的仿作,可視爲文言小說的中興。以後一直要待到《聊齋誌異》問世,文言小說又出現新高潮。但有一個過去少爲硏究注意的現象是元明時出現文言中篇小說如《嬌紅記》、《鍾情麗集》、《天緣奇遇》、《花神三妙傳》、《如意君傳》、《癡婆子傳》等,篇幅較過往的文言小說長,能容納更多的故事情節,有較多的細節描寫和對話;文體也由純粹的文言加入白話詞語。這批作品大多爲言情之作,亦有豔情如《如意君傳》、《癡婆子傳》等,時代早的作品,文體較純;後來者則已雜有白話成分了。此類文言中篇,到萬曆年間發展出白話的才子佳人和豔情小說兩大流。由文言而白話的發展過程,在我們這套書中,也是有跡象可尋的。明淸小說硏究上常有繁本簡本出現先後問題的爭論,過去多從進化的觀點來處理這個問題,認爲簡本在前繁本在後。最顯著的例子是馮夢龍將二十回本的《三遂平妖傳》擴充爲四十回本的《新平妖傳》。但硏究者也發現有些小說是繁本在前簡本在後。豔情小說同一部小說出現繁簡本的情況甚多,爲我們提供硏究這一問題非常豐富的資料。其實繁本簡本雖然情節大致相同,但文字則有兩種不同的情況,一是彼此文字大部分相同的,一是彼此文字沒有直接的關係。後者我們還沒有深入硏究,印象中似乎都是簡本在前繁本在後的。前者則可肯定是繁本出現在簡本之前,沒有例外。這或可成爲我們判定明淸白話小說繁簡本出現先後的通則。其原因並不複雜,在著作權沒得到保護的情況下,書商翻刻別家出版的小說時,爲了爭奪市場,或以降低售價,或以增加故事情節去招徠,有時甚至是兩種手段並用。要降價勢必要減低成本,最省事的方法是減少篇幅;保存原來的故事梗槪而刪去細節描寫和詩詞韻文。刪節者爲求省時省力,不會另外寫一本書,而是就手上現成的書加以削減。因而絕大多數的文字是相同的。有的刪節者十分用心,要不是和繁本比較,幾乎找不到刪削的痕跡;也有些甚不用心,使情節斷裂不能通讀。又明淸白話小說情節互相因襲情況嚴重,豔情小說更甚,若能分解構成各書相同或相類的故事情節,觀其流變,對了解各書之形成,彼此間的關係及其價値,也是有幫助的。

中國政治史甚發達而社會生活史料較欠缺,明淸小說是了解當時社會的重要材料。豔情小說除了提供當時一般社會生活史料外,又特別反映了當時的性風俗、性心理等,爲後人硏究此一時期的性文化提供豐富的資料。近年國內外興起性文化硏究熱,注意發掘歷代房中著作,目前除考古所獲得資料外,又從醫部整理出不少房中著述,有人還注意到道家和佛家的有關作品,但對明淸豔情小說中之資料所知無多。本叢書不單保存了一些房中理論,更重要的是記錄了許多此方面的具體例證,展現了明淸兩代多采多姿的性文化活動。在世界豔情文學史中,中國豔情小說有很特出的地位。比較西洋豔情小說,中國小說起源早,類型多,表現出一種比較健康自然的性觀念,和西洋處在宗敎強大壓力下的反抗性的豔情小說,所表現的被扭曲的性觀念大異其趣。日本的豔情小說,則是從翻譯到模仿明淸豔情小說再發展出來的。本叢書的出版,不僅有助於我們硏究中國豔情文學,亦爲世界豔情文學的比較硏究,奠定堅實的基礎。

豔情小說或被稱爲風流小說、猥褻小說、穢褻小說、淫蕩小說等,諸家所指內容各不相同。本叢書所收乃是專以敍寫性愛或以敍寫性愛爲重點之一的小說。《天緣奇遇》、《三妙傳》等文言中篇,《空空幻》、《都是幻》等白話小說,皆以敍寫男歡女愛爲主,因直接的性愛描寫較少,故不選入本叢書中。有些書如《三續金瓶梅》、《梧桐影》等,雖亦未有大量直接的性描寫,然前者爲《金瓶梅》續書,屬《金瓶梅》系列之一;後者爲《肉蒲團》仿作,皆有助於全面了解此類著作,故亦收入。《情海緣》、《歡喜緣》兩書,雖爲民國時期所編,但因雜抄自明淸豔情小說,故亦收入供比較硏究。本叢書亦收入日本漢文豔情小說數種爲附錄,此類資料不多,但亦有參考價値。

西方漢學界較早注意中國性文化硏究,五、六十年代,高羅佩就已寫出《中國古代房內考》及《秘戲圖考》兩書。隨著社會開放,性禁忌被打破,中國性文化亦成漢學家注意的焦點之一,豔情小說自然受到重視。有些豔情小說如《如意君傳》、《癡婆子傳》、《金瓶梅》、《肉蒲團》、《繡榻野史》等,皆有多種外文譯本。若干學者且寫出豔情小說的論文。本叢書的編輯顧問,都是硏究中國古代小說的權威,無論在資料蒐集翻譯和硏究方面,都是出色當行。雷威安的法譯全本《金瓶梅》,韓南的英譯《肉蒲團》都得到漢學界的讃許。但因爲中國豔情小說資料未經全面整理,不論是翻譯還是硏究,都只是使用有限的資料。缺乏好的校勘本,自然不能希望有完好的翻譯本;缺乏全面的資料,亦難苛求出現原創性的論文。這套書將過去被禁毀最慘烈、流散在世界各地的明淸豔情小說,鉅細靡遺,盡數收集,經過校勘整理,匯爲一編,這是硏究明淸豔情小說的堅實基礎。明淸豔情小說之欣賞、翻譯和硏究都可開始了,我們期待著高水準的翻譯和傑出的硏究成果。

\begin{flushright}

陳慶浩

一九九四年八月於台北

\end{flushright}

\chapter*{《思無邪匯寶》編輯凡例}
\addcontentsline{toc}{chapter}{《思無邪匯寶》編輯凡例}
\markboth{《思無邪匯寶》編輯凡例}{《思無邪匯寶》編輯凡例}

一、版本 廣收現存各種版本,了解各版本間之關係,比較其價値,並以最初或最優版本爲底本。

二、文字一般按底本錄入;底本有誤時據別本校改者皆出校記。底本不誤而別本文字可供參考者亦酌量錄入校記。

三、俗體字、簡體字不引起誤會者一般依正體錄入;可能引起誤會者則用校勘符號標出或於校記說明,必要時於出版說明統一交代。

四、本叢書一律以「。」號斷句,原書各種符號皆取消。句讀基本依原書;原書句讀有誤時改正,不一一說明。原書無句讀者補入。

五、校勘符號:

\begin{enumerate}

  \item 原書誤字誤詞應改正者,於該字或該詞後用( )標示。

  \item 原書文字衍出應刪去者,用〈 〉標出。

  \item 原書缺去應補入文字用〔 〕標出。

  \item 原書文字漶漫者逐字以囗號標出。

  \item 原書文字缺去者逐字以〇號標出;數量不明者於校記中說明。

\end{enumerate}

六、書影 所用各版本之扉頁、所有揷圖、首回首頁、印記及其他有助了解版本特徵者皆收;序跋則依其重要性收錄,或全收,或只收首末頁。

七、各書皆有出版說明,述故事梗槪,介紹作者資料,記版本情況及校勘有關事項。

八、附錄 各書後人序跋及有助於了解該書之資料,皆酌量附錄入書中供參考。

\chapter*{《姑妄言》出版說明}
\addcontentsline{toc}{chapter}{《姑妄言》出版說明}
\markboth{《姑妄言》出版說明}{《姑妄言》出版說明}

《姑妄言》首一卷,爲引文;正文二十四卷,卷一回,計二十四回。三韓曹去晶編撰,古營州林鈍翁評。此書〈自序〉署「雍正庚戌中元之次日三韓曹去晶編於獨醒園」,其〈林鈍翁總評〉署「庚戌中元後一日古營州鈍翁書」,是知書成於雍正八年(一七三〇)。作者名其居爲「獨醒園」,蓋取屈原〈漁父〉「衆人皆醉余獨醒」之意。其〈自序〉述書命名之由,謂世人以妄爲眞,以眞爲妄,「余之是書,孰不以爲妄耶,故不得不名之妄言也。」而本書第一回開宗明義,即謂「話說前朝有一奇事,余雖未曾目睹,却係耳聞,說起來諸公也未必肯信。但我姑妄言之,諸公姑妄聽之,消長晝祛睡魔可耳。」此亦爲書名之又一解。

曹去晶生平不詳,彼自署「三韓」人。「三韓」一般爲古代朝鮮南部的馬韓、辰韓、弁韓之總稱,後泛指朝鮮。遼開泰(一〇一二—一〇二〇)中,聖宗伐高麗,以俘戶置高州,又以其中三韓遺民置三韓縣,屬中京道。金屬北京路大定府,址在今之內蒙古赤峰市東。顧炎武《日知錄・外國・三韓》條謂:「今人謂遼東爲三韓者,︙︙原其故,本於天啓初失遼陽以後,奏章之文遂有謂遼人爲三韓者,外之也。今遼人乃以之自稱,夫亦自外也矣。」〈林鈍翁總評〉開首即謂:「予與曹子去晶,雖曰異姓,實同一體,自襁褓至壯迄老,如影之隨形,無呼吸之間相離,生則同生,死則同死之友也。」則鈍翁與去晶爲同地之人,其年齡、經歷亦大抵相似,故可自鈍翁之資料來推測作者生平也。鈍翁自署「古營州」人,按北魏太平眞君五年(四四四)置營州,治所在龍城縣(今遼寧朝陽市),歷代廢置不一,而古營州者,亦指遼東也。書中批語,常將江南與遼東風俗語言作比較,亦可作評者爲遼東人之助證。由此推知曹去晶所署之「三韓」,乃指遼東。有淸以來,常以「三韓」作遼東之代稱。曹寅祖籍遼陽,韓菼《有懷堂文稿》卷六〈織造曹使君壽序〉謂「以余所見,三韓曹使君子淸,乃誠善讀書者」,可爲例證。

或有以爲作者化名評書,亦爲中國古小說所常見,《姑妄言》亦可能如此,否則評者與作者,何能「如影之隨形」?第二十二回回前總批:「鈍翁曰:岳忠武云:『爲將之道,智信仁勇嚴,缺一不可。』誠至言也。余閱此回,方悟尚智諸人命名之由。︙︙」可知批者非作者也。且此類評語,書中也正不少,當可釋然。是知「自襁褓至壯迄老,如影之隨形,無呼吸之間相離」云云,無非示彼此關係之不同尋常也。鈍翁此評旣云書成之時,彼與作者已「迄老」,若以當時爲六十歲計算,則彼等應生於康熙十年(一六七一)左右。《姑妄言》故事背景在南京,全書對南京風光習俗有細緻的描寫,可見作者對南京十分熟悉,非久住者不能爲。首卷總評謂作者「將江寧歷來始末及城中諸景,寫得淸淸白白,曾遊過者一閱,如在目前,固一快事;即未至者,亦可想其風景,不勝神往。」第二十回牧福典妻一節有批註云:「︙︙余親見江寧有一妓卓二官,係揚州人。厥夫酷好嫖而無資,因命其妻接客,得他人之嫖金以作己之嫖資。不知此輩人心腸是何生法!」(頁七〇)第十六回童自大夫妻談家財一節有批註云:「︙︙江南一趙百萬,家私百萬猶有餘。後年將七十,漸漸虧折,僅存十餘萬。逢人即哭道:『我要餓死了,只得十來萬銀子,這日子怎麼過!』彼時余尚年幼,常笑之。︙︙」(頁一七三)第二十三回鍾生、梅生見花子一節批註:「余向在江南內橋過,見兩個乞兒私語。︙︙」(頁二六)按順治二年(一六四五)改南京爲江南省,轄今之上海市、江蘇浙江兩省及江西婺源縣地,治所在江寧。康熙六年(一六六七)分置江蘇、安徽兩省。此書批語所指江南有兩種含意,一是泛指,如一般用法;一則指江南省,亦作爲江寧的代稱。上引三批均表明批者自幼年住在南京,成人後亦在南京,且可推知作者幼年亦曾長期住南京。原籍遼東,康熙年間長期居於南京的曹去晶,與當年遼東大族,三代任江寧織造的曹寅家有關係否?此書和《紅樓夢》有關係否?皆有待硏究。此書評語和脂硯齋《石頭記》的評語口氣頗相近,脂評者是否讀過此書,並受其影響,亦是個値得注意的問題。

本書第一回回前總評謂:「此一部書內,忠臣孝子、友兄恭弟、義夫節婦、烈女貞姑、義士仁人、英雄豪傑、淸官廉吏、文人墨士、商賈匠役、富翁顯宦、劍俠術士、黃冠緇流、仙狐厲鬼、苗蠻獠玀、回回巫人、寡婦孤兒、諂父惡兄、逆子凶弟、良朋損友、幫閒梨園、賭賊閒漢,至於淫僧異道、比丘尼、馬泊六、壞媒人、濫淫婦、孌童妓女、汚吏贓官、凶徒暴客、淫婢惡奴、傭人乞丐、逆璫巨寇,不可屈指。世間所有之人,所有之事,無一不備。余閱稗官小說不下千部,未有如此之全者。勿草率翻過,以負作者之心。」書中正文及批語經常提到或引用善書、戲劇、小說、鼓子詞、唱本、寶卷、吳歌等,可見作者和批者的確熟悉這類作品。其中涉及的小說除《水滸》、《三國》、《金瓶》、《西遊》、《封神》等書外,還有《如意君傳》、《後西遊記》、《燈草和尚》、《鋒劍春秋》等。《鋒劍春秋》現存最早的是同治年間版本,因此一般將他看成乾隆以後的作品。現在有了《姑妄言》的記錄,就可知此書最遲在雍正年間已出現了。第十九回賴盈訴說身體不好,「但用一點力,就傷着了,定要病幾天。」句下有批註:「病魔專凌窮漢,余也受此大累。」可知批者晚年景況不甚如意,故批註常有不平之語。讀本書〈自序〉,其憤世嫉俗之心,溢於言表;〈自序〉之後,又有〈曹去晶自評〉,註明「旣欲看是書,請先閱此評」。謂若讀者「略識數字,以看鼓詞之才學眼力看之,但曰:『好村,好村。』此乃諸公爲腹所負自村耳,非關余書之村也。求其不看爲幸。何故?諸公自恐其汙目,余更恐其汙書」。〈姑妄言首卷〉,署「三韓曹去晶遊戲」,下又註明:「編爲知者道,不共俗人看。」其自負也如此。讀《姑妄言》亦隨處感作者憤激感慨之情。看來作者生平,亦未必暢順,是傷心人別有懷抱,發而成此一奇書也。〈林鈍翁總評〉謂「曹子生平性與予同,愚而且魯,直而且方,不合時宜之蠢物也。」這是我們目前唯一對作者的了解。

《姑妄言》寫萬曆年間,南京閒漢到聽醉臥古城隍廟,見王者判自漢至嘉靖年間十殿閻君所未能解決的歷史疑案,依情理,按其情節輕重,各判再世爲人受報應。其中董賢、曹植、甄氏、武三思、上官婉兒、楊太眞、趙普、嚴世蕃等,生於民家,李林甫生爲阮大鋮,秦檜生爲馬士英,永樂生爲李自成,其相助大臣生爲李氏諸將,因忠於建文爲永樂殺害者如張昺等,則投生爲史可法等一班明末忠臣。是時又有一白氏女及四男子情案,亦判再世以結情緣。此書即以此情案之主角南京瞽女錢貴和書生鍾情之婚姻並宦蕚、賈文物、童自大等四個家庭爲主線開展,旁及其他降世人物,以魏忠賢擅權、崇禎即位殺忠賢、李自成造反入北京,崇禎自吊,福王在南京即位,馬士英、阮大鋮把持朝政謀私利,終至敗亡爲背景。而以明亡,滿淸代興作結。如上引第一回總評所言,此書涉及社會各階層人物,由帝王將相至販夫走卒,無所不有,無所不寫,是一部鴻篇巨著。此書雖只得二十四回,然每回達三、四萬字,全書正文超過九十萬字,批語亦不下五萬字,全書近百萬言,爲中國古本長篇小說中篇幅最大的小說之一。此書〈曹去晶自評〉,自謂著書之意,「無非一片菩提心,勸人向善耳。內中善惡貞淫,各有報應。」而〈林鈍翁總評〉亦謂細閱此書,「乃悟其以淫爲報應,具一片婆心,借種種諸事以說法耳。」此書各回回前總批及書內雙行批註,無時無刻,不提及此一觀念,以警醒讀者焉。此書宣揚報應之觀念,頻頻引用〈太上感應篇〉及〈文昌帝君陰騭文〉等善書,跡近《續金瓶梅》。其中雖寫及種種之報應,而筆力所貫注者則在淫報,故回回寫淫。所寫者有一女多男、一男多女及男女混交、亂倫、男女同性戀,和人獸雜交如人狗交、人驢交、人猴交等。寫採戰法則有採陰補陽,採陽補陰,因採人反被採而致死,仙狐求採人陽精反失丹。寫春宮圖冊、春藥如揭被香、金鎗不倒紫金丹、如意丹等。緬鈴、白綾帶子及角先生等淫具亦時常出現。古代色情小說中之種種套數,種種工具,均出現在此小說中。我們可從中看到《如意君傳》、《繡榻野史》、《金瓶梅》、《癡婆子傳》、《肉蒲團》等小說的影響。《姑妄言》可視爲古本色情小說中集大成之作。《金瓶梅》雖被指爲色情長篇小說之鼻祖,論其內容,實爲社會小說,色情之比例甚微。《金瓶梅》諸續書,亦皆如此。而《姑妄言》實可稱爲眞正性文學長篇。正因作者意在戒淫,遂寫出此一空前絕後之淫書,也是十分詭異的現象。然若非有此「淫之報應」爲武裝,則在十八世紀初,亦難有足夠的勇氣,寫出這樣一部色情小說之巨著吧。

此書之首卷爲〈引文〉,以〈秦淮舊蹟〉介紹故事地點南京之歷史,而以〈瞽妓遺踪〉敍述嘉靖以來當地盛行瞽妓之風俗,如《水滸傳・楔子》以引出全書。此書各回除如一般章回小說以一對聯語爲回目外,又有另一聯語爲附目,此似爲《姑妄言》所獨創,未見於其他古本小說。此實因此本各回字數較任何古本小說爲多,所包含故事內容亦較複雜,非一聯所能槪括,故又加一聯,使回目能更詳盡反映該回內容之故,此亦爲回目中一創新也。

《姑妄言》第四回托言童自宏作《峒谿備錄》,述雲貴諸苗風俗,書中並抄錄其中一大片文字。第十一回寫雲貴之遊,其文字與正文不類。據陳益源硏究,所謂《峒谿備錄》,實爲康熙間陸次雲所作之《峒谿纖志》。而雲貴遊記則雜抄自陳鼎(一六五〇—?)之《滇遊記》、《黔遊記》,許纘曾(一六二七—一七〇〇)之《滇行紀程》、《滇行紀程續抄》、《東還紀程》、《東還紀程續抄》諸書。第三回記烈女杜小英事跡及絕命詞,第七回記高烈女事,第十一回記汪時珍事,皆抄自陳鼎之《留溪外傳》(見〈《姑妄言》素材來源初考〉一文)。按《峒谿纖志》有康熙二十二年(一六八三)刊本。上引諸書中,《留溪外傳》刊刻较遲,有康熙三十七年(一六九八)刊本。可推知《姑妄言》之作,或於十八世紀初。此書有〈林鈍翁總評〉,首卷及各回前亦有總評,例以「鈍翁曰」開頭,故林鈍翁爲此書最主要的批者。鈍翁旣爲作者「生則同生,死則同死之友」,其批語對我們了解作者及此書之創作,自有特別的意義,頗似脂硯齋評語之與《石頭記》。但此書批語涉及作者、批者生平資料甚少,爲遺憾耳。第二卷回前總評記及當日讀者和批者討論之情況,頗有趣味:

\begin{quotation}

鈍翁曰:余一日正評此回書,忽有二三俗客至。一客問余曰:「一部大書,內中無限的人,開首一個就出錢貴,此是何意?」余曰:「如一部傳奇,是誰人事蹟,定是那正生先上場,故此書先出錢貴也。」客曰:「此書雖是錢貴事蹟,然正生當是鍾生,傳奇中,豈有以正旦先上場者乎?」余曰:「不然,此非傳奇,不過借傳奇以做譬喻耳。錢貴猶之正生,鍾生反是正旦角色,故首出錢貴也。」又曰:「錢貴旣是一部書中大有關係之人,定要寫得他高纔是,其父何以名錢爲命?甚不雅觀。」余笑曰:「以錢爲命之人,孝弟忠信、禮義廉恥尚何所知。錢貴旣生於娼家,其父自然是忘八了。此不過信手拈來成趣耳。」座中一人家道素封,頗有愛錢之癖,忿然作色曰:「君語刻毒之甚,豈天下愛錢人盡忘八耶!」余笑解之曰:「非此之謂也,非云愛錢人皆此輩,不過謂此輩中人無有不愛錢者耳。」彼猶含怒而去。前客又問曰:「錢貴旣算正生,係要緊的人了,不但寫他是妓,且又瞽目者何?」余曰:「此別有深意焉,此是作書之人滿腹牢騷,借此以舒憤懣,總見世間之鬚眉男子,只知勢利,惟以富貴評月旦,塵埃中能物色英雄者爲誰?而錢貴以一瞽妓,乃卑汚之極矣,而多少富貴中人他皆不取,獨注意在一貧窮不堪之鍾生,矢心從良,後來竟得全美終身;不過有眼男兒不及一瞽妓女。此是作者一部大主意,須會得此,方許看此書。」

\end{quotation}

此書首卷及各回均有批註,絕大多數不署名者,自可視爲林鈍翁批。其中又有二、三十條批語,以「辱翁曰」起首,當爲辱翁之批。第一回楊玉環自辯通安祿山乃爲壽王雪忿,「不然,這樣三百六十斤的一個大肚皮鬍漢,那被底風流就有限了,有何可樂,有何可愛」句下,有批註兩則。其第二則曰:「余兄辱翁曰:『玉環與此兩人不同,肥而無骨,那怕壓殺。』」(頁八七)此可知辱翁乃鈍翁之兄。又有批者引人話語入評者,如第二回有「王大江先生云」云云一批,即此類也,然數量不多。其中第二十一回述李自成兵攻汴梁,「賊用陰門陣,驅婦女赤身濠邊,望城叫罵;城上點大砲悉倒洩。城上急用陽門陣,令僧人裸立女牆叫罵;賊砲倒洩。」句下有如下之批:

\begin{quotation}

昔明有一帝,見宮內豢豕,謂侍臣曰:「宮闈之中,蓄此何用?」命悉發光祿。後一夜,宮中獲一怪,索豬狗血厭之。而夜深,豬不可得。帝歎曰:「祖宗法自有深意,向之蓄豬,焉知非爲此!所謂寧可備而不用,不可用而不備也。」余曾謂和尚一敎,亦世間可有可無之人,比閱至此,破陰門陣亦大有用處,亦不可少之。然而大有疑焉:男人皆陽具,何故不可破此陣而必用和尚?愚意度之,豈以男子陽物微,不足以敵盛陰,因和尚上下兩光頭,以二陽而破一陰乎?殊不可解,俟高明敎之。一元子曰:「三敎一體,賢愚不一,智者當自悟。作此批者,愚而且蠢,無味。」

\end{quotation}

一元子之語,在整個批語意見一致之下提出批評,且保留下來,甚難得也。《姑妄言》除歷史人物外,其他命名皆有用意,此爲其特色之一,其他古小說人物命名,亦常有意義,惟不及此書之系統全面。部分命名如到聽(道聽塗說)雖甚易知,而又有不少較隱蔽,批語時時提點其含意所在,甚有功於此書也。《姑妄言》書中,有不少笑話,批者也常引用笑話入批語中,粗略統計,不下五十則,亦爲此書批語之一大特色也。

就目前掌握到的資料看來,《姑妄言》寫成後並沒有刊刻,只在小圈子中抄錄流通。淸代文獻中,我們看不到有關此書的任何記載。直至一九四一年,上海優生學會出版了排印殘本《姑妄言》第四十及第四十一回。這大槪是《姑妄言》首次公開出版,但書前標明「會員借觀,不許出售」,只在一個小圈子流通。周越然〈孤本小說十種〉(《大衆》第二期,一九四一年十二月,後收入《書書書》,一九四四,香港圖書供應社,一九六六年影印本)之第六種即談殘抄本《姑妄言》。此爲《姑妄言》首見於公開著錄者。但此書殘卷及介紹文字皆發表於上海孤島時期,不要說一般人看不到,連小說版本目錄專家如孫楷第等都未見,故亦未能引起學術界的注意。一九六六年,李福淸發表了〈中國文學各種目錄補遺〉(《亞非民族》,一九六六年第一期,頁二〇四——二二二),記載蘇聯所藏未見於中國書目的俗文學作品,首提莫斯科列寧圖書館所藏之抄本《姑妄言》,謂「作者爲三韓曹去晶,存二十四卷二十四回,前有一七三〇年序,作者自評及林鈍翁總評(一七三〇)。每頁八行,行二十四字。斯卡奇洛夫(Skachkov)收藏,現存列寧圖書館抄本室,『斯卡奇洛夫藏書』九一九號」(頁二〇五)。此文使我們知道除了上海殘抄本外,又有一個《姑妄言》更完整的本子仍在人間。七、八〇年代我收集《思無邪滙寶》資料時,已讀過優生學會刊本《姑妄言》,知道這是一種重要的資料,曾請本叢書編委中國社科院文硏所劉世德和本叢書主編王秋桂兩位分頭收集。文硏所本已得到列寧圖書館的同意影印了部分稿本,但因蘇聯解體而不了了之。秋桂兄得到李福淸敎授和俄羅斯國立圖書館(原蘇聯列寧圖書館)館長I.S. Filippov敎授的幫助,經過多方週折,終於得到此抄本的微捲。這大槪是《姑妄言》全本首次校點排印。這部佚失了兩百多年的書,終得面世,著者評者,當含笑於九泉。編者特別感謝李福淸和I.S. Filippov兩位敎授的支持和幫助。

如上所述,《姑妄言》現存抄本及殘刊本,今簡介如下:

(一)俄藏抄本(簡稱抄本) 俄國斯卡奇洛夫Skachkov(一八二一—一八八三)於一八四八——一八五九年出差中國時在北京收集到此書。李福淸〈《姑妄言》小說抄本之發現〉謂:「一九七四年莫斯科東方文學出版社出版了A.I. Melnalknis先生編的《Skachkov所藏中國手抄本與地圖書錄》仔細記錄Skachkov收藏的抄本及手繪的地圖、風俗畫三三三種。其中N245著錄《姑妄言》,注明抄本是幾個人抄的,有人寫楷書,有人寫行書。第二卷、第二十一卷有中國收藏家之圖章。用的紙是『仁美和記』和『仁利和記』兩個紙廠的。︙︙」我們現得到的微捲看不到中國收藏者的印章,因而不能了解到抄本的中國原藏者。

Skachkov藏書在俄國收藏情況,可參上引李福淸敎授文章。此抄本後入藏蘇聯莫斯科列寧圖書館,現改稱爲俄國國家圖書館。

抄本共二十四册,計二十四回,第一册首〈自序〉,末署「雍正庚戌中元之次日三韓曹去晶編」。次〈曹去晶自評〉,下註「旣欲看是書,請先閱此評」,末署「書於獨醒園」。下〈姑妄言目錄〉,計引文及二十四回目錄,各回回目例爲兩聯,上聯爲正目,下爲附目。後爲〈林鈍翁總評〉,末署「庚戌中元後一日古營州鈍翁書」。按庚戌爲雍正八年(一七三〇)。接下爲〈姑妄言首卷〉之鈍翁評。次爲正文,首行作「姑妄言首卷」,次行及第三行下署〈三韓曹去晶遊戲〉,下註「編爲知者道,不共俗人看」。第四行低兩格爲「引文」,接下爲引文之目:「秦淮舊蹟,瞽女遺踪」。此可視爲全書之楔子,所謂引文者,爲全書之開篇也,其行款格式一如各回。接下「姑妄言第一卷」之鈍翁評,此爲回前總評,以後各回亦皆如此。評後正文,首行「姑妄言卷之一」,次行低一格書「第一回」,第三、四行低兩格爲回目。回末有「姑妄言第一卷終」一行。自第二册起每册一回,各回格式與第一回大致相同。全書素白紙抄,多人筆跡,多用楷書,一般皆甚工整。然有若干抄手書法幼稚,不依行款,隨意塗鴉者,如第二十三回。第二十一、二十二、二十四回原以行書抄寫,書法美好,又有人再用楷書小字謄抄於旁者。行書、楷書文字間有不同,多數情況爲楷書抄手筆誤,或錯認行書字所致。然亦有行書者錯誤而楷書者改正者。或謄抄時有底本可參校也。此本亦有批註混入正文,正文抄成批註,以及批註抄錄時不規範而錯亂,甚至難以解讀的情形,都是抄錄疏忽造成的,好在數量不多。總體而言,抄錄質量還算不錯。此書正文半葉八行,全書皆無例外,正常情形每行二十四字,然有不少回每行字數不一,在二十二至二十八字之間者。

抄本除個別册於中縫書明葉次外,大部分不寫葉次,各頁書眉左右端,有俄國人以阿拉伯數字後加的頁次。此書原只有回前總批及正文中的雙行批註,但抄本中亦有少量眉批及夾批,就內容及筆跡看來,都是後人加添上的。這些批,有的是改正文字,有的是指明缺葉、錯簡所在,有的是指示抄錄的方式,或表示已作校之類。但也有一些批是就故事或批語發感慨的,我們並不排除其中有漏抄批語,校時再補上去的可能性。抄本偶有缺葉(如第一回、第六回),亦有二、三十處殘破,第十二、十九、二十及二十一册首一葉或數葉破損,第二十一鈍翁之回前總批前面部分已破失。亦有册中或册末葉殘缺者。第八回抄寫草率,有若干處塗汚。此本第一、六、十四、十八、二十三諸册都有錯簡,這是裝釘時不小心造成的。此類情況詳參各回校記。抄本全書二千五百葉左右,雖有若干問題,但大體而言,還是保存完好的。《姑妄言》全書得以相當完整地保存下來,這也是十分幸運的了。此次出版,即以抄本爲底本。由於我們所獲得的是影印件和微捲,原抄本較厚,裝訂時書腦留空少,複製時又不能拆開,因此各書中縫亦有複製不出者。我們曾特別央請白嗣宏敎授就近一一複查過原書,在此亦表示感謝之意。

抄本「玄」字缺末畫,避康熙諱。「萬曆」只一處作「萬歷」,餘皆作「萬曆」,「弘光」亦皆不缺筆,可知不避淸高宗弘曆諱,則其抄寫,當於乾隆前,爲雍正末之抄本乎?

(二)殘刊本 原存一册,爲第四十及第四十一回,第四十回前缺。原藏者誰氏未知,爲周越然藏書歟?今已不知流落何方矣。周越然記錄:「《姑妄言》存四十回,四十一回,四十二回。撰人不詳。淸初素紙精寫本,每半葉九行,每行二十五字。四十二回缺首兩葉。」(〈孤本小說十種〉)刊本封面分三欄,右欄上題「海內孤本姑妄言」,前四字分兩行,「姑妄言」爲大字。中欄作「優生學會逍遙子校」。左欄下方作「會員借觀不許出售」。首〈鄧序〉,謂「︙︙案其事跡,如『借阮大鋮銀子』,『姚澤民造反』,『阮大鋮不知殺死多少大臣』等,似應爲明末淸初著作,與《醒世姻緣》時代,相去或不甚遠。又案《姑妄言》上有刪改二字,疑原有刻本,而經刪改重鈔者。此卷內容爲第四十至四十三(按「三」應作「二」)回,凡三回。︙︙鈔本紙張,大抵爲乾嘉時物,書法亦然,則此當爲乾嘉好事者所爲。」末署「辛未冬,居士山人識」。次爲〈周序〉,謂「細閱四十及四十一兩回,見其文字之美雅,並不在《金瓶梅》之下,︙︙此書著者,決知其爲明末淸初人」云云。末署「民國三十年九月州亞識」。殘刊本雖無署明出版日期,亦無出版者,但可推知出版於一九四一年。〈周序〉後爲正文第四十回,註明「以上原文缺」,自「︙︙他嫁與尋常人家,要選一個做官有錢的佳婿」起,至富新通崔命兒,「他不但慕色,又且感情,時常走來相看」止。第四十一回回目作「司公子漁色破家,崔命兒失丹喪命」,正文自「且說富新正同命兒坐着說話」起,至「這童自大眞是蠢人有蠢福,自從採得這一番之後,精神加倍,面貌生光,大不同往昔」止。下註明「殘篇完」。鄧、周兩氏皆謂抄本三回而刊本只得兩回,當時未全刊出乎?抑有誤記?

比較殘刊本和抄本可發現,殘刊本實爲抄本第十八回刪削修改者。刪去所有的批語、部分韻文和笑話。富新和他的妻妾爲李自成兵所擄,在闖王營種種遭遇至死一段故事,抄本十五、六葉,近六千字,而殘刊本全部刪去,只以數十字草草作結。改動的主要是假道姑通佛姑一段故事,抄本中佛姑將三十歲,刊本作二十四歲。後佛姑有孕,爲其兄藺通發覺,抄本謂藺通將假道姑送官打死,又逼佛姑上吊而死。刊本則謂藺通將假道姑去勢,收爲男妾,佛姑鬱鬱病亡;四、五年後假道姑亦得暴病而亡。殘刊本不及抄本一回,然亦可用以校正抄本之誤字及混入正文之批語,故用作參校本焉。又《姑妄言》素材來源可考者,亦據原書作校。

本書校勘情形,詳見各回校記,或於正文間以校勘符號標示。至於一般通俗小說常見或底本出現次數頻繁之俗體、簡體、異體字,如:「舔」作「㖭」,「雜」作「襍」,「旗」作「旂」,「暫」作「蹔」,「曌」作「瞾」,「站」作「跕」,「拗」作「抝」,「漢」作「汗」,「拎」作「㩕」,「酬」作「酧」,「皺」作「縐」,「鬆」作「松」,「浹」作「夾」,「糖」作「餹」,「癒」作「愈」,「碼」作「馬」,「臀」作「臋」,「踩」作「跴」,「壩」作「垻」,「磕」作「〖足盍〗」,「笨」作「坌」,「擅」作「抎」,「罐」作「礶」,「醃」作「〖鹵奄〗」,「愣」作「楞」,「吵」作「炒」,「餵」作「喂」,「撣」作「担」,「襪」作「韈」,「咿」作「吚」,「嫵」作「娬」,「襤」作「〖衤藍〗」,「哪」作「那」,「佞」作「侫」,「噤」作「禁」,「纓」作「緓」,「槓」作「杠」,「藩」作「籓」,「敞」作「厰」,「衏」作「〖彳完亍〗」,「薑」作「姜」,「媾」作「姤」,「乾」、「杆」作「干」,「掉」、「調」作「弔」,「裝」、「樁」作「粧」,「翻」、「幡」作「番」,「猶」或作「尤」,「村」或作「邨」,「欄」或作「闌」,「凳」或作「橙」,「躇」或作「踷」,「脅」或作「䝱」,「稍」或作「少」,「照」或作「炤」,「勉」或作「免」,「脖」或作「〖孛頁〗」,「稀」或作「希」,「銀」或作「艮」,「芽」或作「牙」,「痛」或作「通」,「樣」或作「檏」,「捨」或作「舍」,「値」或作「直」,「啃」或作「掯」,「褲」或作「庫」,「塞」或作「㩙」,「撂」或作「畧」,「蓮」或作「連」,「懶」或作「嬾」,「怪」或作「〖女圣〗」,「擱」或作「閣」,「竊」或作「切」,「摸」或作「抹」,「梅」或作「枚」,「錠」或作「定」,「摀」或作「侮」,「彎」或作「灣」,「悽」或作「恓」,「餘」或作「余」,「撫」或作「〖扌刍〗」,「恍」或作「愰」,「戴」或作「帶」,「帶」或作「代」,「映」或作「暎」,「挖」或作「穵」,「撐」或作「牚」,「猴」或作「〖犭〗」,「歟」或作「與」,「境」或作「竟」,「覆」或作「復」,「嘟」或作「都」,「蹲」或作「蹾」,「碟」或作「䭟」,「瘋」或作「風」,「坊」或作「方」,「粽」或作「糭」,「敷」或作「傅」,「婚」或作「昏」,「斟」或作「酙」,「薄」或作「萡」,「兩」或作「刄」,「掩」或作「〖扌菴〗」,「擦」或作「搽」,「絲」或作「系」,「揷」或作「扠」,「裡」或作「里」,「鬍」或作「胡」,「鬚」或作「須」,「奔」或作「逩」,「諺」或作「喭」,「園」或作「园」,「瀟」或作「消」,「麵」或作「面」,「纔」或作「儳」,「綁」或作「〖亻邦〗」,「賊」或作「〖月戎〗」,「睏」或作「困」,「數」或作「〖口數〗」,「訴」或作「愬」,「廝」或作「斯」,「瞅」或作「矁」,「整」或作「正」,「腰」或作「要」,「字」或作「子」,「諫」或作「〖亻柬〗」,「廳」或作「聽」,「愧」或作「媿」,「遮」或作「庶」,「忿」或作「分」,「慌」或作「荒」,「慄」或作「栗」,「議」或作「義」,「嘻」或作「喜」或作「唏」,「漿」或作「〖將衣〗」或作「〖將果〗」,「鬟」或作「䯱」或作「环」,「馱」或作「駝」或作「〖馬伏〗」,「蠟」或作「爉」或作「䗶」,「蓬」或作「鬔」或作「逢」,「擋」或作「攩」或作「當」,「一」或作「乙」或作「壹」,「桌」或作「槕」或作「卓」,「閂」或作「〖門丨〗」或作「〖木户睘〗」,「晃」或作「愰」或作「幌」或作「㨪」,「盅」、「忠」或作「中」,「副」、「附」、「咐」或作「付」;與「粱」、「梁」不分,「第」、「弟」不分,「偕」、「諧」不分,「諂」、「謟」不分,「遊」、「游」不分,「址」、「趾」不分,「個」、「各」不分,「效」、「効」不分,「買」、「賣」不分,「曲」、「典」不分,「惑」、「感」不分,「篡」、「纂」不分,「梨」、「黎」不分,「密」、「蜜」不分,「早」、「蚤」不分,「明」、「名」不分,「丘」、「邱」不分,「叉」、「乂」不分,「標」、「嫖」不分,「薄」、「簿」不分,「心」、「必」不分,「厭」、「壓」不分,「縱」、「總」不分,「刺」、「剌」不分,「駐」、「住」不分,「義」、「意」不分,「奄」、「淹」不分,「態」、「熊」不分,「鮮」、「解」不分,「思」、「恩」不分,「哈」、「恰」不分,「敝」、「敞」不分,「傅」、「傳」不分,「說」、「話」不分,「佯」、「徉」不分,「浪」、「娘」不分,「瓜」、「爪」不分,「斑」、「班」不分,「淨」、「靜」不分,「歪」、「盃」不分,「候」、「侯」不分,「慕」、「暮」不分,「煩」、「繁」不分,「箱」、「廂」不分,「梢」、「稍」不分,「俏」、「悄」不分,「淸」、「靑」不分,「慨」、「槪」不分,「防」、「妨」不分,「磬」、「罄」不分,「椿」、「樁」不分,「耍」、「要」不分,「練」、「煉」不分,「屎」、「尿」不分,「襠」、「檔」不分,「衣」、「木」不分,「瞞」、「滿」不分,「分」、「兮」不分,「盼」、「盻」不分,「辛」、「幸」不分,「烏」、「鳥」不分,「嗚」、「鳴」不分,「肓」、「盲」不分,「搬」、「撇」不分,「謬」、「繆」不分,「調」、「綢」不分,「容」、「客」不分,「惱」、「腦」不分,「訴」、「訢」不分,「拆」、「折」不分,「哄」、「烘」不分,「享」、「亨」不分,「僕」、「樸」不分,「慢」、「漫」不分,「那」、「都」不分,「仗」、「伏」不分,「替」、「賛」不分,「怒」、「恕」不分,「卷」、「捲」不分,「璧」、「壁」不分,「趨」、「趣」不分,「著」、「箸」不分,「藉」、「籍」不分,「座」、「坐」不分,「雎」、「睢」不分,「臾」、「叟」不分,「庾」、「廋」不分,「諛」、「謏」不分,「極」、「急」不分,「姻」、「烟」不分,「棵」、「顆」不分,「聚」、「衆」不分,「蠟」、「臘」不分,「但」、「俱」不分,「雨」、「兩」不分,「沐」、「沭」不分,「叫」、「敎」不分,「娑」、「婆」不分,「枕」、「忱」不分,「賭」、「睹」不分,「秣」、「抹」不分,「枯」、「祐」不分,「眠」、「眼」不分,「玉」、「王」不分,「畜」、「蓄」不分,「峻」、「竣」不分,「不」、「下」不分,「群」、「郡」不分,「肋」、「助」不分,「筋」、「筯」不分,「難」、「雖」不分,「妬」、「姤」不分,「件」、「伴」不分,「慚」、「漸」不分,「壤」、「壞」不分,「睽」、「暌」不分,「睛」、「晴」不分,「珮」、「佩」不分,「栽」、「裁」不分,「攏」、「籠」不分,「虬」、「乱」不分,「帕」、「怕」不分,「入」、「人」不分,「租」、「祖」不分,「卿」、「鄕」不分,「苦」、「若」不分,「斜」、「邪」不分,「繫」、「擊」不分,「褌」、「褲」不分,「酥」、「蘇」不分,「炙」、「灸」不分,「啐」、「碎」不分,「去」、「玄」不分,「腎」、「賢」不分,「找」、「我」不分,「響」、「嚮」不分,「釀」、「饟」不分,「伽」、「茄」不分,「癱」、「攤」不分,「髓」、「隨」不分,「廷」、「庭」不分,「進」、「近」不分,「悽」、「棲」不分,「烝」、「蒸」不分,「透」、「逗」不分,「宵」、「霄」不分,「趺」、「跌」不分,「槁」、「稿」不分,「徙」、「徒」不分,「昝」、「咎」不分,「箇」、「筒」不分,「菅」、「管」不分,「褪」、「腿」不分,「招」、「抬」不分,「頂」、「項」不分,「與」、「興」不分,「取」、「娶」不分,「漱」、「嗽」不分,「嬉」、「嘻」不分,「嚇」、「赫」不分,「罔」、「岡」不分,「網」、「綱」不分,「貪」、「貧」不分,「微」、「徵」不分,「惟」、「帷」不分,「紳」、「伸」不分,「渡」、「度」不分,「朽」、「杇」不分,「形」、「刑」不分,「狀」、「壯」不分,「叛」、「判」不分,「休」、「体」不分,「妾」、「妄」不分,「栗」、「粟」不分,「拼」、「併」不分,「紗」、「沙」不分,「礲」、「瓏」不分,「貼」、「帖」不分,「冠」、「寇」不分,「眨」、「貶」不分,「銷」、「鎖」不分,「跨」、「胯」不分,「刀」、「力」不分,「策」、「榮」不分,「壺」、「壼」不分,「和」、「合」不分,「蔑」、「篾」不分,「受」、「愛」不分,「盡」、「書」不分,「甌」、「毆」不分,「河」、「何」不分,「做」、「作」不分,「隊」、「墜」不分,「投」、「頭」不分,「兢」、「競」不分,「驚」、「警」不分,「負」、「員」不分,「惦」、「墊」不分,「惜」、「借」不分,「偏」、「徧」不分,「款」、「疑」不分,「差」、「羞」不分,「淌」、「倘」不分,「尸」、「戶」不分,「垣」、「坦」不分,「宴」、「晏」不分,「怪」、「快」不分,「趁」、「稱」不分,「鍾」、「鐘」不分,「誓」、「哲」不分,「惺」、「猩」不分,「顧」、「僱」不分,「重」、「從」不分,「勢」、「式」不分,「釭」、「缸」不分,「焉」、「馬」不分,「觀」、「覷」不分,「的」、「約」不分,「性」、「姓」不分,「嵌」、「篏」不分,「反」、「返」不分,「辰」、「晨」不分,「用」、「同」不分,「扣」、「叩」不分,「繃」、「綳」不分,「艾」、「哎」不分,「矯」、「嬌」不分,「史」、「吏」不分,「葷」、「暈」不分,「隲」、「〖阝少日小〗」不分,「勵」、「厲」不分,「紮」、「箚」不分,「耀」、「躍」不分,「氓」、「民」不分,「堤」、「提」不分,「績」、「蹟」、「跡」不分,「代」、「待」、「侍」不分,「設」、「没」、「役」不分,「抗」、「扛」、「杠」不分,「妄」、「忘」、「望」不分,「己」、「已」、「巳」不分,「已」、「以」、「矣」不分,「祿」、「綠」、「緣」不分,「德」、「得」、「的」不分,「啓」、「起」、「豈」不分,「倒」、「到」、「道」不分,「博」、「搏」、「摶」不分,「豪」、「毫」、「亳」不分,「直」、「眞」、「貞」不分,「贏」、「嬴」、「羸」不分,「倍」、「陪」、「賠」不分,「麈」、「塵」、「麀」不分,「卵」、「卯」、「卬」不分,「兒」、「尔」、「而」不分,「攙」、「纔」、「讒」不分,「冒」、「胃」、「胄」不分,「機」、「幾」、「畿」不分,「獎」、「漿」、「槳」不分,「困」、「因」、「固」不分,「丈」、「大」、「太」不分,「撒」、「撤」、「徹」不分,「臊」、「燥」、「躁」不分,「令」、「今」、「金」不分,「髒」、「臟」、「贓」不分,「熱」、「熟」、「孰」不分,「也」、「他」、「地」不分,「底」、「低」、「抵」不分,「避」、「僻」、「辟」不分,「向」、「响」、「晌」不分,「問」、「門」、「們」不分,「乎」、「呼」、「吁」不分,「看」、「着」、「者」不分,「土」、「士」、「仕」不分,「吝」、「各」、「名」不分,「麼」、「磨」、「摩」不分,「貝」、「具」、「其」不分,「過」、「遇」、「愚」不分,「腹」、「復」、「服」不分,「輩」、「背」、「肯」不分,「然」、「燃」、「撚」不分,「衲」、「納」、「呐」不分,「護」、「獲」、「穫」不分,「工」、「功」、「攻」不分,「存」、「在」、「再」不分,「豎」、「堅」、「監」不分,「宮」、「官」、「宦」不分,「歹」、「反」、「及」不分,「北」、「比」、「此」、「些」不分,「相」、「想」、「像」、「象」不分,「謂」、「爲」、「未」、「位」不分,「兔」、「免」、「面」、「而」不分,「時」、「特」、「持」、「恃」不分,「末」、「未」、「禾」、「來」不分,「辯」、「辨」、「辦」、「瓣」不分,「汁」、「計」、「記」、「許」不分,「哩」、「裡」、「理」、「禮」不分,「隻」、「支」、「枝」、「技」不分,「又」、「有」、「友」、「支」不分,「勤」、「勸」、「歡」、「觀」不分,「如」、「知」、「只」、「至」不分,「才」、「財」、「材」、「村」不分,「夠」、「勾」、「鈎」、「釣」不分,「姚」、「桃」、「挑」、「跳」不分,「放」、「故」、「敌」、「敢」不分,「扁」、「匾」、「篇」、「遍」、「邊」不分,「句」、「向」、「問」、「間」、「閒」不分,「韋」、「常」、「當」、「嘗」、「長」不分,「干」、「于」、「子」、「了」、「丫」不分,「住」、「佳」、「往」、「拄」、「柱」不分,「見」、「是」、「自」、「時」、「事」不分,「籮」、「鑼」、「羅」、「罹」、「罷」不分,「罷」、「擺」、「把」、「抱」、「報」不分,「巴」、「吧」、「把」、「杷」、「靶」不分,「白」、「日」、「曰」、「旦」、「目」、「自」不分,「悵」、「帳」、「張」、「脹」、「賬」、「腸」不分,「狼」、「狠」、「很」、「恨」、「限」、「根」、「跟」不分,偏旁「木」、「才」不分等等;以及若干慣用語詞如「婊子」作「表子」,「赳赳」作「糾糾」,「汚穢」作「汚濊」,「鸚哥」作「鸚〖哥鳥〗」,「畎畝」作「〖亩犬〗畆」,「燙酒」作「盪酒」,「芙蓉」作「芙容」,「邋遢」作「辣榻」,「抖擻」作「抖搜」,「疙瘩」作「趷〖足荅〗」,「齷齪」作「䠎踀」,「猴急」作「喉急」,「悚然」作「聳然」,「筷子」作「快子」,「世面」作「識面」,「見識」作「見試」,「惺眼」作「星眼」,「咨嗟」作「〖口咨〗嗟」,「倔強」作「崛強」,「仔細」或作「子細」,「蒼蠅」或作「螥蠅」,「親戚」或作「親妾」,「鴛鴦」或作「夗央」,「萬曆」或作「萬歷」,「棺材」或作「官材」,「笑談」或作「笑但」,「豆腐」或作「荳腐」,「翻本」或作「反本」,「不過」或作「不故」,「便宜」或作「便益」,「皇帝」或作「黃帝」,「嗚呼」或作「烏乎」,「矇矓」或作「朦朧」,「眼睛」或作「眼精」,「索性」或作「率性」,「叮咚」或作「丁冬」,「收拾」或作「收什」,「原來」或作「元來」,「齜牙」或作「〖口咨〗牙」或作「咨牙」,「傢伙」或作「家火」或作「家伙」,「蜂擁」或作「蜂湧」或作「蜂〖虫勇〗」,「勾搭」或作「抅搭」或作「鈎搭」,「葫蘆」或作「壺蘆」或作「壺盧」,「窟窿」或作「窟竉」或作「窟寵」或作「窟籠」之類,則依文義統一逕改,不另一一說明。

\chapter*{自序}
\addcontentsline{toc}{chapter}{自序}
\markboth{自序}{自序}

夫余之此書。不名曰眞而名曰妄者。何哉。以余視之。今之衣冠中人妄。富貴中人妄。勢利中人妄。豪華中人妄。雖一舉一動之間而未嘗不妄。何也。以余之醒視彼之昏故耳。至於他人。聞余一言曰妄。見余一事曰妄。余飮酒而人曰妄。余讀書而人亦曰妄。何也。以彼之富視余之貧故耳。我旣以人爲妄。而人又以我爲妄。蓋宇宙之內。彼此無不可以爲妄。嗚呼。況余之是書。孰不以爲妄耶。故不得不名之妄言也。然妄乎不妄乎。知心者鑑之耳。

時

雍正庚戌中元之次日。

\begin{flushright}

三韓曹去晶編於獨醒園

\end{flushright}

\chapter*{曹去晶自評\footnotemark}
\addcontentsline{toc}{chapter}{曹去晶自評}
\markboth{曹去晶自評}{曹去晶自評}
\footnotetext{旣欲看是書。請先閱此評}

余著是書。豈敢有意罵人。無非一片菩提心。勸人向善耳。內中善惡貞淫。各有報應。句雖鄙俚。然隱微曲折。其細如髮。始終照應。絲毫不爽。明眼諸公見之。一目自能了然。可不負余一片苦心。其次者。但觀其皮毛。若曰不過是一篇大勸世文耳。此猶可言也。倘遇略識數字。以看鼓詞之才學眼力看之。但曰好村好村。此乃諸公爲腹所負自付(村)耳。非關余書之村也。求其不看爲幸。何故。諸公自恐其汙目。余更恐其汙書。

\begin{flushright}

書於獨醒園

\end{flushright}

\chapter*{林鈍翁總評}
\addcontentsline{toc}{chapter}{林鈍翁總評}
\markboth{林鈍翁總評}{林鈍翁總評}

予與曹子去晶。雖曰異姓。實同一體。自襁褓至壯迄老。如影之隨形。無呼吸之間相離。生則同生。死則同死之友也。曹子偶以所著之姑妄言示予。予初閱之。見其中多雜以淫穢之事。不勝駭異。曰。曹子生平性與予同。愚而且鹵。直而且方。不合時宜之蠢物也。何得作此不經之語。深疑之必有所謂。復細閱之。乃悟其以淫爲報應。具一片婆心。借種種諸事以說法耳。何以見之。黃金色以蠢然之富翁。好色輕生。而再世得爲才貌雙全之鍾情。復獲高第。而更得美麗之錢貴爲妻者。何故。以其自供生平一惡並無。諸善皆積。而神判中亦云心實善良。以其一善能解百惡之所致耳。後又因其爲多情種子。見色不迷。度量寬宏。謙謙自下。神復庇其發甲爲官。及其居官淸正。爲國愛民。歸時兩袖淸風。而宦實以報德之故。酬以萬金之產。焉知非冥冥之中陰注陽受者乎。此豈非警人當富而好善之婆心耶。白氏以銀錢擇婿。幾墮畜道。因其有感情報德之微。初罰之爲瞽爲娼。後方得爲良婦。其旨深矣。再世爲瞽目之錢貴。一遇鍾情。即矢貞不二嫁。後即置爲小星。後得雙目重明。受封生子。此豈非警人擇婿不當以財。而持身無淫妒之婆心耶。後三生者因係讀書之人。亦好色輕生。故罪黃金色一等。再生爲宦賈童。愚醜癡頑以報之。念其苦學之勤。使皆生於豪富。神恩厚矣。孰不知彼等無惡不作。恃富橫行。猶寬之。未罹惡報。但使之受其淫毒妻子之凌虐而已。若以宦蕚之惡。賈文物之假。童自大之臭。尚不使其妻子淫於人者。因宦賈童未曾淫人之妻女。故此妻不淫人。只不過癡頑凶暴。尚猶可恕。特存一點惻隱之心。留一自新之路與彼等耳。後能幡然自改。皆力行善事。宦蕚見色。能忍人所不能忍。賈童能輕財。捨人之所不能。更得神祐。不但保守家業善終。而且多福多壽多男子。仍暗化厥妻凶淫妒悍之心。使得同偕到老。豈非警人改故遷善。得獲良報之婆心耶。宦實爲朝廷大臣。而依附逆璫爲之假子。賈明以淸高之翰苑。而有萬餘之產。焉知非主考時私弊之得。童山能以刻薄而致富。宜乎生子若是。幾墜家聲。後幸得而守其家業者。雖三子能改過自新所致。或此三老又有隱微之善行。得挽回耳。此豈非警人貴者當盡忠於國。富者勿刻薄於人之婆心耶。侯富鐵三氏。前生皆爲男子。因罪孽深重。致墮畜道。罪限受滿。始得爲奇醜淫惡之婦人。此豈非警人勿造罪墮落之婆心耶。但此三氏之父。何不幸而生此三女。得無亦有失德耶。然其女尚無淫人之醜行。只其形狀醜惡。生性淫妒。乃厥夫刑于之化所致。況後盡化爲賢婦。不足爲父母累也。嬴陽以一梨園。仗妻子淫人而得千金之產。便妄自尊大。且誘人賭博內中。坑陷人家子弟不少。而使其愛女受報若此。此豈非警人忽恃財自妄。誘人局賭之婆心耶。了緣盜而獲命。幸矣。而又加之以淫毒。獄卒已屬凶徒。而又淫騙犯婦。龍颺淫人之女。又負情以揚其醜聲。故皆不得其死。此豈非警人凶險好淫之婆心耶。鍾趨擁婦棄姪。嫌貧棄婿。自後家產即爲不肖之子傾蕩。且隕命絕嗣。此豈非警人勿疏棄貧窮骨肉之婆心耶。鍾悛忘親棄弟。吞產離鄕。只落得骨殖棄於中流。妻嫁子奴。若非賢弟。幾斬其祀。此豈非警人勿薄棄手足之婆心耶。戴遷以好賭之故。傾家蕩產。至棄女爲人之婢。此豈非警人勿貪賭之婆心耶。鐵化好賭貪嫖。日夜飄蕩。致使妻子與狗爲伍。而後有外遇。竟非人類。此豈非警人勿晝夜貪於嫖賭之婆心耶。鄔合雖係諂脅小人。而不助人爲虐。後亦得重酬。使其嬴氏有此一番淫行者。因其已係廢人而誤少年女子。隱寓老翁蓄少婦之輩。豈非警人當自量。不可誤少艾婦女之婆心耶。莫氏覓媳而誤於媒。鄔合娶妻而誤於媒。鐵氏賣婢幾坑於媒。此豈非警人勿爲狡媒所誤之婆心耶。梅生能親厚貧窮之友。初獲艷妻。後得千金之報。鮑信之只以本分和氣四字獲利。而後得功名。含香以多情之故。而得良善之夫。嬴氏初雖淫蕩。而後能改過。竟得夫婦偕老而有子。豈非警人當做好人行好事之婆心耶。竹思寬幼而不孝。己身已好賭。而反誘人以賭。旣誘人以嫖。而又私人之妻。娶老鴇爲之婦。買龍陽爲之子。納妓婢爲之媳。已純乎其龜矣。此等一分人家。尚可言哉。誠所謂之忘八。卑卑不足數者矣。此非警人當上進。勿蹈下流之婆心耶。鍾悛因一文之故。破產而喪命。此豈非警人生意中勿見小苛刻之婆心耶。以上諸人。係書中要緊節目。故爲提出。如馬士英阮大鋮奸貪誤國。牛質易于仁好色貪淫。游混公卜通誤人子弟。屠四人屠戶局賭坑人。皆有惡報。其他種種。不可枚舉。明眼人一見而即知之。何必予之多喙。倘有一竅不通。有眼如盲之輩見之。強做解事語曰。此書一村淫之小說也。不但玷汙此書。豈不負曹子此一片婆心耶。予故不憚煩瑣。表而出之。有見之者。須細心思其報應處。學其改過處。而(勿)但注目觀其淫艷處也。故爲之評。

\begin{flushright}

庚戌中元後一日古營州鈍翁書

\end{flushright}

\setcounter{footnote}{0}

\part*{姑妄言首卷}
\addcontentsline{toc}{part}{姑妄言首卷}
\markboth{姑妄言首卷}{姑妄言首卷}

鈍翁曰。開首一段。原是敍瞽妓出處。別無深意。然將江寧歷來始末及城中諸景。寫得淸淸白白。曾遊過者一閱。如在目前。固一快事。即未至者。亦可想其風景。不勝神往。

永樂之設官妓。萬世仁人君子爲之腐齒痛心。先說是建十六樓。直是盛朝富麗。忽夾以此係永樂皇帝造爲漁利之所一語。復感嘆十六樓一作。把許多綺言一筆抹殺。眞皮裡陽秋。不覺令人失笑。

內中說癡頑公子富家郞效用加納等語。並非罵此等人是如此。正欲警此輩人不可如此也。一片婆心。看書者勿錯會其意。

\chapter*{姑妄言首卷\\
三韓曹去晶遊戲\footnotemark\\
引文 秦淮舊蹟 瞽女遺踪}
\addcontentsline{toc}{chapter}{三韓曹去晶遊戲}
\markboth{三韓曹去晶遊戲}{三韓曹去晶遊戲}
\footnotetext{編爲知者道。不共俗人看。}

詩曰。

\begin{quotation}

阿房宮裡稱佳麗。誰識秦淮艷六朝。

風嫋綠楊穿畫鷁。月明紅粉步虹橋。

滄浪夜夜聞鼉鼓。臺榭年年吹洞簫。

最是八行書未盡。渡頭又見酒旗招。

\end{quotation}

這一首詩是贊秦淮之作。你道這秦淮在何地方。乃金城中一條內河。這金陵係江南之地。春秋屬吳。戰國屬越。後屬於楚。因楚威王埋金於此以鎭之。故謂之金陵。嬴政改爲秣陵。孫權更爲建業。西晉曰建康。東晉曰丹陽。隋曰蔣州。唐曰昇州。宋仍建康府。元時稱集慶路。至明太祖建都於此。改爲應天。今之江寧府是也。秦始皇時。太史奏金陵有天子氣。那時他方自稱爲始皇帝。滿心以爲天下是他嬴家一己之物。欲傳之子孫於萬萬世。聽得這話。猶恐幾千萬年後或生聖人。奪了他家天下。遂忙忙發駕南巡。欲將龍脈掘斷。以洩王氣。自東至西濬成一河。城分兩半。引淮水灌之。因係始皇所開。故名曰秦淮。俗有兩句道得好。

\begin{quotation}

世無百歲人。枉做千年調。

\end{quotation}

就是他了\footnote{寫盡愚人之愚。千百年後之王氣不曾洩去。反把自己的王氣洩盡。一傳而絕。照遠不照近。千古同然。}。這秦淮東有三十三洞。但通江源。而人不能出入\footnote{在通濟門之南。東門也。}。西有一十八洞。設立水關。可行舟楫\footnote{在水西門之南也。}。諺云。三十三天無人走。十八地獄有人行。此之謂也。起初不過是條河而已。直至吳宋晉齊梁陳六朝。皆都於此。方纔富盛。到明洪武建都之後。將城改築外城。袤延一百二十里。門有十八\footnote{有瑤芳。土橋。鳳臺。安德等名。}。內城周六十里。門有十三\footnote{有聚寶。三山。石城。定淮。淸涼。通濟。儀鳳。仙鶴。麒麟。金川。太平。得勝。洪武等名。}。東則龍蟠。西則虎踞。建皇城宮室於其北。復將此河自南至北。開至雞鳴山下而止。雞鳴山之左。乃古之鍾山。形如覆舟。又名覆舟山。因蔣子文追賊至此山下被殺。孫權於此立廟祀之。故又名蔣山。山側有臺城舊基。建章宮含章殿遺址\footnote{此即壽陽公主人曰梅花落額山處也。}。陳後主辱井。山後即玄武湖。山之巓。右有梁武帝所創之雞鳴寺。此寺乃寶誌公監造。地址不過數畝。內中週廻曲折。深邃若大刹焉。至今有誌公遺像。漆裹裝金。造塔如室以供之。其左則明太祖所建之觀星臺。山之下。東則太學。西則帝王廟。功臣廟。蔣廟。高廟\footnote{合城機匠祀之。廟中有泉極佳。}。城隍廟。關帝廟等十廟。金碧輝煌。至於兩河岸上。有泮宮。泮宮二字乃宋朱熹所書。天下文廟之內皆爲明倫堂。獨此名明德堂。乃宋文天祥所書。文廟之側即貢院焉。又有黃公祠。桃葉渡。邀笛步。十景牆。橋側有靑溪\footnote{今呼內橋。}。淮淸。文德。武定\footnote{靖難時。黃觀夫人有詩云。不忍將身配象奴。手持麥飯祭亡夫。今朝武定橋頭死。一劍淸風滿帝都。即此處也。}。利涉\footnote{乃木橋也。自來相傳此橋映蘇州風水。宜木不宜石。至今蘇人年年來修。亦一古蹟也。}。大中。上浮。下浮\footnote{二橋乃船搭浮橋也。}。珍珠。蓮花。陡門。四象。笪橋等名。如飛虹橫跨河上。將一條秦淮妝點得十分富麗。十餘里樓臺夾岸。千百處樹木參差。畫舫飄遊。從朝至暮。笙歌繚繞。以夜繼日。天下相傳爲名勝之地。繁華之邦。凡過往紳衿商賈僕隸。無不買舟遊賞。本處富貴的人不消說。雖貧窮屠販。亦典衣棄物。必常常遊鑒。倘有一人不至。衆口咸稱俗物。因此遊人如蟻。往來絡繹。故那兩岸河房多居美妓。或隱約於珠簾之內。或徘徊於花柳之間。或品洞簫。或歌新詞。或倚雕欄而獻媚。或逞妙技以勾魂。或斜溜秋波。或嫣然獨笑。引得這些遊人浪子。無不魂迷色陣。骨醉神飛。日夜如狂。四時不息。這一段便是秦淮的佳話。後來明太祖升遐。太孫繼立。燕王朱棣爲惡禿姚廣孝所蠱惑。自北平起兵篡奪了建文天下\footnote{敍事中已把二人的罪案伏下。妙。}。改元永樂。恨靖難諸公不肯臣附。遂大殺忠良。男子老幼盡戳。妻女大小悉充官妓。於城裡城外建造

\begin{quotation}

重譯。石城。鶴鳴。醉仙。樂民。集賢。輕煙。淡粉。梅妍。柳翠。鼓腹。謳歌。南市。北市。淸涼。來賓。

\end{quotation}

共十六樓。以分貯之。設敎坊司掌管。隸於太常樂籍\footnote{敎坊司紗帽角帶。圓領白菜補子。有衙署。有公座。硃筆。吏役。刑仗。籤筒之類。儼然一官。但遇客不敢拱揖耳。}。終歲斂一年之利。交於宮中金花庫。爲后妃脂粉之資\footnote{醜極。以胯下得來之物。爲后妃面飾。可笑。}。美其名曰金花銀兩。這十六樓乃永樂皇帝造爲漁利之所\footnote{永樂於地下若有知。亦當愧殺。但不知可悔此一著錯否。}。與他處娼樓妓館自是不同。眞個是雕梁畫棟。玉宇瓊樓。檐飛走獸。窗斲菱花。一到晚來。紗燈照耀。玉燭輝煌。火光熒熒。如同白晝。淺斟低唱。妙舞嬌歌。觥籌縱橫。絲竹迭奏。朝夕爽心。日夜聒耳。至於其中美妓。則不可勝數。眞古今第一盛蹟。即也是亙古新創第一奇政也\footnote{奇則奇矣。虐亦虐也。}。曾有一詩感嘆這十六樓道。

\begin{quotation}

南北繁華十六樓\footnote{語褒而意貶。}。管絃吹動一江愁\footnote{勝於罵。}。

勸懲自有先王法。罪辱何須及女流。

陌巷花連秦苑曉。歌臺鶯囀漢宮秋。

當年只爲通商賈。不解而今有妓囚。

\end{quotation}

看了此詩。便知那時光景了\footnote{此句內中。贊美也有。唾罵也有。}。直到了嘉靖年間。此風稍息。然又生出一種瞎妓來。說起來尤爲可笑\footnote{瞎妓來因。}。你道一個女人生在世上。五官俱足。猶有醜陋不足觀者。況少了一對眼睛。可還看得。至於妓者。全要在秋波寄意。眼角傳情。若緊閉雙眸。有何趣味。相傳昔人有愛一眇娼者。寵癖異常。娶之而歸。人皆笑之。以爲異事。彼云。予自得斯人。視天下婦人無不多一目者\footnote{秦少遊有眇娼傳。}。此不過一人之癡情耳。與嗜痂者何異。但一女子至於雙目皆瞽。猶可相親者乎。你道這些人爲何作興到他\footnote{聖人云。見瞽者變。與孟夫子惻隱之心同意也。然若輩烏足語此。}。因內中有個緣故。那時十六樓的風景雖不能如初。又興出一個勝地來。名曰舊院。人稱之曰曲中院。門前對武定橋。後門在鈔庫街\footnote{明太祖造鈔之所。}。妓家鱗次比屋而居。室宇精潔。花木蕭疏。畫檻雕欄。綺窗絲幃。恍若仙居。逈非塵境。院中盆景盡異卉奇葩。房內擺設皆古瓶舊鼎。字畫悉唐晉宋元。器皿俱官哥汝定。焚香必鳳餅龍涎。烹茶定龍圍(團)雀舌。池中金鱗耀目。架上翠羽傳言。雖一拳太湖石。必透瘦可觀。即數朶枝上花。亦鮮妍可愛。各各爭妍獻媚。家家鬥勝誇奇。有客到門。則銅鐶半啓。珠箔低垂。升階則猧兒吠客。鸚哥喚茶。登堂則假母肅迎。分賓抗禮。進軒則丫鬟艷妝。捧娘而出。坐久則水陸並至。絲肉競呈。定情則目挑心招。綢繆宛轉。入夜則擫笛搊箏。梨園搬演。聲徹雲霄。喧塡達旦。到了夏月炎天。有一番佳致。卯飮淫淫。蘭湯灧灧。薰風徐來。衣香一室。至日亭午。裙屐少年。油頭半臂。提籃挈榼。高聲唱賣逼汗草。孩兒菊。茉莉花。嬌婢捲簾。攤錢爭買。捉腕捺胸。紛紛笑謔。頃之。烏雲堆雪。竟體芬香。請想在這去處行動的人。以千金買笑。白鏹纏頭。可是窮人做得來的。自然都是膏粱公子。富室嬌兒。或是效用的先生。或是加納的濶老。且這幾種人。不但使幾個憨錢。且要假裝一個名士。必定要嫖名妓。宿美娼。好使人羨慕他道。某名妓是公子的令翠。某美姬是財主的相知。他倒也不圖甚麼風流實事。只要博一個識貨的虛名而已\footnote{說盡狂奴心事。}。要知這名妓二字也不是容易加的。必定才貌驚人。技藝壓衆。衆口稱揚。逢人說項。這纔算得一個名妓。他無奈墮落煙花。身居下賤。那果然名稱其實的。未免自負。眼空一世。必須美如衛璧人。才過曹八斗的人品。纔得他心悅誠服。可是幾個臭銅錢輕輕動得他的\footnote{明末有名妓曰劉元。佻達輕盈。目睛閃閃。注射四筵。有一過江名士與之同寢。元回身向裡。不與之接。其人拍其肩曰。汝不知我爲名士耶。元回頭曰。名士是何物。値幾文錢一個。相傳以爲笑。彼輩視名士猶如此。而況於此類乎。}。你想。就是一個醜陋的妓女。也未嘗不思量接一個美貌男兒\footnote{說透人心。}。況旣是名娃。豈肯與酒囊飯袋衣架肉桶爲伍。且這種做癡頑公子的。拿着老子魚肉兵民幾個錢。仗着乃尊爵位勳赫。一番勢。一段驕傲之氣。雖長親父執。財勢稍次。尚不屑以正眼視之。何況將錢挾妓。不效占花魁傳奇中万俟公子身分者。能有幾人。至於富家郞。他祖父的財主可是輕易得來的。陽貨云。爲富不仁。這是財主們生前的官銜。死後的謚號。都是他刻薄窮人。心機盤算。日掙一日。積少成多。你想這種錢與強盜劫人相去幾許。可能保得常久受用。自然要生出不肖子孫。替他花費。這起孽障。身上穿幾件虼蚤皮\footnote{虼蚤皮。所謂輕跳(佻)之意耳。}。腹中無一點文墨氣。糟包着一個肥臉\footnote{唐歐陽詢謂長孫無忌云。只因心混混。所以面團團。可做此註解。}。高腆着一枚屎肚\footnote{此則不獨富家兒。}。腰中仗幾個臭銅錢。眼內無一個大丁字。談吐時俗惡之氣沖人。舉動時驕傲之態可掬\footnote{不但是此輩一幅行樂圖。而且是一篇揣骨相。}。勿論賢愚。稍有識者。未嘗不爲之噴飯。未嘗不爲之嘆惜。當時人稱他們爲麒麟楦。一絲不謬。何爲麒麟楦。人有假裝麒麟者。製一麒麟形狀披於蹇驢之上。望之儼如麒麟也。旣至脫去假飾。仍龐然一蠢驢而已。這些人以皮相之。相貌癡肥。衣冠齊楚。居然人也。窺其底裡。獸焉何別。請想這種人。可是那名妓眼中所有的\footnote{了却許多富家郞。}。再者。這些效用的先生。加納濶老。自然都是有錢人做的。他弄了一頂臭烏紗\footnote{自然是銅腥臭。}。不自己回想。我一資郞耳。滿身銅臭\footnote{頭旣臭矣。滿身自不能免。}。混濁衣冠。貽羞當世。縮頸藏頭。猶恐人知不雅。孰意毫無忌憚。意氣洋洋。以爲尚書宰相。是他分內之物\footnote{罵盡小人。如見其肺肝然。}。傲然自得。恬不知恥。終日鮮衣駑馬。俊僕豪奴。晝則橫行里巷。欺凌鄕黨。夜則投入煙花。美酒羊羔。要知道這原也怪他不得。你想他囊中有鈔。腹內無書。旣不知四書五經八索九丘爲何物。又不解孝悌忠信禮義廉恥是何話。終日無聊。不敎他嫖賭。却做何事。但可憐有一種不第的窮儒。三年燈火。十載寒窗。不能奮飛。終身困鈍。眞是控天無路。吿訴無門。言之令人酸鼻。還有無限抱經濟之才者。埋沒於草莽之中。懷韜鈴之略者。棲身於畎畝之內的。眞令英雄氣短\footnote{千古同聲一哭。}。眞所謂。

\begin{quotation}

時來頑鐵生輝。運去黃金失色。

\end{quotation}

就是此了。可笑這些沒字碑。自幼不受先生的氣\footnote{一樂。}。大來不受宗師的氣\footnote{二樂。}。仗祖父遺留的些寶鈔\footnote{三樂。}。托自己生來的些頑福\footnote{四樂。有此四樂。纔好配後之四妙。}。公然做起甚官來。稱起老爺來\footnote{此不足怪。江南之和尚道士。遼東之醫生。無一不稱老爺者。}。相與起當道來。扛擡起大轎來。長班跟隨起來。篾片奉承起來。紗其帽而圓其領。腰其帶而補其花。冬烘頭腦。雖皇帝在上。亦不知其比己尚尊。此身如在雲霧中。捉摸不定。雖欲不自大。其可得乎。這等人。人人見之欲嘔。個個聞之齒冷。況那嬌嬌滴滴的名妓。身邊可容得如此惡俗之物\footnote{了却許多加納效用的先生濶老。}。因這幾種人在妓館往來甚密。若(惹)得那些名妓都厭惡起來。雖不敢明明拒絕。恐其使勢也。有在言語中譏誚他的。也有作詩文嘲笑他的。也有假歌詞代罵的。也有在背後指搠的。久而久之。轟傳里巷。人皆以爲美談。這些簇新時興的老爺\footnote{簇新時興四字。加得刻毒之甚。}。旣不能博一個虛文。反添了一篇醜贊。弄得認眞不得。認假不得\footnote{苦極。}。欲留戀而自覺無顏。欲嗜惡而又無指實。因此不約而同。再不敢輕遊妓館\footnote{更苦。}。但這些人是浪蕩慣了的。如無韁野馬。縱轡狂驢。身子如何拘束得住。無可奈何。不得已而思其次。千籌萬算。在妓女中想起一種瞎妓來。他想頭也甚妙\footnote{眞妙想。}。去嫖這瞎妓。他却有許多燥脾處。緊閉雙睛。不能辨我之好醜。無從褒貶。一也\footnote{一妙。}。瞎女中百無一人能通文墨者。任其一肚臭糞。滿口胡柴。只是贊好。二也\footnote{二妙。}。日間一度風流。百文定價。每夜通宵行樂。額例四星。價錢又廉。纏頭省費。三也\footnote{三妙。}。彼瞎婆向日所接。不過屠戶販子僕皀輿人。彈琵琶唱野詞。侑燒酒臥破蓆而已。今忽有顯者大老光臨。猶如天降。公然日間陪着。肆筵設席起來。夜裡睡着錦衾繡帳起來。出自意外。聽其驕矜使氣。只是一味趨承。何等爽心湊趣。豈不樂哉。四也\footnote{四妙。}。爲有此四種妙處。向日爲名妓所輕薄厭惡者。今日皆趨移於瞎子矣。且這種瞎妓。他當日未得際之時。爲人所賤棄。成年屢月。那陰戶尚不能開市大吉\footnote{笑倒。}。間或有臁瘡乞丐。禿頂遊僧。要來點綴點綴。只圖幾文爲糊口之計也。一槪笑納不辭。今日所遇俱皆肥馬輕裘之客。眞如登天界。奉承之不暇。雖受鞭笞之辱。猶覺其榮。又曷敢少有所忤乎。所以這些濶老更加親愛。視之如掌上之珍。惜之猶心頭之肉。尚有一等可笑的人。他向日原也不屑頑瞎子的。今日見這些老爺們皆如此鄭重。視同尤物。彼不知他之苦衷。但垂涎羨慕。道。今日之富翁大老。皆以瞎妓爲命。我何人斯。豈可不一爲領略耶。視之猶如至寶。得共席一飮。欣然如赴瑤池之宴矣。得聽一曲。樂哉如聆鈞天之樂矣。得贈一物。如漢皋之解佩矣。得共一寢。如高唐之入夢矣。尊榮得這些瞽妓。不啻巫山神女。洛浦仙妃。皆踴躍視之。趨蹌恐後。悉尊稱之曰姑娘。甚而竟有跪之拜之。稱親娘者。因此瞎姑之名重於一時。而名妓之門。反可羅雀矣。雖係俗人之眼內無珠。然亦巫下之風俗此矣。雖然亦不可執一而論。竟有才貌雙全。恩情畢至的。但千萬中僅見之一人耳。你〔道〕我爲何敍此一段。因當時有一個瞎妓兩世姻緣的公案。欲續在後文。故引此以見瞎妓之來踪。不致突然。使觀者詫異耳。請閱下回。便知端的。

\setcounter{footnote}{0}

\part*{姑妄言第一卷}
\addcontentsline{toc}{part}{姑妄言第一卷}
\markboth{姑妄言第一卷}{姑妄言第一卷}

鈍翁曰。此一回方入正意。說神說鬼。正是本書命名姑妄言之意。然如此。方見得來路分明。或謂一部書中不下百人。而托生者寥寥數十而已。其餘或善或惡。何不皆一一註明。更覺可據。余曰。若如所言。不是著書。竟是作一本大點鬼簿矣。或又謂。旣如所云。何不竟不用此一段神鬼的話。余笑曰。若不引此數十人出處。後來憑空生出多人。又是一篇無影的杜撰了。要識作者之意。方見其苦心。

到聽塗說之人。天下皆是。聖人採童謠。亦未必句句皆有實驗。妙在到聽說莫愁湖之魚。却是假。人信以爲眞。說城隍廟之鬼。明是眞。而人反謂之假。世上過耳之言。眞而假。假而眞。不可但因其人而定眞假也。見此可長一番學問。黑姑子一段。要他後來授術於崔命耳。故不得不生出他來。以受道士之術。若不寫這個姑子。將來何以傳那個姑子。又可見此輩中守戒律者少。非謗之。實勸之耳。

峨嵋山人首篇即出。直貫至十五回內又見。可見一部書是一氣呵成。並非捏攏湊合。

寫道士之遇昌氏。似乎蛇足。此有深意焉。一部書中淫婦人不少。而開手寫一極淫之昌氏做榜樣。昌氏之淫。量可謂無敵矣。遇道士而得病。再遇竹思寬而身死。可見貪淫之婦。無不因淫而死。特死有異同耳。鄰家小廝同昌氏調戲一段。入情入妙。男貪女愛。滿心要私合。却都在幼年。又怕羞又膽却(怯)。想出法來。先猜枚。贏打手批。繼而贏親嘴。逐漸而入。此調戲彼。彼調弄此。彼此親厚了。纔放膽去做。的是一對孩子行徑。看他兩個調戲的那番光景。畫也畫不出。即出無關係處。亦不肯輕意草草寫出。

如黑姑子住在一條小僻靜巷內。門口一叢黑松樹。一個小小的圓紅門兒。進去裡面甚是寬敞。到聽提着一獨(角)盧(蘆)瓶水白酒。肥肥的一段騎馬腸兒。兩個醃鴨蛋來望他。此所謂像形也。書中似此等趣語不少。

此一回淫婦人則小姑子與昌氏母子。淫男子有名者。則到聽于敷道士三人而已。其餘雖多。而和尚則不可勝數。豈獨寫和尚之惡。實此輩較諸人尤淫毒也。

一部大書二十四回。內中無限的人。頭一個就是一個閒漢。這一個閒漢引出莫愁湖閒蕩的四五個閒漢。這四五個閒漢又引出同到聽斑駁的許多閒漢。這許多閒漢又引出看花的無數閒漢。雖有一個道士。還是閒漢一流。何天下閒漢之多也。士農工商。各執一業。便不是閒漢了。終日遊手好閒。不至不做賊不止。這許多閒漢引出後來千千萬萬的流賊。無非都是閒漢。此是一部書的大呼吸。

此一部書內。忠臣孝子。友兄恭弟。義夫節婦。烈女貞姑。義士仁人。英雄豪傑。淸官廉吏。文人墨士。商賈匠役。富翁顯宦。劍狹(俠)術士。黃冠緇流。仙狐厲鬼。苗蠻獠玀。回回巫人。寡婦孤兒。諂父惡兄。逆子凶弟。良朋損友。幫閒梨園。賭賊閒漢。至於淫僧異道。比丘尼。馬泊六。壞媒人。濫淫婦。孌童妓女。汚吏贓官。凶徒暴客。淫婢惡奴。傭人乞丐。逆璫巨寇。不可屈指。世間所有之人。所有之事。無一不備。余閱稗官小說不下千部。未有如此之全者。勿草率翻過。以負作者之心。

此一回書雖係正文。猶文之餘文也。如傳奇之副末開場一齣。雖與正文無涉。然係必不可少者。看者須知。

此開卷說到聽。謂他上無父母。中鮮兄弟者。何意後來引出鍾生。也是無父母鮮兄弟來。遠遠相對。這一個便流落做了閒漢。那一個便成了正人君子。愈見鍾生之不可及也。又謂到聽惟以聽新聞說白話爲事。近日此輩人幾遍於天下矣。

\chapter*{姑妄言卷之一\\
第一回 引神寓意 借夢開端\\
附 接引庵黑尼姑受異術 西湖畔小寡婦縱奇淫}
\addcontentsline{toc}{chapter}{第一回 引神寓意 借夢開端}
\markboth{第一回 引神寓意 借夢開端}{第一回 引神寓意 借夢開端}

話說前朝有一奇事。余雖未曾目睹。却係耳聞。說起來諸公也未必肯信。但我姑妄言之。諸公姑妄聽之。消長晝祛睡魔可耳\footnote{二十四回書。從這兩個妄字生出。}。你道此事出自何時。係當日萬曆年間。南京應天府有一個閒漢。姓到名聽。字圖說\footnote{一部書。頭一個出名的便是道聽塗說的閒漢。閒漢一。}。家住旱西門內\footnote{旱西門是與接引庵小紅門相對者。}。他上無父母。中鮮兄弟。孤身一人。不事家產。終日無所營爲。只在街市閒遊。惟以聽新聞說白話爲事。他有一件奇處。古人是過目成誦。他却能過耳不忘。每常聽人說甚演義。千言萬語。能一字不遺。他相識甚多。說鬼話之名遍於一城。故此人起他一個混號。叫做毛空。一日。他在街上閒行。遇着四五個人。說着閒話走來\footnote{又是四五個閒漢。閒漢二。}。內中有兩三個認得他\footnote{應前相識甚多。}。便一把拉住了。道。你說個白話我們聽。他故意匆忙之態。掙着要跑。道。我今日有要緊的事。不得閒。改日來說罷。那人拉住不放。道。你有甚麼事。對我說了。纔放你去。到聽道。方纔幾個朋友說。莫愁湖近日出了許多魚。他們都借網打魚去了。我回家去取個筐子。要些來下酒\footnote{原似可信。}。說完。忙忙掙脫跑去了。衆人信以爲實\footnote{孰知竟似(是)假。}。商議道。我們何不大家去看看。倘有熟人在那裡。落得要些來吃。遂興興頭頭一齊走出水西門。到了莫愁湖。惟見煙水茫茫。菰蓴佈滿。半個人影俱無。方知爲他所哄\footnote{趣。}。及至走了回家。魚不曾得了一個。反走得通身是汗。改日遇見了他。說他道。莫愁湖何嘗有魚。你怎耍我們空走一回。到聽道。你們原拉着我。叫我說白話。我說弟(的)就是白話了。誰叫你認眞\footnote{妙極。趣極。}。衆人大笑一場。偶然一日。他四處遊蕩。天色將晚。無可圖餔啜之處。意欲歸家。不意在途中遇見相好的一個酒友\footnote{這酒友無非也是閒漢。閒漢三。}。邀他到酒市中坐下。要了兩碟子小菜。沽了幾壺藥酒。二人對酌。說了些無稽的白話。談了些脫空的俚言。豁了幾件無徑的拳。唱了幾句無腔的曲。多飮了幾杯。醄然大醉。遂辭了那朋友回來。酒醉路黑。一路踉踉蹌蹌。走到古城隍廟前。一時酒湧上來。見廟門半掩半開。就走入門內。倒在側邊泥馬足下。不覺睡去。直至三鼓。因遍身僵冷。方矇矓少醒。似夢非夢\footnote{此句好。若竟說明明白白看見。便是活見鬼了。}。見殿上燈燭輝煌。正居中坐着一位袞冕王者\footnote{神。}。傍侍許多官吏。夜叉鬼卒\footnote{鬼。}。羅列庭下。到聽知是神道顯靈。嚇得汗流浹背。不敢喘息。遙聞得如神問事狀。側耳而聽。偷目而視。只見一個黑臉虬髯的判官。上前稟道。地府十殿閻君遣崔判官。齎到冊籍並若干人犯。送大王發落。那王道。叫他過來。隨見一個白面圈鬍。紅袍烏帽的神道。在檐下參見畢。立起稟道。地獄中夏商周三代以前並嬴秦時所有輕重罪犯。皆已斷訖。自漢室初興起。從大王歸神以後。以至唐宋訖今明朝之嘉靖末。將二千年來。人心不古。犯重罪者甚多。漢朝如王莽董卓梁冀曹操之流。唐朝如李林甫安祿山盧杞朱泚之輩。宋朝如王安石賈似道蔡京童貫之徒。明朝如胡惟庸汪廣洋藍玉宸濠之類。有應墮畜道者。已久矣送轉輪托生。有永沈地獄者。皆發十八司受種種之罪孽。尚有許多疑案。至今尚未能結。昨地官大帝奉天玉帝旨。到陰府查核。獄中有沈滯者。可速了結。因查得各種疑案。命小神將冊籍並犯人送到大王臺下剖決。王笑道。森羅殿上。業鏡分明。況且十殿閻君。皆冰心鐵面。有何持疑不決之處。那神又稟道。人在世間所犯罪戾。或輕或重。有一定之律。自易分剖。陰府斷事。必須情罪俱當。纔稱得鐵筆無私。比不得陽官。胡胡塗塗。可以任己心行事。諸案中有一種罪實輕而情頗重者。又有情可恕而罪難逭者。因此故難下筆耳。王又笑道。這有何難。罪輕而懷(情)重者。榮其身而罰於後。情輕而罪重者。亦就其事而斷之。何難之有。你將一起起文卷並人犯挨次呈來。聽我分剖。那神呈上一册。道。此董賢父子一案。只見一個老兒。一個婆子。一個美男。一個美婦。齊跪堦下。王問那神道。董賢罪犯甚實。有何疑處。那神稟道。董賢父子。若謂蠱惑朝廷。幾危社稷。則罪擢髮難數。然而實未嘗殺人害人。若與操莽等同科。似乎太過。若從輕議處。又無以爲後來者戒。所謂罪重而情輕者以此。王怒道。董恭夫婦不能訓子以義方。反藉子之聲勢赫奕一時。今把他托生。仍做一個富家翁。還借他族間之聲勢。享用五旬。可不償還他不會害人的好處麼。却使他妻子淫人而假種。雖有子而絕其嗣。這就暗暗的報應了。死後發阿鼻受罪。豈不完他的宿孽麼。至於董賢。冶容眩色。幾至漢哀帝那昏君有禪代之事。以鬚眉丈夫而效淫娃舉動。情已難恕。且將妻子亦以奉朝廷而博寵榮。此又以龍陽而兼龜子者也。尚列衣冠。晉位司馬。更令人髮囗(指)。仍著他與董恭爲假子。使之帶一暗疾。專善人淫。其妻以婦人而不知三從四德。乃獻媚要君。今還托生爲婦人。與董賢仍配爲夫婦。授以不男不女之形。奇異宣淫。後使不得其死。以報其夫婦之罪。使他享福者。情輕之故。受惡報者。償罪重耳。豈非兩得乎。因問那神道。我斷得是麼。那神道。大王金判。不但小神欽服。即董恭父子夫婦亦無容多喙矣。王吩咐鬼卒道。此地有一牛姓。兩代刻薄成家。素性陰賊良善\footnote{看到此等處當着眼。}。可使董恭爲彼眞子。董賢爲其假孫。董賢雖育多男。俱非眞種。後同歸於盡。絕其後而兩報之。牛董二家同結此公案可耳。董恭之妻。托生苟姓。仍與作配。喝一聲下去。寂然不見。那神又呈上一卷。就有一個金貂少年。一個珠冠美婦跪下。王看畢。問道。曹植與甄氏罪狀顯然。當年蕭何之律法三章。不足爲據。以今日之大明律斷之。叔嫂通奸者。絞。更有何疑。那神道。二人私心相愛則有之。然而實在奸情則未有也。況曹植曾爲遮須國王。甄氏亦爲洛浦仙妃。欲重擬之而不敢。欲輕擬之則不可。所謂情重而罪輕者。故爲疑耳。王勃然變色道。是何言哉。王子犯法。庶人同罪。普六茹堅云。豈天子兒另有一律耶。陽間斷罪以事。我陰曹斷罪以理。曹植甄氏雖未成奸。誅其心。豈不欲奸者耶。那一篇洛神賦就是他的罪狀了。非我以莫須有三字加人之罪也。曹植以才美如斯。甄氏已貴爲皇后。尚復如是。故罪比愚夫愚婦未成奸者加一等。要說他一爲國王。一爲仙妃。只可勢利凡夫。我這裡顧他不得。曹植以如此才華而無行。今着他托生爲一美男兒而仍無行。但他生爲王死爲王。使之爲民太卑。令其爲官不可。叫他去做個假道姑。庶乎不貴不賤。甄氏初旣不能死節於袁熙。後又失貞於曹丕。旣云他是仙妃。再世可爲佛女。我看得有一藺姓夫婦。廣信佛法。佛法豈謂不好。但佛門中所當行之善事甚多。彼以一己之愚。惟以養僧贍道爲善。孰不知僧道中十無一良。故罪比不信佛法者加等。甄氏使爲之女。敗壞門風。與曹植苟合。以了前緣。皆死非命。以正有服通奸之罪。那神稟道。小神聞得齋僧布施。功德無量。與恒沙河等。而大王如此斷之。小神不知其中所謂。望大王諭之\footnote{問得好。若無此一番問答。不得醒愚人之迷。}王道。人在世間。當行之善事不一。如文昌帝君陰隲文云。濟人之急。救人之危。修數百年崎嶇之路。造千萬人往來之橋。種種甚多。即如去道傍之一石一木礙人道路者。何非善事。能力行不倦。自可獲福無窮。若只任愚迷。惟以齋僧布施爲事。果能供養高僧。自然邀福不淺。但如今這些和尚能持戒律者。千百中能有幾人。他處無可奈何之際。只得暫守淸規。你反齋之給之。助他貪淫嗜酒。破戒行凶。在家人所不忍爲者。彼竭力爲之。豈非以油添火乎。孽雖由彼。而助彼爲虐者。非此而誰耶\footnote{善男子信女人不擇僧而一例亂施者。可將此語常閱之。}。韓昌黎云。人其人。火其書。同此意耳。神道。大王尊諭。眞聞所未聞。開小神茅塞多矣。王顧左右道。將此案人送到轉輪王處交割。再將袁熙托生爲藺馥之子。使曹植甄氏皆死於彼手。以了前孽。鬼卒答應一聲。帶了去了。王又道。還有何案。神道。漢家只有此二件。唐室甚多。尚求大王區畫。王道。把唐家的人犯全帶上來。就有許多男婦在丹墀跪下。那神指着一個標致少年稟道。此張昌宗也。求大王判之。王神目一睜。呵呵笑道。蓮花似六郞者即爾耶。又忽然大怒。高聲喝道。爾烝淫母后。已罪不容於死矣。武曌久淪苦海。不必再議。爾尚可末減者。以武氏之淫。不成其爲母后者耳。然而爾之罪亦不容緩。不意尚得悠遊於地獄也。命鬼卒道。楊國忠本他之遺孽。又幾壞唐家。可押他去。仍與楊姓爲子。姓其子之姓。爲龍陽一世。以償臣主宣淫之罪。後殘廢不得其死。前生面似蓮花。再世遍體楊梅。死後再墮抽腸地獄。庶可消此忿恨矣。王又指着一個道。這是誰。那神道。這便是昌宗之兄張易之也。王點頭道。他之罪與昌宗等耳。也着他生爲龍陽。死於非命。足以報之矣。可押去龍家爲兒。那神又指着一男一女道。此武三思韋庶人也。三思一禽獸者流。韋氏一淫鴇者匹。此可謂罪爲次而情難逭者。願大王察焉。王作色道。你閻君太覺迂濶了。武三思不但以臣子而烝二母后。且以姪奸姑。罪尚何言。韋氏以母后而下淫。且鴆夫而殺子。罪更甚焉。姑以無知之淫娃。生爲下流之淫鴇。今著三思爲竹姓之子。始篾片而終龜。以酬邪慝。有一竹淸夫婦。吝刻異常。宜生此子。蕩產破家。韋氏罪爲郝\footnote{音好。}老鴇。初爲妓女。爲多人之妻。以償淫孽。後逢思寬。以完後愛。配爲夫婦者。非遂其淫心。使之一以貪淫而亡。一以好淫而斃。死後均下刀山地獄。足以報之矣。那神在旁不住點頭。暗暗贊是。王又指着一個宮娥。問那神道。這是何人。神稟道。上官婉兒。王道。你父上官儀爲唐室忠臣。爾不思父爲武氏所害而恨。反與三思通淫。你初生時。謂你能權衡天下的人才。這番行事。大約就是你的權衡了。你又勾引韋氏與三思私淫。不但不孝。而且不忠。罪當云何。婉兒道。妾父爲武后所殺。籍沒入爲宮婢。切齒之痛。寧不思報。但武后一世之雄也。妾何能爲。因仇無可復。故誘三思。以淫韋氏。假手以死中宗。爲父報仇耳。望大王上察。王笑道。其然。豈其然乎。果如爾所說。你就不該與三思通淫了。我跟前豈容你巧辯。叫鬼卒。押他去火宅。托生爲女。今姑示薄罰者。以汝之罪尚有可原。此去若能改過。來時再一墮畜道。以償勾引淫主之罪。輪迴再轉。便得善地受生。若淫心不改。仍通三思。即爲三思淫死。則難拔苦海矣。押去。鬼卒答應一聲。帶去了。只見一個人高叫道。大王。我是楊再思。別無過惡。不過善於逢迎。閻王說我罪輕情重。繫獄千餘載。求大王爺超拔。又一個婦人叫道。我虢國夫人楊氏。也無大過。閻王道我恃美奢淫也。入罪輕情重案內。至今未得超生。求大王矜憫。王笑道。楊再思。你雖無大過。但贊昌宗蓮花似六郞一語。可謂諛醜之至。也就遺笑千古了。楊氏恃一時之寵。奢淫侈慾。無所不爲。彼時人道你。却嫌脂粉涴顏色。淡掃蛾眉朝至尊。揚汝耶。抑汝耶。你二人昭昭史冊。可謂遺臭萬年矣。雖然。皆猶可恕。楊再思再生爲鄔合。使爲天閹。雖名曰陽。而毫無陽氣。以你生前雖係男子。而柔媚如婦人耳\footnote{善諛者留神。勿後世爲天閹也。}。爲一世幫閒。以完其善諛之性。楊氏即爲爾之妻。貪淫而可淫。旣得淫而又苦於淫。後因創於淫而息其淫。來世或可爲不淫之人耳。帶去。方纔帶過。那神又稟道。這是楊國忠同妻子裴氏。王睜目大喝道。國忠以奴隸之才。借妹氏而邀相位。逼祿山反。以危唐社。裴氏假云夢合而生子。汝愚國忠乎。欺鬼神乎。速押去。國忠爲嬴氏之子。梨園而龜。裴氏爲陰家之女。戲旦而妓。國忠向借妃妹之榮而致相。今戲臺上。官兒時時任做。裴氏有多夫之樂。那巫山夢也不必再尋了。王忽然呵呵笑道。妙哉。虢國前爲伊妹。今復爲伊女。仍站門楣。可謂是夫是婦。是父是女了。去罷。一陣陰風。三人皆無影響了。那王向下一看。見一個肥美婦人。翬翟之服。如后妃裝束。頸垂素練。王笑道。你壽王配耶。抑楊太眞耶。李三郞妃耶。安祿山母耶。衛宣之新臺遺臭。其媳尚未偶其子。猶萬世所譏諷。汝旣久爲壽邸之配。又爲李三郞之妃。在他父子聚麀。已非人類。貴爲天子。爲家奴李輔國所弑也。就算現報了。你一個婦人。竟肯叫他父子同門。也就無恥之極矣。你今日若見壽王。將置身於何地。況還反妒梅妃。又私祿山。言之令人汙頰。以你所爲。當墮畜道纔是。只見那婦人辯道。古人云。爲人莫做婦人身。百年苦樂由他人。妾一婦人耳。焉能自主。明皇以君父之尊。欲下淫兒婦。我如何敢拗。至於祿山一事。更有下情。求大王諒之。我一個靑春少婦。與壽王正是佳偶。明王一個雞皮老翁。將我占去。所謂不敢言而敢怒者是也。我之私祿山。正是爲壽王雪忿耳。不然。這樣三百六十斤的一個大肚皮鬍漢。那被底風流就有限了。有何可樂。有何可愛\footnote{余見此數語。因想起兩個笑談來。一男子胖甚。同妻子交媾。因樂極時向下一壓。將妻壓死。此婦到〈到〉陰司訴寃。冥司將男子拿去。男子辯道。非我有意將他壓死。因一時酥麻無力。往下一壓。因而致斃。我有何罪。冥司笑道。你這蠢材。你行房時。將一條小板凳墊在胸前。便無此患了。慮不及此。焉得無罪。一幼女身材甚小。所嫁之夫有三百餘斤。彼父母兄嫂常以爲慮。恐彼壓殺。彼竟無恙。滿月歸家。其嫂私問道。我每常以爲慮你壓死了。竟造化無事。如何幸免。女子道。他兩手拄定。豈無百餘斤力氣。兩膝跪榻。又減去無百餘斤。只剩數十斤。腰間那物撑住了。還有何害。祿山之於玉環。不知是用板凳墊胸。又不知是手足腰三處用力之故。雖起玉環而問。亦未必肯述。附此可做一笑。余兄辱翁曰。玉環與此二人不同。肥而無骨。那怕壓殺。}。至於妒梅精一事。又係婦人之常。不得深責於我。況馬嵬一縊。慘痛非常也。可以相准了。王道。也罷。你還去托生做一個美婦。你前生旣是不后不妃。今世仍做人之不妻不妾。你憎李三郞是個雞皮老翁。你還去配一個鶴髮老叟。你生前做了一場假道姑。今去做一個眞禿尼。你能潛心釋典。革去淫心。尚得好死。若仍縱淫不戒。就使你淫樂而亡。雖然比馬嵬受用些。再來却難免地獄之苦了。且帶過一邊。那神指着一個峨冠博帶的人道。此祝欽明也。王微哂道。五經掃地者爾耶。你爲人之師範。那一番高麗舞眞可謂面甲千重。虧你如何做得出。躊躇道。他尚無大罪。只善媚耳。此等人。如今天下皆是也。罪不得這許多。還許你去做一個的資郞。配你一個淫悍之妻。也足報你了。你前世旣學高麗。今使你去做一個回子。又想了一想。道。好好。那上官婉兒是你同時的人。就把他配與你罷。神又稟道。這李林甫十世爲牛。九世爲娼。皆遭雷震。惡報已滿。送到大王臺下發落。那王不住點頭嘆息。那神問道。據小神愚見。李林甫之罪。與歷代奸邪誤國者等耳。尚未如莽操輩弑君弑后。而受報獨重者。何故。求大王見示。王道。李林甫本仙官。應劫降凡。若能再立功行於世。則返列仙班。永無輪迴之患矣。不意他自己墮落至此。豈不可惜。我之長嘆者。正爲此耳。當日安祿山謂一術士云。我見天子猶不畏。但見李相則心悸汗流。何也。此人能視鬼。云。公有銅頭鐵額魔兵五百爲護從。何得畏彼。俟異日來。我當觀之。後李林甫來。此人見林甫前有一對仙童。手執提爐前導。護祿山之鬼皆踰牆越壁而奔。術士無(撫)祿山言其故。復曰。李相乃仙官降世。非等閒人也。此即可證。汝言諸人受報皆輕。而他受報獨重者。則非也。諸人永沈獄底。受諸苦惱。萬劫不能超生。其罪隱。故以輕耳。林甫雖爲牛娼被震。其罪顯。故以重耳。但他尚有出路。可以自新。他若再生陽世。能屢立功德。十世之後。尚可復立仙班。其所罰輕矣\footnote{妙哉此語。破醒世間多少疑惑事。即如善人受摧殘。貧賤而夭。惡人享福祿。安逸而多壽者。同一理也。焉知無後報也耶。古人謂。善惡到頭終有報。只爭遲早耳。誠至言也。}。但恐此去再奸爲(譌)不忠。殺害良善。縱惡恣淫。貪得無厭。不但生前受妻淫。妾淫。女淫。媳淫。種種惡報。此後永墮地獄。再無出期矣。李林甫道。某千餘年備嘗苦毒。自悔無及。焉敢復蹈前轍。王搖首道。噫。但恐你一得人身。却又忘了今日矣。你此去雖不能得相位。也還貴顯爲鄕貳重臣。可以有爲。切不可又萌邪念。負了上帝恩德。鬼判可送他阮家去托生。那神又呈上一冊。道。唐家只此李義府一案了。王恨道。李貓兒耶。笑裡藏刀。腹中懷刃之人。情罪皆難恕者。發去聶家爲子。若能改過則已。倘凶頑肆惡。不但陽世不得善終。死後再受孽報。也足正其罪了。那神稟道。趙普一事。宋太祖屢訟天庭。謂他因一言而害德昭廷美。可謂稔惡。但查他之相業。頗有可觀者。所以也在疑案中。上呈大王金判。王嘆道。此何言哉。負心報。冥府報最重\footnote{余見諸勸善書云。負心者。冥司極惡。但今人負心者。車載斗量。但恐冥司報不得許多。}。況負聖主之恩而害其子弟耶。他不過貪富貴之心重耳。今著他生於吳姓。還做一個富貴顯官。酬他的相業好處。使他老而無子。斬其血嗣。家資仍爲衆分去。貪富貴而富貴俱失。害人子而亦絕其子嗣。死後永不出地獄。每日受拔舌之苦也。就可以報他媚人害人了。因叫道。玉環過來。就把你做他的續配。以完前孽罷。玉環道。我在生時。初爲王妃。後做天子之亞后。我此去寧可不要丈夫。豈肯配一臣子。王搖着頭。笑道。你不要說這體面話。他不比安祿山還高幾分麼。又笑着道。你也認不得他了。判官可把趙普前世的原形揭出來。那判官上前。吹了一口氣。玉環一看。原來就是壽王李瑁。羞慚滿面。低頭無語。王笑道。你認得了麼。雖係今世之事。乃生前未了之緣耳。那趙普欣欣自得。玉環粉面低垂。一同去了。那神又稟道。宋家奸邪各案俱已完訖。只有秦檜父子祖孫一案。昨日岳忠武王親降陰府。向十位殿下道。秦檜罪惡雖重。受罪多年。亦不爲少。替他說情。叫他放往陽世去走一遭。看他改過不改過。給他一自新之路。衆位殿下因他罪重。不曾放他來。命小神口稟。看大王尊意如何。可放他去不放。王道。你可知岳王的心事麼。那神道。小神冥曹下吏。焉能知上聖襟懷。王笑道。岳王在那時身爲大元戎。秦檜雖是奸相。焉敢就私自害他。高宗聽信奸言。扭於和議。有多一半是他之過。故賊檜尚可從輕議。況且岳王若不爲秦檜所害。不過與張浚。韓世忠。劉錡。楊沂中諸君。後人稱爲名將而已。焉能到今日血食千秋。廟貌而祀。你看杭州府他的墳塋。湯陰縣他的故里。何等崢嶸。他之功於岳王亦不小。在當日爲岳王之罪魁。今日又可謂之功首了\footnote{此是實情。不知岳王果同此心否。}。岳王欲放他往陽世去者。或他能改過遷善。尋一自新之路。亦未可知。此正是岳王以德報怨。正直慈憫之心。但不知此去若何。旣然有此。不可負了岳王的美意。且放他去做一個編氓。到艾家爲子。倘能力行善事。後世漸漸的超拔他。若還悛惡不改。他一個小民。尚不能流毒於衆。在生受殺身之慘。回來沈於獄底。永無出期。豈不是公私兩盡。我主意如此。你回去說了。看閻君尊意定奪。那神道。小神謹遵。又稟道。閻君說。秦檜父子若十分斷重。他非秦檜之親子。若稍從輕判。又不足盡秦檜之惡。所以也置疑案中。他父子現帶在臺\endnotemark[1]〇〇〇〇〇〇〇〇〇〇〇〇〇〇〇〇〇〇〇〇〇〇〇〇〇〇〇〇〇〇〇〇〇〇〇〇〇〇〇〇〇〇〇〇〇〇〇〇〇〇〇〇〇〇〇〇〇〇〇〇〇〇〇〇〇〇〇〇〇〇〇〇〇〇〇〇〇〇〇〇〇〇〇〇〇〇〇〇〇〇〇〇〇〇〇〇〇〇〇〇〇〇〇〇〇〇〇〇〇〇〇〇〇〇〇〇〇〇〇〇〇〇〇〇〇〇〇〇〇〇〇〇〇〇〇〇〇〇〇〇〇〇〇〇〇〇〇〇〇〇〇〇〇〇〇〇〇〇〇〇〇〇〇〇〇〇〇〇〇〇〇〇〇〇〇〇〇〇〇〇〇〇〇〇〇〇〇〇〇〇〇〇〇〇〇〇〇〇〇〇〇〇〇〇〇〇〇〇〇〇〇〇〇〇〇〇〇〇〇〇〇〇〇〇〇〇〇〇〇〇〇〇〇〇〇〇〇〇〇〇〇〇〇〇〇〇〇〇〇〇〇〇〇〇〇〇〇〇〇〇〇〇〇〇〇〇〇〇〇〇〇〇〇〇〇〇〇〇〇〇〇〇〇〇〇〇〇〇〇〇〇〇〇〇〇〇〇〇〇〇〇〇〇〇〇〇〇〇〇〇〇〇〇〇〇〇〇〇〇〇〇〇〇〇〇〇〇〇〇〇〇〇〇〇〇〇〇〇〇〇〇〇〇〇〇〇〇〇〇〇〇〇〇〇〇〇〇〇〇〇〇〇〇〇〇〇〇〇〇〇〇〇〇〇〇〇〇〇〇〇〇〇〇〇你可知嚴嵩的來歷麼。那神道。小神正在疑惑。他當日往生。並不曾經由地府。不知何故。求大王詳示。王道。他原是一個歷劫魔王。上在無厭國中\footnote{果爾。則無怪乎當日有錢癆之稱了。}。下至苦海。皆爲他所據。帥領魔兵十萬。稱爲無厭大王。他殺害生靈無限。上帝將他囚於天獄。數千年來。頗知悔心改過。上帝慈憫。見他略有善念。不忍將他終棄。故使他托生陽世。位極人臣。富可敵國。原要他做一番好事。便可超拔爲神。不想他得了人身。惡性復萌。欺君誤國。戮害忠良。饕貪無厭。自墮惡孽。今我體上帝好生之仁。還叫他去做個宰相。若能做個忠臣。致君澤民。尚可以蓋前愆。還不致於墮落。倘仍肆惡如前。陽世現報。其父子死後。永化蛆蠅之屬。再想人身。萬劫不能矣。愼之愼之。送他往貴州馬家爲男子去。嚴世蕃他那裡是嚴嵩之子。一個魔王焉得有後。乃嵩乞他人之子而撫之。冒爲己子耳。他害人利己之罪。生前已斬首梟示報之矣。其奢侈淫汙之罪。也還要去受一受。問嚴世蕃道。你當日可覺得太過些。咳唾用美人之口爲香唾盂。便溺以銀婦人爲溺具。交合以白綾帕爲淫籌。你就不想一想今日到這裡來麼。今罰你去充家爲男。一生逐臭。流爲糞壤乞丐。仍不得其死。以正你奢淫之罪。那趙文華以嚴嵩爲父。陷害張經。胡宗憲等。皆出其謀。做了朝廷大臣。乃以金虎子諛世蕃。更鐫其姓名於上。在當時便有盛㞠子的官兒之美號。爾只圖容悅一時。獨不懼遺羞萬年乎。我看你的心腸眞異於他人。你還有些餘福未盡。再去受用一番。看你悔過不悔過。再來定罪。此一去雖是人形。却是獸種。易于仁就做你的名字。你須顧名思義。不可再錯脚跟。把董賢之妻就與你做假女。你不應有嗣。只好得兩個假子罷了。王哈哈笑道。你前世爲人之假子。後世人又爲你之假子。是可假也。孰不可假也。倒也可笑。那神向他道。大王一番恩德。放你去自新。不可負了。那大王不住點頭沈吟道。嚴鵠嚴鵠。忽然笑道。祖孫父子在生時。人都稱他爲錢癆。今叫他去做個龜子。名叫錢爲命。就把韋氏配與他暫爲夫婦。再拿回來受罪。正說着。那王舉目往下一看。見下面跪着非人非畜。一個奇形怪狀的東西。問道。那是個甚麼怪物\footnote{此乃獸心人質者也。}。那神稟道。此乃元世祖忽必烈所供養之國師番僧楊璉伽眞也\footnote{好國師番僧。}。閻君痛惡他發宋帝諸陵。每一日夜輪受十八地獄之苦。已三百餘年。閻君說他在陰曹受罪。世人不知也。送到大王臺下。叫他陽世受一番顯報。回來再受諸苦。王切齒蹙額道。這廝原非人類。叫他世間去。又要殺人淫人。如何行得。閻君旣送了來。只得叫他去走一遭。還叫他做個和尚。或可以稍有慈心。或不受其害\footnote{王誤矣。和尚而有慈心者幾人耶。這兩個或字。已是決無而僅有矣。}。若再凶淫奸盜。使其身爲齏粉。以飽鳶鳥犬豕之腹。回來再聽閻君發落。帶去。王對那神道。宿案俱完。你可去回覆閻君。倘有不合處。不妨改正。那神道。大王鐵筆之下。不但無寃人。而諸人亦自以爲不寃。復下來叩首道。小神辭去矣。恍惚之間。不見形影。到聽見了許多奇異\footnote{夾敍。到聽決不可少。}。正在驚疑之際。忽見一片金光。照耀半天。仙樂盈空。彩霧繽紛。異香馥郁。猛聽得半空中大呼道。天符下。只見那王忙趨下丹墀。俯伏在地。衆鬼判一閃。盡皆無影無踪。頃刻間。一位金冠黼黻天官從空冉冉而下。如世間所繪三官大帝之像。兩位金甲神人持節前導。到地傍列。天官立在殿陛中間。宣上帝玉音道。有明建文皇帝因永樂篡奪一案。屢控天廷。至今未結。今明朝氣運將終。前靖難諸臣。如方孝孺。景淸等。或係天星下謫。或係諸神下凡。應歷劫數者。已經歸位勿論外。其屈死諸人並首逆朱棣暨姚廣孝等助逆諸臣。皆着托生。了結前案。以造罪之大小定報。施之重輕。切勿過殺。以損皇仁。欽此。宣畢。騰空而去。霎時金光潛滅。仍舊燭影輝煌。那王復登寶位。鬼判依然羅列。王吩咐判官道。可將在地獄中永樂並有名衆犯都拘來。聽候發落。傍邊鬼判齊應一聲。眨眼之間。見一個沖天冠袞龍袍的人。面惡鬚長。眉愁臉苦。在前後有許多文武官員隨著。有戴枷鎖的。也有閒散著的。那皇帝站立階前。衆皆遠遠跪下。聽得那王道。適逢天符建文吿你篡奪一事。你家國運將終。你可托生。身爲逆賊。殘滅爾之子孫。破壞爾家天下。碎磔其身。稍償稔惡。當日是你費盡心力篡奪了天下。今日就使你混亂了天下。付與有德者。纔叫做善惡之報。如影隨形。今天上已生聖人。神器已有所歸。與你朱家無干矣。其助逆諸人。仍着托生隨你。皆受慘報。以舒神人之忿。那皇帝道。我是一個親王。也是奉玉帝勅旨降生的。我有何罪。復使我爲賊。況我當日欲淸君側之惡。效周公輔成王之耳。建文自己遜去。誤傳以爲自焚。彼時國利長君。我不得不徇衆人之情。今日爲何使我殘滅自己的子孫。破壞自家天下。負罵名於萬世耶。我縱有罪過。在生已不得其死。屍爲賊殘。僅存一腿。負痛至今二百餘年。也就可以爲報了。爲何還要我去受孽報。那王大怒道。你此言只好在陽世欺人耳目。今在我臺下。尚敢搖唇鼓舌。巧語飾非耶。你說要淸君側之惡。天地間之惡。尚有忍於你以臣而篡君位者耶\footnote{妙論。}。你說恥去做賊。你以臣子而篡天位。非賊而何\footnote{問得更妙。}。你說不忍殘爾子孫。那靖難諸人。他的九族十族難道不是他的子孫麼\footnote{何辭以答。}。爾當日殘毒若此。今日叫他人屠子孫。不若使爾自屠之更暢快人心。你說怕負罵名於萬世。當日方孝孺說你萬世之後免不得一個篡字。久矣有罵名了。又何在此\footnote{爲方正學先生吐一口氣。}。你說怕去怕(受)孽報。方孝孺敲牙抉舌而磔其身。鐵鉉以油鍋煠之。景淸則剝皮揎草。靖難諸公。無毒不備。你當年何不想人皆血肉之軀。他難道是不痛的麼\footnote{又爲靖難諸公吐一口氣。}。爾背君滅祖。毒害忠良。上天有好生之德。爾何殘刻若是。況且上帝命汝爲王。已恩隆極矣。又復奸天位。罪復何辭。且自古來篡弑諸人。至惡者莫過朱溫。至醜者莫如趙炅。其醜惡兼備而更甚者。則你一人而已。我今細剖一番。看你還有何辯。建文乃爾太祖親立之太孫也。太祖骨肉未寒。爾即篡奪之。是不孝也\footnote{是眞不孝。何所辯。}。懿文太子已久正位之儲君。又係你之嫡兄。爾旣篡其子。又去其孝康之謚。只許你做眞皇帝。哥哥死後的虛名也不許他領受。此是何心肝\footnote{眞問得甚妙。}。呂太后是你的長嫂。你更置他不得其死。屍骨無踪。且他一婦人何罪。你也太狠。這樣看起來。爾兄若在。爾亦必篡弑之矣。是不弟也\footnote{誅心之言。是眞不弟。何所辯。}。建文已正君位四載。繼嗣之天子也。爾竟篡奪之。猶以覓璽爲由。遣人遍天下以至海外物色。況他旣爲天子。普天之下孰非臣妾。豈有不知之理。又削其年號不錄\footnote{辱翁曰。已經歷過之年。實亦改去不得。即革除建文。仍紀洪武。後人言談不便。遂稱爲革除。則革除二字仍依然是建文也。是爲燕逆之不智。}。你以臣篡君的年號倒用得。他一個大公至正承嗣天子的年號反用不得。你是何算計。是不忠也\footnote{眞可笑。是眞大不忠。何所辯。}。據我看起來。你的年號倒該自己削掉。你纂位一場。反用叛賊方臘永樂的年號。明明以叛賊自居了\footnote{譏得妙。}。雖是你不學無術。正是天奪其魄處。眞正可笑。你今日尚有何言。也罷。你也是一座破軍星。免你肆諸市朝。此去爲鄕人撻死如泥。也就如受醢一般了。那皇帝滿眼垂淚。俯首無言。王喝道。鬼卒帶去。俟托生之期。送到陝西米脂縣李家爲子。以結前案。鬼卒答應一聲。扯洩(拽)而去。王又道。帶那高煦上來。鬼卒帶上一人。遙見略似人形。渾身上下竟是一塊烰炭\footnote{好一位漢王。}。王喝道。汝在生欲篡奪太子之位。助父爲虐。空負篡弑之名。徒爲惡死之鬼。爾前生旣係爾父之愛子。還隨他同去。做他的心腹愛。後死於鎗刃之下。以完前孽。那黑鬼道。我在生不過奉父命耳。雖篡了建文天下。皇帝又不是我做。況我生前被銅缸煉死就夠了。還要我去受一刀一鎗之厄。求大王寬釋罷。王大笑道。你助父叛君。尚未償報。何如算得。今去受刀鎗之痛。還算輕恕了你。更有何說。鬼卒可帶去了。同他父親先後托生到陝西史家爲男。但他的心腸都是黑的。這個黑形骸也不必變白了\footnote{若如此說。難爲了黑鬼子竺。}。來世還是一個大黑漢罷。說畢。帶去。又喝。帶那禿賊姚廣孝上來。鬼卒押過一個大胖和尚。那王拍案震怒道。你這賊禿。旣皈依釋敎。就當守你淸規。自幼奸淫好亂。就該下犁泥地獄了。後復逞你凶心。屢勸燕王篡逆。你去想一想。當日只圖你做一個開國元勳。獨不念殺了多少無辜之忠義。弄得個人族滅身亡。皆由你之作俑。我看你故鄕尚有你當年奸生之子孫在。今著你仍生姚家。旣爲爾孫之子。好酒貪淫。敗辱家庭。醜流後世。爾初受國恩。後復歸燕王造逆。還受賊封公爵。遂你生前之願。因而覆宗滅族。碎磔其身。仍剖棺戮你前生之屍。以報往愆。庶可稍快人心。且爲方鐵諸公稍雪其恨。速速帶去。勿久汚我之殿陛。一個惡鬼上前。伸手拿住脖項。按倒夾於胯下。只露一個光頭。像個大腎囊一般\footnote{若遇眊眼醫人。見之必曰。你如何生了這樣個大氣脖。若有厚謝。我當包替你治之。一笑。}。那和尚哭哭啼啼。如驢子一般爬去了。只見人叢中一個尼姑大喊吿狀。王大喝道。何物野鬼。擅敢到我臺下叫寃。帶過來。衆鬼卒如鷹鸇搏兔一般。拿到臺下。王睜目喝道。你是何鬼。敢吿何人。那尼姑道。小鬼在生原是極守戒律的一個姑子。從未犯色戒。被姚光(廣)孝百般引誘。遂成苟合。又替他生了兒子。他後來得了好處。把我棄擲不顧。因此抱恨而歿。今聽得大王爺命他轉生。我求同去。以報前仇。王笑道。你與姚廣孝通奸。是他引誘之罪了。你復私伊弟廣忠。是誰之過。我看你三人緣尚未盡。何(你)可去桂家。托生爲女。仍爲廣孝之妻。淫醜不堪。以報他前生負你之罪。再着廣忠托生爲廣孝之姪。爲你之私夫。了結前緣。俱免不得一刀。以正奸淫之罪報。那女鬼欣欣而去。王又喝。將一起從逆重犯都帶上來。衆鬼卒遂將一夥戴枷杻的人都推過案下。指着一個道。袁珙。你一相士耳。輙敢串通姚廣孝。勸那燕王爲叛。情殊可惡。今着你托生游混姓爲子。但你惡還未甚。姑免項下一刀。但遭癰疸惡病而斃。以報爾慫恿謀逆之罪。爾子忠徹。亦以相貌邪說。致害張昺諸人。乃成燕王之逆謀。其罪過於爾。乃着他爲爾之子。初受妻之毒虐。復罹極刑。以滅爾後\footnote{袁珙選擇二婿。一爲水淹死。一爲賊被殺。彼但談相。其妻即詈之曰。爾旣能相。何爲相這等兩個女婿。珙無以爲答。但云。我只能相其面。不能相其心。燕王之叛。實成於袁珙父子。此輩爲天下之害不小。}。又叫一人道。陳瑛。爾爲臣不忠。私下黨逆。爲衆人攻擊。建文赦而不誅。爾當感恩不盡纔是。你更反面是仇。仗爾蛇蝎之心。羅織忠良家屬。殘刻極矣。李友直。一小吏耳。漏洩軍機於燕逆。希圖佐命之功。獨不思爲爾一人之榮祿。害了多少的性命。你二人事雖不同。罪名總一。押去阮家爲子。陳瑛逢君之惡。難逃斷頸。李友直長君之惡。罪尚可全屍。然皆受妻子淫人。斬其血嗣之報。又叫李景隆。爾乃國之至戚。受朝廷厚恩兩世。爾督兵無狀。喪數十萬性命於沙場。建文宥爾不戮。恩莫重焉。爾反開門迎寇。不忠不孝出於爾一人矣。你私意要爲燕之功臣。不思燕王之忮刻。他的麒麟閣上如何容得你。與其後日死於他手。抱不忠之名於萬世。曷不同靖難諸人爲罵賊成人(仁)之忠魂乎。你不過因富貴這二字橫於胸中耳。今着汝托生與馬家爲子。奇蠢癡頑。人形獸性。雖擁萬貫之資而不知受享。雖爲顯宦之兒而如木偶。有父母而不識爲何人。有妻子而不知爲何物。係他人之種。嗣續暗地斬絕。仍死非命。以報你了。爾張信。建文以心腸待爾。授爾密詔擒燕逆。爾反以此爲進獻之功。今爾可托生勞宅。病體懨懨。後與袁忠徹同歸姚廣孝幕下。俱正典刑。以結前案。但張信之罪實成於伊母之言。其夫其子世受皇恩。奈何以死夫無稽之語。命子爲叛逆之事。因係女流。其爲無知。姑從寬。罰他去始爲大家之婢。終做賈人之妻。其餘朱能。張玉。譚淵。丘福。李彬等從逆諸文武。俱着各處托生。同歸燕王標下。或死或脫。論生前獲罪之輕重報之。又道。可將袁忠徹。張信。李景隆。李友直。陳瑛五人妻子。也着托生。仍配爲夫婦。皆各宣淫。以爲厥父不忠之報。一個判官上前稟道。查得袁忠徹生前無妻。何以報之。王想了一想。道。長舌婦也無夫。當年秦檜送了高宗。做了個不孝不弟之人。今日袁忠徹送了燕王。做了個不忠不孝之人。先後一轍。正好爲長舌之夫。就配了他罷。說完。喝道。都帶了去。衆鬼卒一擁上前。牽住鐵繩。盡皆悲啼。一陣陰風。倏然不見。王又命。將那些忠義文武叫上來。有數十人一齊上前跪下。王道。爾等忠魂義魄。俱起來聽我發放。衆人立起。王道。張昺。謝貴。人患不得其死耳。若死忠孝。又何恨焉。你二人被奸謀誘殺。已名載靑史。今張昺爾托生史家。後爲閣部。遣將殺賊。以洩生前之忿。後仍死於忠義。更流美名於不朽。爾可明不能善終之故麼。張昺道。某愚昧無知。求王見諭。王道。燕王之變。雖逆心已久。實汝衆人逼之速發耳。烏得無罪。汝雖死。而爲千秋所仰慕。便何憾焉。謝貴托生樂宅。位蒞尚書。殺賊功成。名垂竹帛。憂國勤勞。得終正寢。亦可報爾之前生了。瞿能已破北平。爲景隆忌功而不得入。平安鎗將及燕逆之背。馬蹶而不能及刺。天也。非人之尤。後以一陣亡。以一毒弊(斃)。葛誠爲燕藩長史。爾乃帝室。忠心未遂。反被橫誅。皀旗張勇冠三軍。奮不顧身。不幸陣歿。今爾等皆去托生。齊心殺賊。旣爲今時之義士。又報昔日之深仇。亦可以釋憾矣。瞿能托生林家。天生神力。勇猛絕倫。獨重爾者。以爾父子皆忠勇而亡之故耳。爾始祖爲殷之忠臣。萬載之下孰不知有比干焉。此林姓之所始也。爾此父又係今日之隱君子。故使爾爲之嗣。可乃心王室。報効國家。榮其身。以報爾父之隱德。爾此去勿負林之一姓可也。爾後仍死於沙場者。正所以令爾殺身全忠。垂令名於不朽耳。爾知之乎。瞿能大呼道。王恩厚矣。敢不盡心報國。王又道。平安托生慕室。武勇如前生。葛誠托生尚姓。爾原係文臣。今授爾文武全材。抱經濟之術。負衝鋒之勇。倡義殺賊。以遂宿願。皀旗張。爾生前好執皀旗。故得此名。可去托生國姓。今世則銀鎗素鎧。白色者金也。金有肅殺之氣。又能殺賊之意耳。爾道好麼。皀旗張道。大王厚恩。生生世世感戴不盡矣。王又道。瞿能二子。皆在幼年。便能捐軀報國。死於忠孝。今爾父子三人同生一處。雖隔世不能相認。一姓卓。一姓常。爲爾偏裨。協助殺賊。其餘陣亡諸將。皆係忠肝義膽。各擇善地受生。皆爲勇武之將。以復前仇。因向衆人道。我這斷判。你衆位心下何如。衆人異口同心(聲)道。荷蒙大王厚恩。我等皆心悅誠服。二百年之積憾。俱一時冰釋矣。皆歡欣舞躍。俯伏拜謝。王亦立起道。着判官備幢旛寶蓋。送他諸公去。忽見一土地咆(跑)得喘吁吁的。忙來跪下。稟道\footnote{此一轉尤妙。如元宵放大桶花。若一放即了。有何趣味。放完之後。又忽然另冒出一陣花火來。然後止之。方覺醒目。此一段正是此意。}。小神係建文時東湖樵夫。聞燕王篡逆。建文駕崩。我義忿塡胸。即痛哭投東湖而死。上帝憐小神一介編氓。有此忠心。即勅爲東湖土地。今二百餘年。此忿未消。聞大王着靖難諸公去復前仇。小神亦願附驥尾。幫助殺賊。以雪前生未了之恨。求大王恩允。王贊道。好好。你一個無官無祿之樵夫能死於忠義。使世間爲人臣而有二心者。置身無地矣。你旣願去。可往鮑家爲男。就同瞿能等同心殺賊。爾再生之時。有官有祿以榮身。有妻有子以居室。即將張信之母配你爲妻。爾壽考而終。死仍爲神。也可報你了。那土地笑逐顏開。再三叩謝。王道。你同他們一起去罷。只見一對童男女。手執幢旛引領衆人。一陣香風而散。到廳(聽)自思道。我非是做夢麼。想著這些說話。並這許多人衆。却是明明白白聽見看見。正在躊躇\footnote{處處拿到聽似夢非夢光景。方見得句句話件件事俱是他耳聞目覩。非白話也。此等極易忽略處而不肯遺漏。纔見作者之細心。}。心中甚是驚疑。又見傍邊一個綠袍紅鬚的判官。呈上一卷。如人間之文案。跪稟道。此係白氏的金重一案。上呈聖覽。那王看畢。就吩咐。帶那白氏上來。只見那個少年女鬼。面目如生。神情帶慘。然而體態輕盈。腰肢嬝嫋。雖所隔頗遙。燈影下見其嬌艷動人。容光飛舞。金蓮半露。款促湘裙\footnote{此處不但贊白氏之美。連後世錢貴都贊在其內。}。走到神案前跪下。王問道。爾陽壽未絕。何故來此。女稟道。女在生係本地白物好之女。父母只生女鬼一人。並無兄弟。因珍愛如寶。云比兼金尤貴。故喚乳名爲金重。生長二九。尚未適人。父母爲愛女心切。難於擇婿。女因摽梅期過。未免傷情。緣此情未遂。故抱恨而亡。王說道。汝父母旣鍾愛於你。爲何不與你早擇一婿呢。女稟道。父母見女頗有姿容。難求坦腹。欲覓一才〈子〉如子建。貌似潘安的人品。方肯許允。如此揀選。故爾難得。王笑道。似此議論。亦是愛女擇婿之常情。但姻緣自有天定。世事豈容人謀。爾父迂腐庸人。不足較論。但此等人物雖未易得。以爾之貌或不至於終棄。倘爲爾覓一才貌稍可之婿。亦未可知。爲何就至捐軀。女又稟道。天公最妒。不能全美。那才貌兼備的人。大約貧者居多。向曾有三人。雖敷粉何郞。豪吟太白。才貌也不多讓。但他家徒四壁。一貧如洗。雖女父慨然有允諾之心。而女鬼誓死無相從之意。王又道。才貌雙全的人。本山川之秀氣而生。一時也是難得的。因南京虎踞龍蟠。江山秀美。故生多俊。難道三人中就沒有一個中你意的。女道。以我之容貌。雖不能賽西子。壓王嬙。然選於今日美艷之中。亦可以自雄一世。雖不敢望以金屋貯嬌。安肯配蓬茅下士。一心欲嫁一富勝石季倫。貴如郭令公之夫。方纔遂願。女旣係一時絕世之嬌娥。故發誓要嫁一個敵國巨富之財子\footnote{眞奇想。}。王不禁大笑道。此事不特罕見。此語抑且罕聞。你不愛無貝之才。反愛有才之貝。眞爲可笑\footnote{舉世皆然。不獨此女可笑。}。我看你容貌若許。爲何具此一副俗腸。妍皮不裹癡骨。誠謬言也。然紅顏薄命。你旣有幾分顏色。焉能得配才郞。但城中富貴者頗多。你爲何又不嫁呢。女道。曾有一富家之子。姓黃名金色。家資巨萬。富壓南畿。慕女花容。曾求袒腹。在女傾心悅意。願效舉案齊眉\footnote{白物之女。作配黃金。理所當然。}。奈父執性不從。以致花殘玉碎。王問道。你父爲何不依。女道。父母說他形如傀儡。貌似修羅。故他家雖有好合之媒。而我家竟不中雀屛之選。女恨父母。難以明言。傷己身暗悲。奄蹇原不解。害相思而不覺相思害矣。本待要效鸞鳳。誰知鸞鳳分飛。今一命雖赴幽冥。九泉難免遺恨。王勃然怒道。你不知以才貌擇夫。反以銀錢求配。可謂目無珠矣。可惡可恨。女又稟道。黃家郞雖然貌醜。却甚情深。彼聞女之美麗。數四相求。父憎他之醜態。再三推執。彼竟思慕成疾。一病而亡。臨終惟呼女乳名者再。我聞之。故爲心死。因感他一種癡情。愈動我萬分想慕。古云。情之所鍾。正在我輩。又云。女爲悅己者容。彼旣爲我而死。我豈能捨彼獨生。下情若此。上聖鑒察。王道。論你初具嫌貧愛富之蠢念。本當永墮阿鼻。變豬變狗。憐你後有感情報德之深心。尚可保全人體。爲瞽爲娼。正欲判斷。只見三個文士。衣巾破敝。面貌淸奇\footnote{癡肥者多鮮衣怒馬。淸奇者盡衣巾破敝。眞令人不解。}。共持一狀。上呈神案。長跪訴道。念某等在生時。腹富三冬。胸藏二酉。不得飛騰黃甲。空自困守蓬茅\footnote{學富者困蓬茅。肉食者享富貴。千古同聲一哭。}。未蒙賢守宰之吹噓。反爲癡女子所擯棄。慕色雖非正道。好逑自是人論(倫)。各害相思。抱思而歿。情實難甘。故同上吿。王將他三人文狀看了一回。大笑。反怒道。爾輩讀書人具此才華。焉知非瑚璉之器。有斯品格。豈料匪梁棟之材。爲何輕擲此軀。自棄若此。所謂雖讀書而猶未知書者也。今雖一死。尚有可憐。不過供人笑哂耳。正說間。只見又有一持狀者。面貌猙獰若鬼。身軀彷彿如人。自稱姓黃名金色。呼寃不已。情色慘然。王問道。爾有何寃。所吿何事。那人道。鬼在陽世。慕白氏之咨(姿)容。苦懇萬端。白氏亦羨小人之富厚。樂從一諾。奈他父母只愛那才貌兼優。指指說靑雲有路\footnote{此則大謬不然。}。孰知我金銀滿庫。看看就紗帽籠頭\footnote{財旺生官。自古同然之理。}。以一不識時務之老迂。致害我一對妙齡之蟻命。況鬼在生時。雖然貌醜。却甚心良。惡並一無。善皆萬積\footnote{有此數語。方可再世爲才貌兼全之人。非無因也。}。今受報若此。情甚不甘。且人命關天。願求追斷。王聽罷。援筆判曰。

\begin{quotation}

白氏金重。艷色如花。癡心似水。不思嫁才貌兒郞。但願配銀錢子弟。妍媸莫辨。貧富是論。未嫁女即害相思。婦道可知矣。擇丈夫尚圖富貴。親戚何有哉。本當押入酆都。今且從寬譴謫。旣愛金銀。應與錢家做女。不分好醜。當使瞽目爲娼。恨其自負嬌容。想殺才人三命。初做賤妓。償還宿債。憐其以後矢貞。能爲醜子捐軀。終爲良婦。了却前緣。今生誤愛富兒。再世當求才子。黃金色自恃富豪子弟。苦苦求妻。白家翁只重才貌兒郞。殷殷却婿。以致彼緣未遂。此命是捐。查彼貌雖醜惡。心實善良。今著彼托生陽世。與錢氏初諧露水之歡。後遂于飛之願。才貌兼優。以掩前生之醜。家徒四壁。以報恃富之橫。錢氏作配鍾情。鍾有貌而瞽女不能見貌。要知色即是空。鍾情固得錢氏。縱得錢而貧士仍舊無錢。方是空能得色。雖嗔他性墮癡愚。尚念彼情猶可憫。法外施仁。故從寬貸。至此三生。具此才華。不知自檢。旣自恃才貌。使托生愚蠢癡頑。以報自棄之罪。又怨恨貧窮。使再世豪華富足。以償苦學之勞。咸配淫醜悍妒之妻。以懲好色輕生之戒。爾大衆與錢氏買笑追歡。了却前生宿願。你諸人須自己回頭是岸。勿結來世寃愆。鐵筆無私。照判發放\footnote{以上一段全是對偶句。一部書所無者。}。

\end{quotation}

寫筆(畢)。發與判官。判官高聲宣白一遍。那王又叫道。帶那三獸上來。只見鬼卒帶過一隻尖嘴母猴。一隻咆哮牝虎。一隻鐵黑雌狐\footnote{妒婦原身。幻想奇絕。}。伏在案下。若有所訴。王道。爾三畜前生孽重。致變畜生。罪恨已滿。今着轉托婦人。配此三生。獸心雖不能全革。若不傷害性命。來世尚可保全人體。不然又墮畜道矣。着鬼卒送他到轉輪殿去。那三獸連連點頭。如叩謝之狀。擺尾搖頭。回盼三生。欣欣然隨鬼卒而去。判官在傍呼喝。將前之判文傳與鬼卒。隨亦將衆人帶去。倏忽雞鳴。驀然不見。展轉之間。不知東方之旣白\footnote{住得好。赤壁文風甚是可笑。}。到聽凝神自思。宛然在目。回憶前語。一字不忘\footnote{好記性。}。正在驚訝之際。値廟祝出來開門\footnote{廟祝。}。見了到聽。驚問道。你是甚麼人。爲何夜間存在此處。到聽訴說昨晚酒醉家遙。故而在此睡倒。因將夜來之聞見。備述一番。廟祝聽了。以爲跪(詭)辭。大笑而去。到聽自己以爲一件奇事。每遇見親友。無不相吿。雖於途中遇一面之識的人。亦詳細道之\footnote{這方應他大號圖說二字。}。衆皆不以爲然。以其平素好傳新聞說白話之故\footnote{這又應他毛空的別號。}。人雖不信其實。亦皆以爲奇談。轉相傳說。有一種與他同類。亦好道聽途說者。四處談講。竟普傳於白下。至今里老猶有能言之者。這是後話。且說那到聽。一日在稠人廣中(衆)之衆(中)。\footnote{這更有許多的閒漢。閒漢四。}高談濶論。講這一段新聞。正說得興頭。內中一個少年問道。兄這些事醒着聽見的。還是睡着了夢中聽見的。到聽道。我是醒着聽見的。那人道。兄此時是醒着說話。還是睡着了說話。到聽道。你這位兄說話稀奇得很。大靑天白日。我站在這裡說話。怎說我睡着了。那人道。兄不要見怪。你旣是醒着。爲何大睜着眼都說的是些夢話\footnote{大睜着眼說夢話的人。正自不少。}。衆人哈哈大笑\footnote{哈哈大笑。}。到聽纔要分辯。又一個道。不是這樣說。兄這些話是獨自聽見的。還是同人聽見的。到聽道。半夜三更。就是我一個。那裡還有別人。那人道。兄自己錯了。怪不得人說。到聽道。我怎麼錯了。那人道。兄方纔說看見有許多判官小鬼。該把那判官也罷。小鬼也罷。拉住一個做個證見。此時這些鬼話。就不怕人班駁了。你不曾想到這上頭。豈不是錯。衆人拍手打掌。又笑了一場\footnote{拍手打掌的笑。}。到聽發急道。我是千眞的話。你們當我說謊。這樣省剝我。內中有認得他相厚的便道。毛空。你旣要說新鮮謊。老着臉憑人說罷了。又急得是甚麼。又一個道。這位原是(來)就是有名的到兄。面荒失敬。我們大家說頑說。兄不要發急。等我替兄尋個證見。包管他們再沒得說了。到聽當是好話。笑着道。兄替我尋個甚麼證見。那人道。兄那日在那個去處聽來。到聽道。我在大門內泥馬脚下睡醒了。聽得這些說話。那人向衆人道。如何。我知到兄決不是假話。列位都這樣白他。這不有了證見了。衆人道。誰是證見。那人道。他說在泥馬脚下睡的。那不有個拉馬的馬夫站在那裡。我們同去問他。是眞是假就明白了。何須大家只管辯駁。衆人道。那馬夫是個泥人。怎會說話。兄也來跟着說新聞了。那人道。列位有所不知。我去問他。正要他不會說話纔好。若是會說話。他也要說到兄是扯謊。越發講不淸了。衆人聽了。笑得幾乎打跌\footnote{起初是哈哈大笑。次是拍手打掌的笑。此是笑得幾乎打跌。寫笑亦有層此(次)。寫得好。}。到聽要辯。又說不過衆人。不辯。又氣得慌。臉脖子通紅。頸子上的筋急得有指頭粗疊暴着\footnote{畫出一個發急人的形像。}。只見人叢中走出一個道士來\footnote{道士。這道士也是一個閒漢。閒漢五。}。上前笑着道。天下奇怪的事何所沒有。這位居士也未必全是謅出來的假話。或有些影兒也不可知。列位何必如此認眞。若信他是眞話。就聽他這一遍新聞。若疑他說鬼話。就不必信。人還拿着錢給說書的。聽瞽(鼓)兒詞上的瞎話。如今聽說這新鮮話又不要錢。何等不樂。只管班駁些甚麼。衆人看這道士。兩道濃眉。一雙大眼。五尺身材。四旬年紀。竹冠布氅。麻履絲縧。好一個齊整相貌。衆人道。這位師傅說的是。我們打柴的不要跟着放羊的。各人做各人的事去。一轟而散。到聽垂首喪氣。也就要走。被這道士一把拉住。道。居士且住。到聽道。師傅叫我。說甚麼。道士道。古人說。惱一惱。老一老。笑一笑。少一少\footnote{此十二字。便是延生秘訣。}。大家頑笑。何須認眞。氣惱的是甚麼。我同居士去小飮三杯。消消閒氣。到聽聽見請他吃酒。氣惱全無。一臉的笑。光(先)嚥了兩口唾。然後說道。今日腰中不曾帶得一文。改日請師傅罷\footnote{已暗含着今日且奉擾五個字。不曾說出。妙極。}。道士道。我請居士。何用你破鈔。拉着手到一個酒肆中去。到聽口中說道。豈有此理。怎麼好擾師傅。雖如此說。那兩隻脚已隨着到酒店中來了。對面坐下。走堂的送上兩壺酒。幾個小菜碟擺上。到聽等不得他讓。先一氣飮過了數杯酒。方纔問道\footnote{飮過數杯方問話。畫出一個好酒饞吻的人來。}。師傅貴處是那裡。在何處住。我每日在這裡走。從未曾會過。道士道。貧道祖籍陝西固原人氏\footnote{會採戰。自然能固本還元。所以是固原人也。}。自幼在峨嵋山投師訪道。近來四處雲遊。爲人治病\footnote{看官記着。}。今到此不多幾日。在朝天宮作寓。獨坐甚悶。出來閒步。纔見居士生氣。故約來同飮幾杯。我們說說白話\footnote{正投到聽所長。}。也可消遣。又讓他吃了幾杯。道。我寓處也無伴侶。居士若無事可常到我敝寓來。別無他物。就是一杯水酒相待。到聽滿臉堆下笑來。道。有了酒吃就儘夠了。我聽得人說。無鈔一身輕。有酒萬事足\footnote{學套文字。不意到聽亦善此。}。別的還想甚麼。若承師傅不棄。我來奉陪。我是閒着一點事也沒有的。道士讓他吃酒。他也吃過有兩壺。把白話口袋打開了。講天說地。論古談今。都是不見經傳。稀奇古怪。無影無形的天話。他說得津津有味。道士聽得倒也耳中爲之一新。微微的笑着聽他謅說。又同飮了數杯。到聽口也說乾。等不得他讓了。自斟豪飮起來。杯杯一乾到底\footnote{古詞云。杯行到手莫留殘。亦同此意。}。吃了一會。方覺得有些不好意思。反客爲主。一鍾一鍾的倒讓起道士來\footnote{到聽豈不聞癡客讓主乎。}。道士的酒量頗雄。鍾鍾乾過。二人又飮了多時。到聽有了八九分酒意。覺得滿到喉嚨跟前。不下去了。纔起身道擾\footnote{古人云。人生有酒須當醉。云。不飮。傍人笑我。到聽兼有之矣。}。舌頭短短的。不明不白說了幾十遍。道士會了帐。同他出來。他晃晃蕩蕩的去了。次日。到朝天宮尋着了道士。一來奉拜。二來道謝。道士又留他吃了半日酒。他無以爲敬。不過說些白話以答盛情而已。道士聽他說的。倒也不覺寂寞。臨別時。道士道。居士無事可常來閒話。他滿口應諾而去。到聽吃着了甜頭。他又是個無事的閒身子。況他要到街上來。必由朝天宮後門卞公祠過\footnote{晉朝卞壺死難之地。墳即在此。建祠祀之。}。所以他無三日不來。來無不醉。他吃得多次了。一日。聽得各處桃花盛開。他在史家墩。小桃源。黑龍潭。虎踞關各處去看熱鬧。見那些男男女女看花之人往來如織。別人都是三五成羣。有攜着春盛的。也有擡着食盒的。或在酒棚內飮酒的。或在茶棚內吃茶的。絲竹管絃。長歌短調。其然熱鬧\footnote{這一夥閒漢更不計其數。閒漢六。}。看了一會。眼飽肚飢起來了。他因囊中無鈔。四處混撞。忽然到一棵桃樹之下。見金晃晃一件東西掛在上面。忙近前取下來一看。是一枝鍍金銀花。也不知是那個婦人在花下過。掛了下來的。他滿心歡喜。也不看花了。欣欣然袖中。回來到家中。取出估値。道。這個也値七八錢銀子。五錢銀擡一大罎酒。剩的買些柴米。夠我幾日大醉。想道。不好\footnote{一算不妥。}。目下天氣漸暖了。買件單衣服穿穿是正經。又想道。也不好\footnote{再算又不妥。}。我擾這道老爺多次了\footnote{江南僧道尼姑皆稱之曰老爺。而縣中知縣反稱縣裡大爺。}。也有些不好意思。不如請他一請。還了席。後來又可以擾他幾十次。這樣一本幾十利的事爲甚麼不做。就是這個主意好\footnote{三算方成。可謂三思而後行矣。一笑。}。況且是人說的。吃在肚裡是細絲。穿在身上是九成。我放着細絲的事不幹。倒做九成麼\footnote{此等算計的人不少。}。只當是不曾拾着這件東西。又算計道\footnote{此一算是算字餘文。}。家中碗盞鍾碟一樣沒有。是來不得的。酒館中肴饌又貴。不如買兩樣擋戧的物件。這兩日接引庵碧桃盛開。請他到那裡坐坐。小姑子又是我的厚朋友\footnote{玉簪記舟子說陳妙常云。我小老兒活了六十九。不曾見姑子同秀才做朋友。今這小姑子是到聽的厚朋友。可見亦非異事。}。問他要茶要水燙酒還便宜些。定了主意。明日舉行。且說這接引庵在旱西門北首一條小僻靜巷內。門口一叢黑松樹。一個小小的圓紅門兒。進去裡面甚是寬敞\footnote{昔人題半截美人圖云。堪笑良工無見識。動人情處不曾描。今未見其人。先寫動人情處。若遇前詩人做試官。定考第一。此門中乃和尚出入之所。今到聽竟要請道士進去。奇事。}。內中三間大殿供着接引菩薩。東西六間廂房只有兩個姑子。東廂房是兩明一暗。兩間做客位。一間是那老姑子的臥房\footnote{姑子。}。這老姑子有七十多歲了。動擔(彈)不得。成年家睡在床上。西廂房內一間做廚房\footnote{後姑子張道士弱(溺)尿處也。}。一間做庫房。一間是小姑子做臥室。這小姑子纔有十八九歲。雖不叫做奇醜。却也說不得個俊字。肥胖胖的一個團臉。深紫棠色。五短身材。圓滾滾的却胖得緊。就做人甚和氣。見人滿面春風。一臉的笑。到聽家離此只有三四箭遠。時常來隨喜。大約與這姑子有些暖昧的帳。人却不得而知。且說到聽次早起來。把那枝花拿到錢鋪中去換。雖然大樣。是纍絲的。稱了稱。只得七錢多重。首飾做八成。換了六百文錢。買了一隻大板鴨。一個爛熏蹄。並些果子。又買了些好茶葉\footnote{細。}。一直到庵前敲門。那小姑子來開了。笑嘻嘻的道。你今日買這些東西做甚麼\footnote{是個相厚問的聲口。}。到聽進來。小姑子關上門\footnote{一絲不漏。}。也隨了進來。到他房中。到聽道。我今日要請個人。借你這裡賞賞花。煩你收拾收拾。再把樹底下打掃打掃。改日我腰裡用些勁酬謝你。那姑子笑着瞅他一眼。道。你肥肉能吃得幾塊。好像根豆芽菜兒似的。不要討我貶別你了\footnote{大形容不堪。似此較之。那道士之物只算得一根芹菜。}。說得到聽笑着把他脖子摟過來。親了一個嘴。道。你且不要關門。我去買了酒來。少頃。又拎了一小罎酒來。道。你就預備下。燒好了茶等着。吃過早飯我就同人來了。說着走出。便到朝天宮來。這道士正要吃飯。見他來。讓了坐下。道。這兩日爲何不見。今日來得甚早。便飯且用一碗。到聽道。這兩日花開的盛得有趣。〈偏師傅〉我去看了看。所以沒有來。望得今日。我備了一杯水酒。請師傅去賞賞花。那道士道。居士是那裡的錢。我怎好相擾的。到聽道。師傅在客邊。我倒擾過幾十遭了。論理也該還還席。沒有甚麼東西吃得。不過看花而已。我都預備下了。師傅用了飯。我們早些去頑頑。道士見說買了東西。知他是實心相請。便不推辭。說道。我領情便是。只是心中不安。讓他同吃了飯。道士鎖了門。一同出來。二人說着閒話。慢慢的步着到接引庵來。不多時。到了門首。到聽上前敲了兩下。等了一會。不見裡面嘖聲。道士道。何不再敲幾下。到聽笑道。師傅你是外路來的。不知南京城姑子庵的暗號。先敲兩下。應着開門兩個字。等一會再敲三下。是快開門三個字。他自然來開。若一陣亂敲。他聽見知是外行。再不肯開的\footnote{確是個姑子厚朋友的說話。}。說着。又狠狠的敲了三下。只聽得脚步響。一個小姑子把門開了\footnote{此是道士聽得看見。若到聽。不待開門。便知是小姑子矣。}。笑嘻嘻的道。我收拾後院子來。先敲門就沒聽見\footnote{妙極。照前。開門兩個字不曾聽得也。}。那道士把他一看。心中一動。道。好個爐子。是絕妙的鼎器。到聽讓了進來。到東廂客屋內坐下。少刻。小姑子送了茶來。他心愛上了這個道士。好個儀表。目不轉睛看着他笑\footnote{先寫衆人看這道士好個相貌。不過一看而已。此處寫這姑子。一見他好個儀表。便有就交之意。隱寓許多男人不及一個姑子之眼力。直貫到鍾生貧窮爲親友所棄。獨一個瞽錢貴能識之也。此是後文的一個影子。看者須知之。}。道士也有了他的心。望他笑了一笑。不住拿眼睃他。吃罷茶。說了些閒話。讓到後院。打掃得果然潔淨。道士看那花時。有七八株都有一抱粗。花朶比茶鍾還大。紅白燦爛。開得甚是好看。樹下鋪着蘆蓆。上面墊着氈子。二人席地而坐。不一時。送了果肴來擺下。那姑子又去拿了熱酒來。到聽斟了一杯。送與道士。道。沒有甚麼請師傅的。不要見笑。道士接過酒來。道。居士這等費心。何須客話。二人說話飮酒。吃了多時。那姑子穿梭也似。兩頭拿酒服事。道士道。小師傅。勞動你了。我們不安得很。你請坐坐。那到聽忙起身。篩了一杯酒讓他。他笑道。我不會吃。就要跑。早被到聽拉住袖子。道。這位師傅不是外人。你吃一杯怕甚麼\footnote{到聽之於姑子亦外人也。而此云這位師傅不是外人者。儼然以野家公自居。寫得甚妙。}。送到他嘴上。他推辭不得。纔要飮時。被到聽一灌。只得嚥下。到聽放手。他跑了去了。二人又飮了幾巡。道士要散步散步。起身到園中各處走走。走到西牆角一個小欄中看看。不防那小姑子蹶着滾圓的一個黑屁股。背着臉那裡溺尿。衣服摟得高高的。自己低了頭看着他的陰戶\footnote{昔有一孀婦臨嫁洗浴。低頭看着牝戶道。鬍子鬍子。今晚你有肉吃了。此時姑子看他的陰戶。大約也道。肥嘴肥嘴。你幾時纔有肉吃呢。}。因他屁股蹶得高了。那一件肥物全全露着。正對着道士的眼\footnote{一隻無珠的大眼。對着兩隻有珠的小眼。好笑。}。道士一看。眞正一件好東西。牝峰老高的凸起。宛然一個大饅頭上裂開了一條細縫\footnote{饅頭倒好。可惜麵黑些。}。他一回頭看見了道士。笑着忙扯衣服蓋住。站將起來\footnote{這姑子有心乎。無心乎。試猜之。}。道士也笑笑撤身退出。那姑子繫了褲子出來。望着道士嘻嘻的笑\footnote{寫生手。}。往前邊去了。那道士也回來坐下。到聽讓着又飮。那姑子送了酒來。看着道士只是笑。道士恐到聽看見。也一面笑着。一面同到聽說閒話\footnote{寫得二人活跳。}。飮到日色將暮。道士起身謝別。到聽款留不住。道士又向着小姑子道。小師傅。勞動你了。改日酬情罷。他只嘻嘻的笑。也不說甚麼。到聽送了道士出門。復身進來。拉着小姑子同飮了幾杯。二人相摟相抱。一時興發。到聽就去扯他的褲子。那姑子也正騷到極處。任他褪去。到聽爬上身。抽了三五下。早已吿竣。原來到聽自做主人。過飮了幾杯。不能自持。竟從門流涕。那姑子正然興濃。見他纔挨着早已完事\footnote{豆芽菜原沒甚趣味。無怪乎乃爾。}。急得叫道。你掙着命再動動是呢。到聽再要抽時。陽物已稀軟縮了出來\footnote{豆芽原軟。}。姑子十分情急。在他項上咬了幾口。身上擰了幾下。到聽也甚覺沒趣。起來同他收拾了傢伙\footnote{細。}。含愧而去。却說那道士回到寓處。心中想道。這個姑子看他那種光景。大有情意在我。況他是件寶物。難得相遇。不可輕放過他。須如此行事方妙。原來這道士旣會採陰。又善煉汞。他有的是銀子。四處雲遊。遇着有好鼎器。他就採補一番。今日見了這姑子是個妙物。他遇過的婦女甚多。好歹一見便識。却不揀醜俊。他留了心。次日飯後。獨步到庵中來。記着昨日到聽的話。只將庵門敲了兩下。只見那姑子來開門。見了是他。笑臉相迎\footnote{親熱。}。心中暗喜。原來這姑子因生得黑醜。無人愛他。雖然相與了一兩個契闊。都不過是到聽之類。他昨日見了這道士生得相貌魁梧。心愛得了不得。剛剛的在那裡溺尿。又被他看見了風流穴。竟有個要就交之意耳\footnote{大約少年姑子無一個不願與人就交者。}。所以昨日故做騷態。只是望着他笑。又被到聽引動淫心。不想一場掃興。眞是慾火如焚。眠思夢想。夢魂顚倒了一夜。今日見他獨自走來。心中猜了個八分\footnote{老見家。}。定然有些妙處。故此暗暗歡喜\footnote{這一喜是喜其好事在邇。}。忙道\footnote{這個忙字是喜極語。}。師傅請裡面坐。道士進來坐下。他道。師傅坐坐。我去燒茶。道士道。我不用茶。倒去看看花罷。他道。旣然這樣。我拿個東西去坐。遂到房中拿了氈蓆。同一床小獨睡褥子。到樹下鋪好。讓道士道。請在褥子上坐。還厚些\footnote{雖是心中。或更有他。}。道士道。小師傅。你也請坐坐。他笑道。師傅請坐。我不消得。道士道。你請坐了。我有話說。儘過一頭讓他。他笑嘻嘻就坐下。道\footnote{旣肯同坐。已無所不肯矣。}。你和我說甚麼話\footnote{你我二字。親愛之甚。但太怎麼早些。昔有一女子私問嫂子道。我明日嫁去。叫你姑夫做甚麼。嫂道。先不過你我相呼。等生了女兒。便指着孩子叫大兒老子。此女嫁之次日。新婿帽子被門簾掛歪。女呼道。大兒老子。你的帽子歪了。與此姑子你我相同。}。道士道。賞花不可無酒。買得些酒肴來麼。他道。酒還可以買來。只說有俗家奶奶們來賞花。打酒請他。還可以得\footnote{此係姑子沽飮之法。}。葷菜如何好去買。道士聽說。在腰取出一包銀子來。打開看。約有二三十兩。拈了一塊。別的付與那姑子。道。你拿錢數銀子。替我打些好酒來。別的你就收着。他笑道。金銀不過手。我怎好收得。你稱些我買去罷。道士笑道。多大事。你若要。就全送你也有限\footnote{姑子中不愛色者或有之。再無不愛財者。道士又以利動之。可謂算無遺策。}。他笑道。我也沒福要這些銀子。道士遞與他。他也就接着。道士道。你去打酒。我去買菜。你若先回。不要閂門。他要了一個筐子。拿着出去了一會。買了許多熏雞臘肉。燒鴨熟蹄。並上好果品。滿滿一筐。推開門進來。閂好了門\footnote{細。}。只見小姑子在西廂房門口站着。道士拿到他跟前。道。小師傅。\endnotemark[2]煩你整理整理。小姑子就到他房中。道士也隨了入來。道。原來你的臥房在這裡。小姑子見了許多果菜。笑道。你就買這些東西。要請客麼\footnote{明知故問。何不道。要請安麼。}。道士笑道。就是特特請你。二來替你昨日酬勞。他笑道。我們僧家是不用葷的。道士笑道。你就破破戒也不妨。我見別處的女師傅。不要說吃葷不論。連甚麼都是不戒的\footnote{妙語。却係實話。}。那姑子瞟了他一二眼。笑着道\footnote{騷態動人。}。不當家羽化的枉口拔舌。你看見來。說着。就忙去料理。道士走到花下坐了一會。到西牆小欄中去小解。只聽得北窗內笑了一聲。道士往內一張。是小姑子正在廚房燙酒。聽見窗外響聲。向外一看。見道士捏着陽物溺尿。他故意笑了一聲。好與道士知道他在那裡賞鑒\footnote{昨日姑子之物在此處被道士看見。今日道士之物也是此處被姑子看見。正可謂之還席。姑子這一笑。余因想起一笑談。家母女二人。其母有事出門。囑女兒道。對門那小子壞極。你切不可被他討了便宜去。女應諾。抵暮母歸。問女兒道。我去後。那小子可曾來。女兒道。他來了。今日却被我討了他的便宜。母問其故。女道。他一來就抱着我親了一個嘴。被我把他親了二個。他把我抱到床上。扯去褲子。弄出許多血來。母驚道。你這樣吃虧。還說討了他的便宜。女道。我不曾說完了呢。過了一會。我把他的腦子都夾了出來。豈不得了便宜。這姑子大約也想占這便宜。}。道士鑒貌辨色。知道好事十有八九。心中暗喜。轉身到花下。只見他捧了一個大托盤。碗碗碟碟擺了許多。又取了酒壺。一雙箸。一個杯。道士道。小師傅。你請來坐着。他倚着棵桃樹站住。笑道\footnote{古詩云。人面桃花相映紅。但這姑子面黑。不合。}。我不坐。道士拉着他袖子。道。我原是請你。你不坐就掃興了。他也就笑笑坐下。道士斟酒敬他。他不肯接。道。我不會吃。你請自己受用罷。道士強遞在他手中。道。你昨日怎麼吃來。今日又假推辭。他道。我再取個杯子來。道士道。不消了。就是這一個輪流吃罷。他笑嘻嘻也就吃了\footnote{合巹猶用雙杯。他二人只用一杯。更親熱。}。道士又斟了一杯。送在他口邊。道。好事成雙。再一鍾。他道。你也吃一杯。道士強送到他嘴上。他喝了一口。被道士拿過。一口飮乾了。道。好香甜\footnote{調情老手。}。有趣。他笑着瞅了一眼。又讓他吃菜。他先不肯。道士再三相讓。他也就不辭。吃了一會。這姑子三杯落肚。有些酒意了。烘動春心。兩隻眼水汪汪的也(乜)斜着\footnote{上眼如此。下眼不知何如。}。道士又讓他吃酒。他笑道。我不吃了。吃多了不好過\footnote{因此句想起一個笑話來。一女子在門口閒立。適兩個少年過。一個道。好個女子。只太月巴子些。那一個道。可好一個毛非。這女子進來問娘道。方纔兩個人。一個說我月巴子。是怎麼說。娘道。他說你身子肥胖。女又道。那一個又說我好個毛非。是說甚麼。娘不好說。謊答道。看見你手上有幾個疥瘡。女信爲實。一日。母女往鄰家赴席。主人讓他飮酒。女道。我不敢吃。吃了毛非會癢的。大約姑子也是怕毛非癢。故說吃了不好過。}。你請用罷。笑嘻嘻反儘着讓道士吃。道士見他這個騷樣子。也有些耐不得了。望着他笑道。你不吃只是讓我。我吃醉了回不去。看你怎麼樣打發我。他笑着道。回不去就在這樹底下睡。道士道。這園子空。沒人做伴。你要肯陪我睡。我巴不得不去呢。他把眼睛瞟了瞟\footnote{騷態可掬。}。笑着也不做聲\footnote{這個笑字也有刻不容緩之意。}。道士又強讓他吃了一杯。他推辭道。我的酒實在夠了\footnote{昔有一女子問娘道。人開口就說酒色。酒是吃酒了。色是甚麼。娘不好答。說道。色是吃飯了。一日往親戚家去。備酒飯相待。飮過數杯。再讓他。他道。我的酒夠了。倒是色罷。大約這姑子亦是此意。}。道士看他那光景。也有了五七分酒意。臉上紅紅紫紫。眼睛餳瞪瞪。不住嘻嘻的笑。暗想道。火候到了。下手他罷。便道。你旣夠了。我們歇一會再吃。就站起身來。那姑子也便立起。道士佯醉。假裝站不穩。往他身上一倒。小姑子當他醉了。上前一扶。道士就勢撲到他懷中。剛剛的嘴對了嘴。親了一下\footnote{有一舊笑話。一男子途遇一婦。上前摟住。親了一嘴。婦人大怒。男子道。奶奶息怒。我恐你要如此耳。在我何須如此。大約道士亦恐姑子要如此耳。}。姑子笑着將他擰了一下。道\footnote{浪極。}。我好意扶你。你倒這樣不識好\footnote{好意扶他者。原圖此好意。}。道士一把摟住道。旣承你好意。我再親幾個。那姑子只是笑着推。也不動怒。道士見事有可成。就伸手要摸他下身\footnote{道士要享用饅頭了。}。他用手攔着道。我叫起來。你就了不成。道士那裡聽他。把他抱住。放倒在褥子上\footnote{此時纔正經用着。拿來與道士墊坐的。反是自己墊着睡。不知先拿來時是有意否。}。壓在身上。連親了幾個嘴。道。你同我相與\footnote{也有要做朋友之意。}。我有大好處到你。補你的情。那姑子也情動了。不嘖聲。道士趁勢扯他褲子。他再要假掩時\footnote{假字刻。}。已被褪下。露出肥臀來了。他只閉着眼笑。道士忙取肉具弄將進去。肥美至極。一連幾聳。盡沒至根\footnote{一部書若許奸夫淫婦。却以一尼一道開首。見此輩中能持戒律者少。大書之。爲彼等下鍼砭耳。}。道士伏在他身上也不動。那姑子見他弄進去之時也不多大。過了一會。裡面翻滾熱起來。脹得滿滿的。那龜頭在內中如蛇吐信子一般。不住亂戳。麻癢難當。嘻嘻的笑個不住。他初嘗這種異物。頃刻就丢了一度。道士把陰精吸了個乾淨。定了一會。又是那樣亂鑽起來。只見他屁股扭着。兩眉皺着。似有些難忍的樣子。矇矓着眼睛只是笑。不多時又丢了。道士覺得這一次陰精更多。吸了個暢快。那姑子一連丢了兩次。渾身痛快。說道。夠了\footnote{酒夠了用色。色夠了用何物。}。拔出來歇歇罷。道士笑着道。沾住了。拔不出來了。他道。你讓我歇歇逗逗氣。怎麼只是皮臉。道士道。你就拔了。看他兩手推起道士來。屁股往後褪。果然陽物在陰中脹滿了拔不動。姑子急了。道。這怎麼樣好。你使些力拔拔呢。道士笑道。我沒力氣。你上我身來。用力拔了看。抱着他一翻身到了上面。騎在道士身上\footnote{先是道士騎驢。此時是尼姑騎牛。趣。}。兩手按着道士肩上。雙膝跪住。儘力往上拔。沾得死緊。他把屁股亂扭混扯。撐得陰門生疼也拔不出來了\footnote{道士後與昌氏交接。並淫姚宅諸婦。再未見如此。獨這姑子如此者。何故。他兩個是開首的奸夫淫婦。謂他鏈在一處如狗之交耳。借此兩個。罵盡一部書中之奸夫淫婦皆是狗之一類。故後不復寫。}。道士道。你且睡在我身上。少刻自然會出來。你急得是甚麼。他只得伏下身子。道士把他摟緊。叫他伸過舌頭來。緊緊含住。陽物在戶中又是一陣混鑽。覺得他舌尖冰冷。又丢了一度。裡面陰精更盛。道士吸得他興足了。放了一口氣。道。你再拔拔看。他探起身子。屁股加力。往上一擡。聽得不洞一聲響。好像小孩子們喞了一個水砲。早已拔出\footnote{小說中之寫淫事多矣。未有如此奇喩。}。姑子把他陽物一看。嚇了一跳。長有七寸多些。根子底下粗不過一圍有餘。上半截竟像一根大菜瓜。所以內中塞滿陰門。却脹得不痛\footnote{此所謂一個小圓紅門。裡面倒寬敞也。}。先是他閉着氣。其堅如鐵。陽物粗。陰門小。就如狗鏈幫一個理\footnote{恐人看不出。特特提明。余前評是否。}。故此拔不動。放了氣。略綿軟了些。所以一拔就出了。姑子道。你怎麼有這麼個稀奇東西。先也不多粗。怎麼一會就長成這麼個硶樣了。道士道。我是煉成的活寶。可大可小。先起弄時一送便入。着了陰氣就長大了。他是就着女人陰戶的。女人內中多深多大。他就長多粗多長。就是沒有破身的女兒也弄得。就是任你多深多大的陰戶也弄得。那姑子喜歡得兩手捧着\footnote{寫出愛極。}。細細賞玩了一回。不忍釋手。道士道。我也見過許多婦人。你的這件東西也是一個寶貝。姑子笑道。這件扁東西那個婦人沒有一個。怎麼見得我的是寶。道士道。別的婦人弄頭一次陰精都盛。第二次就少了。第三次還有沒有的。間或還有受不得的。你的一回多似一回。再吸不盡。豈不是寶。姑子笑着穿上褲子。重又熱了酒來。二人不像先了。摟肩並坐。親親熱熱的一遞一口。吃了一會。日色將西。道士笑着道。多擾你的寶物了。過幾日再來看你。起身要去。姑子也笑道。不堪匪敬。免勞道謝。你這去。幾時來。銀子帶了去\footnote{一絲不漏。所以爲妙。}。道士道。那銀子送你盤纏罷。我不過五七日定來看你。那姑子依依不捨。送出庵門。道士去遠了。他還站着目送。遠遠見有人來。他纔縮了進去收拾。這道士隔着六七日又來望他\footnote{已伏後。要過七日。方纔又採得也。}。就帶了下酒之物。大袖籠來同飮。飮得興濃。就在花下做一齣\footnote{這一齣名爲花下佳期。}。後來花謝了。就在他禪床上做了快樂窩。他愛這姑子有一百分。姑子愛他也是兩個五十。親愛無比。來往了有半年光景。姑子也就不能如起初的精脈盛了。道士恐傷了他。意欲別去。一日。對他道。我看你老實可喜。我有一種異法傳你。你一生受用不盡。姑子道。你傳我個甚麼法。道士道。我有採戰之法。傳與你罷。你學會了。自有許多的益處。遂同到床上。附耳傳了他許多的秘訣。那姑子歡喜得了不得。道。你今晚不回去罷。夜裡好傳授得。道士應允\footnote{相厚半載。將別纔留共宿。是一部書所無者。}。一連住了有四五日。見他學會了。又叮囑了些話。把內中利害二字都詳細與他說知\footnote{此處利害二字。這暗含着說。留在後來姑子傳命兒时方纔細述。妙。}。然後道。我傳授你此法。可也補你的情了。我明別你他去。又取了二三十金相贈。小姑子聽他要去。吃了一驚。一把拉住了他的手。掉下淚來。道。我同你這幾個月的恩情。你怎忍就撇了我去。道士道。我爲你在此半年。也不爲不久了。你今學會了此法。我在此也沒用了。後會有期。不必傷感\footnote{伏後。}。替他拭了淚。又叮嚀了幾個保重。出門而去。這姑子一來感他相愛之情。二來喜他那個異物。他去後。悲切了好幾日。心纔放下。過了些時。正想要得個人來試試法。恰好到聽提了一角蘆瓶水白酒。肥肥的一段騎馬腸兒。兩個鹹鴨蛋來看他\footnote{像形。饟腸得如鴨蛋之粗。則姑子大得矣。}。姑子道。你這半年多往那裡去來。怎不見你。到聽道。自從那日別你回去。第二日。有個朋友約我同他往湖廣去了一回。事忙沒有來別得你。昨日纔來家。今日特來看你\footnote{道士遇姑子半年有餘。而到聽係姑子厚友。多日未曾到庵。豈非疏漏處。有此開(問)答。方見久不來之故。甚妙。}。就在小姑子房中。二人飮了一會。到聽笑道。那一日我多了一杯。辜負了你的美情。沒有盡興。今日來替你陪罪。就去扯他的褲子。小姑子正想拿他試法。欣然解衣。二人幹了一會。姑子幾鎖。到聽便丢了。姑子覺得果如醍醐歡(灌)頂。甘露沁心。樂不可言。到聽也覺得快活無比。戀戀不休。一連三度\footnote{與道士初採姑子時作對。}。弄得猥頭搭腦。頭葷(暈)眼花而去\footnote{較後之衆和尚鼻塌嘴歪猶不濟。}。但這姑子居處旣僻。貌又不甚動人。外面的招牌不濟。誰知他內中有好貨。所以主顧甚少。後來老姑子死了。他獨自一人。只得又招了兩個姑子來做伴。人眼多。越發難招攬主顧了。只好偶然遇巧。偶一爲之而已。眞是。

\begin{quotation}

雖有驊騮千里駿。不逢伯樂待如何。

\end{quotation}

他這個法後來傳了何人。到底可有大展試驗之日否。後來便見端的\footnote{伏後十八回內。}。且說道士別了小姑子之後。要往別處去雲遊。又想遇一個美物。心內道。西湖名勝冠於天下。何不到彼一遊。況這山川秀美的地方。定產異物。或有所遇也不可知。遂搭船到了虎丘。到寺內去遊賞。那寺門外兩邊俱是鋪面。賣泥人物並搬不倒。精細甲於天下。有客(賣)各種盆景的。收拾得十分好看。賣宜興壺碗各種器皿的。有賣斑竹几杌椅凳的。而織虎丘燈草細蓆者居多。眞正熱鬧。進入寺中。看了看試劍石。到了千人石上。四圍俱是茶房酒肆。又看了看劍池。登了登寶塔。遊玩了兩日。又雇船到了杭州。就在西湖邊覓了個寓處住下。靈隱淨慈三竺各寺內。並岳墳于墳。四處玩賞了十數日。一日遊倦了。傍着湖邊一個旅亭中小憩。臨窗坐下。獨飮了數杯。見水光接天。山色葱翠。時値深秋。芙蓉夾岸。桂蕊飄香。眞乃快心爽目。想道。古人贊這西湖說。

\begin{quotation}

若把西湖比西子。淡妝濃抹總相宜。

\end{quotation}

果然不謬。正欣然得意。隔席四五個少年\footnote{又是四五個大閒漢。閒漢七。}。也在那裡吃酒。內中一個道。世間上稀奇古怪的事何所不有。又一個問道。有甚麼奇事。那人道。前邊湖嘴子上那昌家的女兒\footnote{淫婦。}。就是個怪物。這一個道。怎麼見得。那一個道。他家男人死了。他如今也不嫁。也不守。却零碎嫁\footnote{零碎嫁三字。千古奇聞。}。他今年纔二十一二歲。只他娘兒兩個。做了個半開門。我聽得人說。當日初出來還不怎麼的。近來竟成了個鐵屄\footnote{屄字之上。從未見有加一鐵字者。不但奇聞。更令人可畏。}。這個騷浪的法。大約也就淫得無對了。任你什麼好漢子。再敵不過他。一夜弄到天亮。他再不得個飽足。同他睡一夜。第二日定是七死八活。還有病幾日纔起得來的。他誇嘴說。人去嫖他。只要三錢一夜。就有三五個人同去。他都不辭。若有本事把他弄得飽足了。他反倒貼十兩。先我還不信。都不過是父母的皮肉。當眞是鐵的不成。後來聽得竟是實話。我們前日約會了八個人。商議了一同走去。原是取笑。諒他見人太多了。決不肯留。誰知到了那裡。他竟公然笑納。八個人齊心輪流。想弄輸了他的嘴。大家取笑一場。誰知從沒有點燈時弄起。一上一下不歇氣。足足弄到次日日出。一個個弄得盔歪甲斜。他還不曾足興。反討他笑話道。不要說你們這幾個膿包。像你們這樣不濟的東西。再有八個。也不在我老娘心上。你們若有本事。從此時再弄到黑。看老娘可怯一怯。若沒本事。老娘饒了你們的命。去罷。幾個人就沒一個敢說一句硬話。大家掃了一鼻子灰出來。這些人如今替他起了個混名。叫做女敬德。鼓兒詞上說胡敬德日擋八將。取這個意思。你說這個女人豈不是個怪物\footnote{見怪不怪。其怪自壞。衆人先以爲怪就氣餒了。焉得不敗。}。衆人聽了大笑。道士聽了這話。暗想道。旣然有這個怪東西。我何不去會他一會。試試他的本事。遂起身還了酒錢。一直到湖嘴上來。且說這昌家女兒。父親自幼亡故。母親孀居\footnote{昌家女兒者。娼女也。其母老娼矣。故不必用姓。}。也時常同人活動。這昌氏十一二歲時就有些妖模妖樣。一日。在門口站着。兩個少年經過。見了他。一個說道。好一個金童玉。那一個道。得同他靑天白一下子就快活了。先那一個道。還七大八個呢。同他着笑去了。他聽了進來了。白他娘道。恁兩個砍千刀的忘八在門口過。夾着走他娘的村路。走罷了。說我金童玉。靑天白。又甚麼七大八的。恁個嚼舌根的囚。爛了嘴的龜子。喃喃嘟嘟罵個不休。他娘〔聽〕不過。說道。他說金童玉。說你是個女兒。也沒有甚麼壞話。你儘着罵甚麼。昌氏道。他還說要靑天白呢。那娘不好說。便道。靑天白月。說你如月一般白。倒不好說。昌氏道。你不要哄我。我知道是日。他想要日我呢\footnote{妙。}。他又說七大八。說我小。還肏不得。你當他說好話麼\footnote{更妙。}。那娘倒無言可答。又一日。他娘女兩個到門口看看。恰遇一個男子在他門外牆根下溺尿。他一眼看見。攆出去罵道。人家有黃花女兒在家。你瞎了眼了。在這裡來撒濃溺血。那人不好意思。提着褲子飛跑。他趕到街上去罵。娘拉他進來。道。那也是個黃花郞。失錯撒尿。跑了就罷。還罵甚麼。昌氏道。哎呀。好黃花郞。一個㞠頭子像紫李子一般的。還是黃花郞呢。到了十三歲。見他娘常同人做些不三不四的事。就竊聽或張張。看了幾回。見那娘的樣子有個十分快活的局勢。想道。這件事原來這樣受用。我怎得也受用受用。看是怎樣快活。他隔壁有個小廝。姓于名敷。比他大三歲。十六歲了。自幼在他家走動。與昌氏像兄妹一般。頑耍戲謔。無所不至。兩人十分契厚。他也愛昌氏。但年小膽怯。不敢動手。昌氏也一心愛他。一日。他娘往親戚家去了。只他一個在家。恰好這小廝走了來。昌氏一見了他。道。我娘不在家。你來同我做做伴兒。那小廝巴不得。便道。我們坐着做甚麼。尋個甚麼頑頑。昌氏道。我們猜拳罷。輪〔輸〕了的打一個手批兒。那小廝道。不許賴。要打的呢。昌氏道。那何用說。取了幾個錢做拳碼兒。兩人猜。昌氏輸了。那小廝一把拉着手要打。昌氏不肯。緊緊的把手縮着。那小廝用着力拉。道。你說過不賴。如何又賴起來。昌氏掙不過。睡在他懷中滾。小廝道。憑你怎麼賴。要給我打一下纔罷。昌氏滾了一會。見他拉住不放。坐起笑道。你打。遂將袖子擄起來。露出雪白的膀子伸着。那小廝愛得了不得。笑道。我那裡捨得打。你讓我咬咬罷。遂將嘴含了含。放了他。道。再來。昌氏笑道。你不打我。我贏了要打的呢。小廝道。那憑你。二人又猜。是昌氏贏了。小廝伸出膀子。道。你打。昌氏笑道。你不打我。我也不好打你的。饒你罷。那小廝見他嘻皮笑臉。也笑着同他說道。我捨不得打你。你又捨不得打我。這怎麼論輸贏。我們贏親嘴罷。昌氏笑道。我怕你麼\footnote{妙答。}。兩人又猜。又是昌氏贏了。那小廝把嘴送到他臉上。道。你親。昌氏笑道。也饒你罷。我不親。小廝道。不好。你不親我的。我贏了又不好親你的。怎算得輸贏。定要他親。他把個臉扭過去。嘻嘻的笑。那小廝一把抱住。定把嘴送到他嘴上。挨了挨纔罷。放了他。笑道。你還敢來麼。昌氏瞅了他一眼。笑了一會。又猜。是那小廝贏了。道。送了嘴來。昌氏笑着。站起要跑。被他一把拉到懷中。用兩腿夾着他的腿。兩手捧定了臉。連親了四五個。此時那小廝也興動了。一個㞠子鐵硬。在昌氏腿上亂撞。昌氏被他撞得春心大發。故意在他懷中滾。混擰混掐。笑說道。你原說過親一個。你就親上這些。嘻嘻哈哈。頑成一處\footnote{眞一對頑皮。}。那小廝見他有些俯就的意思。把他一下對面抱住。說道。我們摔個交頑罷。將他抱到床前。一下壓在他身上。就把陽物隔着他衣服亂戳。昌氏也情動得很了。說道。不要頑了。你去關了大門來。我替你說話\footnote{不意小女子竟是老作家。}。那小廝懂局。知有妙處。放了他。忙關了門進來。見他坐在床上。問他道。我關了門了。你說甚麼。昌氏笑道。我哄你放我起來。有甚麼說的\footnote{答得不即不離。妙極。}。那小廝也跳上床。將他推倒。掀開衣。就扯褲子。昌氏也不拒。只笑着打。道。你越發這樣頑起來了。被他扯下褲子。壓在身上。然後伸手去扯開自己的褲子。取出肉具向他腿縫中亂戳。他只是笑。那小廝見他肯了。親了個嘴。道。你不要混動。我同你試試。昌氏也就不動。那小廝起來。看明了關竅。用了些唾味(沫)。然後對準門戶。漸次而入。兩人弄了一會。俱是初次開葷。並不知內中趣味。昌氏想道。又疼又脹。一點樂趣也沒有。我娘每常是那個樣子。大約是熟了方妙。須臾事畢。各自散去。自此以後。一得其便。兩人就做一齣。漸得佳趣。昌氏方知個中果有滋味。到了十五歲。他娘也有些知覺了。倒煩人去對那小廝父母說。情願將女兒白與他。家中無人。却要招贅過來。那小廝的父親也是個窮漢。見兒子十八歲了。長成了一條大漢。巴不得替他娶個媳婦。今遇着了這不費錢的便宜事。何樂不爲。況只又一牆之隔。出贅何妨。就允了。遂成了親事。昌氏雖同他偷了二年。一月中尚不得一二次。甚不暢意。今得成了夫婦。一對淫物相聚。朝弄暮弄。日弄夜弄。不到半年。把那于敷弄成癆症。虛火下行。越病陽物越硬。越硬越要。弄到那病倒動不得了。陽物還是鐵硬。昌氏那顧他死活。騎在他身上。通宵到亮。不肯少歇。把那于敷弄得昏一會。醒一會。首尾不到一年。信物一信。親自往閻羅王處投到去了。昌氏這一年來。除了行經之日不得已而暫歇。餘日是再不放空的。今丈夫死後。整整捱了一個月\footnote{虧他。}。慾火如焚。自陰戶中一把火直攻上頭頂。一個臉時時發紅滾熱。幾乎似坐地的眞僧。那三昧火要從丹田下起荼毘了的樣子。耐不得了。不住走到門口望望。一日。只見一個精壯少年。也還齊楚。一面走着。偶然看了他一眼。他此時那火益發冒了上來。忍不住笑說道。你走路罷了。看我怎麼。誰知那人也是個色字號的先鋒。見他話來得有因。又一臉是笑。便站住脚。放膽笑答一句道。因見娘娘標致可愛得很。故此斗膽看看。昌氏笑道。你看我。想把我怎麼樣的呢\footnote{正是你要怎樣的呢。}。那人笑着近前道。這憑娘娘下顧了\footnote{二人針鋒相對。正是一對老手。}。昌氏笑着瞟他一眼。往裡就走。那人隨後就跟進來。昌氏低聲道。我家有老娘娘呢。你且站着。因伸頭一望。不見他的娘。〔道。〕快跟我來。到了他房中。不暇開言。上床各自解帶脫衣\footnote{急得有趣。}。那人有一副本事。二人足弄了有兩個時辰。尚未肯歇。昌氏初經大敵。如登天之樂。那裡肯放他。他娘半日不見女兒。看看關着門。打窗洞中一張\footnote{先是女兒張娘。此時娘又張女兒。絕妙。}。見他同着一個小夥子好弄。那小夥子像同他女兒有仇一般。下死力亂舂亂搗。他女兒像抽瘋似的。渾身亂顫亂扯。他只得迴避。等那人去了。他說女兒道。你一個新寡婦就做這樣事。不怕傳出去人笑話麼。昌氏道。我嫁過的女兒。娘管不得了。我見娘也常做來。難道你是舊寡婦就該做的麼\footnote{妙語。趣甚。}。把他娘說得臉通紅。反沒的答\footnote{眞沒的答。}。那人是個色精。遇了他這個色鬼。正是一對。三日不來。間或也送些盤費。或帶些酒肴來吃吃。來則必弄。弄則必盡興而後止。他娘料也禁他不得。各尋主顧。同居各弄\footnote{奇語。}。各幹各事。那人到數月之後。力不能支。不知是病倒了。又不知是躱過了。再不見影。昌氏等了數日不見來。他自經過這人之後。益發貪之不已。他生得風騷俏麗。又有鈎人之術。絲毫不費力氣。只用放下鈎去。人隨鈎而入。況且全不計利。男子中能有幾個王狀元。十年前已薄相如的。無不樂從。後來人知道的多了。就有街坊閒漢氣不憤。道。放着我們本坊本里的人不相與。倒同遠處人來往。就打磚撂瓦的囉唣。昌氏同他娘商議道。這裡不好住。我們到西湖嘴子上僻靜些的地方。尋幾間房子去安身。那裡近着天竺靈隱淨慈各寺。這些和尚。人稱爲色中餓鬼。又說有不歇不洩的本事。況他十方錢糧。來得容易。不但圖了快樂。且又可掙錢享用。豈不是好。他娘也四十多歲的佳人。雖相與了些朋友。但白擾的多。送分資的少。要靠女兒掙錢度日。以他在下之一豎口。供在上之二橫口。況連年他母子把這件事也做慣了。以爲這是婦人家理所當然的事。不足爲異。就依他。在西湖上尋了三間房子。單門獨院。倒也僻靜。搬上去住下。那西湖各寺中。禪和雖然也有。那吃酒養婆娘的和尚却反甚多。能有幾個如參寥子說的。凡心已做沾泥絮。不逐東風上下狂。這樣的高僧何可多得。又有那些串寺院的道婆子替他招攬。不一二日。就被他鈎上一個。一傳兩。兩傳三。這些和尚以化緣爲由。盡來賞鑒。且拿着施主布施的不心疼的銀錢。都送他做纏頭之費。且終日大酒大肉。買來受用。他娘兒兩個此時惟恨上下只有兩口。吞噬不盡。眞個是其門如市。大門中大和尚絡繹不絕。小門內小光頭出進無休。昌氏不但領略了這禿驢的本事。且大獲其利。他從朝至暮。那卵袋之頭的人穿梭般出出進進。他皆不辭。尚不能飽其所欲\footnote{昌氏可與河間婦作對。}。過了年餘。這些和尚被他弄得鼻塌嘴歪。囊內已空。袈裟度牒都典了。就來得稀疏。他索性做了個半開門\footnote{門未必止於半開矣。}。索價甚廉。只要三錢一次。若本事高強。可以遂他的心。便不受價。你想這樣價廉而工巧的寶貨。誰不願來交易。後來總不足興。他出一個新令。即二三人同來。只受價五星。四五人只價一兩。如有能弄得足興。不但價銀不收。反以十金爲贈。以做勞資。這些少年聽得這話都瘋魔了。舉欣欣然。皆摩拳擦掌而來。想白受用了。還反得辛苦錢回去。皆三五成羣。相約而來。不想皆弄得棄甲曳兵而走。吃酒的那人。也有一具好成文的陽物。又有一分耐久的本事。他聞了這名。約了一夥八個少年。湊了一兩分銀到他家來。原只說他見了八個人自然不肯。以爲大家取笑的意思。不想他正在恨英雄無用武的時候。見了。竟慨然笑納。這八個人沒有說害怕竟走了出來的理。也自恃着這樣八個精壯小夥子。可有弄不倒他的。遂輪流轉上半日一夜。皆拱手納降。被他痛貶一番。忍愧吞聲而出。這一日。那人在酒肆中當件奇事說給衆人下酒\footnote{新鮮淡菜。絕妙下酒之物。}。被這道士聽得。到了湖嘴邊。只見一家門口。一個妖妖嬈嬈的少年婦人站在那裡。道士近前打了個稽首。道。女菩薩。借問一聲。這裡有個姓昌的在何處住。那婦人道。你問他做甚麼。道士道。貧道尋他有句話說。那婦人把他上下打量了打量。不像個化緣的道士。笑着說。你想是要來相與相與他麼。他不是好惹的。道士道。正爲慕名纔來相訪的。那婦人道。我就是。你尋我說甚麼。道士聽說就是他。把他一看。雖爲不美麗。眉目中另有一種騷態。令人魂醉。便笑着道。到裡邊好說。那婦人讓了他進去。道士坐下。向身邊取出銀包。拿了有三兩多一錠送與他。道。久仰大名。意思要來親近親近。領敎大才。無可爲敬的。這些須微物聊表寸心。那婦人笑道。師傅禮太厚了。道士道。不堪匪薄。請收了。又笑着附他耳上低聲道。但有一件。我的本事頗雄。況且我出家人見了婦人。如蒼蠅見血一般。再沒有厭足的時候。只求你不要推辭。就是盛情了。昌氏道。但願你有這樣本事。我倒也不懼。道士又拿了有一兩多一塊與他。道。煩預備個小東。那昌氏見他肯出手。又見他說大話。忙把銀子遞與他娘買辦去了。這婦人是騷淫極了的。聽他說有好本事。旣發狂言。或有實學。滿心要想試試。便道。我娘去買東西。還得一會。我兩個何不先各顯本事看看\footnote{倒是他要先試。不但眉目中有騷態。滿腔內皆騷意也。}。道士欣然道。甚妙。關了門。進房脫衣。那婦人見道士的陽物只好四寸多長。一圍大小。心中暗笑道。我以爲他口中旣出大言。腰中定有雄具。誰知是這一點子東西。還摸不着內中的邊兒。縱有通宵的本事。也是有限。多半這牛鼻子是沒有見過世面的。心中暗笑\footnote{昌氏誠婦人之見也。技勇精嫺者。豈皆魁梧大漢耶。}。他睡倒。兩腿大揸。那道士伏在身來。一送到根。就伏着不動。昌氏見他這個樣子。疑他是從不曾幹過這事。笑着敎他道。你還抽抽動動。怎這麼個樣兒。道士也不答應。片刻之後。婦人先覺得陰戶中微熱。後便如火炭一般。漸漸脹滿。直抵內中極深牝屋之上。那龜頭一時如頑蛇吐信。一時如小兒咂乳。下下戳着癢筋。快活難當。不多時。昌氏丢了一度。方知這道士手段果是高強。將他摟得緊緊的。道。你果然好本事。我遇過多人。算你頭一個了。道士得了這番獎勵。那東西在裡邊更鑽得利害。那昌氏樂得皺着眉。只是嘻嘻的笑。不過頓飯工夫。他又丢了。道。夜裡再弄罷。我娘娘將要來了。我要起去開門。那道士也就放了口氣拔出。那昌氏覺得陰門脹了一下\footnote{前日姑子便有許多艱難。今昌氏只覺一脹便拔出。雖謂兩人有寬窄之異。余前謂如狗鏈幫。後不復寫。評得是否。}。不像先進去時不知不覺的樣子。向他腰中一看。竟長將八寸。粗如鍾口。如獲至寶。忙起身一把捏住。道。你原來竟是個活寶貝\footnote{比那姑子尤愛。}。這個樣子。我今夜有個飽足的光景\footnote{女贊男。}。喜笑不止。二人穿了衣裳起來。那婆子也買了東西來了。收拾酒飯齊來。三人吃畢。昌氏先嘗了兩次。纔高興動頭。等不到黑。老早同道士脫衣上床。看那道士的陽具時。還是像先的那般小巧。兩下上手弄起。不多時。樂得昌氏嘻嘻哈哈。一時又哼哼喞喞。像是內中鑽得難過。一夜未睡。丢了有七八次。却也得了個心滿意足。道士暗想道。這個婦人。怪不得七八個男子敵他不過。果然是個騷淫極了〔的〕奇物。別的婦人經我採到三四次。再沒有不哀求吿止的。他竟全不在意。因向昌氏道。我生平閱人多矣。像你。眞算一個鐵陰了\footnote{男贊女。}。睡了一會。穿衣起來。道士見他夜來斲喪太過。恐傷了他。意欲辭行。那昌氏那裡捨得。抵死留住。不但不要歇錢。連東道都是他拿出己囊來預備款待。日間還不放鬆。弄了一次。一到晚。忙攜了道士上床。就弄上半夜。他還喜樂如前。到了下半夜。就有些勉強承受。道士要歇。他定不肯。又到天明。也就懨懨的不似那精神了。吃了早飯。要想去睡。又恐道士去了。悄悄吿訴他娘。叫伴着道士。千萬不要放了他去。他床上去睡了。那婆子纔四十多歲。雖不似女兒奇淫。也是個酷好此道的。聽得女兒說得這等津津有味。將道士拉到自己房中。要求他賜敎。道士見他有年紀了。不敢下手。他苦苦哀求\footnote{苦苦哀求着如此。從來未聞。}。道士沒奈何。同他弄了一度。婆子嘗着這樣美口的奇物。不肯就歇。死摟住了不放。還要求閏。道士只得又弄了一次。把個婆子暈了過去。半晌方醒\footnote{諺云。爽口物過多做病。此老婦僅嘗二次便至如此。其量遜乃愛多矣。}。倒把道士嚇了一身冷汗。見他醒了。方纔放心。忙忙穿衣下床。那婆子要起來。身子動不得。又怕他去了。一手拉着道士的衣服。坐在床沿上。他伏在枕頭上張着嘴發喘\footnote{形容得有趣。}。等女兒醒了。將道〔士〕交付與他。睡了一會。纔爬得起來。道士要去。昌氏那裡肯依。道士勸他道。不是兒戲的。性命要緊。他道。牡丹花下死。做鬼也風流。我春間同人吃河豚魚。有一個人說。當初有一個蘇東坡愛吃河豚。他說道。吃河豚。値得一死。據我看起來。遇了你這個寶物。値得一死\footnote{孰知不死於道士而死於竹思寬。不但道士料不到。即昌氏亦自料不到矣。}。我好容易今日遇見了你。就死也無怨。到晚。他還興興頭頭的要弄。只弄了一次。他覺得頭暈眼花。腰酸背痛。陰戶內中最深處也有些疼得受不得。心有餘而身不能。方纔興止。道士知他要病\footnote{道士謂昌氏要受病。不意昌氏病尚得癒而婆子反得病而死。豈昌料不到。即此老昌亦想不到。與前一對。甚妙。}。次日臨別。送了他二十餘金爲扶養之資。他娘兒兩個都有些捨不得。但弄不得了。留他無益。眼睜睜的只得放了他去\footnote{眼睜睜。妙。寫出萬分捨不得的光景。}。這昌氏覺支撐不住。便睡倒在床。病了數月。幾乎喪命。吃了許多補益的藥。纔起得來。雖然好了。也就不能似前番壯健。他經過了這一番。淫心微略稍止。心上念念不忘那道士。央煩緊鄰的一個屠四。四處尋了數次。不見踪影。那道士又不知往何處雲遊去了\footnote{看官記着。}。話休煩絮。暫且結過一邊。端的到聽所聞古城隍判斷之語。並諸人托生之事。可是眞否。等我細細敷演後文。來因便見。正是。

\begin{quotation}

無事關門著書。談空不如說鬼\footnote{二語總結一部書大意。}。

\end{quotation}

姑妄言第一卷終



\endnotetext[1]{原書書眉註明「缺一頁」。}

\endnotetext[2]{以下有錯簡。自「煩你整理整理」至「只見他捧了一」爲一葉,原在後,書眉註明「此篇在前」;自「個大托盤」至批註「大約姑子也是怕毛」爲一葉,原在前,書眉註明「此篇在後」。今據文義改正。}

\setcounter{footnote}{0}

\theendnotes

\part*{姑妄言第二卷}
\addcontentsline{toc}{part}{姑妄言第二卷}
\markboth{姑妄言第二卷}{姑妄言第二卷}

鈍翁曰。余一日正評此回書。忽有二三俗客至。一客問余曰。一部大書。內中無限的人。開首一個就出錢貴。此是何意。余曰。如一部傳奇。是誰人的事蹟。定是那正生先上場。故此書先出錢貴也。客曰。此書雖是錢貴事蹟。然正生當是鍾生。傳奇中。豈有以正旦先上場者乎。余曰。不然。此非傳奇。不過借傳奇以做譬喩耳。錢貴猶之正生。鍾生反是正旦角色。故首出錢貴也。又曰。錢貴旣是一部書中大有關係之人。定要寫得他高纔是。其父何以名錢爲命。甚不雅觀。余笑曰。以錢爲命之人。孝弟忠信。禮義廉恥。尚何所知。錢貴旣生於娼家。其父自然是忘八了。此不過信手拈來成趣耳。座中一人家道素封。頗有愛錢之癖。忿然作色曰。君語刻毒之甚。豈天下愛錢人盡忘八耶。余笑解之曰。非此之謂也。非云愛錢人皆此輩。不過謂此輩人中無有不愛錢者耳。彼猶含怒而去。前客又問曰。錢貴旣算正生。係要緊的人了。不但寫他是妓且又瞽目者何。余曰。此別有深意焉。此是作書之人滿腹牢騷。借此以舒憤懣。總見世間之鬚眉男子只知勢利。惟以富貴評月旦。塵竢(埃)中能物色英雄者爲誰。而錢貴以一瞽妓。乃卑汚之極矣。而多少富貴中人他皆不取。獨注意在一貧窮不堪之鍾生。矢心從良。後來竟得全美終身。不過有眼男兒不及一瞽目妓女。此是作者一部大主意。須會得此。方許看此書。

此書寫好賭者。竹思寬。鐵化。戴遷。曾嘉才。牧福諸人。各人有各人賭法。各人有各人輸法。累累數千言。無一句相雷同。故妙。

竹思寬。竹絲而寬。自然是篾片了。篾片又自然從竹靑竹黃中來。所以父爲竹靑。母爲黃氏也。竹靑之刻薄。黃氏之慳吝。只知有銀子哥哥。而親友皆不知爲何物。宜乎生此等兒。竹思寬始而賭。繼而篾。終而龜。此報應刻薄慳吝者亦盡矣。警醒此輩之功不小。

郝氏之遇竹思寬。不過謂此等淫鴇須此如驢之具始可娛之耳。且作後來郝氏歸竹思寬張本。不然錢爲命死後。錢貴又適鍾生。郝氏何所歸。若竟到鍾生之宅。儼然爲之岳母。嗚呼可。故千算萬計。算出一個絕大陽物之竹思寬來。郝氏戀之不能捨。後成夫婦。始不玷及鍾生錢貴也。

亙古及今。極壞的事非極聰明的人做不來。非謂聰明人則壞。特恨其錯用聰明耳。如鐵化之尖酸促恰。豈非一段聰明。然壞了許多心術。所以有奇淫奇悍之火氏。降夫如鼠。與狗爲樂。竟同畜類。雖是尖酸促恰之報。聰明反被聰明誤一語良然。

人屠戶屠四叔姪以放賭爲生。壞人家子弟一生品行。喪人家兒孫多少身家。他家門中行同禽獸。此等人雅當如是。這一段不但是一篇勸戒賭的婆心。且更勸好賭人知此中的大害。

昌氏一宗淫案。隨手結去。筆下何等乾淨。

\chapter*{姑妄言第二卷\\
第二回\endnotemark[1] 錢貴姐遭庸醫失明 竹思寬逢老鴇得偶\\
附 鐵化有心弄人 火氏無聊戲狗}
\addcontentsline{toc}{chapter}{第二回 錢貴姐遭庸醫失明 竹思寬逢老鴇得偶}
\markboth{第二回 錢貴姐遭庸醫失明 竹思寬逢老鴇得偶}{第二回 錢貴姐遭庸醫失明 竹思寬逢老鴇得偶}

話說明朝萬曆年間。應天府上元縣地方有一個樂戶\footnote{洪武欽定。樂戶裹綠頭巾。繫紅布腰帶。連毛豬皮靴。一見而即知其爲龜子矣。}。姓錢名爲命。他妻子郝氏\footnote{郝音好。以錢爲命之人。再有一個好妻子。自然是忘八無疑。樂戶。老鴇子。}。小字翠娘。舉止風騷。語言嬌麗。少年時在美妓中也算錚錚有名的。他年過三旬方生一女。夫妻二人愛之如掌珍。惜之如至寶。將週歲時。此女生得眉目如畫。身體如脂。但有見之者無不憐愛。悉呼之爲粉孩兒。至六七歲他就學弄粉調妓(脂)。描眉掠髻。他父母見他姿(資)性聰明。將他送入鄰館中就學。那先生姓卜名通\footnote{一個不通的先生出現。}。先生謂他道。你門戶人家。所重者無非財帛。況你又是姓錢。可即命名爲錢貴。豈不巧合。道。妙\footnote{他的名字是這不通的先生取。如此起出。}。遂將他留在館中。每日敎他讀書寫字。作對吟詩。誰知這女子穎悟異常。竟能過目成誦。未及二載。連詩詞也覺頗通。他父母心中歡喜自不必說。而傍人亦爲他欣慶。盡說道。錢家之錢樹子自此興矣。又過了年餘。雖纔十歲。儼然成人。其丯姿婥約。不能盡言。只見他。

\begin{quotation}

眉黛春山。眼含秋水。唇猶紅豆。臉若桃花。十指尖尖玉笋。一雙小小金蓮。腰肢似荷莖翻風。皮膚如海棠經雨。語言嬌麗。聲音不亞淸簫。行步輕盈。體態可欺弱柳。隱微處雖然未許人窺。想個中一定是件妙物。

\end{quotation}

他生得眞令人一見魂消。且不必說。孰意那一年城中疫癘大行。他也偶染時症。伏枕數月。他父母延醫問卜。打卦求神。無不備至。後來病雖痊可。只雙眸微暗。漸漸不明。城中之名醫國手百樣醫治。毫無效驗。但那時醫生的技倆\footnote{醫生。}原是有限。而內中又有兩等。一等是窮的。一等是富的。若是那窮的。只好守着藥箱。袖手在家高坐。十日半月。藥都霉爛了。間或賣出一兩劑。聊爲餬口。大約終身不過如此。或者等到十年運來的時候發財也不可知。不然再無望矣\footnote{此段無時之窮醫見之。必點頭嘆曰。不謬。不謬。}。這是爲何。因那一等富的。他家中有幾貫錢財。每日雇上三四個轎夫。扛上一頂油衣紅頂小轎\footnote{三四個轎夫。甚是體面。接寫扛上一乘油衣紅頂小轎。不堪之甚。轎本是擡。此謂曰扛。尤其不堪。}。不論陰晴。大街小巷。擡了亂跑。到晚來。或買燒鵝板鴨。或火腿熏雞。着背藥箱人拿了。跟在轎後。故意使人看見。好說此人一日到晚這等興頭。且如此大吃大用。定是時醫無疑。好與他四處馳名。哄人延請。孰知他只好自費幾百文錢。擡在街上搖擺。究竟一日到晚。藥箱還不曾發市\footnote{此段有錢之富醫見之。亦點首曰。誠然。誠然。}。有那倒運的這個人請着他看病。他不過是撞自己的造化。拿別人的命來試手。胸中千般算計。口內一片胡謅。凡湯劑定要人參。是病症皆做丸藥。起發人些錢鈔。養活自己妻兒。病若好了。誇他的手段高強。索謝不休。醫死了呢。說人的命數修短。潛身無語。眞個是。

\begin{quotation}

招牌下寃魂滾滾。藥箱內怨氣騰騰。

\end{quotation}

況且大明律中。雖有庸醫殺人的罪款一條。從來可曾見用過一次\footnote{此段勿論醫道中窮富見之。必含笑曰。巧手丹靑不過只能畫人形像。此人竟說盡我們的肺腑了。何刻薄若是。}。所以這些人任意胡行。那裡有窮究醫書。精硏脈理的。就是那馳名的國手。也不過是他的造化頗高。遇着都是不該死的症候。多看好了幾個。就傳說是名醫無雙。一匕回生。到底何嘗有絲毫實學。所以說那富的還糊得去。只可憐那窮的眞是寸步難移。近時岐黃中大都不過如此\footnote{此段非是痛貶醫道中人。正是勸其用心精究一番。不可將活人醫死的慈心耳。古云。不爲良相。當爲良醫。謂其能救人耳。若只圖殺人。何不去學劊子手。學醫何爲。扁鵲云。越人豈能生人。但遇不死之症。不致殺人耳。願學醫者效之。}。因此那錢貴不多時竟兩目皆盲。雙眸緊閉。把一對嬌滴滴的秋波。被這些庸醫弄得個視而不見\footnote{諺云。如今的世情。只好睜着一隻眼。閉着一隻眼罷。據我言之。不若視而不見者爲尤妙。}。他父母雖然心疼愛惜。然到此地位。亦付之無可奈何而已。又過了二三年。錢貴已經長成。愈生得美貌可愛。有一詞贊他的妙處。道。

\begin{quotation}

舉止甚蹁躚。體飄颻。態若仙。妖嬈不亞嬌飛燕。梅粧淡添。潘妃兩彎嫌汚。輕掃梨花面。羨嬋娟。秋波緊閉。恰似玉環眠。

\begin{flushright}右調黃鶯兒\end{flushright}

\end{quotation}

且說着郝氏見女兒雖少了雙眸。那丯姿出落得天仙相似。要尋一個好主兒出一樁大錢。纔與他梳籠。但錢貴小時雖有人知他生得標致。後來都聞他損了雙目。皆以爲是個殘疾廢物。誰知他眼雖沒了。還是一個才美佳人。郝氏見他年雖十三。長得如成人一般。可以破身的時候。況他這種人家。無非所愛是錢。巴不得早梳籠一日。早覓一日的利。見沒得財主來相看。貧窮的自然又不肯與他。心中急了。有他相交一個貼皮貼肉的厚友。叫做竹思寬\footnote{王大江先生云。天下無不近臀之卵。亦無不近卵之臀。世上人相與朋友。彼此一弄。自然就親厚了。以此論之。郝氏與竹思寬貼皮貼肉。是厚朋友了。}。托他替女兒招攬個好主顧來。若得一注大財。不但重重相謝。還許他臍下那一品老淡菜常常到口。槪不取利。竹思寬聽了此話。不但爲人。而且爲己。自然去替他上心打合。你道這竹思寬是個甚麼人。他也是個篾片行中朋友\footnote{篾片}。自幼好賭\footnote{第一個賭賊出現。}。又好偷他父母的東西做賭本。雖還不曾在外邊做梁上君子。而家賊之名已遍於親戚鄕黨。人背後送他一個美號。叫做貝者貝戎\footnote{四個字的號甚新。大約是仿金元時叫法。}。不懂拆白道字的人。就直呼之曰賭賊\footnote{如今有此美名者。天下幾半。}。他祖籍是江西人。父親姓竹名淸。母親黃氏\footnote{竹多產於江右。故他是江西人也。}。遷移到南京來住的。那竹淸手中原有五六百金之蓄。他的一個宗叔也是江西人。名叫竹耉。是看守孝陵的太監。他倚着這個聲勢\footnote{好大來歷。可謂遙遙華胄。}。開了一個錢鋪。放印子錢。每月放出大錢一千文。要每日活打。一日收四十文。一月滿。足收錢一千二百文。有人要借死的一千錢。每月加利三百。若這個月沒得還他。下月這三百文又加利九十。你想這樣重利。誰敢去借。都是那挑葱賣菜。窮得沒飯吃的人。只得借來做本。一日圖掙些錢。除了還他之外。下剩幾文度日\footnote{說盡窮漢之苦。}。還有一種好賭的人輸了。借錢作本的。借得來翻梢。贏了送還。輸了又借\footnote{此種人不足惜。}。或是有體面的人。暫時貧窮。少了人些零碎帳目。逼得利害。要惜臉面的。沒奈何了。明知是個火坑。只得去借來且挪一肩\footnote{見此數語。不覺令人長嘆。}。若多欠他些日子。便擡出他令叔的名目來嚇人。這是陵上竹老公的本錢。叫我替他放的\footnote{好勢要小人心腸。令人可恥可笑。}。你若少了他的。他對知縣官一說。捱了板子。雙手送來。還怕遲了。人聽見這話。誰敢短少。賣兒賣女也顧不得。且還他要緊。他屢年也積趲了有二三千金。他生性嗇刻。親友們到他家來。不要說款待酒飯。從來不知與人一鍾茶吃。他或有所求於人。或有體面朋友光臨。沒奈何。忍着心疼。備一餐粗飯相留。這也是十年九不收的事。他妻子黃氏是來到本京娶的。也還是個做買賣正經人家女兒。但生性奇異。說起來更爲可笑。他只夫妻兩口。又無多人。間或買斤肉來。何妨公明正氣收拾來吃。他生怕有人來看見。搶去吃了一般。弄一個小廣鍋。在床後馬桶根下炒熟。揀好的落起些來藏了。餘的剩出來。關了房門。兩口子如做賊似的。忙忙偷吃了纔開門。等竹淸外邊去了。他將那所藏之肉拿出來獨享。每每如此。一日他生辰。他哥哥家送了四斤肉。兩尾魚。兩隻雞。兩盤麪與他來做生日。他哥哥嫂子姪兒姪婦都來拜壽。竹淸陪着大舅內姪在堂屋裡坐。這黃氏把那肉割了有四兩。炒了一盤。將那雞頭雞翅膀雞脚去了。下了炒做一盤。魚尾巴去下小半截來做了一盤。別的忙忙收起。將些白水着些鹽下了一撮麪\footnote{看書者勿謂形容太過。此類人世竟有之。}。每人剛有大半碗。叫拿出去款待哥哥姪兒。他嫂子看不過。說道。姑奶奶。外邊三個大人。這一點子那裡夠吃。少還罷了。你湊四個盤子也好看些。不尶不〖兀介〗。三個成個甚麼樣子。他艴然曰。誰不叫他送四樣來的。他只送了三樣。那一樣叫我那裡變去\footnote{責人則明。責己則昏。眞有此等人。}。他嫂子道。不論片粉也罷。或韮菜白菜之類。那能値幾個錢。添一盤便了。黃氏皺着眉道。可憐見的。家裡要半個刮沙的錢也沒有。拿甚麼去買。他嫂子又道。那肉還多哩。再割些下來。做不得一盤麼。他聽了。由不得那眼淚撲簌簌往下滴。道。先割那一塊。比割我身上肉還疼呢。還叫我割。你們不是來替我做生日。是要來送我死了\footnote{先是皺着眉哭窮。後方堕淚捨不得。寫盡吝嗇醜態。}。他嫂子見他這個光景。也不好再說。任他拿了出去。竹淸把盤子品字放了\footnote{異想。}。只陪着舅子內姪吃完了那半碗麪。也不叫添。也不再讓\footnote{可謂夫婦同心。}。衆人只得放箸。還剩了些骨頭魚刺之類。他忙忙收進。藏在抽屜內。他嫂子也知機。料想坐着也沒用。決無再留他們吃的事了。肚裡有些飢餓。就帶着媳婦要家去。黃氏心中暗喜。也並不假留一聲。送到門口。看他坐上了轎。見轎夫擡起來了。他纔說道。我要收拾飯待嫂子呢。你又不肯大坐坐\footnote{等擡起轎來纔說。妙極。不擡起。尚恐其回來也。將鄙吝人說得無立身之地。然此等人竟有之。}。空空的回去。他嫂子微微含笑而去。他夫妻二人到四十歲尚無子息。心中想道。人家求子都供一尊送子觀音。我要畫一軸來供養。不但要費銀錢。況我家現供着玄壇財神爺。每日要上香。再供一尊菩薩。又要費一分香錢。大不可。小算零碎。不覺一年總起來就要好幾十文。如何行得\footnote{好精細算盤。}。兩口子商議道。觀音是佛。這是神。菩薩旣送得子。難道神道就送不得子。我弄個畫的娃娃貼在玄壇爺懷中就是了。偶然擡頭見房門上貼着一張耍娃娃。喜道。湊巧。湊巧\footnote{眞是擡頭見喜。}。拿刀子就把那娃娃剾了下來。捨不得錢買麪打糨糊。兩口子刮下來許多牙黃\footnote{好算計。好想頭。刮下許多牙黃來。令人絕倒。}。沾在玄壇懷中。他夫妻二人每人上了一炷香\footnote{人家上香或三或九。他只上兩炷。新款。}。倒虔虔誠誠禱吿了一番。叩了十多個響頭起來\footnote{或香少而頭多也。一秀才送敎官節禮。封筒上寫節儀五十文。門生某百五十拜。多五十拜算五十文。官云。你可添百文來。只用五十拜足矣。他夫妻因省了一炷香。故多叩些頭以補之。}。竹淸對黃氏道。人家求財求子都要許個願心。願是我不敢許。設或養了兒子。拿甚麼還。古人說。寧許人。莫許神。神道爺跟前不是扯得謊的。但俗語說。小本不去。大利不來。我們旣求神道慈悲送子。也要時常有些供養纔是。黃氏道。你這算計不好。若時常供養。倒費得大了。你竟大大的許個豬羊願心。設或養了兒子。我們竟封幾兩銀子折乾。神道是不會用銀子的。仍舊還了我們。這豈不省事。竹淸搖頭道。萬萬行不得。事情要深謀遠慮。倘或神道一時竟把銀子收了去。那時怎麼處。黃氏想了想。道。不然把我許了神道爺罷。料道神道爺是不要人的。竹淸道。越發行不得。倘神道爺一時靈感起來。賜了兒子。把你拿去做小奶奶。我可不得了子。倒把老婆送掉了\footnote{他夫妻越算越奇。}。黃氏道。這不好。那不好。你就想個主意出來。竹淸道。我有個好道理。每日兩頓飯是我們要吃的。你每頓飯好了。不論葷菜素菜。先送了去供供。也就算得供養了。古語說。心到神知。這豈不妙。黃氏連聲贊道。這主意好。這主意好。自此日爲始。他倒也着實虔心。每飯必供。供必叩頭禱吿一番。白菜豆腐去供。他還不在心上。或買些肉來。他怕神道吃了去。拿個小碟。少盛幾塊。心驚膽顫的拿去試試。少刻去收時。竟絲毫不動。他試過幾次。皆是如此。膽大了。後來全送了去供過。纔收下來吃\footnote{一路敍來。直欲笑殺。}。一日買了個魚。也全送了去供。不想剛剛被一個野貓銜去吃了。他來收時。只得一個空盤。驚得目瞪口呆。忙走來吿訴竹淸道。哎呀呀\footnote{如聞其聲。}。了不得。了不得。竹淸見他面目更色。倒也吃了一驚。忙問其故。他道。原來神道愛吃魚。我當每常他是不吃的。一尾魚全拿了去供。誰知吃得精光。可惜了的。心疼死人。竹淸聽了吐舌道。你前日還說拿銀子折豬羊。把你許神道爺呢。倒是虧我沒有聽你的話。黃氏道。造化果實。虧你見得到。就是這魚。今後是再也斷斷供不得的了。從此以後。他家再不買魚了。過了數月。他夫妻兩個睡到半夜。似夢非夢。如每常日裡一般。同到神前去叩頭求子。那神道忽然說起話來。道。我看你夫妻二人倒還虔心。因指着案邊蹲着的一個猛獸道。把他賞你兩個做兒子罷。他夫妻又驚又喜。驚的是畫的神像會說話。喜的是與他兒子。叩了許多頭。再看那獸時。原來是一隻金錢大豹\footnote{豹報同音。謂此等人宜生此獸子以報之也。}。竹淸道。我每常見爺爺的聖像傍邊是一隻黑虎。怎麼如今又換了一個金錢豹子了。神笑道。如今世上壞人太多。我管世間財帛一道。有十分在銀錢上刻薄的。故遣他去暗暗的啃些人的腦髓\footnote{銀錢上刻薄的人留神腦髓。}。所以又換這個豹子來。你旣求子。故把他賜你爲兒。竹淸道。爺爺。小人求子一場。旣蒙慈悲。賞我一個人做兒子纔好。這等一個凶惡畜生。如何要他做兒子。神笑道。你不要看輕了他。他是唐朝武則天之姪武三思。生前曾封過王位的。因他貪淫凶惡。故墮畜生道。一來如今該他轉世\footnote{應前到聽所聞神語。}。二來你夫妻又懇懇求我。故此拘來與你。你這種人刻薄到萬分。生個畜生也罷了。還想得好兒子麼。竹淸道。兒子倒也罷了。怕他啃我的骨肉\footnote{刻薄人着眼。}。神大笑道。你一生把那些窮人的骨髓都吸盡了。就不許他把你啃一啃麼\footnote{貪得刻薄之輩仔細聽者(着)。}。因用手將那豹子一指。那豹吼了聲。望着他二人一撲。驚得他夫妻一齊大叫哎呀。醒來時原來是一場大夢。心中還跳個不住。夫妻彼此相問所夢符合。心內常常憂疑。過了數月。黃氏經水不行。吞酸懶食。知是有孕。喜的是得了胎。又怕的是那豹子。到了五六個月上作怪起來。在腹中橫撐直撞。痛得那黃氏摀着肚子流眼淚。一日定有數次。連夜間睡覺亦不能免。間或睡着了。還撐撞得疼醒來。黃氏十分憂懼。向竹淸道。不是求了兒子來。是求了寃家來了。我的命還不知怎樣呢。竹淸也着實擔心。到了分娩之期。黃氏四十多歲纔破盆生育。骨縫硬了。萬分艱難。兩個收生婆守了三日三夜。纔生了下來。黃氏只得一口悠氣。心中雖然害怕。這樣年紀纔得個兒子。也還有幾分歡喜。況且是個肥頭大臉的娃娃。又甚心愛。但這孩子一個㞠子有三寸餘長拖着。見者無不驚異。三朝這一日。他舅子約了些親戚。都送了賀資來吃喜酒。黃氏睡在床上動不得。是他嫂子來代庖。也還豐豐盛盛的款待來人。他家每常待客。那肴饌不過名而已矣。連盤子底下靑花還蓋不嚴。今日忽然豐滿過盛。竹淸心疼得了不得。暗暗抱怨道。這是我那不會當家的內嫂做的事了。來破碎我的家私。我不吃還等別人吃了去麼。自己遂大嚼大啖。不住喝酒。已吃了個五六分醉意。衆人替他道喜。敬他喜酒。他鍾鍾不辭都領了。衆人見他吃得爽快。又敬個雙鍾。他到口就嚥。多了幾杯。有八九分醉了。衆人臨散。他送客。剛跨門檻。不防踩着一塊骨頭。站不穩。把脚一搖。一交跌倒。把踝子骨〖扌歪〗錯了骨縫。疼得滿地亂滾。叫苦連天。衆親戚倒都着實不安。他舅子內姪忙替他揉對了骨縫。擡他進去睡下。又跑到接骨的醫生處。買了膏藥來與他貼上。他家並無餘人。他舅子見他夫妻二人都睡倒。只得家中叫了個老婆子來服事。過了半月有餘。他夫妻二都掙得起來了。因舅子家那老婆子在家中。一日要多費些米菜。忙忙打發他回去了\footnote{說得此等刻細人行事。令人絕倒。}。將到滿月。他大舅同妻子商議道。妹子這樣大年紀纔得了個外甥。前日替他做三朝。把妹夫的腿幾乎〖扌歪〗折。我倒很不過意。如今滿月了。我再約些親友攢些分資。一則賀喜。二則替他起病。你道好麼。他妻子道。前日三朝。姑娘睡倒了。是我在那邊照料。還成個樣兒待那些人。如今他起來了。是他自己料理。送了分資去。他藏起來。弄些不堪菜蔬待人。連你的臉面都不好看。你還不知他的刻嗇麼。依我的主意。你齊了銀子。買一口豬。叫屠戶宰了。再擡一罎酒。剩多剩少與他買柴米。這或者他還收拾的好看\footnote{主意固妙。孰意竟大謬不然。這或字下得好。亦慮及在有無之間。}。他舅子依着妻子。如法送去。到彌月之辰。有十四五個客到了他家。等到晌午。纔放下兩張桌子。八個人一桌\footnote{大約是取吉利。八仙慶壽之意。}。少刻搬上菜來。你道是些甚麼東西。每桌上只得四個盤子。一盤豬肝炒腸子。還墊上許多葱。一盤心肺熬蘿蔔。一盤豬頭肉膾豆腐。一盤是蹄爪子同槽頭肚囊皮炒白菜\footnote{此四品描寫得令人笑殺。叫這些嗇細人何處生活。}。都只鋪過一個盤底子來。空處尚露着靑花。八個人一舉箸。只剩了四個空盤同幾塊骨頭。竹淸只拿着寡酒相讓\footnote{大約黃氏不善飮。不然此一罎酒亦藏起矣。}。原來黃氏把那豬的四隻腿。兩塊大肋巴。都落了下來\footnote{余竟見過此等人此等事。並非謬語。}。拿到房中床後去醃。正然歡喜。忘了鍋中煮着飯。他添了一把柴出來的。那柴掉了出来。就把竈前的餘柴引灼。煙就大起。黃氏忙去一看。見火焰焰的燒着。嚇得大聲喊叫。衆親友聽見。都跑了來。大家同救熄了\footnote{醃得好肉。得無妄之福者。即有無妄之禍隨之。黃氏不知之耶。}。及至出來。只見他家的兩條狗餓得瘦骨伶仃。見人不在跟前。跳上桌子。吃得盤中的骨頭餘汁酒鍾。都掉下地來。打得粉碎\footnote{眞正奇想。}。衆人也沒興坐了。吿別而去\footnote{竹淸夫婦當感激此狗。虧他省了許多酒。}。他舅子到家吿訴了妻子。又是氣又好笑了一場。竹淸見屢屢不妙。向黃氏道。自生這孩子。你我二人幾乎喪命。今日又險些遭了火燭。將來不知如何。終日憂愁。這孩子倒無病無災。易養易大。到了五六歲。就同父母相拗。叫他往東。他決定往西。從不肯一事順手。竹淸夫婦見兒子長得淸淸秀秀。數年來也沒有甚麼禍患。他雖性拗。父母再沒有不疼兒子的。那黃氏更姑息得不成話說。凡事不敢拗他一拗。慣得那孩子天也不怕。到了十歲纔送去讀書。先生聽得說他性子拗。凡事拗戇。因起名叫做思寬。要他變化氣質之意。他在學中纔坐了兩日。便想出逃學的方法來。向先生道。我家爹爹身上不好。家裡沒人使喚。叫我家去使喚呢。先生放了他。他躱在外邊。先還同小孩子們跌錢下城棋。輸了時回家。見他母親那裡有藏着的錢。便偷了出去。後來就漸漸同人搗丁拐擲四子。但輸得大了。就將家中零東碎西偷出去賣了還人。黃氏全然知道。只瞞了竹淸。竹淸每月白送了學錢去。他總不到館中。淸晨出去。餓了回來吃飯。到放學時回來睡覺。黃氏又護短\footnote{子弟之不肖。無不起於護短之母。}。不肯吿訴丈夫。說兒子逃學。在外賭錢。並家中偷東西的話。間或背地勸勸他。他便狠頭強腦。嘴中不乾不淨。連爺帶母的混罵。到了十四五歲。長成了一條大漢。他那陽物竟長將一尺。粗如鍾口。竹淸思量要替他娶個媳婦。或可絆住他的身子。因想到他那樁物件。可是女子容得的。遂爾中止。他每日在外戲耍。索性不歸。後來連老子都不怕了。他娘再要說他一兩句。他一搡一路筋斗。罵是不消說得。竟有輪(掄)拳之勢。如此數次。後來黃氏見了他。竟眞是見了活豹子的一般害怕。連哼也不敢哼他一聲。他自幼知道他媽媽藏些梯己肉菜。他一時餓了走回來。惡狠狠的問他娘要菜吃飯。黃氏怕他得很。忍着心疼。忙取出與他吃。一日。黃氏留了幾塊好肉自享。他又來要菜。黃氏捨不得拿出來。回他沒有了。他四處去翻。在床脚背後翻着了半碗肉\footnote{藏得固然妙。翻得更妙。}。怒道。這不是肉。你就說沒有。不給我吃。大家吃不成。連碗摔到院子裡去。便宜那狗吃了。他氣狠狠而去。黃氏雖是心疼。却敢怒而不敢言。見他去了。放聲大哭道。我的兒喲。你好狠心。可惜我的肉喲。我心疼死囉。肉喲。可惜肉喲。我的命好苦喲。儘着鼻涕眼淚數說着。哭個不休\footnote{他哭個不休。看書者却笑個不休。}。有個鄰家的婦人偶然到他家來。見他這等數着哭。倒吃了一驚。只當是他兒子死了。忙進來相問。他哭着實吿。那婦人忍不住掩口含笑而去。恰好竹淸來家。看見院子裡那破碗。跌足叫道。哎呀。這是怎的來。把個碗打破了。可惜了的。黃氏聽見丈夫的聲音。纔住了哭。竹淸進到房中。見黃氏淚痕滿面。問他爲什麼。黃氏不肯說兒子摔了肉。說道。我方纔失手打破了一個碗。由不得心痛墮淚。竹淸道。經過這一次。下次小心些。把兩隻手捧得緊緊的要緊。再說竹思寬先在家中。還是偷着東西去賭。後來但他父親不在家。他竟走來。不拘衣服器皿。可當賣的。拿着就走。他娘又不敢阻攔。及至竹淸回來。黃氏還不肯吿訴。等着要用的時候沒了。他方纔說兒子拿了去。竹淸查查家中東西。已不見了許多。暗暗的叫苦。後來要出門。將値錢的物件都鎖在櫃中。鑰匙自己帶在身邊。一日。竹思寬輸了錢沒得還人。着了急。走回來尋當頭。一無所有。問黃氏要。黃氏道。可憐可憐我。那裡有甚麼當賣的東西。穿的在身上脫不下來。他四處翻了一會。只有黃氏的一條藍布單褲\footnote{翻着了一條褲子。趣極。}。他見不濟事。見老子床上的被。夾着就走\footnote{夾着就走。妙。是個輸急了的樣子。}。黃氏急了。攆出來道。褲子我不穿罷了。這被是你爹晚上要蓋的。你如何拿得去。他頭也不回。一直去了。竹淸來家。見床上沒有了被。問起來。黃氏方說兒子連他的褲子都拿去了。竹淸臉都氣白。這是晚上要蓋的。各當鋪去問。贖了回來。黃氏忙把褲子捲緊了。暗藏在那財神的案桌底下\footnote{這一藏。妙。料兒子再想不到。}。此後竹淸輕易也不敢出門。一日。竹思寬回來。竹淸問他道。你也不小了。儘着往下流裡頭走。一個錢朝死裡賭他做甚麼。你想從小頑錢。到如今輸了多少。可曾見你贏回一個錢來。你這樣一想就該改了。他怒目而視道。你說我下流。我偏下流個樣子與你看看。你說我賭。我先還是小賭。你不說我還好些。你旣是這樣說。我且去大賭賭着。口中嘓嘓噥噥的去了\footnote{此等下流的逆子多甚。吾聞其語矣。又見其人也。}。他果然竟走到屠家去賭。這些放賭的都有耳目。知他家有數千之產。就讓他擲。一場就輸了一百餘兩。同他頑錢的。不是光棍。就是大老的兒子。到他家門口來要。竹淸先也捨不得。終日來打鬧村辱罵得不堪。聲聲叫他娘出來剝褲子。竹淸受不得。忍着疼。沒奈何。替他還了。他見老子替他還得容易\footnote{世間多有此類。正經處不捨一文。替兒子還輸贏帳則不惜。吾不知是何肺腸也。}。越發放心去賭。如此多次。竹淸也替他還過有千餘金。又不敢奈何兒子。只自己氣得抱生怨死。有相好的親友叫到衙門去吿。他因係獨子。又捨不得。一時間疼起銀子來要去吿。過後心疼兒子。自己又中止。\endnotemark[2]因此竹思寬越發肆無忌憚。他一日同着幾個光棍耍錢。他的手氣順。從早至午。贏了有三四百兩籌碼。歇了算帳要銀子。衆人道。稜子磨了水了。把你那妄想心打掉了罷。爺們的錢都是好贏的。只好等你那一日輸了。慢慢的准帳罷。他急了。道。每常贏過我的不知多少。輸了就要。我好容易今日贏了。想賴我的。衆人道。實話對你說罷。爺們原想贏你這腫嘴。今日不幸輸了。是你的造化。不要講三四百兩銀子。你想要三四百文低錢板子。大約還不能夠呢。竹思寬又氣又急。就罵了幾句。被這三個人齊上。拳頭嘴巴打得嘴鼻中都是血。滿臉紅紅紫紫。大包小瘤。把頭上的瓦楞帽子。身上的海靑。扯得稀爛。正鬧着。恰好他舅舅路過。喝住了。問起緣故。竹思寬將前事奉吿。他舅舅向衆人道。這個不長進的奴才。每年來輸了頭二千兩。今日纔贏得這一場。列位就沒有。也該好說。不犯着就動手。贏了他的要。輸了他的打。自己也過不去。這是鼓兒詞上說趙太祖的賭法。輸打贏要了。衆人見他有些體面。不敢回言\footnote{體面人處處行得去。可慨矣夫。}。況自己原也理虧。還洋洋的道。饒他這一回。再要想問爺們要。叫他試試爺們的利害。就走去了\footnote{是起光棍的行徑聲口。}。他舅舅送了他到家中。忿怒向竹淸道。旣有本事養兒子。怎麼就沒本事管敎。叫他在外邊賭錢闖禍。作何了局。你旣不敢管他。送到官。連同賭的人一齊處治幾個。也戒戒他的下次\footnote{果是上策。}。那竹淸半晌吐出一句道。我何嘗不想到。倘送到了官。怕親戚們看着。沒臉面\footnote{何沒臉面之有。老牛心性。令人不解。}。他舅子大怒道。好好好。你兒子這樣不長進。倒有臉面。你這等出奇的心腸\footnote{眞是奇心腸。}。就怪不得有這樣好兒子了。虧你怎麼活了這樣大年紀。說得越發怒氣上來了。道。呸\footnote{可謂不顧而唾。}。孽障\footnote{眞是孽障。罵得不差。}。後來不知怎樣現世呢。就忿忿的出去了。竹淸望着竹思寬。道。今日你試着了。輸了白白送與人去。贏了不能得。還要捱打。你想你輸過了多少。有這兩千輸過的銀子。要開個鋪面。做上生意。又操練出人來。何等體面。今日叫舅舅這樣罵我。你也過意麼\footnote{眞老牛。還有姑息兒子嗔怒舅子之意。}。竹思寬道。你要肯給我銀子開鋪子。我好戲得賭錢麼。我是閒着沒事做。纔幹這營生\footnote{人生在世。何事不可做。閒着沒事便去賭錢。奇語。非下流人不能說此下流語。}。竹淸道。給你銀子開鋪子。又好拿了去賭。他道。要開上鋪子。做了買賣。還要賭錢。那也不是人養的。竟是驢子肏出來的了\footnote{他倒也罷了。難爲他令堂。}。竹淸道。據你想。做個甚麼買賣。他道。小本生意。硶滋滋的。我不做他。本錢大了。你又不放心。得五百兩銀子。開個錢米鋪也罷了。竹淸聽得兒子說有生意做就不賭了。父母愛子之心無所不至。巴不得他望成人裡做。遂取出五百兩來。租了三間鋪面。搭了一個夥計看銀水寫賬目。又替他做了一身新衣服帽鞋之類。擇吉開張。他果然竟有三四個月不曾去賭。把個竹淸夫婦喜得沒入脚處\footnote{眞是出奇。不但竹淸夫婦歡喜。看書者亦以爲異。}。竹思寬人物生相也好。口中言談也好。見人一團和氣。又捨得。這些在街上開紬緞鋪布鋪雜貨鋪的人也都相與。時常請到茶館中吃茶。或大葷館中吃酒飯。衆人也都還席請他。見他少年圓活。倒都看得他甚高\footnote{偏是伶俐小夥子好幹此等下流事。余不解是何心也。}。他足足戒了有半年。忽然賭興又發。忍不得了。走到屠家。一夜就輸了五百餘兩。就把錢米算與了人\footnote{倒也爽快。}。人來擡錢米的時候。夥計纔知道。要攔阻時。竹思寬反拿刀子要同他拚命。夥計無法。只得連忙去報與他父親。竹淸跌跌舂舂跑了來時。錢米已去。只剩了個空鋪子。連竹思寬都不見了。搥胸跌足。怨天恨地而回\footnote{可謂。兒子一去不復返。錢米今已空悠悠。}。你道竹思寬往何處去了。他把鋪子輸去。要想翻本。手頭無鈔了。走向素常相識的這些鋪子裡說謊道。水西門外上江到了幾船米。客人家中有事急於要回。只照本錢就賣。就照眼下時價也有四五分利錢。家父的銀子都放在外邊。一時收不回來。鋪子裡又沒有許多。家父叫我到寶鋪。懇祈暫挪一用\footnote{題目甚佳。可惜把文章錯作了。}。或五十兩。或三四十兩。三五日內米一發了。如數送來奉還。衆人見他現開着鋪子。也有與他父親相熟的。又知他家殷實。況他說得甚是委婉。可有不相信的。各鋪中三五十兩不等。共借了四百有餘。拿到屠家。全全送入他人囊中。只落得辛苦了半夜。這些鋪家在他鋪子門口過。見關着。還以爲是他夥計們同去照料發米。過了四五日。仍然高鎖如故。訪問左右鋪子。方知他做的那些妙處。衆人全知道了。約會到他家來問竹淸要。竹淸見是兒子做的事。又都素常相識。情理兩個字都說不去。只得咬牙跌足。如數償還。這一下。將他生平刻薄所掙之物。盡行罄囊抖出。所剩房產田地不過五六百金。還是他三十多年前的原本\footnote{諺云。人有千算。天只一算。刻薄一生。終歸烏有。刻薄者何益。此等處須當着眼。}。竹思寬這兩場送去了千餘兩。他雖然不怕父母。自覺無顏。老老在屠家住着不回。零零星星又輸了一二百兩。衆人得慣了濟。又來尋竹淸。竹淸此時囊中已無物了。只得學那脫空祖師妙法。兩隻推聾的耳朶。一個裝啞的嗓子。塞耳弗聽。緘口不言。後被辱罵得不堪。他此時也將七十歲了。出來說道。我幾千兩的一分家私。被你衆人勾引我那不成器的孽障。弄得精光。如今只剩我一條老命。你們拿刀來殺了我罷。走到街上大聲叫屈。拉着衆人撞頭磕腦要尋死。衆人先還以爲他像當日好騙。不想老兒弄光了。着了急。要來拼命\footnote{眞叫做人急智生。}。誰不怕事。一轟就走了。回來叮着竹思寬要。竹思寬沒法。想出個妙策道。我家的銀子雖沒有了。房產地土還値千兩。但文書在我老爹手中出不來。我寫下一張欠約。等老爹死後磬一響就還錢。今日且叫我擲擲。翻翻本着。衆人知他家的產業還値數百金。就依允了。兩三個老把勢同他下場。一夜就贏了他七八百兩。立逼着將房產地土都寫了賣契。同夥許多人做保。這幾個贏了的。拿出幾兩銀子來。備了幾桌酒席酬謝衆人。竹思寬却也吃了一飽。欣欣自得\footnote{眞便宜。七八百兩賒帳還了一頓現飽。余有一親曾錫侯擁資巨萬。衣食不浪費一文。頭髮長約寸餘亦捨不得錢剃。到親友家遇有剃頭者。方擾一剃。其吝如此。遇賭則不惜。他有一茶館。名曰爽月居。連房子器用家伙。係二千五百金所置者。偶一日夜輸去三千金。以館算與他。喜謂人曰。我二千五百銀子的產業算了三千金。豈不便宜。竹思寬心亦類此。}。此後衆人知他是屬太監的。淨了身了。再不同他大賭。只賭現梢。他身邊一文賭本皆無。着了急。想起他一個表姐夫來。這人姓蘇名才。就是黃氏的姪女婿。他有千餘金資本。在外路販買雜貨。竹思寬走去看他。蘇才見了。甚是歡喜。說道。你姐姐對我說。你竟改過不耍錢了。開了鋪子。這樣往成人裡走還不好麼。這是姑老爹的積行。他借因兒說道。開鋪子。奈本錢短少。轉不過來。老爹放的帳一時又收不起來。今日買了一樁米。差二三十兩銀子就撅住了。我聽見姐夫回來家。一來看看。二來想問姐夫挪二三十兩銀子權用一時。三兩日就送來。蘇才道。我的貨物還沒有發動。銀子是沒的。旣等着要用。把你姐姐的頭面且當幾兩用罷。遂叫妻子拿出幾件首飾。說道。這當得二十兩銀子了。你拿去罷。竹思寬道。一客不煩二主。旣承姐夫姐姐美情。索性全美了我的事罷。再得十兩就夠了。省得我又去求第二家。蘇才想了一想。又對妻子道。把你我穿不着的衣服借些給他罷。他姐姐又將新衣服包了一大包袱與他。他說了一聲多謝。笑嘻嘻拿着去了\footnote{樂哉。}。到了當鋪中儘力一當。當了三十五兩。走到賭場輕輕送去。過了半個多月。蘇才不見他送來還。竹淸待親戚極淡。人都不甚上他的門。蘇才因要問他要東西。借此來看看姑丈姑母。坐下敍了幾句閒語。方說起竹思寬借的當頭來。竹淸聽得氣得兩淚交流。把竹思寬歷來所做所爲前後細說。蘇才聽了這話。知道這項物件他萬不能還了。去尋他要當票要緊。辭了出來。正走到街上。見二三個屎皮辣子揪住竹思寬在那裡鬧。蘇才看時。他連衣服鞋襪都沒有了。上穿一件小衫。下着一條褲子。赤着兩片精脚。蘇才上前問故。衆人道。他輸了我們十多兩銀子。只將一身衣服給我們。値不得頭二兩銀子。就想罷了。如何饒得他。蘇才道。列位看他這個樣子。還問他要命麼。勸列位撂開罷。衆人那裡肯依。這個一拳。那個一脚。蘇才看不過意。說道。列位不必動手。打死人是不要償命的麼。向順袋中掏出有兩數銀子。遞與衆人道。這個列位拿去買杯酒吃罷。放了他。如不肯。聽憑尊意。我就不管了。衆人先看竹思寬的樣子。知是逼不出來的。不過打幾下出出氣。見蘇才拿出銀子來解紛。實出望外。做好做歹放下他。向蘇才假說了幾句好看的話。笑吟吟往酒館中去了。蘇才向他嘆了口氣道。你這樣不成人。如何是了。我的東西你料道不能還了。把票子給我罷。幸而當票還在身邊。取出付與蘇才\footnote{疏財之姐夫遇着這不才之小舅。奈何。余閱此。偶憶起一故事。當年祝枝山在京兆。無以度歲。向各親友家借白領。詭云往人家弔孝。借得十數件。盡送質庫。新年人不好來要。燈節後皆來取討。答云。早來好來。遲到如今。當票也不知何處去了。竹思寬當票竟還在。較此向(尚)妙。}。蘇才道。你這個樣子。還有臉面在街上走麼。我送你家去。他還不肯。蘇才拉住不放。送他到了家。把上項事對竹淸說了。然後回去。竹淸見賢郞這樣個形狀。也無言可說。只嘆了幾口氣。落了幾點淚。老牛䑛犢。沒奈何。把舊長衣又給他一件穿上。忽一日。黃氏的姪兒騎了頭驢子如飛而來。說道。母親偶然得了暴病。叫我來接姑媽媽。快家去見一見。黃氏道。你快去碼頭上叫乘轎子來。他忙忙去了。及至叫了轎來時。驢子已不知何往。找竹思寬也不見了。他急得暴跳道。我怕走得慢。借隔壁磨房裡驢子騎了來。這沒得說。又是大兄弟拿去做賭本了\footnote{偷得有趣。}。竹淸在房中羞得連聲也不敢嘖。他急了一回。沒奈何。只得步行同黃氏去了。竹思寬把驢子偷去。做了二兩五錢銀子耍子籌碼。頃刻送得精光。他打聽得舅母沒有了。到六日上黃家。正念首七經。他毫不覺恥。走了去幫忙。他娘舅表兄見了他。雖是一肚子的氣。家中有許多親戚男婦。當着人又不好發洩。看妹子姑娘的面上又不好攆他。到晚間和尚施食。至三鼓方歇。人都困倦了。一齊睡着。次早起來。靈前的供器都沒有了。衆人不見了許多孝衣。連白布桌圍都拿了去。出去看時。門已大開。查點衆人。單單不見了這位姓竹的賢甥\footnote{這一偷更趣。}。他娘舅急得亂叫道。你寧可把別的東西偷些去罷了。把孝衣拿了去。這忌忌諱諱的如何重做。這是如何說。忙叫兒子拿了銀子到屠家賭場上找着了他。要了票子贖了回來。把個黃氏羞得要有個地洞也就鑽下去了。過了幾日。黃氏歸家。把乃郞妙處吿訴了丈夫。竹淸有年紀了。羞愧氣惱齊集胸中。漸漸飮食少進。懨懨成病。這竹思寬從此也不想回家了。在屠家做了幫閒。他沒得錢。却也沒人再同他賭。他在傍邊拈幾個飛頭。十日半月積得幾文。就同人小耍。他雖輸完了家業。却把武藝練精。竟不得輸了。屠家見他伶俐。相幫照看賭帳。拿拿紅兒。倒離不得他\footnote{可謂學成看賭藝。貨與放頭家。}。且說竹淸久不見兒子回來。門口也無索賭帳的來鬧。家中所餘也還儘可供穿吃。眼耳淸淨。病倒覺好些。久不出門。一日。拄着根拐。到街上茶館中坐坐散散心。走堂的送上一壺茶來。他忙道。不用茶。我略坐坐就去。那掌櫃的素常認得他。知他吝嗇。怕費茶錢。笑道。送你老人家吃。不要茶錢的。他方留下。篩了一杯吃着。見隔座兩個人也在那裡吃茶說笑。他聽了聽。是談他的家務。一個道。爲人在世。銀錢誰不愛。要十分刻薄。觸了鬼神之忌。遠報兒孫近報身。再躱不掉的。像竹思寬的老子那老孽障。我雖不曾會過他。聽得人說他的刻薄嗇細。也就是天地間少有的了。窮苦人吃了他多少虧。掙了一輩子。弄了這麼個家私。也沒有享用一日。養了這麼個好兒子。輕輕的送了個乾淨。背後還落了人多少笑罵。那一個笑道。我前日在老屠家。見竹思寬把房產地土都輸了。寫了文書給人。只等老兒一倒頭。都是別人家的。那老孽障不知道兒子的這件事。還坐在鼓裡呢。他知道這話。大約也就要氣死了。竹淸聽了這一片話。一口氣幾乎回不過來。把腿都氣軟了。定了半晌。纔掙着回家。向黃氏說知。夫妻悲切了一場。他的舊病原未曾大好。復着了這口重氣。成了一個氣蠱。又捨不得錢醫治。臨危時。心中想道。這個孽障。我同他前世不知是甚麼寃家。今生相遇。那裡是甚麼父子。他同我拗了一生。我如今要說我死後要他埋葬我。他是決不依的。不是燒了。就是棄之於水。我只要叫他火化。然後水葬。他就定然埋了我。煩鄰舍到屠家尋了他來到跟前。說道。我生了你一場。養你三十多歲。我不曾得你一日的孝養。爲一個賭同下流。我勸了你幾千百遍。越勸你越要拗着去做。我如今要死了。也管不得了。任你去罷。但我死後。料道也沒人將來到我墳前燒錢化紙。你不必土埋。把我燒了。棄在水裡頭罷。倒還乾淨。說畢。就閉目而逝。竹思寬每當他老子勸他不要賭。他更賭得利害。勸他不要下流。更往下流裡走。他何嘗不知道自己的不是。他常見有同他一般的人。也勸道。你們這是何苦。不要像我這樣不長進。但他是生來的逆種。明知故犯。今聽了父親臨終的話。他一時心中也覺難過。忖道。實是我同他拗了一生。父子一場。他今日臨死的言語。再不依他。也覺太過不去些。他在生時我恨他者。爲他時常在我耳邊絮聒。以不入耳之言相加。所以拗他。如今想起來。他掙了一生。一分家私我全敗盡。他也並不曾敢把我怎麼樣。憑良心說。我要有這分家私。他要花了我的。我也還有好些依不得呢\footnote{世間忤逆心腸惡子聲口。大都如是。}。只想他的好處。不要想他的歹處。我後來或者生了兒子。也要想他孝順呢。人常說。死了死了。外人還人死仇解。何況一家。罷罷罷。把寃仇解了罷。我依他的遺言罷。遂買棺裝殮擡出去。一火焚之。揀了骨殖。家中拿了個舊瓶盛了。去到城外賽虹橋上投於中流\footnote{不逆父命。眞是孝子。}。這些債主見他父親死了。都來索逋。他將房產地土並囊篋中所剩盡情付與。黃氏是兒子降服了的。可敢擅發一言。暗氣在心。又是悲痛丈夫。不數日而亡。竹思寬想道。他雖然不曾說土埋火化。但他夫妻自然該在一處。也就燒了。棄於賽虹橋下。他的房子俱無。孑然一身。就依身在屠家賭場中過日。他雖把一分家私送盡在這賭之一道。倒也熬成了一個相識。屠家賭場上來耍錢的財主。官宦門的子弟多。也個個奉承。又習會了這篾片道路。雖吃穿二字不愁。但他自幼花用慣了。所以到三十餘歲。並無家業。也不想要妻子。他有個混名叫做賽敖曹。他這根陽物生得其實放樣。橫量寬有二寸。豎量及一尺。休說是良家女子。就是淫娼宿妓。見了他這驢大的行貨。也驚個半死。有那大膽淫浪的妓女。貪他加倍的嫖錢。又想嘗嘗這頂大的滋味。略試一試。就肉綻皮開。啼哭而遁。後來妓女中拿他做了誓辭。凡他的同類中有說誓者便道。若沒良心。叫他遇了竹思寬的㞠子。他有這個大名在外。妓女中再不敢招惹他。因有這個緣故。把娶妻一念丢向九霄雲外。再也不想。他雖遇幾個婦人。只算做登門奉拜。並不曾做入幕嘉賓。那陰戶之形雖然熟識。却還未曾嘗着個中滋味。不想天配奇緣。偶然遇着郝氏的這件傢伙。竟是生成替他裝本錢的一個皮袋。郝氏雖是個半老佳人。風騷比少年尤勝。當日也素常聞竹思寬的大名。不敢造次。後來想道。彼人也。我亦人也。我何怯乎哉。竟同他試了一試。誰知悠然而入。毫不覺其煩難。竹思寬遇了這個開大飯店的主兒。方得飽嘗一頓異味。始知婦人裙帶之下眞有樂境。起先竹思寬以爲自己腰間這廢物是沒用的了。今日方知天生一物。必有一配。因此鑽頭覓縫。去弄了錢來奉承郝氏。圖他歡心。可以常常領敎他這個妙物。但他一個好賭的人。如何得有餘錢。有個緣故。他雖好賭。比不得這些少年孟浪的人昏頭昏腦。脖子上揷一面小黃旗。做那送錢的鋪兵。他於此道中花了數千金。練了二十餘年。而却甚是在行。他在賭場中着脚久了。某人有錢。某人沒鈔。某人是把勢。某人是雛兒。個個都有一本老册子在他胸中。他或遇着有錢大老。又都是在行的。他不耍。就只在傍邊撮趣奉承。或是幫着算算籌碼。或是記記帳目。誰人贏了。他拈些飛頭。這些在賭場中頑錢大老。十個中有九個肯撒漫。見他又善於幫襯。又會奉承。且相識久了。分外肯多給他些。或者造化。遇着兩個有錢的雛把勢。他便勾上一個老手上場。他在此道中歷練久了。鉗紅捉綠。手段也自高強。所以十場中倒有九場被他席捲而去。他得了這種錢財。別處一文不捨。只做件把衣服穿穿。每日飯食是在賭場中擾的。終年連柴米都不消買得。積得多了。只留些賭本。餘者盡送與郝氏。爲陰戶錢糧之費\footnote{竹淸生他一場。不曾孝養一日。郝氏之陰戶。他供了無限錢糧。竹靑之嘴竟不如郝氏之陰。刻薄人宜生若是之子。}。數年來也塡還了他不計其數。郝氏這個陰戶。就像和尚們化緣的銀櫃一般。捏上兩個泥娃娃。張着一個鍾口大的小口袋。站在櫃上。任你撂上多少錢。都掉了下去。他這樣個小肉窟窿。竹思寬塡了許多錢。總不見一些影響\footnote{一羊客販羊數百隻。貨賣。偶嫖一妓。相得甚歡。陸續將羊盡准與彼。一日臨行。謂此妓曰。我同你相厚一場。可將你此物與我細看一看。其妓即與看之。此客嘆曰。這樣一個牙也沒有的一張嘴。怎麼就吃了我幾百隻羊。幾百羊入內尤不覺。況於竹思寬之零星錢乎。}。郝氏自從幸會過他這件放樣的陰物。他的自然成了個出楦的陰戶了\footnote{陰戶而曰出楦。與鐵陰是一樣新聞。}。間或有嫖客來與他相交。此訝其小。彼訝其寬。都駭然而走。郝氏有個最相厚舊孤老。極善詼諧嘻笑。他的陽具當日也是郝氏贊揚過。考在一等數內的。偶然來看他。溫溫舊帳。帶了個包兒來做東道之資。郝氏備酒飯款待他。同他吃了飯。留下過夜。二人解衣上床。那人將他陰戶一摸。竟如兩片破瓦。吃了一驚道。婦人中有如此巨物耶。眞可謂三日不見。當刮目相待了。我見武則天小說內。說他陰如片瓦。我以爲後人罵他的話。據此言之。想亦不謬。只得上他身去試試。寬而無當。陽物在內如鐸中木舌一般。左右晃蕩。總無涯際\footnote{妙譬。}。又宛如措大走路相似。任着兩邊搖擺\footnote{此譬更妙。}。郝氏見他在腹上一動一動的。內中却全然不覺。問道。你弄便弄罷了。又不放進去。只管亂動做甚麼。那人暗笑道。好大物。拔出道。我撒抛(泡)尿。來到窗下。見一個搗蒜的石杵。有手腕粗。有六七寸長。悄悄拿了進來。假意爬上身。用手將那石杵往陰中一塞。一下全入。郝氏道。你怎把陽物凍得冰冷的了。那人吐舌道。好利害。我定要試試有多深多大。又道。我還要出個大恭去。又下床來。燈影之下見床側有一個搥衣的大棒槌。笑着拿了上床。又爬上肚子。將那棒槌對了陰門。兩三搗送入大半。郝氏覺內中有些搗着底子。他暗想。惟竹思寬的可以至此。他何得亦有此異物。忙用手去摸時。原來是一個大棒槌。笑罵道。促恰鬼。這是我掙飯吃的本錢。又不是石臼子。怎拿大棒槌搗起來了。那人也笑道。你不聽見古詩上說的。長安一片黑。萬戶搗屄聲麼。郝氏大笑道。我聽得是一片月。搗衣聲。那人道。月下自然是搗衣。你這個屄只好黑地下搗。雖兩件事各有不同。總要用的是這個棒槌。兩人一齊大笑\footnote{昔有一張姓之兒與陰姓之女聯姻。臨娶時。張姓之妻命媒人傳話親母云。我家大大一張。粧奩須入得我張家門。纔出得他陰家的門。親母向媒人云。你拜上親家母。他雖是大大一張。我的陰門也不小。正是郝氏之謂。}。那人知弄不得的了。見他這種奇牝。不住用手摳挖。郝氏被他引得不疼不癢。甚是難過。淫水長流。那人手皆精溫。將五指捏攏。戲往內中一塞。不想滑〖氵韲〗〖氵韲〗把一隻手送了進去。直至手腕。郝氏猶然不覺。那人大駭坐起。將一隻脚往陰門內一蹬。進去了半截。郝氏摸着。笑罵道。我這東西是給你當破皮靴穿的麼\footnote{此何足異。有一笑談。一妓陰大無比。〈一〉有一熟客到他家。此妓正赤身晝臥。此客戲將他鞋脫下。塞入陰內。妓醒。覓鞋不得。問他鴇母。鴇母道。你穿在脚上。如何得不見。此妓上淨桶小解。鞋自陰中掉出。妓笑呼鴇母道。不知那個促恰癆。把鞋塞在我這裡頭。纔掉了出來了。鴇母道。前日不見了兩把大酒壺。想也是人同你頑耍。塞在裡頭了。你尋尋看。酒壺可以塞上兩把。而況於半隻脚乎。郝氏若與此妓相較。算緊美之甚了。一笑。}。那人笑得滿床亂滾了會。方纔睡了。次日回去。當一個笑話吿訴人。就有編出個吳歌來唱道。

\begin{quotation}

郝老鴇兒忒子個騷。一個陰門賽子個破瓢。被人拿了當子個皮靴套。只好賽敖曹做他子個孤老\footnote{個音故。}。

\end{quotation}

人聽他有這件奇物。再也沒人來領他的大敎。因此這郝氏愛竹思寬的肉棒槌猶同性命。今見女兒大了。有他這件豆腐腦兒似的嫩貨接待。不愁那財源不滾滾而來。做個富婆\footnote{富翁則有之矣。富婆此方僅見。}。況且自己已四十多歲。成了老家人。也是過時的了。恐怕竹思寬憎嫌他這個乾蝦癟鮝\footnote{奇語。蝦則謂其形。鮝則喩其臭。}。一時見棄。那裡再去尋這驢腎般的佳配。所以托他只要替女兒尋得個好孤老來。不但分惠與他。且自此以後。有女兒做了穿衣吃飯的本錢。他這件老朽牝物情願奉申謝敬。白送與他受用。一文不復再索。竹思寬聽了這話。銀錢還是末事。若謀事不忠。恐他惱惡起來。出諸大門之外。何處再尋這深鬆濶大的妙物\footnote{此等妙物或者還有。}。豈不守了活寡。因而十分上心。一日。在賭場中有一個舊相識。姓鐵名化。是個回子\footnote{回子。}。有三十多歲。他自幼刁鑽古怪。促恰尖酸。所做所爲。出人意外。八九歲時。他父親送他到一個老學究館中敎他讀書。他別樣的事件件皆能。惟到了書上便念不下去\footnote{此等學生多極。}。這先生姓眞名佳訓\footnote{一個好先生。不愧姓眞。}。是個迂板的老儒。毫不放鬆。常施夏楚。無一日不見敎他幾下。他懷恨在心。這先生年紀雖纔五十多歲。却是一嘴白鬚。一日將要科考。聞得新宗師係少年進士出身。最愛少賤老。少者雖文章欠通。他以爲靑年可以培植。皆取前列。老者縱是宿儒。盡置末等。這先生鬚髮如銀。自覺難看。恐怕一時考低了。不但壞了聲名。且不得科舉下場。要尋些烏鬚藥來烏黑了。方好去考。又不知何處有好方。但是會着朋友就問\footnote{一老漢納寵。有一嘴白鬚。用烏藥烏黑。其寵一日見之大慟。此老駭問之。答曰。我見了你烏乎。我怎不哀哉。娶妾者。烏鬍自是常情。不意應考亦烏髮也。}。鐵化揣知其意。向先生道。我家老爸有上好的烏鬚藥。先生道。你如何知道。他道。先生當我老爸的鬍子是黑的麼。也是雪白的。我時常看見他到晚間臨睡時用些藥包了。過了夜。第二日早起。就烏油黑的。先生聞言甚喜。向他道。你晚間回去時。請了你父親來。我有話說。他道。我老爸出外做買賣去了。這一向還沒來家。先生要藥。家裡有。我問母親要些來送先生。先生道。也罷。你不可忘了。到了放學的時候。將散時。先生又叮囑他道。我還等着你拿來纔回去。他滿口應諾。如飛的跑到家中。忙忙的摘了些紅鳳仙花。同些礬搗爛如泥。用紙包了。送到館中來。詭對先生道。我母親說來。這個藥見不得風。不可打開了看\footnote{妙甚。打開恐看出假也。}。只到臨睡時用塊小絹帕包在鬍子上。明日就漆黑的。兩鬢也擱上些。再用包頭紮住。也就黑了。那先生是至誠的人。信以爲實。到了家中。果然到臨睡時方打開。包了就睡。過了一夜。次早起來。對鏡打開一看。吃了一驚。不但一嘴通紅的鬍鬚。同兩鬢連臉上。斑斑點點都弄紅了。若再有個紅臉。竟像一個火神。他有個女兒見了。說道。這是誰拿染指甲的鳳仙花捉弄爹爹的\footnote{閒中提出此女。後來嫁干不驕。方不是劈空謅出。}。眞佳訓被他提醒。方知爲鐵化所耍。一兩日就要赴考。眞急得要死。忙用水洗肥皀搓。越洗越紅。反被肥皀搓得更光亮起來。沒奈何了。只推有病。等到後來趕遺才吿大收罷了。門也不敢出。足足在家躱了有一個月。紅方退了。他起先是一嘴白鬚。到如今竟弄成鵝黃顏色\footnote{眞先生若是白面。倒合了相書。相書云。銀面金鬚。大貴之相也。}。曠了一個多月的館。那日一肚忿氣走到館中來。傳齊了衆學生。鐵化也來了。先生要打他。他道。我又沒有犯了學規。先生爲何打我。先生道。你這樣小小年紀就這等壞心術。你前日弄的是甚麼藥哄我。他道。我何嘗敢哄先生。我母親包了藥。對我說了。放在桌子上。我往外邊出了個恭。怕先〔生〕等晚了。忙進去就拿了來送與先生。誰知一時慌忙。就拿錯了來。把我妹子染指甲花拿了來。我回去。妹子問我要花。我再去看。那個烏藥包還在桌子上。纔知道拿錯了。我要送到先生家去說這話。我又小。天也漸漸黑了。不意妹子將那一包藥搶過去。摔在地下。脚踏得稀爛。我再問母親要些藥。等先生第二日到館來送給先生。又沒有了\footnote{眞刁鑽。此想更妙。不如此說。恐先生再要。何以答之。}。次日就聽見先生有病。我敢戲弄先生麼。我在家被妹子罵了兩日。說把他的花弄掉了\footnote{此語不但不受過。且還有居功之意。暗含錯送了藥。因先生而受妹子罵也。眞頑皮。}。他此時要強說是烏藥。自然是他弄鬼無疑。定然是要打的了。他眞認是錯拿了。倒不好打他。先生聽他說得委委曲曲。有頭有尾。也就半信半疑。況前日問他小孩子要藥。自己也有些差處。也就饒過了他。這館中有個學生貝餘。那一日書背不熟。被先生責了十板。那日鐵化也責了幾下。先生回家吃飯。衆學生都回去了。單不許他二人去。貝餘喃喃嘟嘟罵個不歇。我們的皮肉被他打得生疼。鐵化道。你罵他。他又聽不見。如何出得氣。我有法兒報這個仇。我家遠。你家就在隔壁。你去要兩個大針來。揷在他坐的墊子上。等他坐了下去。把那屁股戳他兩下。只當替我們的屁股報仇。貝餘道。好是好。只我兩個在這裡。查起來。不是你。就是我。又捱一頓好打\footnote{貝餘有此想頭。尚不至大愚。但鐵化過於狡獪。故被其愚耳。}。鐵化道。我恨他不過。你只管依我行事。你再寫個帖兒。說鐵化拿針戳先生。他看見了。我破着再與他打十板。且出出氣。一絲也累不着你。那貝餘歡天喜地跑到家。要了兩根針來揷在墊子上。又寫了個帖兒放在底下。少刻。學生都來齊。先生也來了。到椅子上一坐。穿的是單衣。兩根針戳進去半截。疼得暴跳起來。忙把針拔出。拿起椅墊一看。只見底下一個帖兒。寫着鐵化用針戳先生。叫過鐵化來。大怒道。你這畜生。書不會念。單會做這些壞事。鐵化道。學生多多的。先生怎麼就知道是我。先生拿帖兒與他看。道。這上頭現寫着是你。鐵化哭道。我坌(笨)些。不會念書。人見先生常打我。就捉弄害我。要是我戳先生。我還敢寫名字放在這裡麼。先生想他說得甚是有理。遂叫衆學生來對筆跡。却是貝餘。先生要打他。他說是鐵化敎他做的。鐵化道。我就這麼呆。要是我叫你做的。肯敎你寫我的名字。你先在先生座上翻。我當你尋什麼東西。你做的事。倒反賴我。先生道。這與鐵化不相干。明明是貝餘這個畜生。因我早起打了他。他故下此毒手戳我。故意寫個帖子。想嫁禍與鐵化。這等奸詐可惡。那貝餘痛哭。只說寃賴他。口口咬定是鐵化。先生也還有些不決。有一個大學生。名叫干壹。說道。先生只究這兩根針從何而來。便知是誰了\footnote{隨手便出干壹。省筆。}。先生問鐵化。鐵化道。我不知道。貝餘說要出恭。去了好一會纔來。就在先生位上去翻。先生便打發干壹到他家去問來。回說道。他母親說貝餘說先生要根針用。拿了來的。先生笑道。畜生。你還有甚麼說。貝餘道。是鐵化叫我要去的。先生怒道。你還敢賴。鐵化叫你吃屎。你也肯吃麼。按在凳上。結結實實將貝餘重責了十板\footnote{甚矣。世間之寃枉事不少也。明是鐵化。反累及貝餘。鐵化狡黠便能脫禍。貝餘愚鹵便受其枉。以小槪大。片言折獄難矣哉。}。貝餘被鐵化耍了這一下。眞有口難分辯。却也背地被他罵了十數日\footnote{先生猶被其愚。而況於此蠢材乎。}。隔了些時。先生有事出門。回來時。正在鐵化家門口過。只見十多歲一個孩子。彎着腰在那裡哭着叫罵。走近前一看。原來是一個賣雞蛋的。在那一塊馬臺石上。把兩着膀臂圈着。把些雞蛋壘得高高的。彎着腰抱着。動也不敢動一動。一個筐子放在傍邊。問他緣故。那孩子哭道。這家十來歲的一個人要買我的蛋。叫我過數。又沒處放。他叫我把手圈着。他數了。說進去拿錢來取蛋。這半日總不見出來。我又不敢動。怕蛋滾下來打掉了。這一回又沒個人過。我腰也彎疼了。膀子也木了。再遲一會。都是打掉的數。造化遇了老相公。救我一救。先生知是鐵化所爲。恨聲不絕。替他拿過筐子。把蛋拾在內。裝完了。那孩子連腰還直不起來。向先〔生〕千恩萬謝。方提了筐子走去。先生到了館中。那鐵化已打後門早來到學館裡了。先生叫他過來。問道。你門口那個賣蛋的。可是你促恰做的事。他道。我吃了飯就到學裡來。並不知道甚麼賣蛋的。先生道。他明明說十來歲的一個孩子。不是你是誰。怒狠狠的要打他\footnote{方寫耍過貝餘。又寫耍這孩子。見得總是孩子。却沒有鐵化之尖酸狡獪耳。}。他道。我家有好幾個十來歲的。難道就是我。先生方纔不該放他去。該叫他來認認我。看是不是。先生此時打我。可不寃屈了我麼\footnote{眞頑皮。實是強辭奪理。先生亦無奈他何。}。那先生倒被他說得無言可答。又饒恕了。這館中有一個學生。姓白名華。他父親曾做陝西華州吏目。因爲無子。禱於華山所生。故命此名。這白華伶牙俐齒。善於搗鬼。衆學生替他起個混名。叫做白白嘴。因兩個白字重在一處不好叫。見他的嘴略有些癟。又都叫他白癟嘴。一日。先生他出。鐵化道。我講個笑話你們衆人聽聽。白華同衆學生都攢攏來聽。鐵化道。

\begin{quotation}

一個婦人往井上汲水。這日大冷。遍地都是冰。這婦人一時尿急了。見左右沒人。就蹲下去溺。溺完了纔要起來。不想一滑。站不穩。一個坐跌。把個陰戶就凍得粘在冰上。爬不起來。只得坐着。他丈夫見妻子不回。忙走了來。看見妻子坐在冰上。問他緣故。妻子吿訴他。因溺尿。凍住了。這男人沒法。想了一會。道。除非呵化了冰。纔起得來。只得爬倒。用嘴呵。不意把嘴同陰戶凍在一處。也動不得。忽有幾個挑脚漢過。見他二人如此。問其所以。男人嘴凍住了。說不出話來。婦人只得忍羞實吿。那幾個漢子上前看了看。內中一個道。這事容易。若要開時。我們拿過扁擔來。大家別嘴的別嘴。別屄的別屄。

\end{quotation}

衆人聽了大笑。白華見是罵他。說道。我也有個笑話說給你們聽。衆人側耳聽他說道。

\begin{quotation}

一個人念詩道。一色杏花紅十里。狀元回去馬如飛。傍邊一個人道。你念錯了。古詩是歸去。這人笑道。你好不通。歸字就是回字。回字就是歸字。

\end{quotation}

衆人笑得打跌。鐵化道。你們不要笑。我再說一個。

\begin{quotation}

一個人在畫鋪中賒了幾幅畫兒。家去貼着。畫匠要了幾十回。他總不肯還錢。畫匠氣不過。罵道。我肏你貼白畫的親娘。

\end{quotation}

衆學生齊拍手笑道。白癟嘴吃了虧了。白華也不答應。說道。你們不要笑。且聽我說了着。

\begin{quotation}

一個人纔睡覺。聽見外邊叫門。起來開了看時。不見有人。剛回來睡下。又聽見叫。只得又起來開了。又沒有。如此者四五次。這人急了。罵道。開了門不見人。關了門又叫門。我肏你叫門的祖奶奶。

\end{quotation}

鐵化見傷了他祖上。就面紅耳赤。爭競起來。幾乎相打。那大學生干壹。雖也是個少年。却板板策策。從不同人頑笑。衆人都懼怯他些\footnote{屢寫干壹少年老成。後來方見是成材也。}。是他一陣〖口么〗喝。纔鎭壓住了。鐵化又讀了一二年。他父親見他仍然一竅不通。叫他辭了先生。下來學做買賣。他在館中先生管着。還時常逃學。何況到了鋪子裡。他可肯安坐。終日在外閒撞。一日。遇見一個人。穿得甚是齊整。斯斯文文。也像個讀書人的樣子。遠遠走來。到了跟前一看。是一個大糟鼻子。他心有所觸。暗暗含笑。上前深深一揖。那人見他身上華麗。知是正經人家子弟。也回了一揖。道。小相公。素不相識。何勞賜揖。他道。我見先生這樣一個儀表。可惜把土星壞了。怎不治他一治。那人顣頞道。正是呢。也曾各處尋方醫治。再不能好。他道。家父倒有絕妙的奇方。一治就好的。效驗至極。那人歡喜得一把拉住。道。小相公。旣然如此。煩你引我到府上奉求令尊。倘醫好了。我自當奉謝。鐵化詭對道。本當奉陪同往。但晚生有些要緊的事到一舍親家去。不能相陪。先生只到三山街。問開氈貨店的鐵爸爸。人都知道。那就是家父。那人道。你原來是鐵爸爸的公郞。令尊雖不曾會過。是久聞名的。府上在禮拜寺間壁。我也認得。此時就去奉求。遂同他拱手別了。一直走到鐵家。煩門上人說了進去。老鐵回子迎了出來。讓到廳上坐下。問其來意。那人看見這老回子也是個大糟鼻子。紅腫如拳。甚是疑心。只得答道。適塗間遇見令郞。他見弟鼻紅腫。他說爸爸有上好藥方。特求(來)奉求。老回子大笑道。先生被那畜生哄了。因指着自己的鼻子道。若有好方。我的鼻子如何到這田地。他哄尊駕來同我會會糟鼻子的。那人恍然大悟。也大笑作辭而去。他一日走到一條僻靜巷內。見一家門內一個少婦。同街上一個老婦人說話。他見那少婦頗有幾分姿色。便站住。目不轉睛的呆望。那老婦見他年紀雖小。然看得太着相了。說道。你走你的路罷了。儘着站住看甚麼。他道。朝廷的官街。你站得我就站不得。是你看我。我何嘗看他來。老婦怒道。你明明的看着。還強嘴。把眼珠子剜了你的。鐵化笑道。你剜了我的眼睛。千萬撂在那位奶奶的褲襠裡。那老婦聽了。又好笑又好氣。攆着要打他。他纔跑了。他到十四歲上那一年。敎門是七月初一日過年。老回子把了一個六月的齋。大長的天氣又是那熱。一日餓到晚。還要幾次禮拜。直到星月上纔吃上一飽。到五更時。又撑上一肚子的牛羊肉油香哈哩洼。好捱一日。有年紀的人。饑飽不均。傷了脾胃。成了禁口痢。十數日就病故了。請老師傅同滿剌念回回經。即日下葬。都不必細說。過了數月。他一日偶然在門口閒站。只見一個斗笠草鞋漢子。問隔壁一個牛肉鋪內道。這裡有個鐵回子在那裡住。那鋪子裡的人就指着鐵化道。那戴孝的就是鐵相公。那人走到跟前說道。我是北門橋吳相公差來的。有封字送與相公。鐵化先聽見叫他鐵回子。已心中含怒。接過字來一看。假意道。原來你相公等着借這東西。你不要就去着。趕着拿了去。他忙忙的走進內邊。取了一個大圓盒。將磨盤拿了一扇裝入。四面封了。寫了一個回字封好。叫家人將盒子掇了出來。對那來人道。你家相公急等着要用。你路上萬不可歇。叫家人幫着他擡上肩頭扛着。那人道。重得很。是甚麼東西。鐵化道。都是要緊磁器。不要歪動。看打掉了。又將回字替他揣在懷裡。那人沒奈何。扛着去了。原來那人是莊子上纔上城來的\footnote{應前斗笠草鞋句。}。鄕下人老實。信以爲眞\footnote{說得活像。即城中人亦不得不信。}。一氣扛了七八里。肩頭也壓腫了。兩手扶着。肩也不敢換。生怕歪動打了。累得渾身是汗。面紅耳赤。到了家中。走到內邊。叫道。快來接接。壓死了。他主人忙跑出了一看。不知何故。用手來接。覺得甚重。那人道。正正的好生拿着。看打掉了。他主人問道。是甚麼東西。那人道。我知道是甚麼。鐵相公說是相公借的。急等要用。叫我一氣扛了回來。不可躭擱。他主人甚是疑心。道。我並不曾問他借甚麼。忙打開一看。是一扇石磨。不知其意。問他有回字沒有。那人喘吁吁的道。有。在我懷裡。取出來。汗都濕透了。拆開了一看。上邊並無多言。只得九個大字。寫着。

\begin{quotation}

來人無禮。罰扛磨一回。

\end{quotation}

下面有一行小字。道。仍着送回。庶可償罪。他主人笑着問道。你怎麼得罪了他。被他耍了這一下。那人道。我何得罪他。我到了那裡。問那牛肉鋪裡道。鐵回子在那裡住。他正在隔壁門首。那鋪內人指與我。我將相公的字遞上。他就進去拿了這東西。叫我扛了來。他主人大笑道。他惱你叫他鐵回子。故罰你當這回苦差使。那人方明白這個緣故。又是那可惱。又是那好笑。他主人道。說不得。你歇歇。還替他送了去。萬不可再叫鐵回子。那人嗗嘟着嘴。歇了一會。只得又與他送去。一日端陽佳節。秦淮河遊船如蟻。他家的小廝來向鐵化道。方纔奶奶打發我送粽子到火爸爸家去。我在貢院門口過。看見哈相公。鎖相公。馬相公。伍相公四五位擡着食盒。都遊船去了。鐵化想道。這幾個人都是我家敎親好朋友。他們就偏我去作樂。令人可惱。我如今給他個大家樂不成。遂叫那小廝忙去捉了些大靑螞蚱來。到家中尋出一個魚鮓罐子。裝了些稀糞淸。把那螞蚱拌上。用紅紙封好。吩咐小廝。如此如此行事。你到那裡切不可笑。那小廝。甚是伶俐。點頭會意。接過來拿着。一直到河邊來。遠遠看見這幾個人的船到來了。高聲〖口么〗喝道。哈相公。我家相公可在船上麼。那哈回子一看。認得是鐵家小廝。見他手內拿着個罐子。遂同衆人商議道。小鐵兒這促恰鬼。到處他占人便宜。他這小廝拿着的。定是人送他的東西。我們且騙了來吃了再講。遂叫船攏了岸。誑那小廝道。你相公纔上岸同人說話去了。就來的。你拿的是甚麼。那小廝見他說謊。忍着笑。用眼睃他船上。正中放着張桌子。鋪着猩紅絨氈。一個大宣窰花瓶揷着蓮花。香爐碁子之類。擺得好生富麗。面前一張金漆方桌。五個人圍坐着。鮮果美肴堆了一桌子。答道。我們家的夥計纔打安慶來。帶了幾罐魚鮓送我家老奶奶。老奶奶說相公不在家。定然是來遊船。叫我送一罐子來。衆人聽了甚喜道。你來得好。拿上來。你家相公就來了。那小廝將機就計。遞與船上人接了。他道。千萬交明與我家相公。我回老奶奶話去。說着。笑嘻嘻如飛的去了。衆人欣欣得意。拿過來揭開了紙頭。正要倒出來嘗嘗。誰知這些螞蚱悶久了。見了亮。一陣亂跳。衆人滿頭滿臉。渾身上下。無處不是臭糞。先螞蚱一跳時。大家齊叫。哎呀。不好。這一聲叫是張着嘴的。濺得那糞屑滿口都是。幾乎連腸肚都吐了出來\footnote{神情寫得逼眞。}。這桌子擺設的肴饌果品都成了屎拌了的。滿船臭不可聞。方知吃了他的這一場大虧\footnote{受得好便宜。}。連跟隨家人。在船頭船尾老遠的伺候。都還沾了些餘光。臭得都坐不住了。東西也吃不得了。倒在河裡。一場掃興。大家散去。歸家洗沐去了。累得船家把船都重新洗過。還不能除盡臭氣。再說鐵化房分中的姐姐妹妹嫂子。他母親接了五六個到家中來過節。都說道。今年人說秦淮河熱鬧得很。有一二十隻燈船。堂客們遊船的多得了不得。一年一度。奶奶帶我們大家去頑頑。也沾你老人家的洪福。他的那個胖女兒\footnote{胖女兒者。童自大之妻也。順便即帶出。用筆之伶便若此。眞妙。}。撒嬌撒癡的道。媽媽。你帶我同姐姐嫂嫂們頑頑去罷。這個一嘴。那個一舌。念誦得那老回婆倒也有些念動興了。叫了鐵化來。道。我聽得說河下今年十分熱鬧。我老人家了。也該去散散心。你可雇隻船。我同你姐姐妹子嫂子們大家去頑頑。他道。人山人海的。到那裡有甚麼趣。不如在家坐坐。還受用些\footnote{婦人遊船看燈。江寧之惡俗也。他此說却是。}。他娘怒道。只許你終日在外邊取樂。我就頑不得一頑。難道怕花了你的家私麼。鐵化不敢違拗。出來尋思道。我娘從沒這樣高興。定然是他們總成的。我且叫他衆人吃些虧。纔知道這船不是好遊的。主意定了。次日雇船。上面掛上簾子。他預先來囑咐道。旣要遊船去。不要多吃茶水。船上沒處溺尿。大家留神些。衆婦人歡喜非常。果然多不敢吃茶水。大家淸早吃了些飯。坐轎子到船上來。撑開遊賞。眞是熱鬧。看別的遊船上。有淸唱的。有絲管的。有挾妓的。有帶着梨園子弟的。還有吹打十番的。那兩岸河房。全是來玩賞的男婦。雖然耳中眼底有趣。但此時五月上旬。天氣正長。一輪火傘當空。四面日光透入蒸着。已是熱氣難當。又且是口中發渴。到了午後。衆人都是絕早吃的飯。此時也餓得很了。他娘催了三四次。他只答應。就有了。却不見拿上來。又停了一會。方纔送上。你道是些甚麼。都是鹹鵝。臘鴨。牛羓。醃魚。烘糕。薄脆。眉公酥。玉露霜。閩薑。橘餅。糖梅。圓片之類。衆人已餓得發昏。見了這些東西。儘飽一吃。過了一會。時已下午。越發炎熱。先已是渴了半日。又吃了這些鹹的甜的乾的東西。那喉管中都冒出煙來。如何受得。一個個都渴得昏頭昏腦。忙問他要茶吃。取了兩大壺溫茶來。衆人那裡還顧得。右一碗左一碗只是呷。渴了的人。忍着倒還罷了。一吃些涼茶。越發渴起來。只是要吃。兩壺不夠。又要了兩壺來。都吃了。大家灌了個滿肚。渴雖止了些。又過不多時。都有些尿急了。旣沒處溺。又說不出來。正在難忍的時候。誰知鐵化拿出些預做就的安息香來。他把皀角製成極細的末子。裹在這香上。捏了數十根。一齊點上。叫船家把船頭迎着上風。他靠着簾子坐着。那香煙同皀角末。順着風一陣陣的吹入艙中。這皀角末一聞着。噴涕打個不住。這些婦人正在那尿急的時候。勉強忍着。已是難過。這一頓噴涕。打得下邊的尿長淌。那裡還忍得住。都穿的是單紬紗羅之類。把裙褲衣服後面盡皆汚透。連膝褲同鞋都濕了。滿船板全是尿。忙忙叫攏船。叫轎子回家。他到了家中。反抱怨衆人道。我說不要去。你們定要去。我叫少吃茶。大家朝死裡呷。弄得滿船是尿。人看着是甚麼意思。明日被船家傳得人知道了。臉面何在。衆婦人都紅了臉不作聲。他娘也是一褲子的尿。也說不出來。大家只怨吃的茶多了。不聽他的好話。那裡知是他弄的鬼。過了兩年。他十八歲上。娶了媳婦火氏來家。他母親也就是那年死了。過了些時。他舅子火大生日。他去拜壽。有許多親友都在那裡留着吃麪。他偶到後園中去走走。見他舅子的後窗底下放着一個淨桶。就知是他舅姆子的。四顧無人。忙向鍋底下刮了些鍋煙子。將淨桶邊上週圍擦了。把蓋子蓋上。他留心少刻。又進來看看。淨桶已不在那裡了。知是舅姆子掇了去用。他走出來。在席上笑個不住。衆人問他。他只是笑。再三強問。他道。我說了。怕大哥惱。他舅子也不知是甚麼事。便道。你有話只管說。我惱的是甚麼。他笑道。我剛纔到後邊去。不留心撞見嫂子在那裡撒尿。雪白的屁股上一個大黑圈子。故此忍不住好笑。內中那哈回子同他最相熟。笑着罵道。你這砍頭的促恰鬼。單管嚼蛆胡說。他道。我一些也不胡說。你叫大哥進去看。要沒有黑圈。任憑怎麼罰我。他舅子也當是他眞正看見。倒不好認着犯頭。大家說別的話。就扠了過去。到人散後。火大走入房中。埋怨他妻子道。你可知道鐵家妹夫這個促恰鬼。你怎不留心撒尿。被他看見了屁股。當着衆人說得我怪不好意思的。他妻子道。哎呀。這是那裡的話。我在屋裡關着門撒尿。又不曾在外邊。他如何得見。火大道。他還說見你屁股上一個大黑圈子呢。那婦人道。呸。他難道見了鬼了。理那砍千刀的胡說。我好好的屁股。如何得有甚麼黑圈呢。火大道。你也不必罵人。也不必多講。看一看便知道了。叫他伏在椅子上。屁股撅着。掀開衣裙。把褲子扯下。果然一個黑圈。却被褲子擦得模糊的了。火大道。現有憑據。你還強甚麼。用手將他陰戶一擰。道。大約連這個紅圈也都被他看見了。那婦人紅着臉。氣忿忿的想了半晌。忙忙的去將淨桶揭開。點上燈一照。用手週邊一擦。滿手烏黑。方悟到是他弄的鬼。夫妻二人罵了幾句短命促恰鬼。大笑了一場。過了些時。鐵化又到丈人家來。他舅子不在家。丈人房中坐了一會出來。偶然瞥見舅姆拿着兩張草紙。往後邊毛廝房中去。關了門淨手。南京人家大家小戶都有個毛廝。大人家深宅大院。日間則用淨桶。晚間僕婦侍婢們去倒。小戶人家後窗之外即是毛廝。日間大小便皆在內中。淨桶只備夜間之用。這鐵化見他進去了。忙忙走到廚房內。兜了些米來。自廚房口悄悄直撒到毛廝門外。進來對丈人道。老爹。不知是誰偷米。把米撒了一地。直躱到毛廝裡頭去了。那老兒是當家的人。聽得有人偷米。走出來一看。果然一地。〖口么〗喝道。是誰偷米。說着。就走到毛廝門口。見門關着。當偷米的人躱在內中。就來推門。那媳婦聽見公公〖口么〗喝着來推門。又不好作聲。忙忙的靠住。連褲子也不及拽上。一個罵着往裡推。道。是那個奴才白日裡偷東西。這樣大膽。一個使着力往外頂。正在相持。鐵化跑到丈母跟前道。奶奶。你看老爹這樣大年紀的人。嫂子上毛廝。他老人家跟了去推門呢。那老婆子聽了。跳起身。忙趕來一看。果然那老鬼還〖口么〗〖口么〗喝的推呢。被這婆子氣狠狠上去兩個大巴掌。把那老兒打得愣愣掙掙的。他罵道。老沒廉恥的。媳婦在裡邊解手。你推門做甚麼。那老兒聽了。滿面羞慚。道。女婿纔說道是偷米的。我當是眞。攆了來拿。那裡知道是媳婦。及至出來尋女婿對話時。那鐵化已回去久了。過後不但老頭子好笑。連老婆子同媳婦想起他這促恰來。也暗笑了幾回。鐵化一日在街上閒蕩。有一個鄕下人上城來賣棗刺。那刺綑不緊。揸揸巴巴的兩大綑。用鐵尖擔戳在中間。挑得老高的走。不想晦氣。就在身上抓了一下。把衣服也就戳破了些。他正要動怒。那人看見。忙歇下擔子。上前陪禮道。小人一時失錯。相公看我鄕間窮苦人。高擡貴手。饒恕了罷。笑嘻嘻的儘着陪小心。鐵化見他這個樣子。俗云。嗔拳不打笑面。一時怒不起來。便道。你非有心。失錯了何妨。我正要買擔棗刺用。你要多少錢。賣與我罷。那人見他不怒反要買他的。忙道。相公饒恕了小人。我應該奉送的。府上在那裡。我就送了去。鐵化道。我如何肯白要你的。自然不虧你。你挑着跟我來。那人挑上肩。跟着他走。是鄕下人。認不熟城中的路。跟他到了一條小巷口。鐵化指着道。走大街繞遠好些路。打着小巷內過去。就是我家了。那人當是眞話。走了進去。擠住了。走不動。他在前面叫道。你狠狠的使力擠。過了這一節路。那前邊就寬了好走。那人果然用力往前擠。越走越窄。動不得了。再叫了幾聲相公。要問話時。已不見答應。那棗刺兩頭擠住。人在中間。要往後退。那刺先是用大力擠進來的。此時要退。那刺都倒揷在牆上磚縫中掛住。動也不能動一動。那兩邊來往的人都攔住了走不得。罵道。你瞎了眼。這個窄巷可是走得過去的麼。那人在中間叫寃叫屈的道。是一位相公要買我的。領我到了這裡。他不見了。何嘗是我自己來的。衆人知他被人哄了。等不得。都往別處繞去了。這賣刺的站了一會。人急智生。沒奈何。將身子睡倒。還打進來的這邊。從那刺底下爬了出來。他出便出來了。這擔刺却動不得。又想了一會。身邊又沒一文。只得脫了一件大布衫。當了幾十文錢。買了一根粗麻繩。打刺上撂過去。他又爬進去。拉着繩頭爬了出來。用力倒扯。那裡扯得動。你想這鄕間的人。自三四更天挑着一個重擔。幾十里走上城來。指望着賣幾十文錢。買碗飯吃。剩得多寡就回去的。那裡知道遇了這位盛德君子。耍這一下。弄得已是下午。力也費盡。腰也餓酸。要撂這擔刺。又捨不得那鐵裹的尖擔。只得到街口。再三央求了幾位過路的人幫着。纔拉了出來。看時。刺都掛掉了。料道日色將西。還要趕了回家。也賣不及。賭氣撂在空地方。把買繩子剩得幾文。買了碗飯吃。挾着尖擔回家去了。一擔刺不曾賣得。反當了一件布衫。又得了一根繩子。你道這個窮人可氣苦不氣苦。再說那時䘕中一個妓者。小字玉仙。生得雖不叫做美人。在他姊妹行中就要算出色的了。因此名重一時。熱鬧之甚。鐵化聞知。接了三番五次。總不得閒。這並不是他故做身分不來。天地間偏有這樣不湊巧的事。他閒了的時候。鐵化又不去接。到去接時。他又不得閒。鐵化那裡想到這上頭。見接了幾次不來。恨道。這臭娼根。他倚着有點名頭。這樣可惡。我把他的飯碗搗碎。他纔知道我的利害。這個陰隲老兒遂算計了一條毒計。那日他備了一分厚禮。又封了數兩嫖金。親自到玉仙家來。他果然不在家。那老鴇兒接着。讓進坐下。鐵化道。我慕令愛久了。來接過數次。都遇無緣。不曾得會。我今特備些須薄禮在此。媽媽收了。但是令愛得閒。就着人對我說去。我倒不定日子。老鴇兒也知鐵家是個財主。今見他尚未會面就這樣大出手。定是個好主兒了。那識他的深意。遂笑吟吟滿口道謝。應允不迭。過了兩日。玉仙家的鴇兒來說他姑娘今日在家得閒。叫他來請。問或是相公到他家去。還是接了來。鐵化心中暗喜。便道。我就差人去接。忙着人去定河房。吩咐家人到他敎門館中定了桌席。又着人去邀了四五位朋友來。無非是哈回子。馬回子。鎖回子。伍回子幾個同敎。然後叫個伶俐小廝。附耳囑咐。如此如此。這般這般行事。他遂到河房中來。玉仙也來到。看時。果然生得還好。他便善於詼諧。碩於酒量。所以人都愛他。少刻。這幾位朋友也來了。大家坐下。衆人見了他。都來奉承。也有贊揚他美貌的。也有說慕他大名的。那哈回子道。今日鐵兄同玉仙。眞是一對佳偶了。那玉仙微笑道。當日琵琶記上原有一句。

\begin{quotation}

這回好個風流婿。

\end{quotation}

衆人大笑。鐵化見打趣他是回子。心中雖怒。却不好發洩。也笑了一笑。叫拿上酒肴來。入席共飮。鐵化道。我素知玉仙大量。我們今日較一較高下。每人面前放一把自斟壺。自斟自飮。豁拳打關。不許代酒。不許錯斟。違者罰三壺。衆人都說道。好。玉仙自以量大。也不推辭。大家直吃到二鼓時分。都有八九分的酒意。衆人道。酒夠了。不要躭誤了你二人的好事。鐵化也就止住。又叫烹茶來。小廝們送上茶。此時酒多口渴。衆人都吃了幾杯。鐵化道。夜深了。衆位弟兄不回府罷。床鋪都預備下有。在此下榻罷。這是鐵化要留他們在這裡。明早好做大家一笑。衆人雖不知其中就裡。見天氣遲了。怕夜緊。也就道。旣承厚情。我们遵命。於是大家道了安置。各自去睡。鐵化同玉仙到了一間房內共寢。少不得脫得精光做一番生活。看那玉仙時已辭(醉)得動不得了。鐵化有心算計他。如何容他就睡。服了春藥。安心捉弄他一場。翻來覆去。弄個不歇。婊子被孤老接了來。可攔阻得他不弄。只得任他翻騰。直到四鼓方住。旣說玉仙有好量。如何衆人還好。他倒大醉起來。這就是鐵化的惡計。他是主人。如何自己行令打關。自斟自飮。他預先備下的兩樣酒。衆人吃的就是隨常的酒。那玉仙吃的是他特尋下十多年窨下的醇醪。吃着了爽口。玉仙所以不覺。後被熱茶一衝。那酒力發將上來。就有支撑不住。上床又被他一陣鼓搗。頭暈眼花。受不住了。雖忍住了不曾吐。却有醉得不知人事。鐵化有心。徹夜無眠。到了天明。把他一看。還昏昏的不醒。他昨日吩咐的那小廝。將他做的那假糞拿了來。你道是甚麼東西。是黃酒糟拿來揉得稀爛。搓成長條。從竹筒中楔出。儼然乾糞無二。他接過。輕輕揭開被。放他屁股底下。又將些抹在他糞門上。然後大叫起來道。不好了。小廝們快來。這丫頭撒了屎在被窩裡了。幾個家人跑進來。那玉仙已驚醒了。鐵化罵道。沒廉恥的臭娼根。如何把屎都撒在褥子上。玉仙吃了一驚。精光着身子。忙坐起來一看。果然兩三撅屎在那褥子上。糞門內還覺有些黏達達。也疑是自己醉了撒出的。那知是那鐵化弄的楦頭。急得只是哭。那時衆朋友聽見。都起來跑了出來看。大家鼓掌大笑。鐵化恐怕人看出假來。忙忙的喝那小廝道。髒巴巴的。還不拿了出去。那小廝拿了兩截蘆柴棒來。將那糞夾住。故意把鼻子捏得緊緊的拿出。鐵化吩咐家人道。快叫轎夫送他去到他家。對他老鴇子說。他撒了屎。汚了我的鋪蓋。饒了不要他賠。把我前日與他的東西都要了來。只許玉仙穿了衣服。也不容他梳洗。叫家人拉上了轎子。啼啼哭哭而去。三四個家人到了他家。把前話說了。那老鴇子見自己女兒装了醜。無辭可對。又怕聲名張出不好聽。只得把原物繳還。一口氣把女兒打了個半死。不題。鐵化請的這幾個人都是些惡少。玉仙昨日戲言。說了那一句頑話。他們都是回子。一棒打了幾個。那時雖然大笑。却蓄怒在心。今有這件因頭。四處一陣轟傳。玉仙睡着了會拉屎。這個美名一出。弄得鬼也沒得上門。他這樣促恰的事做得甚多。也不能盡述。不過姑舉數件。就可槪見他爲人的刻薄了。他家中有數萬之富。他的病症與竹思寬一樣。喜的是賭場中盆內六塊又紅又黑又金的骨頭。愛的是婦女們胯下兩片又尖又圓又扁的精肉。旣與竹思寬臭味相投。自然就道同契合。這日在屠家賭場上歇了局。大家小飮閒敍。且說這開賭局的姓屠的。雖然提了他多次。尚未說他的來歷。一筆不能雙寫。此時得空即補。這屠四他原是浙江杭州府錢塘縣人。在那西湖嘴子上住。與女敬德昌氏緊鄰。每日在湖中以戳鱉賣錢度日。昌氏家中無人買東買西。常煩屠四替他走動。昌氏無可酬謝。見他好一條壯健漢。一日煩他沽酒買肴。二人同飮。以當合巹。遂將腰州臍下褲襠縣裡那一件人又怕又愛的鐵牝奉敬。不意那屠四竟有一具好陽物。不但他人會戳鱉。此道更善戳蝦。昌氏雖不能稱心滿慾。然較之別人。一個可抵二三。着實心愛。叫他常常來家中走動。昌氏自遇那道士之後。被他採了兩夜陰精。傷竭得一場大病。幾乎害死。吃了許多補藥。保養了兩個月纔起得來。後來淫興雖略減了些。不過不能向日精壯。可以日夜不倦。但兩三個男子他也還不放在心上。倒是他的娘有年紀的人了。精枯血敗。被道士那兩下弄傷了。先還不覺。後因扶持女兒的病。起早睡晚。受了辛苦。及女兒病好時。他就病倒了。懨懨纏纏。總不能癒。因此他家中越發離屠四不得。遂向他道。你孤身一人。穿吃有限。況且這戳鱉一事也非正經買賣\footnote{是極。戳鱉不如當龜。賣鱉不如賣蝦。}。不如搬來我家同住。現成衣食。不過相幫走動。又沒費力的生活做。你心下如何。那屠四巴不能夠。不但日間有吃。而且夜間有弄。喜孜孜滿口應允。他係租的半間房子。退還原主。只數樣舊傢伙。幾件破衣服。頃刻就搬了過來。昌氏取些私囊。替他製了幾件衣帽鞋襪。裝束起來。倒也好條漢仗。他兩人也不待父母之命。亦不用媒妁之言。做了一對名色夫妻\footnote{夫妻而有名色之謂。奇談。}。日則同食。夜則同衾。或有嫖客到來。屠四日裡買買酒菜。夜間聽聽梆聲。若無人到。他就頂缺。這種人的官銜。南京叫做湯保。北京呼爲撈毛的。屠四就充了這行職役。過了些時。衆孤老知他是昌氏的假夫\footnote{前曰名色夫妻已奇。此曰假夫更奇。}。因叫他屠四。不好直呼其名。都稱他爲屠半八\footnote{半八有理。因係假夫。故只算得半個忘八也。}。他也欣欣然居之不疑。昌氏的娘臥病年餘死了。火化葬於湖中。起先昌氏娘女兩個做這販棒槌收水銀的買賣。人倒無甚閒言。今見他娘死了。這屠四公然在他家享用。有些無賴的少年就吃起醋來。一日向屠四說道。古人說。急風暴雨。不入孤兒寡婦之門\footnote{看這些無賴先以大義責之。}。你係他家鄰舍。旣非昌姓親戚。他又不曾明公正氣嫁你。你如何公然與他同住。霸占寡婦\footnote{次以罪名加之。}。這樣無主的美物。你受用得。我們也受用得\footnote{此方吐出心腹事。妙。}。你要同我們公用便罷。若不然。我們往縣中公舉。吿你一狀。叫你打官司。再不然。你趁早迴避了也可。你回去與女敬德商議。三日內沒有回信。你試試我們的手段。叫做前打後商量。屠四見人多勢重(衆)。回來把這些話向昌氏說了。要辭去。那昌氏恨道。我的命中偏生遇着這些小人駁雜。當日在城中是這樣。纔搬到這裡來。如今又是這樣\footnote{昔有一婦。嫁到夫家。即有外遇。屢嫁屢被出。一日偶與鄰婦閒談。知此婦數被休。勸道。一個婦道家。何苦只是這樣。此後自己檢點些罷。此婦道。這如何怪得我。我嫁着的就是忘八。叫我奈何。昌氏不知自責而責他人。與此婦心腸無異。}。心中捨他不得。說道。我同你過得好好的。你爲何要去。旣衆有閒言雜語。我們竟說是夫妻\footnote{竟說是三字愈出愈奇。自有夫妻二字以來。大約未有此奇稱也。}。暗暗的搬了。便沒是非。屠四道。旣承你過愛。我此地有(也)無存身之處。我有個叔叔在南京開賭場。無兒無女。屢屢帶信來叫我。我因無衣服盤纏。不能動身。如今除非投奔他去。不知你可肯離鄕遠出。昌氏道。我母親又沒有了。別無一個至親。眼前你就算親人。我此處有甚麼戀得\footnote{即有親戚。焉能如這個沾皮貼肉的實在。}。屠四道。雖然如此。只是沒有路費。奈何。昌氏道。我幾年來也還積趲了些。遂將歷來陰戶所掙之物。取出來與他看。約有百餘金。屠四喜得滿臉是笑。道。兩人有三四兩銀子就夠盤費了。別的留着到那裡做本錢。尋個生意做。又道。房子是租的。撂了就可走。但這些器皿傢伙。若要變賣。恐那些惡人知道了。攔阻起來。就走不脫了。昌氏道。幾件舊東西。所値幾何。也還差房主兩個月房租。留下。鎖了門。准了他罷\footnote{屠四戳鱉。是沒本錢的人。故其見小。昌氏賣蝦。隨身便有寶貨。故其見大。故兩人所見不同也。}。二人算計明白。將所有細軟都打了包。傍晚叫了一隻船來。搬上了行李。到了北新關。次日過了壩。雇了一隻滿江紅。由蘇州到丹陽出江。過鎭江金山。直抵南京石城橋泊下。屠四上岸去尋着了他的叔叔家。接了昌氏上岸。一同住下。昌氏此時說不得假夫的話。只得認眞的拜了叔公嬸婆。這屠四的叔叔開屠(賭)場久了。人起了他個美號。叫做人屠戶\footnote{又一個開賭局的。}。他家中來賭錢的着實熱鬧。日夜不斷。這人屠戶自幼好嫖。後來因開了賭場。銀錢來得容易。嫖得更甚。他前妻陶氏因丈夫好嫖。不同他親厚。他也就嫖起來了。家中但有來賭的人。他揀那鼻大身強的。無一不嫖。偶然嫖着一個知疼着熱快心蜜意姓強的朋友。他想五倫中只可盡得一倫。竟撇了那夫婦。與那朋友同生同死去了。人屠戶也吿過官。屢年未獲。他因內裡無人照料。有嫖厚了的一個婊子。說是姓通。也不知是眞是假\footnote{自然是通。焉有妓婦而不與人通者。雖姓通。通猶可也。}。他費了許多錢買了來家爲妻。不想一年之後。人屠戶得了一個下疳。竟將陽物蝕去。上面還是鬚眉男子。下面竟無男子之具了。正是。

\begin{quotation}

孰意腰中小和尚。化爲烏有一先生。

\end{quotation}

這通氏纔三十多歲。酷喜的是人胯中那小和尚同他通一通。那人屠戶把根通條沒有了。他家夜間人來賭博。人屠戶守定抽頭。傍邊有看的閒人。通氏就暗約到房中。請那小和尚到他那紅門裡去小酌。硬幫幫的進去。定要吃得那小和尚撒酒瘋。撞頭搕(磕)腦。吐得猥頭搭腦軟叮噹。纔肯放出\footnote{這和尚與通氏纔是眞正通家。}。如此多次。人屠戶也有些知覺。他大雅得很。毫不介意。通氏大發慈心。正要學西遊記上的寇員外。想齋萬僧。數年來尚未及百。突然屠四兩口子到來。東西屋住着。甚是礙眼。整熬了數日。過不得了。一晚。悄悄的約了一個舊朋友進來。在床上小敍濶悰。不想那人進來時已被昌氏瞥見。這昌氏是一夜也不能離此道的。前水路來十多日有屠四相伴。他因感恩盡力。也還將就過了。到了此處。屠四夜間又去幫叔叔。竟川中犬百姓眼起來。多年未慣。甚是難過。雖要學戰國四君去延攬三千食客。一來新到。不知誰可做主顧。二來嬸婆咫尺。不好意思。今忽見了這事。暗喜贊道。原來嬸婆也與我同類。是個招賢納士的女英雄。須衝破了。大家好做事。遂悄悄的到窗下來聽。正在響動\footnote{當年張他娘乃見其形。今日聽嬸婆只聞其聲。前後遙遙一對。}。他回房點了一枝蠟燭。輕輕走來。將門一推。隨手而開。忙進去把帳子一掀。見他二人正在綢繆。通氏同那人見了。吃了一驚。那人忙拔出。要下床跑。昌氏笑嘻嘻的一手拉住。道。你這麼個小膽子。就敢來偷野食吃。我來看你們怎麼個弄法。你怕的是甚麼。可有個女人來捉奸的。通氏同那人見他如此說。都放了心。那人知他是就敎的意思。上前抱住。親了個嘴。伸手就去摸他下身。只着單裙。不曾穿褲。把燈接過\footnote{此等小末處亦不漏去一筆。眞細心。}。放在桌上。將昌氏抱到一張椅子上仰着。掀開裙子。弄將起來。輕輕一送。便沒至根。纔抽了幾下。昌氏用手推住。道。不濟事。你還同奶奶弄去罷。我不稀罕這樣東西。打水不渾的。那人一團高興。被這一掃。拔又不好拔出。抽又不好再抽。被昌氏雙手推開。站起笑道。旣做這樣的事。也尋個像樣些的來頑頑。這有名無實的物件。要他做甚麼。仍拿着燈出去了。那人好生沒趣。勉強同通氏弄了一陣而去\footnote{此非寫那人之不濟。通氏之不擇美惡。正寫昌氏淫心猶勝通氏之宿妓也。}。次早。通氏笑向昌氏道。昨晚那人也將就用得過了。你爲何那樣貶他。叫他甚是沒趣。你不曾試着他的本事。他有半更天的好熬手呢。昌氏微笑道。不瞞奶奶說。這件東西我正正經經見過了些。像他那樣的。只好備數。要緊處用他不着。不要講別人。就是你姪兒。也比他強多哩。他有一二更的工夫。還不在我心上。通氏又笑道。這樣看起來。你是個多見廣識的了。也不瞞你。這物件我也經過了些。覺得都大同小異。沒有見過那個放樣的。只有一個人的此道又太放樣了些。我也曾約他來試過。用兩手圍着道。有如此粗。又比着道。有這長。我同他弄了半夜。唾沫用了有兩鍾。費力(了)多少力氣。只弄進了一個頭子去。把我的幾乎裂開。睡了兩日。纔起得來。還腫疼了好幾日。再他不敢惹他了。你若是有大量。我約他來同你試一試。你見了不要害怕。昌氏聽了。渾身慾火直冒。笑道。我們生了這件東西來。就是隨身的利刄。世上男子好漢不知死了多少在這裡。可有反怕他的道理。這人在那裡。奶奶你約了他來。看我怕不怕。通氏道。就是每常在我家住的老竹。他是有名的賽敖曹。說他總沒有遇過對子。只有一個老鴇可以同他弄得。我先聽得這話。心裡也不信。人身都是父母的遺體。男人的縱大也不過略大些罷了。況且我們這東西也不過是一樣。也不過稍有寬緊。一個旣受得。個個都受得。誰知約了他來。竟是一個大棒槌。嚇得我要不得。心裡雖怕。眼見稀奇物。却也愛他得很。二來又不肯折了我們女將的銳氣。況且他旣來了。怎肯空去。只得仗着膽子同他試一試。誰知這東西只好看而已。是用不得的。白吃了一場虧。你旣說不怕。只好夜間私約他來。此時大靑天白日。不怕姪兒來撞見麼。昌氏道。他管不得我。奶奶你只管去約了他來。通氏也着實高興。要看看他二人可果然弄得。就走了去。那竹思寬是日夜在他家的。頃刻便同着進來。通氏已悄悄將昌氏的話向他說了。竹思寬喜不自勝。忙同通氏到昌氏房中。深深一揖。通氏笑道。就是他。你二〔人〕請試。我是要觀陣的。竹思寬將昌氏抱到床上。就去脫褲。昌氏毫不推辭。任他脫了。竹思寬也褪了褲子。昌氏一眼看見他那異物。心中暗喜道。這眞是生平見所未見了。有西江月贊他道。

\begin{quotation}

偉長足有一尺。粗圓將及雙圍。頭如剝兔紫巍巍。柄上蚓筋幡綴。乍看渾疑桌腿。端詳果勝擂槌。敖曹大號不虛推。喜得淫心如醉。

\end{quotation}

竹思寬將他兩腿分開。見他牝戶大張。如鍾子口一般\footnote{眞正可怕。}。也與別的婦人頗異。也有一個西江月贊他的道。

\begin{quotation}

開閃寬皮兩片。中間一個紅門。猶如鼠洞一般深。定是曾經大陣。牝蓋豐盈滿滿。毳毛漆黑森森。看他窈窕一佳人。動人情處却恁\footnote{二字貶極。}。

\end{quotation}

竹思寬見昌氏的陰門雖然寬大。但因自己孽具太大。不敢冒失。也還用了些唾津。對着一頂。輕輕就將龜頭送進。知道是一員猛將。較郝氏猶雄。一連幾下。送到了根。通氏把牙咬了幾咬。倒替他打了幾個寒噤\footnote{俗謂聽彈詞掉眼淚。替古人擔憂。通氏之謂也。}。暗暗吐舌。昌氏覺竹思寬之物比那道士粗雖有限。却長了寸餘。頂在極深處。甚有妙境。那竹思寬見是一盤對手棋子。却放鬆不得的了。盡力搗將起來。那昌氏淫聲艷語。腿顫股迎。騷態百出。甚是難看。通氏賞鑒了一會。面似火燒。陰如水浸。忙走出來。恰好屠四進來。不知他尋甚麼。通氏正在難過時候。想起方纔昌氏誇他技勇。叫他到房中。一把摟住。親了幾個嘴。道。我聽得你娘子說你腰間有個好本錢。我同你試試看。屠四道。這怎行得。怕叔叔來看見怎處。通氏急了。你快同我弄弄就罷了。不然我就叫喊起來。那屠四禽獸一般的人。知道甚麼叫做倫理。見通氏纔三十多歲。也生得風騷可喜。就一同上床脫褲。屠四見他的陰戶雖與昌氏的形狀相似。那門洞却緊密了許多。通氏見他的陽物昻昻然。果覺可觀。較之昨夜那人大了半倍。而且較生平所遇之具尚還出色。一個初逢小陰。一個乍遇大陽。自然快樂無比。通氏被屠四弄得丢了二度。心愛不過。摟住不放。屠四道。我進來有事。外邊等着我呢。放我去罷。改日有空。我同你大大的盡一盡興就是了。通氏只得放他起去。屠四穿衣出來。聽得昌氏聲息異常。響聲大震。忙在窗外向內一張。見他二人正在賣解。忙避開了。通氏揩了陰戶。穿了褲子。又走了過來。見他二人還在弄呢。那竹思寬已被昌氏弄洩了二次。奈他緊緊摟住不放鬆。竹思寬只得掙着還抽抽扯扯。怎奈那個陽物漸漸軟了。昌氏覺內中沒趣。纔放了他。道。你這樣個好東西。可惜不長久。若再有通宵的本事。眞是天下無兩了。即如一個赳赳大漢。一點勇力也沒有\footnote{此等甚多。}。一個翩翩少年。一毫文墨也沒有\footnote{此類更多。昌氏却不知彼等偏能享福也。}。空自好看。濟得甚事。各穿衣下床。昌氏在一個匣中取出一個小瓶。倒出兩丸藥來。遞與竹思寬。道。當日是個人送我的。屢試屢驗\footnote{不知可是那道士送的。}。只剩得兩丸。你晚上用燒酒服一丸。那一丸也用燒酒硏開。擦在陽物上。我同你夜間做一個整工夫\footnote{夜間這一整工夫。也不知工價多少。}。試試我的本事。竹思寬笑吟吟接着出去了。通氏笑道。你果然好手段。我看不但你不怕他。他還有些怕你呢。兩人笑了一陣。通氏出去。昌氏自從經那道士到今。算第二次爽快了。上床養神。安排夜戰。晚間衆人在外邊賭錢。竹思寬吃了藥。又擦了藥。不多時。覺陽物發脹。溜了進去。那昌氏已經在那床上脫光等候。竹思寬忙脫了衣服上床去。就弄起來。通氏聽得響動。又走來。坐在床沿上。燈光下細看了一會。按納不住。忙叫了屠四進來。同他着着實實弄了一場。然後纔睡。那竹思寬趁着藥力。或疾或徐。或深或淺。弄個不休。乏了。定一會又弄。弄了又歇。直到五鼓。那昌氏也不知丢了多少回數。雖覺得精神倦怠。四肢酸軟。但他陽物在內中熱硬有趣。況只此兩丸藥了。後來欲求此樂境料不能得。那裡肯捨。竹思寬見天色將明。圖解藥力。更奮勇長驅。一陣亂搗。正然弄時。只見昌氏手癱脚軟。聲息皆無。眼睛緊閉。像昏迷的樣子。忙用手摸他口鼻。只微有溫氣。嚇得連忙拔出。嘴對嘴度了一會。纔漸漸醒來。問他道。你怎麼來。昌氏道。我不怎麼的。方纔只覺得心窩裡一快活。渾身一麻。就不知道了。竹思寬道。這是你一夜精脈去多了的緣故。養息養息罷。我這藥力不得過怎麼處。昌氏覺得再弄不得了。說道。你喝些涼茶。再把下身用溫水洗洗。弄洩過就好了。竹思寬見昌氏這個樣子。不敢再弄。忙別了。到郝氏家來。此時郝氏尚未起床。他忙喝了些涼水。洗了洗下身。同那郝氏撥戰了一場。方纔洩了。郝氏覺他比每常分外粗硬。脹熱有趣。問他緣故。他不肯說昌氏的話。只說偶然得了一粒金丹。特來奉承他的。郝氏也就信了。更感愛他了不得。那昌氏只圖快樂。不想這一夜精脈流枯。他睡了一會。覺身子底下黏齏齏的難過。只得掙了起了。看那褥子濕了半截。連他兩股腰間都是陰精浸濕。揩淨了。換了床褥子。然後又睡下。通氏梳洗了。過來看他。見他還睡着。說道。外邊早飯時了。你還睡呢。昌氏道。我身子懶得動。通氏笑道。你兩個這一夜也不知怎樣弄。大約是弄癱了。一個可口的美物。吃飽了就罷。何苦定要吃傷了。昌氏也微微的笑笑。在通氏只說他一時乏倦。就是昌氏也以爲過兩日定然就好。孰不知他被道士弄傷了的。那時因身子壯。故逃得性命。今日舊病復返。自然難支。漸漸飮食不進。渾身打骨縫裡邊發熱。五心煩躁。日漸黃瘦。每夜還央通氏約竹思寬來弄上一度。他也無力動了。只如死人一般仰臥。憑他抽拽而已。竹思寬同通氏勸他暫歇幾日。將養身子要緊。他道。我自幼到今。恨無敵手。今得遇此。一死何恨。我當年曾說牡丹花下死。做鬼也風流。今果應其言了。所恨者相遇未久。若同他相聚一年。就死也無遺恨了。我今已病入膏肓。古語兩句說的好。

\begin{quotation}

臨崖勒馬收韁晚。船到江心補漏遲。

\end{quotation}

我如今忙忙的日夜行樂。猶恐無及。你如何還說止歇的話。二人勸他不醒。惟嘆息而已。屠四延醫調治。服藥無效。捱至月餘。僅存皮骨。臨危時還約竹思寬來。將他陽物撫摩了一會。長嘆了兩聲。落了幾點淚。竹思寬也甚傷心。掩面而出。到了半夜。氣絕而亡\footnote{不圖爲樂一至於此。}。只得二十四歲。此亦貪淫不節之報也。正是。

\begin{quotation}

浪魄不知歸何處。淫魂今夜落何方。

\end{quotation}

屠四感激昌氏提攜之情。不但陪他白睡了許久。還遺下若干之物。也哭了兩場。買棺殯葬。延僧超度。都還熱鬧。自昌氏死後。通氏將姪兒做了副夫。屠四在當日也想盡力以報昌氏。無奈窮主人請了大肚漢的客。再不能使他飽足。此雖竭力鋪排。彼並不見感謝。今遇通氏。見他還易於打發。只仗着本事。儘力可供他飽足。他二人恩愛得了不得。只瞞着人屠戶一個。通氏雖然好淫。竟還知足。自從有了屠四。把外邊向日的舊主顧一槪謝絕。不去招攬\footnote{此非寫通氏知足。正反襯昌氏貪淫不堪耳。}。人屠戶見妻子忽然貞節起來。暗暗稱奇\footnote{貞節之上有忽然兩字。眞奇聞。}。那知他寵幸了可心可口的愛姪\footnote{可心二字是矣。可口大約是下口。此事與上口無涉也。}。過了年餘。通氏忽生一子。人屠戶方纔大異。究問其從何而來。通氏還(道)。是你當日好的時候我受得孕。人屠戶道。我已病廢了這幾年。那裡有懷七八十個月的道理。通氏只是笑說道。你有了兒子就罷了。管這些閒事怎麼\footnote{奇談。只論兒子之有無。不必問其所從來。千古未聞之奇語。丈夫問奸生子之來歷。而通氏謂之管閒事。此更奇。}。人屠戶也料到是姪兒之種。也還是他屠家的骨血。就葫蘆提認了\footnote{人屠戶以姪兒之種是他屠家骨血。便認爲己子。尤奇。眞是一對奇夫婦。}。誰知這孩子不妨眞父而妨假父。不尅親父而尅叔祖。甫及一週。人屠戶疳瘡大發而死。通氏屠四口內乾嚎。心中暗喜。忙殯送了。他們在人前還假爲嬸姪。到內中儼然夫妻。一個語語要做節婦。一個聲聲要做義夫。一到晚來。上床之後。節婦義夫合成一體。雖係通氏之無恥。屠四之滅倫。亦由人屠戶開賭。一生不知陷害了人家多少好子弟。一妻同朋友而逃。一妻爲姪兒所據。身死嗣絕。也就可以報應。凡以賭局誘人者。急改弦易轍。切勿蹈此\footnote{看此書。但到此等冷語處。細味之。}。屠四接了叔叔衣鉢。他又有昌氏所遺之物。揀有好主兒放頭接賭。比他叔叔當日更覺興旺。來者越多。屠四鑒通氏昔日之事。恐曠了他。又去齋僧布施起來。每夜偷空必進房幹訖一度。方纔出來照料。這日。竹思寬同鐵化衆人都在局上歇了。飮酒中間。正說閒話。鐵化偶然道。偌大一個京城。就沒一個絕色的妓女。眞也可笑。竹思寬正有郝氏所托之事在心。遂答道。怎麼沒有。那十分才美的佳人。他要高擡他的身價。怎肯做那毛遂自薦的事。所以人知道的少。鐵化見他說話有因。遂問道。兄是此道中的老在行。必定知道誰家有好女兒。竹思寬道。只這眼面前錢家的女兒就是個絕色才女。大爺如何忘了。鐵化道。小時我常見來。果然生得好。後來說他雙眼瞎了。如此無心想到他。有三年來沒見。雖然他模樣生得標致。但沒了眼睛。也就算不得十全的美人了。竹思寬極力打合道。大爺是此道中老見家。這一句話又來得外行了。請看那畫上的楊妃春睡圖。他不是閉着睛睛的麼。相傳以爲妙事。果然是絕色佳人。何在那眼睛之有無。還有一句笑話。到了那高興的時候。有眼睛的還要閉着呢。大爺若果然相與了他。還有多少人贊揚。鐵化道。這是甚麼緣故。竹思寬道。假如如今大爺出一股大錢梳籠了他。人知道了。定然誇說大爺是個多情種子。識貨的奇人。錢貴雖少雙眸。單重他才貌。取人於牝牡驪黃之外。肯費若許大錢。偌大京城。有多少風流子弟沒他的眼力。被他奪去頭籌。再被這些妓女們聽見了。人人欽仰。在䘕衏中着脚一場。做一個風流魁首。也不枉了。不瞞大爺說。一來我年紀多了。二來我手內無錢。我要比得上大爺府上百分之一。我也早奪了這趣了。鐵化聽他說得天花亂墜。也動了心。便道。我們幾時閒了去看一看。再做商議。竹思寬道。大爺尊意差了。不做此事則已。旣有此興。定要占在人先。況佳人難得。雖然他母親韞櫝而藏。待價而沽。但他的靑春也是緩不得時候了。難道他的美名只我一人知道不成。別人倘然知道。有好風流美名的。先去採了鮮花。大爺這樣福人。是吃殘湯剩水的麼。鐵化被他奉承得快活。甚覺動火。笑道。旣然如此。我們此時乘興。何不就去。竹思寬道。古人說。輕人輕己。大爺要去相看這絕色佳人。不備分厚禮去打動他。覺得不是行家了。況他母親少年時。大爺知道也是見過大世面的人。我們卒然走去。闖起寡門來。豈不落他背地譏誚。鐵化道。據兄說。當如何行事。請見敎一番。竹思寬道。大爺果然有此興。今日送一個大大的東道封兒去。就說大爺慕他的令愛。要一親色笑。叫他家預備酒席。明日再送一分厚禮做見面錢。然後大爺駕去。他門戶人家是識竅的。見大爺如此舉動。自然百般趨奉。何等光彩。鐵化道。兄說得有理。就煩兄去做個月老。叫過小廝來。將帶來賭本取出一封。稱了二十兩。遞與竹思寬。道。煩兄今日送了去。叫他整理下東道。我回家備了禮物。明日親往。兄於明日在他家等着我。若果中了意。就煩兄說合。我自有厚謝。竹思寬道。我承大爺相愛。多年契厚。何敢當謝字。總成大爺一個風流榜首。我也叨得餘光了。說定。大家散去。竹思寬見事體有幾分妥意。他心中暗喜道。他女兒的事若成就了。他母親的這件妙物我便可以長久受用了。遂忙忙走到錢家。向郝氏就把怎樣打動鐵化。怎樣起發他東西的說獻了功。將銀子遞與他。道。這是辦東道的。他明日還有厚禮來。若造化事成了呢。是你的一炷大財香。就不成。且白得他這一分厚禮。郝氏歡喜得了不得。就忙設佳肴美酒謝了媒人。就留他同宿。然後將他巨陰中的淫水着實澆了一澆梅根。正是。

\begin{quotation}

令愛未曾試新。乃堂且來溫舊。

\end{quotation}

且說這鐵化。他承祖父做的那氈貨生意。夥計們專走北京。也有兩萬本錢。本京城中又還開着幾個大氈貨鋪。他只十八歲上父母相繼亡後。止有他一個大胖的妹子以外。別無兄弟姐妹。娶的那賢妻火氏。生得有五七分姿色。倒有八九分風騷。論起來。那樣一個俏人兒。就該性格溫柔了。誰知人再不可皮相。這婦人淫而且悍。降伏那丈夫的手段。比降龍伏虎的羅漢還利害幾分。鐵化初娶來時。愛他美麗。凡事順他的性兒。後來縱慣了。就有些動手動脚的起來。鐵化順慣了他。一時翻不轉來。弄成了一個情怕。何爲情怕。起先娶他來時。因十分愛他。百樣事不忍拗他一拗。且每夜上床之後。定要做一番生活纔睡。請敎。這件佳品雖然味好。只當得點心偶然吃些的。可是當得家常茶飯的。日日離不得的東西。他雖然姓鐵。身子與陽具却不是鐵的。如何夜夜來得。久而久之。未免就要操三歇五的了。先因鐵化愛他的很。又是新鮮美味。自己做慣了例。上床之後。必定把功課完了。方纔睡覺。火氏也道是例當如此。況乍嘗着個中滋味。如何肯歇。忽然見他怠惰起來。就如那小學生上學定要背書寫字。完他這一日的事。方纔放館。忽然不待先生吩咐。竟公然自己逃起學來。如何使得。但這鐵化幼喪父母。無人拘管。自小在賭場妓館中着脚。這是他的事業。初因戀燕爾新婚。寸步不離。過了些時。新鮮妙物吃了多次。也有些厭了。身子也拘束得久了。終日只想往外邊溫溫舊業。那火氏正同他打得火熱。忽然見他朝出而不歸。覺得冷冷淸淸。寂寞之甚。雖有一個小姑。生得又醜又惡。因幼無父母。無人敎訓。鐵化自己還少一個人管他。如何能管他的妹子。養得他這個性子。眞像嫂子娘家的姓。竟是一個火\footnote{趣。}。一日打了丫頭罵僕婦。惡狠狠的。雖纔十七八歲。長成胖大無比的一個身軀。他也不理這個嫂子。故此火氏也不去親近他。這火氏獨自坐在房中。無可消遣。到晚鐵化回來。他定嘓噥抱怨個不住。鐵化因橫了一個愛字在胸中。見他生氣。晚間少不得替他消氣。鞠躬盡瘁的陪個禮。但這個氣如何有本事夜夜替他消得。又過了些時。竟像窮百姓躱差一般。逃在外邊。做了個夜出而不歸了。這火氏旣姓了火。他一身到底竟無處不是火。孰意胯下那穴道中。其火更甚\footnote{丈夫姓鐵。陽物却不是鐵。妻子姓火。陰內全然是火。笑倒。}。日間火往上升。還可以消得下去。到夜間忽然獨守孤闈起來。火往下行。把一個救火的水炮又不在眼前。如何過得。一夜槌床搗枕。咬牙切齒的氣恨。等得鐵化回來。先時還哭哭罵罵。後漸抓抓打打起來。鐵化本還要替他陪陪禮。消消氣。無奈力量不加。知道這件事是無可挽回的。只得聽之而已。先只是愛之一字。到如今愛中又生出怕來。所以說是情怕。那火氏先也還想施施威。等他好來陪罪的意思。那知他自知罪惡深重。將至隕滅。陪不來了。任他處治\footnote{絕倒。}。竟不來修飾。火氏見他如此。焉得不急。急中生怒。火氣直騰。與鐵化竟像有不共戴天之仇的一般\footnote{此等趣語。令人笑殺。}。見了面就罵。罵上氣來就咬上幾口。向鐵化臉上亂蒯。那鐵化見了他。竟合了他夫妻二人的貴姓。又合了自己的尊名。鐵見了火。自然會銷化起來。竟怕他如母夜叉一般。日夜躱在外邊。輕易不敢見他尊面\footnote{他夫妻兩姓。妙喩。妙譬。}。但火氏是個淫物。又有吃有穿。無所事事。自然就飽暖思人肉了\footnote{人肉二字乍見。}。上面這張橫嘴。珍饈百味。要吃就有。下邊這張直嘴。想一點粗糲之食充充飢也不能得。熬得他日夜淸水也不知淌了多少。總有要打隻野雞吃。救救饞的意思\footnote{兀的不笑殺人也麼哥。}。但他家雖非仕宦門第。也是個財主人家。深房大屋。閒人誰能到得裡邊。不但想吃野雞肉沒有。連想根野雞毛看看也不能夠\footnote{苦惱。若不見雞。只見雞毛。更覺難過。}。他一日心中躁急。又是那困倦。打算要去睡睡。欲睡又先愁不穩。走到廊檐下靠着欄杆。正在怨恨。只見二個小哈巴狗兒在那裡高興。那隻雄狗伸着大長的舌頭。替那母狗舔陰門。母狗翹着尾巴任他舔咶。動也不動。舔了一會。爬上去聳了幾聳。不多時跳了下來。兩個已黏在一處。竟成了一個身子。八隻脚。兩頭狗了。他看到此處。

\begin{quotation}

上面的火一陣陣燒將起來。熱得他臉皮通紅。眼睛中火星亂爆。

下邊的水一股股流將出去。淋得他兩腿皆濕。陰門內熱癢難抓。

\end{quotation}

不由得怨氣沖天。切齒恨道。可以人而不如母狗乎\footnote{昔有念別字一先生死去。陰間冥司謂。誤人子弟。着他去變狗罷。他求道。變狗不敢辭。願變個母狗。問他何故。答道。臨財母狗得。臨難母狗免。所以願變母狗耳。可見人而不如母狗多矣。}。忽想起。方纔見那雄狗舔得母狗的陰門。看得那光景。似乎也有樂境。我何不試他一試。想了想。有了主意。又等了一會。那兩隻狗已分開了。將那雄狗喚着。那狗是主母每日吃飯他在傍邊分惠慣了的。一呼即來。他喚着。走到樓梯跟前。吩咐丫頭。我要睡午覺。怕人吵鬧。將樓門關着。不許擅開。非呼喚不許上來。丫頭豈敢不遵。說了。他上樓梯。低聲喚着。那狗竟跟着他。一蹬一蹬跳了上去。丫頭們將門帶上。他到了上面。這進樓一連五間。下邊東兩間是他的臥房。西兩間是小姑的臥房。當中一間堂屋。樓上隔做三明兩暗。儘東兩間三面皆是窗。是他收拾了午睡之所。床帳桌椅。香爐骨董。花揷書燈。癢槌孝順。筆筒硯臺。種種俱備\footnote{詳敍擺設若許之物。只有床椅書燈癢槌筆後來用着。其餘皆是陪寫。泛然看到此處。不過謂敍事而已。孰不知竟有要用之物。眞令人莫測。}。他將狗喚到房中。將門關好了\footnote{下邊樓門丫頭已帶上。此又云關了門者。非怕人來。怕狗跳去耳。此等細心處。須看得出。方算會看書。}。外衣寬下。裙褲脫光。一把將狗抱在懷中\footnote{此狗何幸而得此。}。上床來\footnote{床。}。仰臥着。兩腿揸開。將狗放在胯下。把狗嘴對陰門。那狗雖常見過母狗的陰戶。却與款式大不相同。並不認得此是何物\footnote{昔有一人。以販賣骨董爲生。因要遠出。値母臥病。囑其妻曰。我若歸遲。恐母親病故。亦須留一件貼體的東西。等我來家一看。如見母音容。此人去後。其母即故。其妻想道。他再三囑托。叫我留一物。不知何者是貼體之物。因想婆婆之陰。乃生他之門。方爲貼體。以刀剜下收貯。其夫歸。以此付之。夫見一乾圈。不知何物。以問妻。妻笑曰。連你娘的屄都不認得。還在外邊看甚麼骨董。識骨董者尚不識得娘的屄。又何況此狗而能識火氏之陰戶乎。}。見主母如此舉動。疑是餵他東西。也用鼻子聞聞。旣無葷味。又無他物可食。只一條縫兒。水漓漓的。不知何故。只道是哄他來頑耍。掙着撲的一下跳下床來。火氏把他又抱上來。他又跳下去。如此數次。急得火氏那慾火。打遍身毛孔中都冒了出來。正在沒法。忽然看見那個書燈\footnote{燈臺。}。想道。狗愛舔的是油。何不搽些油。或者聞得香氣。肯舔也不可知。起身把燈盞中油蘸了些\footnote{油。好悟性。}。搽在陰門兩邊\footnote{世之罵人曰。油嘴光棍。火氏可謂油屄的淫婦。}。復將狗抱上床來。如前作用。果然此番那狗不像先那樣死板了。聞着了香油氣味。便伸出舌頭舔將起來。但有油處無不舔到。原來這狗的舌頭又熱又糙。舔得癢酥酥。無比受用。雖然外邊有趣。裡面不曾嘗得是何滋味。又想了一想。還是以前的這個題目。只是文章又深一層。復起身將一枝新筆\footnote{筆。}。醮着油。送入牝中一攪。蘸了數次。攪了幾回\footnote{自有筆以來。其至貴者則吾夫子之春秋筆。其次則董狐之史筆。朱衣之點額筆。江淹之生花筆。孔循所獻之畫日筆。相如題橋之筆。班超所投之筆。蕭曹之刀筆。以至如椽之筆。無私鐵筆。種種不一。於閨房之私。則有張敞之畫眉筆。爲千古韻事。不意火氏有此一枝蘸油筆。同一筆也。何此筆之不幸也若此。}。又上床來臥下。這狗先將外邊舔淨了。聞得裡面還有香氣。將舌頭伸入去舔。越舔裡面還有。又伸長些。惟獨狗舌最長。這狗雖小。他舌頭竟有五寸餘長。伸在內中絞着亂舔。這樣又長又熱又糙又活的一件東西\footnote{古有四其御史。此有四又狗舌。可稱的對。}。在裡面活動起來。你道他快活不快活。將這婦人舔得骨軟筋酥。陰精一陣陣流將出來。那狗雖將油舔完了。後有些黏黏涎涎的東西流個不住。又有些腥味。他還當是主母用鮝魚湯和的稀糨糊餵他的\footnote{異想奇譬。}。越發舔得高興。越舔越有。越有越舔。這火氏眞生平未逢之樂境。直舔得他丢了數次。遍體酥麻。火氣盡洩。興足而止。有四句打油說那火氏道。

\begin{quotation}

人畜相投趣味眞。不膠不漆自親親。

一團春色融懷抱。妙舌強多躱懶人。

\end{quotation}

然後起來。那狗心猶未足。以爲主母捨不得與他吃了。還搖着尾巴亂跳。有個親益的意思\footnote{火氏當曰。吾倦。一笑。}。火氏穿了衣褲。重復睡下。暗想道。我若早知有此妙事。稀罕那忘八做甚麼\footnote{而今而後。取狗而捨忘八矣。}。同他弄時。我正興濃。他已吿乏。十次中倒有四五次不得像意。今日這一番。我興已闌。他舔猶未足。況那陽物在裡邊只直進直出。四面尚有空〖阝少日小〗。這舌頭亂絞亂舔。無微不到。勝似他〈他〉的百分。深悔早不悟到此處。癡癡空守着這懶惰的忘八\footnote{鐵化與狗。竟百不及一。可憐。}。不覺酥酥睡去。一覺醒來。睜眼一看。那隻狗蹲在傍邊。還有個候舔之意\footnote{此狗的職銜。可稱陰門候舔。}。火氏笑了笑。下床開門。喚着他跟了下來。自此以後。但是興動。就上樓去假睡。那狗自嘗過這甜頭。也不用喚了。但見主母上樓。他就跟着往前飛跑。這丫頭們見了。以爲是主母恩養餵慣了他。所以跟了去做伴。那裡知其中有這些奧妙。後來舔熟了。連油都不消用得。他一聞得那一種鮝魚香。舔得好不興頭。夜間丫頭們在房中伴宿。雖不好喚他上床。但日間不拘度次。乘興即來。興盡方止。即如那吃飯的一般。日間飽足了。夜裡也就不覺得餓。再說這鐵化雖然怕他。輕易不敢相親。沒有個永不見面之理。偶然進來。他見了就像寃家。非罵即嚷。當日尚圖他來夜間陪罪。還留三分情義與他。如今有了這根強似他的數倍的妙舌。越發不留一絲的好氣。那鐵化那知內中就裡。還說躱得久了。叫他守了活寡。自然氣忿。自己過意不去。間或夜間來陪他睡。着意溫存。就是陪罪。也必定要強而後可。雖竭盡心力。他總不如意。再不能討得一毫喜歡。還有半夜裡打嚷一番攆了出來的時候。弄得鐵化後來成半年連房裡也不敢進來。且說他妹子自幼許了童百萬做妻子。他生性已自憊賴。又看了嫂子降服哥哥的這番法術。以爲天下人的丈夫都該妻子如此管敎的。他學了個滿腹經綸。巴不得嫁了丈夫試試手段。他哥哥見他大了。正値童百萬家要來娶。盛備了數千金粧奩。買了六個丫頭\footnote{記着六個丫頭。}。幾房男婦作媵嫁到童家去了。再說鐵化見妻子這像(樣)性格。不容他近身。以爲妻子賭氣。做有夫的節婦罷了。我如何做得這有妻的義夫\footnote{有夫的節婦。有妻的義夫。千古奇談。的對。}。每日出去。非嫖即賭。耳邊無人吵鬧。倒也甚覺遂心。只他這種人。心是無主的。這個嫖得兩三夜。厭了又換那個。那個嫖幾夜。厭了又想去換。雖說是棄舊憐新。請想他妻子生得如此風騷美麗。又是經他開闢的妙牝。弄厭了還想去尋野食。何況這些顏色平常的妓女。又是宏敞的陰門。今日聽見竹思寬說起這錢貴來。十來歲時。他見了就愛。那時尚小。故不經心。後來聽說眼睛壞了。就不在意。今聽得如此標致。焉不動心。當日回家。買了幾疋紬緞。換了數件首飾。準備次日到錢貴家來相看。不知梳籠成了不曾。且聽下文。便知詳細。正是。

\begin{quotation}

欲知好事能成否。但把來因仔細看。

\end{quotation}

姑妄言二卷終



\endnotetext[1]{「回」原作「卷」,據全書體例改。}

\endnotetext[2]{此段原有眉批「天理倫常」四字。}

\setcounter{footnote}{0}

\theendnotes

\part*{姑妄言第三卷}
\addcontentsline{toc}{part}{姑妄言第三卷}
\markboth{姑妄言第三卷}{姑妄言第三卷}

鈍翁曰。鐵化梳籠錢貴。不幸失身於此狂且。正是爲其抱屈處。非寫鐵化之幸得錢貴也。

寫童自大之呆。自始至終竟未能改。非謂呆人能做財主。正寫財能呆人。可發嘆耳。何以言之。余常見擁巨萬之資者。猶晝夜持籌盤算。眉未刻舒。非呆而何。揆其意。不過爲兒孫做馬牛耳。獨不憶古人云。兒孫強如我。要錢做甚麼。兒孫不如我。要錢做甚麼。聚斂不已。非呆而何。百年駒〖阝少日小〗。終日營營。嗚呼老矣。死去一文帶不得。貪之何益。非呆而何。唐詩云。昨過老人宅。不解老人心。何事殘陽裡。栽松欲待陰。此意雙關。寫盡自不知死之將至。猶爲後人算計也。此詩可爲呆財主做一喝棒。正見童自大之呆。乃財主之常。不足笑也。

世間婦人醜者或有不悍。悍而醜再未有不淫者。鐵氏便是樣子。

仙桃木也。鐵氏金也。木遇金必傷。寫鐵氏凶暴若此。而仙桃相隨數載。竟未受其摧殘。乃仙木非凡木矣。仙木豈可久在臭銅之室而鄰金鐵

之險。必移根別植。庶有榮茂之期。故歸錢貴。得侍鍾生。旣貴之。又得鍾情之人而愛惜之。自能結實。故隨鍾生而生子也。仙桃雖得好處。但錢於金。鍾於金。始終爲金所制。故只能爲之小星。此等處。心不如髪。如何看得出。

葵心蓮瓣。此二物即鐵氏下體之形。豈可須臾離者。故獨留此二婢也。用一童自大引出魏如虎。魏如豹。巨金知縣。許多怕婆人來。不過謂陰道漸長。陽道漸消。女帥之威風日熾。弱男子甘拜下風。寫得世情可笑。當補在怕婆經之後。

夾入杜小英一段。正顯錢貴身辱煙花不得已之苦心。看他聽代目念詩後之言便知。

因錢貴引出祁辛。雖與正文無涉。正見錢貴之慧心。不爲富貴所惑。高出庸流萬萬。又借之以警戒少年。不可薄棄妻妾。私淫他人之婦。不但送去性命。其妻妾即歸所淫婦人之夫。報應分明。孰苦孰樂。人皆能作如是觀。淫之一字可化爲烏有矣。此一段於不可少。

何幸之葵花者。不過因其愛日取意焉耳。

極力寫火氏之淫者。一以見鐵化交不擇人之愚。一以見竹思寬無良奸淫之惡。世上竹思寬之流不少。明眼者當避而遠之。勿蹈鐵化之愚。錢爲命信手拈來。隨筆結去。讓出郝氏。後來好贅竹思寬。乃行文之省法。

\chapter*{姑妄言第三卷\\
第三回 瞽女矢心擇婿\endnotemark[1] 虔婆巧說迎郞\\
附 怕婆男小心更受非刑 貪淫婦大膽竟試巨物}
\addcontentsline{toc}{chapter}{第三回 瞽女矢心擇婿 虔婆巧說迎郞}
\markboth{第三回 瞽女矢心擇婿 虔婆巧說迎郞}{第三回 瞽女矢心擇婿 虔婆巧說迎郞}

話說那鐵化次日打扮得齊齊整整到錢家來。竹思寬昨晚未回。已在此拱候。見他來到。迎了進來。郝氏出來相見了。讓了坐下。鐵化叫家人送上禮物。郝氏看見約値百金。喜出望外。拜謝收了。然後扶出錢貴來\footnote{此扶字乃寫其嬌羞。非寫其瞽目也。}。見禮坐下。鐵化一見。果然生得美貌非常。雙目雖瞽。却不癟塌。不凸暴。眼皮微垂。似好目人含羞略閉一般。滿心歡喜。如雪獅子向火。不由得酥了半邊。與火氏比並起來。那一個美而淫惡。這一個麗而嬌羞。如何不愛。少頃安席。搬上酒肴來。上面鐵化坐了。竹思寬下面相陪。錢貴在東。郝氏在西。共坐而飮。那錢貴雖是妓家之女。還是個未破瓜的女孩。嬌羞滿面。低頭坐着。一語不發。鐵化越發看得中意。心愛得了不得。撤席之後。拉了竹思寬在背處。煩他講梳籠的財禮。竹思寬自然是爲郝氏的。假意兩次三番。說定了二百兩銀子。衣服被褥首飾在外。鐵化也算一個財主。這些須他那裡吝嗇。一應都依。又擺上換席來。吃了一會。那鐵化面前放着這樣美人。一時不能到手。心癢難抓。那裡還坐得住。約定了日子就起身回去。次日請竹思寬到他家。就煩同他家人送了禮物來。額外又是二十兩酒席之費。到了吉日。他到了錢家。郝氏預備了精緻豐盛的酒席。叫了一班彈唱的雜耍。熱鬧了一番。晚來成親。見錢貴是眞正處子。婉轉悲啼。憐愛至極。

\begin{quotation}

不覺數點牛精髓。傾入錢姑兩瓣中\footnote{痛惜錢貴語。}。

\end{quotation}

有一調憶秦娥憐惜那錢貴道。

\begin{quotation}

香馥馥。此中有個人如玉。人如玉。恨庸醫誤損他雙目。煙花已慟身埋沒。遭逢又對癡頑物。癡頑物。痛悲傷感。慘切心骨。

\end{quotation}

後來有人知道鐵化梳籠了錢貴。都道。可惜一塊好羊肉落在狗口裡了。就有會打油的人。編了四句口號。說他道。

\begin{quotation}

一顆驪珠圓又圓。奇珍應讓你爲先。

今朝誤落村夫手。異寶塵埋實可憐。

\end{quotation}

且說這錢貴。他雖只十三歲。却聰慧異常。滿心想遇一個風流才子。付此一點元紅。只是女兒家此話不好出口。只得聽父母主張。今失身於此狂且。怨恨之氣充滿肺腑。不覺傷心。枕上含淚。隨口編了一調二郞神道。

\begin{quotation}

憂心悄。斷送一生身窈窕。惡姻緣偏向奴身繞。吹簫誰和。梅花片落江皋。空思弄玉諧同調。沒緊要的良宵偏杳。窗櫺小。恨那冷月偷窺。使人煩惱。悲悼。嗟容貌如花命似草。魂消魄落。一天風雨飄飄。滿地落紅誰個掃。好含恨。狂咀(且)惡少把玉山攪。霎時間。夭桃嬌柳。摧殘傾倒。

\end{quotation}

悲拗不已。欲睡不能。又成了一調轉(囀)林鶯道。

\begin{quotation}

滿腔悲怨多縈繞。聲聲啼血噍嗷。恨難消。似閏的更難曉。何不把殘生來棄了。驀想梁國夫人後從良。嫁着韓王好。怒難消。望他年好景。且耐今宵。香軀相伴狂且嬲。好似烏鴉彩鳳同巢。傷心恨怎消。此情試問人知否。只有空煩惱。倒不如惜花園內雙飛鳥。難忍淚珠抛。嘆今朝花謝。昨日曾嬌。

\end{quotation}

此二詞他後來常常自唱。故爾傳出。他每日眼含珠淚。那一種萬不得已的光景。每每現於詞色。況這鐵化是三十多歲的回子。嘴唇上的鬍子剪得齊齊的。偶然親嘴搵腮。將他那粉森森的嫩臉戳得又疼又癢。好不難過\footnote{眞正苦惱。}。錢貴自幼愛潔。他每日渾身上下。被褥以及衣服。定用好香薰得撲鼻。鐵化敎門中常享用的是牛羊等物。他那身上的一種羶臭。自十萬八千毛孔中透出。甚是難聞\footnote{絲毫不爽。}。那裡有夜深私語口脂香。那錢貴不由得氣苦。在那暗中的眼淚不知落了多少。怎得還有心情同他歡樂。這鐵化雖然愛他。總不見他有一毫喜色。不上一月。他一個財主性兒。只要人奉承他。今反要他去奉承別人。如何行得。他雖會奉承火氏。那是名正的夫妻。抛棄不得。二來怕服慣了。無可奈何。今在錢家雖費了數百金。倒也不在他意中。況且又有個厭舊取新之意。因此也就漸漸淡了。先還三日五日一來。後來或十日半月來一次。到數月之後不復再至矣。這錢貴自從梳籠之後。心中只鬱鬱不樂。又過了多時。雖又歷過數人。都是竹思寬引來的麒麟楦。總非他之所願。他雖然雙目皆瞽。秉性原極聰明。常靜夜自思。我門戶人家。人所重者無非色藝。人人盡道我有沈魚落雁之容。閉月羞花之貌。但今損却雙眸。未免減了許多風韻。老天老天。旣生我如此嬌姿。何吝秋波少許。何苛刻若是耶。若是留得我雙目。雖不敢與天下之美女爭銜。在這平康隊裡或博得個風流榜首。還擇一個才貌情郞。終身有托。於不可知。豈料今日至此。奈何奈何。他心中傷感。遂題了自嗟薄命的四首詩。

其一。

\begin{quotation}

定是前生作孽多。敎濃(儂)今日目無波。

幾回辜負菱花鏡。空有嬌容用彼何。

\end{quotation}

其二。

\begin{quotation}

憶兒幼讀女兒經。衆口咸誇貌娉婷。

孰意十齡遭此疾。煙花日日類浮萍。

\end{quotation}

其三。

\begin{quotation}

不知天暗與天明。但聽傍人說雨晴。

獨有琵琶能解恨。調中哀怨訴幽情。

\end{quotation}

其四。

\begin{quotation}

可憐晨夕伴狂且。怨雨愁雲那得舒。

祇有更闌方少息。將明又喚把頭梳。

\end{quotation}

此詩一出。聲名愈重。哄動一城。往來之人無不憐愛。但他自己另具一段隱衷。常想道。我之此目已經雙瞽。無策可療。我之此身雖落火坑。尚可自拔。於當拿定主意。萬不可隨波逐流。誤却終身。倘有緣得遇一個有才有貌的情郞。當以此身相許。若只圖財帛。與輕薄兒郞醜陋子弟爲伍。不但人笑我心盲。我於豈不自誤。他因執定這個主意。那來訪的人定要選擇纔留。這話在他胸中。無人可吿。眞所謂。

\begin{quotation}

不如意事常八九。可與人言無二三。

\end{quotation}

錢貴矢心立了個擇婿之念。要覓一個伶俐丫頭托以心腹。凡是來訪之人姸媸。叫他預報。這主意不肯向娘說。只說要尋一個好丫頭作伴。那郝氏此時靠他如泰山一般。敢不遵依來命。四處托媒人找尋。不惜重價。一日。媒人領了一個丫頭來。說是童百萬家打發出來的。小名仙桃。纔十四歲。郝氏看了。果然生得性格溫柔。齒牙伶俐。就買了與他。過了數日。錢貴見這丫頭動止端莊。至誠可托。細問他的來歷。也還是正經人家的女兒。因父親不才好賭。將他賣出。幼時曾讀過書。又還識字的。這錢貴甚喜。竟待之如親妹一般。不叫他做一點重活。食必同桌。若無客來。臥必同榻。那丫頭也感激不已。錢貴遂將心腹吿之。丫頭也盡心允諾\footnote{錢貴能待彼如此之厚。故彼後來於厚報之。人見田橫之五百義士。葛誕之三千甲將。與之同死。以爲異。謂後人無此義氣者。但此等義士自有。特無人如二公能赤心待人者耳。}。替他改名代目。因自己眼看不見。取其代己雙眸之意\footnote{仙桃得錢貴以代其目。重之甚矣。可無後患。漸有生茂之意矣。}。話分兩頭。且說童百萬家是南京城中數一數二的財主。如何賣起丫頭來。內中有一個可笑的緣故。這童百萬名自大。原籍徽州府人氏。他高祖之上。在元朝曾做到行省平章政事。掙下了一個偌大家私\footnote{做到行省平章。不曾掙了些聲名政績。只掙了一分偌大家私。宜乎子孫做財主也。}。因愛江南繁華。遂留寓於此。已經數代。到他祖父。雖不曾出仕。却善於經營。專於刻薄\footnote{財主小像。}。所以做了有名財主。他父親名童山\footnote{是個財主姓名。大約字是金穴。}。生他弟兄二人。他排行第二。他哥哥名喚自宏。父親故後。兄弟拆居。他搬回祖籍新安去了。只他一人在此。這童自大雖算不得奇蠢。也有三分呆氣。旣是一字不識\footnote{無怪乎做財主。}。却又半分難捨\footnote{余見大通的財主也有此病。}。他娶的妻子就是鐵化之妹。這鐵氏不但生得性子凶暴無雙。且嬌容更長得奇異無兩。有幾句贊語贊他的妙處。怎見得。

\begin{quotation}

兩道濃眉。濶如柳葉\footnote{眉曰遠山。本自不小。}。一雙怪眼。大勝桃姿\footnote{眼曰杏眼。大杏原只稍小於桃。}。櫻桃口。三寸還寬\footnote{相書云。口大福也大。宜乎做財主婆。}。蒜頭鼻。一拳稍小\footnote{土星高聳。相於稱佳。}。面如皮鼓。兩腮肉有十斤\footnote{是個財主婆相貌。}。體似綿包。渾身重餘二百\footnote{是個財主婆身軀。}。拳眞柳斗。足賽鯿魚。高聲大喝。不啞(亞)虎嘯空山。細語低言。還像洪鐘夜度\footnote{相書云。聲如洪鐘。祿享千鍾。童自大之福。焉知非乃妻之福。}。仰眠繡榻。肥乳峰一尺猶高。側坐牙床。胖屁股十圍還大。陰門寬濶。似兩瓦合成\footnote{怕人。諺云。撂出半邊來還嚇死了你。鐵氏此物於應如是。}。牝蓋豐隆。如一盂扣住\footnote{日用之唾盂乎。和尚之鉢盂乎。這一件却妙。歷來美婦人不可多得者。或楊玉環若是。}。走來時。儼同一座肉山\footnote{肉屛風只用他一人足矣。}。睡下時。全然一隻皮袋\footnote{以乳爲枕。以軀爲褥。於可比漢成帝溫柔鄕也。}。

\end{quotation}

請敎這樣一位佳人。令人害怕不害怕。童自大自娶了他來家。也不曾領敎過他的打罵。只見了他那一種不惡而嚴。不怒而威的樣子。眞如鼠見貓。如獐見虎相似。那鐵氏天性萬種咆哮。只有一件與丈夫相合。却是千般吝嗇。這鐵氏在家時。見他令嫂管敎他令兄的那些法則。學了個滿心滿耳。本要拿厥夫做個小試行道之端。不想這尊夫心悅誠服得很。每見他雙眉略豎。不覺屈膝尊前。忽然兩眼微睜。早已稽顙頓地。這鐵氏雖然凶暴。古語道。大蟲不吃伏肉。他見了這個局面。也竟無所施其威。可以不必用其打了。但只是學了這幾年的閫政來。竟用不着。未免有抱負經綸沈埋草莽之嘆。只好慢慢等待機緣。相時而動罷了。一日。該他發令施行。開張第一的良辰到了。這是爲何。鐵氏在家時。他哥哥鐵化尋了六個丫頭與他媵嫁。買了四好二醜。四個好些的與妹子做針黹。侍梳妝。鋪床疊被。貼身服侍。兩個粗笨些的。爲灑掃漿洗之用。四個好的裡頭有一個頂尖出色的丫頭。他也是好人家女兒。因他父親戴遷好賭。輸了鐵化的錢。無可償還。沒奈何。將女兒算來准賬。那來時纔得十歲。就與了妹子。鐵氏見他生得乖巧伶俐。心愛非凡。每日替他梳頭打扮。與他好的吃好的穿。替他起了個名字。叫做仙桃。這丫頭也讀過二三年書。因他資性聰明。竟識許多字。還動筆寫得來。女紅件件都略知些。說話行事能看人眼色。鐵氏這樣一個急如火暴如雷的性子。別的丫頭一打非數百不饒。一罵非半日不住的。三四年來。不但惱彈不曾彈他一個。連哼也不曾哼他一聲。自嫁到童家。丫頭跟了過來。已半載有餘。那一日淸晨。鐵氏在窗前一張桌子上放了鏡臺梳頭。童自大就在桌橫頭一張椅子上坐着。看他抹脂膩粉。刷鬢掃眉。看得十分親切。只見他。

\begin{quotation}

醬色臉上。濃堆鉛粉。襯成靑紫二色\footnote{世間偏是黑面婦人愛堆鉛粉。添醜即增美耶。令人不解。}。濶大唇中。重點胭脂。染做血紅兩片\footnote{此方可謂之硃唇也。}。牙黃齒垜。眞像金嵌玉山。面白頸烏。果是銀杓鐵靶。髮像金絲。也學個時樣梳妝。腕如鐵杵。還帶副起花金鐲\footnote{俗謂醜人有醜福。正此謂也。}。

\end{quotation}

童自大見了。不由得膽怯。心中凜凜然起來。他打扮已完。要水洗手。忽見仙桃掇了一銀盃(盆)水來\footnote{銅臭之家焉得有銀盆。借這銀盆二字。以襯鐵氏鐵杵之腕。以作一笑耳。}。只見他。

\begin{quotation}

黑臻臻靑絲細髮。喜孜孜俏麗嬌容。面上紅白相兼。身材高矮廝趁。裙下一對小小金蓮。盆邊十個尖尖玉笋。頭上簪一朶嬌滴滴仙花。耳上帶一雙黃烘烘金墜\footnote{要知此非贊此時鐵氏之婢之美。乃贊異日鍾生小星之美耳。看者眼光須長。}。

\end{quotation}

童自大看了這半日的魔母。忽然見了天仙降世。頭頂上錚的一聲。魂已出竅。癡呆呆大張着嘴。口水順着嘴丫流出\footnote{人見美食多有口中流涎者。見美色則非可食之物。當下口流涎。而往往人於上口流涎。不知何謂。豈自下而上耶。辱翁曰。所謂秀色可養(餐)也。}。不轉睛的望着。難道丫頭來了這些時。童自大不曾見過不成。爲何今日忽做此形狀。但他每日看見鐵氏。都是梳洗過了。妝飾起來。雖然醜陋。看慣了還不覺得。今日細窺底裡。見了本來面目。眞正醜到十分地位。二來每常因懼夫人的虎威。丫頭偶然一見。不敢詳視。不過偷目一覷。況又另外站着。也不覺十分俏麗。今日忽主婢在一處。相形起來。佳者更覺其佳。醜者愈增其醜。不覺出神。竟看癡了\footnote{忘却夫人虎威。眞可謂色膽天來大。}。那丫頭掇着水。一擡頭。忽見姑爺的這個呆樣。不由得嘻嘻一笑。他也並非有心。這一笑剛被鐵氏看見。這鐵氏身子胖大。他有這個放樣的肥臀。特做了一張放樣的大杌做坐具。他洗手時側過身子去的。所以不曾見乃夫的尊容。今見丫頭笑得有因。急轉身一看\footnote{轉身二字寫得妙不容言。何以見之。只此兩字。便畫出一個胖得沒有理的人來。若他人回頭。只須頸項一轉。他因胖得極。脖子過粗。頭回不轉。只得連身轉過。此等處於必寫得入神入理。余不知作者之心何細若此。}。那童自大忽然見丫頭一笑\footnote{古云一笑傾人城。仙桃一笑。童自大便殃及其身。可見佳人之笑。非國家之祥也。}。以爲有情到他。益發昏了。還呆着臉癡砢砢的\footnote{呆人的有此呆態。}。鐵氏見了他這個形狀。把那幾年學的閫政施將起來。數月鬱的醋氣發將出來。伸出胡蘿蔔粗的五個嫩指。兜臉一掌。一手的水。異常響亮\footnote{趣甚。}。童自大正在妄想之際。被這一下。嚇得攛的老高。打得個發昏章第十一。正打得愣愣掙掙的。被鐵氏擰着一隻耳朶。拎將過來\footnote{拎字妙甚。一見鐵氏勢頭之凶。一見童自大怕懼之弱。眞可謂耳提面命。}。寃家路窄。適纔丫頭們撣桌子上灰的一個雞毛撣箒還不曾收。恰巧放在傍邊。被他抓將過來。有毛的一頭攥在手中。將那一頭有大指粗的紫竹桿。夾光脖子上就是十多下。打得童自大頸如刀割。淚似雨流。跪在地板上亂轉。鐵氏罵道。殺剮的奴才。你好大膽。在我眼跟前公然對着丫頭調起情來。你背着我。兩個不知偷了多少回數。實實地說來。饒你一死。童自大哀哀吿求。奶奶你寃死我了。我成日守着你。寸步不離。或是有事就往外邊去了。我遵奶奶的王法。每常連丫頭們看還不敢看。可還敢生這個心腸。就有這樣狗心狗肝。也沒有地方去做。你請詳情。那鐵氏雖然性如烈火。聽他說得頗有情理。又見他脖子上腫得一條條比指頭還粗。便道。我饒過你這一遭。下次再要大膽。休想得活命。起去罷\footnote{鐵氏尚還有憐惜之情。還算不得第一個惡婦。}。童自大如鬼門關放赦。不住道。謝奶奶天恩。爬起來揉着脖子。往前邊去了。鐵氏餘怒未已。叫過丫頭來要打。這丫頭雖從未曾嘗過此味。主母的酷刑是常常見過的。今聽要打。眞嚇得心膽墮地。跪着哭道。我跟隨姑娘這幾年。蒙姑娘恩典。如此待我。我何敢欺心。適見姑爺的樣子好笑。實忍不住。笑了一聲。敢有甚私情別意。求姑娘開恩饒恕罷。鐵氏數年來罵也捨不得罵他一句。一時如何打得下去。見他柔語悲啼。似梨花帶雨。心中暗想道。這個妖貨。我看了這個樣子。還疼愛得了不得。何況男子漢見了。可有個不愛的\footnote{我見猶憐。何況老奴。纔是眞情種語。鐵氏之不肯留仙桃。雖云是妬。却正是愛。}。這個禍根放在跟前不得。我惱後無眼看不得許多。古人說。老虎還有打瞌睡的時候。倘弄出來。那時悔就遲了。不如趁此時打發掉他罷。主意定了。說道。我跟前如何許人弄鬼。我養你幾年。也不忍打你。你只收拾收拾。打發你別處去罷。丫頭痛哭起來。道。我服事幾年。蒙恩擡舉。今日非有心之過。姑娘如何就要棄我。我情願與姑娘打死。我總是不願出去。鐵氏見他哭得傷心。胸中也覺慘然。因醋念橫在胸中。違着心罷。定頭不允。那丫頭知不能留。雖感他數載之恩。又懼觸了他此時之怒。磕了個頭。哭着收拾他的衣服被褥去了。鐵氏聽他哭得甚是悲慘。心中好生難過\footnote{愛心竟不能奪醋念。婦人之醋。誠可畏也。}。叫了一個家人童佐弼來。吩咐道\footnote{童佐弼。謂其媒人同做弊也。}。將這丫頭帶到媒人家去。不拘身價。揀個好人家與他做媳婦去。不可混配了人。坑了這孩子。童佐弼答應。領着出去了。鐵氏復沈思\endnotemark[2]道。這三個像樣的丫頭也是禍根。萬不可留在上邊。將家中選了三個無妻的僕人。即日配了下去\footnote{古云。城門失火。殃及池魚。仙桃一笑。惠及三鬟。此夫婦六人當尸祝之。}。單留兩個醜婢。一個名葵心。一個名蓮瓣。在傍使用。纔放了心。有一調西江月贊這兩個醜婢道。

\begin{quotation}

面黑難施膩粉\footnote{較主母猶大雅。}。髮黃罩個包頭\footnote{可謂善於藏拙。}。腰粗全仗汗巾收。大脚幸虧裙覆\footnote{如此喬妝。獨不畏主母醋發乎。}。掃地鋪床能事。尿瓶馬桶常丢。料然難與主人偷。可免姑娘獅吼\footnote{孰意大謬不然。反竟列爲小星。}。

\end{quotation}

不想仙桃這一笑。倒便宜了這三個丫頭。即日得嘗妙物。只當是替他們做了一個媒人。眞可謂一笑姻緣。却是總成了別個。與自己倒不相干。這童佐弼領了仙桃到媒人家來。因見他生得有幾分姿色。又主母吩咐不拘身價。思量在他身上發一主橫財。遂暗暗與媒人商議。許他加一酬謝。媒人道。非賣與門戶人家不得重價\footnote{惡僕奸謀。一樣黑心。}。適逢錢家要買丫頭。講明身價銀八十兩。賣與他家去了。媒人分了八兩。童佐弼落起六十兩正。只拿了十二兩銀子來回鐵氏的話。假說受了財禮十二兩。嫁與江西一個木商做兒媳而去。鐵氏聽得。心中慘切了一會。見說與木商做媳婦。倒又替他歡喜\footnote{鐵氏之於仙桃。始終相愛。故後仙桃相會時。毫無怨意也。}。那童自大被打了這一頓出來。到書房中想道。我一個大財主。誰不敬我三分\footnote{調侃世人。}。我這樣小心奉承他罷了。倒還這樣凌辱我。我見他就怕。是沒奈何了。難道官府衙門也怕他不成麼\footnote{妙想。孰不知雖不怕此。而各有所怕。奈何。}。我去吿他一狀。後來或者好些。也不可知。別的大衙門我不敢去。我到縣裡去吿。又想道。這個狀子不好雇人寫的。用口訴罷。又道。不好。一堂的人聽着。怎麼好說被奶奶打了。不怕人笑話麼\footnote{千算萬計。活畫出一個呆人的肚腸來。}。躊躇了一會。猛然想起道。我那姑表大舅魏如豹。他現當着上元縣刑房書辦。何不去同他商議。又轉念道。但恐他爲護表妹。未必肯管。又想道。甚麼相干。做衙門的人。世人說的。公人見錢。如蠅見血。要有幾個錢給他。吿他的娘他還未必管呢。何況遠房表妹\footnote{不意此呆人竟有此奇見。}。我許他個厚禮。他自然肯爲我出力\footnote{財主都捨得厚禮送人。我不敢信。據他說許他個厚禮。單只許。或有之。}。定了主意。遂到魏家去尋魏如豹。只見他哥哥魏如虎迎出來。道。舍弟不在家了。妹丈請裡邊坐。童自大到了廳上坐下。魏如虎道。老妹丈尋舍弟說甚麼。童自大道。尋他商議一句要緊的話。魏如虎道。他衙門中有事。淸早起就去。到晚方得回來。若要尋他。明日絕早到縣門口就尋着了。忙進內捧了兩鍾茶來。讓童自大吃着。又道。老妹丈有甚麼要緊的話。也可以對我說得麼。童自大嘆了一口氣。將護領捲下。伸着脖子與他道。請驗驗傷痕。魏如虎見都是指頭粗的紫印。腫得老高。驚道。甚麼人敢大膽打老妹丈。了不得了不得。童自大道。還有誰。就是令表妹了。遂把無心看丫頭被打的話吿知。魏如虎大怒道。豈有此理。天地間那裡有這樣的事。婦人都凌虐起丈夫來。不要怪我說老妹丈。你太不濟。容他放肆。要是我麼。哏\footnote{道家書云。多言護(無)益。不如默而無言。魏如虎只圖奉承財主妹丈。忘記了夫人虎威。宜乎後來受罪也。}。還不曾說出下句。聽得屛門後他妻子接口道。要是你。便怎麼樣呢。他說話時手中正拿着一杯茶。聽得問了這一聲。打了一個寒噤。把杯子掉在地下。跌得粉碎。面上便失了色。答道。要要是我。我就咬着牙死死捱\footnote{這纔算眞正好漢。}。童自大暗暗含笑。上前作了個揖。那夫人也回了一福。便把眼望着魏如虎。瞪了一瞪。他低着頭。面如死灰。童自大見不是好光景。也不再坐。就辭了出來\footnote{童自大竟能鑒貌辨色。竟不呆了。}。魏如虎送着。伸着舌頭悄聲道。倒是沒有說甚麼別的話呢。造化造化。童自大笑道。我看你比我還怕。你怎麼先又說那硬話。他忙伸手把童自大的嘴摀住。道。我的活祖宗。你悄聲些。不要替我惹禍。因附在他耳朶上低聲道。怕老婆的人。難道硬話也不許說一句麼。二人哈哈大笑。一拱而別。童自大回家。見四個標致丫頭都不見了。只剩醜婢二人。又不敢問。晚間見鐵氏惡狠狠的睡了。他在床脚頭穿着衣蹲了一夜。也不敢睡\footnote{蹲字趣。不知這一夜他可曾睡着否。}。次日起個大早。悄悄下床。出來看見童佐弼。私問他四個丫頭的下落。方知三個配了家人。仙桃已經賣去。他恨了幾聲。就出門到縣前來尋魏如豹。見衙門口靜悄悄也沒有人。等了好一會。見魏如豹手中拿着兩個膏藥。一瘸一踱的走來。他一眼看見童自大。忙拐着上前問道。昨日失迎。老妹丈淸早到這裡有甚麼貴幹。童自大道。有一件事特來尋老兄商議。魏如豹道。這門首不是說話的去處。請到裡面科房中坐了再講。遂同他進了儀門內。到科房中一條凳上。讓童自大坐下。他就挨了坐着。問道。老妹丈有甚麼事見敎。童自大道。我受令表妹的氣。實在過不得了。我又不敢奈何他。想要吿他一吿。要雇別人寫狀子不好意思的。要借重老兄寫寫。因把脖子伸與他看。道。傷痕現在。便是干證了。魏如豹聽了。只是嘆氣不做聲\footnote{嘆氣不答者。欲寫不敢。不寫又恐拂了財主妹丈之意。又貪或有筆資。故做難耳。}。童自大道。我不白勞老兄。少不得有個薄儀奉謝\footnote{可謂錐心入耳之言。}。魏如豹忙道。倒不是爲此\footnote{童自大一許謝儀。便撞着他的心事。便逼出下文一篇說話。至於倒不是爲此一句。乃違心之言。假體面話耳。把衙門中吏胥心腸說盡。}。低聲道\footnote{先魏如虎一個低聲道。此處魏如豹一個低聲道。後文巨金一個低聲道。寫得一夥怕婆人。有許多張致醜態。不覺失笑。}。實不相瞞。我寒家祖墳上的風水有些古怪。大約是陰山高。陽山低。祖傳代代有些懼內。到了我愚弟兄。越發是馬尾穿豆腐。提不起。我家兄那樣個好漢。咱衙門裡算他頭一名。番子二三十人也打他不住。憑你甚麼狠強盜。見了他。俯伏在地。家嫂那樣個肌瘦人兒。到他跟前。纔打到他奶胖。老妹丈是常見的。家嫂間或一時動怒。要打他一百。打到九十九下。不但不敢爬起來。連動也不敢動。我不是說大話。我每常打到捱不得的時候。還大膽討討饒。他連饒也不敢討。啞巴似的咬着牙死捱。因他叫魏如虎。外邊人知道這事。說當年李存孝會打虎。是個肌瘦小病鬼的樣子。恰巧家嫂也姓李。又生得小巧。人都叫他母存孝\footnote{肌瘦的旣怕。}。大約老妹丈也有所聞。到了弟益發可憐。說起來連石婆婆也掉淚。那些作踐的事也說不盡。一句結總的話。也不怕老妹丈見笑。他此時若叫我死。大約也不敢再活\footnote{不意夫人之威竟同君父。}。也怨不得。一來我的賤體比老妹丈小了好些。賤內的尊軀與舍表妹相彷彿\footnote{胖大的又怕。不知婦人的身子生得如何。丈夫纔不怕。}。他要打起我來。一隻手像拎小雞似的。輕輕就撂在地下。一屁股坐在脊梁上。就如孫行者壓在五行山。還想動一動麼。憑他揀着那一塊。愛怎麼打就怎麼打。我叫做擡轎的轉彎。滿領就是了。總是我賤名的這個豹字當初起的不好。童自大道。怎麼見得。他道。我賤內姓師。獅爲百獸之尊。豹見了獅。可有個不怕的。我常想。就是豹子眞見了獅。不過是個死罷了。也未必怕到這個地位。我見了他。心驚膽碎。說不出的那個怕法。若見他個笑臉。我就比做神仙還快活。但見他有些怒容。我渾身肉都亂顫。那心撲撲的跳到口裡來。話都說不出一句。我背地上了他個尊號。稱他爲九靈母元聖。這是西遊記上太乙天尊騎的那九頭獅子的名號。那是個獅祖。必定纔這樣利害。因笑着把那膏藥與他看。道。你說我買這東西做甚麼。童自大道。據老兄說起來。想是被嫂子打傷了那裡了。魏如豹道。那打提他做甚麼。老妹丈。你脖子上那幾條傷痕也算得個打麼。要在我賤軀上。就算天字第一號的輕刑罰了。可憐我一年三百六十日。渾身上下那一處沒些傷痕。若貼起膏藥來。不但沒這些錢買。竟把衫子褲子襪子總攤了膏就是了\footnote{何必費許多事。拿一床單被攤着。一個大膏藥裹上。何等省事。}。說着。將襪帶解開。把褲脚攎起來。只見他兩個膝蓋紅腫有飯碗大。全是碎血眼。童自大忙問道。這是怎的來。魏如豹笑道。寃屈死人。昨日一個敝友請我吃酒。回家去遲了些。我是個官身子。每常回去或遲或早。都是家兄出來開門的。他也還沒得甚說。昨日家兄不知同老妹丈說甚麼來。家嫂着了惱。從昨日午間在屋裡。家嫂叫他頂着淨桶跪着。不放起來。是賤內出來開門。驚動了他了。發起性來。說我定是在外邊嫖老婆。不然爲甚深更半夜回家。我把嘴都分說破了。他也不信。眞是口中淌出鮮血來。他還說是蘇木水。你有甚麼法。他拿些碎磁瓦。砸爛了墊在我膝下。足足跪到天亮。也還罷了。他又把一塊死沈的大槌衣石。叫我頂在頭上。壓得那碎磁都戳進肉裡頭去。你道刻毒不刻毒。到了今早還不放起來。虧我爲(苦)哀求\footnote{眞虧他。}。再三吿說。今日衙門裡有要緊公事。恐怕誤了。纔饒了起來。我出來時張了張。家兄還像空陽文。頂着個花盆在那裡跪着呢。我到了外邊。一步也挪不動。看了看。全是血眼子。都是那碎磁戳的。兩腿幾乎要折。沒奈何。只得慢慢的捱到外科藥鋪裡。買了兩個膏藥來貼。爲甚麼今日來得遲些。你不見我方纔走路一瘸一點的麼\footnote{也有便宜處。未曾考滿。已先做了典史。}。我若替你寫了這狀子不打緊。後來設或舍表妹知道了。會着我賤內一說。我還想活麼\footnote{畏妻如蠍之妹夫。又尋着這兩個懼婦如虎豹的大舅。濟得何事。}。那就是眞正的死無葬地了。就是老妹丈也有些不妙。這事不是兒戲的。性命相關。不可輕舉妄動罷。我勸老妹丈忍忍罷。童自大聽他說了這些話。也不知是眞是假。見他有些作難。袖中取出個草紙包兒來。送上道。這算不得甚麼。老兄買一鍾茶吃。果然替我出了氣。我後來還有重謝\footnote{此即先所說許他個厚禮也。}。魏如豹一見了包兒。便一臉的笑。道。我倒想了個主意。不知可做得來\footnote{見了包兒。就一臉的笑。便有了主意。活畫一衙門中人。妙極。}。假推道。一個至親家。如何好受禮的。童自大道。老兄旣有主意。你要不收這薄意。我也不敢奉求了。塞在他手中。他也就接過去。道。老妹丈旣如此說。我且權收下。便裝入鈔袋中。然後說道。據我想。這件事也不必定要吿。況本官病了。這幾日不曾出堂。不見衙門口靜悄悄的麼。就有狀子也吿不進去。內邊管轉桶的管家巨大爺巨金\footnote{尋那懼荆的管家。中甚用。}。同我最相厚。等我請他來同他商議。煩他稟聲老爺。出根簽。差兩個人到你府上。只說官府查訪得他欺凌丈夫。要拿來處治。唬嚇唬嚇他。舍表妹一個婦道家。到底膽小。他聽得自然害怕。若後來改過。也就罷了。況且你我都站在不敗之地。沒有甚麼干係。不怕他們知道。一興詞動訟。那就有指實了。你說可行得麼\footnote{作商量語者。此原非妙策。若不出一主意。何以銷繳那個包兒。}。童自大見說官府不上堂。也沒奈何。只得說道。聽憑老兄尊意罷。魏如豹煩了個門子到穿堂後去請巨金。等了一會。見他來了。童自大看他好一條大漢。方面大耳。一部落腮鬍鬚\footnote{偏是此等好相貌好身材的人。更怕得利害。不知何故。}。左手捏着一塊藍紬手帕。將左眼摀着。二人起身。讓他坐下。他問魏如豹道。這位是誰。魏如豹道。這位是舍親童百萬。巨金忙施禮道。得罪得罪。聞大名久了\footnote{余聞(閱)至此。掩卷長嘆曰。甚矣。銀錢之令人起敬也如此。人生世上。勢爲富厚。蓋可以忽乎哉。}。魏如豹道。數日不會。不知大爺患目。失候得很。巨金哈哈大笑道。我那裡是害眼。魏如豹道。不是害眼。是怎麼的來。巨金笑着說道。魏師傅你不是外人。童大爺旣是令親\footnote{世上有錢人。勿論老少。無不尊稱之曰太爺者。可嘆。}。也都是自己。實不相瞞。前日敝恩上同主母偶然角口。敝主母就拿我賤荆出氣。罵了一頓。我正在家裡吃酒。桌子上放着一把大壺。賤荆回來。摔碗摜碟的。我又不曾敢說多話。只說你在上邊受了奶奶的氣。怎到家裡來使性子。魏師傅\footnote{叫一聲又說。妙。如聞其聲。}。你就說我這句話也沒有衝撞了他。我不曾防備。被他拎起酒壺來。夾臉就是一下。虧我躱得快。打在眉毛頭上。幸得是我這樣個漢子。也還掙住了\footnote{眞好漢。}。要是軟弱些的。不死也有個小發昏。一來是祖宗保佑。二來虧我靈泛\footnote{果然虧他。}。不然眼睛珠子也打出來了。他一把揪住我耳朶。還要撏鬍子。幸喜我的力氣大。死命掙脫了。往桌子底下一鑽。纔得跑掉了\footnote{偏是此輩人。專會說體面話。}。要是撏掉半邊。今日還不得出來會你呢。因把汗巾拿下。道。你看看。魏如豹同童自大一看。眉稜骨烏靑。眼睛腫得像桃子一般。只有一縫。魏如豹道。這一下利害呢。巨金道。先還腫得大。連眼都睜不開。這兩日好了許多了。便問道。你尋我說甚麼。魏如豹遂將童自大的事對他說了。他儘着遙(搖)頭咨嗟。魏如豹道。舍親不敢白勞。少不得還要奉酬。巨金道。魏師傅。不是這個話。我們是好朋友。我若可效力。童太爺難道還不値一個相與麼。內中有個緣故你不知道。因低聲道\footnote{描寫入神。}。前日敝恩上偶然同主母說頑話。敝恩上說。大凡做官的人。誰沒有幾個小老婆。你今將五十歲的人了。也該讓我娶個小。樂一樂。還哈哈的正笑着。不想被主母跑上去。把臉同脖子抓得稀爛。一條條的血口子。好不難看。怪是也怪不得敝主母。原是敝恩上的不是。這樣的話可是亂說得的。還虧主母很心疼的一位小相公。有八九歲了。每常老爺帶他出來頑。你也見過。是他哭喊着抱着老爺。奶奶纔饒了。不然還利害。因上不得堂。故推病這幾日呢\footnote{大約官府推病不出堂。多半爲此。}。我賤荆受氣。我造化低。都同在這一日了。如今敝恩上在主母面前千小心萬陪罪的時候。我若去一稟。家主母一知道。要怪我替男人吿妻子狠惡。這還了得。敝恩主正在奉承的時候。不要說用刑。只吩咐我賤荆處治。那就即死無挪\footnote{閫法重於官刑。令人那得不怕。}。是這個緣故。所以不敢奉命。向童自大道。尊夫人還算賢慧呢。一個少年的標致丫頭。見了遠遠的躱開。還怕惹是非。那是大膽望着得的。這是自己失於檢點。如何怪得人。不曾打斷脖梁骨就算萬幸了。要是敝恩主同我犯了這樣的法。哏。恐怕連性命都難保。我奉勸是好話\footnote{眞是好話。}。請息息怒。此後凡事小心些。樣樣自己留神。就不妨了\footnote{的是良言。保身妙訣。}。因立起道。不能奉陪。賤荆上去了。一早起。恐要回來吃飯。我照看去。拱拱手去了。童自大只是嘆氣。魏如豹道。我爲老妹丈。不過如此盡心罷了\footnote{這一句話。把那包兒消結了去。}。說不進去。却沒奈何。老巨說的也是好話。老妹丈得忍就忍。我有幾句護身符的藥言奉傳。你但記熟了。便可保無後患。

\begin{quotation}

他要打區區。區區先睡倒。他若罵區區。區區只贊好。他又省力氣。我又省煩惱。這個波羅密。的是個中寶。但能知道此。保身直到老。

\end{quotation}

老妹丈千萬記着。請回罷。衙門中無事。弟也要返舍了。倘回去得遲。又生禍患。童自大見他如此說。只得別了出來。因大淸早來尋他。此時又渴又餓。到一個茶館中去吃一壺茶。軟飽軟飽。正坐在吃茶時。聽得隔座幾個人在那裡說笑。一個道。江寧縣喜老爺。做官也風厲。人品也生得好。五短三粗的一條漢子。一嘴連鬢鬍。頗有三分殺氣。他是福建人。酷好男風。他衙門裡有個門子。姓董名混。叫做小董賢。生得細皮嫩肉。比女人還嬌媚些。喜老爺愛上了他。在奶奶面前說衙門中事繁。日間辦不完。夜裡還要料理。一個月倒有二十日在書房中同小童(董)兒睡。後來不知怎麼被奶奶知道了。那日有三更天。忽然開了宅門。奶奶帶着丫頭僕婦們。點着幾個燈籠。直奔書房。打開門進去。喜老爺正同小〔董〕兒睡着呢。奶奶上前把被一掀。兩個都是精光。誰知奶奶手裡拿着一把大環錐。把那小董兒嫩屁股上戳了十來下。那小廝疼得滾到地下。還戳了兩錐子。他鑽到床底下去纔罷了\footnote{老爺之肉錐猶可。奶奶之鐵錐太利害些。}。奶奶把喜老爺的頭抱住。儘着薅鬍子。薅掉了半邊\footnote{余有一友看至此。笑曰。此與鬍子何涉。何不以環錐錐其陽具。方爲切當。余曰。不然。因此陽具被小童(董)占去。方錐其股。焉捨得復錐之。薅鬍子者。意其嘴於必與之相親。故遷怒於鬍。此謂不得已而思其次。}。就揪着半邊鬍子。像牽羊的一般拉着\footnote{陳慥變羊。尚不至此狼狽。}。衣服也沒有穿。披着床被。拉上去了。古人說。好事不出門。惡事傳千里。這是他衙門裡事。不知怎麼就傳出來。第二日就有人寫出謠言歌兒。貼在兩府裡照壁上。我還記得是四句。道是。

\begin{quotation}

夫人半夜鬧書齋。嫩股遭錐實可哀。

滿都(部)虬髯將去半。縣公風厲在何哉。

\end{quotation}

不想被府尹大老爺知道了。說他爲民父母。怎縱容內眷半夜鬧到外邊來。加他不禁兩個字。取了職名。封門聽參。喜老爺着了急。他同大老爺管事的堂官雪太爺名叫雪機。素常交好。他托人去問雪太爺。說本地鄕紳中誰同大老爺契厚。好去求了來說請(情)。雪太爺說。大老爺性情倔強。是個鐵面無私的人。從來不聽情面。如今只有一條路。舅老爺新近纔到。叫他尋着舅老爺的門路。向太太求求情。太太若對大老爺一說。一天大事都完了。喜老爺就煩雪太爺送了舅老爺一分重禮。舅老爺向太太說了。太太也不知向大老爺怎樣說。就不得知道\footnote{這一句頓挫得好。若詳詳細細的講。他衙中內裡的事。外人如何得知備細。}。那日大老爺坐在穿堂上儘着出神。搖着頭沈吟\footnote{畫出個有心事人的樣子。}。恰好本房吏上去呈稿。大老爺看了。說道。這件事我正在這裡爲難。今日太太再三說叫我饒了喜知縣罷。本府想。旣取了他的職名要參。怎麼好忽然歇了。若不聽太太的話參了上去。太太若知道。笑道。本府又是喜知縣之後車了。你的主意怎麼說。那本房道。大老爺取喜知縣職名。闔屬皆知。忽然中止。儼有情弊。恐科道兩衙門知道不便。大老爺道。我在躊躇。正是爲此呢。本房道。如今只好當着太太說饒了他。瞞着暗暗參了上去。等旨意下來。太太也便沒法了。大老爺連連點頭道。你這主意有理。正贊着。忽見大老爺頭上。像個黑老鴉一般。一翅飛得老遠。落在地下。衆人忙看。原來是大老爺戴的紗帽。再回頭看大老爺時。不知太太如何知道了\footnote{雪機者。洩機也。前旣洩機與喜知縣。此洩機與主母。不待言矣。}。拿着個棒槌走出來。在大老爺惱後一下把紗帽打得飛去。大老爺震昏了。就伏在公案上。那本房見勢頭不好。一擡頭。見太太的棒槌已對惱門劈下來。他叫了一聲不好。忙把頭一歪。連耳朶帶肩胛早捱了一下。得了命就往外跑。太太拎着棒槌便往大堂上攆。衆管家爺們跪了一地。攔住稟道。求太太給老爺留體面。外邊多少書辦衙役看着。太太如何出得去。太太還不依。虧得走出一二十個管家娘子們來苦哀求。纔進去了。管家爺們也把大老爺扶了進去。頃刻。雪太爺出來吩咐。喜知縣免參。照舊開門理事。大老爺的名字叫做都三畏。說是畏天命畏大人畏聖人之言。如今人叫他都四畏。說兼畏夫人了。又還有人稱他都元帥的。喜老爺雖造化。保住了功名。近來奶奶做了禁子。他成了犯人。但是出堂。奶奶在暖閣後監押着。退堂便一齊上去。他原是一嘴鬍子。因去了半邊。不像模樣。索性剃掉了。他成了光下頦。好不難看。乍見竟認他不得。這些時走路把腰彎着。我先以爲或是奶奶打傷了腰。我有一個朋友在他衙門裡當差。前日向我說。如今喜老爺但出門。奶奶拿他個喜圖南的名字圖書\footnote{余先謂此知縣何故姓喜。今見其名。方知所謂。}。印在龜頭上。回來要驗看。若是擦掉了便了不得。所以如今走路彎着腰。說了。衆人大笑。童自大聽了這一段話。心中暗想道。可見如今世上也沒一個不怕老婆的。做官的人都怕到這個地位。又何況於我。我今後只是一味小心。凡事順着他。再沒有無緣無故只管打罵的道理。他拿定了這主意。他的一壺茶早已吃完。又要了兩壺水也呷了。灌了個滿肚。與了四文茶錢回家。不題。再說魏如豹送童自大去後。心中喜道。這個嗇鬼從來連水也沒有擾過他一杯。今日却也得了他個包兒。方纔我若嫩些。再要推辭。他管情就收了回去。昨晚我那娘着了惱。今日做個大大的東請他一請。陪個不是。大約就好了。況且衙中也無事。早些回去罷。出了衙門。到一個錢桌子上。腰中取出那包兒。打開一看。攧攧約有二錢重。却紅不紅黃不黃的顏色。那鏨口上還上了些銅靑。遞與櫃上一看。那人笑道。我店鋪中只換銀子不換金子。你拿到首飾籠子上去換。魏如豹道。難道一些銀氣也沒有。你夾開來看看。那人夾開又看了一看。足足四成。道。要換便換。不換請別處去照顧。魏如豹暗罵了幾聲吝鬼。這樣銀子也拿來送人。沒奈何。道。換了罷。那人一稱。只得一錢八分\footnote{本是二錢。因是財主家戥子。斷了二分。窮人的戥子或不至此。}。換了幾十文錢。算算買別的不夠。買了三斤牛肉。用了二十四文。打了二斤燒酒。也是二十四文。拎了回來。剛到家門口。他妻子師氏正在門內看看街上兩條大獅子狗鏈在一處\footnote{師氏有(看)獅狗鏈幫。也可謂方(物)以類聚。}。正看得有趣。一見了他來。怒問道\footnote{打斷興頭宜乎該怒。}。你替誰買的酒肉。魏如豹正低着頭走。猛聽得這一聲。嚇了一攛。幾乎把酒瓶掉在地下。定了一定神。陪着笑。掙了一會。掙出幾句來道。我見娘這幾日熬淡得慌。心裡急得了不得。今日造化。弄得了幾分銀子。買二斤肉打斤酒來孝敬你。那婦人嚥了一口唾。登時一個惡鬼臉變做笑嘻嘻的龐兒。道。好好。我正想些牛肉頓(燉)絲瓜吃呢。纔過去一個菜擔子。你叫了來。問可有絲瓜。魏如豹忙〖口么〗喝那賣菜的回來。那賣菜的來到門首歇下。道。買甚麼。魏如豹道。要絲瓜。那人道。我賣的是肥韭菜。沒有絲瓜的。魏如豹道。我不要韭菜。那人挑上擔子。口中嘓噥道。韭菜是興陽的倒不吃。絲瓜那東西是眠陽的倒要。那婦人聽見這話。忙問道。你怎這樣死相\footnote{此婦眞活泛。}。旣沒有絲瓜。韭菜炒肉還不好麼。快多買些\footnote{多買些。趣甚。旣圖興陽。何不連擔子留下。}。魏如豹又叫回來。買了幾斤進來。見哥哥還跪着呢。李氏見小叔買了肉韭菜同酒來。滿心歡喜。向魏如虎道。饒你去罷。快幫二叔切肉擇菜去\footnote{也不是兄弟。竟是救命王菩薩。}。魏如虎將淨桶輕輕放下。腰彎背折掙着去相幫。到廚下炒了。盛了一大盤一小盤。大盤中肉多韭少。送與嫂嫂同妻子享用。魏如虎幫着盛飯篩酒。伺候他妯娌二人吃了。然後將那小盤子掇過來。他兄弟二人吃。這盤中肉少韭多。那魏如虎只翻着肉吃。魏如豹單吃韭菜。他妯娌二人看着。那李氏問嬸子道。二叔怎麼不吃肉。單揀韭菜吃。是甚緣故。師氏低聲道。纔賣韭的說韭菜興陽。故此他儘着吃呢。李氏聽說。釘釘的望着魏如虎。還在那裡尋肉吃。心裡急得忍不住了。罵道。你害了饞癆了。你把韭菜也吃些是呢。那魏如虎正在找肉吃。嚇得把手中箸子掉在桌上。回頭望了望。不知是甚緣故。忙拾起箸將韭菜一連吃上幾大口。李氏笑着道。看這纔是理。他娌妯二人彼此心照。笑了一場。閒話休題。且言正傳。再說這仙桃自賣與錢貴之後。改名代目。凡來之人好歹。叫他預報。這錢貴一時在盛名之下。閱人雖多。並無一個知心中意的人。皆不過淡然相處而已。他又自負才華。不肯與白丁相對。遇着那稍通文墨。面目可對的。雖貧窮之士。還可博他一笑。若那形容醜陋。氣質粗俗的。雖縉紳公子。富老大商。他雖沒奈何。違心承奉。然那一種萬不得已的光景。未免露於辭色之間。這些大老官都是好頂花盆戴高帽的人。見他如此。往往含怒而去。他父母雖然懷恨。緣係親生之女。又自幼嬌惜慣了。故拾(捨)不得難爲他。他所以任性到底。那衆人中有種俗人笑話他。也有一種情人憐惜他。那俗人笑他呢。說他門戶中人。原是倚門獻笑圖幾個銀錢。況瞎了雙眼。還要揀甚麼兒郞。聰俊富貴的倒不陪奉。反喜那餓鬼窮酸。有何好處。那情人憐他呢。說他立志如此。也是妓女中有氣槪的。有這一段好心。將來定有一個好收圓結果。兩種話傳到他耳中。他只執定主見。毫不動移。但他父母雖然疼女。未免愛錢。那錢爲命是一生全在銀錢上做工夫的人。他當日靠着郝氏。滿心中想掙一個烏龜中大大一個財主\footnote{大大的財主甚軒昻。加上上數字甚不堪。}。不想郝氏自從遇了竹思寬。把個妙牝被他楦得其大無當。主顧一個不來上門。他也甚驚異。況且郝氏也還算不得很老。怎便爲人棄擲若此。他同郝氏雖名爲夫婦。因他以錢爲重。穿吃次之。屄爲輕的。素常也不甚與郝氏交合。一日。他疑心郝氏的此竅或有別故。故招攬不來主顧。偶然同他試試。孰意弄了進去。渺無邊岸。竟如一粟納之大倉。他方知閉門謝客者緣此。他撫着郝氏之陰。竟慟哭起來。郝氏驚問其故。他道。我仗你的這件東西做一個錢庫。滿心想做個財主。誰知弄得如此。如今門前冷落車馬稀。這財主是無望的了。叫我怎不傷心。說了。更放聲號咷大慟。郝氏由不得好笑。安慰他道。你不必傷心了。我的雖然沒用。目今女兒已長成人。有他接了衣鉢。將來這個財主不怕不是你做。你但放心。他聽見這話。方纔住了哭。他每日在白眉神案前焚香叩禱\footnote{龜子家所供白眉神。赤面虬鬚。白眉直豎。問之。云係柳盜蹠。但盜蹠當爲強盜祀之。何龜奴祀之。豈謂妓婦之心。於如強盜之惡耶。}。保佑女兒招財進寶。以遂初願。不想這不順親心的女兒。今又立志如此。大辜生平所望。除了他母女二人。別無掙錢之物了。這個財主只好看別人做。自己是無分的了。着了重氣。染成瘋癲。一日。走到朝天宮山後。竟跳在一個臭泥坑內淹死\footnote{錢爲命畢。}。這郝氏原也不以他爲夫的。不過借名而已矣。買了一個火皮匣盛貯。雇土工擡出城外。燒而棄之水濱。但他。

\begin{quotation}

旣無九肋能爲藥。又乏軀形可卜筮。

\end{quotation}

此等物何足道哉。那錢貴一日在書房中閒坐。正倚枕沈思。只聽得代目到跟前說道。姑娘。我纔在門首見賣的烈女傳小本兒的。我買了一本來。錢貴欣然坐起。道。你念與我聽。看是那裡人。是怎樣的烈女。代目念道。

\begin{quotation}

烈女杜小英。係湖廣辰州府諸生杜楷之女。母姜氏。夢見一女\endnotemark[3]子。絳衣執玉。再拜而吿曰。吾英臺女也。敢就母僦居。姜氏許諾。覺後有孕。及誕。即以小英字之。八歲。母舅愛其聰慧。授以閨訓。諸書一目了然。及讀木蘭詩並黃崇嘏傳。乃掩卷嘆曰。此二女不足以法也。夫以女子混跡男兒中。縱完身無玷。亦失貞靜之道矣。舅聞。大異之。及長。已字巨族。流賊張獻忠大寇湖南。將近辰郡。闔城人俱逃躱。杜楷攜舉家於潛避山中。官軍無糧。素無紀律。到處搶掠。婦女被擄者無數。小英於被一軍士搶到營中。欲犯之。小英號泣求死。誓死不從。軍士怒而懼。進上主帥。主帥好色貪淫\footnote{好主帥。主帥如此。軍士可知。}。一見大悅\footnote{明末之將大都此類。焉不被流賊所敗也。}。小英正色道。聖天子命將軍討賊以救黎庶。今將軍反縱士卒搶劫良家子女。與賊何異。不但將軍上負天子。下何以副衆百姓之望耶。妾以爲無知軍士貪淫劫擄。將軍定不知之。得見將軍。將軍定下令召人領回。今將軍反欲汚妾。不但威令何以督三軍。獨不畏人譏議耶。主帥於不怒。反大笑曰。自古道。佳人難得。我幸獲汝。且作目前之樂。死於何懼。人言何畏哉\footnote{好將軍。見一女子便不惜命。眞可謂朝廷之干城柱石。}。納於幕內。欲淫之。英詭辭泣吿曰。妾身已在此。尚何能辭。囊妾因母病篤。矢志茹素三年。今已兩載十月矣。倘蒙寬假。以完\endnotemark[4]宿志。不然。惟願速死。主帥心甚憐愛。許諾。旣而流賊過去。主帥挾小英回武昌。泊舟江滸。將及兩月。意欲犯之。英恐不能保全完璧。乃作絕命詞十首。自敍章首。內之油囊。貯於衣間。投江而死。其敍略曰。

\end{quotation}

洋洋洞庭。妾非不能死也。恐投之荒烟野水中。無有知者。則二親終不得我存亡矣。武昌省會之區。楚南賢士大夫多集於黃鶴白雲間。且當貢舉之年。吾郡應試。必多其人。故隱忍至此而死。希長者爲妾婦報高堂耳。

\begin{quotation}

其詞曰。

厭聽軍中唱凱歌。幾回斷腸嶺猿多\footnote{此二句無限悲鳴}。

將軍不下搜羅令\footnote{罪及首惡}。遮莫紅妝馬上馱。

其二。

淚痕濕透舊羅衣。夢到家鄕身未歸\footnote{讀之悽愴}。

滿目風濤誰是侶。低低遙祝兩靈妃。

其三。

舟師乍轉五溪津。載得佳人泊水濱\footnote{紅顏薄命。千古同悲}。

寄語雙親休涕泣。入江猶是女兒身\footnote{難得。眞烈女。}。

其四。

憶昔深居畫閣時。詩書曾就渭陽師。

於今飄泊干戈裡。猶夢挑燈讀楚辭。

其五。

生平十五未簪筓。自古紅顏福不齊。

河伯\endnotemark[5]有心憐薄命。東流逆繞洞庭西\footnote{果符其言。烈女有靈。}。

其六。

泣斷江聲怨亂離。永辭鸞鏡缺雙眉。

朱門空自聯秦晉。死後相逢總不知\footnote{傷心哉。}。

其七。

身雖如葉墜江邊。豈肯隨風逐浪圓\footnote{烈女之性。死於不變。}。

萬古不消天地恨\footnote{烈。}。幽魂只合化啼鵑。

其八。

滾滾江濤掩暮空。妾心寧與水俱東。

山川有恨家何在。誰爲招魂魚腹中\footnote{自有傳芳名者。}。

其九。

鬚眉雖愧奇男子。立志偏期豪傑儔\footnote{不愧女中丈夫四字。}。

完潔此身還碧落。江皋一任泣鵂鶹。

其十。

骨肉於今嗟已矣。承歡惟在夢中迎\footnote{死不忘親。非但烈而更孝。}。

貞魂即向家園去。歸報高堂已不生\footnote{讀竟不墮淚者。必無仁心。}。

旣死。逆流六十里\footnote{奇事。勿謂鬼神無靈。}。至荆口驛。土人撈屍得其詩。遍傳南國。讀者無不垂涕焉。

\end{quotation}

念罷。錢貴聽了。潛(潸)然流涕。〔道。〕爲女子者不當如是耶。我生不辰。出於煙花。身已汚矣。死於無及。雖失之於始。尚可悔之於終。倘異日得遇才郞。必當潔身以待。萬不可隨波逐流。笑殺多人也\footnote{入杜小英一段者。錢貴聽此之後。從良之心已十分拿定。}。終日眉頭不展。毫無笑容。一日獨坐。他母親郝氏到房中坐下。問道。我兒在此做些甚事。錢貴道。春色惱人。欲眠不得。無計消遣。焚香煮茗。供淸興耳。郝氏道。好有趣呀。我看你生得如此容顏。又有這些才調\footnote{先奉承幾句。好做巧說的引子。虔婆舌妙。}。老娘何福。得你爲女。遂滿臉堆下笑來。道。我兒。有一句話要對你說。你這樣聰明識字。決無拗我做娘的道理。錢貴聽道。母親有話。但請敎訓。郝氏道。兒呀。我們門戶人家。好容易得一個才貌雙全的女兒。別人家呢。還要千方百計覓來掙錢。何況你是我親生。反不着己。當初你七八歲的時節。人見你美秀異常。都說我家將來必定興旺。後來你雖不幸壞了雙目。如今看你的容顏。在姊妹行中也不能有二。做娘的在你身上。想圖一個小小富足。以娛老景。你想如今肯使幾個憨錢的人。定是王孫公子。濶老富翁。你如今只揀甚麼才貌。把這樣好主兒常常得罪了去。倒親近這些窮酸秀士。況從古來。但是有才貌的人。沒一個不是一貧徹骨的。就如女子中紅顏薄命是一理。古來這些有名的美人。有幾個嫁得才貌丈夫。你旣有此嬌容。已是薄命了。又想接標致才郞。如何能夠。你執意如此。叫我做娘的如何過活。且你只管如此任性。恐怕後來遇着作惡的呆公子。還要弄出禍來呢\footnote{伏後。}。故做悽慘墮淚道。你爹爹因你執性。氣成瘋癲死了。只有我在。你再執拗。我也不能久了\footnote{以死動之。}。你可替做娘的去想一想。錢貴道。娘言自是有理。但我生在娘家。今日做這等下賤的勾當。已是出於無奈。況天旣生我如此才貌。我豈可反不自惜\footnote{男子中有才貌而趨下流者甚多。見此語當愧殺。}。雖在風塵中。也要想一個出頭的地位。豈可終落火坑。如此結局。就是今日揀擇這些才貌兒郞。也不過是於中要選一個終身的夫婿。並非圖買笑追歡。風花雪月的行樂。那些膏粱紈袴。俗氣衝人。兒對之。每每欲嘔\footnote{大約是被鐵化熏怕了。}。豈肯圖他幾個臭銅錢。捨身屈意去奉承他。我係娘之親生。怎就不體愛孩兒。郝氏道。我視你如心頭之氣。豈有不疼愛你的。但你旣生在我這樣人家。說不得這些執拗的話。我如今並不叫你棄却才貌情郞。只留富貴蠢物。但要你彼此兼收。庶不寂寞。你說要圖一個終身之配。你是我親生之女。豈不願你得一個佳婿。但你年尚靑春。還可少待。況我方纔所說。才子配佳人。千古無多。一時如何能夠遂願。不過等待機緣而已。兒呀。你可知道占花魁上勸嫁的故事麼。錢貴道。兒自幼眼盲。未曾見過。郝氏道。趁今日家中無客。烹一壺好茶來。我對你慢慢細講。叫了個鍋邊秀的丫頭來。名喚財香。煮了一壺好岕茶。代目斟上。同吃了兩杯。郝氏便開口道。我兒。當初宋朝有一個宦家女子。只因避金人之難。被人拐去臨安。賣入煙花。更名王美。兒呀。說他生得就如你一般。姿容絕世。才藝驚人。故此都稱他做花魁娘子。他起初也不肯接客。定要從良。他娘央了個結拜的妹子勸他。道。你旣落在門戶人家。可是輕易跳得出去的。你說要去從良。固是好事。若從良不着。不若不從。你不如今日順了娘的意思。那做娘的自然愛惜你。況以你之才貌。自能傾動一時。且受用幾年。積攢些私房財帛。等遇着有可意兒郞。那時再嫁未遲。你若十分執拗。那時娘惱恨起來。或凌辱幾場。或轉賣別家。旣難跳出。仍要意從。豈不反低了聲價\footnote{雖是明說王美。却是暗指錢貴。其說眞巧。}。後來勸醒了他。竟自從了\footnote{郝氏一篇說話。重此二句。}。數年中聲名馳譽。掙了數千金之物。後選中了一個知心識意的秦小官。做了一對嬌滴滴的好夫妻。以完終身結果\footnote{錢貴之肯聽從者。乃重此二句。}。這是古人的事蹟。我兒。你想一想。若這樣效法做來。豈不兩妙。兒呀。只願你學他。就是我做娘的福了。再過三五年。替我掙下些錢鈔。那時憑你選一個情郞自嫁。可不是好。你若有了好處。我也還要從良呢\footnote{眞肉嘛(麻)。}。你多大年紀。就想遇着同心合意的情郞。我在這風月場中經歷了多少年。纔遇着個知心人兒\footnote{他這知心人。恐選遍天下。也難得此驢大的行貨了。}。兒呀。你談何容易。錢貴沈吟了一會。見他娘說得情理皆有。便說道。母親敎導。兒敢不依。但只是後來倘若選着才郞。我是定要嫁去的呢。郝氏道。乖兒。你旣聽我之勸。我可有不依從你的。但從良雖是好事。只要你自己拿得穩認得眞纔妙。若一時錯誤。後悔便難\footnote{這幾句却是良言。}。不是輕易的事。錢貴道。母親但請放心。孩兒自有主見。但母親那時不可失信。那虔婆見女兒依從了他。叫了幾千聲乖兒。許了幾百個肯字。歡天喜地而去。錢貴見娘去了。自己思量了一番。頗覺有理。自此以後。遇着呆公子。蠢富翁。俗濶老。腐科甲。雖不屈己奉承。也不似當時拒絕。這正是。

\begin{quotation}

明知不是伴。無奈且相親。

\end{quotation}

他無事之時。作了春夏秋冬四闋詞兒。道。

\begin{quotation}

春

傍花隨柳。雕輪驄馬。紫陌踐香塵。巧囀黃鸝。翻飛粉蝶。風景醉人魂。笙歌勸飮垂楊下。嬌鳥喚遊春。狼藉杯盤。玉山頹倒。歸去日西沈。

夏

彩鴛戲水。黃鶯織柳。庭樹盡濃陰。水閣榴丹。廻廊桐碧。風過覺微薰。方床石枕淸無暑。碧筒勸頻斟。瓜李冰涼。芰荷香滿。坐待月華生。

秋

寒蛩泣露。銀蟾吐月。萬戶搗衣聲。桂蕊飄香。菊英初綻。新釀醉花陰。金風簌簌驚黃葉。天際雁聲頻。玉燭淚流。金爐香燼。側耳聽殘砧。

冬

玉梅纔放。瑤花亂舞。朝野慶昇平。炭熾紅爐。歌揚白雪。紅粉侑金樽。樓臺似玉輕寒透。痛飮已微醺。膾鯉炮羔。淺斟低唱。莫負好靑春。

\begin{flushright}右調少年遊\end{flushright}

\end{quotation}

此調傳出去。人人皆羨他是才貌雙全的尤物。猶恐親之稍後。因此車馬闐門。絡繹不絕。他也漸漸積了些私財。以爲日後從良之計。這是後話。一日。有一個富家公子。姓祁名辛。慕他之名。特來相訪。一見了面。心愛非常。就送了三十兩花粉之資與郝氏。過了一宿。次日就替錢貴做衣服。製頭面。成大塊的銀子付與郝氏。每日預備極豐盛的酒肴。把個郝氏喜得屁滾尿流。錢貴見他豪爽可喜。雖不十分親厚。却也不像待那別個膏粱紈袴不得已的樣子。那祁辛一心愛上了他。毫不吝惜。時興各種的珠翠紬緞。無不買來相贈。過了數日。祁辛私向他道。我愛你不啻至寶。我素常聞得人說你一心有從良之願。你若不棄我。以我之力。爲你贖身甚易。你到我家。我當以金屋貯之。你意下何如。錢貴微微而笑。不答。又過了幾日。祁辛又道。我前日之言。乃心腹至語。你笑而不答。莫非疑我家中有正室麼。實不瞞你。我雖有妻有妾。前生未結夫婦之緣。名爲夫妻。實同陌路\footnote{輕薄小兒語。要知錢貴聽得此語。決不肯相從矣。}。你若肯嫁我。我當別置一室以處你。定以你爲正。豈肯屈做小星。古云。女爲悅己者容。我這一番情深向你。你難道竟無戀我之意麼。錢貴道。人非木石。豈不知情。承你垂愛。我深爲感激。況我旣身薦枕蓆。又何妨更掃箕箒。但你係貴介公子。我乃瞽目娼家。焉敢爲君家之配。我前之所不答者。爲此故耳。承君不棄。只可做煙花友。不能爲中饋婦。君其諒之。祁辛再三苦說。錢貴執意堅辭。這正是。

\begin{quotation}

落花有意隨流水。歸燕無心戀墮泥。

\end{quotation}

祁辛見錢貴決定不肯嫁他。也就興致索然。漸漸淡了。還留連了數日而去。有四句打油說他二人。道。

\begin{quotation}

莫認桃夭便好逑。須知和應始雎鳩。

世間多少河洲鳥。不是鴛鴦不並頭。

\end{quotation}

代目乘間問錢貴道。據我看。祁公子相貌也還可觀。家資旣富厚。又是貴公子。況且性又粗豪可取。待姑娘的情意也可謂親切之甚了。旣要替姑娘贖身。爲何堅執不肯。且姑娘又素有從良之志。失此機會。恐後來難遇這等有心人了。姑娘豈不憶魚玄機的兩句。道是。

\begin{quotation}

易求無價寶。難得有情郞。

\end{quotation}

姑娘尊意。令我不解。錢貴笑道。知人不易。難爲你言。祁公子人固可嘉。但心性非能常久者。且髮妻猶可棄。況於他乎\footnote{錢貴可謂深會其所厚者薄。而其所薄者厚。未之有也數句。}。我一會面。即知其爲人虛花輕佻。決不能保其始終。因他情意殷殷。較那肉食之輩差強。故不得不爲之周旋。豈終身之偶耶。我旣欲從良。必得兩意眞篤。方能保得能夫妻白頭相守。若只圖目前恩情富貴。將來不能善後。不但自悔無及。且恐笑破多人口嘴也。且他之愛我者非情也。乃愛我之色耳。古云。色衰而愛弛。異日將奈之何。我今日試說在這裡。你但記着。此人將來決不能有成。更不得有壽耳。我旣識之。復以身歸之。愚者猶不爲。而況於我乎。代目聽了。雖不敢與辯。深以爲不然\footnote{不可無此一番辯論。不然不足見錢貴之深心巨識也。}。話分兩頭。且聽我說這祁辛的出處並結果的事。便知錢貴的慧心了。我且先說些板(假)道學眞迂腐的話。做個引子。再歸到祁辛身上來。看官請聽。夫妻一倫乃五倫之始。有夫妻然後有父子兄弟朋友君臣。且古人云。妻者。齊也。夫妻相敬如賓。又云。上床夫妻。下床賓客。到了床上。那就不拘怎麼相戲狎罷了。當日張敞說。夫妻房幃之私。豈止於畫眉而已哉。別的話就可以不必言而喩了。至於白晝相對。自應相敬相愛。要說竟去跪之拜之。受其打也罵也。那却也無此理。然而把他辱之棄之。拳焉脚焉。視同奴婢。亦決乎不可。況妻與妾婢大不相同。婢字乃卑女。原是卑卑不足數者。即妾之一字。亦立女二字合成。不過比婢女之一道又略高些。其爲物也。原是取樂之具。可以放去。可以贈人。可以換馬。王將軍放妾。蘇東坡換馬二事。亦不必細說。單講這贈人的。馬鐸之母已生馬鐸。乃父念李姓好友無子。贈之。後生李騏。一妾從二姓而生兩狀元。千古奇聞。生子之妾猶可贈人。可見是不足爲重的了。至於妻子。要他生兒育女。爲宗祧之計。主持中饋。爲當家之用。何可十分輕賤得他。若把他當了一個可有可無之物。與妾婢一般。如何行得。我這一段話是要人夫妻和美。琴瑟相調之意。諸公莫錯會了。當是我勸人做那怕婆的好漢。譬如那人把他妻子十分作賤不堪。如寇仇陌路一般。離心離德。焉知那妻子心中又不懷別。念古來這些死節烈的婦人。雖是他的心如皦日。也必定是生平夫妻恩愛。情義甚篤。故願相從於地下。再沒有個兩口子素常如活寃家。朝打暮鬧。那女人肯去死節的\footnote{說的盡情盡理。}。豈但如此而已。我曾聽得一個迂腐老道學先生說。男人日裡看了他人之婦美。夜間與妻子行房。心念美人。借妻子之身以行樂。焉知那妻子不心中也想着美男子。借丈夫之身以行樂耶。此心尚不可萌。而況於棄其妻以私他人之婦。安得保其妻又不私於他男乎。我因要說祁家的事。故先說了這段熟話\footnote{雖是熟話。却是勸人夫婦和美的勸世文。}。言歸正傳。且說祁公子撇了自己的嬌妻美妾。去淫他人之婦。送了性命。反把妻妾被人去受用。還貼賠了一分大家私做了嫁粧。豈不可笑。當是這個膏粱公子。姓祁名辛。祖籍原是山東萊州府人氏\footnote{山東萊州府而來流寓。故後祁辛死時。別無一親戚也。}。他父親曾做湖廣黃州府知府。後因吿老。路過南京。愛這地方富庶。遂流寓於此。他父母已經亡故。他年紀未及三旬。他妻子莫氏就是黃州府同知之女。他一娶過門時節。那莫同知就陞了廣西梧州府知府去了\footnote{梧州府。妙。故後杳無音耗也。}。那莫氏生得也還有幾分姿色。但月下老人當日不知怎麼把赤繩繫錯了。把兩個寃家繫成一處。莫氏性格也還溫柔。不知何故。祁辛同他像有仇恨一般。只娶進門來。好了沒有幾日就相反目。那莫氏是個新人。不好同他相鬧。只得忍受。過了滿月。也就不肯十分相讓了。也就言悖而出者。亦悖而答敬。祁辛先見他不敢回言。以爲他的夫綱嚴肅。所以妻子畏而不言。發一會狠就罷了。今日見他嘴中不遜起來。那裡依得。竟掄其拳而飛其脚。不但搥其體而且嘴其巴。如此者數次。先不過是分床而臥。後來竟連話都不交談了。一對夫妻竟同陌路。祁辛賭氣娶了兩個妾來。一個姓須。一個姓有。都還生得標致。也只過了月餘。比待莫氏那個樣子還利害幾分。這兩個雖不敢與他相抗。不過是強笑強迎。假趨假奉而已。論起來。他夫妻大小都在少年。家中要穿有綾羅紗緞。要吃有美酒羊羔。出外堂上一呼。階下百諾。入內嬌妻艷妾。翠繞珠圍。眞是除了神仙淸幽快樂。就要算他繁華受用了。孰意這祁辛不知他是什麼奇異心腸。倒把家中之美棄了。專去外邊尋那閒花野草。他有一個窮朋友。姓何名幸。是一個少年飽學之人。生得人品淸秀。舉止端方。與祁辛曾同學念書。何幸仗着腹內文章進了學。祁辛虧了孔方之力也遊了庠。雖然各別。少不得算同案的朋友了。他二人年相彷彿。倒也來往得着實親厚。這何幸的肚中雖比祁辛通透。那祁辛的腰裡却比何幸厚實。何幸命旣不如他之豪富。且年將三十。小兒尚未有母。他母親當日在生時使的一個小丫頭。叫做葵花\footnote{又一個淫婦。}。生得不叫做美。那一種風流騷浪的態度。是他胎中帶下來的。非所學而能也。將二十歲了。何幸就把他收在身邊。也不說妻。也不謂妾。混焉而已。一日。祁辛到他家來尋何幸。恰好葵花在門口站着。祁辛一眼見了。魂靈兒飛去半天\footnote{此正可謂五百年風流孽寃。}。忙走到跟前。深深一揖。葵花素常在門縫之中。窗洞之內。曾見多次。雖認得是他。却未曾看得親切。今日覿面相親。見他那一種輕狂的體段。華麗的裝束。着實相愛。笑吟吟回了一拜。閃入門內。露着半個身子。說道。相公到此。有何貴幹。祁辛道。特來相尋何兄。不知在府上不在。葵花笑答道。不在家了。失迎相公。也虛讓一句道。相公請裡面坐。誰知這祁辛是調婦女的班頭。偷私情的領袖\footnote{有此兩句罪案。宜乎不得其死。}。見了葵花這個俏寃家。正無門可入。聽得讓他進去。巴不得這一聲。竟跨進門來。葵花只得閃身讓他到了內邊。滿臉的笑。重又作揖。葵花讓他坐下。自己在臥房門內站着。祁辛無可拔談。東扯西拽。說了些沒要緊的淡話。葵花毫不避嫌。也就一往一答的說了一會。祁辛只得起身吿別。葵花又送他出來。二人大有留戀光景。祁辛路上走着。心中想道。我同何兄相與幾年。竟不知他家有這樣個尤物。我看他大有綣戀之意。怎樣得個妙法。纔弄得他到手。想了一會。道。有了。須如此如此。不怕他不落在我的彀中。其計已定。歸家準備行事。且說那何幸回家。葵花對他說。祁辛來尋你說話。何幸不知是做甚事。就到祁家來。祁辛聽得。心中大喜\footnote{喜其落在彀中也。}。忙接了進來。書房中坐下。何幸道。適間失迎得罪。不知長兄賜顧。有何見敎。祁辛且不答。忙叫小廝拿上果酒來。二人對飮。然後說道。弟造府並無別事。因今歲大比。弟想做一做三場的工夫。癡心想一個進步。弟孤陋寡聞。苦無良師。素知長兄滿腹珠璣。欲屈長兄到舍下做一個益友。脩脯自不敢薄。府上的薪水都是弟這裡供給。吾兄也不必往返。就在這敝齋下榻。不知尊意如何。何幸的家中甚是寒薄。正要想潛心靜讀。以應秋試。但苦日用不繼。少不得要在外奔波。今聽他有這一番美意。可有不喜的。說道。弟才疏學淺。恐不能有砥礪之益。倘承不棄。敢不從命。但寒家無應門三尺之童。只有小妾在家。抵暮而歸。淸晨造府。也還不妨了功課。祁辛道。天時暑熱。設或再遇陰雨。來往也甚是費力的。因笑道。長兄若不能捨房幃之樂。弟則不敢強。若慮老嫂獨居無伴。舍下僕婦頗多。着一老媼到府上去。不但可以相伴老嫂。並汲爨之事都可以替老嫂代勞。長兄以爲何如。何幸道。雖承長兄如此見愛。但弟何以克當。祁辛道。我輩斯文骨肉。何必更做客套\footnote{昔人有云。此語出自其母。則爲賢母。若出自其妻。則爲妒婦。今祁辛此事若出於眞心待友。豈非君子。但出於不正。則爲眞小人矣。}。明日吉辰。弟有些微不腆之儀送到尊府。就打發個婆子過去。長兄把家務料理料理。也就請過來罷。何幸再三謝了。作別回家。把前話向葵花說知。他聽得有了盤費日用。而且又有人來替他燒茶煮飯。何等不樂。雖然夜間被底孤悽。日裡却得受用。再三慫恿。次日。祁辛送了十兩束脩並柴米之類到何家。又叫了一個能言善語的老婆子馬姓。附耳囑咐了許多話。到何家要見景生情。事成重賞。那婆子笑嘻嘻應諾。到了何家。何幸見祁辛如此用情。柴米銀子都有。也無可料理者。就到祁辛家中。謝了盛情。祁辛又設了一席。算入館的酒。二人談談講講。痛飮了一番。祁辛雖說納他來同念書。只早間一會。同在館中坐坐。飯後便說有事。不知何往。何幸也以爲他家業大。富貴人家應酬繁瑣。不好強他念得。且樂得三茶六飯的受用。潛心誦讀。且說那馬婆子在何家百般慇勤。不拿強拿。不動強動。連那葵花的淨桶也都去倒。葵花有得吃有人用。一日高閒自在。心中感激祁辛了不得。過了有四五日。祁辛到何家來。竟入到內中堂屋裡站着叫馬婆子。那婆子聽得是主人聲音。向葵花道。我家相公來了。葵花前次見過他的。也不害生。就走到房門口相見。祁辛忙作了揖。說道。我纔出門拜個客。在尊府過。因何兄不在家。恐怕尊嫂家中少長缺短。我心裡記掛。着時進來問問。葵花道。前日承府上送了盤纏柴米。拜領感謝不盡。不差甚麼東西。不敢勞費心了。祁辛道。我同何兄多年契厚。就是同胞弟兄一樣。與尊嫂也似嫡親叔嫂一般。彼此通家。怎還說個謝字。尊嫂若少甚麼物件。只管吩咐。我無不奉命。本當請尊嫂到舍下走走。嘆了一口氣。說道。但我這個賤內是死人一般的。不會知人待客。若像尊嫂這樣和氣。早請去會會了。因吩咐馬婆子道。你小心服事何奶奶。就像伺候家中奶奶一樣。不許懶惰。要是少甚麼。就回去對我說。說罷。辭了出來。葵花與何幸雖然夜間爲妻子。日裡仍是爲婢的。今被祁辛這一番奉承。自己尊貴了許多。覺得心窩裡都是快活。又見他話中帶着憐愛。不但感激。竟動了一點相愛之情。那馬婆子見主人又吩咐了幾句。更加勤謹。葵花一日偶然同他閒話。問道。你家相公說你奶奶是個死人。是甚麼緣故。馬婆子道。這總是各人的緣法。我家奶奶也不叫生得醜。頗有幾分姿色。〈個〉夫妻兩個不知是甚緣故。總不同床。還有兩個姨娘生得也好。也不中他的意。三日吵兩日鬧的。前日在家裡同奶奶拌嘴。相公說道。我前世不曾修。今生娶了你這樣個老婆。像何家那嫂子。見人又和氣。說話又能幹。我要娶了這樣個婦人。眞正頭頂着他過日子\footnote{上頭頂乎。下頭頂乎。此話難解。}。我的命薄。可惜就沒有這個緣分。我前日來時。再三吩咐。叫我小心服事奶奶。說你這樣個嬌嫩人兒。如何做得粗重生活。又罵那兩個姨娘道。你們這樣東西。揷金戴銀。穿紬着緞的受用。我看何家嫂子那樣人物。布裙荆釵。家中無樣不是自己去做。眞是老天沒眼。我想起來。好不叫人心疼。大約他心裡記掛你。故此昨日又來了看看\footnote{此媼可謂利口。先以情義動之。次以富貴感之。繼以憐愛感之。婦人水性。焉有不動心者。雖是受主人之托。然壞此心術。後之一死。亦爲不枉。}。實實是我相公沒緣。若是有緣。娶了奶奶你這樣個心上人兒。還不知怎樣恩愛呢。葵花聽了。呆了半晌。說道。那是他沒緣。是我沒修了這樣的福來。婆子道。說起來也奇。我家相公因同奶奶姨娘不睦。成年在外做這些偷情的勾當。也相與了好些婦人。從沒聽見他誇獎一個有得意的。前只見了奶奶一面。上口不念下口念。刻刻在心。像是有些緣法罷\footnote{此婆之口可畏。見葵花呆了半晌。知其已爲所感。乘空便入。又將此語誘之。眞善說。}。葵花道。今生不中用了。修得好。來世同他結個緣罷了\footnote{此話已明明心見矣。}。那婆子見他這話來得有些因頭。便嘻着臉說道。奶奶。我說個戲話。你不要見怪。我看他這個愛你的心腸眞是沒有的。何不兩下暗暗成了姻緣。要甚麼穿的戴的他不送你\footnote{更進一步。}。葵花笑笑。也不作聲\footnote{此一笑。已是千肯萬肯矣。}。婆子見有幾分光景。又逼一句道\footnote{惡甚。}。奶奶。少年夫婦誰不做些風流事兒。從沒聽見貞節牌樓蓋在那有丈夫不偷情的婦人門口\footnote{奇談。可駭。}。葵花初見祁辛時。心中也就有些愛他。今聽見婆子說他這些相愛的話。更動了知己之感。嘆了一口氣。那馬婆子見他也有些活動了。便道。奶奶。你請自己坐坐。我回家去取點東西來。葵花道。你取甚麼東西。馬婆子道。這兩日天氣熱。身上有些汗酸臭。我取兩件衣裳來換換。設或我來遲些。奶奶只管把門掩着。你但請安歇。我是必定來的\footnote{詭譎可畏。不如此說定。恐關了門。晚間同祁辛來時。葵花出來開門。或看見也。}。說着。就去了。到家把前話向祁辛說知。便道。等夜晚些。我同相公去。悄悄進他房中。竟硬做起來。大約他也情願。祁辛大喜。到了天黑。同馬婆子一路到了何家門口。婆子推了推。門是掩着的。推開。同祁辛進去。關好。房中也不曾點燈。葵花已睡下了。婆子道。奶奶。你睡着了麼。連燈也不點。葵花道。等你到晚。不見你回來。自己一個人心裡怕怕的。我就上床睡了。我還怕你不回來了呢。婆子道。我可有不來的。因相公問奶奶這裡家長裡短的話。說了半日。故此來遲了。葵花道。問你些甚麼。婆子道。話長呢。蚊子咬得慌。奶奶你不嫌棄。我到床上細細的說給你聽。葵花聽說祁辛問他。不知說些甚麼。正要問問詳細。便道。也罷。你進帳子來罷。那祁辛忙脫光了爬上床。同他一頭臥下。就伸手去摸。因天熱。葵花也是上下沒一根絲。祁辛不由分說。上了他身子。緊緊摟住。葵花只當婆子合他戲耍。遂笑道。媽媽。你癡了麼。話還未了。已被他直抵紅門。忙問道。你是誰。婆子在帳外道。是我家相公。因怕奶奶府上沒人。特來與奶奶做伴的。那葵花將錯就錯。便不做聲。被他着實高興了一度。二人千般旖旎。萬種溫存。重整旗鎗。又大戰了一場。葵花每常何幸間或同他如此。不過是古板正傳抽弄一會。適興而已。並無奇異的做造。這祁辛是此道中的慣家。弄得葵花意亂心迷。身搖股湊。不能自主。事畢。摟抱而臥。講說的無非是相思相慕。相憐相愛的話。兩人睡至天明。猶戀戀不捨。看看紅日三竿。只得要起來。還摟抱着親熱了一會。方纔別去。此後別沒三日不(必)來。那何幸是個書呆。一心要想成名。在他家苦讀。況家中柴米盤費都有。無內顧之憂。且葵花何幸原也不把他取重的。因家中又有那馬婆子。他也不便在家中過夜。只十日半月間或日裡回家看看。問問家常。就去館中高坐。祁辛也同葵花走動多次。夏盡秋來。被一個前生寃孽看見了。你道是甚麼人。這個人姓暴名利。是個凶頑惡棍。見財貪財。見色就愛色的人。就與何幸緊鄰。你道他生得怎個模樣。

\begin{quotation}

一臉橫肉。滿面疙瘩。色似羊肝。腮如豬肚。唇上倒豎幾莖黃鬚。鬢邊蓬鬆數根紫髮。純乎戲臺上扮出魍魎。宛然廟門首塑的惡鬼。

\end{quotation}

他每常見於葵花獨自在門口閒站。他知何幸軟弱可欺。就想去勾引他。嘻皮笑臉。做出那風流調情的樣子。他若生得略似人形。或者葵花也還肯苟就。這樣三分似人七分像鬼。醜騾乍見了還要體戰心悸。婦人中可還有愛他的。常被葵花大罵也多次了。葵花吿訴何幸。何幸道。那種人同他一般見識做甚麼。你只不到門口去便沒是非\footnote{此語妙極。一婦人在街上步行。一男子目不轉睛看之。此婦怒曰。各人走路。你看我怎麼。那男子笑道。你若不看我。怎麼就知道我看你。葵花若自己不出來。暴利焉得而調戲之。}。也就撂過一邊。這些時。暴利見何幸總不來家。那祁辛暮來朝往。他醋氣大發。怒道。這淫婦。我想相與相與他。他就做張做致。假撇淸不肯\footnote{假撇淸三字。葵花不能辭。}。也還情有可恕。你罵了我不知多少。就該貞節到底\footnote{這責備得甚是。}。今日在我眼皮子底下偷有錢的漢子。明明的囂我。我叫你試試我的手段看\footnote{這一轉念。是便惡棍心腸矣。}。今晚這廝若來。我悄悄過去綁上了他。不但囮他一大塊銀子使。且借此囮這淫婦。弄他一個痛快。弄過之後。將來就不怕他不是我的一個外宅了\footnote{初心不過如此。原非有仇欲殺。後殺之者。激之使也。敍事有先後輕重。妙極。}。又想道。恐他們不怕。我帶了刀去唬嚇唬嚇。也不敢不受我的挾制。拿過切菜刀。在石上磨了磨。磨去了鏽。亮錚錚的。天色將晚。看見祁辛進他家去了。約將三鼓。他腰間揷了刀。此日正是七月十五\footnote{七月十五者何。一則點明前夏去秋來句。二來俗謂中元放鬼。今日七月十五。故有此惡鬼來行凶也。妙甚。}。月明如晝。他越牆而過。見房門關着。推了推。如鐵桶相似。就去掇門。用得力猛掇下一扇。那一扇向地下一倒。劃剌一聲大響。把葵花祁辛一齊驚醒。原來他二人掛着帳子。點着燈。照着大幹。搏弄了半夜。都乏倦了。方纔合眼。被這一驚。一睜眼。見一個人站在地下。葵花慌忙坐起。連聲大叫有賊。暴利又是那氣。又是那急。拔出刀來。上前儘力一下。葵花臉上正着。尚未砍死。倒在床上。兩足亂蹬。那祁辛驚得要死。下床不及。也叫道。殺人了。說猶未了。也被一刀砍着。就跌倒了。便不做聲。有四句說他們道。

\begin{quotation}

忿激凶怒動殺心。奸人被害却緣淫。

持身正直邪淫斷。暮夜應無禍難侵。

\end{quotation}

那老婆子一板之隔。聽他二人響動了多時。方纔寂靜。一時老興勃發起來。摸了一個搗蒜石杵。睡在榻上。扯開褲子\footnote{不脫褲者。以便少刻提着好跑。此等沒要緊處。亦必細心寫出。}。正然一出一進的搗。纔有些趣味。先聽得響了一聲。正在吃驚。又聽得葵花叫有賊。後聽得主人叫殺人。撂了石杵。連忙爬起。一手提着褲腰要往外跑\footnote{嚇慌。拽不及也。}。暴利攆了出來。馬婆子跪到天井中。回頭一看。月下認得是他。說道。是你麼。暴利道。也饒你不得。剛舉起刀來。那婆子腿嚇軟了。一交撲倒。暴利夾脖子也是兩下。見那婆子不動。以爲死了。復進房來。見兩個屍首都精光着。他拿燈照了照葵花的下體。笑道。你這淫婦。活着不肯給我弄。我且肏個死屄。着將葵花的身子放正。他還淫媾了一番。方踰牆而回\footnote{余見書中赤眉賊淫呂后屍一事。一死屍也。尚何有此高興。不知此輩是何肺腸。}。暴利行凶時。他那切菜刀先砍了二人。已鈍缺了。及至砍那婆子時。他也心忙。雖然砍了兩刀。又在脖子上。只疼昏了過去。尚未曾傷命。到天色將明。甦醒過來。掙着爬起。拽上褲子\footnote{一絲不漏。}。進房看時。兩個都赤條條的。主人頭顱兩半。葵花額鼻平分。俱殺在床上。血濺滿處。他只得掙着開門出來。悄悄報與鄰舍。衆人約了地方總甲。一齊到暴利家來。他正還睡覺\footnote{好放心。好受用。}打進門去。血刀血衣俱在。還有何說。將他綁縛了送往縣衙。那馬婆子先倒還掙了起來。此時反又昏迷了過去\footnote{此一部書。總不越情理兩個字。即寫此等沒要緊處。亦情理所必然。所以爲妙。}。只得拿塊門板將他擡着同到衙門。知縣聽見是殺人公事。連忙陞堂。地方街鄰上去稟了。知縣先問暴利這事如何起來。暴利將他二人通奸的話說了。道。小的係緊鄰。因何相公不在家。小的替他殺奸\footnote{奸那是替殺得的。寫無知凶徒強辯處。妙。}。知縣笑道。奸固可殺。但你非殺奸之人。你圖囮奸是眞。後至於殺死二命。則非爾之本意。可是麼\footnote{這知縣可謂片言折獄。}。暴利被他一句話說着了心腹。無言可對。知縣喝道。你還不實招麼。取夾棍上來。暴利知道是不能免罪了。徒受刑也辯不出。把從前引誘不從。以至後來他二人通奸。本意囮詐。不想他二人叫喊。只得殺害。從實招了。知縣命畫了供。打了二十板收監。知縣又問馬婆子奸自何時起。何以得成奸。他親夫知情不知\footnote{問得細。}。婆子將主人如何誘何幸到家讀書。如何叫他引誘葵花。如何成奸。他丈夫並不知情。也細說了\footnote{婆子不殺死者。留爲此處用耳。不然這些詳細。他人如何得知。看者勿爲作者所瞞。認眞是切菜刀鈍。不曾殺死。未免爲作者暗笑也。}。知縣嘆道。誘人夫而遙(淫)其婦。有玷黌門。一死何惜。吩咐典史。帶忤作相驗兩屍傷痕。以便呈報。本夫不知情。不究。兩屍各家領埋。馬婆子雖奉主人之命。不該引誘良家婦女。以致殺傷二命。本當重處。姑念身受重傷。免究。着本家人領去扶養。馬婆子祁家人領了回去。次日即故\footnote{話已說了。用不着他了。}。也報了知縣。定暴利的罪。引殺一家非死〈罪〉三人。律剮。他三人雖非一家。但暴利欲囮奸而致殺三命。罪應加等。剮不爲過。申了上臺。達部。准了下來。暴利一剮。不用多說。何幸回家。雖恨葵花淫賤。念他數載勤勞。要存厚道。買了一口棺材裝了。雇人擡去埋葬。莫氏將祁辛的屍首擡回。製棺入殮。延僧道念經。那些熱鬧生人眼目的事。少不得都要做。買墳地。做紙扎。開喪出殯。十分體面。莫須有三氏寡居了一年。他夫妻俱係外省人。並無一個親戚。又年少無出。夫妻做了幾年寃家。還守甚麼。思量要贅一個丈夫做個倒蹋門。恐一時不得其人。又似前夫薄倖。那怎麼處。因想起何幸來。家人素常都誇他老實。婦女們又說他相貌淸秀。莫氏就動了一點相愛的心腸\footnote{夫愛彼之妻。其妻即愛彼妻之夫。毫厘不爽。}。又是丈夫故交。情願嫁他。倒煩人去替他講這親事。何幸先還不肯。說。古人道。朋友妻。不可欺。朋友妾。不可褻。他雖不仁。我同他相與一場。今日如何好娶他的妻子。衆朋友知道。勸他道。你不要太迂了。你要去謀占他的妻子則不可。今日他情願明公正氣的嫁你。何不可之有。他欺你。偷淫你家的人。你今日做個鳩奪鵲巢。也不爲罪。衆人慫恿他。竟成了秦晉之好。何幸一介寒儒。今日忽來享妻福。華其衣而美其食。呼其奴而使其婢。且又是極美的妻子。雖然不到勢怕的地位。也着實相敬相愛。莫氏同祁辛仇敵一般。今見他如此溫存。也十分相得。何幸當日同葵花半妻半婢。原沒有伉儷之樂的。今遇莫氏這等恩愛。二人方知世上夫妻有如此之恩情。莫氏身已有主。要須氏有氏改適。他二人見何幸待大奶奶如此情厚。大約決不忍薄了如夫人。況且嫁去。又不知良人心性如何。也情願嫁與何幸。莫氏同他二人相伴久了。也捨不得相別。見他們不願去。心中也甚喜。勸何幸也並納了\footnote{祁辛偷淫何幸之婢。以爲是得便宜。孰不知妻妾皆明歸於何幸。便宜安在。何幸固然何幸而得此。祁辛亦可謂之奇心也哉。}。何幸後來走了幾科。再不得中。終身一儒。大約也是娶朋友妻妾。享朋友家產之故\footnote{又是喝棒。}。雖非他圖謀之過。未免隱微中傷了些德行\footnote{此書於報應二字。毫末不肯放鬆。令人不寒而慄。尚可謂之淫書耶。}。雖不曾中。却也享福終身。一妻二妾。皆生有子女。後來竟成了一個巨室。這又他做人端方好報應。可笑那祁辛。撇了美妻艷妾。反去戀那葵花。以致喪身絕命。不知是何心腸。正是。

\begin{quotation}

祁辛眞是奇心。何幸誠然何幸。

\end{quotation}

這一段事。費了許多唇舌紙筆。說了這一會。雖與正傳無干。一來也是一番大報應。二來可見錢貴之慧心卓識。一瞽目女子。初相會便知人之終始。龜鑑若此。把世上有眼男兒一齊抹殺\footnote{因此數語。所以纔有此一部大書也。}。後來錢貴得知祁辛的這一番事。想起他的舊情。慘嘆了幾聲。因向代目道。我向日之言何如。代目道。姑娘眞好慧心。我輩淺人。如何得知。暗暗心服。且說那鐵化之妻火氏。自從得了狗舌之樂。總不許鐵花(化)沾身。那鐵化也躱在外邊。成半年也不敢見他的面。他有個心腹丫頭。叫做巧兒。聰明伶俐。善能體貼火氏的心腹。所以火氏愛他如親生女兒一般。時常帶他一床同臥。以消寂寞。他看見主母喜。也就做個喜顏相對。主母憂。他也是滿面愁容。見主母時刻氣恨。知是爲主公\endnotemark[6]之故。他無話也謅出些話來。時常說說笑笑。解主母的愁顏。因而火氏更加疼愛。偶然叫他打聽鐵化在外面做些甚事。他打聽明白了。一五一十。全全奉吿。說主公在外如何貪嫖。今日張。明日李。並不歸家。要不嫖。就在賭場中取樂。火氏聽了。切齒怨恨道。結髮夫妻身上萬分躱懶。一毫情意也沒有。撇了不理。倒去貪嫖\footnote{獨不思結髮夫妻身上一點情意也沒有。倒同狗取樂。你可以同狗樂得。他也可以嫖得。}。他旣然可以嫖得。我也可以嫖得\footnote{好嫖者留心乃政。}。當初礙着小姑戳眼。如今只我一人在此。就嫖嫖也無人知道。心中雖如此想。却無可嫖之人。心中想上火來。便到樓上去。且拿狗舌解釋\footnote{無可嫖之人。且嫖狗。}。一日。在房中正然胡思亂想。忽聽得西屋裡幾個僕婦在那裡說笑。他走到堂屋中來聽。只聽得說長說大。嘻嘻哈哈的笑成一堆。說不明白。也聽得不眞。他走將進去。衆人見了他。都迸(繃)着笑臉。便不做聲。火氏問道。你們在這裡說甚麼。這麼好笑。衆婦道。大家講閒話。沒有說甚麼。火氏道。我聽見你們說說笑笑的。有話說罷了。怕甚麼。內中一個僕婦指着一個說道。他剛纔見了個稀奇的東西。嚇掉了魂。在這裡吿訴我們。所以大家在這裡笑。那一個笑着瞅了他一眼。道。你們難道就沒有說句把兒村話。單是我說來。火氏動疑。道。你見了甚麼。怎樣好笑。快快說來。那個僕婦見追問得緊。只得笑說道。我纔纔到毛廝上去倒淨桶。不防每常在我們家的那個竹相公在那裡溺尿。撞了一個滿懷。他的那個東西軟叮噹的。還有八九寸長。鍾子口粗。就像驢㞠子一樣的。要是個硬起來。還不知有多大。纔在這裡同他們說笑。人身上怎生這樣個硶東西。虧他的老婆怎麼捱來。量一量。差不多頂過了心口。我想女人遇了他。不搗斷腸子弄死了\footnote{此一語爲火氏將來結果之讖。}。也要穿裂了陰門\footnote{此句爲火氏初試之先兆。}。火氏聽說得好生動火。又笑着追問道。他們又說甚麼村話。這個婦人指着一個道。他說要遇着這東西。慢慢的也還弄得進個頭進去。又指着那個道。他說要吃四兩燒酒還捱得半截。火氏也笑了一陣。那巧兒丫頭也在傍邊聽着。嘻嘻的笑。那個僕婦道。丫頭家不害羞。你笑甚麼。他纔跑了去了。火氏回到房中。半晌不做聲。想道。我家忘八這樣沒良心。我走走邪路也不爲過。這老婆子方纔說的話。料未必扯謊。若相與了他。不枉捨身一場。如果有這樣一個大東西。豈不又強如那狗舌頭幾倍。只是怎麼得會着他。有四句寫那火氏的心事道。

\begin{quotation}

嫁夫莫嫁此無徒。嫖賭齊行私婢奴。

我今也學乖伶俐。各自相交小丈夫。

\end{quotation}

火氏想了一會。道。這事瞞不得巧兒。須得他做個牽頭。纔可遂心。叫巧兒同他上樓去。叫他樓門關上。誰知那狗見主母上樓。他就先跑了上去\footnote{一絲不漏。}。火氏到樓上。在椅子上坐下\footnote{此時方用着椅子。}。對巧兒道。我有一件事要托你。你不可洩漏纔好。巧兒道。奶奶的恩典這樣待我。我可敢走洩。火氏欲言又止。巧兒知他疑心。忙說誓道。奶奶疑我麼。我若不盡心替奶奶做事。要洩露與人。後來遭刀砍斧刴。一世沒有漢子\footnote{好狠咒。}。火氏見他發誓。知他實心相爲。遂拉着他的手。臉紅着道\footnote{善於寫生。臉紅着三字入神。是初學偷漢淫婦。羞惡之心尚未絕滅。}。我這樣年少靑春。你主子總不顧我。他旣沒恩情。我也可以有得外遇。方纔說的這竹相公。我心裡要想會他一會。除非你做個引進。你可肯麼。你若替我做成了。後來我揀個好人家嫁了你去。還厚厚的賠嫁。報你的情\footnote{重賞之下。〈不〉必有勇夫。火氏即此意。故厚餌之。}。巧兒說道。這是奶奶的恩典了\footnote{他先發誓時重在第二句。此却在頭一句。}。我每常見爺這樣沒良心。不要說奶奶氣。我也在這裡成日的氣呢\footnote{趕脚的也來哭。}。但只是他們方纔說得怕人子剌剌的\footnote{先寫巧兒也在傍邊聽了笑笑。此一句甚覺無味。此方見他先聽得之妙也。}。奶奶不是當頑的。另尋別個人。小巧些的好\footnote{是個小丫頭說話。}。火氏微微的笑道。呆子。旣是這麼說。難道他一生就沒見個女人麼。總不過是皮肉。一個受得。個個都受得。況且別人又往那裡去尋。巧兒道。旣然這樣說。如今奶奶的主意叫我怎麼做。我就依着行。決不誤事\footnote{活畫出一不知事獻勤的小丫頭來。}。火氏歡喜得了不得。道。此時大約竹相公同你主子在前邊吃酒。今日說不得別的話。我拿件東西。你看巧沒人。悄悄遞與他。同他約下。若你爺明晚不在家。千萬叫他來。多話不用說。恐人聽見。他要是明白人。自然懂局。巧兒道。這事有甚麼難\footnote{此原是乖巧丫頭長技。}。等我去。奶奶你拿甚麼送他。可交與我。火氏將臂上金鐲除下一隻來。用一條大紅縐紬汗巾包了。遞與他。道。好好藏着。萬不可與人看見。小心在意。再三叮嚀。巧兒接了。興興頭頭而去。火氏每常一上樓來。就脫衣叫那狗舔。今日上來同巧兒說了這一會的話。那狗急得圍着他。搖着尾巴亂跳。不住用口扯着裙子。有個要他上床之意。火氏先說話時已看見了。此時巧兒已去。見那狗急得好笑。把門閂了。恐巧兒來撞見。不脫衣服。在小床坐着。要褪褲子。那狗等久了。急得把頭儘着往褲襠中亂鑽。火氏想竹思寬那又長又大的驢腎久了。也火動得很。忙脫了褲子臥倒。那狗如得了寶貝一般。你看得那好舔。舔得那火氏酥麻了一會。恐巧兒來回信。要推開他起來。那狗舔興正濃。那裡肯歇。火氏只得又讓他舔了一會。然後起來。穿好了褲。開了門。坐着等候。不一時。只見巧兒笑嘻嘻上樓來。火氏忙問道。事體怎樣了。巧兒道。事有湊巧。這是奶奶的洪福\footnote{此事亦謂之曰福。奇談。}。我剛到外邊。一個人影也沒有。恰巧竹相公走出來。想是要溺尿。見了我。撤身就要回去。我低低的叫住他。把東西遞與他。把奶奶的話悄悄向他說了。他打開看了看。藏在腰間暖肚裡。歡喜得了不得。他道。多上覆奶奶。我明日把你爺哄在外邊過夜。我一定來。說着。聽見大爺說話。他忙忙進書房去了。火氏聽說。滿心歡喜。拍着他肩背。道。好孩子。這樣中用。不枉我疼愛你一場。巧兒道。奶奶恩養我們的。這點事若做不來。還要我做甚麼\footnote{奇語。主母恩養丫頭。原來是爲此。}。遂下樓歸房。以俟明夜佳期。且說竹思寬在嫖賭行中過了半世。甚麼事不知道。見火氏送了他這件東西。知道是做表記的。心中暗喜。進書房中同鐵化吃着酒說着話。心內想道。我雖然遇過些婦人。都是妓女。那陰戶俱是經過千百人陽道的。却從不曾見過良家婦女之物是怎個樣子。因爲我這東西過當。也不敢去尋人。今承他這番厚愛。且又聞他生得標致非常。得會他一會。就做着弄不得。且見見這樣妙人兒的妙物。也是造化。須將老鐵騙出去耍錢。纔好行事\footnote{好賭者當防之。}。想了想主意。便道。大爺這幾日怎不到屠家去耍耍。鐵化道。前日你看見的人。旣不對樁。又沒有大錢。倒把我輸了兩場。總沒有個好主兒。耍得一點興頭也沒有。竹思寬道。昨日他家局子裡有幾個人。都是外路來的。我看他都是些雛兒。成千家銀子拿着。我因沒有現梢。不敢下場。大爺何不明日去贏他些來。翻翻前日的本。鐵化道。說是這樣說。輸贏也是定不得的事。竹思寬道。只怕短歇就沒法了。上場時說明了要耍一夜。頑長了。到了夜間。大爺弄些本事出來。怕不一股擒之。鐵化心中大悅。道。明日我同兄去。竹思寬道。明日上半日我有些小事。大爺請先去。下午些我來奉陪。又飮了幾鍾。辭別去了。次日。鐵化帶了幾百金到屠家賭局來。果然有三個江西木商在那裡。正少一把手。屠四見了鐵化。大喜道。爺來得好。我正要煩老竹去奉請。因他兩日不曾來。這三位都是現梢。大爺頑頑。鐵化道。我因爲昨日聽見老竹說的。故此今日帶了銀子來。先要說過。要頑除非長局。正正經經見個輸贏。頑個通宵。我纔來的\footnote{你在此要頑個通宵。那知令政在家亦要頑個通宵也。不知事少年當深戒之。}。那三個道。這位爺說的是。夜局更妙。說定了。擺下壇場。就擲起來。再說那竹思寬自鐵化家出來。要打點明晚行事的。便不到屠家。恐次日鐵化去。掛住了身子。便到郝氏家去宿。他因心中想着火氏。將郝氏之軀當他。足足弄了半夜。因困乏了。睡到次日已飯時纔起來。日色將午。他到屠家門口。打聽鐵化已來了。上了局。喜不自勝。到各處去閒撞。捱到天色已暮。到鐵家來。已將關門。故意問看門的道。大爺可在家。門上人道。大爺從早間去的。此時不回。大約是不來了。竹相公此時來。有甚麼話說。〈做〉〔竹思寬〕故意咨嗟道。我尋他有要緊的話說。不在家怎麼處。遂走到書房裡。道。我在此等等罷。那家人道。恐今曉不回來。天黑了。怕等不得。竹思寬道。我有要緊的事同他商議。定要面會的。他就不來。我在這裡過夜。明早他必定回來。家人都知他是主人的厚友。常常來往。過宿也是常事。便道。旣然相公在這裡。我去點燈。叫收拾晚飯來。竹思寬道。我吃了飯來了。你只點燈來罷。須臾。點上了燈。竹思寬道。你們都請去安置。我自己在這裡睡了。不用人做伴。家人們見主人不在家。落得去受用。都各回家高臥去了。那火氏昨日聽得巧兒說竹思寬許了今日必來。猶恐鐵化在家阻了好事。不住叫巧兒打聽。早飯來說鐵化帶了銀子賭去了。心中一喜。還怕他晚上回來。到了日落未回。知道在外過夜。越發放心。但不見竹思寬來。正在憂悶。只見巧兒一臉的笑走進來。到耳傍悄聲道\footnote{眞伶俐。}。竹相公來了。要在書房過夜。等爺明早說話呢。火氏知是假圈套。喜不可言。想道。如何得他進來。又想了一想。道。不好。還是瞞了丫頭們。我悄悄同巧兒出去爲妙。原來鐵家的房子正樓五間。廂樓六間。獨院獨門的。門外橫隔一條小巷。面前就是大廳。廳院東邊有一個小圈門。進去又一個獨院。三間書房。後邊也是一個院子。前後都有假山花木。廳後那條巷。東西儘頭處都有角門。西邊角門通着廚房衆家人下房。東邊一個小角門通着書房後院上房。出來就不走大廳。從角門直達書房。甚是便宜。火氏叫巧兒去。若沒人。可通知竹相公。叫他關了前院門。把後邊角門開了。等夜靜些好出去。你來時。可就把大廳門同西角門閂好。巧兒出去。一個人也沒有。他對竹思寬說了。進來把兩處門都閂好。到房中悄悄回了火氏的話。火氏雖有三四個丫頭。只巧兒在他屋內睡。別的都在西屋。他此時淫念一動。坐臥不寧。心中好不難過。只把頭梳了梳。將牝戶用香肥皀挖洗了一番。老早吩咐丫頭們都去睡覺。他也故意上床假睡。那些丫頭是巴不得的。每常主母坐着。還要偷空去睡。何況主母吩咐。可有不睡之理。倒下頭就如死人一般。火氏叫巧兒聽聽丫頭都睡熟了。下床同巧兒出來。帶上房門。輕輕開了堂屋門。也反帶上。趁着微月。開了院門。也帶好。順着東邊小巷。走到書房後角門來。輕輕推開。二人進了門。閂好。到書房中來。竹思寬正坐等。專候仙姬降世。神女臨凡。側着耳聽。夜靜了。隱隱似有婦人高底聲響。忙走出來一看。月光下巧兒扶着一位美人來了。歡喜欲狂。忙讓到房中。竹思寬忙把燈剔亮了。將他一看。眞好一位風流標致的女郞。也不梳妝打扮。他是安心出來做一番大生活的。頭上緊緊挽了一個蘇纂。結結實關着兩根金簪。穿着隨身大紅縐紗。窄袖襖兒。鵝黃絲紬裙子\footnote{是個回回家婦人的打扮。}。手中控着一條白紬汗巾\footnote{只道他拿來揩嘴。原來是預備揩那個的。}。他雖是一個淫浪婦人。一來年幼。二來乍見生人。未免含愧。臉上一紅一白。竹思寬見了這段嬌羞。魂都沒了。忙作了揖。道。我有何福。敢蒙奶奶這樣見愛\footnote{看了許久方纔作揖。是渴想極了的樣子。神情逼眞。}。如何纔報得這種深情。那火氏只回了一拜。並無言可對。竹思寬也忍不得了。一把抱到床上。替他寬衣褪褲。他也並不裝假推辭。臉紅紅的微微含笑。兩眼半閉半睜。任憑脫去。見他一對小小金蓮。穿着靑緞子高底花鞋。白綾褶褲。大紅絲帶。他自首至足。燈光照着一身雪白光滑精肉。眞個消魂。竹思寬也忙忙脫光。火氏心中想他那件物事太大。有些害怕。悄悄向他耳邊道。聽得說你的東西大得很。不可冒失。探起身子將他一看。竹思寬見了這尤物焉不動火。早已直豎着一根大肉棒槌。火氏見了又愛又怕。嬌聲道。只怕放不進去。不是兒戲的。竹思寬摟着親了個嘴。道。親親。你放心。我自然有法子。你不要膽怯。將他扶正了睡好。竹思寬知他這件傢伙。除了郝氏的巨牝。再沒有對子。後雖遇過昌氏。那是婦人中的異物。不可比列。今承他厚愛。不得不同他試驗試驗。見他生得這等嬌嫩。可敢造次。先縮了下去。將他陰戶一看。潔淨無毛\footnote{是極。回回家男婦但有毛處無不扳淨者。相傳敎門中專有一種爲婦人剃陰毛者。名曰剃小臉兒的。然不知果否。或婦人爲之剃則有之。若男子決無此理。或人笑罵之言耳。有一笑談。一婦呼人剃小臉。剃畢。其人興動。以陽物送入頻抽。婦怒曰。你這是怎麼說。其人陪笑道。奶奶旣剃了小臉。自然要取取大耳。}。肥嫩已是動人。且他不但不曾生育過。而且不曾經過大物。尚還是緊揪揪一條細縫。微露指頂大一點花心。竹思寬生平見所未見。愛之如寶。將腿分開。聞了一聞。是方纔他用香肥挖洗的噴鼻馨香。把嘴對了他的陰門。一陣亂舔。又將舌頭伸入戶中絞刮。火氏覺得雖不如那狗舔得受用\footnote{竹思寬之舌雖不如狗。他的陽物却勝似驢。}。但慾火動久。被他舔得癢癢酥酥。淫情更熾。那淫水一股股的冒出。竹思寬知他情濃。牝物也濕透了。連忙起來。把自己龜頭抹上許多唾沫。叫他腿揸得開開的。然後對着門往裡頂。那裡進得去。略略重些。火氏就叫疼說苦。弄了許久。還不得其門而入。竹思寬急得沒法了。想了一想。對火氏道。這進不去怎麼樣處。我想來我在上邊弄。不知輕重。倒是你上我身來往下坐。該輕該重。該進該出。你自己酌量着行。這唾沫不如油滑。把你我兩件東西都多擦些油。或者就好了\footnote{火氏前日用油。此時竹思寬也要用油。可謂二人同心。}。火氏點頭依允。竹思寬下床來。拿了燈盞中油。自己抹上些。又將指頭蘸着。替火氏把陰門內外擦上許多\footnote{先則香。此是油臭矣。}。上床來。扶起火氏。他仰臥着。叫火氏跨上身來。兩手拄定。竹思寬一手搊着他。一着手捏着龜頭。對正了他的陰門。道。你往下坐坐看。火氏往下坐了坐。雖覺得滑溜了些。還穿得陰門生疼。此時舞弄了半夜。尚不曾嘗着是甚滋味。心中也騷極了。顧不得疼。咬着牙狠命往下一坐。竟進去有三四寸。火氏哎呀了一聲。覺得迸急如裂。似刀割的一般。眼淚痛得長流\footnote{先是下面那一隻眼冒水。此時是上面的兩隻眼流淚。他旣姓火。如何有許多水。}。伏下身子道。受不得。下來罷。竹思寬遇了這樣淫美少婦。弄不進去。陽物硬脹得難過。正急得要死。忽見進去了些。箍得龜頭緊緊的。妙不可言。生怕他害疼抽了出去。忙把他屁股用兩手扳住。道。你略忍一忍。就好了\footnote{因此一句。想起一笑語來。一和尚買了一個大鯉魚來。刷淨放入鍋內煎。鯉魚容易不得死。尚首尾亂跳。此僧用鍋鏟按住道。你〔略〕忍一忍。〈略〉就好了。}。頭子旣進得去。底下就容易。火氏也就依他不動。二人親嘴咂舌。頑笑了一會。竹思寬道。這會兒可好些。火氏道。雖比先略好些。還疼得很呢。竹思寬道。你抽抽看。用手扶着他兩胯。一起一落。動了幾下。火氏雖然覺得龜頭在裡面塞得脹滿有趣。但陰門痛不可忍。嘴對着他的嘴。道。行不得了。脹得疼得很。改日再來弄罷。竹思寬也不敢強他。答道。憑你的意思。火氏擡身拔出。覺得陰門又疼痛了一下。跨下來睡倒。疼得甚是利害。拿他那白紬汗巾擦了一擦\footnote{寫汗巾只云紬字便可。先見用一白字。疑必有所謂。至此方知昨日者焉能顯出血跡。作者之細心若此。}。拿上來看一看。竟有許多鮮血同油跡\footnote{鐵化當日娶他時。不知曾有此否。}。用手摸了摸。原來是把陰門撑裂了\footnote{可惜。}。竹思寬接過汗巾來。也將陽物試(拭)淨。對火氏道。你這汗巾與我罷。火氏道。髒巴巴的。你要他做甚麼。竹思寬把他抱得緊緊的。道。心肝。你雖不是女身。今日同我弄出這些血來。也算是開首的恩情一樣。我留着。一時間想起你來。不得見面。見了汗巾上的血。就如同見了你一樣。便連親了幾個嘴\footnote{人乍見此。不過是竹思寬一番相憐相愛的話。又帶三分奉承語。要知此別有深意。竹思寬豈不自知齒已非幼矣。與火氏大不相侔。而貌又不足以動人。火氏之所以愛他者。只因此孽具耳。今旣受創。恐後竟棄之。奈何。故想出要此汗巾。拴住他一片心。常於此物上着想。以圖長久相處也。}。火氏見他說得這等恩愛。弄都弄了。還怕羞不成。一把摟過他脖子來。也連親了兩個嘴。說道。親哥。你這樣疼愛我。我就給你弄死了。也是沒得怨的。把嫩生生的舌尖遞入他口中咂了一會\footnote{淫婦人水性易動。已入其圈套中矣。}。他同鐵化正經夫妻一場。也不曾有這番恩愛\footnote{二語雖是閒言。却是入火氏的罪案。}。火氏道。這弄不得怎麼處。竹思寬道。你今日是初試。下回再弄。包你就不這樣艱難了。火氏道。等我養好些。你過幾日再來。但只是你怎麼得在這裡過夜。竹思寬道。這個只好看機緣。我想法在嫖賭兩個字上把你家鐵大爺掛在外邊。我就好來親近你\footnote{好嫖賭者着眼。}。只恐我來了你不得知道。火氏道。只要你把我家的哄了出去。我時常叫巧兒出來探聽。他二人約定。摟抱着睡了一覺。醒來時。月已西斜。將及天曙。火氏道。我去罷。天將亮了。起來穿衣。二人捨不得。又摟抱着親嘴咂舌了一會。火氏將頭上的金簪拔了一枝。替他關在頭上。道。親哥。我送你這個。取個結髮恩情的意思。千萬不可忘了今日。但切不可與我家的看見。竹思寬接住。道。親親。你的深情我殺身難報。豈敢負你。但承你厚情屢屢。我沒一點東西送你做個記念。心中甚覺抱愧。火氏道。兩情相愛。要甚麼値錢的東西。把你的褲帶換與我。我繫在腰中做個想念。你若捨得。再把下身陰毛拔幾根與我\footnote{此却是敎門難得見的罕物。}。我做個小荷包裝着。日夜帶在身上。如同與你相伴一般。這個就強如送我件寶貝了\footnote{愛其巨物如寶。推及於毛。亦視如寶。寫淫婦寫得盡情不堪。}。竹思寬忙把褲帶解下換過。伸手將陰毛拔了一把。遞與火氏\footnote{一把。趣。要做刡子用乎。一把至少有數十根。昔有一鬍漢。偶然持鬍。掉下一根。連道。可惜可惜。其妻曰。一根鬍子。何可惜之有。鬍漢道。你豈不聞一根鬍子値一條片(牛)麼。其妻脫下底衣。笑指陰戶謂夫曰。若如你說。我這些鬍子値得一山牛呢。竹思寬一把陰毛也値了許多牛。回敬不爲不厚。}。火氏捲在衫子袖內。方纔下床。看那巧兒時。倒在一張醉翁椅上。兩腿大揸。放在兩邊椅軸上擱着濃睡\footnote{他因睡熟。不曾聽得二人行事。故後來問竹思寬可弄得是弄不得。前後照應。絲毫不謬。}。火氏笑着把他推醒。開門出來。猶依依不捨。不忍分離。攜着手叮嚀了又叮嚀。囑咐了又囑咐。送到角門口。方纔分手。竹思寬目送火氏。那火氏也一步兩回頭的望。只等火氏進了內院子門看不見了。竹思寬方纔關了角門。回到書房去睡。火氏到了屋內。巧兒關了院門。火氏上床坐下。重又脫衣而寢。那陰門次日大腫。裂破處疼了好幾日。直等結了疤兒掉去纔好了。那竹思寬一覺睡到日高三丈方醒。想道。世間有這樣多情女子。我料無可報他。只有竭力同他大弄一弄。得他稍遂歡心\footnote{火氏原不過圖此。}。纔可報了他萬一。只要想法騙得老鐵在外過夜纔可行事\footnote{這是鐵化好厚朋友的算計。}。正想着主意。只見鐵化笑吟吟走進來。道。我在屠家專候兄。何反在我舍下呢。竹思寬道。昨日早間有些俗事脫不得身。直到夜了。我只當大爺回府。特來看看采興。誰知竟不曾回來。夜深了去不得。所以在府上借宿。大爺采頭如何。鐵化道。兄言不謬。果然三個都是鄒(雛)兒。被我大勝。贏了將及千金\footnote{贏得他人千金。折去妻子一竅。愛便宜人往往如此。}。方纔回來。正要着人去請兄。幾時叫老屠勾了他們來。讓我再贏他一場\footnote{不勞多囑。他心比你還勝。}。門上人說兄在此間。昨夜失陪得罪\footnote{昨夜有令政奉陪。何必勞你。}。竹思寬聽了。正中下懷。他出去了。好來同火氏親熱。忙答道。這容易。都在我效勞\footnote{豈只效勞於鐵化。更欲效勞於火氏也。}。對老屠說了。約定日子。我來奉請。鐵化將小廝們搭連中扛來的銀子。拿出一大封遞與竹思寬。道。承兄指引。些須奉敬。倘再弄着他們。我贏了還有酬謝處。竹思寬道。怎敢當大爺這樣厚賜\footnote{連他夫人的那件寶貝都拜領了。何況此些須之物。}。鐵化道。你我相契間不必客套。請收了。竹思寬道了謝。收入腰中。起身作辭。鐵化要留他吃飯。他道。大爺辛苦了一夜。乏困了。請安歇安歇罷。改日再來奉擾。拱手去了。鐵化也正要睡睡。見他這樣體貼。好不感激。因昨夜不在家不曾陪他。又甚不過意。不知尊夫人已陪他過夜。連陰戶都被他弄裂了。鐵化同他這等相好。又待他如此厚情。還淫汚他的妻子。可見世上結交不可不絕匪類。正是。

\begin{quotation}

畫虎畫皮難畫骨。知人知面不知心。

\end{quotation}

那竹思寬得了五十兩銀子。心中暗喜道。這個阿呆。我睡了他的老婆。又還得他的厚贈\footnote{受人如此之情。反淫人之妻。當內愧自責不暇。而更笑人之呆。此等心腸。較惡獸尤毒。謂竹思寬之淫火氏。係火氏起意。彼罪尚可減。但竹思寬負鐵化之深情。其罪何可恕。然而有說焉。彼父母猶不知爲何人。又何朋友之足論。於禽獸又何難焉。}。世上那裡有這樣便宜的事。歡喜不盡。一路又尋思道。錢貴這妮子。自從梳籠之後。這幾年越發嬌得愛人。我但瞥見他那舉動言笑。連精魂俱失。久要想親近親近他。我雖同他母親相厚。不好白開口的。今拿這五十兩頭送他。要同他女兒睡一夜。但見錢眼開。再沒有不肯的。我先怕我這孽具太大。他那嬌怯怯的身子恐不能容。今看鐵家娘子與他身材相彷彿。這都弄進去了。何況他經過多人。自然與鐵家娘子又是不同。可以得一場快樂。也不枉爲人一世。且他母親的那件東西也有些癟了。換一換新鮮嫩物嘗嘗。遂欣欣然到錢家來同郝氏商議。這種壞人。

\begin{quotation}

纔奸了多情淫婦。又妄想才美嬌娃。

\end{quotation}

他不知可能想得上錢貴否。下文便見。

\begin{quotation}

竹思寬權時按下。錢貴姐再接來因。

\end{quotation}

姑妄言三卷終



\endnotetext[1]{「婿」原作「配」,據書前目錄改。}

\endnotetext[2]{「沈思」原作「思沈」,據文義改。}

\endnotetext[3]{「女」原作「絳」,據陳鼎《留溪外傳》卷十五《杜烈女列傳》改。}

\endnotetext[4]{「完」原作「寬」,據陈鼎《留溪外傳》卷十五《杜烈女列傳》改。}

\endnotetext[5]{「伯」原作「泊」,據陳鼎《留溪外傳》卷十五《杜烈女列傳》改。}

\endnotetext[6]{「主公」原作「公主」,據文義改。}

\setcounter{footnote}{0}

\theendnotes

\part*{姑妄言第四卷}
\addcontentsline{toc}{part}{姑妄言第四卷}
\markboth{姑妄言第四卷}{姑妄言第四卷}

鈍翁曰。鍾情是一部書內的一個正經脚色。自然要寫得他高。然說他幼無父母。爲兄所棄。此是何意。雖是寫鍾悛忘親棄弟之惡。正是高擡鍾生處。以十齡幼稚之童。無父兄管敎。先雖依傍外祖家中。後復獨處。竟能少年成立。所謂不遇盤根錯節。無以別利器者是也。鍾生之遇錢貴。用梅生許多婉轉。方得成就。眞好媒根。錢貴聽代目說鍾生之美。想起素常久聞其名一語。後來親愛便不覺突然。二人定盟。一邊寫鍾情多情。一邊寫錢貴多識。他二人皆從情愛中來。深於情者方得知。峒谿錄一段似乎不必。一則恐童自宏太冷落了。隨筆帶出。二則寫苗蠻風俗。不無開卷有益。且使正文略間一間。看得更覺醒眼。再者鍾生梅生宦賈童鄔皆係一部書中始終要緊之人。鍾生有父母叔兄姪兒外祖表弟許多親戚。梅生則有林報國多必達之姑表。宦蕚則有父母妻舅姑父。賈文物則有岳翁。皆敍之詳。雖鄔合猶有嬴陽之岳。而童自大百萬富翁僅有鐵化一舅。再別無親戚。只一胞兄。但言其回原籍去了一語。便不再提。豈非筆墨疏漏處。故不得不使之一現也。寫童自宏之性情乃與弟迥別者。非揚其兄而抑其弟。若再寫他呆。不但作書者說呆話。且太把富翁說得不値了。則主中豈無高人。特僅見耳。

代目於錢貴前只極誇鍾生之美。雖十分心愛。却並不曾勸及錢貴從良一事。因見其貧窮。恐錢貴未必心肯耳。孰不知二人竟成良緣。非寫代目見淺。乃是要極力擡錢貴迥出尋常之見耳。

梅生雪氏眞是一對好夫妻。不幸中拆。梅生黯然傷神。不肯再娶。可謂篤於夫妻之愛矣。因此始能與鍾生相厚。世間未有薄於夫婦而能親於朋友者。其所厚者薄。無所不薄矣一語可鑒。

竹思寬將來爲郝氏之夫。錢貴嫁與鍾生。竹思寬儼然有後岳之尊。若上門未免辱及鍾生。固不可。旣係至戚。竟不上門。又不近情。不得不思一絕之之法。故想出錢貴一罵。錢貴之罵。乃因其要嫖己而怒。不如此。後來不可以絕之也。作者心細如髪。

火氏竹思寬一段。乃寫淫婦之淫至於此極。竹思寬只算得此文中應用之一物。並不曾用正筆寫他。

寫巧兒。活是一個伶俐獻勤丫頭的身分。

代目雖不足爲重輕。然係鍾生生子之妾。故不得不替他長些聲價。乃祖叔祖爲良善正經之人。祖母又是賢德之婦。父雖不肖。後能改過遷善。仍不失爲成家子。總不過說他是好人家兒女。落爲人之小星。尚有爲之負屈之意。雖抑其父。實揚其女也。戴遷之好賭。不如此寫。代目何以得歷鐵童錢三姓而到鍾生之宅爲妾也。因借他賭之一字。故撰出一篇戒賭文來。少年孟浪好賭之人。當書一通於座右。

\chapter*{姑妄言卷之四\\
第四回 梅子多情攜愛友乍入煙花 鍾生無意訪名娃初諧魚水\\
附 鍾悛吞產潛蹤 火氏偷情滿意}
\addcontentsline{toc}{chapter}{第四回 梅子多情攜愛友乍入煙花 鍾生無意訪名娃初諧魚水}
\markboth{第四回 梅子多情攜愛友乍入煙花 鍾生無意訪名娃初諧魚水}{第四回 梅子多情攜愛友乍入煙花 鍾生無意訪名娃初諧魚水}

且說那時城中有一個書生。鍾姓情名。麗生爲字。他家世代業儒。他父親鍾越。乃一懷才抱德之士。生性慷慨。積德好施。娶妻咸氏\footnote{丈夫積德。妻子又賢。宜乎得生令子。}。夫妻舉案齊眉。琴瑟和諧。鍾越父母亡後。只有一個胞弟。名叫鍾趨。也列名黌序。但他的性情與哥哥迥別。惟知損人利己。敬富欺貧\footnote{古云。一母之子有賢有愚。誠非謬言。觀此即知兄能越過於人。做了一個盛德君子。弟則趨利嫌貧。做了一個小人。何迥避(別)如此也。}。他每見哥哥揮金如土。暗暗心疼。想道。我家祖遺有限。若任着哥哥的豪性揮霍起來。其盡可立而待。他雖博了一個虛名。我却受了一生實害。如何行得。後來忍不得了。定要分拆。鍾越也知他的私意。只得從公。將家產剖而爲二。分居各住。這鍾越二十八歲上始生一子。命名鍾悛。到六七歲上。也曾送去讀書。資性也還聰明。孩童頑戲的事是樣見了就會。惟到了書上。便如仇敵一般。不但不上心去讀。尚不屑正眼一視。讀了三五年。仍然一塊白木\footnote{近日人家子弟如此者不少。}。他父親一心望兒子成器。屢屢囑托先生嚴訓。無奈鞭扑之時。他一般害怕。一住了板子。便只袖手高坐。先生再三呵叱。他眼睛四處去望。口中咿咿喔喔。也不知哼些甚麼。及至背書時。他翻着白眼。只聽得咿呀呢哪的哼。一個字也記不得。寫倣的時候。衆學生都寫完了。他容易再寫不完一般。見他不住手的畫。及至拿上來時。看他滿臉滿手滿嘴無處不是黑墨\footnote{此一段是頑劣小學生的小像。}。再看字時。東一個西一個。大一個小一個。微有形似而已。寫字與他認。他口中但說這是那這是那個三字。正經叫他認的。那個字再說不出。手心也不知打過多少。日日仍然如是。敎他作對。嘴都磨鋊了。他總不懂。一日。先生出了個對叫他對。道。

\begin{quotation}

靑驄馬。

\end{quotation}

還講解與他聽。靑是色。馬是獸。他妙極。想了一會。對道。

\begin{quotation}

白嚼蛆。

\end{quotation}

先生聽了。反忍不住大笑。只得向鍾越細道他賢郞的這些妙處。鍾越以爲館中學生多。故他心野。辭了先生。帶他回來自訓。亦復如是。無日不打數次。但不打他。雖不知他念甚麼。還哼哼有聲。越打連聲氣都沒有了。鍾越也沒法了。惟有切齒恨怒。咸氏三十多歲只此一子。未免愛惜\footnote{婦人雖賢。未有不姑息兒女者。}。勸丈夫道。做父母的誰不願兒子成器。但當因材而施。這孩子天生不是個讀書的材料。雖打殺了何益。士農工商。各執一業。等他大來不拘敎他做那一行事罷。鍾越見他是塊朽木。不能雕斲的了。無可奈何。只得由他。他到了十六七歲。心雖險仄。刻薄寡恩\footnote{二語他一身定評。}。却一文不肯浪費。鍾越常想道。此子惜錢如命。雖非成家之道。若能中正自持。還可爲守成之子。無奈心術不端。恐將來一敗塗地耳。時常發嘆\footnote{可謂知子者莫若父。}。因係獨子。未免望孫。十八歲上。替他娶了個鄂秀才的女兒爲媳。這鄂氏雖不到那潑悍無知的壞處\footnote{有此一句。後日方可回來與鍾生同居也。}。至於孝順翁姑。相夫持家的道理。却也一絲不識。惟知食粟而已。咸氏十七八年不生育了。到了四十六歲忽又懷起孕來。次年生下一個兒子。粉面朱唇。淸眉目秀。鍾越歡喜無限。一則見鍾悛已是廢物。圖得此子。或可接紹書香。二則見鍾悛孤立。有一手足。將來可以彼此相靠\footnote{父母心則做如此想。孰不知爲其兄者視之爲贅疣也。}。這些親友見他老來添子。盡來稱賀。鍾越是素性豪爽的人。又是心中歡喜。預備極豐盛的筵席款待衆賓。那鍾悛自己每常以爲是獨子。將來的家產是他獨承。看見生了兄弟。不但不喜。反甚不樂。又見父親如此用度。心下老大暗急。雖不敢明說。暗地嘓噥道。這樣大年紀從新養甚麼兒子。不害羞恥\footnote{奇想。豈老年人皆不許生兒者耶。}。倒反賀喜宴客。花錢費鈔。做這樣沒要緊的事。一個血胞子。還不知養得大養不大。就算着養大了。將來撂得血餬零拉的。還是我的大累\footnote{甚矣。人之發言不可易也。鍾悛今日說兄弟。不意後來應在他乃郞身上。可發一嘆。}。鍾越也有所聞。不去理他。過了二年餘。鍾悛也生了一個兒子。他夫妻愛如掌珍。取名小狗子。謂易生易長之意。鍾越見次子到了五歲。聰慧異常。每日敎他認幾個字。他再不遺忘。半年來竟認得許多。鍾越想長子已是無用的了。此兒尚有讀書之資。不可再誤。此時已五十餘歲。下過九次科場。無奈才高命薄不售。竟吿了衣衿\footnote{九者。數之奇也。旣不售。應當吿退。若到十次。便沒趣了。}。閉戶在家。惟以課子爲務。因長子性情刻薄。遂將次子取名鍾情。字曰麗生。無非欲其天倫中多情之意。這鍾情雖不能過目成誦。凡是經書。他念過三五遍。無不純熟。不但記得。且個個字認得。鍾越愈加歡喜。況是幼子。老夫妻未免過於疼愛。鍾悛更覺不平。背地道。我是長子。我兒子又是長孫。倒不相干。倒把他當倭寶兒一般。等着等着。等他大來做了官。好來封贈娘老子的\footnote{鍾悛雖是氣恨語。孰意後來竟應其言。}。我的兒子也不讀書。看他後來趕得上這讀書的趕不上\footnote{豈但令郞趕不上。連令尊還趕不上也。}。因此他見了兄弟就如眼中釘一般。鍾越也知因次子年小。也只忍在心中。每日細心將小學並各種故事。孝弟忠信的話。諄諄講解與鍾情聽。他聽了便能記憶。八九歲上。就知孝父母敬兄嫂。那小狗子雖纔五六歲。頑劣甚於其父。並不知祖父母父母叔叔爲何物。一日混頑混跳混罵。他聽見爺爺叫叔叔做鍾情。他也便叫。任你怎麼叫叱。叫他不許稱呼叔叔名。他總不理\footnote{倒虧他這一叫。因叫熟了。後來相認時纔記得叔叔名鍾情也。}。那鍾悛鄂氏疼愛他到無可容言處。一任他的性子。鍾越再要管他。見大兒子已刺嫌兄弟。再要打了孫子。兒子媳婦定以爲父母疼幼子。不疼長孫。弟兄將來越參商了。每每隱忍\footnote{說盡家庭苦情。}。常常嘆息。小狗子但見叔叔拿着些甚麼。劈手就搶。不給就罵。鍾情從不同他爭鬧。倒反疼他\footnote{可見孝弟慈愛。皆天性中帶來者。}。因此也還相安。鍾情九歲上。經書皆講熟。已經成篇。筆下甚淸亮。鍾越以爲可以見此兒取金紫。娛暮景。不想得了一病。日重一日。奄奄不起。鍾悛視若罔聞。鍾情衣不解帶。親嘗湯藥。時刻不離的服事。鍾越看看危篤。鍾情每夜禱天。願以身代。一日。鍾越的岳父咸德來看他。鍾越涶(垂)淚道。小婿之病不能起矣。別無他囑。大外孫已成廢物。小外孫資性還是個讀書種子。小婿死後恐誤了他。望岳父念翁婿之情。將小外孫帶去。擇師訓導。將來不墜家聲。小婿於九泉之下也瞑目了。因顧鍾情道。看你哥哥可在家。鍾情去了來道。嫂嫂說。今早朋友們約哥哥往雨花臺耍靑去了\footnote{老子病得待死。兒子且去耍靑。此等惡子頗多。勿單謂只一鍾悛也。鍾越欲托心腹與丈人。恐大兒聞知。故命去看。寫得精細。}。鍾越嘆了兩聲\footnote{此嘆了兩聲乃兩爲也。一嘆生此逆子若此不孝。二嘆欲說托孤言。先覺傷心。不覺嘆而又嘆一聲耳。}。執丈人之手。低說道\footnote{怕媳婦聽得也。}。大兒非友愛者。俟小兒成立之時。岳父將小婿家產爲他二人分之。不然。必爲大兒所獨吞矣。今日小婿若爲他兄弟分拆。但小兒尚幼。恐倘有不測之禍。今有小婿家私單一紙。岳翁留爲異日分拆之憑。萬望岳父留意。遂在枕邊取了一張賬單。遞與咸德\footnote{鍾越做事可謂密矣。後日鍾悛竟知之。盜賣而去。可笑世人但做機密事。開口便曰可瞞着人。孰不知人並不曾瞞得。只瞞了自己耳。}。咸德也墮了幾點淚。應允了\footnote{鍾越之慮幼子。可謂盡善矣。豈意鍾悛後來更有先着。父母臨死猶爲兒孫慮後者。終無益也。}。過了數日。鍾越自覺沈重。叫了二子在傍。向鍾悛道。我死後。你是長子。須孝順母親。撫恤幼弟。得他成人。我亦瞑目。鍾悛也不答應。只鼻孔中似答非答。似笑非笑的坑(吭)了兩聲。鍾越見他這個樣子。也再不說。嘆了一口氣。便閉目而逝。鍾悛喪葬之事凡百從儉。苟且了事而已。鍾情雖在孩提。守定棺材哭泣。晝夜不絕聲者數日。竟至哀毀骨立。親友來吊者。無不暗暗稱異。殯葬之後。咸德將鍾情領了家去。送在一個朋友館中讀書。那先生姓廣名德厚。是飽學盛德名儒\footnote{又一個好先生。}。且訓徒甚是有方。這館中許多窗友。一個姓司名進朝的。是個宦家之子。一個姓劉名顯。他父親名劉太初。也是個有德行的老儒。一個姓梅名根。一個名多必達。是梅根母舅多誼之子。一個名陳仁美。是多必達的姊丈。一名咸平。就是咸德之孫。乃鍾情的表弟\footnote{因鍾生進館。陪出許多窗友。後來一個個的出現。筆力何等簡便。}。衆人之中。惟鍾情梅根獨肯用功。先生見他二人又聰明又苦讀。着實心愛。更加一番敎導講究。他二人彼此問難。互相切磋砥礪。情同骨肉。親愛無比。過了兩年。鍾情到了十一歲。他母親咸氏又復臥病。鍾情聞知。辭了外祖同先生。歸家侍奉。咸氏道。我病未必就死。不可誤了你讀書。你還在館中去。鍾生道。父母生子原圖孝敬。子弟讀書原是要知孝弟的道理。不然念書做什麼事\footnote{常見讀書人而不知孝弟者多矣。}。況古語說。羊有跪乳之恩。鴉有反哺之義。人不知孝。眞禽獸不如了\footnote{鍾生此語。不懼令兄聞之耶。而今世上人之不若禽獸者。比比皆是。}。過了數日。咸氏的病愈沈重。他父親七旬外的人倒還康健。常來看視。咸氏向父親哭道。女兒五十餘歲。不爲夭了。況女婿已故。兒之死何足恨。但放不下你小外孫耳。望父親念女婿臨終之言。撫養他罷。兒死。分之當然。父親年尊了。也不必悲慟。說畢。奄然而逝。咸德也哭了幾場\footnote{女婿死時。咸德只落了幾點淚。女兒死。他哭幾場。寫盡人情。}。那鍾生哀慟迫切。淚盡繼之以血。水米不入口者數日。咸德再三勸慰。始進勻水。喪葬已畢。咸德仍帶他家去讀書。那鍾悛見父母雙亡。遂起了一點私心。將父親所遺產業盡思獨占。他雖欲獨擒。一來怕親友談論\footnote{怕親友談論。還算良心未曾喪盡。}。二來恐兄弟大了。外祖做主。仍要分去\footnote{所懼者此耳。怕人談論還在次之。}。豈不白做一場惡人。遂暗暗變賣了。帶着妻子鄂氏。兒子小狗子。連夜遷彼(徙)他鄕而去。他那個親叔鍾趨。久矣分家各戶。也不來管他。咸德過後方知。不勝惱恨。但鍾悛已不知影響。只得罷了。鍾生虧得外祖撫養成人。到十五歲上。他外祖年已八旬。到老病將危之時。憐外孫孤苦無依。娘舅又死了。只舅母喪居。表弟幼小。料到後來未必能盡心養活他。暗地與了他些私房。叫他各自另尋安身之地\footnote{寫咸德慮自己死後。舅母孀居。未必能養活一語。有深意焉。鍾生若始終依傍外祖舅母家中。不能顯其孤身竟自成立。一也。君(若)不出來。何以得遇錢貴。二也。不得不想到他出來另住。故說他外祖慮及於此。乃借他舅母一用。非說他舅母之壞也。看者須知之。}。他遂隻身出來。在鳳凰臺下典了眞敎官的一間斗室棲身。喜他有志上進。埋頭讀書。十七歲就批首進學。他生得面如冠玉。唇若塗硃。經文時藝。一掃千言。歌賦詩詞。援筆立就。有幾句贊他道。

\begin{quotation}

書生之態。弱冠之年。神凝秋水。學冠雲煙。瓊姿皎皎。玉影翩翩。春情吐面。詩思壓肩。性躭情種。骨帶文顚。問誰得似。靑蓮謫仙。

\end{quotation}

他且存心不苟。立志端方\footnote{這八個字是鍾生一生〔定〕評。〈定〉}。雖係少年。眞是個才行兼優的人品。那時的人都好奉承\footnote{今日更勝。}。他不但不會奉承人。且不同愛奉承者對面。盡都喜容悅。他豈但不去容悅人。更不與要容悅者交談。入泮之後。也算學中數一數二有名的一個秀才。從來應試再不出三名。但只孑然一身。眞個家徒四壁。雖有滿腹才華。難免終年頓困\footnote{腹中有了才華。窮鬼便來相親。財神便去躱避。豈窮鬼喜文而財神妒文耶。殆將誰問。}。喜他志氣亮爽。毫不介意。年已二旬。尚未受室。他也曾幾次央人求婚。但風俗囂薄。人家擇婿只重這財不重那才\footnote{其所由來者久矣。}。人見他家業飄零。孤寒特甚。親戚視同陌路人。朋友盡皆遠避。無一肯就。爲此他發了一奮志。定要先金馬玉堂。然後纔洞房花燭。終日閉戶讀書。足不出外。雖不曾囊螢映雪。刺股懸梁。却也是三更燈火五更雞的苦誦。一日二月下旬。他見春光和藹。小院中數株花木都綠嬌紅艷。讀書之暇。詩興偶作。信筆揮成一絕。

\begin{quotation}

春光嫵媚萬花妍。正是尋芳二月天。

兀坐竟忘春意好。撩人蛺蝶兩蹁躚。

\end{quotation}

興猶未已。復題醉花陰一首詞。道。

\begin{quotation}

杏蕚枝頭紅盡吐。紫燕蹁躚舞。春事半闌刪(珊)。滿徑蒼苔。微染如酥雨。頻斟綠醑留春住。切莫催花去。一歲幾多時。劇飮高歌。醉倒花陰處。

\end{quotation}

寫完擱筆。正在推敲之際。忽聽門外有剝啄之聲。啓戶視之。原來是他自幼的一個窗友。這人姓梅名根。字合山。他有個姑父叫做林放梅\footnote{得便就出林報國。省筆法。}。取林和靖先生孤山種梅之意。他也與此意相合。故取了這個名字。他與鍾生兩人是總角之交。同窗讀書又是同案進學。那梅生雖不能稱富足。也還是小康之家。他知鍾生家寒。時有所贈。雖不能衣食全然管顧。然一年不至凍餒者。多半虧他\footnote{好朋友。今日恐無其人。後食千金之報。不爲過也。若今有此等人。吾當拜之。}。故他二人素來莫逆。時常相晤。梅生十六歲時娶妻雪氏。生得如玉人一般。有古人的一調玉女搖仙佩。正好移來贊他。

\begin{quotation}

飛瓊伴侶。偶別珠宮。未返神仙行綴。取次梳妝。尋常言語。有得幾多姝麗。擬把名花比。恐傍人笑我。談何容易。細思算。奇葩艷卉。惟是深紅淺白而已。爭如這佳人。占得人間。千嬌百媚。

\end{quotation}

他夫妻十分相得。那一種恩愛綢繆。莫能言喩。梅生也美如璧玉。那時他的衆朋友套了古詩二首贈他。一首是贊羨他夫婦的。道。

\begin{quotation}

有梅無雪不精神。有雪無梅俗了人。

今日雪梅相配合。兩人得做十分春。

\end{quotation}

又有一首是戲謔他夫妻的。道。

\begin{quotation}

梅雪爭妍未肯降。詩人擱筆費平章。

梅須遜雪三分濶。雪却輸梅一段長。

\end{quotation}

他夫妻見了。幾乎笑倒。那雪氏不但有如花之貌。且有咏雪之才。不想成親只二年光景。那一年天氣甚暑。雪氏偶染了一場熱病而歿\footnote{雪遇大熱。自然化去矣。}。眞是。

\begin{quotation}

世間好物不堅牢。彩雲易散琉璃脆。

\end{quotation}

梅生面上雖不覺十分悲痛。而黯然傷神。竟幾幾乎似當年荀奉倩。有個骨化形銷的樣子。鍾生再三苦勸。他方少釋。過了年餘。有人愛他的人品淸俊。家道厚足。要將女兒嫁他續絃\footnote{愛其家道耳。若人品。鍾生何無人愛。見而愛之者。只一代目。聞而愛之者。只一瞽目錢貴耶。}。他執意不娶。鍾生正色諫他道。兄與尊嫂雖夫妻恩愛至篤。但繼嗣更重於私情。兄讀書人豈不明此。梅生謝道。吾兄以大理敎我。敢不從命。但佳人難再得。容緩圖之。數年來。他尚鰥居未娶。今日來訪鍾生。一進門。相遜揖罷。便道。吾兄終日閉戶。自然學業大進。讀書雖係妙事。然不可苦功太過。損耗精神。還該散步散步。以活文機。鍾生道。小弟鶉衣百結。羞見親友。在家無事。不過將這些斷簡殘篇拿來翻閱。聊舒悶懷。有何進益。梅氏道。兄言謬矣。聖人說。素貧賤行乎貧賤。且貧乃士之常。又何足爲愧。貧窮二字可是人笑得的。兄不憶原憲譏子貢曰。予貧也。非病也。子貢終身自愧爲失言。談笑人貧窮的人。那不過是市井之徒。略明道理的人豈肯有此。況以兄之大才。取金紫如拾芥。焉可限量。兄萬不可把志氣自餒了。況還有說衣敝縕袍與衣狐狢者立而不恥。這有何妨。鍾生道。吾兄見愛。則有此語。若世俗炎涼之輩。靑目者誰。衣敝縕袍與衣狐狢者立雖不足恥。爲今之際。那衣狐狢的人與衣敝縕袍者立反以爲恥\footnote{說盡世情。}。小人心胸另有一番評論。且不可以今比昔也。梅生道。兄說得也是。世俗惡薄至此。殊屬可笑。然此等人也不足與較。弟連日未晤兄。可有甚佳作麼。鍾生道。春色惱人。小弟連日爲睡魔所侵。神思昏昧。並無拙作。只方纔見小園中花草可愛。謅得一詞一絕。正欲求斧政。遂將所作的詩詞遞與梅生。道。請敎。梅生接過看了。贊道。可謂滿紙琳瑯。字字珠玉。足見吾兄用功之效。鍾生道。小弟俚言請敎。吾兄反一番謬獎。此非弟請敎之本意了。梅生道。果然佳妙。非弟過譽。因將手中的扇子遞過來。道。弟値有便面在此。祈吾兄將尊作一揮。鍾生笑道。此等鄙俚之言。豈可有汚尊搖。梅生道。兄不必過謙。你我莫逆兄弟。何必用這些套語。鍾生推辭不得。笑着提起筆來寫了送過。梅生接來看了。道。三日不見。刮目相待。兄不但佳章精妙。連大筆近日也越發純熟了。鍾生笑道。汚兄佳扇。幸勿見責。二人閒談了一會。梅生順手在案上取過一本書來翻閱。見內中夾着幾張字紙。說道。這想是兄的窗稿了。鍾生笑道。不然。昨日小弟無聊之極。偶謅得一篇戲語。雖是不經之言。恰中我輩貧寒之病。梅生打開看道。

\begin{quotation}

九州巡察使臣鮑奮謹奏。爲乞恩剿除巨惡。以甦蒼生事。臣奉命巡視九州。兢兢業業。不敢稍怠。密訪得有巨惡九名。乃盛世之大凶。爲天下之深害。生民被其塗炭。萬姓受其摧殘。惡貫滔天。罪着九地。眞不可一刻留於世者也。臣訪得彼等罪惡。鑿鑿可據。非係風聞。乞大奮乾斷剿出。以甦生民困苦。古謂殺一人而甦萬命。若除此九惡。使天下億兆窮人皆被其澤矣。令將彼等罪惡。謹開列於左。

嬴藺錢堅二人者。表裡爲奸。志同氣合。嬴藺則助人賄通關節。大干法紀。錢堅則與人詭詐通神。奸謀百出。專與正人君子爲仇。但同鄙吝貪夫契合。遇富貴者則趨附之。刻薄非爲。縱淫縱惡。見貧窮者則膜(漠)視之。毫不相恤。爲寇爲仇。石崇一宵小者流。郭況一椒房之嬖。嬴藺則依之爲鷹犬。嚴世蕃範美人爲溺器。慕容彥超鑄鐵胎做大錠。嬴藺則助之爲奸邪。鄧通一嬖幸小人。蕭宏一膏粱紈袴。錢堅則附之妄作非爲。暴珍(殄)肆惡。至於貧窮者。即如聖門顏淵原憲之流。彼不但不助之結之。反凌之棄之。又何況於蓬茅下士。閭閻小民。不困其慳吝。不受其荼毒耶。且使人父子失其親。兄弟失其愛。朋友失其誼。夫婦失其和。以至正人君子困苦飢寒。無賴小人流爲盜賊。皆嬴藺錢堅使之也。此二人者。趨富欺貧。親貴凌賤。罪猶其次。而助人爲奸淫。黨人爲凶惡。罪狀多端。不可僕(擢)數。似此窮凶極惡。無刑可加。乞敕火力士鐵金剛。粉其身碎其骨。遍給天下之貧士窮民。庶可以酬往愆。以消衆忿。此其一也。

薛泰罪惡雖未着於四時。而刻毒久施於一季。一至三冬。萬姓苦寒之時。不但不能如太陽普臨天下。使貧者可以負暄。彼反漫空飛舞。遍地飄揚。假做輕模輕樣。其實如刃如鎗。陰賊陽善。倍加楚毒。使無衣無纊之人。骨砭肌裂。口噤體僵。袁安高士幾至捐軀。角哀賢者竟遭畢命。古今來受其害者。亦不能屈指而記。封厲冷盛二人。與彼結爲死黨。惟以害人爲事。薛泰之惡已無窮。而封厲鼓舞助之。冷盛阿諛輔之。同惡相濟。使天下之窮人。破膚墮之者有之。抱臂縮頸者有之。齒鬪號寒。身僵哭冷。呼天莫應。叩地無門。眞有不可形容者。窮苦無吿。萬姓含寃。乞敕皎日消其雪。封姨禁其風。元惡不能逞凶。冷盛助桀爲虐之流。不但不敢施其威。當亦隨之而滅矣。除此三凶。則生民皆受和煦之澤。庶免其苦冷號寒之痛。此其二也。

古謂民非水火不生活。水火固有功於人。而於人爲害者亦不淺。然功不能掩其過也。上古帝堯之時。汎濫於天下。幾至民無所安息。後雖爲大禹所平治。然至今數千年來。水煥常逞志恃凶。妄作威福。良田美稼漫渙沈淪。麗室華居漂流淹沒。懷山襄陵。沈竈產蛙。使受害之人無粒米之炊。無立錐之地者。皆水煥之罪也。至於火熾之罪。雖因人而起。似可稍逭。然亦彼助之爲虐。不可全恕。咸陽三月之焚。江都竟月之焰。謂出於項羽世民。尚有所諉。而歷來焚宮室。燬民居。蕩產破家。殞身畢命者。多有其人。其罪亦非淺鮮。乞敕祝融禁其火。馮夷制其水。痛加懲創。嚴行防飭。使人但受其功而不罹其害。救民水火亦一要政也。此其三也。

上古茹毛飮血。后稷敎民稼穡。人始得五穀而食之。此聖人憂民愛民之至意也。孰意萬惡米諸者。恣意妄爲。亦效嬴藺錢堅之習。趨炎附勢。棄賤欺貧。富貴之家盈倉積廩。以致紅腐而棄之。彼猶歸之弗止。至於苦寒之室。懸釜待炊。兒啼女哭。彼亦弗顧。如殷紂鉅橋之粟。李密洛口之倉。紅朽作踐。何可勝言。及至人遭貧困。彼更鄙吝萬端。使韓信乞食於漂母。子胥丐漿於瀨女。曾子三旬九食。梁武餓死臺城。介之推割股啖君。張睢陽烹童賞士。皆米諸之所爲也。甚至孔子萬代之師。亦猶厄之陳蔡。其罪尚未擢髮而數耶。更有羅雀熏鼠。敲骨吸髓。夫妻相食。易子而炊者。傷心慘目。尚忍言哉。皆米諸稔惡之所致也。乞敕風伯五日一風。雨師十日一雨。蜡不爲災。蝗不爲害。天下之粟賤如塵沙。人人得而積之。則米諸不能妄自尊貴。與人爲難。且使人人得而食之。碎嚼其軀。勿論貧富。無枵腹之患。皆鼓腹擊壤。衢歌帝力。其功於萬姓豈淺尟哉。此其四也。

薪者天下無地不產。或草或木。或葦或蒿。無不可而爲之。乃至賤之物也。而辛貴一葑菲下材。草木賤質。亦自矜其能。視之如桂。效惡薄趨世之風。作逐臭附羶之態。亦與貧者爲難。竟至寒士之家。突內無煙。窮民之室。竈不舉火。誠可深惡而痛絕者也。乞敕五岳四鎭以及各省郡邑城隍社令之神。無地不生。無處不茂。使辛貴及其子孫。人人得而誅之。戶戶得而炊之。化爲灰燼。棄之溝壑。然後辛貴之威庶可稍殺。此亦濟民之一端。此其五也。此五者皆天下之窮民而無吿者。臣有巡察之責。旣得其實。敢不備細陳之。如不以瞽言爲謬。乞賜施行天下。幸甚。謹奏。賫奏官戴天命\footnote{此一本雖是戲語。爲後來鍾生上裁監軍本作一對。前後遙遙一對耳。}。

\end{quotation}

梅生看完了。道。兄之尊作固佳。其如上帝無庸議。奈何。二人大笑了一回。梅生又道。兄方纔說神思昏倦。這是坐久了的緣故。今小弟奉陪到外面閒步一步。看一看春色。把胸襟稍舒。就精神健旺了。鍾生道。承兄雅愛。但弟平素倦遊。不敢從命\footnote{初次邀是如此辭。}。梅生道。吾兄眞讀迂了。今春光明媚。花柳動人。各處仕女如雲。車馬咽道。若不出遊賞遊賞。豈不爲花鳥所笑。說畢。拉了鍾生要走。鍾生再四推辭。道。閒花野草。小弟實不願看。辜兄美情。容當荆請\footnote{二次邀是如此辭。}。梅生道。兄旣無此高興。弟也不敢過強。然旣不去賞春花。同兄去訪一訪解語花何如。鍾生道。請敎吾兄。此言何謂。梅生道。兄終日在家。不知外面的事。近來平康中有一瞽妓。姓錢名貴\footnote{此一回題目便是梅生攜鍾生訪錢貴。却不開首便說出。從約去看花。兩次三番。彎彎轉轉纔說到他身上來。筆墨曲折之妙若此。若一開口便邀了到錢家去。不但梅生是約人嫖妓之損友。且突然而來。不成語矣。}。生得肌如白玉。面似桃花。那一段嬝娜的身材。風流的態度。百口也贊他不盡。雖是少了一對秋波。那一種嬌媚嫣然。令人魂醉的樣子。眞是形容不出\footnote{此是極力贊他之貌。}。小弟當日聽得人說。也不肯信。後來親去一訪。果是名下無虛。弟還記得當日令叔所棄的令坦干不驕兄。曾贈他一調浣溪紗的小詞。是贊他妙處的。遂念道\footnote{得便就順筆帶出干生是鍾趨的棄婿。後來照應。便不覺突然。此雖是作者之長技。實可謂之黃絹幼婦。}。

\begin{quotation}

紫玉風流白玉身。嫣然一笑欲傾城。淡粧濃抹總宜人。

蜜意難窺吞吐語。柔情易覺淺深顰。不須迴眼已牽情。

\end{quotation}

兄聽此作。可見彼之嬌艷了。我同兄去一訪。也可寬些眼界。兄意如何。鍾生笑道。兄愛小弟過厚。故說得這瞽妓如天上人。欲弟去一遊耳。弟雖生平不曾會過妓女。曾聽得人說。近日大街中並無一個名娃。大非昔日之比。何況瞽妓中尚有此等人物。梅生道。我與兄自幼相知。可曾有一語相欺。若謂瞽妓中無美人。昔日王嬙西子綠珠之輩。就不該生於鄕僻了。兄何固執若此。鍾生道。小弟非敢固執。但想他一個瞎妓。縱有幾分容貌。自然胸如黑漆。只好娛市井之徒。我輩讀書人對着一個白木。單只大嚼屠門肉。牛飮幾杯回來。有何趣味。又不若對着那嫩草嬌花。聽那枝頭小鳥嘹嚦。痛飮一番了\footnote{三次邀是如此辭。}。梅生笑道。兄可謂唐突西子了。兄旣不知。也怪兄不得。這錢貴自幼穎悟異常。八九歲時就詩詞歌賦無不涉獵。後來十歲上纔壞了雙目。他至今終日咿唔。著作甚富。皆膾炙人口。小弟記得他十三四歲時。有他自嗟薄命的四首絕句。念與兄聽。看是如何。遂將他的薄命詩念了一遍。又道。弟還見過他的少年遊四闋四季詞兒。還聽人傳念他編的囀林鶯。更妙一時。記不得許多。兄到他家要出來一看。便知弟言非謬\footnote{此是極力贊他之才。}。鍾生聽罷。也不禁容色飛舞。道。果爾佳作。可不愧兄之贊揚矣。梅生道。兄旣以弟言爲不謬。弟做薄東。請兄一樂。鍾生道。承兄厚意殷殷。本當從命。但他旣是名妓。又有如此才華。相交的自然都是富翁大老。小弟一介寒儒。那裡在他眼界內。恐去反受他輕薄。那時進退兩難。還是不去的好\footnote{四次邀是如此辭。}。梅生道。吾兄吾兄。人不易知。知人亦不易也。吾兄此言是皮相英雄了。兄還不知錢貴的心跡。他極重的是風流才貌。最厭的是銅臭烏紗。他向日遇着俊俏才郞。雖不得他曲意奉承。也還頗親色笑。若是那癡蠢子弟。雖富勝陶朱。他不但不肯相陪。還有許多的譏誚。所以那些膏粱紈袴。往往乘興而來。弄個敗興而返。後來因他母親苦勸。他如今纔略肯通融。我還聽得人傳說。他曾立一誓願。倘遇着個才貌兼全的知心伴。不拘貧富。願托終身。吾兄這一去。不但不受他輕薄。恐還要在他知心之列呢\footnote{不意此語竟成先兆。}。鍾生道。若果如兄所說。此女可謂妓中英雄。以瞽目之人而有此心胸。又高出梁夫人紅拂妓之上了。但恐此言容或有之。未必如兄所說若此鑿鑿可據。梅生道。不患弟言之不實。猶恐我揚之不盡耳。今同兄去看一會。若弟謬言。兄此後竟視弟爲妄人可也。鍾生見他說得如此眞切。未免少年心動。答道。弟豈敢疑兄之妄。私心竊料恐世間無此尤物。今日之鬚眉男子無一人能塵埃中物色英雄\footnote{此句是一部書的骨子。}。況此一瞽女而具此俠腸。有此巨識乎\footnote{此是一個題目。一部書從此二句敷演而出也。知否。}。梅生道。兄到彼見之。若不符弟言。竟罰弟以金谷酒數。鍾生道。旣成(承)見愛。敢不趨陪\footnote{五次邀方肯同去。只一同遊寫得屢屢次次。一見梅生之愛友過甚。一見鍾生之少年老成。}。梅生大笑。鍾生抖了抖補道袍。按了按舊紗巾。拔了拔破朱履。撢了撢身上灰塵\footnote{大約錢貴家中從未見此等打扮的大嫖客。此數語非極寫鍾生貧。不如此描盡一番寒態。不足以顯錢貴取之之奇也。}。鎖上了房門。同梅生出來。又鎖了院子門\footnote{細。}。遂同攜着手。一路說些閒話。彎彎曲曲。不覺已過朝天宮大街。到錢貴門首。只見一帶疏籬。數竿修竹。樹木掩映。一個小小靑門樓兒。迎門一座花臺。栽着一叢天竺。點綴着幾塊宣石。門口站着個丫鬟。約有十六七歲。生得面白唇紅。指柔足小。靑衫潔淨。黑髮光明。在那裡買花。梅生指對鍾生道。此幽舍乃錢娘居也。又指着那丫頭。笑顧鍾生道。兄未覩麗人。先見艷婢。只這一小鬟。也就算嬌美了。隨問那丫鬟道。你姑娘家中有客否。我同這位鍾相公特來相訪。那丫頭原來就是代目。梅生原常在他家行走過的。他却認得。將鍾生一看。不覺滿面春風。笑容可掬。忙向鍾生歛袵\footnote{寫出十分相愛的樣子。}。道。姑娘正獨坐無事。二位相公請進客屋內坐。我去通報。讓了進去。坐下。他忙到房中對錢貴道。恭喜姑娘。向日那梅相公同了一位鍾相公來訪姑娘。錢貴道。癡妮子。這有甚麼喜處。我今日心中不樂。懶於應酬。你可去回他說。得罪相公。改日再會罷\footnote{有此一頓。妙甚。後聽得是鍾生。方更覺欣喜也。}。代目道。姑娘不可錯過。我跟姑娘數載了。雖見過幾個俊俏郞君。怎如這鍾相公是天上謫仙。人間罕有。雖然衣敝履穿。窮酸打扮。但那一種風流。恐巧妙丹靑也畫不出。他才人丯韻。雖不知他才學何如。姑娘也該會他一會。大約世間有才而無貌者有之。有驚人之貌而無才者未必。姑娘一心想遇一個俊俏的郞君。今日却遇着了。我先說恭喜者。就是這個緣故。他比那祁公子不但風流過之。且另有一種藹然可親之態。較之他人就有雲泥之隔了\footnote{先是梅生在鍾生前極力贊錢貴貌。此是代目在錢貴前極力贊鍾生之貌。兩處一對。代目也算是一個媒根。}。錢貴聽了。笑吟吟的道。窮何妨\footnote{當年只有章臺柳謂韓翊云。韓夫子豈〔長〕貧賤者。今又聞錢貴道鍾生窮何妨。此三字不聞者多年矣。}。但可果然如你之所云。竟是這樣瀟灑風流人品。代目道。向蒙姑娘以心腹托我。我怎敢欺誑。誤姑娘的大事。錢貴想了一會。道。我常聽得人說。有一個小秀才叫做鍾麗生。算當今才貌雙全第一個人品。他因四壁蕭然。故閉戶在家苦讀。我雖神往久矣。却無緣相會。莫非就是此人。叫代目替他輕攏雲鬢。淡點硃唇。起身。喜孜孜扶了代目。慢移蓮步。款蹙湘裙。嬝嬝娜娜走將出來。朝上拜了兩拜。三人相讓坐下。梅生先開口道。久濶錢娘。渴想之甚。今我這敝友鍾兄因久慕芳名。特同來奉訪。喜錢娘今日得暇。誠爲三生有幸。錢貴道。賤妾葑菲下材。蒲柳陋質。怎敢當相公過譽。聞得鍾相公神仙中人。今得屈臨賤地。乃妾之萬幸耳。正說間。代目捧上茶來。三人吃罷。錢貴附代目耳道。快備酒飯。代目點頭去了\footnote{點頭二字妙絕。錢貴說話。只代目聽得。代目點頭。錢貴看不見。反是梅生鍾生看見。}。梅生顧鍾生道。兄今見錢娘丯韻。弟之前言妄否。鍾生道。弟先以兄之言恐其太過。今細看起來。兄之所贊尚未能盡錢娘之萬一。眞胡然而天。胡然而地(帝)。大約古來相傳之名媛。恐尚未若是。梅生對錢貴道。我這敝友鍾兄。表字麗生。是黌門中第一個才貌兼全靑年的才子。眞可謂倚馬千言。才華絕世。今日與錢娘初會。定有些新詩相贈呢。鍾生道。小弟不過背地吟哦。邯鄲學步。久聞得錢娘精通翰墨。小弟豈敢弄斧班門。錢貴聽說。果然是他數載神馳。聞名未會的那〔人。〕喜動顏色。忙笑答道。相公言重。妾久仰高名。如雷灌耳。眞如三神山。可望而不可即。今竟得相遇。何幸如之。妾陋質寡文。恐不敢當相公珠玉。或蒙不棄。賜我佳章。勝錫我百朋矣。梅生道。適間找到鍾兄府上。鍾兄正在豪吟。錢娘可要聽麼。錢貴欣然道。相公若記得。幸爲賜敎。梅生遂將扇上的詩詞念了與他聽。錢貴聽了。贊道。名下無虛。妾何幸得聆佳作。鍾生道。俚言粗鄙。有汚尊聽。令我愧殺。聞得梅兄說錢娘著作甚富。祈假一觀。錢貴笑道。拙作眞要汚目了。幸遇高明。敢不獻醜求敎。喚代目將他歷來所作的詩詞取出來。遞與鍾生。鍾生看了。贊不絕口。道。錢娘佳作。眞可擲地金聲矣。錢貴道。不但相公汚目。且使賤妾汗顏。梅生道。你二位皆不必過謙。俟酒闌後。等詩興發作。少不得要彼此賡和。正說着。內邊捧出酒肴來。彼此相敍坐下。觥籌交錯。賓主甚歡。擲了一回骰子。說了一回口令。郝氏也出來各奉敬兩杯。梅生暗暗把東道之資遞與他去了。錢貴又叫代目取過絃子來。彈着唱了一支紅拂記上虬髯落店的崑腔曲子。道。

\begin{quotation}

我看你丯姿灑落。儀容俊俏。自合雙飛雙宿。姻緣分定。千里非遙。多感你好逑君子。擇配佳人。一見相傾倒。我看你二人呵好一似秦樓乘鳳弄璚簫。可笑楊素那老頭兒他銅雀焉能鎖二喬\footnote{唱此調巧甚。儼然以紅拂自居。鍾生比李靖。虬髯比梅生也。}。

\end{quotation}

他玉指輕挑。檀唇慢吐。眞有繞梁裂石之音。令人聽得心曠神怡。唱了一曲。侑了數杯。看看日色將暮。酒闌上來。梅生道。有勞錢娘妙音。我們已賞鑒過了。鍾兄此時詩興動否。可作將起來。以助飮興。鍾生道。小弟拙作不拘何時皆可應命。但恐俚句不堪。有汚錢娘淸聽耳。錢貴道。相公勿過謙。定要請敎。遂自己到房中。取出一柄重金牙骨佳扇來。雙手遞與鍾生。道。求相公即將尊作揮於粗扇。賤妾當留爲終身珍玩。隨命代目掌上兩支大燭來。又自己進去了一會。代目捧着一個螺甸方盤。梅生鍾生看時。盤中放着一方端溪舊硯。一錠方于魯的佳墨。錢貴將一枝純毫湖筆遞與鍾生\footnote{錢貴不令代目。而兩次三番自己進去者。一見喜之至。一見重之至也。}。命代目將墨磨起。那梅生不住贊道\footnote{只梅生贊而鍾生不贊。深意妙絕。鍾生非不欲贊。因腹內打詩稿耳。}。不要說錢娘著作之妙。只這筆硯精良。也是難得見的。錢貴道。妾因目瞽。不善塗鴉。凡有拙句。俱是小俾代寫\footnote{照前代目竟還寫得來語。}。此妾特特製下。以待高賢。藏之數年。今日得遇鍾相公佳作。何(可)謂筆墨之幸。亦見妾一段苦心之有靈也。鍾生道。錢娘可謂深情。敢蒙錯愛若此。因提起筆來。蘸濃了墨。要逞才思。不假思索。一揮五首。

其一。

\begin{quotation}

雪兒饒綽約。惆悵隱秋波。

蜜意流纖指。柔情託緩歌\footnote{妙。謂先歌時以紅拂自喩。以鍾生比藥師也。}。

看勻深淺黛。裙織綠緋羅。

話到傳心處。明眸愧爾多\footnote{罵盡不識人者。}。

\end{quotation}

其二。

\begin{quotation}

閉目如思婦。開喉盡妙歌。

動人年最小。謔客趣尤多。

不飮頻呼酒。催乾欲捲波。

醉餘偎倚處\footnote{想當然。}。香氣透春羅\footnote{映其時季春之交。江南天暖。俱穿羅衣。}。

\end{quotation}

其三。

\begin{quotation}

不見偏能識。心靈會晤多\footnote{二句贊盡錢貴一生。}。

愛傳絃上調。情露坐間歌\footnote{又我前意。}。

花好藏深髻\footnote{贊其髮。}。肌香透薄羅\footnote{贊其軀。}。

餘思何處覓。去去緩凌波。

\end{quotation}

其四。

\begin{quotation}

天意何幽渺。盈虛事頗多。

旣然予月貌。曷以吝秋波\footnote{此四句旣痛惜又深恨。無可問者。歸之於天矣。}。

淡鎖吳宮恨。輕披越國羅。

浮杯一繾綣。況復有淸歌。

\end{quotation}

其五。

\begin{quotation}

無意逢佳麗。風情動我多。

軟腰欺嫩柳。柔體怯輕羅。

玉指挑新調。朱唇吐艷歌。

花魁應避步。何必在秋波\footnote{此想更深一層。謂有如此見識。何必用目也。有眼而不識人。又要之奚益哉。}。

\end{quotation}

寫畢。梅生接過來朗誦一遍。贊道。兄之佳唱。精工敏捷。雖靑蓮復生。不能居兄之右。非兄不能有此咏。然非錢娘亦不能當此贊也。絕色高才。可稱二美。眞是千秋佳話。小弟有幸得預斯會。錢貴聽了。忙出席深深拜謝。命代目斟上二巵。自己雙手奉一巵與鍾生。道。賤妾慕才如命。今幸得遇相公。乃前緣所致\footnote{語中已含深意。}。但蒙過獎垂憐。妾不能當此耳。僅敬一觥拜謝。又奉一巵囗(與)梅生。道。承相公不棄。同鍾相公來賜顧。遂妾數載之願。薦引之恩。亦當拜謝。梅生道。此係鍾兄與錢娘宿緣所致耳。我不過偶介紹乎其間。何足居功。焉敢當謝。鍾生亦回敬錢貴一巵。道。小生乃貧寒下士。親友皆所不齒。今錢娘見愛若此。可謂生我者父母。愛我者錢姑也\footnote{鍾生初遇錢貴。不懼其鄙薄貧寒。便吿以心腹實話。錢貴即知其爲誠實君子矣。焉得不願托終身。}。敢不爲知己謝。錢貴道。相公是何言也。韓夫子豈長貧賤者哉。妾得遇相公。實出萬幸。彼此遜謝一番。大家飮畢。錢貴叫代目取出一方新紬帕。將扇子包好。收入匣內\footnote{鄭重之至。}。他先聽得代目說。鍾生果然容貌無雙\footnote{過(果)然兩字。看者極容易忽略過去。謂當日之是耳聞其美。或其未必眞。今經代目見之。果然是實耳。}。與向來所聞無異。今覿面又見他才美若此。不勝心折。就存了一點要托終身之意\footnote{此時從良之心方十分決定。先雖有七八分相愛。因未聆其才。尚未敢決。寫他心事深淺都有層次。}。只是一時不便開口。而那一番綢繆之意。甚是慇懃。梅生見了。笑道。我聞得錢娘數年來無一人得其歡心。今遇鍾兄即相愛若此。眞是姻緣宿定。非人力所能強。錢貴道。妾何人斯。敢雌黃人物。但從幼有誓。願得遇一個才貌兼全的情郞。今遇鍾相公已符宿願。敢不致敬。梅生道。鍾兄。我看錢娘可謂愛兄已至。兄今在此留宿何如\footnote{眞好撮合山。}。鍾生道。小弟寒酸體態。怎敢伴天上姮娥。今承錢娘不棄。只可做詩酒交。安敢結鸞鳳侶\footnote{古謂聆音識意。今鍾生數語。已滿心願留。但自鄙形穢。不知錢貴心內如何。故語謙而不決烈。然而好色人之所慕。況係煙花。鍾生雖少年老成。而心非木石。豈能不稍動也哉。}。錢貴滿心要留他。不好驟然啓齒。今聽見梅生相勸。心喜非常。見鍾生推辭。忙道。妾乃娼門下賤。怎敢汚相公玉體。但得侍一宵衾枕。雖於九泉亦無遺恨\footnote{至情語。}。說了。面有慘(慚)色。梅生道。錢娘之言若此。吾兄若要推辭。豈不辜錢娘一團美意。倘再拘泥。不但殺風景。就覺太不情了。弟且吿辭。明早再來扶頭。因起身作別。鍾生見他二人如此說。也就立住。道。人非草木。孰能無情。非弟推辭。但只恐無福消受耳。說完。與梅生作別。送了出門。隨與錢貴攜手進房。見房中焚蘭爇麝。幽雅非常。繡帳錦衾。又富麗至極。鍾生雖是一個才子。却是一個寒儒。每常住的是衡門茅屋。睡的是紙帳梅花。今到此溫柔鄕。如登仙界。他此時眞是。

\begin{quotation}

身雖未到蟾宮裡。如在瑤臺瓊室中。

\end{quotation}

錢貴又叫代目烹了一壺好茶。各吃了兩鍾。說了些久聞未會的知心話。鍾生在明晃晃銀蠟下重新把錢貴細細一看。燈下看佳人。分外嬌嬈。眞美麗也。

\begin{quotation}

鬒髮如雲\footnote{髪。}。黑臻臻挽一個時樣梳粧。柔軀似柳\footnote{軀。}。嬌滴滴着大套細輕衣服。眉彎新月\footnote{眉。}。淡淡掃兩道春山。牙排嫩玉\footnote{牙。}。齊齊露兩行瓠子。雙眸似睡\footnote{眸。}。如未醒之楊妃。嬌面不勻\footnote{面。}。似嫌涴之虢國。鼻若垂珠\footnote{鼻。}。臉同瓜子\footnote{臉。}。口中香氣氤氳\footnote{口。}。唇上殘脂腹(馥)郁\footnote{唇。}。十指尖尖\footnote{手。}。眞如玉筍。雙彎窄窄\footnote{足。}。實賽金蓮\footnote{錢貴之美。豈獨鍾生今日始見之。數年來他人皆無目者耶。要知他人眼中見錢貴如此。不足盡錢貴之美。鍾生雖是男子。貌勝婦人。他見錢貴尚美如此。可謂美之至矣。此不但贊錢貴。連鍾生都贊在內中也。}。

\end{quotation}

相攜上床。脫衣共寢。鍾生又將他遍身細細撫摩。眞是。

\begin{quotation}

體滑如脂。骨溫如玉。上口似櫻桃。下口包含紅芍藥\footnote{喩其色。}。橫唇如赤豆。直唇微露紫雞冠\footnote{喩其形。}。乳頭新剝雞頭肉。捏着已足魂消。牝戶劈開菡萏瓣。摸到勃然興發\footnote{自頂至踵。無不贊到。獨於此處却不曾十分大贊。妙極。更有妙者。鍾生摸着此物如此。便以爲婦人如此皆是。並不知未破瓜之處女却非如此也。是寫一個乍近女色的少年。}。

\end{quotation}

情致如火。雲雨起來。一個初嘗滋味。一個久慕丯標。一個憐才。一個愛色。他兩個彼此相愛之情。一番綢繆之態。雖浴水鴛鴦。穿花鸞鳳。猶不足以喩也。事竣就枕。錢貴枕鍾生之臂。悄語道。妾有心腹一言。欲君見憐。君肯垂聽否。鍾生道。卿之深情。沁我肺腑。有何見敎。敢不勉從。錢貴道。妾乃錢家親女。不淫(想)隸在樂籍。這接客迎人。原非妾之本意。奈迫於父母之命耳。妾今雖倚門獻笑。然自幼曾立一誓。願得遇才貌郞君。定以終身相許。妾今虛度十九齡矣。數載做這風中柳絮。也因是未得其人。今遇郞君。妾心已足。若徒效露水之歡。非妾之願。必以此身相托。誓死不移。倘鄙妾下賤煙花。留爲妾婢。亦所甘心。君若不從。妾當以一死。自矢此志。決不他移。君能憐念妾否。言畢。不覺嗚嗚咽咽。哭將起來。有八句道他二人。男貪女色。女慕郞才。道。

\begin{quotation}

爲雲復爲雨。相愛又相憐。

美配當良夜。佳期正妙年。

撫郞郞似玉\footnote{撫字妙。眼看不見其貌。但用手摸。}。覷女女偏妍。

更有銷魂處。低低枕畔言。

\end{quotation}

鍾生聽了。惻然道。卿可謂交淺言深。但我自幼父母雙亡。爲兄所棄。家徒壁立。親友皆疏。向來幾次求婚。人皆鄙我寒賤。故年已二十。尚無室家\footnote{鍾生錢貴是一部書中之正生正旦。故寫他二人獨詳。前寫錢貴自生時至襁褓便有人贊愛。後七八歲上學攻書。十齡損目。十三歲爲鐵化梳籠。今十九歲得遇鍾生。鍾生也是自生時至五歲便能識字。七八歲就便爲(會)作文。九歲喪父。十一歲喪母。十五歲自外祖家出來另住。十七歲進學。今二十歲得遇錢貴。何以(似)太史公之年月表。自他二人之外。再無第三人費若許筆墨者。}。我因想書中自有黃金屋。書中有女顏如玉。故立志芸窗。矢心發憤\footnote{錢貴矢心擇配。鍾生矢心發憤。二人皆得如願所爲。有志者事竟成也。}。或皇天不負苦心。倘獲徼倖。再尋配偶。今幸得遇芳卿。承你百般垂愛。我心已醉。感你以終身相托。何幸如之。本擬如命。但我一介寒儒。恐負你終身結局。二則我囊罄如洗。焉能爲子贖身。三則你係他親生愛女。安肯輕易配人。四則我原說徼倖之後。方可畢婚。今豈有出乎反乎之理。且我一個薄命寒儒。焉有福配你這天姿國色。因此數種。故難從命。賢卿請自細思。錢貴道。以郞君之才。蛟龍豈池中之物。不日升騰。這何足慮\footnote{破其寒儒句。}。至於贖身一事。妾係他親生之女。安得論價\footnote{破其贖身句。}。且妾數年來替母親所掙不下千金。若定要身價。妾當自辦。不用君費心。若說親女不肯捨輕易嫁人。當初妾原不肯接客。是我母親苦勸。原訂過得遇才郞許我自嫁。向有斯言。我方依允。今若萬不肯從。妾當誓以一死\footnote{破其親女不捨配人句。}。今日旣已侍君。此身決不再辱。妾心已死於君。自茲以後。生爲君家之身。死則君門之鬼矣。君所說脫却藍衫。方纔納偶。今日我不過欲爲君妾足矣。豈敢望與君作配\footnote{破其徼倖後方娶句。數語釋破鍾生前數語之疑。}。何妨今且歸君。爲君權主中饋。亦可免分君讀書之心。俟君捷後再覓夫人未遲。妾籌之熟矣。君能憐念妾否\footnote{此數語雖非本心。不過謂此身即爲小星亦願。一以明己志之堅。二欲感鍾生之心不能復辭耳。眞慧心。}。鍾生感激不盡。道。子言至此。可謂深心。我尚有何推阻。但你說今且相從。倘我僥倖。再尋匹配。此言非知心人當出口。我有何能。承你這般厚情。誠令我感激泣下。我自然以你爲室。豈有列做小星之理\footnote{先破他這一句。妙。不但謂你不當說。且我不願聽也。愈覺情深。}。但今日若與你老母言之。他見我一介寒儒。未免有許多張致\footnote{洞見小人肺腑。}。你且不必露於辭色。俟今秋大比。或上天憐我二人情癡。稍得寸進。然後娶卿爲室。不幸即落孫山。又當設法別議\footnote{勿謂鍾生情種。即鐵石人見錢貴如此一番相愛。亦不忍辭他。此二語娶之之心亦決。}。錢貴道。聆君之言。妾之深願。況數月光陰亦容易過。但恐君高中後。那豪門閨秀。富室嬌娃。誰不願得此風流佳婿。恐致妾有白頭之嘆耳\footnote{雖未必疑鍾生是此等人。然不得不慮及於此。不若先說破之爲妙也。}。鍾生長嘆了一聲。道。我命名鍾情。豈肯作薄倖人。況女子中尚有多情美麗如子者耶。若異日負卿。我終身前程不吉\footnote{此數語破他另娶之疑。又自明決非負情者。}。錢貴聽了。忙欲披衣起謝。鍾生摟住道。你我何須乃爾。但你此後仍如昔日承順母意。俟到我家。再守婦道未遲。錢貴道。君此言視妾同畜類矣。我旣以此身許君。此身乃君之身矣。敢有辱君之理。若母親不念天倫。或行威逼。妾九死弗移。以此報君\footnote{只見錢貴三志之堅。伏後姚澤民來訪時。}。鍾生道。我正恐如此。故爾勸你。我二人旣已定盟。便是終身夫婦。倘你不堪受凌辱。如此豈不使我抱一世鼓盆之嘆。況你之心跡。我豈不知。俟出火坑。再做良家腔調未晚\footnote{此數女(語)鍾生之情更深一層。可以死錢貴之心。}。錢貴道。君情至此。妾雖死九泉。亦含笑矣。因笑道。我錢貴好造化也。得此多情多義才郞。終身之願已足。又對鍾生道。目今郞君請寬住數日。聊盡微忱。此後無事望常來看。免妾身記懷。鍾生道。我豈忍瞞卿。我家一貧如洗。此地豈能常到。且大比在邇。還要用功。若有稍暇。自來看你。不必注念。錢貴道。君高志若此。妾豈敢擾亂君心。今求寬住數日。稍伸繾綣。若恝然別去。情何以堪。鍾生應允。二人相敍到親厚之際。情興復萌。重又春風一度。正在綢繆之時。不覺天色已曙。日映紗窗矣。二人起身下床。鍾生將他一看。眞個消魂。但見。

\begin{quotation}

雙眸雖緊閉。顏色勝芙蓉。

月掃娥眉淡。雲偏寶髻鬆。

\end{quotation}

又看着錢貴梳洗。親爲之掠鬢。代爲之畫眉。一種親愛之情。不能言盡。梳洗方畢。只聽得梅生一路叫進來。道。鍾兄起來不曾。小弟來扶頭了。鍾生忙迎出來。道。吾兄來何早也。梅生笑道。弟恐兄乍入陽臺。好夢不能即醒。特早來驚夢耳。相視大笑。到堂屋中坐下。代目捧出兩盞茶來。二人吃了。梅生攜了昨夜嫖金。今日東資。交與代目。代目進房對錢貴說。錢貴不肯收。叫代目定還了梅生\footnote{此一事決不可少。不然鍾生白嫖固不可。自出嫖金又不能。昨日是梅生勸他留宿。今日代出。方是知己。錢貴不受。他二人私下定盟。則非梅生所料矣。}。梅生只得收回。少頃。錢貴出來同坐。早飯畢。談了一會。又拿出酒肴來。三人入席而飮。無非說些新詩。行個妙令。且說郝氏昨日見了鍾生。看他衣衫襤褸。甚不慊意。因女兒叫備酒飯。少不得整理送出。後接了梅生東道之費。也還不十分着惱。以爲他到晚就去。不想女兒竟留下了他。不見一文宿錢。滿肚忿氣。正是。

\begin{quotation}

未曾見慣奇嫖客。惱斷虔婆愛鈔腸\footnote{實在未曾見慣。怪他不得。}。

\end{quotation}

今日又見女兒自己拿出私囊製東。越發氣得了不得。因看女兒面上。不好發話。惱得只在他自己臥室坐着。總不來瞅睬。一應都叫代目財香\endnotemark[1]料理。不在話下\footnote{描寫盡鴇兒愛鈔。小娘愛俏兩種心事。}。他三人飮過數巡。梅生問道。兄今日可回府麼。鍾生道。小弟也要回去。蒙錢娘苦苦相留。不忍拂其雅情。還住一日。梅生笑道。諺云。得魚豈可忘筌。你二位如此相親。何以謝我這月下老。他二人同應道。多感厚德。容圖後報。決不敢忘。今且以一巵爲壽。二人起身。各斟一巵。奉與梅生。梅生笑着立飮了。又皆回敬坐下。梅生又問道。鍾兄遇錢娘。昨已有新詩相贈。錢娘可有佳章酬答否。錢貴微笑道。鍾相公佳作。陽春白雪在前。妾巴人下俚之言。豈敢相和。因鍾相公說自幼貧寒。爲親友所不齒。妾見世態炎涼。人情冷暖。不勝慨嘆。謅得一調木蘭花慢。不敢獻醜。恐相公噴飯。梅生道。錢娘不必太謙。就請賜敎。錢貴遂念道。

\begin{quotation}

想人生貴賤。皆前定。有何妨。嘆人盡欺貧。衆咸趨富。出醜張狂。思量從來世事。儘多更何必恁匆忙。富貴焉知不敗。貧窮豈便無昌。悽惶。有限幾時光。誰弱又誰強。復何須乃爾。千般醜態。萬種無良。推詳事多反覆。況人生怎定得滄桑。堪笑人皆睡夢。安能洗盡汚腸。

\end{quotation}

梅生聽了。道。妙極妙極。罵盡世情。錢娘眞鍾兄之知己矣。又向鍾生道。錢娘旣有佳作贈兄。吾兄不可無答。或詩或詞。也請敎一首。鍾生道。旣承兄命。敢不呈醜。弟荷錢娘厚愛。亦有數言以謝之。故美其名曰意難忘。鄙言誌意而已。幸勿大噱。遂念道。

\begin{quotation}

漂母流芳。憫王孫進食。義俠充腸。章臺英俊眼。貧賤識韓郞。紅拂伎。目非常。奔李靖歸唐。適蘄王。梁妃顯達。千載稱揚。負羈哲婦無雙。識文公終復。傑士從亡。逃吳胥乞食。浣女獻壺漿。豪傑事。屬閨房。試說姓名香。到今朝。垂靑顧我。又有錢娘。

\end{quotation}

錢貴道。妾何人斯。何敢當郞君如此高比。所謂投之以木桃。報之以瓊瑤了。叫代目取出筆硯。並一幅白綾。請鍾生寫。鍾生將錢貴之詞寫於前。他自己的寫在後。寫畢。梅生接過。念了一遍。贊之不已。錢貴道。以妾之俚語與鍾相公尊作同書。眞正是精金配頑鐵。美玉並瓦礫了。梅生道。你二位都不必謙。兩調佳章。若傳出去。都可紙貴洛城。錢娘何不以此兩調被之新聲。長歌一番。我們洗耳靜聽。何如。錢貴欣然應允。各送巨觥。先將鍾生的詞歌了。二人飮畢。梅生酬了一杯。歇息了一會。又各送上酒。錢貴又將他的詞歌了。二生大喜。彼此歡飮酬酢。飮至天晚。梅生別去。鍾生錢貴二人。如並蒂芙蕖。穿花蛺蝶。百般恩愛。又住了一日。苦辭要回。錢貴知不可留。遂在篋中取出銀一封。道。此內約有三十餘金。係妾向來所積。今贈君權爲燈火之費。若有不敷。將來再取。妾倘有衷腸欲訴。托人請君。望君即至。鍾生道。卿若見招。我必就到。但你之情愛。我已難當。此贈如何好受。錢貴道。君何外妾。妾身旣已屬君。況此身外之物。妾之所有。皆君之所有也。鍾生感其言。也就收下。二人依依不捨。攜手流淚。錢貴又道。郞君萬分自愛。秋闈後妾當洗耳以聽佳音。鍾生道。卿亦當自愛。前言須緊記。萬不可因我而受辱。使我愈不自安。彼此鄭重而別。正是。

\begin{quotation}

無眸瞽妓。勝於有眼男兒。

鬚眉丈夫。不若巾幗女子\footnote{大書特書。此二句是一部書大主意。}。

\end{quotation}

且說鍾生到了家中。開門進去。他這間房子。原是那老先生眞佳訓的書室。這眞家訓後出了貢。選了敎官。一家數口俱帶去上任。此房典與鍾生。其價甚廉。只當替他看房子一樣。雖然是間斗室。四面俱有小院。院中還有幾棵綠蕚西府。碧桃紅杏之類。他室中竹床木几。紙帳布衾。裡外倒也還收拾得十分乾淨。鍾生素常在家時。因貧窮特甚。三旬九食。也是他的常事。但無長遠枵腹之理。少不得終日要去奔波柴米回來。又要親躬汲爨。做那竈州府的炊官。還要掃地澆花。一日中只好半日讀書。今日錢貴贈了他一封銀子。他就坐下來。打開一看。都是上好錠兒。不覺墮下淚來。道。我自幼椿萱見背。兄嫂將家私變賣。不知何往。依傍了外祖數載。後外祖先逝。虧得與我些私蓄。纔覓了這間房子棲身。並盤纏了兩年。數載來。多承梅兄間有所贈。以佐薪水。纔苟延到了今日\footnote{此處提明。後日千金之報方不爲過也。}。其餘骨肉至親。盡同陌路。不意今日與錢姑無心之遇。不但贈我若許之資。且以終身相托。此情此德。沒齒難忘。我趁此有餘之時。可以苦攻。今秋倘百尺竿頭。得進一步。完他終身大事。就是報德了。次日到書鋪廊買了許多墨卷表論策判之類回來。又製了幾件隨身的衣履\footnote{此句伏得好。不然後來那得一衫一褲贈郗氏。}。備了數月的柴米。恐自己炊爨。誤了讀書之功。雇了一個江北小廝。叫做用兒。來家使喚\footnote{即帶出鍾用。妙。}。每日工價一星。他然後自己擬了些題目。選了些文章。足跡總不履戶。只有會文之期纔出去。閒常只埋頭潛讀。眞是雞鳴而起。三鼓方歇。以俟秋闈鏖戰。權且住筆。鍾生前日在書坊中見一冊新書。名曰峒谿備錄。翻開一看。係本京新安人姓童名自宏\endnotemark[2]近日的著述。他也買回來閒閱。你道這童自宏是誰。他就是童自大的胞兄。與他乃弟的胸襟大不相同。滿腹文章。却不願出仕。一意陶情山水。愛閱歷名山大川。民風土俗。他家中也是巨富。將家事付與兒子主持。只在外邊遊覽。有人勸他道。何不在家享用。常常奔波道路。何苦乃爾。他道。大丈夫志在四方。豈有個做看財奴。守這故園空老\footnote{此等財主。吾見其語矣。未見其人也。}。一日想道。東西兩粤。吳楚秦蜀。我都曾遊過。只不曾到過滇黔。我聞得苗蠻之地雖近中原。而人畏其險峻。細探之者甚少。我何不一遊。把蠻中風景紀出一段故事來。不但自己豁了心胸。也可留爲後人長些見識。決意要去。親友咸勸阻道。苗蠻煙瘴之地。何可因遊觀之小事而輕萬金之軀。寧不聞千金之子。立(坐)不垂堂乎。他笑道。如諸君之言。床榻之上。屋宇之中。皆不死人者耶\footnote{達者之見。}。遂帶了數個家人。攜了若干途費。到了南京。在童自大家只住了一日。見兄弟那鄙嗇的樣子。十分難看。遂遷到朝天宮道士房中作寓。那時應天府學敎授姓廣\footnote{第九回內廣敎官薦干生到李太家。此處已伏其人。}。祖籍徽州。與童自宏原是社友。當日在家時甚是契合。今到此處。次日即去拜訪。廣敎官聽得他來。忙倒屣迎入。敍了許多久別渴想的話。又閒談了一會。童自宏見他的學署牆欹壁塌。甚是不堪。說道。社兄在此爲一方之師範。怎魔(麼)貴署傾圯至此。也不申呈府縣修理一修理。廣官嘆了一口氣。道。豈但弟之敝署。連聖人的大成殿同兩廡都有倒漏處。曾呈稟過數次。皆置若罔聞。奈何。昨日正有一個笑談。弟與兩位敝同僚在那裡同閱諸生的月課。門斗進來說道。外面牌坊上那個掉下來了。弟不懂所謂。問他掉下的是甚麼東西。他說。就是那個了。我知道叫甚麼。弟還罵他道。死蠢材。必定有個名色。甚麼那個那個的。遂出去一看。原來是牌坊柱子上那瓦套兒。因柱頭朽了掉了下來。弟也不知叫做甚麼。只得解嘲。向門斗道。這個掉下來就是了。儘着那個那個的。我如何知道。後來各書去查。始知名叫護朽。老社翁請想。一個文廟大門外的牌坊。乃衆人觀瞻之地。而(尚)且如此。又何況於他。童自宏顧家人道。拿五十兩銀子送廣師爺收拾房子。家人取出送上。廣敎官道。老社翁駕臨。弟連一杯薄酒還不曾奉敬。怎敢當此厚賜。然不敢過却。有負雅愛。此屋雖弟居。乃官舍也。弟定將老社翁這一番義舉申報上臺。童自宏道。此萬不可。弟非沽名者。不過贈故人稍加修葺。以蔽風雨耳。廣敎官領諾。作謝收了。童自宏別了回寓。廣敎官即刻回拜。次日設席奉請。他自知童自宏尚樸素。不喜虛華的人。請了兩三個得意的窮門生相陪。彼此談講。甚是相投。童自宏寓中無伴。約他們常去。以消寂寞。這兩三個秀才知他是好客的富翁。何樂而不往。便日日到他寓中陪談。大嚼豪飮。那是不消說的\footnote{到聽日日到朝天宮陪那道士。這兩三個秀才日日到朝天宮陪童自宏。遙遙一對。}。一日。童自宏同他們到三山街承恩寺閒步。見許多的骨董鋪。遂挨着家看去。並無一件好物。看到一家。還有幾件看得的東西。他衆人中有一個朋友。見一個匣內放着一隻玉椀。便伸手取過來看。那開鋪子的。先見他們幾個都是酸丁打扮。料非售主。坐着揚揚不睬。此時見他拿椀。忙站起來說道。哎呵呀。看仔細。好閒賤手。遠遠的看看罷了。一下失錯打掉。你還賠得起麼。便伸手來奪。童自宏見他小量那朋友。心中暗怒。便一手接過來。問道。你這椀値多少銀子。就敢量人賠不起。那人見童自宏說這話。估了他兩眼。見他穿着也甚是平常。料不是主顧。遂冷笑了一聲。道。要是別人買。一百八十的要。相公你若要。讓你些\footnote{買賣小人小量。人猶可恕。稱呼這幾個你字。則可惡難忍。寫盡小人勢利的心腸。}。稱二十兩現銀子。拿去了罷。童自宏聽了這話。拿着向街中石上儘力一下。摜得粉碎\footnote{陳子昻摔胡琴是博名。童自宏摜玉椀是出氣。然而兩件事都暢快。}。吩咐家人道。稱二十兩銀給他\footnote{余有一李姓長〔輩。〕新任江陰副總。新歲到省謁制臺。因往評事街燈市看燈。裝束如兵丁形常(象)。見一家列紗屛。一架花梨架甚精工。問道。這架屛要賣多少銀子。那賣燈的道。你料道買不起。問他做什麼。又一個笑道。便自送你。恐你家還沒處放。你若愛。稱三十兩銀子。擡了去罷。李公家即在省城。回來差四名軍卒。拿了三十兩銀去擡屛。吩咐云。他若不肯。可將兩個掌櫃的拿來軍牢。到彼言其故。二人自悔無及。只得將屛付與。二事相同。故並及之。以快心胸。}。那人爭道。這是人的寄賣的。定要五十兩。昨日人還到四十兩。尚不曾賣。如何摜碎了他的。先那朋友被他譏誚了兩句。一肚暗氣發洩不出。今見童自宏摜碎了。心中暗喜。便說道。你要二十兩。他就給你二十兩。還有甚麼說的。你先貶淺我罷了。他是徽州有名的百萬童老爺。像你這樣的鋪子開得起幾萬個呢。你也小量他。這條街是極熱鬧的所在。此時圍着許多人看。這朋友向衆人細說了其故。衆人一來也惱他眇(渺)視人。二來人情所使。自然要奉承富翁。都說開鋪子的不是。他方忍氣吞聲。沒得話說。童自宏同衆人談笑着踱出聚寶門外。到了報恩寺。走乏了。投知客寮去。只見挪一個大胖和尚。肥頭大臉。穿着一身紬緞僧衣。光着頭。坐在一張大圈椅上。見了他們。屁股略擡了擡。道。請坐。他衆人也都坐下。那和尚毫不瞅睬。也不叫茶。童自宏見他那樣子可惡。笑問道。老師就是知客麼。那和尚帶答不答的道。正是。童自宏道。請問這報恩寺以前是甚麼寺來。知客道。以前是長干寺。童自宏道。長干寺以前呢。那和尚忙(茫)然了一會。道。這却不知。童自宏笑道。寶刹也算南京第一大寺了。無限的貴官財主來往。像我輩窮酸不足論了。倘遇了那種人盤問起來。連本寺的來歷都不知道。不但於寶刹削色。就是有願布施的也不肯出手了。那和尚問道。相公可知道麼。童自宏道。我安得不知。那和尚忙立起。滿臉陪笑。足恭問訊道。適纔着實得罪。小僧以爲是等閒人。不知是廣見博識的老先生。叫小和尚送茶。茶罷。就叫掇果碟子上來。一十六樣上色果品細點。再三讓着。吃了一會。又叫備齋。頃刻撤下果碟去。送來十二碗豐盛素菜。包子雲捲。南鄕米飯。細粉鮮湯。吃飯畢。又叫烹了一壺好毛尖茶來。漱了口。那和尚笑吟吟躬身問道。請問老先生。敝寺長干寺以前端的是甚麼寺。童自宏道。當年梁武帝要建長干寺。特選了這一塊地基起蓋的。長干寺以前是一塊大空地了。這有甚麼難解處。衆朋友先也以爲童自宏必知其詳。都側着耳朶聽。見他說這話。都忍不住的哈哈大笑。那和尚先當童自宏是實話。陪了無限的小心奉承。備茶果。備湯飯。盛款了要請敎。此時方知是耍他。又說不出口。心中暗急。光頭上的汗珠有指頂大。順着往下滴\footnote{寫和尚一路屁滾尿流的奉承請敎。原來是這句話。焉得不急。偶憶一笑談。一個僧冒雪歸家。到屋內。雪花頭上滴水。徒弟問道。師傅頭上是那裡的水。師云。是雪洩了。此僧頭上大約也是洩了。}。童自宏笑着起身一拱。道。多擾了。笑着同衆人別處去隨喜。吩咐家人道。稱二兩香資送這師傅。那家人便向身邊取出一包銀子來稱。那和尚見給了二兩銀子。除茶飯之費。還多餘兩數。方纔暗喜不急。因見他這樣出手。不像個窮酸。問那家人道。你們這位相公姓甚麼。在那裡住。口聲不是我們本地人。那家人道。我家老爺是徽州有名的童大百萬。你們這城裡住的童百萬就是他的親兄弟了。那家人也惱他出家人先那大樣。說他道。他先來時。你不那大模大樣。奉承得他快活。要化他一千五百。只當氈子上去了一根毛。說着。連忙趕主人去了。那和尚後悔無及。後來倒也敎乖了他許多。再不敢以衣帽相人。不論貧富人來。都以禮相待。按下不表。那童自宏在城裡城外各僧房道院遊了月餘。買舟而去。或水或旱。到了貴州雲南一帶。住了年餘回來。果然紀了一册手抄。名爲峒谿備錄。遂命匠人刻了絕精的板刷印。傳到各書坊中都有。腹中稍有文墨者。無不喜閱。獨他乃弟不善。他令兄帶了數十本來與他。童自大翻開一看。大笑道。花花綠。綠綠花。一個字。兩個叉。他認得我。我不認得他\footnote{人生在世。要認得銀子足矣。何必要認此。}。又笑道。有用有用。付與一個管賬目的小廝。叫做美郞。道。留着覆醬瓶蓋醋甕。也省幾文錢買紙。不要可惜抛撒了。你道端的這本書上紀的是些甚麼。聽我細細述來。上面道。

\begin{quotation}

峒谿種類不一。聞見同異各殊。余係目覩。辭雖簡而事詳。苗人。盤瓠之種也。盡夜郞境多有之。有白苗。花苗。靑苗。黑苗。紅苗。其衣各別以色。散處山谷。聚而成寨。睚眦\endnotemark[3]殺人。仇報不已。故諺云。苗家仇。九世休。近爲熟苗。遠爲生苗。熟苗徭役之苦。勞同牛馬。男子椎髻當前。髻纒錦帨。織布爲衣。竅以納首。婦人以海〖貝巴〗銅鈴\endnotemark[4]結纓絡爲飾。耳環盈寸。髻簪幾尺。以十月朔爲歲首。揉魚肉於木槽祭盤瓠。羣號以爲禮。見流官。無論尊卑。皆稱曰老皇帝。稱內地人曰漢人。以漢始通西南故耳。九股苗在興隆凱里二界。以十一月爲歲首。楚王馬殷遣將鎭八番。遂成土著。多樓居。衣靑衣。婦人被細摺裙。摺如蝶版。古致可觀。以六月六爲正旦。其俗尚鬼。喜造蠱毒。身帶刀弩。多爲盜賊。喪食魚蝦而禁禽獸之肉。葬則以傘蓋屍。期年發而火之。宋家蔡家。春秋宋蔡二國之裔也。性樸不詐。衣冠盡廢。宛然苗類矣。夭苗多周後。姓姬。尚行周禮。祭祖推其家長唱土語讃祝。紫薑苗裝束與漢人同。多力善戰。亦曉讀書。嗜殺尤甚。得仇人。生啖其肉。夫死。妻先嫁而後葬。曰。喪有主矣。賣爺苗在白納。賤老貴少。雖父老亦拽至他方賣之\footnote{不知誰人買這老者何用。}。克孟牯羊二種。處於金筑。擇懸崖鑿竅而居。高百仞。\endnotemark[5]或垂竹梯。或緣藤上下。如同猿狖。西苗尚勇好鬪。葬不用棺。不知拜掃\footnote{此是效法上古所行。}。飮醉相殺。醒復相好\footnote{國中雖不至於殺。而醉後相打。醒後歡好者甚多。}。東苗性悍。衣藍短衣。婦着花衫。無袖。遮覆前後而已。細摺裙僅蔽其膝。䮾氏之裔。死用棺。以石作墳。以七月七日祭先。甚敬。四龍家衣尚白\footnote{回回遺製。}。喪服易之以靑\footnote{諺云。穿靑衣戴孝。死鬼肚裡明白。大約因此而云。}。有張劉趙三姓。一曰大頭龍家。男以馬牛尾鬣雜組髮中。盤之成蓋。覆以尖笠。一曰狗耳龍家。婦人作髻。狀如狗耳\footnote{近日婦人挽長髻如騾腎。不知當作何呼。}。一曰小頭龍家。一曰曾竹龍家。俗與龍家無異。土人在新添司者。與衛人通婚姻。漸染漢俗。在施秉者。播入流裔。在邛水者。鬪狠輕生。里人亦名夭苗。身衣木葉\footnote{省了許多布帛。}。新添丹行之間。蠻人性獷戾。以漁略(獵)爲生。衣簑衣。峒人以苗爲姓\footnote{好個大族。}。性喜殺。片言不合。即起干戈\footnote{尚強如中國人腹內之干戈也。}。在石阡朗溪二司者。多類漢人。在永從者。居常負固在洪州。地頗膏腴。然不事耕作。惟喜剽掠。粤西有狪人者。好彈胡琴。吹六管。女善漢音楚歌。生女還之母家。曰。一女來。一女去。八番其俗。女勞男逸\footnote{夜則男勞女逸。庶可相均。}。勤\endnotemark[6]於耕織。長裙曳地。白布裹頭。以十月之望爲歲首。葬不當晝。必於靜夜。曰。不忍使親知之也\footnote{這纔叫做瞞鬼。}。犵兜衣靑。身不離刀。狇老叛\endnotemark[7]服不常。死則俯屍側葬。云。爲死者避壓也。〖犭羊〗獷生理苟且。荆壁無門。出則以泥封戶\footnote{何不憚煩。}。父母死。焚其衣冠。有如贈鬼\footnote{此俗近來盛興。}。僰人號十二營長。玀鬼犵狫言語不通。僰人爲之傳譯。被氈衫。女吹篾。有淒楚聲。六月二十四日星回節。吃生肉。祭天過歲。朔望日不乞火。性悍好鬪。盧鹿同風。又好佛。手持數珠。善誦梵咒。有禱輙應。〖犭罷〗。僰人後。住元謀。女負擔。男抱兒。\endnotemark[8]最潔。日杵米。不食宿糧。其人能咒咀(詛)變幻報仇家。又善變犬馬諸物。又有二形人。上半月爲男。下半月爲女\footnote{近日中華國少年。晝則爲男。夜則爲女。甚多。}。犵狫其種不一。有花犵狫。紅犵狫。赤脚善奔。不知惜命\footnote{此則不止於犵狫。天下多有之。}。布圍下體。謂之桶裙。善造毒箭。當之立死。受其氣者亦死。死則有棺而不葬。置之厓穴。或臨大河。剪頭犵狫者。男女剪髮。僅留寸許\footnote{梳箆二物置之無用矣。}。豬屎犵狫者。喜不潔。與犬豕同食。豎眼犵狫。蠻人之尤怪者。兩目直生。惡人衣靑。\endnotemark[9]云。遇之有禍。去麻陽百餘里。亦不常見。播州。古夜郞地。其苗信鬼好詛。射獵爲業。衣用虎皮。以虎尾揷首爲飾。黎州蠻。白馬氐\endnotemark[10]之遺種。其類凡十一。曰。西箐蠻。三王蠻。邛部蠻。風琶蠻。保塞蠻。淨浪蠻。阿宗蠻。烏蠻。白蠻。兩林蠻。山後蠻。交易不用銀錢。漢以絹帛茶布。蠻以鹽馬紅椒。其俗尚鬼。稱其長曰都鬼主。\endnotemark[11] 建昌。俗陋性剛。與黎州相似。松潘。古冉䮾地。積雪凝寒。盛夏不解。人居累石爲室。高者至十餘丈\footnote{危矣哉。較立危牆之下者何如。}。名曰碉房\footnote{名甚新雅。}。親死。斬衰布衣\footnote{強於遼俗遠矣。}。五年不浴\footnote{這却是關東強。有終身不浴者。}。奸淫事。輸金請和而棄其妻\footnote{金多者樂甚。}。惟處女嫠婦勿禁。有罪者。樹一長\endnotemark[12]木。擊鼓聚衆而殺之\footnote{較依律問斷者。甚覺爽快。}。富者貰死\footnote{有錢人到處得便宜。}。燒其室。奪其田畜。部落甚衆。無總屬。各推一人以爲長。麥坌住白沙。牝牛聘婦。吹笙飮酒。刻木人祀祖。負薪荷蕢。治生辛苦。玀玀本名盧鹿。有黑白二種。黑爲大族。深目長身。面黑齒白。故名玀鬼。其人佩刀挾弩。左肩背脊拖羊皮一方。兵爲諸苗之冠。諺云。水西〖犭罷〗(玀)鬼。斷頭掉尾。男女貴者。寢不同席。至夜半密通之\footnote{俗謂。婢不如妓。妓不如偷。此玀鬼大約於此數語中悟出者。}。男去鬚\footnote{一老童應試。將鬚剃盡。其友駭問之。答曰。時人不識愁苦。將謂偷閒學少年。玀鬼豈亦學少年耶。}。女\endnotemark[13]辮髮\footnote{省了許多妝飾。}。氈衫爲禮。雞骨占年。死不用棺\footnote{同回回敎。}。招魂以葬。女以善淫名者。人爭取之以爲美\footnote{這眞是尋忘八當當。}。白猓\endnotemark[14]玀住麻地。迎春邛州府。吹笙跌足而賀。〖犭栗〗蘇居茂連山。種菽而食。玀〖犭面〗在鋪西硝井等處。採薪拾菌。攜柴棍乞醴酒。醉臥中途。可供一笑\footnote{中國此等可笑者亦不少。}。金齒。古哀牢國。其苗人皆九隆之後也。其裔蕃衍。散處荒域。\endnotemark[15]其人有數種。有以金裹兩齒者。曰金齒。有漆其兩齒者。曰漆齒。有刺面者。曰繡面蠻。有刺足者。曰花脚蠻。以綵繩撮髻者。曰花角蠻。惟居諸葛營者。衣冠禮儀。悉如中土。八百媳婦。其人性緩。刺花鳥於眉目之間以爲飾。俗同緬甸。相見把手以爲禮。木邦亦名孟邦。其人多幻\endnotemark[16]術。能以木換人手足。又能置汚穢於途。人觸之者。變爲羊豕。以錢贖之。復變爲人。有知之者。易置汚穢於他方。則其人反變爲異類。其俗男衣白。文身髠髪截髭。女飾金圈象鐲。居皆竹樓。男貴女賤。民皆奴視其妻\footnote{此風較中國大佳。定無怕婆者矣。}。役之耕織。老撾。其民性悍。徧體花繡\footnote{水滸之燕靑不能獨擅於前矣。}。居高樓。其上寬廣。猺一名軬客。其種有八。曰。天竺。咳首。僬僥。跛踵。穿胸。儋耳。狗軹。旁脊。又有飛頭蠻。鑿齒。鼻飮。花面。白衫。赤褌之類。猺俗童時燒鐵烙足心。沁以蠟油。重趼如鞹。易登險峻。婦人黥面成花。嫁則荷傘懸草履。歸於夫家\footnote{此一禮。在六禮之外加設。}。好劫掠。然信鬼畏誓。可以要結\footnote{較中土強多矣。當面設誓。轉背即忘。}。外有打寮山猺。柁親山猺。獞人。居五嶺之南。冬綴鵝毛木葉爲衣。能用毒矢。中之者。肌骨立盡\footnote{劍仙以少藥化人頭如水。大約亦是此藥。}。雖猺人亦畏之。苗人欲舉兵攻殺。先期集衆。揷牌於山\footnote{兵不厭詐。並詭道襲人。全用不着。不意此猺竟是堂堂正正之師。}。偵知得以預備。峒苗仇殺之後。漢官爲之講歹。兩造各積草爲籌。每講一事舉一籌。理詘者棄其籌。籌多者勝。負者以牛馬歸勝者。即彼此殺人。亦較其人數多寡而以牛馬賠償之。紛乃解。講歹之時。兩造苗民各踞兩山之上。而立牛於其中。講旣明矣。一苗持刃從牛頸下。於是兩山之苗呼噪而集。各割牛肉一塊。歸而祭祖。若相誓。曰。有負諭者\endnotemark[17]如此牛。蠻獠\endnotemark[18]有事爭辯不明。則對神祠爇油鼎。謂理直者探滾油手無恙。愚人憤激。信以爲然。往往焦潰其膚。莫能白其意者。各峒歃血誓約。緩急相救。名曰門款。戰鬪進止。以發喊助威。曰鸕鷀號。朱漆牛皮以護頭頸。名曰固項\footnote{若遇利刄。恐項難固。}。六月二十四日名火把節。苗相聚。生啖牛豕。苗人把忌。以元日爲始。二七而解。〖犭羊〗獷以三月爲忌。二十五日而解。俱不容人犯忌。午日。苗盡閉門把忌。先二日鎖鈕所擄之人。善逸者於是日走。苗不敢追。追懼不吉。鬼方之民。信鬼。椎牛而祭。謂之做鬼\footnote{眞是活見鬼。}。初夏徙居數日。讓鬼居之\footnote{奇想。豈春秋冬三季無鬼耶。}。謂之走鬼。平居寢不解裙。亦恐犯鬼故也\footnote{豈彼地皆淫鬼。專犯人裙內耶。眞可笑。}。犵狫謂席地而居則近鬼矣。爲屋宇。必去地數尺。架以巨木。上覆杉葉。有如羊栅。故名羊棲。獞人之室。緝茅衡板。下畜牛羊。謂之麻闌。苗童之未娶者曰羅漢。苗女之未嫁者曰觀音。皆髻揷雞翎。於二月羣聚歌舞。自相擇配。心許目成。即諧好合\footnote{視六禮爲多事。}。苗人之婚禮曰跳月。跳月者。及春月而跳舞求偶也。其父母各率子女擇佳地而相爲跳月之會。父母羣處於平原之上。子與子左。女與女右。分別於原隰之下。原之上。相讌樂。燒生肉而啖焉。操匕不以箸也。漉咂酒而飮焉。吸管不以杯也。原之下。男女皆艷粧盛飾。男反褌不裙。女反裙不褌\footnote{睽其意。負去時以便相合也。}。男執蘆笙。笙六管。長有二尺。女執繡籠。繡籠者。編竹爲之。飾以繒。即綵毬也。原上語女歌則皆歌。語男吹則皆吹。其歌哀艷。每〖聿皿〗一韻三疊。曼音以繚繞之。而笙節參差。與爲縹緲。吹歌之時。手翔足揚。睞轉肢廻。首旋神蕩。是時有男近女而女去者。有女近男而男去者。又數女爭近一男而男不知所擇。有數男競近一女而女不知所避者。有相近復相捨。相捨仍相盼者。心許目成。籠來笙往。忽焉挽結。於是妍者負妍者。媸者負媸者。媸與媸不爲人負。不得已而後相負者。有終無所負。羞愧浹涕以歸者。彼負而去者。渡溪越澗。選幽而合\footnote{反裙不褌者便於此。}。解錦帶互繫。相攜還於跳月之所。各隨父母以返。而後議聘。聘以牛必雙。以羊必偶。先野合而後儷。苗之俗如此\footnote{此俗或不止於苗。}。獠人夫妻異宿。晴晝牽臂入山爲樂。於路口揷松枝以斷往來。謂之揷靑。見者即避。如或誤入。刀斧相加。谿峒\endnotemark[19]男女相歌於正月朔。三月三。八月十五。而三月謂之浪花。歌尤無禁忌。龍家苗立木於野。謂之鬼竿。春時男女旋躍其\endnotemark[20]下。以擇配偶。玀鬼之俗。新婦見舅姑不拜。裸而進盥\footnote{進盥則古禮。裸則甚不雅觀。}。謂之曰奉堂。苗人取雞卵畫墨。祝而煮之。剖視吉凶。又有將葬其親。以雞卵擲地。\endnotemark[21]視卵不碎之處。即以爲吉。於焉卜兆者。苗人臘祭曰報草。祭用巫。設女媧伏羲位。苗祀神多書孔明天子之位。苗人親死則聚親族笑呼歌舞。謂之鬧屍\footnote{中國更有唱戲者。大約是染苗之俗。}。又曰唱齋。至明年春月。聞杜鵑聲。比戶而號。曰。鳥有一歲一來。吾親不復至矣\footnote{孝哉此苗。近日詩禮之家。親柩有在室而不悲號者多矣。}。苗人每遇令節。男子吹笙撞鼓。婦隨男後。婆娑進退。舉手頓足。疾徐可觀。名曰踹堂之舞\footnote{禮失而求諸野。夫妻唱隨之樂。不意反出於苗。}。八番之蠻臨炊始舂稻。不宿舂。宿舂則頭痛。臼深數尺。相杵而下。其聲叮咚。抑揚可聽。名曰椎堂。苗人醉後以長柄木杴躍舞。名曰舞杴。獞人遠歸。止三十里外。家遣巫師提竹籃貯其裡衣前導而還。謂之收魂。獞人親死。慟哭水濱。投錢於河。汲水而返。用之浴屍。謂之買水。否則爲不孝\footnote{余聞此不禁傷心。苗獞醜類。猶知慟哭其親。尚懼爲不孝。而近日簪纓世族。詩禮名家。親死不慟者甚多。聖經云。喪。與其易也。寧戚。有幾人哉。}。播州苗所歌。十數輩連重袂而舞。以足頓地節歌。名曰水曲。葫蘆笙大如盂。止六管。韻頗悠揚。猺人之樂狀如簫。縱八管。橫一管以貫之。即古鳳簫之製。銅鼓多馬伏波及武侯所製。故稱曰諸葛鼓。大苗峒方能有之。琵琶只二絃。彈之應律。苗人合樂。衆音競發。擊竹筒以爲節。峒民爲筆用雞毛。彼蟲鳥之文。非此不可。苗錦大似苧布。巾帨尤佳。藻彩雲霞。悉非近致。謂之花綀。土俗珍之。南蠻豪家以鵝毳毛爲被。溫麗勝於純錦。犵狫諸種則以茅花爲被。苗人年十六無不帶刀。其鐵自始生時煉至成童。故最銛利。以黑漆雜皮爲鞘。\endnotemark[22]能者擲刀空中。接之以手。曰跳雞模。苗人之弩名曰偏架。以毒塗矢鏃。中者必死。掉鎗長餘二丈。用以護弩。戰則一弩一鎗。相依成對。苗人火器有過山鳥者。能打越重山。絕無障礙。蠻地多楠木。刳以爲舟。有絕大者。猺人截大竹筒煮食物。而竹不燃。亦異製也\footnote{鍋到彼處亦無用矣。}。猺獠睡無床褥。以三木支板。燃火炙背。板焦則易\footnote{較睡炕者尤暖。}。名曰骨\endnotemark[23]浪。處以瓦屋。居之溫室。則病而不安。溪洞收蟻卵。淘汰爲醬。非尊客不以供饌。粥雜魚肉蛆蟲。叢嘬\endnotemark[24]以爲珍美。謂之曰䤃。苗之矜富者。則曰。其家蓄\endnotemark[25]䤃幾世。咂酒一名釣藤酒。或有以鼻飮者。謂由鼻入喉。更有異趣。富峒以九月一飮羣苗。謂之大設\footnote{中國却無此等慷慨富翁。}。牛羊腸臟略一擺洗\footnote{近日人吃羊腸。尚有不擺洗者。}。煮以饗客。臭不可近。必欲客盡之乃喜。曰不乃羹。凡殺牛。以骨浸於淵泉之中。歷久乃酥。取出食之。以爲至美。殺牛多者。將牛角掛之屋上。以矜豪富。苗人請客。先到者上坐。子先赴席。子居父上。到遲者立飮\footnote{有一種大老富翁故做身分。遲延後到者。皆當以此禮待之。}。苗人渠帥謂之精夫。其相呼謂之姎徒。獠人尊有力者謂之火郞。獠人之百姓謂之提陀。洞酋妻皆稱媚娘。苗人同類稱曰同年。苗人幼稚謂之馬郞。能通苗漢語者謂之客語。爲苗人判論是非者謂之鄕公。漢人潛入苗洞者謂之漢奸。\endnotemark[26] 熟洞溪文移者謂之專事。犵狫之隨行者謂之隊小。狑(犵)狫之爲傭者謂之奴狗。苗人買\endnotemark[27]人。量人以拳。一拳價一金\footnote{防風氏身長三丈。若在彼處。値許多銀子。}。諸苗負物不以肩。用木爲半枷之狀。箝其項。繫帶於額。背籠以行。猾苗坐茂草中。見孤客過。暗鈎曳入。綁之貨販。苗人得漢人。恐其逃逸。以木靴着之而墩鎖。終身莫能出。有逃走拿回者。用板一片。以釘釘於足上。墩鎖之外。六月曝日中。曰曬日。冬月去衣使露處。曰曬霜。

\end{quotation}

其事甚多。不能盡錄。擇其異者載之。其全部則書房中有之。鍾生細閱了一遍。倒也胸目爲之一新。按過一邊。且說竹思寬那日別了鐵化。攜着他所贈的那一封銀子到錢家來。恰好大門開着。走進內中。悄悄躡足走到錢貴房門口。伸頭一張。見鍾生已去。錢貴靠着桌子。手托着香腮。一隻手做着手勢。虛空模擬。面孔上笑吟吟。不知心內想些甚麼\footnote{活畫出一個瞽女來。}。竹思寬見了這個樣子。不由得骨軟筋酥。忙到郝氏房中。郝氏正在床上睡着。上前抱着親了個嘴。就伸手到他褲襠內。摸那大而且癟的朽牝\footnote{牝字之上加此數字。難乎其爲牝矣。一笑。}。笑着道。你這件寶貝東西。比當日更肥範有趣了\footnote{欲說違心之言。故未語先笑。善奉承者連此物也奉承到。}。郝氏笑道。知道不堪。不勞你假奉承\footnote{郝氏竟有自知之明。}。你昨夜爲甚麼不來。想是那裡又敍上新人了。你此時有這些假親熱。竹思寬道。也沒甚麼新人。一來我前晚在你這裡弄了一夜。不曾合眼\footnote{接前不漏。}。昨日乏了。去歇息歇息。二來我如今不敢常常到你家來。心裡有些過不得。郝氏道。我同你相與了這幾年。今日重新講這句鬼話。有甚麼過不得。是甚麼緣故。竹思寬親了他個嘴。道。不瞞你說。你的那個女兒是個狐狸變的。會懾人的魂魄。我一瞥見了他。就掉了魂。你要叫我同他沾一沾身。我晴(情)願死在你肚子上。在你家替你當個老烏龜。你就拿棍也攆不出我去。郝氏含笑把他打了一個嘴巴。道。我同你相厚了這些年。我一心還想要嫁你。他也算你的一半女兒了。你還想做這樣的事。況且你想想你這東西。可是輕易近得人的。我那嬌滴滴的女兒。不要說弄。他要摸着\footnote{目不能見也。摸字妙。雖泛常語。亦不錯誤。}。管就嚇死了。竹思寬道。你這些話說的一點也不相干。難道雞巴硬了不認親。況外國的風俗說。生我者不淫我。生者不淫。除了自己的親娘同親生的女兒。別的一槪混弄。像這樣的女兒。十個指頭扯扯。關着那一條筋。你若肯容情。我把你娘兒兩個當做素珠。一串兒穿起來。你說我的東西怕他禁不得。我想有其母必有其女。你的這件寶貨難道生成的這樣大。也不過是我揎開了的。你恐我吃白食。故有這些推托。遂在腰間掏出那封銀子。打開道。五十兩細絲相送。你總成我一總成。我後來還重重的謝你。豈不強似他前日接那窮鬼。郝氏道。還提他呢。我只接了梅相公的一兩東道銀子。被他吃了兩日去還不打緊。女兒白白的陪他睡了兩三夜。一個錢也不見。竹思寬道。可又來。只許他白接人。難道你就叫他留不得我。郝氏道。這丫頭情性古怪。只好等他那一日歡喜的時候。我慢慢的對他說。他若肯依。就是你的造化\footnote{極寫老鴇之醜惡。見了銀子。連親生女兒都不惜了。}。有一句先要斷過。這不過只許你嘗嘗滋味。不要說得了甜頭。戀着他。撇了老娘。我把你的肉零碎咬了下來\footnote{身上的肉零碎啐(咬)下來還罷了。若將陽物也零碎咬下。何處再覓此如驢之具。}。竹思寬道。我原不過想嘗嘗。怎敢得新忘故。你但請放心。竹思寬昨夜同火氏未曾盡興。方纔又張見錢貴那番舉動。此時手摸着郝氏的老陰。說了這一會話。總未離手。摳摳挖挖。滿手淋淋漓漓。動火之甚。抱住了郝氏。道。承你慨諾。我且先謝謝媒著。郝氏被他挖得難過。也正想他這種謝儀。同脫光了。架起兩足。弄將起來。他二人一個是驢腎般的陽物。一個是皮袋樣的陰門。這一場非同小可。那樣結實的金漆楡木床。還搖得格支支亂響。兩個帳勾叮叮咚咚。一個陰戶搗得瓜瓜答答。財香在隔壁房中聽得好生難過。走到窗下。張見他床上枕頭推在半邊。郝氏平平仰臥。像是渾身被他搗酥了。四肢張開。宛然是一個大字\footnote{奇想。像形。}。竹思寬還橫舂豎搗。財香見他兩個的那樣子。笑得肚疼。他二人耍夠兩個時辰。方纔歇手。竹思寬要求他做媒。把吃奶的力氣都拿出來奉承了。他這一下。叫他把銀子收了。又懇求他去看看緣法。郝氏得了他的銀子。又被他弄得渾身痛快。推辭不得。叫他坐聽佳音。遂走到錢貴房中。那錢貴因與鍾生訂了終身之約。心中歡喜。誠於中。形於外。未免那喜色就露於面上。郝氏見他喜氣洋洋。心中也暗喜。便道。兒呀。我看你一臉的喜色。大約是有喜事臨門了。錢貴道。兒處在這活地獄中。有何喜事。郝氏道。事倒有一件。你若肯依從了。也是件小喜。遂將竹思寬送了五十兩銀子。要請他歇一夜的話說出。錢貴不等他說完。大怒道。這奴才。連畜生都不如了。他與母親相處了多年。怎麼又想起我來。這豬狗不如的下流。該拿驢糞塞他的嘴。我自幼見他是個舔癰䑛痔。不端的小人\footnote{此一句是暗含着總成鐵化來時。}。屢屢要辱罵他。因他係母親相知。我看母親面上。容忍多次。他今日反這等無知妄想。放這屁起來。我當與他性命相搏。我雖眼睛看不見。我若聽得他聲音。遇着這大膽的豬狗。與他誓不俱生。千小人。萬匪類。罵不絕口。那郝氏恐竹思寬聽得。惱了不來怎處。便道。你不肯便罷了。何必這等破言。忙抽身出來。原來竹思寬正在房門外。一團高興來聽好消息。誰知被他罵得狗血噴頭。郝氏怕他羞怒。忙拉他到房中陪話。道。那丫頭嬌養壞了。嘴不値錢。你宰相肚裡好撐船\footnote{他肚裡未必能撐船。胯下倒有一個大篙攢。}。看我薄面。不要記懷。我替你陪禮。叫財香收拾酒肴來與他消氣。又將銀子還他。道。你請收回罷。我沒福要你的。那竹思寬如何捨得撇了郝氏這個對子。便道。你女兒不肯。你是肯的。銀子就送了你罷。叫我拿了那裡去。郝氏也就笑納。二人吃到天晚。上床。竹思寬道。你女兒的惡口罵我。我且拿你的屄出出氣着。使出蠻力。足足拿郝氏出了半夜的氣。搗了個無數。郝氏心中暗暗感激女兒了不得。竹思寬把力氣也費盡了。睡下想道。婦人中賢慧的太賢慧。潑賴的太潑賴。鐵家娘子那樣溫柔嬌媚\footnote{以偷漢婦人爲賢慧。爲溫柔。非此等下流人無此異想。}。這妮子看他也還好。誰知這樣可惡。眞是。

\begin{quotation}

鼠狼未獲得。空惹一身騷。

\end{quotation}

我還是串通了老屠。把小鐵卦(引)了出來。同他娘子去親熱是正經。想了一會。一覺睡到日出起來。別了郝氏。往屠家去了。此後錢貴但是聽得竹思寬來。便在房中大罵。你道錢貴果是爲要來嫖他的仇恨麼。自從竹思寬打合了鐵化來梳籠了他。直恨至今。礙着母親發洩不出。恰遇有這個因頭。把這數年的鬱氣都發了出來。且他要杜門守貞。先撒個潑樣與郝氏看看。後來竹思寬要來看郝氏。悄悄的瞞着他。郝氏又囑代目。但是竹思寬來。不要吿訴他。錢貴見他許久不至。纔氣攤了。所以後來錢貴嫁了鍾生。郝氏招了竹思寬。竹思寬再不敢上他家的門。就是此時結下的仇恨。這是後話。再說那火氏自經了賽敖曹之後。雖弄得陰門腫裂。他不以爲苦。反心中私喜道\footnote{因今日不爲苦而反喜。所以後來方死於此也。}。不意天地間生此異物。若陰門不痛。內中之樂自然不可言盡。過了數日。腫消痂退。依然好好的一個妙牝\footnote{恐未必似當日之妙矣。}。心中想道。雖然不腫痛了。若仍然還弄不得。豈不枉受了這番苦楚。我何不去試他一試。纔可放心。遂走上樓去。將褲子脫下。睡在床上。用手指摳挖。竟是一個大窟窿。與當日那一條細縫大不相同。甚是得意\footnote{火氏雖得意。鐵化若試着。甚不得意。}。想道。局面有些好了。但得個甚麼試驗試驗纔妙。滿屋顧盼。忽見壁上掛着兩個搥癢的花梨棒槌\footnote{第二回內敍樓上擺設之癢槌。此時纔用着。}。有鵝蛋大小。比蛋略長些。一個大指粗的把兒。忙起身取下一個來。用手箍了箍。道。這個與他的差不多粗細。若這個弄得進去。他的也就弄得進去了。遂用許多的津唾。將棒槌潤濕自己的陰門。內外也用上許多。仰臥着。蹺着腿。揸得開開的。拿着往裡面塞。雖覺有些難入。却不甚痛。想道。料不妨事。手婉(腕)用力往內一送。一下攮了進去。似乎微有疼意。摸時已全然入內。只剩個把兒在外。大喜道。好了。這次却弄得了。復沈思道。寬處容下了。但他那長得利害。內中容不得怎處。又想了想道。有了。到臨弄時叫他放入。只儘我裡邊。到了底。剩在外的。拿汗巾裹住。便無礙於事\footnote{悟性頗通。}。笑道。我的道場雖排下。不知幾時纔遇得這着和尚\footnote{窮道場。只用一個和尚。}。他擺弄了一會。有些火動。就拿那槌兒一出一進的抽。正弄得有些趣味。那狗在胯下搖着尾。將鼻子混拱。因棒槌塞在戶中。他尋不着門。在腿縫中舔幾下。又在糞門上舔幾下。或在手上也混舔舔\footnote{狗知之乎。汝之情人將棄汝而取竹思寬也。}。礙着手。抽得不爽利。倒把棒槌拔出來。用兩手扳住腿彎。屁股疊起。牝戶大張。叫那狗舔。舔了幾下。內中覺得比每常分外有趣。用手摸時。原來當初只一個小圓眼。狗但伸得舌頭進去。如今被大物楦開。此時又被棒槌撐得像鍾子口似的一個大洞。狗小嘴尖。聞見裡面腥氣。嘴拱進去有二寸許。舌頭入內深處。所以較常愈樂。舔夠多時。淫興已足。穿褲下樓歸房。他先那幾日因牝戶裂疼。知道行事不得。倒也不想去弄。此時好了。又試過無妨。可以大舉了。把那個粗大東西時刻在念。吃着飯拿着箸子。就想起他的長來\footnote{一想。}。吃茶掇着鍾子。就想起他的粗來。看見燈盞。就想起那夜用油\footnote{三想。〈想〉此想令人絕倒。}。又把那大而且粗的放在心上。連睡都睡不着了。每日叫巧兒來回在外打聽。使得他如走馬燈兒一般。來來往往個不住。心裡一動就叫他去。一日何止百十次。到晚睡下。那丫頭出不去了。纔得少歇。把他的腿也走腫了。脚底心上泡都磨出。他要圖主母歡心。也顧不得勞頓。一日。忽見巧兒來說道。大爺今日又去賭錢。吩咐家人說今夜不回來了。火氏雖然歡喜。又愁着竹思寬不知可知道。如何望得他來。凝眸盼望。一刻三秋。比那秀才望報錄。與那農夫望歲。還着急幾分。正合了曲子上的兩句道。

\begin{quotation}

望將穿。不見情人到。

\end{quotation}

將晚時。望得悶上心來。神思困倦。伏在桌上。不覺睡去。忽見竹思寬走進房中。慌忙爬起。笑逐顏開。上前一把拉着手。同在床沿上坐着。道。你來得好。我望得眼睛幾乎滴出血來。你剛纔進來沒人看見麼。竹思寬摟着他。道。我也幾乎想殺了。恐你懸望。纔在外邊。見沒人。所以走了進來。忙去把房門關了。兩人攜手上床。不暇脫衣。只褪了褲子。二物相接。方要送入。正纔高興。忽被一推。猛然驚醒\footnote{掃興。昔有一人睡覺。爲妻呼醒。其人大怒。痛撻之。妻問其故。恨曰。人請我吃戲酒。方才上席。被你叫醒。豈不可惱。火氏將到妙境。被巧兒喚醒不恨者。竹思寬來強如做夢故耳。}。原來是夢。睜眼看時。却是巧兒笑嘻嘻站在床前推他。火氏因叫巧兒不住來回打聽竹思寬的消息。走到角門口看看。見門罅着縫。疑內中有人。走進去到北窗下一張。只見竹思寬在內獨坐。他忙進去道。你多昝來的。爺今日不在家。奶奶望你連眼都望穿了。叫我出來看了十數次。竹思寬笑道。我來了好一會了。就摟他在懷中。親了個嘴。巧兒笑道。那一夜我睡着了。你同奶奶可弄得。竹思寬道。你奶奶的那東西緊小得很。弄了一會。他怕疼。只得罷手。把我幾乎急死了呢。巧兒道。我聽見他們說你的有多粗多大。我就疑惑弄不得。可應了我的話。旣然這樣。他還想你來做甚麼。竹思寬道。那是頭一次纔試新。第二回自然就不妨了。巧兒笑道。我就不知道這件事有甚麼趣。甚麼好吃的餹棗兒。何苦這樣忍疼捱痛的還戀着他\footnote{是個未曾嘗過滋味的小丫頭說話。}。竹思寬笑道。你不曾嘗着味兒呢。後來嘗着了還更愛。你的雖然弄囗囗。囗囗你在門口晃晃。你看可有趣。就掀開衣服。扯他褲子。巧兒故意不肯。竹思寬強替他脫褲。就將他仰臥在椅子上。看他的囫圇美物。只條細縫。巧兒比火氏的又自不同。十分可愛。眞是生平頭一次纔乍見也。唾上一朶津唾。用手攥着陽物。將龜頭在他那縫上擦晃。巧兒被他擦得癢酥酥的。不住嘻嘻的笑。晃了一會。也有些淸水流出。巧兒笑道。晃得不好過。你放我起來。我去對奶奶說。好出來同你做正經事。竹思寬放起他來。他穿了褲子。上來與火氏報信。見他醒了。附着耳道。原來竹相公來了。我方纔出去看看。前邊一個人也不見。書房院子門倒關着。我先疑是家人們在裡面賭錢。我走到後邊角門口聽聽。門是虛掩着的。我進去看。只見竹相公自己一個坐在裡面呢。他說昨日串了開賭場的屠家。今日請了爺去耍夜局。他知道不回來。故此傍晚來了。到了門上。不見一個人。想是知道爺不來家。都吃酒耍錢去了\footnote{可謂上有好者。下必有甚焉者矣。}。他悄悄走進書房。倒關着門。開着角門等我。可可湊巧遇了我去。他見了我。歡喜得了不得。叫我拜上奶奶。請奶奶早些出去。火氏聽了。笑容滿面。精神頓長。那個喜那裡還說得出來。連忙爬起\footnote{忙一。}。忙下床來\footnote{忙二。}。忙到鏡臺前\footnote{忙三。}。把頭髮挽了個結實。兩鬢刡光。忙忙的勻了勻臉\footnote{忙四。}。點了點唇。忙拿出一條大紬汗巾\footnote{忙五。}。塞在褲帶上。正收拾着。見捧了晚飯來。他心忙意亂。也無心去吃。吩咐道。我心裡不自在。要早些睡。不吃飯了。你們都去快快的吃。吃了都早早的睡了罷。丫頭們拿去了。受用一飽。伸開鋪。倒頭而睡。覺得他們比火氏還快樂幾分。巧兒問道。奶奶怎麼不吃飯。火氏笑着低聲道。他的那東西長得利害。吃飽了。怕頂斷了腸子。空着些肚子好\footnote{奇想。}。忙叫巧兒掇了一脚盆水來\footnote{忙六。}。熏水澡牝。忙拿了一雙大紅睡鞋\footnote{忙七。}。用塊絹帕包了。叫巧兒籠在袖中。外面有起更時分。丫頭們大約睡沈。恐書房中無燈。忙叫巧兒點了兩枝安息香\footnote{忙八。}。拿了兩枝燭並焠燈。然後忙忙出來\footnote{忙九。}。纔到角門口。那竹思寬正站在那裡潛潛等等。一見了面。也顧不得巧兒在傍。兩人忙摟抱着\footnote{忙十。先是火氏獨忙。此是兩人同忙。}。親嘴咂舌。親熱了一會。相攜進房。巧兒忙點上了燭\footnote{忙十一。這是巧兒幫忙。}。竹思寬見火氏比前夜愈加俏麗。等不得敍寒溫。情急如火。忙拉着火氏一同上床\footnote{忙十二。這是竹思寬獨忙。}。巧兒遞過那個包兒。火氏接過。放在枕傍。忙忙各自寬衣解帶\footnote{忙十三。此是兩人同忙。}。脫得精光。火氏忙把睡鞋換上\footnote{忙十四。此又是火氏獨忙。}。竹思寬見他一身雪白肌膚。燭下照耀。細膩如放光一般。兩隻小脚剛有三寸。穿着大紅平底睡鞋。神魂飄蕩。那陽具翹然直舉。忙叫火氏睡倒\footnote{忙十五。}。竹思寬兩手捏着他兩隻金蓮。分得開開的。看他的牝物時。比前大了許多。兩瓣大張。中間一朶花心。碎糟糟如一個楊梅一般。微微紅紫。心愛極了。忙縮下身去\footnote{忙十六。此又是竹思寬獨忙。}。親嘴也似的連親了幾親。把舌尖將那花心舔了幾下。忙上身要弄\footnote{忙十七。}。火氏前次與他初會。那個大物雖然看見。却不曾細細賞鑒。此時要仔細領略一番。便道。你且住着。待我起來看看。遂爬起來。那話正猙獰跳躍。他一把攥住。仔細端詳。果然好個異物。

\begin{quotation}

紫威威一個和尚光腦袋。鼓稜稜一枚頭陀大戒箍。粗將雙圍。長約一尺。靑筋蟠繞。如皮繩亂纏鐵棒。黑鬚倒豎。似毛纓上托鋼鎗。若非那騷淫寬大之陰。怎容這堅粗長大之腎。

\end{quotation}

那火氏見了。眼中都爆出火來。心愛極了。縮下身子。也將嘴來含住\footnote{先是竹思寬舔他的。此是他舔竹思寬的。可謂還禮。}。他那一張未及三寸的櫻桃小口。只含了一個頂兒。就撐得嘴了。有些疼。笑道。好大東西。連嘴都含不進去\footnote{喜殺。火氏之嘴反不如其陰矣。}。他用舌尖把那馬口挑弄了幾下。竹思寬筋骨皆酥。忙推他睡倒\footnote{忙十八。}。兩人都情急了。用上唾沫。一頂而入。毫無難苦。火氏心中之喜不消說的。把個竹思寬幾乎樂殺。問道。這次何如。火氏笑嘻嘻搖頭。道。不疼了。只有些脹脹的。竹思寬放了心。忙抽拽起來\footnote{忙十九。}。抽了有十數下。淫水滑溜。漸漸送入有多半截。還剩有三寸餘在外\footnote{伏此一句。爲後日死火氏之根。}。火氏覺得頂到底了。再入就受不得。忙伸手去攥住。道\footnote{忙二十。}。進不去了。就到這裡罷。竹思寬也覺龜頭頂在軟濃濃的肉上。甚是有趣。知道他的牝戶大而不深。也不敢再進。火氏用手捏住那則寸。叫他拔出來。起來拿過帶來那條汗巾。替他裹在根子上\footnote{先只疑是帶來做陳媽媽。不想是做如此用。}。忙將褶褲帶取過來\footnote{忙二十一。}。紮了個結實。然後臥下。忙蹺開兩足\footnote{忙二十二。}。竹思寬就勢扛在肩上。一挺而入。這回將弄起來。響聲震耳。好一番動作也。怎見得。

\begin{quotation}

那火氏牝中與口內齊鳴。竹思寬陽物共腎囊亂撞。男子婦人。上下並用。陰門厥物。兩件同忙。弄夠多時。抽扯半晌。火氏初經這番風雨。心窩內受用難當。竹絲(思)寬乍嘗這宗美物。遍身上酥麻樂極。有半個更次。將一個時辰。竹思寬情濃精洩。那火氏也興足火消。

\end{quotation}

兩人拭抹乾淨\footnote{這不知是用大汗巾。是用包鞋的小帕。}。並肩疊股〔臥〕下\footnote{以上共寫二十二個忙字。到此方忙畢。世間惟此一事。雖極罷緩之人。到此事無有不忙者。偶意(憶)一笑談。也(夜)遊神到一家訪察善惡。正値他夫婦行房。入問他家宅神云。他二人所作何事。答云。造人。神問一年造得多少。答云。一年只造一個人。笑道。一年造一個。何須如此忙。附此一道。}。竹思寬道。方纔若不戲(是)巧大姐出來。我幾乎空費了這場心。白等了這一夜。火氏道。這幾日我那一日不望你。時時刻刻叫巧兒出來打聽。那一日不走二三十次。今日也是他伶變。要不是到角門來看看。豈不悟(誤)了天大的事。竹思寬道。總有個緣法。應該我兩人姻緣湊合。所以他纔走了來。火氏道。你進便進來了。明日怎麼出去。竹思寬道。我想到了。明日約略有開大門的時候。我到廳上。只說來會鐵老爺的。說是不在家。我就出去了。要是遇見鐵大爺回來。他也只當是我纔來找他。那裡疑心我在此過夜。你道這想頭好麼。火氏歡喜得了不得。摟緊了他。親了個嘴。道。親親。你眞好想頭。竹思寬道。我承你這樣深情。這幾日我的心思也費盡了。串了老屠。尋了幾個賭友誆了鐵大爺出去。我纔得來親近你。火氏緊緊的摟着他的脖子。道。親親的哥。你要留心想出個妙法兒來。常常把他弄在外邊去\footnote{妙極。下句不曾說出。謂常常把他弄在外邊去。你的那個纔常常弄得我這裡邊來也。此一句。寫淫婦之淫。至於此極也。}。我同你終日相親纔好。竹思寬道。我自然留神。何用你說。你那條有血的汗巾我帶在身上。簪子關在頭上。一日摸着一百遍。就想你一百回。連夜裡睡覺都是魂夢顚倒的。火氏道。可不是呢。我比你還利害。你的那幾根毛。我剪了幾根頭髮包在一處\footnote{夫妻稱爲結髮恩愛。奸夫淫婦以毛髮相結。當作何稱。}。我拿了幾個珠子石寶。一塊金子。一個銀錁兒。寶貝似的裝在花包裡\footnote{自有毛以來。未有重之至此者。}。掛在褲帶上。來(走)着坐着。但把我的腿挨一下。就想起你來\footnote{若帶在褲內胯在(之)前。刻刻與陰戶相挨。豈不甚妙。}。剛纔望你不來。纔閉上眼。就夢見你來了。正講得親熱。被巧兒推醒。說是你來了。又摟着親了個嘴。道。親親。我看這個樣子。同你今生今世同生同死\footnote{將來之讖。}。再拆不開的了。說着話。竹思寬看那火氏兩隻眼已乜斜着。一點點個鮮紅嘴兒微綻。似笑非笑。兩個眼眶通紅。兩隻手不住的捏弄陽物。知他又有些情動。看了他這騷態。心愛不過。又堅(昻)然直豎。兩人這一場潑戰。非同小可。火氏竟自輕車熟路。越覺有味。交溝(媾)多時。竹思寬雖把筋力費完。那火氏也算飽其所慾。事畢之後。竹思寬伏在火氏肚子\endnotemark[28]上。咂口調笑說道。俗語說。婦人嘴小。陰戶也小。我看你這樣一張櫻桃小口。不意你下邊的。竟可容得一個大秋(約)半斤的桃子。好像開棺材鋪的招牌。外面放着小棺材做樣子。裡邊的却大得放樣。火氏笑着道。要不虧我這大棺材。你這東西裝在那裡。竹思寬笑道。可是人罵的。我竟是短棺材厥的了。火氏道。這是怎麼說。竹思寬抽了兩下。你不見只裝了多半截。還剩這些在外面麼。笑說了一會。又抽一陣。竹思寬將陽物拔出。縮下身子。再看火氏的陰戶時。有幾句比方道。

\begin{quotation}

牝戶大張。如喜極人裂開笑口。花心外吐。似饞勞鬼牙縫流涎。又如那善說人臨死一言難吐。惟張嘴而似嘆似語。又像那啞巴子欲說無聲。只吐舌而或閉或張。從前細細一紅溝。今此寬寬一黑洞。

\end{quotation}

二人又頑笑了一會。都乏困了。並枕而臥。只苦了巧兒。聽了半夜梆聲。那小牝中也點點滴滴流了好些淸水。有打油四句道他三人。

\begin{quotation}

覆雨翻雲錦被中。漏聲短促興匆匆。

獨憐識趣知情婢。聽得淫腔一夜風。

\end{quotation}

他熬困了。以椅代榻而睡。一覺醒來。出去溺尿。見天色將明。忙推醒了火氏。穿衣而別。古人有兩句道得好。

\begin{quotation}

最是五更留不住。喚人枕畔着衣裳。

\end{quotation}

正是這個光景。那竹思寬穿衣起來。也不敢復睡。見紅日將出。開了院子門出來。往外一看。大門已開。家人知主人不在家。尚都酣睡。管門的開了大門。大淸早料無客來。且回房中高坐。竹思寬滿心歡喜。忙忙趨步而去。看官且住。前說竹思寬的這根孽且(具)。只有一個郝氏是他的老對子。除他之外。老娼淫妓遇着他。肉綻皮開。今這火氏是良家少年嫩婦。且又是一個嬌怯怯的身軀。如何倒反弄得。要知事有不然。理無足異。竹思寬當日嫖妓時。有一個妓字在心中。以爲他老的少的。村的俏的。見過了千萬。此竅何所不容。況嫖妓可還有用唾沫的理。爬上身。猛然一下。自然弄得狼狽而走也。未必幾千百個妓女都受不得他的。也不過遇了幾人受了他的虧苦。互相傳說。人就不肯招惹他了。他後來遇了郝氏。正是棋逢敵手。心滿意足。已出望外。也就不想去再尋別人。今遇火氏這一番下愛。眞是夢想不到的美事。可還有推辭之理。見了他這樣個靑年嬌嫩的人兒。不敢像當日冒失。去下辣手。唾而油。油而破。兩次三番。用了多少水磨工夫。纔得漸入佳境。且男人的陽物旣有大小不同。婦人陰戶豈無濶窄之異。奇矮極小之男子有極大極粗子(之)壯陽。何見得嬌怯秀美之婦人而無深鬆濶大之牝物乎\footnote{俗謂觀婦人之面色。可以知陰好歹。黃鬆黑緊白邋遢。大約火氏之面皮是個黃白淨了。}。閒話不必多言。且看正傳。那〔錢〕貴自從與鍾生定盟之後。私心竊喜。以爲終身有托。遂吟一詩以誌意云。

\begin{quotation}

半生心願一朝酬。意蜜情殷不自由。

何日桂香來枕畔。夢魂先到曲江頭。

\end{quotation}

叫代目代他寫下收貯。錢貴因代目一見鍾生。便識他是個佳客。慫恿他相會。得遂了生平之願。越發待他親厚。暗對他道。\endnotemark[29]此事只你知我知。不可再傳六耳。異日我此身有歸。決不使你失所。代目感之不盡。暗暗也自歡喜。且說這代目之父姓戴名遷。戴遷之父親名叫戴善。他家祖上也還是書香一脈。到了戴善。讀書不成。因而學賈。他雖非絕頂的好人。還是個一邦之善士。四十無兒。他的妻房氏屢屢的勸他娶妾。戴善不肯。道。我若命中無。雖娶十妾奚益。應不絕嗣。焉知你就不生育。何必又多做這番事。誤了人家的兒女。房氏見丈夫執意如此。也無可奈何。光陰迅駛。歲月如流。不覺又是十載。他夫妻二人同到了五十歲上。房氏道。我年已五旬。是萬萬不能生育的了。你娶妾一事。似不可緩。戴善還不肯意。房氏道。不孝有三。無後爲大。凡事要盡人事以聽天命。你娶了妾。若再不能生子。這就是命了。況且你一生並無惡過。未必就到絕嗣的地位。前日二叔帶了信來。他尚無子。你再又無子。將來戴門宗祀豈不斬絕了。戴善見房氏說得大義凜然。便道。你這樣賢德的話。我安得不聽。但我今娶妾是爲生子。非圖慕色也。不必拘定要少年標致處子。就是中年略像樣的寡婦。可以生育的就罷了。房氏聽得這話也甚有理。托媒人去訪。不拘女孩寡婦。只要沒殘疾宿病。遇巧便成。過了幾日。媒人打聽着了一個小寡婦。來說道。這個寡婦二十歲了。先守着個小兒子。不幸死了。公婆憐他靑年。叫他改嫁。他娘家姓繆\footnote{他姓繆。生得乃郞雖不甚妙。孫女幸而還妙。}。人物生得也好。我們提起府上要尋二房。他素常知道府上是良善人家。也竟願意。房氏大喜。一應禮物俱全。擇日娶了進門。就在西屋內住。房氏見這繆氏生得端荘穩重。心中甚喜。如姊妹一般相待。過了一年多些。就生了一個兒子。這老兩口歡喜無限。只生過這一胎。以後雖也還常常下種。總不見收成\footnote{這纔妙。再生一個便不妙了。}。這核(孩)子他也無病無災。易長易大。到了八九歲。送入學堂。起名戴遷。他這讀書不過應卯而已。讀書幾年。虧他聰明。竟可上上賬目。寫寫包皮。到了十六七歲。老婦人望孫子心盛。就替他娶了一個那氏爲媳。頭胎生了一個女兒。就是代目了。後來又生了二個兒子。這戴遷到了二十多歲。他父母相繼吿終。都是七旬外的人了。五十無子。方纔娶妾。竟還得見孫子。這也就是天眷善人\footnote{看官當於此等冷處着眼。}。他老夫妻也自瞑目了。他生母繆氏也將五旬。這戴遷自幼因他嫡母房氏姑息太過。嬌縱得他無所不爲。他家與竹思寬昔年准與人的舊宅比鄰相接。竹思寬久已看上了他的家私。因他父母在堂。不敢動意。他父母死後。喪事完畢。被竹思寬輕輕一鈎。就鈎到賭場上去。下了場鍋。這個昏頭昏腦的少年。乍見了一個雪白碗中裝着紅紅黑黑。金晃晃的六塊骨頭。以爲是天地間第一種高貴上流有趣的美事。死命貪住。人先哄他上鈎。小小的輸兩場與他。他便欣欣得意。道。我的本事高強。纔初上場。就把多年耍錢的老把勢都被我贏了。若再頑熟些。我定是頭一把交椅無移(疑)。那裡知道是別人下的香餌。這一件事原來也有些邪處。初去學他。心中何嘗不懷着個我是初學。恐怕要輸。若果然一上手輸上幾場。也就興致索然了。惟獨這一毫不知的雛兒。不要講甚麼盆口。連叉快還認不淸。自己擲了兩個六。兩個三。兩個二的三三靠六六的快。不會贏人。反被人擲了四個六。兩個二的黑隔子眼。假說快。倒贏了錢去。豈不可笑。他這樣被人弄去哄了。手起就該背了。竟大不然。混擲瞎擲。滿手枒裡都是。明明五個骰子坐着是個臭了。那一個還滾出一個快來。譬如坐三個六。一個金么。一個白么。那一個看着是個二四的樣子。他一陣跳。不是么就是三。反贏五注。諸如此類。定要與他贏過幾場。夢魂中都想着這個甜頭。但是略知道了些。這就拾着倒運的票子了。便一日一日的輸將下去。因戀着先贏的那幾場。決乎不肯放手。到後來大輸過三場。他心中不服。道。我前幾次怎麼贏來。這輸不過是手氣不順。故此偶然失利也。並不知是入了人的圈套。再要想去翻本。越翻越輸。間或僥倖贏得一場。貪心不足。又想去贏第二場。不但不能贏來。反將前次贏的貼了利錢送去。這些孟浪不知死活的小夥計們賭錢。更有可笑之處。譬如那人來賭。只有十兩銀子。把他贏到了九兩九錢還不肯歇。定還要想贏他二十兩。就不知那人輸到十兩零一錢。連那一錢都沒有。設或那人色子順了些。翻回一二錢。越發不肯住。道。他十兩銀我先贏到差一錢。尚不肯饒他。何況此時反又少了一二錢。安肯心死。一時被那人手快起來。不但十兩翻回。倒反贏幾兩去。那人先已輸到將盡。此時翻本。而且又贏。焉有不歇之理。到了這個時候。睜着眼。張着嘴。纔嘆氣後悔。他心中何嘗不想剛纔休說贏了九兩九錢。就是贏四五兩也是個采興。就該歇了。萬不然被他翻了本去也就罷了。決不該反輸了自己的。及此時懊悔。那雪白的細絲錠兒已被他捲而懷之。倒不出來了。還有一種可笑的人。一上場去。色子又順。手氣又好。三文五文。一吊兩吊的贏了幾拾兩。心猶未足。竟像在這幾塊骨頭上要贏出個大財主來的樣子。拿在手中。總不肯歇。人擲這件東西。一日到晚。若手氣不改。俗語說得好。這叫做十回九不遇的事。那裡拿得穩。後來手氣一敗。被人幾擲翻了過去。只剩得不過些須。他倒反歇了。豈不可笑。而且可嘆。這是說那不肯歇的。還有一種不但可笑可嘆而又可憐者。這是什麼緣故。可憐他抱着個色盆不放。連死活都不知的人。還要貪着頑錢。他擲色的時候。別人擲擲是快。他像個悶昏雞也是(似)的。可是人說的歇後話。瞎老婆奶孩子。混乳。還趕着下注。自己擲擲是臭。一個快星兒也沒有。他還大着腦袋混擲混下。裡外盆被人贏得死死的。十擲中還強一(不)過一擲來。他還強着色色去下。並不知說。我今日的手氣不好。歇了罷。斷然不肯。只等贏家贏足了。見他輸得可憐。歇了。他倒還急怒道。爲何歇了。不容我翻本。就不知這件邪物順起來却也爽神。從心所欲。想快就快。叫叉就叉。至於要輸起來了。下了注。人的手快。單捏就擄。人的手略皮。自己就擲臭送去。任你甚麼能幹老手。不急不熱忍得\footnote{諺云。少輸更是贏。此六字眞是賭中妙訣。惜乎此輩不解耳。}。這上面占些應想。拗相是再拗不過來的。這些初出世耍錢的少年輸大了頭。那裡知道這些奧妙。這說的是那個不肯結的。所以賭局中有一句話。道。贏不歇。輸不結。眞是個呆賭。南京賭場中有個市語。送了這種人一個暗號。名之曰酒。雖不知他的深意。大約說一個人全成了酒。昏沈沈。連死活都不知的意思。更有一件。人在賭場中每一場輸贏都算十兩。若十場中贏得七場。就算極好的時運了。他自己也說。我贏的次數多。別人看見這人場場贏。拈飛的。打算的。不計其數。他以爲這何足惜。不過五個指頭動了動就贏了來。費了我甚麼力氣。及至輸了之時。並沒人幫出一文。少不得自家全全拿出。他就不曾細算。這贏了七次。名雖得了七十兩。是不心疼的錢。三文不値二文的花銷了。傍人拈飛。自己浪費。實在收入囊中之物。未必有二十多兩。到了輸上三次。這三十兩雪花銀却要自家拿出。究竟還要倒貼出己囊。贏的却在何處。有錢的人還罷了。沒有錢的有得當賣。還算體面。竟有偷人之物。騙人之物。以還賭賬。百醜俱備。這是何苦。惟有這些\endnotemark[30]無知少年。見了色盆。心都死透。再勸不醒\footnote{此一段不是霹空撰出。非久歷於此中者。不能知內中到(利)害若是之詳也。好賭之人。將此一段當細細讀之。}。戴遷是個乍出來賭錢的酒。全犯了這些病症。所以不幾年。把產業家私。被這六塊骨頭送去。他心還不死。猶想去翻本。一日。輸了鐵化的三十兩銀子。無可償還。被他辱罵打鬧了幾次。受氣不過。只得把女兒准了與他爲婢。這種好賭錢人的心腸。情(竟)有一件奇處。令人猜測不出。他雖該〈他雖該〉了私錢官債。被打被罵。情願領受。却捨不得還。到賭輸了。還得也沒有這樣爽利。還有家中無衣無食的人。寧可死捱。及到場上輸時。鑽頭覓縫。弄來塡還他。美其辭曰。這是好漢錢。要還人的。這種人眞不可解。更有異處。人有極剛拗的性氣。閒常他人或有無心一語之失。他便攘袂奮衿。怒目切齒。恨恨不休。到該了賭博賬。或被人辱罵。或以拳脚相加。不但一點氣星兒也沒有。還滿臉陪笑。直受之而不辭。這戴遷自從把女兒准了賭賬。他母親繆氏。妻子那氏。終日啼啼哭哭的咒罵。家中又穿吃俱無。方纔後悔。痛恨旣往之非。已是遲了\footnote{能知悔恨。還算良心未死盡者。予見今日至死無悔者多〈者〉矣。}。他祖父都是正經人家。自從把他女兒輸了與人。不知被親友談論笑罵了多少。人都不理他。下眼看成。他自知做得不是。也沒有顏面見人。躱了三幾年。全靠婆媳二人針指度日。月月還要出租房錢。戴遷一來躱着不是常法。二來家中供個日食還不敷。一寸布也添不上。一口氣瞞着母妻。雇與船上做縴夫。往北京投奔他叔叔戴良去了。他叔叔在北京張家灣住家。開了個雇船的埠頭老行。甚是興旺。也是六十多歲了。他先也無子。因戴善夫妻七十歲時。他把店托了夥計照看。他到南京來替哥哥拜壽。二則別久了。都有年紀。來會一會。見了戴遷。說道。\endnotemark[31]兄弟二人有人接讀(續)香煙的了。心中歡喜無限\footnote{這一段補得好。不然戴遷如何認得去投奔也。}。戴善又勸他娶小。道。你嫂子勸我多次。我先不肯。到五十歲。纔娶了繆氏。今年也就有了十九歲的兒子。且又有了孫女。你今也纔五十多歲。回去也趕着娶一個。焉知不生兒子。戴良見哥哥娶妾得子。他住了些時。辭了回去。也娶了個妾。也竟生了個兒子。方得七八歲。他恐自己年老了。草霜風燭。一時或有不虞。這幾千金家業。兒子幼小。如何承管。知哥嫂已歿。正要想帶信叫姪兒來同居料理。今見他到了。心中甚喜。見他襤褸不堪。問其所以。他哭訴自己不知事。爲人所誘。花費了家私。把女兒都准了與人家。直言無隱。全全說出。並說如今雖悔心改過。已是無及。無顏見家中親友。故遠來投奔叔父。〈見〉戴良見姪兒這個樣子。心甚不忍。說道。書上說。過則勿憚改。你若能改過。我叔叔家產也還夠你們穿吃。再要賭錢。這就不可定了。戴遷道。姪兒此後若不改過學好。再要做這一件下流的事。不要說將來死後不能見祖宗父母於地下。今日就狗彘不食其餘了。戴良連連點頭道。好好。你若能改悔自新。就是我戴門之幸了。戴良的原配顧氏已故五載有餘。現今就是生子的這個妾蕭氏當家。戴良遂領着戴遷進去相見了。他的兒子也來拜了哥哥。隨叫他換了衣服。留住了十數日。戴良對他道。你只顧你來了。家中母親妻子靠誰養活照看。你可去接了他們搭船到這裡來同住。你那裡旣無家業。我又年老。你兄弟幼小。你可來幫着照料家務。再者我們雖不是甚麼仕宦之家。也還是有些臉面的。怎麼把女兒與人爲婢。你可贖了他來。就加些利錢也說不得。但速去速來。免我老人家懸望。他家現當埠頭。搭船是極易的事。恰有一個苑寺少卿。姓侯。在他行裡。寫了兩隻官座往雲南去。戴良就叫戴遷跟着船同往。預先擇着個出行的黃道日子。打點了行囊。取出一百銀子交與他。道。這個做來的盤纏。並替他們做兩件衣服好上路。又付五十兩道。這個千萬贖了孫女兒來。敎他都打在腰中。叮囑再三。然後分手。上船等候着侯少卿一同起身。他這一番氣象。與前番來時那個光景大不相同。一日。到了家。見了母妻。他母親見了兒子衣服光鮮。心中甚喜\footnote{看至此一句。不禁淚落如豆。人家母親未有不望兒子光鮮者。奈兒子不能光鮮以副母父之望何。}。復又悲道\footnote{寫盡慈母。}。你去了數月。我倒當你流落到那裡去了。同媳婦眼淚不知流了多少。你在那裡來。怎得這樣光鮮回家。戴遷詳細把叔父的話說了。一家大喜。他把銀子取出交與母親。次日拿了五十兩銀到鐵家去贖女兒。鐵花(化)道。幾年不見你來贖。陪了舍妹到童百萬家去了。戴遷疑他說謊。又到童家門口來探問眞假。却剛剛問着了賣仙桃的那個家人童佐弼。他聽說是仙桃的父親來贖女兒。暗吃了一驚。答道。你這個女兒。我們奶奶疼他得很。不見你來贖。恐誤了他的靑春。打去年已嫁了人家去了。戴遷見他說嫁了人。知不可贖。便問。嫁了甚麼人。家在那裡住。我好去看看。他怎肯說是現在錢貴家。答道。這就不知道。聽得說是個外路人。不在本地的。戴遷不放心。又面見了童自大根問詳細。童自大當日聽得家人說是嫁往外路。也就是這話答他。戴遷無可奈何了。只得回家復了母親妻子。那婆媳二人又哭了二三日。他家收拾了衣服行李停當。上了墳。就一家搭船上北京去了。他父女祖孫可還有相會之期否。後來便見端的。你道戴遷搭他船來南京的侯少卿是何出處。且聽下回分解。正是。

\begin{quotation}

欲知侯姓人詳細。再接來文仔細看。

\end{quotation}

姑妄言四卷終



\endnotetext[1]{「財香」原作「香財」,據第三回及下文改。}

\endnotetext[2]{「自宏」原作「士宏」,下文多作「自宏」,今予統一。}

\endnotetext[3]{「眦」原作「皆」,據陸次雲《峒谿纖志》上卷改。}

\endnotetext[4]{「鈴」原作「鈐」,據陸次雲《峒谿纖志》上卷改。}

\endnotetext[5]{「仞」原作「忍」,據陸次雲《峒谿纖志》上卷改。}

\endnotetext[6]{「勤」原作「勸」,據陸次雲《峒谿纖志》上卷改。}

\endnotetext[7]{「叛」原作「判」,據陸次雲《峒谿纖志》上卷改。}

\endnotetext[8]{「抱兒」原作「胞兄」,據陸次雲《峒谿纖志》上卷改。}

\endnotetext[9]{「衣靑」原作「靑衣」,據陸次雲《峒谿纖志》上卷改。}

\endnotetext[10]{「氐」原作「氏」,據陸次雲《峒谿纖志》上卷改。}

\endnotetext[11]{「主」原作「王」,據陸次雲《峒谿纖志》上卷改。}

\endnotetext[12]{「長」原作「丈」,據陸次雲《峒谿纖志》上卷改。}

\endnotetext[13]{「女」原作「友」,據陸次雲《峒谿纖志》上卷改。}

\endnotetext[14]{「猓」原作「玀」,據陸次雲《峒谿纖志》上卷改。}

\endnotetext[15]{「域」原作「或」,據陸次雲《峒谿纖志》上卷改。}

\endnotetext[16]{「幻」原作「幼」,據陸次雲《峒谿纖志》上卷改。}

\endnotetext[17]{「者」原作「有」,據陸次雲《峒谿纖志》中卷改。}

\endnotetext[18]{「獠」原作「撩」,據陸次雲《峒谿纖志》中卷改。}

\endnotetext[19]{「峒」原作「同」,據陸次雲《峒谿纖志》中卷改。}

\endnotetext[20]{「其」原作「以」,據陸次雲《峒谿纖志》中卷改。}

\endnotetext[21]{「地」原作「之」,據陸次雲《峒谿纖志》中卷改。}

\endnotetext[22]{「鞘」原作「銷」,據陸次雲《峒谿纖志》中卷改。}

\endnotetext[23]{「骨」原作「滑」,據陸次雲《峒谿纖志》中卷改。}

\endnotetext[24]{「嘬」原作「撮」,據陸次雲《峒谿纖志》中卷改。}

\endnotetext[25]{「蓄」原作「畜」,據陸次雲《峒谿纖志》中卷改。}

\endnotetext[26]{「奸」原作「汗」,據陸次雲《峒谿纖志》中卷改。}

\endnotetext[27]{「買」原作「賣」,據陸次雲《峒谿纖志》中卷改。}

\endnotetext[28]{「火氏肚子」原作「肚子火氏」,據文義改。}

\endnotetext[29]{「他道」原作「道他」,據文義改。}

\endnotetext[30]{「這些」原作「些這」,據文義改。}

\endnotetext[31]{「說道」原作「道說」,據文義改。}

\setcounter{footnote}{0}

\theendnotes

\part*{姑妄言第五卷}
\addcontentsline{toc}{part}{姑妄言第五卷}
\markboth{姑妄言第五卷}{姑妄言第五卷}

鈍翁曰。此一回寫宦蕚之愚蠢。亦可謂至矣盡矣。後來竟到了希聖希賢的地位。何始痛貶之而終過褒之也。古云。相逐心生。相隨心滅。此必至之理。即如一個人有一個上好品格。只往下流處一走。那相貌不因不由。全然改變。就是那下流的形狀。一個極醜惡的人。他一心向上。不知不覺。那醜惡之中就生出許多的慈祥和靄的樣子來。宦蕚之始貶終褒。同此一理。他起初是個癡頑公子。惟知驕矜使氣。那一種呆氣勃勃自然日盛一日。那呆就無所不至。與禽獸幾希。忽爾洗心改變。刻意要做好人。那呆便一日減似一日。久之純是一番仁慈愷惻的心腸。把那呆竟不知往何處去了。孟夫子云。雖有惡人。齋戒沐浴。亦可以祀上帝。何況他不止於齋戒沐浴而已哉。李笠翁奈何天傳奇中兩句說得好。世人莫道形難變。欲變形骸早變心。此之謂也。

此書中不堪之先生者。游系卜通是也。極好之先生者。眞佳訓。\endnotemark[1]廣德厚。劉太初是也。罵游系卜通之先生固然刻毒。獎那三位好先生亦不爲不重。或有先生見此而慍曰。先生與作書者何恨。罵至於此。噫嘻。先生誤矣。但學那三位好先生。自然一字罵不着。若竟要學游系卜通之先生。恐罵破多人口。又不止此書而已。

司富之與宦蕚。千古來兩個奇師生。一旦便豁然貫通。可入詼諧錄。

侯氏之貌之性。人人皆得而畏之。特宦蕚不幸遇之耳。〇翁解嘲曰。我若遇之亦畏。不但宦蕚。

鄔合一段。勿謂其形容太過。舔癰䑛痔之輩。衣冠中代不乏人。由竇尚書雞鳴京兆。拂鬚參政。嗅中丞之足香。嘗太尉之屎苦者。豈非其類耶。又何況於鄔合也。但恐世上更有過於鄔合者。若不自知。鄔合猶不足爲貶也。

姚廣孝之惡。但有知靖難時事者。人人無不痛恨之。今寫他這一番再世之淫惡。更彰其當日之凶毒。諒仁人君子見此一段。只有拍案稱快。決無爲之稱寃者。偶有其人。或亦是不以忠孝爲心。乃此禿之類歟。更有暢快者。姚澤民雖是烝他的繼母庶母。却是姚廣孝淫他的孫婦孫妾。姚華胄爲榮國公之孫。固可稱遙遙華胄。但所生一予民。一澤民。愚者不過只愚其身。賊者則今日辱及家庭。後來敗及王事。且又生一步武乃叔之賊孫。其覆宗滅族宜矣。

萬緣和尚非特寫他以見緇流之壞。借這現在的和尚。罵那過去的和尚。那再來的和尚烝繼母淫庶母。這現在的和尚就淫他的嬌妻艷婢。針針相對。毫釐不爽。

目錄云。現報嬌妻偷僧人淫姪男。此三字妙到至極。明是姚步武私偷桂氏。此不曰姪淫嬸。而曰嬸偷姪者。不如此寫。不見桂氏之淫。不彰姚澤民醜惡之報也。盛旺之奸桂氏者。雖極寫桂氏之醜。然亦有深意。合而言之。姚澤民之腎不旺。裘氏諸妾不爲其淫。姚步武萬緣之腎不旺。桂氏諸婢亦不被其淫也。他一家皆吃了腎旺的虧。

姚澤民奸衆妾以蓮榴起者。二花開於炎天。故二人淫心較諸人更熱耳。後即接寫臘姨者。熱極而冷。時序之理然也。到了雪姐。則冷之至矣。臘盡又當春回。故即出丹姨芍姐矣。終於桂菊者。二花之後。冬即至矣。花俱盡矣。故以他二人收煞。此書雖係小說。作者胸中原有一番大見解。若大槪一看。如何看得出。即此一段中。亦有剝復之理也。勿忽略看之。

素馨同姚步武成奸在佛堂。後同萬緣淫又在佛堂。人家修蓋佛堂。原來留作此用。笑笑。

此書寫各人小傳。無有重者。此寫侯宦兩家是夾敍法。先敍侯敏。次敍宦蕚。正敍侯恭鳳。又接敍宦實。參差錯落得好。

姚澤民訪錢貴。不過替錢貴生色。是歇尾的後文。不可作正文看。

寫姚家諸婦。姚華胄之妻褚氏者。褚鼠同音。謂鼠乃世間第一可憎嫌厭惡之物。且又生下一蠢然之愚子。一狡獪之賊子。此物尚留之奚益。故寫其死去。更騰出此位以讓裘氏。其諸妾丹芍蓮榴桂菊臘雪。及諸婢夭桃紅杏碧梧翠竹紅葉雞冠水仙天竺多人。不過敍四時之景物。

顯而易見。至於裘氏。裘者。繡毬也。繡毬係數十朶花攢成一毬。故以之居衆花之首。後始私姚澤民者。乃爲衆妾做總結也。素馨之氣極香。素馨香兒共係一花。故二人同爲姚澤民之婢。綠蕚。花也。靑梅。子也。本係一物。故二人同爲桂氏之婢。桂氏者。桂花也。桂本極高極貴。古詩云。桂子月中落。天香雲外飄。今反寫他不堪者。桂至北地則不作花。不花之木。樗櫟等耳。賤之可也。故寫他淫其姪男。私及牧馬之圉夫。賤濫至極。較之諸妾婢猶不及。又借之極罵姚澤民耳。裘氏之婢名春花秋月者。

春花喩其時。謂繡毬及春即花也。秋月喩其形色。謂如中秋之月。圓而皎潔也。此等處若不細細指出。看者泛然閱過。豈不負作者之苦心。

\chapter*{姑妄言第五卷\\
第五回 諂脅小人承衣鉢爲衣食計 膏粱公子仗富勢覓富貴交\\
附 再來和尚烝繼母私父妾 現報嬌妻偷僧人淫姪男}
\addcontentsline{toc}{chapter}{第五回 諂脅小人承衣鉢爲衣食計 膏粱公子仗富勢覓富貴交}
\markboth{第五回 諂脅小人承衣鉢爲衣食計 膏粱公子仗富勢覓富貴交}{第五回 諂脅小人承衣鉢爲衣食計 膏粱公子仗富勢覓富貴交}

話說戴遷搭船南來的這少卿。他名字叫做甚麼。他單名一個敏字。他父親原任南京太常寺正卿。致仕歸家。父子別了多年。他吿假回廣東省親。帶着家眷一同還鄕。他有一個妹夫。是個錢可欺人。勢可壓衆的呆公子\footnote{嗟乎。有錢便可欺人。有勢便可壓衆。二語令人慟殺。}。姓宦名蕚。表字盛之。現在南京居住。他到了南京。同妹夫妹子相會了。宦蕚接了舅姆來家。演戲掃塵。不用細說。住了數日。換船起身去了。且說這宦蕚。他父親名叫宦實。以科甲出身。歷仕多年。年將耳順。先在陝西做了十年布政。後陞南京戶部侍郞。目下現任北京工部尚書。他不但官居八座。那家中之富也就不下數十萬了。眞是庫有積金。廩有餘粟。富貴兩個字在南京他家也就要數一數二。後來見魏忠賢威權震主。舉朝文武皆爲之假子。他恐有差跌。也隨衆拜在他門下。做個乾兒。魏忠賢因他是齒爵俱尊的兒子。自然又格外垂靑\footnote{伏後陳忠參本一案。}。因此勢焰滔天。威名嚇衆。夫人艾氏。單只生得宦蕚一個。那宦蕚自幼生得性質粗頑。面皮醜陋。混混沌沌。就像不曾斲開七竅的頑石一般。他父母却十分珍愛\footnote{世間之父母。任兒子愚醜癡頑。未有不珍愛者。此所謂父母深恩。昊天罔極也。}。宦實任南〔京〕戶部侍郞。自幼就替兒子定了侯敏之父侯太常之女爲婦。這侯太常名恭鳳。是廣東南雄府人。家住梅嶺之下\footnote{是個猴族。}。曾中兩榜。先任南太常時\footnote{可謂沐猴而冠。}。正同宦實在一處。他夫人袁氏所生。二子一女。他長子侯敏。任北京苑馬寺少卿。次子侯捷。任太僕寺丞\footnote{一胞生兩個辟馬瘟。奇甚。}。他這位令愛。貌旣不揚。生得尖嘴縮腮。揸耳短項。且是一雙痘風紅眼\footnote{是個猴面孔。}。喜得身肢還嬝娜。手足還纖細\footnote{是個猴形骸。}。却性氣甚潑。纔七八歲。人惱了他。他力小不能打。拉住混咬。把他的乳媼並丫頭們。手上膀子上咬得都是連環血印\footnote{是個猴性情舉動。}。那袁夫人將週花甲。他是個老女兒。一心鍾愛。百般縱容\footnote{世間之物。惟母猴愛子更甚。}。侯太常生平雖不叫做懼內。却也不敢違拗夫人\footnote{此二語大約天下皆是也。}。心中常想道。這女兒如此形狀。恁般性情。等大了。人若知道。那個富貴人家肯要這等媳婦。定成滯貨難嫁。不如小小的。趁我做着官。許下了一個女婿。後來就知他這般醜惡。諒也反悔不得。又想道。女兒這個樣子。要配個聰俊兒郞。不但誤了人家兒子。將來決不能相安。倒是覓一個癡癡蠢蠢的爲妙\footnote{侯太常此想。眞是擇婿良方。不但是想女配夫。且存了許多仁德之念。感應經云。人有善念。天必從之。他不但不曾誤了人家兒子。還成全了女兒。一生享福。皆此一念之所致也。}。他與宦實是會榜同年。往來契密。他每常也見過宦蕚。忽然想起。道。那孩兒眞是個蠢物了。況他父親旣是年誼。且又是現任戶侍。恰是門當戶對。改日遇巧。須如此如此。不愁他不入我彀中。想定了主意。一日。他偶到宦實家來。對坐閒談。眞是個姻緣湊巧。宦蕚已八九歲了。偶然出來頑耍。一個將三十來歲的僕婦背着他\footnote{僕婦也。何必下一年紀。因後日敎導宦蕚。故此處提明耳。}。他手中拿着個播郞鼓兒\footnote{學還不曾上。便知播鼗舞。眞神童。}。幾個丫頭跟着。拿着銀盤子堆着果子的。拿着鬼臉竹馬兒的。還有一個拿着汗巾。貼身跟着。不住替他揩口水鼻涕\footnote{令人笑倒。這個形狀畫也畫不出。}。侯太常一見。便笑着對宦實道\footnote{這一個笑字寫得入神入理。欲說違心之言。故忍不住先笑也。}。弟看這位年姪生得着實敦厚有福。後來大有造化。不在老年臺之下。弟倒有個小女。但恨衙門冷淡。官閒俸薄。不敢攀結。當面失此佳婿爲可惜耳。古人說。知子者莫若父。那宦實豈不知兒子是個蠢蟲。但因是獨子。不得不疼。況家中富貴二字已將到了極處。只要他大了度得出種來。宗嗣不絕就罷了。心中也慮着。將來顯要人家誰肯把女兒配他。門戶低微的又不屑同他結親。正常常以此事躊躇。今聽見侯太常這話。心中甚喜。忙對道。老年臺尊見差了。弟與老年臺何等年誼。多年契厚。何出此言。弟雖知府上有一位閨秀。但恐老年臺將來要回貴省。老年嫂不捨令愛遠留在此。二者因豚兒頑鈍。不足坦府上東床之腹。雖有兼葭倚玉之心。但不敢啓齒耳。倘蒙老年臺俯結絲羅。說到這裡。深深一躬。道。愚夫婦感佩無盡\footnote{心中有欺人之念。故此反被人欺。其宦實之謂歟。然而宦實還算好人。}。那侯太常忙還禮。道。旣老年臺不棄。替女兒結此終身大事。是妙極的了。二人言下而定。宦實猶恐過後有變。就擇了吉期。煩相厚的當道做媒通信。到期拜謝。又擇日請酒。納采下聘。禮幣甚豐。定下了纔放了心。宦實以爲兒子攀了這樣一門好媳婦。那裡知是親家翁使的一肚子猴\footnote{祖晉朝阿智故事。}。這宦蕚到了十三四歲時方延師敎訓。那先生姓游名系字混公\footnote{名與字俱佳。何今日此輩之多也。}。是在宦家一個顯要朋友處謀了薦書來的。宦實一來看情面。二來他原不望子成名。不過說我這樣大人家。且又是科甲門第。豈有不請先生敎兒子之理。圖一個念書名色。故不計好歹。就留下他敎子。那游混公是個捐納的秀才。要他的才學一二三萬萬不能。四五六是考得起的\footnote{好秀才。}。自到了館中。見宦蕚是大老的公子。又是嬌縱慣了的。他雖名曰生員。乃畜生之生。圓活之圓。全沒有絲毫品行。把這位高徒只是一味奉承。不敢稍加拘管。那宦蕚在館中每日只好坐得一兩個時辰。這一兩個時辰之內還是吃果子打瞌睡而已。讀書不過是借他名色上的。一句人之初三個字。敎上千遍。他只是不會。更有妙處。起先敎着。他還跟着念。後來他聽厭煩了。任你怎麼敎。他並不做聲。惟點頭而已\footnote{生公說法。頑石點頭。人以爲異僧。游混公竟敎得宦蕚點頭。也算異師。}。游混公也沒法了。又不敢呵叱他。憑他讀也罷。不讀也罷。那宦實又是溺愛的人。以爲兒子是現成的恩廕。現成的紗帽。何必苦難去讀書。況古人說。何必讀書然後謂學。他縱一字不識。仗我的財勢。將來不愁不富貴。所以總不稽查。那游混公也自有個主意。說。他父母旣不嚴緊。我又何苦與他爲難。況我不過一年。只要束脩不少。每日只要酒食充腸。且我名雖秀才。不過名色而已。況這連年替人做干證走衙門。拿轎馬折酒飯。把書本久已丢去。若忙忙把三字經敎完了。敎到了四書時。倘字眼難認。一時敎不出來。公子倒也混過去了。若被傍人聽出。傳入東家耳中。我這肥館就有幾分不妥。況且如今做先生的有五字密訣。缺一不可。何不遵而行之。那密訣頭一個字就說道。

\begin{quotation}

鬆。

\end{quotation}

我又何苦去緊他。倘得罪了學生。他望着父母說先生利害。父母心疼兒子。恐怕拘管壞了他。一時把二個山字磊將起來。這把館就像喇嘛的帽子。黃到頂了。非徒無益。反害之。這鬆字是第一件要遵的了。第二件兩個字道是。

\begin{quotation}

揸鬔。

\end{quotation}

這兩個字妙絕。古今如今的人。不要說是做先生要穿得體體面面。以起東家之敬。就是傍人看見這樣體面人。可是混學錢騙飯吃的人。定要揸揸鬔鬔。館纔得穩。就不是做先生。如今人眼皮很淺。勢利太重。見穿得略襤褸些。雖至親好友。他向着你只作半個揖。穿得華麗起來。人見了一躬到地。畏而敬之。況我這把持衙門。越要盛服。不但官府肯聽說話。人見我體面。他來尋我的更多。這一副齊整行頭萬萬少不得的了。我曾記得唐朝有一個人。不知叫甚名字。他曾有一首詩道。

\begin{quotation}

而今不用好文章。只要鬍鬚及胖長。

更有一般堪羨處。衣裳漿得硬幫幫\footnote{千古同然。}。

\end{quotation}

當年已是如此。又何況於今日乎。第三件三個字是。

\begin{quotation}

不要通。

\end{quotation}

這個不過說先生太通了。遇着愚鹵的學生。難以爲情。況且人太通了。滿腹珠璣。豈肯做無恥的勾當。去騙館穀篾東翁。館就有些不妥了。要美館把穩。所以說不要通。但這三個字與我合拍之極。不用去學。此時拿了去上剮樁要我通起來也不能夠。可見我做先生。竟是禿子做和尚。天生成的。第四件道。

\begin{quotation}

篾片東翁。

\end{quotation}

這四個字我更在行。不要說叫我奉承。雖使我舔癰䑛痔。我的舌頭比別人伸得還長些。不但於此。就是叫我嘗糞。也只得就學勾踐了。第五件是。

\begin{quotation}

小心待館童。

\end{quotation}

這有何難。我豈但館童而已哉。連闔府大叔。長於我者兄事之。倍於我者父事之。何愁不得其歡心哉。他有了這幾種密訣。熟習於胸。所以宦實宦蕚暨闔家之人。莫一個不歡喜他。數年之中毫無閒言。他敎那宦蕚整整讀了三年。一本三字經方完。完了從新又理。理了重復又念。又讀了二年餘。尚猶不能記全\footnote{宦蕚讀了五年書。三字經不能全記。還算好資性。我見人讀了一生的書。連孝弟忠信禮義廉恥八個字全記不得者多甚。又將奈何。}。宦蕚自己以爲經已讀過數遍。普天下才子恐也無賽於我。因此再也不去念別書。那游混公也不敢勸他再念別書。因因循循。不覺宦蕚年已二十。雖然長成一條肥壯大漢。還是渾然天理。一毫人事不知\footnote{後之享福。焉知不因此。}。他丈人侯太常因年老了。無意功名。吿了病要回故鄕。女兒也二十歲了。催宦家迎娶。宦實見兒子呆呆獃獃。穿衣吃飯還要因人。如何娶得媳婦。甚是着急\footnote{宦實尚有知子之明。過馬士英遠矣。}。沒奈何了。與夫人艾氏商議。叫自幼帶宦蕚的寡僕婦。名喚司富\footnote{名甚佳。}。有四十來歲了\footnote{即前背宦蕚將三十歲之婦也。宦蕚八九歲他將三十歲。今宦蕚二十歲他四十來歲。此等無關係處。一筆不錯。}。吩咐夜間敎他成親的那種妙技。那司富一者不敢違主母之命。二者敎會了小主。後來也有個依傍。與其做這不關痛癢之乾奶媽。不如做沾皮貼肉之實師傅。到晚同他上床。盡心傳授心訣。起初抱他上肚子時。嚇得幾乎哭起來。虧那司富循循善誘\footnote{眞虧他。}。先拉他的手來摸陰戶。又替他捏弄陽物。弄了一會。竟硬了起來。哄着抱上肚子。敎他弄了進去。又扶着他兩胯。叫他抽動。初次還不知道甚麼。做過了兩三次也就領了些。天下事都要學而知之。惟獨此事皆是生而知之。再沒有個學而不能的。這宦蕚人雖愚蠢。倒生得一根成文的好大陽具。又還堅久。

\begin{quotation}

形骸固是同頑石。腰下垂懸有異僧。

\end{quotation}

司富也是久渴了的。每夜定然敎兩三次。雖是假公濟私。也虧他盡心訓導。不幾日。宦蕚竟豁然貫通。不但會而已矣。而且在其行也演習熟了\footnote{好師傅。不但口傳心授。且推身置腹的敎訓。焉得不通。}。司富回覆了主母。宦實纔放心替他娶了媳婦來家。他自從娶過親之後。館中十日半月偶爾一到。到了坐下。不過彼此相混自去。又過了年餘。宦實陞了北京工部尚書。將先生辭了。帶他夫妻同往京中。住了幾年。宦實見兒子年將三十。想已老成\footnote{想已二字妙。誰家父母不心中看着兒子想已老成。孰不知他人見之甚不老成也。}。又見他比當日伶範(俐)了些。況因家資漸漸重了。故此打發兒子媳婦回家照看。那宦蕚不久到家。他因跟着父親在都。宦場中混了幾年。大非昔比。竟是心地如劍如槍。行徑似鬼似蜮。學問雖不曾長進分毫。只他眼眶越發大了。體統越發尊了。勢利越發重了。身軀越發胖了\footnote{活畫出一個貴公子來。}。雖學了些文文縐縐半通不通的話語。却面目生得甚是可笑。有一調西江月爲證。

\begin{quotation}

團團一個肥臉。鬈鬈幾撮黃鬚。眉粗鼻大體如豬。雙眼微微近覷。腹內空空無物。言談字字粗疏。不知何物是詩書。使勢橫行到處。

\end{quotation}

他這妻子侯氏。自幼嬌養。惡性成習。就有河東獅之風。具鳩盤荼之貌。宦蕚這人連天也不怕。父母也不畏的。但是見了他。不由得心中就畏懼幾分。他也常自己想。他一個瘦弱婦人。我這樣一條壯漢。打也打得過他。罵也罵得過他。怕的是甚麼。想到了此處。膽子就壯了起來。走進房去試試。不想一見了面。侯氏把眼一看。他渾身便打一個寒噤。心裡劈劈的跳起來。不知不覺四肢都軟了\footnote{上床後禁不得再看一眼。便不妙了。}。問一句話。那顙(嗓)子不知甚麼堵住。連應都應不出來了。若再三追問來說甚麼。臉脖子都掙紅了纔答應得出兩個字。我不不。試了幾次俱是如此。知道這個硬漢做不成。躱避着些爲妙。喜得腰中有一副爭氣的好本錢。夜裡還可以博他一個歡心。日間輕易不敢入內。只在外廂起坐。他終日在家無事。飮酒食肉之暇\footnote{這也是此輩的兩件正事。}。或欺凌里巷。或唬嚇善良。或嬉戲梨園。或邀遊妓館\footnote{眞是貴公子的要務。}。至於親戚朋友。長親父執。一槪不相往來。只有一個篾片。姓鄔名合。祖代以幫閒爲事。傳到了他。越覺精妙。那諂笑脅肩。撮臀捧屁的身段\footnote{諂笑脅肩撮臀則知之。但屁不知如何捧法。大約非此幫閒世家。他人不能也。}。是他衣鉢。自不必說。更有一種能識人意的聰明。湊趣奉承的話語。人却難及。也有一個西江月贊他的。

\begin{quotation}

撮屁捧臀篾片。伸頭縮頸如龜。假裝一臉笑容堆。䑛痔吮癰慣會。在座惟聞贊好。出門時刻相隨。呼來喝去疾如飛。若論幫閒無對。

\end{quotation}

因他有這些妙處。那宦蕚十分喜他。一刻也離他不得。宦蕚雖是個目無親友一毛不拔的主兒。與他倒相合得來。却常常得他的資助。飽他的酒食。有一首醜奴兒令說他二人道。

\begin{quotation}

脅肩諂笑世皆然。鄔合何尤。更有當羞。今日衣冠盡效尤。

驕頑公子癡愚性。衣食無憂。酒色爲謀。說道詩書勝似仇。

\end{quotation}

宦蕚家中有一座花園。他父親曾請了個文人起個園名。那人取綠竹猗猗有斐君子的意思。題曰斐園。果然山石玲瓏。樹木掩映。樓閣參差。池沼婉曲。十分富麗。一日。初夏天氣。百花盡謝。蓮葉初舒。他斐園中有一叫啖蚊軒。面向蓮池。四圍有數十棵榴樹。前後翠竹參天。桐陰匝地。四面皆窗。一望無際。眞好一個涼爽的去處。你道何謂啖蚊軒\footnote{啖蚊者。何物也耶。罵癡頑公子太毒。}。因取古時齊景公的一個故事。說當年齊景公天暑獨臥。聽得帳外蚊聲喧然。景公道。白鳥營營。是以飢耳。開帳放入。任意恣啖。此軒是他避暑之所。取其豪邁之意。故命此美名。一日。那宦蕚坐在上面一張大涼床上。垂頭喪氣。滿臉愧懼之色。你道他爲何這個樣子。原來侯氏有兩個貼身丫鬟。是他在北京時買了帶來的。一名嬌花。一曰嫩蕊。嫩蕊還小。嬌花有十五六歲了。生得甚是妖嬈。惟獨那一雙眼睛更是動人。竟是一泓秋水。但他斜溜一眼。由不得身上就一麻。他又是北京生長。說話\endnotemark[2]嬌聲嫩氣。身段柔浪風騷。有四句贊他道。

\begin{quotation}

梨影拖肩柳折腰。綠羅裙子繫紅綃。

雖然不比嬋娟貴。亦有婀娜一種嬌。

\end{quotation}

宦蕚久已垂涎。想採他的那一點花心。只因畏懼侯氏。不敢放肆。間或背了侯氏的眼。或望他笑笑。或撂句把邪話勾引。他也不答。只把眼睛斜瞟一下。宦蕚渾身就酥一下\footnote{侯氏看一眼便打一個寒噤。嬌花腰(瞟)一眼便渾身一酥。前後一對。然與其寒噤也寧酥。}。也不知酥過幾千百遍了。這丫頭更有一樁妖樣。宦蕚或向他做醜臉調笑時。他便正顏厲色。竟像不可犯的樣子。及至宦蕚不敢惹他。他又做出那嬌態。扭頭捏項。抿嘴咬唇。或斜溜一眼。或微微一笑\footnote{此所謂撩漢的班頭也。}。把個宦蕚引得魂都不在身上。急得不死不活。這日起來。嬌花服侍侯氏梳洗了一早晨方完。急急的往外走。宦蕚也要往外邊去。一眼瞥見他忙忙向後園裡走。宦蕚悄悄跟到北窗下。往來一張。原來那丫頭一時尿急。到園中蹶着個白屁股正在那裡。

\begin{quotation}

衝破綠苔痕。遍地珍珠濺。

\end{quotation}

看得好不動火。不想侯氏正走了出來。一眼早已看見。正値傍邊放着個棒槌。拿在手中。輕輕走到宦蕚身後。夾肩一連兩下。那宦蕚。

\begin{quotation}

窗〖阝少日小〗始張嫩股。棒槌已及肩頭。

\end{quotation}

幸喜侯氏力弱。不曾打折了肩骨。疼得咨(呲)牙咧嘴。慌忙躱避。侯氏罵道。沒廉恥的。那丫頭溺尿。你偷看的甚麼。宦蕚一手揉着肩頭。掙紅了臉道。我我何嘗看丫頭來。我來看看院子裡可有甚麼花。採些來送你戴戴。如何寃賴我\footnote{該打。丫頭胯下之花固可愛。可是奶奶頭上戴得的。}。侯氏把兩隻紅眼一瞪。道。你明明在此偷看。還敢強嘴。你懷的是甚麼心腸。舉起棒槌又要打下。嚇得他一溜煙跑了出去。被門檻絆得往前一失(交)幾乎跌倒。又吃了一驚。跑到啖蚊軒。坐了一會。又氣又疼。看見兩邊僮僕林立。又羞又惱。甚覺無聊。因命取酒來吃。左右答應了一聲。不一時。海錯山珍。嘉肴異果。羅列滿案。家人將一個蓮蓬頭的紫金鍾。篩了一杯驢精粉調的補腎酒奉上\footnote{是呆公子享用之物。宦蕚改過之後。不復再見用此。}。他獨飮了幾杯\footnote{此可謂壓驚酒。}。愈覺悶將起來。叫過一個家人宦英。吩咐道。你到老鄔家去。說他爲何幾日不來找老爺\footnote{近日如此自呼者甚多。}。今有要緊話對他說。叫他就來。宦英領命就跑。又叫他回來。說道。你說我老爺在園中吃着酒等他。快來纔妙。宦英道。小的只說老爺吩咐。他若來遲了。下次不許他上門。他若聽得這鈎語。自然如飛就到。宦蕚喜道。你好想頭。停當用得。宦英見主人贊他。一團高興而去。須臾。宦英來稟道。鄔相公來了。宦蕚道。叫他進來。原來那鄔合已在槅扇外站着。聽得他叫來二字。就忙忙曲着腰跨進門檻。便一揖到地。道。門下晚生連日未得趨侍。有罪有罪。宦蕚也不起身。只把手略舉了一舉\footnote{是個大老官樣子。}。叫看坐。從人早將一張杌子在桌橫頭放下。鄔合謙遜一番。方敢就坐。宦蕚命斟酒。左右斟上送來。鄔合忙立起身接在手中。滿面假堆一個笑容。說道。連日不曾侍奉大老爺。罪已擢髪難數。怎敢反蒙賜酒。宦蕚道。便酒不必過謙。你且乾過。鄔合深深一恭謝了。然後一飮而盡。方纔坐下。宦蕚道。你連日不來。使我悶極。你在家做些甚麼事。鄔合嘻嘻的笑道。以晚生不曾服事。致大老爺抱悶。門下該萬死。又深深一恭。道。因舍下有些俗冗。幸求寬恕。宦蕚道。你家有甚麼事。鄔合道。因小人終日在家侍奉。那一日傍晚回去。賤內在家。不知何故被人拐去了。因在兵馬司投狀。求他緝捕。故此忙了數日。未曾得覲尊顏。宦蕚道。你這樣一個趣人。怎麼娶這不才的妻子。你也不防範他。被他逃走了。鄔合道。小人妻子平素極貞靜的。終日關門獨坐。從來足不履戶。毫無苟且。街鄰都稱贊他賢淑。焉肯背夫逃走。這是壞人引誘了他去。與小人妻子毫不相干。他雖走了出去。必定還是守節不二的。所以小人急急尋他。不忍捨這樣良婦。宦蕚道。旣然如此。你何不來稟我。我老爺差人去傳諭兵馬司替你拿捕。他難道敢不遵奉麼。鄔合道。若大老爺傳諭他。他奉命不暇。焉敢不遵。但只是晚生妻子末事。不敢干凟天聽。宦蕚大笑道。說得好。說得好。又問道。你妻子姓甚麼。鄔合道。山妻姓嬴。宦蕚訝道。這怪不得他。一個婦人家姓淫。自然就會跟人走了。怎麼他起這麼個姓\footnote{奇談。姓那是起的。非呆公子不能出口。}。鄔合道。這個姓從古來就有。秦始皇就姓嬴。宦蕚笑道。我前日聽鼓兒詞。秦始皇的媽就會偷漢。這是他家祖代傳流下來的了。又道。你只管放心。我差人拿帖子對知縣處去說。叫他上緊去拿。必定就得。你補個失呈進去。這不強似兵馬司麼。鄔合忙起身拜謝。道。這更妙了。叩大老爺天恩。宦蕚便吩咐長班拿帖子到縣中去說。鄔合立起身來。道。小人同去遞了失呈。就來報(服)事。宦蕚道。不消你去。又吩咐長班道。你到縣門口雇人替他寫了。同帖子傳進去。長班應諾。向鄔合問了姓氏居址。鄔合詳細說明。又向他謝了。長班要了帖子而去。鄔合方坐下。宦蕚笑着說道。你妻子旣有人誘他逃走。必定有些姿色。往常怎麼就不與我見一見。況且我待你又不薄。就叫他同我相與相與。我老爺也未必就玷辱了他\footnote{此等語。非此類人不能道。}。鄔合道。小人蒙恩若此。巴不得獻妻出子。惟慚裙布荆釵。上汚了大老爺龍睛鳳目。且恐寒貧粗陋之軀。有玷富貴金玉之體耳。今後倘蒙若獲着。大老爺若不見棄。留爲外宅。小人叨光多矣\footnote{篾片固不足齒。然而罵之太刻。}。縱他貞烈不從。小人定然勸他依順。以盡野人獻芹之意。宦蕚被他奉承得滿胸快活。摩着大屎肚皮。大笑了一回。因問道。你家離我府中甚遠。今日怎麼來得這等快。鄔合道。晚生昨夜夢見祖父說道。宦大老爺天恩如此。你雖有事。明日可去請安。若是宦大老爺一惱。連我們在陰司都有罪犯。晚生今日淸早就來的。因途中遇見了兵馬司差人。同他到茶館中說了一會話。問他賤內可有些影響。然後急急趕來。路上遇着英大叔。聽得說大老爺呼喚。小人恨不得連手放在地下。如狗一般撂着撅子跑來。宦蕚笑道。你家好在行的祖宗。纔生下你這樣知趣的人來。可愛可愛。鄔合忙躬腰足恭道。不敢當。大老爺過獎。宦蕚道。我終日獨坐。除你之外。再沒第二個人可對。故此少你不得。鄔合道。晚學生蒙大老爺天高地厚。自恨無可奉承。但學生聽得人說。當日有個甚麼孟嘗君。門下有三千客。他不過是個公第。尚還如此。何況今日大老爺一位貴公子。要三萬客也有。何不待鄔合去尋些人來趨侍左右。如何。宦蕚道。你雖說得是。但你那裡知道我的心事。你看我何等門第。可是輕與人下交得的。除非與我勢利相當的兒孫。有錢八座的弟子。纔可交往\footnote{可謂善於擇交。何今日此等心胸之多也。}。你想合城中你(那)裡還有像我的第二家。只因你知心識趣。故與你杯酒往來。不然。我這潭府中可是你此輩人到得的。我若泛然混與人相交起來。豈不辱朝廷而羞當士乎。鄔合道。大老爺這段想頭。非天聰天明不能及此。豈晚學生下愚可到。因打一恭道。承敎。又道。古人說。聰明不過帝王。伶俐不過大公子。果然不謬。但晚學生想來大老爺這樣抱悶。晚學生恨不欲捐軀。但恐怕死而無補耳。以小人一人之便嬖。旣不足使令於前\footnote{有人看至此。謂篾片雖然無恥。未必至於此極。然而旣無恥矣。雖形容太甚。亦不爲過。}。而外邊這些王孫公子。或八座而不富。或金多而位不顯。實在也不屑同他相與。萬不得已而思其次。或大老爺族黨中的叔叔兄弟。揀幾個知竅些的。從新交友起來\footnote{千古奇談。}。朝夕盤桓。他同於祖宗一脈。或還不致於有玷。宦蕚聽了。勃然變色。大怒道。不通。可惡。放狗屁而胡說者也。可惱可惱。鄔合不知是爲甚麼。嚇得戰戰兢兢。忙出位跪在地下。自己打了幾個嘴巴。哀吿道。小人失言。不識忌諱。死罪非輕。小人情願領死。萬不可氣了大老爺玉體。連連叩頭不止。宦蕚見他如此。便道。你起來坐了。我不罪你。那鄔合那裡敢起來。叫了數次。方站起侍立。神色猶自未定。宦蕚叫他坐下。說道。你罪坐不知。尚猶可恕。但你草茅下士。那裡知道我閱閥朱門內中的一團大道理\footnote{可佇目看他說這一團大道理。}。你就說這些窮族間可是招惹得的。就有幾個匪長輩百般會奉承我\footnote{長輩而下匪長輩奇稱。然而長輩若奉承晚輩。加一匪字。亦所當然者。}。我不過不好意思同他鬼混。我豈屑睬他。至於說起祖宗二字。我正在此恨他如醋。一者他當日不能掙一個大大的官做。今日叫我一個八座公子。逢年遇忌替他叩頭。已是氣得發昏\footnote{這想頭奇甚。}。這還情有可恕。還有一件。你當日代代單傳。只生我家父老先生一個\footnote{這也是奇稱。}。今受誥贈敕命。就夠你榮耀得很了。又無緣故生出許多沒要緊的兒女來\footnote{這一想更奇。}。若都做八座的官。都像我家的富。不丢我臉面。不來沾染我。不辱沒我。也還罷了。却又有窮的窮。賤的賤。不是來呵我寫字與老爹去照看他們。就是來搊我要吃我的東西。就把我的脬呵腫了。我只是不快活。我如今疏遠他們。還怕人知道。說此人是宦太老爺房分中的兄弟。或是宦大老爺隔從的叔叔。使我羞臉難當。我不理他們。他還無恥常來纏擾。我避之猶恐不及。若再與他往來起來。我在這世上一刻也存站不住。你知道的。我那姑父劉太初。一個大窮秀才。敎書餬口的人。他見了我。不自己害羞。還要做那姑爹的身分。我氣得要死。總不理他。他倒還知機。總不到我家來纏擾。你想我一個萬人之上。三人之下的一個人。怎肯下氣在這些窮骨肉跟前。豈不懼哉識者所笑。你道我說得是〈甚〉麼\footnote{原來是這一團大道理。眞高出於古聖先賢之上。使人耳目一新。不愧爲貴公子。}。鄔合此時魂纔返舍。見他說得如此稀奇。又不敢笑。忙道。大老爺金語。誰敢道半個不字。但小人井底之蛙也。焉能識此深奧之理。無知冒犯。幸蒙寬恕。粉身碎骨。難報厚恩。但適纔大老爺說。萬人之上。三人之下。所謂謙謙君子。只是未免太謙。據晚學生看起來。今日大老爺可謂萬人之上。無人之下的了。宦蕚道。你所說固是。但只是如今上有皇帝。有魏上公。有老爹先生。我豈非三人之下乎。鄔合聽了。咂着嘴道。是呀是呀。小人愚鹵。見不及此。又出了一會神。笑起來道。晚生蒙恩。無可爲報。今想了一策。爲大老爺高陞一級。竟做二人之下。如何。宦蕚喜道。你必有妙論。快快說來。鄔合道。大老爺所說。只讓皇帝魏上公並太老爺三個。晚學生細想。皇帝魏上公是再大不去的了。只有令尊這一位。雖不能居之在上。還可以與之爭衡。只是晚學生覺得言重礙口。不好說得。宦蕚道。你不要拿班做勢。若果然眞有妙計。我自重重賞你。鄔合道。如今令尊是魏上公的令郞。大老爺何不備一分厚禮。也去拜在魏公的門下。認他做個老子。那時與令尊雁行比肩。序起兄弟來。不認父子。無所統屬。豈非只在二人之下乎。宦蕚歡喜得倒在涼榻上。大笑道。哈哈哈。哎呀。妙呀。好奇想。好算計。起來向着他道。雖封神記上的姜子牙。三國志中的諸葛亮。鋒劍春秋的孫伯齡。也沒有你這樣的妙計學問。我同你相識多時。竟不知你有這樣大才學。可敬可愛。鄔合笑道。小人當日原極愚蠢。蒙大老爺培植得福至心靈。連學問計策都有了。此皆大老爺之賜也。二人說得投機。從新添上精肴異饌。美酒佳釀。吃了一會。宦蕚道。吃酒頑耍。定要三四個人纔有趣。你的學問高。見識廣。還想個妙策。訪得一兩個絕頂富貴的朋友方妙。鄔合一面吃着菜。呷着酒。一面說道。適間得罪大老爺。雖蒙寬恕。至此時猶悚懼不安。如何還敢多嘴。宦蕚道。我不過是一時雷霆之怒。過後即休。你看我此時還有一毫惱你的氣兒麼。你不過不肯上心。故以此推托。鄔合假做吃驚。把脖子縮了兩縮。道。大老爺如此說。小人就當不起了。況大老爺之事即晚生之事。且糾合得幾位大老官來。小人也多幾碗酒喝。於此賤腹。豈無小補之云哉。敢不上心。今晚生雖大啖大嚼。而此事未嘗一刻去懷也。晚生倒想起一家來。不知可敢說麼。宦蕚道。你且說了看。鄔合一手執杯。一手持箸。嘴合在酒杯上。眼釘在菜碗內。不住亂吃。那裡還顧得答應。宦蕚道。你把杯箸權且放下。我同你商議正經話。你若有富貴好人薦了我\footnote{嗟乎此語令人傷心。富貴便是好人。貧窮自然都是不好者矣。}。我們結了酒肉社。那時日日有得你吃。何必此時這等着急。鄔合見說。沒奈何。只得將杯箸放下。袖中掏出塊帕兒來擦了擦嘴。說道。城中有一個富翁。叫做童百萬。大老爺可曾問(聞)名麼。宦蕚道。我也知道此人。却不曾會過。不知果是如何。鄔合道。那童百萬名自大\footnote{今日何此名之多也。}。晚生也認得他。他家裡眞是豪富。金銀滿庫。米豆千倉。圓的是珠。光的是寶。犀牛頭上角。大象口中牙。十數座當鋪。千百間佃房。南鄕的田。江北的洲。山中的大木。江裡的魚套。都是有的。雖比不得老爺府上奢華。在南京也還頗充得第二。晚生愚意。像這樣人家。將就同他相與也罷了。宦蕚搖着頭道。他錢倒也罷了。只是沒有官勢。如何好同他往來。鄔合道。他近日大非昔比。也爲人欺他沒勢。他去年拿了好些銀子。納了一個甚麼團于藍的頭一名監生。他自己說大得很呢\footnote{自己說。妙。人無有不自己說大者。不知他人見之。其小無比。}。不過四五十年就要選州左堂\footnote{還是欺人的大話。大約四五十年尚未必選得着。}。比本縣大爺還大一級\footnote{這是眞。}。這州左堂不知是件甚麼東西。大約大得很了。他還嫌官小。要到黃河裡去效用。據晚生揣度。他這一到河裡。大約鱉都司的前程。他自然有的。昨日回來。竟擡了一頂比四〔人〕轎還大的二人轎。〈夫〉四名轎夫輪班擡着走。那轎衣都是北京屯絹做的。五岳朝天時樣的大銀頂。耀眼爭光。跟着一陣家人。穿得好不體面。都是馬尾織的瓦楞帽兒。一色油靑布直裰。淨鞋淨襪。夾着一個描金護書。說是外國獅子皮做的。裡面放着許多灑金硃砂箋拜帖。又有一把大銀頂雨傘。說是高麗紙裱的。蘇合油搓的。偶然撑將起來。眞是遮得天沒日頭呢。還有一張交床。上面放着一個像小孩墊底尿褥子。灰灰的顏色面。就不曾細看是甚麼做的。大約也自然是件寶貝了。晚間打着一對大罎燈。一邊寫着候選州左堂。一邊是通紅的童衙二個大字。好不官樣。一個長班在前喝道。竟同那些街道巡廳坊官捕衙衆位當道老爺們來往。街上人看見。都哎(咬)指側目。遇見他是猶恐避之不及\footnote{懼其勢耶。畏其臭耶。}。誰敢不叫他一聲老爺\footnote{借鄔合口中。極力貶江南暴發戶之援例輩耳。看者勿責作書人嚼舌。}。鬧熱得緊。晚生曾聽得人說。他七八代前的祖宗。在古時也曾做過八座的\footnote{這才眞是遙遙華胄。}。據晚學生看來。除了大老爺。也就要數他呢。宦蕚道。你說得如此動火。姑算一個。怎麼再得一個纔妙。你再想一想。正說着。長班來回話。說帖子同呈子都送到縣裡了。縣大爺說知道了。自然領命。鄔合又向宦蕚道謝。望長班說了動勞\footnote{好。必有之事。必至之理。}。坐下。忙忙喝了幾杯酒。吃了幾箸菜。又想了一會。道。又有一家。是前科發甲的賈老先生諱文物的。他令先尊賈翰林名播一城。他令先岳富戶部官聞四處。他家中房子住着有幾十進。門面漆得雪亮。彩畫得光燦陸離。正中懸着個伽楠香的匾。斗大進士第三個石靑地的金字。外面豎着四根沈香旗杆。刷得通紅。下邊白玉石雕花鼓子。這個體面豪富。在城中也就要算第三家了。至於他肚子裡。晚生粗人。雖不能窺其際。但聽他說一句話。就文縐縐得可愛。眞是出口成章。間或腆着肚子搖擺起來。果然是那名公的體態。比那俗人大不相同。若除了他。再遍尋也沒有了。宦蕚皺着眉。道。罷。倒也罷了。只是聽得他的舉人進士來得有些不明白。恐人譏誚我這樣一個大公子眼中不識人。鄔合道。大老爺又來\footnote{這半截話妙甚。本要駁他說得不是。却不敢出口。連忙縮住接下語。}。他雖有些不明白。如今公然說是科甲。誰敢說他是假的。他又拜在魏上公門下做了親孫子。誰不尊敬他\footnote{親孫子強於假進士多矣。}。敢道半個不字。況他連詩都會作的。若同他相與了。哏。人還要誇大老爺有眼力呢。宦蕚道。何以見得。鄔合拿個指頭在桌子上儘(畫)着圈。道。人都贊大老爺是富貴才子。所以纔相與這樣富貴文人。有此美名。誰不欽仰。豈不妙哉。宦蕚道。我倒不管他才與不才。旣有財勢。你明日就去對他兩人說。我大老爺從不屑下交的。因慕他的豪富。要同他做個朋友。看他們意思何如。說明白了。就來覆我。我明日下午等你的回話。鄔合道。他二人聽見是大老爺要相與。自然欽此欽遵。敢不從命。晚生明日去說明白了。定來回覆。說了。又連吃了十數杯。酒已大醺。日將云暮。起身作了八九個揖。作別而去。眞是。

\begin{quotation}

朱門諂脅人無數。茅戶親朋半個無。

\end{quotation}

將此後文權且按住。再說錢貴自從遇了鍾生。立誓潔身以待。正想尋個由頭。做個下馬威。恰巧竹思寬要想嫖他。被他一場撒潑。罵了幾日。郝氏也覺得沒趣。過了些時。見他氣性癱了些。又勸他接客。他決意不從。又過了些時。北京來了一位貴公子。拿了五十兩銀子來。要嫖兩夜。鴇兒愛鈔的心腸又動。先好勸他依從。錢貴誓死不依。後便加之凌迫。錢貴幾幾乎喪了性命。郝氏雖愛錢心盛。到底是他親生之女。恐當眞弄出把戲來怎處。只得再三婉求。辭那貴公子去了。你道這個公子何方人氏。姓甚名〈姓〉誰。聽我慢慢敷演出。便知詳細。當日嘉靖皇帝時。偶然想起昔年隨太祖平定天下的這些功臣。後因永樂篡奪了建文。有不肯依附者。盡皆削奪世襲。嘉靖不忍負他們的功蹟。皆繼絕世。命查他們嫡派子孫。承襲封爵。劉誠意徐中山常開平等子孫都襲了公侯伯之爵。又想起少師榮國公姚廣孝。永樂篡奪之力。全是他功爲第一。他雖是個和尚。必定兄弟叔姪還有宗支。奉旨到無錫縣查訪。那時有個姓姚的。名字叫做姚華胄。家私富貴。人也不是個一絲無能的。不論九流三敎。諸子百家。他雖未必件件精通。却也無一不曉。且那一張利嘴。談兵說劍。論古敲今。口若懸河。容易人也說他不過。正在英年。生得好個齊整相貌。姚華胄此時聞了這個旨意。到縣中具訴。說他\endnotemark[3]是姚廣孝嫡派子孫。應當承襲。知縣駁道。榮國公應當襲爵。僧綱司何由有孫。姚華胄初意說是榮國公的親孫。萬無不准。就不曾想到他一個和尚如何有兒子傳代。見知縣這一駁。着了急。暗饋了知縣一分厚禮。改報係姚廣孝胞弟姚廣忠子孫。自來相傳。只聞得姚廣孝有一位賢姐。並不曾說他有兄弟。你道這姚華胄到底是誰的子孫。這姚廣孝本醫家之子。他父親精於岐黃。生性侫佛。只生一子一女。他那女兒眞是個女中丈夫。識字知文。深明大義。夫死守節。敎子成人。他雖是個女流。強似那鐵錚錚的漢子。自從姚廣孝助燕王篡逆。他知道了。恨入骨髓。後來姚廣孝封了國公。衣錦榮歸。那時他父母已歿。來見賢姐姐。他賢姐姐關門不納。隔籬道。我家從無此貴人。姚廣孝識其意。變僧服而往。姐猶不與見。家人勸之再三。其姐不得已開門。自立於中堂。姚廣孝入。拜謁甚謹。姐怒道。世上做和尚不到底的可是好人。便抽身而入。姚廣孝愧赧而出\footnote{姚廣孝固乃姐之罪人。然尚有人心。若今之今(人)少得微名。即伯叔亦渺視之矣。何況於姐也。}。這樣婦人與狄梁公姊爲再見耳。千古何可多得。且說姚廣孝因何幼時出了家。他自幼聰明狡獪。那時神相袁珙見了。向他父親道。此兒目生三角。形如病虎。其劉秉忠之流亞歟。若令習儒。恐其不壽。若使之爲僧。將來貴爲帝師。他父親遂送他〔去〕一個素常相與的和尚法號圓通的庵中出了家。他並不是自己願去苦修。是沒奈何做了和尚的。他那師父圓通。〈庵〉也是那時有名的才僧。他愛姚廣孝聰明淸秀。日間盡心敎他經典並詩詞之類。夜間便同他成了夫婦。這是和尚傳家。留得衣鉢。原不足爲異的事。他到大來。雖有過人之才。却有兼人之惡。且素行不端。無恥特甚。他私偷着一個姑子。生下一兒。他不好認得。他有個族弟叫做姚廣忠。瞞着姚廣孝與這姑子也有相知之雅。姚廣忠無子。姚廣孝就把這孩子與了他做兒子。就算了姚廣忠之後。姚華胄就是此兒的子孫。論起來。却實在是姚廣孝的嫡嗣。姚廣孝當年久而久之。醜名漸張。鄕站不住了。遂到南京投拜太祖信愛的一個和尚。叫做宗泐。宗泐却不知他的壞處。見他相貌才學都好。甚是愛他。替他起個法名道衍。法號斯道。那時有一個王行。看透他的心述(術)。說道。斯道非若他人事佛奉師碌碌久做沙門者也。宗泐將姚廣孝薦與太祖。後來每個親王賜一員僧綱司。就把姚廣孝分與了燕王。太祖上賓之後。太孫建文繼統。他一來見朝廷年幼。二來想做佐命功臣。力勸燕王謀反。篡奪了建文的天下。改元永樂。算他功居第一。遂封了他少師榮國公。永樂賜了他幾個宮女。他此時要假裝活佛一般。不肯拜賜。況且又有小沙彌做了內眷。何須要此。永樂越重他的德行。後來人都說姚少師是位眞僧。不貪女色。那裡知他因位尊了。要博虛名。就不知他少年的醜行。他位至國公。歷蒙恩賜。不下數萬。都給了姚廣忠。以貽他所生之兒。傳流了將二百年。到了姚華胄。尚然巨富。姚華胄起先報是姚廣孝嫡孫。見知縣一駁。故此又報是姚廣孝胞弟姚廣忠之後。只把胞字換了個族字。那縣官得了他一分重禮。竟據他的話呈報上去。上司難辨眞僞。輕易不敢啓奏。又仰縣細細淸查。古人說。有錢十萬。可以通神。縣官受了他的重賄。如何銷繳。況且又沒處查證。竟具了印結。說查係姚少師胞弟嫡派子孫是實。上司據文題請了姚華胄。又關通了嚴嵩父子。雖假亦眞。奉旨召他進京陛見。他到面聖之時。應答如流。嘉靖大喜。以爲非姚少師族裔。焉能有此文武全才的英物。遂准襲封了侯爵。那時天下太平。他談天說地。佈陣排兵。每每有英雄無用武之地爲恨。這些朝中臣也有信以爲實的。着實敬他。都誇是武侯再世。留侯復生。爲朝廷欣慶得人。也有惱他大言不慚的。却不好同他辯駁。他歷過了隆慶萬曆泰昌三朝。也享用了五十餘年。他生了二子。長子姚予民。是個蠢然一物。食粟而已。次子姚澤民。他妻子褚氏生姚澤民的那一夜。姚華胄夢見一個和尚直到內室來。心中大怒。道。何物奸僧。輙敢到我內中。那和尚愀然道。我是你始祖姚廣孝。生前殺孽太重。冥冥之中受罪二百餘年了。你今又無故受朝廷重爵。明朝氣數將盡。天帝敕旨。命我來與你爲子。以完前孽。結此一段公案。說完。往褚氏胯下一鑽。就不見了。姚華胄驚醒。正値褚氏腹痛。須臾生下一兒。姚華胄雖知他是祖宗轉世。却不解他完前孽的話。遂起個乳名。叫做祖官。說他大了就學祖爺平定天下。澤及生民。故此命名爲姚澤民。褚氏生他的那一夜。正矇矓睡着。也見一個精赤條條的和尚爬上床來。褚氏又驚又喜\footnote{喜耳。何驚之有。}。正要問他那裡來的。不想那和尚將光頭向他陰門裡就鑽。褚氏驚道。哎呀。這和尚好呆。這個大頭如何鑽得進去。正要用手來推時。不想一下已全身鑽入\footnote{是極。小頭猶可。大頭如何用得。孰不知竟鑽了進去。夢中之喜可知。一笑。}。只覺陰門一脹。小腹微有些痛。驚醒來已要分娩。這和尚進去旣易。出來更是順溜。褚氏毫不費煩難。就生下一個兒子。好生歡喜。褚氏將所夢對姚華胄說了。姚華胄也將夢吿訴他。夫妻深以爲異。姚華胄平生酷信佛法。家中蓋了一所佛堂。請了他素常相與的一個和尚。法號萬緣。是報國寺的住持。纔三十來歲。來家中供養。這萬緣諸般經典皆能。又生得面白頂圓。身長軀大。好一個相貌。有幾句贊他道。

\begin{quotation}

靑旋旋一具光頭\footnote{夫頭也而以具稱。奇甚。}。白晃晃一枚大臉。兩隻眼半睜半閉。假裝出慈悲面孔。一張嘴一合一開。眞講得天花亂墜。素珠百八。時掛胸前。佛法三千。全抛腦後。口中阿彌陀佛。何嘗住聲。心內窈窕佳人。未曾斷想。姚華胄當他是現在菩薩。誰知他是個色中餓鬼。

\end{quotation}

他且又善說。華胄敬他如活佛一般。闔家都尊稱他爲大師傅。姚華胄就把祖官寄名與他做了徒弟。這祖官生得胖壯標致。夫婦心愛異常。買了個奶娘帶乳。又撥了個十來歲的小丫頭素馨相幫抱持。祖官到了七八歲。生得嬌皮嫩肉。肥頭大臉。心雖聰明。性情狡獪\footnote{前說姚廣孝也是此四字。先後一應。}。他雖這樣大。還是乳母每夜帶他同臥。那乳母同丈夫在被窩中再沒有不做些正經生活的。他夜間偶然醒來。見他二人亂動。還不知道是做甚麼事。到了暑天。二人精光的高興。他纔見是奶公的這件東西鑽入他乳母胯下那洞中去。二人就動起來。方悟向來之故。他那小㞠子也竟知硬幫幫的亂跳。他十歲那一年。那乳母一病而亡。夜間沒人帶他睡。夜中啼哭。褚氏親帶他睡也不肯。因素馨自幼背抱他。他要跟素馨睡。此時素馨二十歲了。已配了漢子。名喚吳實。二年有餘。又另撥了個小丫頭香兒服事他。褚氏便叫素馨帶他睡。他此時又大了〈有〉些。知識大開。常見素馨兩口子。也如他乳母夫婦在被中亂動。却動不多幾下就住。知是那件事了。好不難過。一日。吳實奉差他出。秦馨同他睡着。天明時。祖官醒了。見素馨不曾蓋被。赤身仰臥。兩腿大揸。他悄悄起來。爬到脚頭。向他胯下一看。方知這件美物是如此形狀。他那㞠子也竟有三寸多長。不覺大硬起來。也竟公然爬上肚子。對了門戶。弄將進去。一抽一抽的動。素馨驚醒。見是他。笑道。這點個人兒也學幹這事。還不下去呢。那祖官將他的腰一把緊緊抱住了\footnote{抱住了腰。活是個小孩子。勾不着肩臂。自然是抱住腰了。}。連忙亂抽亂扯。原來素馨的男人吳實。雖然二十多歲。此道與祖官的相彷彿。而且甚快。雖配了二三年。素馨還不曾知道丢過。尚不知其中滋味。此時祖官尚小。無精可洩。儘着抽個不歇。素馨覺勝似吳實許多。見他年小力弱。雙手抱緊他小屁股。往下舂搗了好一會。竟被祖官將他弄丢了一次。他愛祖官了不得。祖官初嘗這件異味。一上床。就在他肚子上不肯下來。素馨因他身子輕小。馱着也不費力。任他壓在肚皮上。一時間抽動起來。素馨乍得樂境。便有許多淫聲浪氣。祖官更覺有趣。兩人也盤桓了十多日。吳實回來了。他兩下便阻了佳期。心中好不難過。一日。吳實隨姚華胄出門赴席。祖官得了這個空。要同素馨敍敍。素馨何樂不爲。但那香兒丫頭隨在身畔。祖官支他道。你到上邊要些點心來我吃。香兒去了。二人忙上床。解帶脫衣。就弄起來。他兩個別了好幾日了。彼此不肯便歇。弄個不休。不防香兒要了一盤點心來。不見祖官。只說他在床上睡覺。掀開帳子。見他兩個。光肚子壓着光肚子動呢。香兒也十四歲了。雖不曾嘗過。豈不知道。便將盤子放下。笑嘻嘻避了出去。素馨覺得有些不好意思。向祖官道。這丫頭被他看見。恐一時傳將出去。老爺夫人知道了。你便不妨。我就不好了。就是我男人知道也不好。你須把他也弄一下。纔禁得住口風。祖官道。你放心。在我。二人穿起衣服來。祖官走到堂屋裡。那香兒望着他笑。他見沒人。上前摟着親了一個嘴。就去扯他褲子。那香兒推着他的手。說道。大靑天白日。人來來往往的。你這做甚麼。還不放手。祖官也怕人撞見。只隔着褲子將他胯襠摳了幾下纔放了。兩人笑個不住。那日天氣甚熱。下晚素馨在房中洗了個澡。香兒就接着水洗。却好祖官走來。素馨道。香兒洗澡呢。你快去。祖官忙脫光了。推開門。跑進房中。香兒正坐在澡盆裡。不防一下被他推倒。撲上身。分開兩腿就戳。香兒雖是個處子。下身被水浸得濕〖氵韲〗〖氵韲〗。一下子攮了進去。香兒哎呀了一聲。已被他抽將起來。祖官覺他的比素馨味更緊美。抽了幾下。香兒推他道。你起去罷。看有人來。祖官已得了趣。也就起來。揩了身上。叫素馨拿衣服進來替他穿了。三人你望我笑。我望你笑。此後打成一家。混弄了幾年。他那陽物竟長得。

\begin{quotation}

量去長將六寸。粗圍一虎餘零。衾中偶爾一交兵。抽送千回猶勁。

\end{quotation}

祖官十五歲上。姚華胄替他娶了個錦衣衛姓桂的女兒爲媳\footnote{明季錦衣衛與東廠表裡爲惡。故罵之生此淫賤不堪之女耳。}。粧奩甚富。陪了兩個丫頭。一名靑梅。一名綠蕚。不到一個月。姚澤民將他主婢三人都受用過了。他貪淫無比。雖有妻婢三人。連香兒四個了。那素馨他還不肯放過。常同他在西間屋裡行樂。那素馨的男人是有名無實的。他時常假說上邊叫他上夜。每每的來就敎。那桂氏也纔十四五歲的小女子。並不知吃醋撚酸。倒過得和美。姚澤民到十八歲上。他母親褚氏死了。姚華胄大兒子姚予民送回故土祖塋安葬。這姚華胄天性有些懼內。那褚氏雖不撒潑降夫。但姚華胄想要娶妾置婢。雖有此心。却也不敢出之於口。姚華胄曾試探過他的意思。一日。夫妻閒話。姚華胄笑說道。人但開口。便說妻妾。此二字相連。可見妾之一物。人家亦不可少者。不過要他來侍奉夫人之意耳。這些做夫人的都錯會其意。以爲是丈夫圖取樂。每每不容。豈不可笑。褚氏冷笑道。古云。四十無兒。方纔娶妾。妻已生子。那妾之一字也就可以不必。人開口也就說婢妾兩個字。可見婢是人家不可少者。那妾在婢之次。是可無之物了。至於說要侍奉夫人。愈覺可笑。豈婢不可侍奉而必欲倩妾耶。那都是沒良心男人之飾辭耳。說得姚華胄閉口無言。只得息了此念。今見褚氏死了。他年將望七。不肯自量。把這數十年的豪興發將起來。娶了二十歲的女子爲繼夫人。是個已故光祿寺裘家的女兒。十分標致。他家中後園內原有春夏秋冬四景。都有房屋樓閣。向來只得幾個蠢丫鬟打掃看守。以備他老夫妻遊玩。如今沒有管頭了。他差人回南京。在應天揚州蘇州杭州買了四個美妾。每人各置一艷婢。又在北京山西也買了四妾。婢亦如之。兩妾二婢同住一室。只供宴樂。其灑掃支使。自有當日的粗蠢丫頭。他那春景有牡丹臺芍藥欄。四週桃杏梅李圍繞。花開時却也芬芳馥郁。燦爛如錦。命二妾一正一副之。一個稱丹姨。一個叫芍姐。夏景四面一池蓮花。池中有水閣。池畔數株石榴垂楊掩映前後碧梧翠竹。薰風徐來。蓮香撲鼻。也着二妾主之。一個叫蓮姨。一個叫榴姐。秋景有幾棵老桂。一片菊圃。海棠玉簪雞冠紅葉之類相襯着。甚是幽雅。到芙蓉半吐。菊英大綻之時。一片鋪如錦繡\footnote{桂花到京不花。亦異事。故不題。}。也着二妾。一喚桂姨。一喚菊姐主之。冬景有許多臘梅。高矮參差。雜着數叢天竺。紅綠相間。屋角又有許多迎春探春忍冬諸類。室內列數盆水仙。玉玲瓏。旱梅。大盤香櫞。佛手。香氣氤氳。頗覺不俗。也命二妾主之一。個叫臘姨。一個叫雪姐。他雖有這些嬌妻美妾艷婢。但將七旬的老漢。精力有限。雖然個個都曾開闢過。要想時常點綴。雖有此雄心。却無此健力。只好把這些婦人做個擺設的肉玩器而已\footnote{奇語。}。要個個鑽硏却不能夠。這些少年婦女如何貞靜得住。但他的家法頗嚴。三尺之童不許入內。雖他長子姚予民。孫子姚步武。也不敢擅入。惟這姚澤民是他的愛子。又見他年幼。只容他一人不時出入。這些妖精般女子守着個髮如彭祖。鬚似李聃的老叟。已是憎嫌。況且又是上面皤然一公。底下公然一婆\footnote{廣西獠人稱老漢曰婆。想亦是此意。但稱老婦曰公。不知何謂。}。沒用的厭物。一月中還不能領敎他一次。即有一次。皮條般陽物。屢屢中止。一毫樂境也無。反引得淫情似火。叫這些人如何過得。見姚澤民這樣精壯少年。年紀又不相上下。眼中都冒出火來。恨不得拿水將他一口嚥下肚去。但一見了他。說也有。笑也有。姚澤民先還不敢放肆。後來日近日親。況他又有淫癖\footnote{二字新奇。}。就想要替他令尊代起勞來。也就同衆人打牙犯嘴的說笑。他因有意於衆人。這些妾婢也沒一個不注意於他。皆因未得其便。故此不曾上手。一日。那蓮姨同榴姐乘着涼。兩人說笑了一會。偶然講到夫妻行樂之處。蓮姨忽然長吁了一口氣。道。我在家做女兒時。我的臥房同哥嫂房中隔着一層籬笆。每夜聽得他們歡笑。我間或張張。見他二人那調笑快活。眞有登天之樂。也不枉叫做夫妻。如今我們不幸跟了個老頭子。雖不愁吃愁穿。却守了活寡。不如嫁個窮漢。一夫一妻還得受用。榴姐道。這是各人的命數。事已至此。怨也沒用了。何不自去苦中尋出樂境。爲甚麼癡癡的枉躭誤了靑春。蓮姨道。我何嘗不想到這裡。但此處除二公子之外。再沒有第二人進來。要敢來尋樂境。除非就他身上。榴姐微笑道。我也正是此意。但不知姐姐心下如何。旣有同心。事不宜緩\footnote{始於他二人者。二花炎天大茂。故淫心更熱耳。}。我冷眼見衆姊妹都有心到他。若不先下手爲強。恐被別人占了先去。我們去雌着。人就沒趣了。蓮姨道。旣安心做這事。丫頭們眼多。瞞不得他們的。倒替他們說明了做。纔可行得。遂叫那兩個丫頭。一個名碧梧。一個名翠竹。到跟前。說道。你兩個在我們身邊。我們待你如姊妹一般。我有句心腹話對你二人說。你若同心協力。包你也有好處。兩個丫頭道。我們蒙姨娘姐姐擡舉。難道是死了心的不成。豈不知道。姨娘有話。只管請說。蓮姨榴姐同聲道。老爺有年紀了。我們都靑春年少。白白的躭誤着。守的是甚麼貞節。我們意思要相與個趣人兒。以消寂寞。你們怎麼說。那兩個丫頭道。這却難。外邊的人如何進得來。我們又出不去。勸姨娘姐姐打掉這念頭罷。蓮姨笑道。癡丫頭。難道我不知道。眼面前現放着有一個。何必要你去尋。碧梧道。要是眼面前的。無非就是二爺。蓮姨笑道。你好猜。就是他。碧梧也笑道。要說他。倒容易。不敢瞞蓮姨說。前日姨娘姐姐到夫人上邊去。翠姐也跟去了。只我在家。他忽然走了來。見沒有人。生生被他把我強奸了。我又強不過他。只得憑他弄了一會\footnote{許多人。起手第一個是碧梧者。有深意也。琴皆以桐爲之。古謂琴心相挑。借此意耳。}。他求我做個媒。要同姨娘姐姐相與。他說不知二位心中如何。不敢自己開口。托我探探口氣。我辭他不敢。未曾應允。若姨娘姐姐有意。這事手到擒來。蓮姨滿心歡喜。笑道。不想你這丫頭倒得丫頭籌。你覺他比老爺如何。碧梧道。大着呢。他那件東西像生鐵一般。那裡像老爺那樣軟叮噹的。況且又長大了好些。工夫又久。把我弄得酥了好一會纔醒過來。那蓮姨榴姐聽得臉上一陣陣火發。商議道。他雖時常相見。怎好就幹這事的。對碧梧道。我假裝睡。你去約了他來。叫他偷我\footnote{眞是掩耳偷鈴。}。等他弄上。就不覺羞了。等他弄着。榴姐來衝破。大家一齊上手。榴姐笑道。旣安心做這樣的事。還怕甚麼羞。我是不怕的\footnote{更老辣。}。就依着姐姐這樣來。向碧梧道。你快些去看他在那裡。約了他來。碧梧纔走出門。遠遠見他來了。忙進來道。來了。蓮姨忙到房中。脫了小衣。只着單裙。在床上假裝睡着。故意蹺着一隻腿。裙幅掀開些。陰門微露。榴姐也躱過。碧梧將出來。姚澤民走到跟前。見沒人。摟了親了個嘴。問道。我托你的事怎樣的了。碧梧道。我怎麼好開口的。他此時正在房中睡覺。你何不去偷他一偷。料道不得變卦。若弄上了。不強如我說麼。姚澤民滿心歡喜。輕輕走進房來。揭開帳子一看。見他上身雪白露着。只穿豆綠廣紗抹胸。下着大紅縐紗單裙。此道微露細細一條紅溝。不覺陽物大舉。脫了衫褲。上得床來。爬上身。看準一攮。就送了進去。抽了十多抽。便送到了根。大動起來。蓮姨假睡不來了。睜眼笑道。活強盜。大白日裡怎跑來奸我。看有人來。還不下去呢。姚澤民摟住笑道。強盜到人家。可有肯空回去的。人來不過是榴姐。我正要搗他的花心呢。遂大抽大弄起來。蓮姨初嘗滋味。覺比那老兒大不相同。聳身上迎。姚澤民一面弄着。一面端詳。眞好一個美婦。有幾句贊他道。

\begin{quotation}

臉如蓮蕊。粉濃濃兩朶蓮腮。體透蓮香。撲香馨一身蓮氣。牝似紅蓮微綻。直搗得蓮瓣大張。足如金蓮高舉。眞像那蓮花挺立。揮(渾)身顫顫。猶蓮葉翻風。嫩牝津津。若蓮房滴露。不愧稱做蓮姨。眞堪居住蓮室。

\end{quotation}

二人風流一度。姚澤民歇了一歇。又復大戰。原來榴姐見他兩個弄時。已在床側張聽。隔着紗帳。看得明明白白。又聽得蓮姨那些聲息。他忍不得了。走來掀開帳子。笑道。姐姐的蓮花心這一會好被你揉碎了。也該略歇歇。你兩個不要太享過了福。蓮姨笑道。讓你也來享享。把榴花心也叫他揉一揉。姚澤民把他一把抱上床。掀開紗裙。見他不曾穿褲。扶起他兩腿看時。鮮紅的一朶花心。眞像紅榴的一個骨朶。就弄將起來。一面抽着。方替他脫得精光。一身白肉。軟嫩得可愛。姚澤民低頭看那出進。他的此竅比蓮姨又緊小些。那一朶花心被陽物帶得吐出吞入。翻覆有趣。鼓起勇來一陣亂搗。弄得他嬌聲婉轉。星眼矇矓。多時方纔住手。姚澤民又同蓮姨來弄。他正看得興致大濃。聳身疊股。竭立(力)迎送。榴姐坐起。彎倒腰。低頭笑着看那出入之勢。蓮姨道。妹子你好死相。你幫他在脊梁後邊推推。他也好用力。你看的有甚麼趣。榴姐笑道。你圖受用。熱巴巴的。我不怕費力麼。蓮姨道。你這人好毒。你費力推了。少刻我也幫你。榴姐笑着到姚澤民背後。雙手抱着屁股。替他推送了一會。又放倒榴姐。蓮姨推着。兩人弄了一陣。都洩了。三人摟抱着親嘴咂舌。摸乳撫陰。頑戲了一會。蓮姨說道。你我有緣。今日相遇。後來却要情長。無事我叫碧梧來約你。你此時去罷。恐有人來。那姚澤民還依依不捨。攥攥這個的乳。捏捏那個的陰。方纔穿衣起身出來了。他二人乍經爽活。渾身通泰。一同小憩。姚澤民到堂屋內。只見碧梧翠竹攔住笑道。我兩個替你做了媒。看你拿甚麼謝我們。碧梧道。我先領過你的情還罷了。指翠竹道。這是新稅官。要上鈔的。你如何越得過去。那翠竹嘻嘻的笑。姚澤民知他們是索謝的意思。後來用他處多。不敢薄待。況他在此道中也還是無厭足的。一手拉了一個。笑道。別的謝儀你們也不稀罕。我有一個金剛鑽送你們去去。我替他把竹子節通一通。梧桐上鑽個眼去。三人同到西屋床上。着着實實每人分惠了一下。把竹瀝同梧洞淚都弄出許多來\footnote{二語巧而趣。}。纔出去了。他五人時常相聚。蓮姨的兩片菡萏已弄成了一朶大開蓮花。榴姐一朶半吐嬌姿也揉成一枝翻花石榴。不用細說。一日。姚澤民正同蓮榴二人在房中高興。碧梧翠竹聽了一會。那梧桐瓢中。竹子節內的水。不住滴將起來。心中難過。他互相摳挖了一會。慾火更炎。悄悄商議兩句。同到西間床上。脫了褲子。兩件光撻撻的東西一上一下的搧打。笑個不住。不想臘姨走了來。要向蓮榴二人說話。忽聽得房中笑聲。向窗內一張。見兩個丫頭磨鏡子呢。笑着想道。他主人那裡去了。兩個丫頭這樣騷發。走到堂屋。不見有人。疑他二人睡覺。悄悄走到房中。覺得床上聲息甚異。也當是他二人做那事。笑嘻嘻雙手把帳子一掀。見蓮姨兩足直豎。姚澤民俯身下搗。榴姐在他背後推呢。抽身就往外走。他三人吃了一驚。蓮姨把嘴一努。姚澤民會意。赤身跳下床來趕上。一把抱了進來。說道。好姨娘。千萬不要聲張。那蓮榴二人赤着身子。也下床來央道。我們好姊妹一場。好姐姐千萬隱瞞着些。原來臘姨也久有心看上了姚澤民。因無其便。今見了他們如此。雖有醋意。却發不出。正要借此相交起。便道。各家門各家戶。你們是有造化的。相與了知心的人。干我甚事。我聲張的是甚麼。快放了我去。讓你們做正經事。姚澤民抱住不放。道。好姨娘。旣承你美情。我謝謝你再去。臘姨道。我是來雌你的麼。我難道沒有家。要在這裡。姚澤民知他是要到他家裡去的意思。便放了。道。你請先去。我穿了衣服。隨後就來。那臘姨徜徉\endnotemark[4]去了。姚澤民同蓮榴商議道。旣被他撞破。不得不去。若不堵了他的嘴。這事就不好了。他二人知道這一去。又有四人纏住。分去了一半恩愛。心雖不捨。不得不放他去。那姚澤民忙穿了衣服到臘姨處。來到他房中。見他獨自一個坐在椅子上。看見他來也不理。姚澤民笑着。忙上前抱住就親嘴。他推開道。你同你心上人樂去。我是不要的。姚澤民道。我想你久了。不敢來親近。今日有緣。特特的同你來相交。你怎這樣冷臉待我。臘姨道。你這樣的鬼話哄誰。你的兩個眼睛好不識貨。他兩個生得標致。你自然該去親近他。我生得醜。不要你違着心到我這裡來。要不是我今日撞見。你肯捨正眼看我一看麼。姚澤民道。寃屈死人。你比他兩個不還標致些。怎說這話。我雖有心要來。不知你心裡如何。不敢造次。承他兩個見愛。所以就同他先相與了。你不過怪我來遲的意思。此後我來勤些。補上前欠就是了。臘姨雖不做聲。却還不動。姚澤民忙脫光。要抱他上床。他扳住椅子道。我是不的。免勞下顧。姚澤民急得跪在他面前道。我替你吿罪。求你上床去罷。不要躭誤了工夫。那臘姨何嘗不愛。這一番做作。因蓮榴二人占了先去。他洩洩醋氣。二來急他一急。好儘力以補前之不逮。見他光身跪着。那陽物又粗又長。不住亂跳。做作不得了\footnote{諺云。不看僧面看佛面。臘姨則是不看姚澤民之面而看小僧之面。}。笑着立起扶他。道。我依了你。看你後來有良心沒良心。姚澤民把他抱到床上。替他脫光。要圖他歡喜。儘力大弄。不多時。他就丢了兩度。姚澤民還要弄。他道。我夠了。雪姐同在一處。偏不得他。等我叫他來。你也同他弄弄。遂叫。雪妹子你來。那雪姐先姚澤民來時他就見了。躱在窗下張看了那些光景。好不動火。他是山西人。纔十六歲。年小害羞。不好進來就敎。諒着臘姨。不好偏他。耐着心等。聽得叫他。故做不知。走進來道。叫我說甚麼。不曾說完。被姚澤民跳下。一把抱上床來。就扯褲子。他也不動。只是嘻嘻的笑。姚澤民看他桃紅紗褲襠上如雪消春水一般\footnote{語雋。}。濕了好大一塊。脫去了。看他那牝戶。又小又嫩。水淋淋的。心愛得了不得。一下直攮到底。他嬌聲嫩氣的道。哎喲。你擡殺俺了。輕些是呢。姚澤民奮力直搗。他笑着。口中擡殺了擡殺了叫個不住。〔不〕多時。兩下都弄丢了。姚澤民居中。一隻膀子摟着一個睡下。這個嘴上一親。那個腮上一咬。好生得意。臘姨道。有句話問你。不許瞞我。你同他兩個也相好有多少時了。姚澤民道。不過個把月。論實事只得六七次。臘姨道。我不信。姚澤民道。我要哄你。叫我發甚麼誓我就說。臘姨道。他們那四個。你同他可有私帳沒有。姚澤民道。不敢瞞你。我雖有此心。却無此事。臘姨道。你把從前的數替我補足了着。我包你還有兩個到手。姚澤民道。是那兩個。臘姨道。你不要管。不過有兩個到你就是了。雪姐笑道。姐姐不要管他。他這樣伶俐的人。怕他自己不會去尋。稀罕我們總成他呢。姚澤民聽臘姨又說上興來。把雪姐的腮輕輕咬了一下。道。你也會這麼油嘴。翻上身。又弄了一陣。又向臘姨起媒。重重的抽了數百。然後纔歇。臘姨道。你明日來。把水仙天竺兩個丫頭。你也施點恩到他。纔好大家做事。姚澤民滿口喜諾。穿衣出去。次日進來。走到西邊屋裡。不見有人。聽得床後滴滴聲響。忙去一看。原來是天竺坐在淨桶上小解。見他走來。連忙站起。姚澤民一把抱住。也不容他拽褲。抱到床前。扯下他的褲子。天竺動也不動。聲也不嘖。乜斜着眼微笑。姚澤民忙取出肉具。站在床前。扛起腿來大弄。弄得他哼聲不絕。誰知水仙在外邊進來。見他二人高興。悄悄走到姚澤民背後。看他兩個賣解。見那天竺的樣子。不由得笑了一聲。姚澤民一回頭。見了他。撂了天竺。將他推到床上。把褲子扯下。也是一陣蠻舂混搗。弄得那丫頭的淫聲浪態比天竺還難聽。弄了一會。他要留些精神應付臘姨雪姐。便歇了。走過東屋。他二人昨日乍嘗甜頭。正在那裡談內中的趣味。見他走來。笑臉忙迎。不暇他言。即上床解衣。弄了一度。姚澤民把那兩個丫頭的事吿訴他。二人齊贊道。當日只說你少年人。不過比老爺強些。誰知一個抵得七八個。若論起滋味來。你是山珍海錯。他竟是藜羹糲飯了。大家笑了一回。姚澤民來過了數次。一日。問臘姨道。你前日說等我補足了。還把我兩個。我的數也補過了。你不要失信。臘姨笑道。你這吃一看二的饞鬼。我總成了你。你不許戀了新人薄了我。雪姐笑道。據我說。姐姐不要管他的好。姚澤民把他摟着。咬他的脖子。道。你這壞人。專會調舌。我明日不弄癱了你報仇。也不算好漢。因向臘姨道。你不要聽他的話。你若成全了我的好事。我若敢沒良心。不逢好死。臘姨笑道。你今日且把梅根澆透了着。明日包你得會新人。姚澤民興發如狂。把二人拉上床。同脫了衣服。欲得他的歡心。儘平生之力。却也把臘姨奉承了個飽足。然後按着雪姐弄將起來。自首至根。加勁搗有千餘。那雪姐年幼身怯。被他弄得氣都接不上來。掙着說道。爺喲。你你要擡擡殺俺麼。俺來來不得咧。你饒了俺罷。姚澤民笑道。你怕不怕。下次可還敢來多嘴了。他笑着哀吿道。俺再不敢了。你饒了罷。好親爺。姚澤民也興足了。又愛他這嬌態了不得。又狠狠的幾抽。也就洩了。大家散去。原來臘姨同丹姨芍姐親厚得如嫡親姊妹一般。有心腹話彼此無隱。他們時常閒話。說起跟着個老兒。靑春虛度。長吁短嘆。也都想到要借姚澤民的這一點甘露來澆滿腔慾火。兩下相約定。不拘誰先得手。不許相瞞。今被臘姨先得了。婦人中件件都可讓得人。惟獨這一件事。雖同胞姊妹嫁了一個丈夫。有些偏處也是不憤的。臘姨因有前約。不忍瞞他二人。到底先偏了幾次。自己心足了。纔肯分惠。那日。他走到丹芍二人處坐下。丹姨道。今日姐姐滿臉喜氣。有甚麼好事。攜帶妹子也好。臘姨笑道。明日是妹妹華誕。我備了幾色薄禮。明午請你到我那裡去祝壽。有一個禮單在這裡。你請看。袖中取出一個紅單來。笑嘻嘻的遞過。丹姨忙接着一看。原來是一隻掛眞(枝)兒。上寫道。

\begin{quotation}

賢妹妹。我是來與你上壽。禮匪薄。全望你一併都收。有一疋捲心紬。還有兩疋核桃縐。靑棍子魚一隻。眼大蒸〖飠卷〗兒裂破了頭。送進了你的門兒也。外邊廂還倒提着一瓶酒。

\end{quotation}

丹姨看了。笑道。姐姐見賜。妹子敢不拜領。但不知姐姐怎樣得了這件寶貝。臘姨笑着將如何得遇姚澤民。不肯偏他姊妹。故此來約他同去賞鑒這肉骨董。芍姐笑道。姐姐請我家姐姐去上壽。我却不好去相擾的。臘姨笑道。請你去做陪客。你可曾聽見人說麼。也不願人請我。也不願我請人。但願人請人。請我去陪人。做陪客是極便宜的事。三人大笑了一場。次日早飯後。丹姨芍姐老早就來等候。臘姨雪姐陪在房中閒話。無非誇他陽物有多粗多長。怎樣堅久勇猛。向來所嘗老兒之物。如飮村醪白酒。今他之此道。如飮醇釀美醞。令人骨軟筋酥。心魂皆醉。二人聽得心忙意亂。火氣直騰。望這救命王菩薩總不見來。儘着拿淸茶澆那心火。將到午刻。方見姚澤民走來。臘姨笑道。你們新人相會。又向姚澤民道。他姊妹等你久了。你們敍敍罷。姚澤民道。我早要來了。偏生今日有個客來。躭誤到這昝。臘姨笑道。你們請做正務。遂同雪姐走了出去。姚澤民忙扶着二人一同上床。便脫衣服。他兩個也等不得姚澤民替他脫。各自脫了。姚澤民見他二人又是一種丯韻。先將丹姨扛起腿來就弄。弄得他聲兒顫着。身子搖着。如弱柳迎風一般。好不動興。再看那芍姐。兩腮紅暈。兩個眼圈被火攻得通紅。眶內水汪汪的。咬着裙帶格支支的響\footnote{丹芍皆春茂。故春心大盛也。}。知他情動得很了。撇了丹姨。又同他弄起來。那芍姐將嫩股老高的亂疊。雙手儘力下扳。姚澤民見他騷得可憐。也奮力下杵。已將他弄丢了。他還摟住不肯放。那丹姨急得笑道。你陪客還讓讓正主兒是呢。伸手在他陰中將陽物生拉出來。塡入自己牝內。姚澤民見他兩個。算六人中騷極了。也竭力以事。輪流轉弄。自正午將及日西。還不肯歇。臘姨笑着進來道。也該歇歇了。不怕弄塌了皮(床)麼。丹姨也笑道。姐姐旣請客。那裡有個主人催客起來的道理。臘姨笑道。客太爛板凳。也就怪不得主人呢。丹姨笑着。纔放了姚澤民起來。此後他六人倒都同心合意。議定一日輪到一家。週而復始。那姚澤民次日到丹芍二人處來。只見他二個丫頭夭桃紅杏。笑嘻嘻攔住道。不許進去\footnote{此雖與碧梧翠竹小犯。却迥然各別。}。人家各有地界。俗語說。管山吃山。管水吃水。管靑山吃碓嘴。我們這裡又不是你的屬下。許你直來直往。也說過四言八句。纔放你進去呢。姚澤民笑着。一手摟着一個。道。小由(油)嘴。你不過見姨娘姐姐同我相好了。不曾同你們親熱。你就吃醋。我怎肯偏你。此時特來尋你兩個的。三人笑着同到房中。姚澤民笑道。你兩個那一個先弄起。紅杏道。我杏花比桃花先開。自然是我先。夭桃道。我桃子比杏子大多。自然該是我。紅杏道。古人說。桃李春風牆外枝。到不得你。況且說。日邊紅杏倚雲栽。自然該我杏花先栽一栽。夭桃道。天上碧桃和露種。你栽得我也種得。又說。一枝紅杏出牆來。你在牆外隔着。輪不着你先呢。紅杏笑道。我一色杏花紅十里。比你那桃花富麗了多少。應該讓我。夭桃道。九重春色醉仙桃。豈不強似你。還不讓我麼。況詩經說。桃之夭夭。難道你不曾聽見。紅杏道。你的夭字原在底下。詞上說。紅杏枝頭春意鬧。劈頭就是紅杏兩個字。可見先是我的了。二人笑着你爭我奪。姚澤民道。不〈不〉用爭。你兩個都脫光了睡着。我一個一陣的弄。就公平了。兩個丫頭忙笑嘻嘻睡倒。姚澤民笑道。但是人說話就稱桃杏。自然是桃貴似杏。該他占先。紅杏笑道。誰說。人開口便說驢馬呢。難道驢強似馬麼。我偏不讓他。夭桃笑道。急鬼。我就讓你先。只要二爺有個乘除加減。就在裡頭了。姚澤民笑着。就把紅杏弄起。弄得他丢了。然後弄夭桃。足弄了有二分工夫還久方歇。紅杏道。一樣的人。你怎麼偏心兩樣待。姚澤民笑道。一點不偏。你得頭籌。他得後趣。可不是一樣。紅杏道。旣這樣說。下次再弄。我先讓桃姐。我也照樣要多弄一會的。夭桃笑道。你怎麼比得我。人說桃飽杏傷人。桃多些無妨。杏子自然該少些的。大家頑笑了多時。方纔散去。過後姚澤民想道。八人我已得六。那兩個可肯放過他。須得設一網打盡之計方妙。時常在秋院中去走踅。那桂姨菊姐也耳有所聞。知他姊妹皆已得了姚澤民。心中何嘗不急。要屈身俯就。又恐被他看得下賤。要等他來垂靑。又不見他動手動脚。猜測不知何意。疑道。定是我兩個容貌不如他們。所以他不來親近。不然八個人中爲何單棄我兩個。他旣無心到我。我去就他也是無益。一腔醋氣塡塞在內。後來見了姚澤民。由不得怒氣勃勃。那臉上竟像刮得下霜來一般。姚澤民見他顏色正厲。越發連戲話都不敢說。孰不知他色厲而內荏。故此倒日遠日疏了。一日。姚澤民偶然到他那裡來。見紅葉丫頭在一張醉翁椅上睡覺。兩足擱在椅軸上。兩腿大揸。由不得失笑。左右張得沒人。輕輕上前。將他衣裙掀起。自己取出肉具。撲他在身上。一把抱住。將陽物隔着褲子混戳。紅葉驚醒。說道。還不放我起來。姨娘心裡不好。在屋裡睡着呢。看他起來看見。姚澤民那裡聽他。只是亂戳。那丫頭被他戳得春興大發。笑說道。冒失鬼。這隔着褲子也是弄得進去的麼。姚澤民也不暇替他脫褲。雙手將褲襠一斯(撕)。扯了一個大口子。就弄起來。那丫頭摟着他的腰弄了一會。說道。你歇了罷。看菊姐回來撞見。不說你這沒廉恥的來尋我。還當我騷發了尋了你來的呢。正說着。那雞冠丫頭驀地走來看見。笑道。沒廉恥的。大白日裡。你兩個怎就鏈在一塊兒了。姚澤民連忙拔出。摟着雞冠親了個嘴。將他按在一張杌子上爬着。扯下褲子。露出光臀。就後邊弄了進去。不住亂搗。紅葉笑問道。菊姐呢。雞冠顫着聲兒道。菊菊姐還同夫夫人下棋棋呢。我我來家走走。不想遇了這活強盜。拿着我這樣。姚澤民笑道。不要屈着你。你旣不願。我不弄罷。雞冠扭回頭笑道\footnote{扭回頭。妙。是自後弄者}。你好自在話兒。我旣被你強奸了。弄得我不受用。還不饒你呢。兩個笑着弄了好一會。方纔住了。又同紅葉復了一帳。恐菊姐回來。只得歇手。姚澤民悄悄問紅葉道。你姨娘害甚麼病。紅葉道。誰知道。他這兩日茶飯也不大吃。口裡只是嘆氣。夜裡叫我替他做伴。翻來覆去。總不肯睡。熬得我要死。你不見我纔在這裡舂盹麼。姚澤民道。大約是春心發了。想個人弄弄的意思。紅葉道。他雖說不出口。大約此時有個趣人兒。他也未必辭。你何不去替他醫醫病。姚澤民道。我何嘗不想他。他看見了我。那哭喪臉難看。不敢動手。紅葉笑道。你做夢呢。他知道你同那三個姨娘三個姐姐相厚久了。他惱你不來親近他。你若去陪個小心。包你成就好事了\footnote{此處用紅葉提醒他。使二人成就好事者。取御溝紅葉爲媒之意耳。}。姚澤民方纔恍然大悟。他向來因此而怒。雞冠道。你這沒良心的。也怪不得他們惱。我菊姐雖不曾同你有甚私事。他待你的情也算親厚得很了。你有了別人倒撇了他。他恨不得咬你的肉呢。我聽他的口聲。口中雖說恨。心裡還有幾分戀你。你若同姨娘上了手。他自然也是肯的。姚澤民心中暗喜。走進房中。到床前一看。見他面朝裡睡着。就坐在床沿上低低叫道。姨娘。你身上那裡不好。我來問安了。那桂姨明醒着。也不答應。姚澤民伸手去撫摸他身上。又問了一聲。他忽然一個翻身。鼻中冷笑道。你到你那些心坎上的人跟前去罷了。你來問我的是甚麼。空勞了你的心。姚澤民道。我聽得你身上欠安。我心裡急得了不得。忙來問候。一團好意。有甚麼心上人心下人的。他又冷笑道。你當我不知道麼。他們六個都是你心上的人。我兩個你看不上眼。是你心下棄了的。你此時冷鍋裡豆兒炸。來說鬼話當甚麼。我幾次要來拿你們的奸。一來怕帶累你\footnote{一則見其愛心。}。二來姊姊(妹)一場。不好意思。他們雖瞞着我。寧叫他不仁。不可我無義。兩次三番。忍耐住了\footnote{一則顯其賢慧。此婦善說。}。論起來。都是一樣的人。磚兒何厚。瓦兒何薄。就是我生得醜些。也不到怎麼東施嫫母的樣子。你就這樣分得淸。說着。就嗚嗚的哭起來了。姚澤民忙扯衫袖替他拭淚。他把臉又轉了過去。用手推道。你去罷。不稀罕你這虛情假意。姚澤民忙跪在床下叩頭。道。要有一點假心者。就天誅地滅。我巴不得來親近你。因見你見了我那氣狠狠的臉嘴。我不敢放膽。若知你有這好情。我早來陪伴你了。是你自己躭誤了好事。如何反怪我。嘴裡說着。就伸手去扯他的褲子。他忙攥着。道。不要屈着你的心。你還去尋你的情人。姚澤民道。我的娘。我這樣說。你還不信。你若不肯。我今日死在這裡也不去了。一面說着。忙自己脫了褲子。強將雙手去解他褲帶。桂姨還要做作。被姚澤民一下將他身子扳正。就伏上身。將鐵硬的陽物向胯中亂搗。桂姨情動。不能自持。手由不得放鬆了些。被他乘勢脫下。弄了進去。抽扯起來。弄過一度之後。桂姨說道。你這壞人。我今日依了你。你後來定不稀罕我的。姚澤民道。我的娘。你不要講這句話。屈死了人。若論模樣。八個人中算你第一。要說風流。也算你第一。我心愛你久了。我要有一句謊言。促死促滅。桂姨此時方有了個笑臉\footnote{倏哭條(倏)笑。活是婦人性情。}。摟着他道。你果有眞心到我。菊姐不消說是你受用。紅葉雞冠也憑你取樂。我們都是一樣的姊妹。我難道要搶他的先不成。要你一個公平心就罷了。若偏了我。我打聽出來。却也不肯與你干休。姚澤民道。蒙你這樣見愛。我還敢欺你麼。\endnotemark[5]他們六個派定一日一輪。\endnotemark[6]今承你不棄。我若偏向你。怕他們爭講。也是挨此輪流就是了。說着。將他臀兒墊起。兩足挾於肋下。這一場弄。足有千餘。把桂姨弄得四肢癱軟。喘息了一會。笑說道。寃家。你有這樣本事。怪不得人人愛你。我雖來了這幾年。今日纔知這件東西有如此妙處。又笑道。他們姊妹是誰先得起。姚澤民將先後原委細細吿訴他。桂姨笑道。好個穿花蛺蠂(蝶)。衆人的花心都被你採了。二人正在說笑。聽得菊姐回來了。桂姨道。菊妹子你來。我同你說話。那菊姐走到床前。見姚澤民在床上。便道。這樣沒良心的人。姐姐容他來做甚麼。就要走。原來這菊姐更風流更騷浪\footnote{伏後得病。}。當日同姚澤民頑笑。把臂捏腕。摟頸接唇。都是有的。只不曾沾在一處。後來因聞他有了衆婦人。且又見桂姨正帥不能到手。那副將焉能得。就漸漸疏淡。菊姐滿懷醋念。不得發洩。此時心中雖暗喜。但他醋意蓄久了。故有此話。桂姨拉他坐下。道。我方纔也罵他沒良心。他說因這些時你我見了他惱嘟嘟的。不知我們是甚麼意思。故此不敢放膽。是我們自己躭誤了好事。據我說。也怪不得他。原是我們多心自誤。自替姚澤民遊說了一番。纔勸他上床。菊姐也就半推半就。同他做了于飛之樂。自此以後。姚華胄的這八妾八婢。他虛躭其名。姚澤民實受其惠。一日。姚澤民想道。他們衆人都已到手了。料道不怕洩露。但常老婆他係夫人心愛的人。又在老爹跟前傳話。況他素常長舌。若露了風聲。如何了得。須得連他弄上。方纔妥當。每日留心看機會。一日。遠遠見常氏在牡丹臺畔小解。他悄悄走近前。一把抱住。他兩人時常也戲謔慣了的。常氏又是個極淫之物。竟逆來順受了。也就兩人見了一見大意。此後姚澤民方放了心。且擱過一邊。那時四海奠安。萬民樂業。治極生亂。到了天啓三年。四川廣西就有些流賊勾引土苗倡亂。也不曾占據城池。只搶擄些人畜。殺了些老幼是有的。此時若有守城好將官領些兵去。這幾個毛賊也就可以殺跑了。只因承平日久。人不知兵。忽聞得這個信。州縣官驚得手足無措。便輕事重報。某處反了。凶猛異常。這些上司一見此報。生怕就殺到他跟前。功名性命還是小事。若把這些宦囊姬妾搶了去。將來兒子拿甚麼享用。也不查問有多少賊。據了何地方。便慌慌張張上本請兵。說得好不利害。天啓見了本章。也恐地方有失。着九卿科道會議。命將出師。衆人薦舉姚華胄老將知兵。推他去征勦。他此時已七十多歲了。他自己說了幾十年大話。今日如何推老了去不得。倒是天啓恐他年邁受不得這煙瘴地方的苦楚。疑問衆臣。衆臣奏道。昔日之廉頗班超趙充國郭子儀馬援皆係老將。故能成功。況且不遇盤根錯節。無以別利器。姚華胄雖過七旬。矍鑠猶如壯年。必能平賊。天啓遂命他領了兵去。那兩處不過是些小土寇。聞得官兵到來。潛伏的潛伏。逃散的逃散了。兵不血刄。地方已靖。他也竟妄自居功。報說。一到就烽煙盡滅。天啓大喜。大加賞賚。恐兵一撤回。賊又復起。就封他爲鎭西將軍。駐鎭廣西。那姚華胄出兵去後。他這位繼夫人裘氏正在妙齡。嫁了恁個白頭皓鬚。軟如棉。濃如涕的老兒。心中之苦說不出來。每每見了姚澤民。便眼中冒火。想道。我正是他的對子。怎這月下老人錯把紅絲繫在他老子的足上。我一朶嫩蕊嬌花。怎被這枯藤老樹纏着。天公雖然錯配。人力尚可挽回。何不把這兒子設法弄來孝敬我。但有繼母之尊。難以開口。且這老兒日日守着。也無空〖阝少日小〗可乘。沒奈何。只得忍住。無奈那不知趣的老兒還假賣風流。說情說趣。及至引得春心舉發起來。他又一點正事也幹不得。間或就強而後可。軟叮噹的一個物件。又沒處尋這麼個小篾片幫扶他進去。弄得不疼不癢。更覺難過。往往慾火熾將起來。只好把那涼茶冷水往下嚥。靠他靈犀一點來澆息了這火。萬不能夠。倒巴不得離開了他。孤眠獨宿。眼不見爲淨。還略好捱些。死捱了幾年。見他去了。如拔了眼中釘一般。心下倒覺得一爽。無奈那姚澤民每日在眼皮子底下晃來晃去。見了他。心頭小鹿就亂跳。臍下那件作怪的東西不由得一吸一吸的難過。心中暗想。料道熬不過去。遲早總是放不掉他的。不如早一刻以救一刻之急。每每要算計同他比翼鶼鶼。共偕連理。做那風流樂事。一則不得其由。二則難以啓齒。那姚澤民雖有十分慕他的心。他有繼母之尊。比衆妾不同。連戲話也不敢亂說。怎敢輕易亂做。二人雖都有心。却不能覿面相訴。裘氏一日正在兀坐躊躇。忽聽得兩個丫頭拌嘴。一個叫春花。一個名秋月。聽那秋月道。你說我浪。你同二爺調情親嘴。他伸手在你褲襠裡。是我親眼見的。那倒不是浪麼。春花道。你也撇不得淸。也不是甚麼淸淨姑姑兒。我見他那一日沒捏着你的奶頭頑來。你還瞞我。我不說出來罷了。秋月大怒。罵道。沒廉恥的淫婦。他不過捏我的奶頭罷了。把你的褲子脫了。看那騷屄上的卵子印也有幾千了\footnote{奇談。此處何得有印。}。春花被他罵急了。說道。臭淫婦。你替我墊腰來麼。知道得這麼詳細。二人幾乎打起來。裘氏有心細聽。出來喝住了。少刻。叫了春花到屋裡。悄悄詐問他道。你同二爺兩個的事我也聽見久了。你可實說。我倒饒你。若要瞞我。我追問起來。你就了不成。那丫頭也只當夫人果然有些知覺。臉緋紅。跪下道。二爺時常望着我嘻皮笑臉的說笑。我也不理他。那日他強摟着我親嘴。我把臉扭着。他也沒有親着。就拉我的褲子。我把腿夾得緊緊的。他何嘗摸着甚麼來。我要叫喊。又怕羞。只得哄他說。你去着。等我有空再約你來。他纔放了我。不想被秋姐看見。他今日就罵我。我也看見他同二爺頑呢。那日二爺壓在他身上。摸他的奶頭。又親嘴。嘻嘻哈哈的笑。他就不說了。裘氏又道。你當眞不曾同他沾身。春花道。我要敢瞞夫人。若看見二爺的東西是怎樣。就滴出眼珠子來。要同他沾了身。把下身爛個洞。連腸肚都掉了出來。他強抱着我親嘴是有的。那一日夫人不在屋裡。秋姐把夫人的睡鞋偷了給他看。二爺還聞了聞。看見了我。秋姐忙拿過去塞在床上褥子底下。還沒有吿訴夫人呢。裘氏笑了笑。又想了一想。道。我有一件事叫你去做。你若做得來。我重重的擡舉你。饒你起來罷。春花站起。道。任憑夫人叫做甚麼。我還敢不去麼。裘氏笑着附在他耳上道。你去尋着二爺。悄悄對他說。只說你約他日落後叫他到百花樓上成就好事。我假冒了你去。同他試試如何。若成就了我。只有好處到你。你却不可洩露。春花道。這在我。包管成就。去了一會。回來道。約下他了。裘氏滿心暗喜。晚飯後。吩咐衆丫頭。我帶春花到百花樓上去乘涼。你們不必來。衆人誰敢不遵。他到了樓上。有現成床榻。就到床上睡下。叫春花躱開。原來那春花同姚澤民偷弄過不計其次。已將裘氏假冒約他的話向他說了。姚澤民喜出望外。打點一片好精神要來孝敬繼母。巴到日落。潛身到百花樓下。輕輕上樓。到榻上一摸。見一個人睡着。還不知可果是裘氏。尚疑春花哄他。自己脫光上床。就去替他脫褲。裘氏等了一會。將要睡着。被他驚醒。不好做聲。任他脫去。姚澤民伸手將陰戶一摸。又肥又嫩。緊揪揪一條溝。指頂大一點花心。微微幾根毛。與春花大不相同。知是眞了。素常見裘氏的一雙小小金蓮尚不足三寸。也伸手捏了捏。心喜欲狂。縮下身子去。一口含住了花心。咂了一陣。又伸舌頭在他陰門中亂舔。舔得那裘氏春心繚亂。他從未經此。腰肢只是亂扭。嫩股往上一擡一擡的就。姚澤民興致大豪。爬上身。大弄起來。一氣抽了數百。裘氏樂所未樂。不好出聲。只將身子亂扭亂迎。姚澤民道。心肝。我同你弄過多次。你今日這樣高興有趣。幾日不見。身子滑嫩了好些。屄又肥緊了好些。脚也小了些。風流也添了些。渾身又香了好些。我想夫人也不過如此。難道還有好似你的。我料道也沒福享用夫人的美物。今日同你弄。就把你當做夫人罷。心肝。你怎麼不做聲。遂將舌頭伸入他口中。又叫他伸過來。裘氏也只得伸出舌。被他含住。咂了一會。又自首至根的亂搗。裘氏先只說春花不曾同他沾身。故假冒了他來。今聽見說弄過多次。自然知道不是本人了。此時弄過一會。不覺羞了。且被他弄得忍聲不住。笑說道。短命的。不要拿着精神使胡塗。你明知是我。鬼張的是甚麼。姚澤民也笑道。原來果然是夫人。我說別人那裡有這樣好東西。我那親親的娘。兒子得罪你了。說着。越弄越利害。那裘氏口中心肝親哥無般不叫出來。姚澤民覺他比那八個妾還騷浪些。兩人足弄到將二鼓方住。裘氏心中快樂無比。緊緊的相摟着。喘息了一會。問道。我聽得傳話。說那八個妖精都纏着你。可是眞麼。姚澤民道。怎敢瞞你老人家。是眞有的。裘氏笑道。你好本事。把你的力量勻些與你爹也好。今日的事。料道也瞞不得他們。你對他們說。我們也不論甚麼大小了。只要同心合意守着你過日子罷。姚澤民道。承你這樣厚恩。誰敢不尊讓你三分。裘氏又笑道。春花你也同他弄過麼。他日裡望着我賭誓發願說沒有。姚澤民道。這一家我只除了你一位不敢。你的兩個美婢被我都弄鋊了。裘氏道。倒便宜了這兩個小淫婦。他是有造化的。早相與了你。比我還強。姚澤民見他相愛甚切。又遍身撫摩了這一會。體骨(滑)如脂。光油油如鏡面一般。頭髮嘴唇面上無一處不香得沁腦。興又大動。又儘力弄了一回。相摟相抱。貼胸交股。睡到天明。又找(戰)了一陣。此時姚澤民見他那種嬌容。遍身如玉。愛得如異寶一般。親了幾十個嘴。方纔穿衣而散。這日。那八個妾都知道了。來替裘氏道喜。彼此不言。惟相顧而笑。晚間衆人備了酒果。同到百花樓上。請裘氏同姚澤民正中並坐。衆人羅圈坐下。都歡喜笑語。飮得半酣。各辭而去。他二人點着大燭。如同白晝。整狂了半夜。比昨夜黑地相親。更覺豪興。此後定了個例。裘氏獨得二夜。那八妾各得一夜。十日一輪。他待衆妾親厚得了不得。衆人感他的情。輪着的這一夜。或去請他來分惠。他不推辭。也竟來領情。這姚澤民魂迷在羣芳之中。他自己房中輕易反不一到。他那妻子桂氏生性已是妖淫。又見了丈夫的這些舉動。可有個不弄出笑話來的。再者大人家這些婦人女子壞事。多由於丫鬟僕婦。這種人可知甚麼羞恥節義。只圖得主母的歡心。做牽頭。做馬泊六。傳消遞息。引奸入馬。遂成了他淫汚之行。然亦起於主人公之罪。若主人公是個正人君子。妻子得了他刑于之化。自然端方貞靜。那些丫頭僕婦可敢去引誘他。只因姚澤民是個淫物。那桂氏也自然被他化成好淫的了。這素馨香兒是他自幼就淫起。那得不淫。靑梅綠蕚也都是被他淫過的。但這桂氏雖有一肚子淫興。他到底是宦門之女。況且年幼。又從不曾嘗過偷漢的滋味\footnote{這滋味自然是甜。}。未經破臉。還惜羞恥。這三個丫頭雖被主人用過。且主人也不過一時間偶然點綴。未嘗日日如此。雖知道這是一件美味。却還未曾十分經歷個中的妙處。且終日伴着主母。即有欲淫之心。也無縱淫之膽。只這素馨同主人弄了多年。深知其中奧妙。今主人一旦別戀新知。將他撇下。若像那三個丫頭獨守孤幃。倒還捱了過去。又每夜同着空負虛名的丈夫共臥。可有食放在嘴邊肯不去吃。及至吃時。如一個極饞的人有一塊肉。只許他咂咂香味。不容他大啖。自然引得越饞起來。他常被吳實弄得毫無樂趣。更覺難過。眞急得要死。每每要尋個救急的人。恐捨了身子。還尋了像自己男人一類的。豈不是糟鼻子不吃酒。虛擔其名了。又不好問人。你的陽物可大。這句話如何出口。一日。該他陰物行運。桃花星進宮。他在桂氏房中下來。要回家去。剛走到大廳後邊。低着頭。心中正然思想甚麼。忽見一個人在那裡溺尿。他是留心的。忙向他腰間一看\footnote{不看人。先看腰間。寫盡騷淫之婦。}。見一個硬幫幫陽物。比姚澤民的還粗長些。又驚又喜。急擡頭看時。原來是姚予民的大兒子姚步武。比姚澤民倒還長三歲。他父親雖愚愚蠢蠢。他却尖尖酸酸。古怪好色貪淫。有乃叔之風。素馨見了他這奇具。望着他。笑嘻嘻的笑着走。姚步武見這光景。知他有羨慕之意。忙攆上去。摟着脖子就親嘴。素聲(馨)也不嘖聲。笑着斜瞅了兩眼。推開他的手。往家中去了。姚步武隨屁股後跟了來到他房中。一把抱住。按在床上。就去扯褲子。素馨也不推辭。只道。哎呀。你怎麼硬開弓。這怎麼行得。撞了我家的男人來呢。說着。已被他扯下。看見了妙物。取出肉具。狠狠一頂。進去了半截。他道。你慢些是呢。冒冒失失塞我這麼一下子。這裡行不得。大師傅今日回去了。我們到佛堂裡去。那裡沒人。你先去。我就來。姚步武也就依他。又親了兩個嘴。還狠狠的抽了幾下。先去了。素馨拽上褲子。腰裡塞了一塊布。鎖了門。來到佛堂門外。四顧沒人。兩三步扠進去。就把門拴上。走進來。姚步武忙脫褲子。那素馨也將褲褪去。就仰臥在禪床上。姚步武伏上身。就往裡頂。兩三下送到根。抽弄了有兩頓飯時後(候)。素馨也丢了有兩三次。姚步武也洩了。素馨掏出那塊布。兩人都揩淨了。各自穿好褲子。姚步武摟着他道。承你相愛。成就了這件好事。我還有一件事托你。你要替我做成了。我打幾件首飾謝你。素馨道。我不要那東西。我男人見了問起來怎麼答應他。你倒是有錢給我些買嘴吃倒使得。姚步武道。這越發容易。在我\footnote{此婦竟上下嘴皆好吃。}。我就送來。素馨道。你托我做甚事。姚步武笑道。我見二奶奶生得可愛得很。我心動久了。不得個門路。你是他貼心的人。替我想個法兒。我若弄上了。定然重謝你。素馨笑道。饞癆鬼。你旣偷上了我。又去偷他。你若同他偷上了。還稀罕我麼。我不管這事情。姚步武親了他個嘴。道。好心肝。你要替我謀成了。你就是我的恩人。敢忘你麼。我不過想嘗嘗他的是甚麼味兒。事成後。我每日空閒就偷工夫來儘力同你弄\footnote{後之事竟成者。得力在此一句。}。報你的情。但得同他弄一次。定然同你弄兩次。你道好麼。素馨喜諾了\footnote{昔人有一妻一妾。在妻處睡兩夜。妾處一夜。其妻成日爭論。人勸道。你處兩夜。他處一夜。也算公平了。妻道。我那兩夜是怎樣的兩夜。他那一夜是怎樣的一夜。素馨尚未聞此哪。一笑。}。又道。這事要看機緣。是急不得的。二人先後出來散去。你道佛堂中供養的這大師傅是個好人麼。這和尚鋪眉善眼。裝出那活佛的樣子。却實在是佛口蛇心。酒肉齊行。男女並尚的惡物。他在報國寺私藏着兩個婦人。還有七八個標致徒弟。時常取樂。他心戀着徒弟婦人。往往回去幾日。又來姚家住幾日。他貪圖姚華胄一年四季衣服銀錢糧米。只得常來。但在他家吃的是蔬。夜間又無人陪伴。捱得兩日。回去樂一番又來。兩下裡走動。再說素馨被姚步武弄了一度。向來積火一旦消釋。好生爽快。剛到房中。不多一會。姚步武拎了兩吊大錢來送他。道。你留着用。用完了對我說。我再送來。他歡歡喜喜接過藏了。姚步武又諄諄托他前事。他滿口應允。姚步武去了。他受了姚步武之托。想成了此事。圖他後來錢與弄兩樁謝儀。忽然想出一計。道。須引動了他的春心纔可下手。這日晚間。素馨上來。在西間屋裡同香兒靑梅在一床睡。綠蕚在桂氏房中上夜。三人睡了一會。香兒笑問他道。你家中放着有伴兒不去受用。二爺又不在這裡。你來同我們受這孤悽做甚麼。素馨道。我可憐見你們這些時熬狠了。我來同你們樂樂。消消你們的火氣。香兒笑道。你的同我的一個樣兒。你還要人替你消火呢。怎麼替我們消法。素馨道。我自然有個道理。就伸手去摸香兒的陰戶。拿個指頭伸進去替他摳挖。香兒笑道。這個消法我自己會。不勞你敎。摳得有甚麼趣。素馨拿出手來。道。有。做個有趣的你試試。遂爬起身。將他屁股墊高。上他身來。牝蓋對牝蓋一陣撞。撞得瓜答瓜答的響。又合着一陣亂揉。揉得那香兒淫心如醉。嘻嘻的笑道。不好了。我的裡頭難過。你下來罷。素馨那裡聽他。揉了多一會。香兒情急得很了。一把摟緊他。乖乖親親的亂叫。也就渾身一麻。陰中流出許多淸水。儘着笑個不住。素馨又爬到靑梅身上。靑梅看了香兒的樣子。急得淫水直流。見他上身。兩足高蹺。抱住了他。親哥心肝叫得震耳。他三人嘻嘻哈哈頑到三更方歇。且說那桂氏一覺睡醒。忽聽得西屋裡嘻笑之聲。側耳靜聽。只聽得說笑。又聽不出說甚麼。心中疑道。這丫頭們有何樂處。這般歡喜。猜測不出。次早起來。衆人都在房中伺候。桂氏問道。你們昨夜做甚麼來。笑一陣說一陣。吵得我半夜睡不着。香兒靑梅都望着素馨笑。素馨也笑。桂氏道。問你們話不答應。呲着牙笑甚麼。香兒指着素馨道。是他做的事。奶奶只問他。桂氏問素馨道。你做甚麼來。素馨正要引誘他。就笑着說道。我昨晚同他兩個睡着。他們久不見二爺的那東西了。心裡火發得很。求我替他們殺殺火。他們受用得很了。所以歡喜得那樣笑。桂氏問香兒道。他怎麼樣的來。你就這樣受用。香兒道。奶奶理他嚼蛆。他壓在我身上。拿他的對着我的一陣混揉。揉得好不難過呢。有甚麼受用。素馨道。沒良心的。要不受用。你怎抱着我心肝哥哥的叫。你若沒有快活。你就賭個咒。桂氏笑道。你當眞快活麼。香兒道。那是被他揉得心裡火起。情急了。也就渾身麻一下。是有的。桂氏又問靑梅道。你呢。靑梅谷都着嘴道。他兩個騷得很了。輪流着一個按着我。一個揉我。也沒有受用。也沒甚快活。揉了半夜。蓋子都揉腫了。這會兒還疼呢。香兒笑道。你沒有快活。你屁股底下那褥子上濕了有冰盤大的一塊。那水是那裡來的。大家笑說了一會。桂氏一個二十多歲的少婦。當日同姚澤民沒有一夜不弄。如今成幾個月纔弄得一次。已情極難堪。但說不出口。今聽了這一番話。那裡還忍得住。到了下午。丫頭們都不在跟前。只素馨在傍。桂氏低聲笑向他道。今晚你到我房中來上夜。素馨知他是要試試的意思了。心中暗喜。偷空去約了姚步武。到晚間。桂氏叫三個丫頭都在西屋去。素馨抱了鋪蓋來春凳上鋪了。伏侍桂氏上了床。他吹了燈。又道。我去看看院子門關好了沒有\footnote{譎智可畏。}。出去暗暗將姚步武帶進房中。在他鋪上睡着。他剛把衣服脫完。聽得桂氏道。素馨你來。他忙走到前。彎腰悄問道。奶奶說甚麼。桂氏笑着道。你昨夜同他們怎麼弄來。素馨趁着話頭。便爬上床來。道。我來同奶奶頑頑。遂去摸他。已脫得上下無絲。素馨就伏在他身上。對着揉起來。揉了多時。揉得他心如火燒。淫水直流。嘴裡哼聲不絕。知他難過得很了。說道。奶奶不要動。我撒脬尿來。包你弄個如意的。遂下床來。拉着姚步武。推他上床。姚步武一翻身。上了肚子。摸着水漓漓的陰門。將鐵硬的陽物一送到根。大抽起來。桂氏正然難過。等他來揉。不想一個又粗又長的東西送了進去。又驚又喜\footnote{大約喜多而驚少。}。急用手一摸。竟是個男人。忙問道。你是誰。他也不答應。只是亂搗。不幾十下。桂氏就丢了。那人摟着加力。又是一場混戰。桂氏又丢了一次。那人略慢了些。桂氏透過氣來。道。素馨。他是誰。聽得素馨在床前道。這是大爺的大相公。他常常求我要來孝敬奶奶。我見奶奶獨自冷冷淸淸的。故此帶他來替奶奶做伴\footnote{雖與祁辛道(通)葵花是一個套子。却兩人說話行事。無一句相重。犯而不犯。眞寫得好。}。桂氏已被他弄了。却又弄得甚好。也無可說\footnote{到了此時。就弄得不好。也沒得說了。}。姚步武見他不言語。知他心服意貼。重鼓威風。又弄了多時。兩下都洩了\footnote{姚澤民此時不知在裘氏處。是在衆妾處。}。姚步武道。多蒙嬸嬸的恩。我此後常常來服事。但我不能過夜。掌燈後來。一更多天要回去的。我同我爺對門住着。恐一時查問。我且去罷。桂氏初次破戒。還有些羞意。也不答應。素馨送他出去。關門。回來睡下。桂氏得了這番快樂。一覺睡到次日飯時纔起來。望着素馨。不住的笑。姚步武乍嘗甜頭。次夜又來承應。點燈大幹。二人熟滑了。方說說笑笑。親嘴咂舌的頑耍。有幾句說他叔姪二人道。

\begin{quotation}

那叔叔抱着繼母。百種歡情。這姪兒摟定嬸娘。千般恩愛。那繼母獎兒子。強如你爹爹數倍。這嬸娘誇姪兒。勝似你叔叔多端。那叔叔叫了繼母幾千聲寶貝心肝。這姪兒呼了嬸娘數百遍乖乖親骨。雖是他家門不幸。却也是天道循環。

\end{quotation}

倏忽月餘。一日。桂氏午睡醒來。聽得西屋裡笑聲。悄悄走到窗下一張。見姚步武精光着同香兒在椅子上大弄。素馨靑梅綠蕚都一絲不着。只見素馨伸手將姚步武的陽物攥住。不容他抽。笑着說道。你兩個肏搗了這一會。也該讓讓我了。又見靑梅將素馨攔腰抱着。綠蕚〖扌扉〗他的手。笑道。你太不知足。你那一日不同他弄一兩回。我們這個把月纔同他弄了三四下。還該讓我們三個。那素馨又不肯放手。香兒急得叫道。妹子。你兩個把那老沒廉恥的拉開。我再弄幾下讓你們。你爭我奪。笑成一團\footnote{一幅出奇的春宮。能手未必描得出。}。頑成一塊。桂氏看得興致大發。走進來。推開門入去。衆人正在爭奪。見了他。連忙放手。跑到床後去穿衣服。香兒推開姚步武。也跑向床後去了。姚步武正在高興。見他來打散。上前一把抱住。到他房中。寬衣解帶。也在椅子上扛起雙足。一場好弄。他們四個也來在窗外張看。見桂氏眼兒乜斜着。嘴兒裡哼喞着。股兒攧着。腰兒扭着。風騷異常。香兒悄悄問素馨道。我們弄着。可是這麼個樣子。素馨笑道。他比你們略斯文好看些。綠蕚道。看得不好過。我們去罷。遂都走開了。他二人足弄到午後。方纔罷戰。過了些時。這桂氏忽又換了心腸。這是何故。自來人心苦不知足。得一望二。得命思財。個個皆然。桂氏前日苦熬的時候。常想怎得一個此道。把這心火洩一洩。就算萬幸了。初得姚步武時。他也心滿意足。以爲奇遇。不想弄過多次。忽又發了侈心。想道。這件事必定兩人終夜同床共枕。談談風情說笑話。說到高興時弄上一下。乏了摟抱着睡一會。興動再弄。纔有趣味。姚步武材雖可取。但急急忙忙應差一般。弄下就要去。及至睡到半夜醒來。還是自家一個。更覺悽惶。有何妙境。怎得個人長遠守着。方得趁心。當日不曾嘗過偷漢滋味。臉嫩怕羞。今日同姪兒弄着。也竟像夫婦一般。羞在那裡。管他甚麼人。只要知他有大物事的。就同他行起樂來。且快活一夜是一夜。生人上身。閉着眼睛。羞過那一會兒就罷了。怕甚麼。他做如此想。就有個機緣來湊他。也因姚澤民烝繼母。淫父妾。惡貫滿淫(盈)。人鬼暗中自然成他妻子的淫行。以爲報應\footnote{此等處皆是借淫說法。}。一日。桂氏叫素馨道。我的枕頭舊了。你到馬房裡去撮些草來塡一個新枕頭。素馨拿了簸箕去了。一會笑着跌跌滾滾跑了來。桂氏見他草也不曾拿得。面紅頭赤。氣喘吁吁的。驚問道。你去拿草。怎麼這個樣子跑回來。他笑道。不要說起。我到了馬房門口。見門關着。一推開進去。不防盛旺那砍頭的。脫得精光。蹲在那裡捉蝨子。一個㞠子多粗多大。一個大疙瘩頭子拖在地下。嚇得我好跑。幾乎跌了一交。這會心口還跳呢。桂氏笑着瞅他一眼。道。你就浪得沒影兒。你還是沒有見過這東西的麼。任憑怎沒(麼)大。就是黃花女兒見了。也不犯着嚇得這樣的。素馨道。哎呀。奶奶沒有見大長的拖着。好不硶看。比二爺的不用說。比大相公的還長着有寸把呢。奶奶若見了他。也要嚇一跳。桂氏動了心。笑道。呆老婆。你要怕。不要看他。好容易遇見這樣東西。你再可同他試一試。你閉着眼睛。叫他塞在你那裡頭去。管情就不怕了。你放了膽子。只管去。素馨笑道。罷罷。奶奶總成別人罷。我不敢惹他。留着我的腸子罷。要一下頂斷了。纔是造化。低笑了一會去了。桂氏心中笑道。我經過他叔姪兩個。粗長都差不多。怎這老婆說得如此長大怕人。我想我們的這件傢伙如口袋一般。多也裝得。少也裝得。男人的東西自然是越大越好。若得把他弄進來。就可以通宵行樂。但只是家奴。不好意思的。低頭暗想了一會。忽然啐了一聲道\footnote{忠臣事仇。節婦失身。皆壞在此一想。這一聲啐了。}。男人沒良心。戀着後娘庶母棄了我。我怕的是甚麼。也落得快活。且叫了他來。弄得。是造化。若太大弄不得。再做道理。素馨膽小沒用。等我哄了香兒去擋個頭陣。遂叫了香兒到屋裡來。悄悄的道。我叫你去做一件便宜事。做成了。後來有得快活。你到馬房裡去取草來塡枕頭。要看見盛旺。若沒人在跟前。你悄悄對他說。一更天人靜後叫他來。不可誤了。你把床底下的錢拿一百與他。叫他洗個澡。他夜間來時。你去門口等着。接他進來。若大相公在我屋裡。你領他在西屋裡等着。我有話說。他此時若要同你弄。你就試試他的本事如何。來回我話。那香兒領了這個美差。眉花眼笑。拿着簸箕。袖着錢去了。到了馬房。那盛旺捉完了蝨子。正在床上〖扌歪〗着。見了他。起來笑道。姐姐來要甚麼。香兒道。二奶奶叫我來取草塡枕頭。盛旺忙接過簸箕。撮滿了草。道。請拿去。香兒取出那錢與他。道。這是二奶奶賞你的。盛旺驚道。草是老爺府中餵馬的。來取草。爲甚麼賞我錢。香兒笑道。有天大的一場好事。我對你說了。你怎麼謝我。盛旺道。我一個大窮漢。有甚麼謝你的。果有好事總成我。我替你叩頭罷了。香兒笑道。誰稀罕你叩頭。拿耳朶來。我對你說。因悄悄向他道。二奶奶賞你這錢。叫你去洗個澡。洗得淨淨的。晚上一更人靜後叫你進去。我出來接你。有大好處到你。看你怎麼謝我。那盛旺聽了。眞夢想不到。心花俱開。一把摟着他。笑道。你不稀罕我大頭叩謝。我拿小頭謝你罷\footnote{有此足矣。尚何他望。}。除此。再沒有別的了。就拉他上床。香兒也不推辭。但道。恐怕有人來。盛旺道。都放馬去了。到晚上纔回來呢。只我一個。再沒人來的。忙關上門。替他脫褲子。香兒道。怕奶奶等我回信。只褪下一條褲腿來罷。盛旺依他。褪下一條褲腿。一眼看見好個滾圓的肥牝。他二十四五歲的小夥子。那陽物不覺挺硬直豎。又粗又長。香兒看見。道。哎呀。你這樣個大東西。如何來得。就要爬起來。盛旺忙按住。道。不要怕。包你沒事。香兒此時又怕又愛。只說道。你留心些。看仔細。我的腸肚要緊。揸開腿。閉着眼。聽他所爲。盛旺雖急。也不敢冒失。將龜頭在陰戶門口左晃右晃。引得有些水出來了。然後慢慢弄了進去。〔往〕裡〈送〉一送。香兒哎呀一聲。盛旺抽拽了十數下。他哎了十數聲。也就毫無餘剩。香兒覺得內中脹滿。有樂無苦。用手摸了摸。已到了根。方纔放心。盛旺見他安然無事。放心一陣亂扯。他久不會此物。只幾十下就洩了。那香兒初逢巨物。工夫雖不長。也被他弄丢了。他坐起。一面穿褲子。向盛旺道。你的這東西雖然長大。只是太快些。恐怕不中奶奶的意。盛旺道。不瞞你說。我又沒有家小。遇着外頭有來扒馬糞的老婆。纔撈着弄一下子。不然。是成年家不見屄面的。熬久了。故此完得快。要時常弄弄。我也還有一更天的本事。你到晚上看。就不是這樣快了。香兒拿着草上來。桂氏見他頭髮散亂。滿面笑容。知他嘗了美味來了。笑問道。比你爺同大相公如何。香兒笑道。大是粗大好些。只是快得很。我問他。他說是熱(熬)久了。若時常弄。也還有更把天的手段。他叫謝奶奶賞。晚上定來服事。桂氏笑問道。果然大得硶看麼。弄進去怎麼樣。香兒笑道。看是果然不好看。及至弄上。也就罷了。桂氏心中暗喜。不住出來看那日色。巴到掌燈。方上床脫衣。恰恰的姚步武走來。推辭不得。只得同他弄了一陣。身在此而心在彼。將及更盡。姚步武方纔去了。只見香兒來說道。盛旺來了好一會了。在那屋裡呢。桂氏道。點着燈不好意思。你吹了燈帶了他來。原來盛旺在那〈裡〉屋裡同靑梅綠蕚香兒更番大弄。香兒來叫他。也不穿衣服。赤身抱着衣服跟了來。走到床前。香兒道。你們去罷。他把衣服遞與香兒。爬上床。掀開被。摸着了桂氏。赤身仰臥。他就爬上身。說道。蒙奶奶天恩。小的來服事了。桂氏不好答應。他摸着此竅濕漉漉的。捏着陽物送進門。有那姚步武的餘精在內。滑溜至極。只兩下便送到根。桂氏覺得內中極深處頂着。甚是有趣。他再抽將起來。一下一下搗着。更覺快樂。那盛旺活了二十多歲。不過同那些扒馬糞的粗醜婆娘在那草堆上行樂而已。何嘗經過這番境界。今在牙床錦被之中。摟着這嬌滴滴香噴噴的美人。那興致加增百倍\footnote{有次(此)數句。愈襯桂氏之不堪也。}。那裡輕易得洩。桂氏先聽得香兒說他甚快。猶恐中止。一時掃興。不想他一口氣就抽了千餘。弄得心越魂飛。丢了數次。眞從來未歷之樂境。渾身都酥軟了。摟着脖子。嬌聲道。你好本事。我來不得了。你歇歇着。盛旺也就歇住。有幾句笑話道。

\begin{quotation}

陽物粗雄。儼是個㔍刀把。陰毛硬勁。好似稻草鬚。周朝嬴非子。牧馬蕃息。得膺天子榮封。姚宅盛後槽。豢馬有功。竟蒙主母寵渥。王良當年。只能車上駕御。盛旺今日。更善被中聘馳。直弄得桂小姐。飄蕩了意馬心猿。低囑那盛圉人。暫時且停韁駐馬。

\end{quotation}

桂氏叫他下來。在新枕上同臥\footnote{閱之偶憶一故事。明崇禎周后之父周奎。賤時爲泥水官匠人。奉差建一府第。不勝辛苦。嘆道。我們費盡辛勤。不知便宜甚麼人住。後崇禎登位。冊立周后。奎后父。即以此第賜居之。盛旺費力切草時。焉能想到此時共枕也。}。說道。我的身子付了你。此後我但叫香兒來叫你。你就來。我自然暗暗的照看你。盛旺道。蒙奶奶這樣恩典。小的殺身也感報不盡。只有儘力服事。盡小的的窮孝敬罷。桂氏着實愛他。一夜弄了數次。五鼓時纔叫他去了。後來隔二三夜定叫他來一回。也常賞他些銀錢。過了數日。素馨知道了。又見香兒三個滿臉喜容。又帶嬌媚之色。他想。桂氏都弄過。安然無恙。方知此物以大爲妙。不足爲懼的。深悔前日之誤。他走到馬房。向盛旺道。當日原是我看見了你的。對奶奶誇獎。纔有這番奇遇。我是你開手的功臣。你倒不謝我一謝。盛旺也是樂得的事。儘力把他謝了一場。他留心打聽。但是香兒去約盛旺。他就上來上夜。以沐餘波。桂氏笑問他道。你如今怎麼不怕了。他笑道。誰知這東西看着可怕。弄着是不怕的。自今放了膽。此後就見驢大的㞠子。我也不怕了。桂氏大笑一會。桂氏一夜同盛旺弄過一度之後。兩人睡着說話。桂氏捏着他的陽物。笑說道。這東西可還有大似他的。盛旺道。別人的我倒也不留心。惟有大師傅。他常到馬房裡去出恭。我冷眼瞧見。他長雖比我有限。他軟着比我硬的時候還粗。大約硬起來像驢子的粗是有的。桂氏聽在心裡。次日偶然想道。盛旺先幾回弄得很受用了。弄過多次。不過如此而已。也就莫(沒)甚趣。再粗大些。自然又有一種妙處。這和尚我家成年這樣白供養他。拿他來當當差也不爲過\footnote{人家供養和尚。想就是要如此當差。}。想了一會。道。香兒嫩。這事做不來。除非激了素馨去。他是騷浪極了的。須得如此如此。任他甚麼眞僧。不怕他不破了戒行。叫了素馨到跟前。說道。我又有一件事叫你去做。你難道連香兒都趕不上麼。素馨道。奶奶就說得我連他都不如。還好呢。眞是老娘不如外孫。蘿蔔不如菜根了。桂氏笑道。前日叫你去你就怕。倒是他做了來。素馨道。那是我先嚇了一跳。故此膽怯。我如今不怕了。桂氏笑道。盛旺說大師傅的那東西比他分外粗大。我想要弄他來見見。你依着我這樣這樣去行\footnote{這樣這樣。即後素馨之那樣那樣也。}。定然成就。你若不放老辣些。事尚不妥。你拿褲子套了臉來見我。素馨也笑道。我去我去。若不把禿驢牽了來。我同他把命拚了。且說那萬緣和尚。他一個月中有十日在姚家來住。這日晚飯後。燈下獨坐。正帶了一本燈草和尚的小說來看\footnote{這正是和尚看的小說。}。看得慾火如焚。陽物脹得生疼。馬口中不住流涎。正無可奈何。忽聽叩門聲響。走到(去)開門。黑影裡只見一個婦人。一手捧着個盒子。一手拿着一把酒壺。走進來說道。大師傅把門關了來。那萬緣不知是甚事。把門閂了。同到屋裡內。燈下看時。認得是素馨。說道。大嫂你此時來何幹。拿的是甚麼。素馨把酒壺放下。將蓋子揭開。絕精緻約(的)幾種葷碟。說道。二奶奶說大師傅在這裡自己靜坐。叫我送這些酒肴來與大師傅消夜。那萬緣盤膝趺坐。說道。阿彌陀佛。貧僧佛家弟子。從來不動五葷三厭的。快快拿去。不要汚穢了佛堂\footnote{果是眞僧決不做作。善做作者決非眞僧。}。素馨一屁股就坐在他傍邊。對着他的臉。笑道。師傅你哄誰。那個和尚不吃酒肉。不鑽狗洞。二奶奶好情送來。你多寡領他些。遂斟了一杯酒。送到他嘴跟前。那萬緣聞得香氣撲鼻。不覺口角流誕。勉強忍住。推辭道。菩薩。僧家第一戒的是酒。貧僧不敢領受。雖有那吃酒肉鑽狗洞不肖之輩。佛囉佛。他那是自墮惡孽。貧僧怎麼肯學他。素馨見他裝模做樣。一手摟着他脖子。一手將那酒杯往他嘴中一灌。那萬緣正有些忍不得。借這意思一口嚥下。道。菩薩。弟子今日破了戒了。素馨又夾了一塊金華火腿讓他。他道。佛喲。酒還罷了。這個實在不敢領。素聲(馨)道。我問你。你和尚們開口是佛。閉口是佛。大約見了婦人的那件東西。管情連佛也顧不得了。萬緣道。南無佛。這樣僧也有。像我貧僧。如槁木死灰一般。心如鐵石。再不動的。素馨笑道。果然。你伸出手來。我同你打個掌。任我引誘。你果然不動心。就算你是活佛。你若把持不住。你就認我做娘。萬緣道。這個貧僧秉得住的。纔伸出掌來。被素馨一把攥住手腕。他原來不曾穿褲。拉他的手在陰門上擦了幾擦。〔道。〕你秉得住秉不住。萬緣急縮回手時。素馨猛伸手在他胯下一掏。一根陽物如鐵杵一般豎着\footnote{這纔是眞僧現伸(身)說法。}。他連褲子一把攥住。笑道。這怎麼\footnote{此一段。前之這樣這樣畢。}。禿驢。你還假做作甚麼。小禿驢這樣硬起來。你這大禿驢還硬到那裡去。快些叩頭認我做娘。放了手。便一下跨在他身上坐着。摟着他的脖子就親嘴。那萬緣那裡還假忍得住。笑道。我的娘。把我的眞僧此時與你破了罷。將素馨放倒。取出肉具。對着陰門頂了幾頂纔弄進去。素馨笑道。我說怎麼叫你們禿驢。原來果然有這麼個大驢㞠子。哈哈笑了一聲。道。我今日造化低。怎被驢子肏了去。萬緣笑着。一連幾下。弄到了根。儘力抽扯。素馨覺比盛旺更加精妙。連聲只叫。好和尚。好眞僧。好師傅。好禿驢。怪不得女人們愛打和尚。弄了有一個時辰。素馨道。我儘夠了。且住了。有話對你說。那萬緣又狠狠的搗了幾下。素馨被他搗得哎喲了幾聲。他纔洩出來。道。有甚麼話說。素馨歇了一會。坐起來笑道。你怎麼不怕汚穢佛堂了。他笑道。佛在西天。他是大慈悲的。那裡管我們這些閒事。你可曾聽見僧尼會上唱的麼。

\begin{quotation}

大的大菩薩。小的小菩薩。他都是爹娘養下。

\end{quotation}

素馨又笑道。你可還戒葷酒了麼。若不戒。我同你吃着說。那萬緣笑着抱他對面坐在懷中。二物套上。一面動着。一面飮酒食肉。素馨向他說了來意。是二奶奶叫來約他相會。桂氏每月初一十五到佛堂燒香。萬緣見過多次。心中愛慕了這幾年。聽說。心中大樂。連聲道。造化造化。忙把酒一口乾了。道。趁早去。不要叫他久等。辜負了他的美情。素馨跨了下來。兩人站起。和尚拽上褲子。素馨蓋上盒子。拿了酒壺前走。萬緣一個出來。帶上了門。一手搭在他肩上。到桂氏房中來。那桂氏自掌燈時着素馨去後。即洗牝上床。眼望旌節旗。耳聽好消息。許久不見來回信。知他二人那裡做好事了。急得慾火上攻。淫水下注。眼睜睜直射門外。許多時。忽燈影下見兩個人影。急看時。只見那大師傅跳上床來。脫了衣服。鑽入被中。素馨將燈掭得大亮。出去帶上了門。那萬緣忙替桂氏脫光。雙手捧着他的臉親了兩個嘴。說道。多承奶奶不棄。貧僧不知修了幾輩子。今日得來親近玉體。眞合了小僧的法名。我有萬分的緣法。方得遇奶奶的這件寶貝。一面說着。一面將硬幫幫的大屌向胯中亂戳。不想桂氏水脹了紅門。東滑西滑。不得進去。桂氏被他戳得淫情似火。不見進去。忙伸手導入牝中。萬緣弄進龜頭。趁着水勢。幾攮至根。那禿驢好弄。或深或淺。或緊或慢。搗個不住。桂氏陰中被他巨物楦滿。已有無窮妙處。又經他這種戰法。高出他人之上。從未嘗此。弄得酥而醒。醒而酥。丢了數次。顫聲嬌語。再三央及。方纔住手。還不肯拽出。桂氏歇了一會。剛喘過氣來。萬緣又要抽動。桂氏笑道。我渾身都軟了。禁不得再弄。明日晚上罷。萬緣笑道。旣然請客。也要等他吃個醉飽。我纔領情。你就要吿止。眞是齋僧不飽。不如活埋了來。求你再布施。便要抽扯。桂氏送了個嘴。吐舌與他。道。好師傅。我實在來不得。叫了丫頭們來。你都弄弄。再來同我弄。就可盡你的興了。萬緣見他這段嬌態。疼愛得了不得。也不忍再弄。點頭應允。桂氏一絲兩氣的叫道。香兒你們來。原來他們四個都在門外聽呢。聽得叫。都一齊進來。桂氏向萬緣道。你下去。萬緣跳下床。就把香兒抱着。脫去衣褲。按在椅子上就搗。他們一個個聽得淫水浸盈。毫不費力。容容易易弄了進去。萬緣同香兒弄着。向他三人道。你們都脫光了等着。我一個一個的打發了來。他三個也就脫去。萬緣俟(挨)次輪流轉弄。自二鼓進來弄起。直至五鼓初敲。他也將要洩了。翻身上床。又同桂氏痛抽一陣。方洩在他牝中。桂氏看了半夜的活春宮。陰中如蛆拱的一般。被他這一下弄得骨軟筋酥。如登仙之樂。素馨他們四個也都被他弄得飽腹充腸。心滿興足。桂氏遇過萬緣之後。姚步武盛旺再來弄時。如吃過江瑤柱。不堪再嚼屠門肉矣。姚澤民享用他繼母庶母。將桂氏久抛。從不見他有一毫慍色。有一句怨言。反見他比當日紅光滿面。笑容可掬。以爲是閨中賢淑。不以此道爲念的。私心欣慶。孰不知他還尋的是極粗極大的美物。夜夜不空。故棄丈夫如敝屣耳。一夜。這萬緣正同桂氏在床上。他靠着枕頭坐住。叫桂氏跨在他身上。對面將兩物套好。學喇嘛供的喜樂禪佛那樣式。一起一落。正做囗(得)高興。忽見香兒靑梅綠蕚喜笑得跌跌滾滾跑進房來。桂氏笑罵道。你這三個小淫婦瘋了。這昝晚跑來笑甚麼。香兒道。我們有句話來問大師傅。萬緣道。你們〔問〕甚麼。香兒道。我們纔在外邊講頑話。我說男人的那東西是筋的。靑姐強說是皮的。綠姐咬定說是肉的。我們賭了個東道。故此來問大師傅。看誰說的是。囗(萬)囗(緣)囗桂氏一面動着。一面說偈道。

\begin{quotation}

三人不須多強。說得都還相像。硬時是段純筋。軟了皮囊形狀。咦。大家仔細試端詳。一團肉在光頭上。

\end{quotation}

把三個丫頭笑得東倒西歪的出去。笑得那桂氏一仰一合。騎不住肉鞍。竟墜下驢來。睡在床上。揉着小肚子笑。萬緣見桂氏仰臥着笑。就扛起腿來弄。桂氏忙用手摀住陰門。道。你也說個偈語。纔許你弄。萬緣笑着道。

\begin{quotation}

生我之門死我戶。人人盡道消魂處。老僧直入等閒看。撞入迷魂深澗去。咦。憑威出入數千遭。佳人水漲藍橋路。大衆聽者。被毛戴角任閻羅。且向此中尋樂趣。

\end{quotation}

說畢。挺起小和尚。一頭撞將入去。把桂氏弄得癱在錦褥之上方歇手。此後萬緣姚步武盛旺任桂氏心中所欲。輪流約到房中取樂。不必多敍。且說姚澤民在內裡弄。桂氏在外邊弄。也弄了幾年。桂氏的陽運當旺。姚澤民的陰運要出宮了。這是爲何。自姚華胄往廣西去後。到了天啓七年。忽然想起他來。問羣臣道。姚華胄在廣西數載。他年垂八十。他家中可有兒子否。有知道的奏道。他有二子。天啓傳旨召見。看他大兒子有五十來歲。迂迂腐腐的。小兒子約將三旬。頗有父風。天啓問他二人名字。大兒子答應不出。小兒子俯伏奏道。臣兄名姚予民。臣名姚澤民。天啓顧婊澤民道。爾父遠去數載。爾爲子者也應當去一看。你今可到那裡看他日食如何。康健還如昔否。速來回奏。爾兄庸愚。只可爲守戶之犬。爾異日即承襲爾父之爵。他兄弟二人領旨。叩頭謝恩而出。這是面奉上諭的事。不敢稽緩。就擇日起程。這姚澤民第一好的是杯中之物。不論燒罎黃細。到口就吞。第二件就是酒字底下的那個字。一夜離了婦人。他也過不得。他這一次是奉旨省親。旱路驅馳。不敢帶婦人同往。在家中同那些婦人終日混慣了。如今竟虎撲兒百姓眼起來。那裡還過得。雖帶了有兩個龍陽小廝。到底與婦人滋味各別。他路上但有婊字(子)。只面上略有人形。腰中有個窟窿。他定要領敎領敎。這大路上的土條妓女。私窠戲旦。可有甚麼像樣的。他不過只算鬆了鬆胯下的窮筋。算不得個取樂。他到了南京。在水西門外店中暫寓。就叫了店主人來問道。如今城中可有馳名的婊子麼。店主人道。近來妓女中也都平常。倒是個瞎姑。叫做錢貴。果然色藝雙絕。但聽得人說。他近來總不接客。不知何故。姚澤民道。他不過因有了個名頭。故做身分。若多給他鴇兒幾兩銀子。怕他不肯。遂問了住處。一團高興。帶了十數個家人。鮮衣怒馬到錢貴家來。錢貴自別了鍾生。一個客也不接。只說有病。郝氏強了他幾回。他執定不肯。因沒有大出手的孤老。郝氏也容忍了過去。這日。錢貴正臨窗坐着。姚澤民問到他家。敲開門。竟走了進來。一眼早已看見。果然好個女子。郝氏忙迎着道。老爺駕臨賤地。有何貴幹。一個家人道。我們主人姓姚。是鎭西將軍侯府的二公子。慕你女兒的大名。特來要同他相與。郝氏道。小女醜陋。且近來有病。恐不能陪侍。姚澤民道。你不過只你女兒在盛名之下。要拿些身分。多要幾兩銀子罷了。何必推辭。我不過但嫖兩夜就要起身。我也不肯薄了你。叫家人取一封五十兩遞與郝氏。他滿臉是笑。道。老爺請坐。我去同小女商議。一面叫丫頭收拾酒饌。一面到錢貴房中來。錢貴先在窗子口。聽見有人進來說話。他忙避過。到床沿上坐着。聽見說要來嫖他。正一腔怒氣。郝氏進來道。我的兒。這是位過路的貴公子。慕名來訪你。只宿兩夜便送一個元寶。這樣好主兒。你總成老娘賺這幾兩銀子罷。錢貴忿然答道。兒之此身是決不再辱的了。母親不用癡想。若定要圖這幾兩銀子。我必以頸血濺地。那郝氏大怒道。我從來沒有聽見門戶人家守節的。就是良家婦人要守節。也必定等有個丈夫死了纔守。也沒有望空就守的理。我養你一場。靠你養老。你不接客。難道叫我養你一生不成。我不過爲你是親生之女。下不得手打你。你再執拗。我就拿皮鞭奉敬你了。錢貴道。母親。不要說皮鞭。雖鼎烹在前。刀鋸在後。我亦不懼。郝氏越發怒道。罷了。你旣是這樣的逆種。不是你死。就是我亡。我且打你個辣手。你纔知道利害。惡狠狠就取鞭子。錢貴道。母親不必動怒。你旣愛錢不惜人。我要這命何用。大呼道。罷罷。我把這命還了你罷。猛然一頭撞在地下。額鼻皆破。滿面血流。便暈了過去。幸得代目在傍。連忙拉住。不致十分重傷。郝氏見不是勢頭。聲也不敢嘖。〔不〕多時看見錢貴甦醒。纔放了心。他被這一嚇。忙走出來將銀子送還。道。小女不肯奉陪。老身也沒福受老爺厚賞。姚澤民見了錢貴。十分心愛。見他不從。着了急。使勢威逼道。我一個侯府公子來相與你。難道還玷辱了你不成。好好依從便罷。不然拿去送官重處。再不然叫小廝們將這臭娼根剝光了按住。我硬弄了。看你奈我何。大明律上沒有個強奸娼婦的罪名。錢貴也怒道。匹夫不可奪志。不要說你是個侯子。此時就是帝子王孫。我頭可斷而志不可移。你要行強盜奸淫之事。我與你兩命俱捐。叫代目取了把剪子。他接在手中。道。你好好回去罷。再行強逼。我即刺喉而死。你雖勢利大。我母親無奈你何。我當爲厲鬼。以報斯恨。郝氏恐怕女兒當眞弄出事來。哀求道。我這小女沒福。不中貴人擡舉。況外邊美妓不少。老爺另尋一位罷。姚澤民還要使威使勢的唬嚇。有一個知事的老管家說道。這件事原圖取樂。這個樣子料道也沒甚樂趣了。況爺是奉旨省親的。倘在此嫖妓弄出人命來。聖上知道。干係非小。不如回去另尋一個適興罷。姚澤民聽他說得有理。叫家人接過銀子。嘴中罵着。悻悻而去。到了店中。在舊院裡接了個有名的夏錦兒。並一個江西新到來往(姓)嚴的婊子\footnote{此處將二人一題。後來宦蕚口中又一題。方不是隨口施捏人名也。}。嫖了兩夜。起身去了。錢貴面上瘡痕養了個把多月纔得全愈。閉門兀坐。連窗前都不近。從此以後。郝氏再不敢逼他接客。凡有人來。都推有病。端的這錢貴不接客。後來作何結局。並姚澤民到廣西去。何時復命。後來一個個自然還他下落。此時再說鄔合的妻子嬴氏。他父親名字叫做嬴陽。是個戲子。係崑山縣人。母親陰氏。隨他父親學得許多旦脚的戲文。嬴陽因有殘疾。唱不得戲了。不能餬口。雖然陰氏會唱。在本鄕本土怎好叫他出去做戲。就是他岳家也不依。陰氏裙帶之下雖有一件掙錢的傢伙。也不好明做這外水買賣。也曾相與了個把厚友。因街坊上議論風生。住不得了。故此搬到南京來。聞得有個阮給事名大鋮。酷好梨園女旦。遂依傍在他門下走動。生得這個女兒。小名皎皎。與鄔合。要知詳細如何。須看下回分解。

姑妄言五卷終



\endnotetext[1]{「眞佳訓」原作「眞嘉訓」,據第四回改。}

\endnotetext[2]{「說話」原作「話說」,據文義改。}

\endnotetext[3]{「說他」原作「他說」,據文義改。}

\endnotetext[4]{「徜徉」原作「徉徜」,據文義改。}

\endnotetext[5]{此句原書右有夾批「豈非欺乎」四字。}

\endnotetext[6]{此句原書右有夾批「均得很」三字。}

\setcounter{footnote}{0}

\theendnotes

\part*{姑妄言第六卷}
\addcontentsline{toc}{part}{姑妄言第六卷}
\markboth{姑妄言第六卷}{姑妄言第六卷}

鈍翁曰。嗚呼。男風一道。雖所由來者久矣。然未有盛於今日者也。此輩幾幾半天下。不但恬不知恥。猶欣欣以爲榮焉。得人人皆有聶變豹之孽。且使此輩聞而畏避。庶可洗盡此頽風。

釵(敍)嬴陽家世並梨園子弟履歷許多趣話。令人噴飯。

突然撰出個閔氏來。不但嬴陽感激。看書的人亦感激。此何故。無閔氏則嬴陽不得生。嬴陽不得生。則無陰氏並皎皎。無陰氏並皎皎。那得這兩回熱鬧書看。

一部書偷漢之婦人不少。並無一相重者。即此一回內。陰氏之偷漢。是衆學生誘他。乃略知竅男子誘一不知竅之幼女。是一種行事。金鑛之偷陰氏。是兩人同誘。兩個都是老手。又是一種行事。皎皎之偷漢。是他先誘龍家小廝。兩個都是知情而不知味的。又一種行事。至於了緣之偷皎皎。則是強盜之行事矣。

嬴氏如不遇了緣。焉知久之不爲良婦。被這賊禿一偷。以至辱身出醜。若非鄔合以天閹自責之夫。使嬴氏不知至於何地也。僧人中如了緣者正復不少。緇流一途。原是盜賊藏垢納汚之所。奈愚人往往爲其惑。有守土之責者。不可不嚴察此類。有佞佛之流。見余此評。必合掌曰。枉口白舌。何苦謗僧。孰不知余非謗之。正是爲大雄氏做功臣耳。

龍家小廝酒後一篇不忿的話。直欲逼走嬴陽耳。不然。住到何日是了。嬴陽不去。聶變豹之仇何日得報。皎皎與龍颺之情緣何日得絕。今日嬴陽之走。異日死龍陽之地耳。許多線索。不留心看不出也。

阮大鋮之請鐵按院。乃嬴陽報仇之節目。鐵按院反覆盤問。足見細心。安得爲官者肯個個如此。則無寃民矣。

寫游混公又爲龍家小廝之師一段。總是寫他到極不堪處。且又使之一現。不致冷落也。

鄔合嬴氏二人成親後。一個無用的天閹。一個貪淫的女子。恰恰合在一處。何(如)何下筆。此段寫得情景逼眞。設身處地一想。不過如此而已。

嬴陽何等人物。暴發二千餘金。眼眶更大。就要做財主身分。嗟夫。錢之能大人也若此。

古語謂搆訟云。無賴不成詞。閱此。誠哉言也。嬴陽在按院前供聶變豹之罪。固係實事。而自護之語亦不少。因說得近情。故能聳人之聽耳。

王酒鬼一個挑水的老兒。泛泛然看去。是個極無關係沒要緊的人。後來洩露幾(機)關。反是個極要緊的節目。此等處令人如何捉摸。

\chapter*{姑妄言卷之六\\
第六回 嬴氏貪淫爲淫累始改淫心 賊禿性惡作惡深終罹惡報\\
附 閔氏垂慈 代巡聽訟}
\addcontentsline{toc}{chapter}{第六回 嬴氏貪淫爲淫累始改淫心 賊禿性惡作惡深終罹惡報}
\markboth{第六回 嬴氏貪淫爲淫累始改淫心 賊禿性惡作惡深終罹惡報}{第六回 嬴氏貪淫爲淫累始改淫心 賊禿性惡作惡深終罹惡報}

話說這嬴陽係蘇州府昆山縣人氏。他家世代單傳。從無兄弟姐妹\footnote{偶憶一笑談。一家世代單傳。一人謂彼云。一個兒子是個險子。其人問道。你有幾位令郞。答云。只有二個小女。此人笑道。我的一個蜆子還強如你的兩個蚌。}。積祖以學戲爲生。他父親是個花面。人都順口叫他做嬴丑子。娶妻養氏\footnote{以他養兒子便妙。若陰癢之癢便不妙了。}。只生得嬴陽一個。嬴陽六七歲時。生得甚是美麗。柔媚如女子一般。他父親視爲奇貨。以爲此子將來不但能克紹祖業。還必振興家門。遂將他送入一小班中做了一個正旦。你道這樣好兒子不送去念書。反送去學戲。是何緣故。但他這崑山地方。十戶之中有四五家學戲。以此爲永業。恬不爲恥。就是不學戲的人家。無論男女大小。沒有一個不會哼幾句。即如杞梁之妻善哭其夫而變國俗是一個道理。故此天下皆稱爲崑腔。因崑山係蘇州所轄。又稱爲蘇腔。但這些唱戲的人家他並無恆產。一生衣飯皆從此出。只可餬得眼前。安能積得私蓄。所以兒子不得不接習此藝。只三五年間便可出來唱戲餬口。他這戲子中生得面目可憎者。只得去學花面。不但怨天恨地。還怨祖墳風水不好。又怨妻子陰戶不爭氣\footnote{這一笑(怨)怨得可笑。陰戶寃哉。}。不得個標致子孫爲掙錢之本。將來何以存濟。若稍有面目可觀者。無不兼做龍陽。他那靑年之時。以錢大之一竅。未嘗不掙許多錢來。但這種人又喜賭又好樂。以爲這銀錢只用彎彎腰蹶蹶股就可源源而來。何足爲惜。任意花費。及至到有了幾歲年紀。那無情的鬍鬚。他也不顧人的死活。一日一日只管鑽了出來\footnote{笑倒。}。雖然時刻撏拔。無奈那臉上多了幾個皺紋。未免比少年減了許多丯韻。那善於修飾的。用松子白果宮粉搗爛如泥。常常敷在面上。不但遮了許多缺陷。而且噴香光亮。還可以聊充下陳。無奈糞門前後長出許多毛來。如西遊記上稀柿衕內又添上了一座荆棘嶺。撏不得。剃不得。燒不得。把一個養家的金穴如栅欄一般擋住。眞叫人哭不得。笑不得\footnote{令人笑殺。}。却無可奈何了。眞是。

\begin{quotation}

一團茅草亂蓬蓬。從此情郞似陌路。

\end{quotation}

要知這就是他腎運滿足\footnote{腎運二字新。大約即桃花星更名耳。}。天限他做不得此事的時候了。到了此時。兩手招郞。郞皆不顧。雖在十字街頭把腰彎折。屁股蹶得比頭還高。人皆掩鼻而過之。求其一垂靑而不能。要想一文見面萬不能夠了\footnote{龍陽君看到此。定然掩袂而泣。}。到了唱戲。伸着脖子板筋疊暴着掙命似的。或一夜或一日。弄不得幾分錢子。還不足餬口。及悔少年浪費之時。已無及矣。纔想到這件掙錢的傢伙。比不得種地的農夫。今歲不收。還望來歲。只好像行醫的話。上下改三個字便是的評。說的是。

\begin{quotation}

趁我十年嫩。有股早來舂。

\end{quotation}

這嬴丑子生得一臉黑麻子。又鬼頭鬼腦。宛然天生得一個醜態。故學了丑。少年時。他見同班中朋友俱有人愛。都會掙錢。獨到了他。人皆一介不與。他睜着兩個眼睛。看得好不動火。人人都穿得齊齊整整。獨他只一件舊布直裰。有人問道。別人都體面。爲何你獨如此。他也無別話可對。但慘然指着面上道。你看我的臉哪。他人無不大笑。他間或做個媚態去撩人\footnote{這也是無聊之極思。}。人皆不顧而唾\footnote{掃興。}。時常對鏡自嗟自嘆。自怨自艾。到那無聊之極的時候。自己摸着糞門。嘆道。我比他們雖不能掙錢。他們放的都是散屁\footnote{散屁二字甚新。}。要像我這個囫圇屁眼也萬萬不能夠了。今見兒子如此標致。以爲是祖宗積德所致。方有此跨竈之子。又常撫摸養氏的牝戶。贊道。不意此癟蚌內產生此一個美珠。這嬴陽又甚聰明。生來該吃這種茶飯。敎的戲就會。腔口吞吐也好。身段囗模更覺窈窕。裝扮起來。宛然一個嬌媚女子。學了三年就可上場去唱。無一人不喝采。無一人〔不〕羨慕。因他年紀太小。故尚有待。到了十二三歲。就有個大老官愛上了他。對嬴丑子說要賞鑒他兒子的嫩臀。他豈不樂從。那大老官送了他一大塊銀子。又替嬴陽做了兩套時款紬絹衣服。替他把聰明孔開闢出來。此後果然技藝益發精妙。見者無不消魂。二三年間。他也正正經經掙了一注大錢。因他年幼。尚不知浪費。得來的銀錢皆交與父母。那嬴丑子夫婦喜得屁滾尿流。把兒子的糞門視同聚寶盆一般\footnote{異想處甚然。非異也。鄧通糞門中更有一座錢山。}。偶然一日。嬴丑子忽然放了一個大響屁。淸越異常。心有所觸。不覺慘然長嘆。養氏笑道。放了一個屁。爲何做出恁個樣子。你捨不得這一響麼。嬴丑子道。我因此屁想起兒子來。他雖掙了幾個錢。今生要像我放這樣個響屁。斷乎不能的了。不覺傷心耳\footnote{近日放屁不響者甚多。不知他父親尚傷心否。}。那嬴陽後來就漸漸不似先。俗云。近珠(朱)者赤。近墨者黑。被這同班中朋友一陣引誘。嫖賭嚼搖四個子一幷施行。銀錢雖有。東手接來西手去。一文也到不得家。那嬴丑子原有個弱症。近來舉發。唱不得戲。一家衣食皆倚仗賢郞。可還敢管他。敢怒而不敢言。抑鬱在心。病漸加重。就嗚呼哀哉了。嬴陽雖是個戲子。他各班中相識者多。都來上紙弔孝。他要圖體面。無不從豐\footnote{近日詩禮之家於親喪無不從儉者。視嬴陽猶不若也。}。及至喪事畢後。他向來所掙家私也就去了多半。那時城中有個財主。姓聶名變豹。生性淫惡。他有個妹子嫁在京中一個皇親家爲寵妾。他倚勢行凶。把持官府。無惡不作。納了一個監生名色。同這知縣衙官分庭抗禮。眼空一世的樣子。人人側目。雖有一個理刑要拿他。但這蘇州欽差來的織造。並駐防太監出京時。那皇親諄諄之囑托護庇他。那時太監的威勢。雖撫按也不敢得罪他。何況以次官員。他因有此靠山。所以更橫行無忌。殺人性命如草菅。占人妻女如囊寄。鄕人皆惡之。就把他的名字同音而改。都稱他爲孽便報。他家房產深邃。姬妾衆多。旣貪女色。又慕男風。女子中雖被他奸淫無數。而男子總未試新。這是甚麼緣故。這樣作孽之人。就生了個作孽之具。他的陽物雖只有六七寸長。竟有鍾口粗細。也還不足爲異。那個龜頭竟如驢腎一般。弄入陰中。一發了興。開了花。就如同一個喇叭。婦人的陰戶門小而內寬。入去還易。拔出時如小碟子一般。這一撑還禁不得。年小些的婦女乍經了他。還弄得七死八活。那糞門中不能容得此物。他在家中同妾婢們弄時還有些須憐惜。若高興去嫖。任意衝突。不管死活。娼妓們多受他的淫毒。因此背地都叫他聶驢子。有此大名在外。這些龍陽雖然愛錢。誰肯做這賈胡剖腹藏珠的事。拿性命來換錢使。古語說得好。

\begin{quotation}

留得五湖明月在。不愁無處下金鈎。

\end{quotation}

若果然有個好臉。再有一個嫩股。何往而非銀錢。豈肯來輕試他這個孽具。況這件事如賣房地文契結尾兩句一般的。此係兩相情願。並無逼勒等情。那小官不願領敎他這件奇物。他也沒法。他屢屢看上了嬴陽。托人多番作合。又以重利誘之。嬴陽再不敢輕諾。這聶變豹恨入骨髓。想道。定設一計使他入我牢籠。一文不得。白白的痛弄一番。更置之於死地。纔出得這口惡氣。一日。他想了個主意。向着他一個愛妾閔氏商議此事。閔氏勸他道。老爺請想。你這件東西。我們婦人家跟久了你。還難禁受。何況姣童。人的性命不是兒戲的。他之不肯。大約也是知道大名在外。不敢應承也是人情。必然有罪。何至於死。據我想來。前後滋味大槪相同。何不棄彼而取此罷。聶變豹大怒道。我這樣的家私。如此的聲勢。況又有此奇具。若不一嘗這美男子的妙臀。是我負天所付了。你旣如此護着他。把你的後庭我試試。我就不要他了。閔氏怎肯從井救人。嚇得閉口無言。半晌道。老爺息怒。我們遵着行就是了。聶變豹又叫了他一個心愛的標致丫頭名喚垂絲來。吩咐道。你與姨娘兩個人明日替我如此如此行事。要洩露了。我也不處治你們。只將屁股每人弄一下。至於死活。那就憑你們的造化。那閔氏同垂絲你我相顧。面容失色。唯唯領命。到了次日。聶變豹傳了嬴陽這班子弟來家中唱戲。到半本落臺時。已有二鼓。合班人吃飯了。一個個都出去淨手。嬴陽落後出來。尿完了剛要轉身。後邊有人將他衣襟拽住。忙回頭一看。月下見得分明是個俊俏女子。却是丫鬟裝束。嬴陽疑心。問道。你做甚麼。那女子近前低聲道。你姓甚麼。答道。我姓嬴。那女子喜孜孜攜着手道。到那黑影處。有話對你說。這裡怕人撞見。嬴陽此時魂都不知往那裡去了。同他到了黑處。那女子反將他摟過來。親了個嘴。附在耳上道。剛纔我家姨娘在簾內看戲。見了你。着實心愛。想要同你會會。有許多好處到你。叫我來問你。明日可有戲。嬴陽道。明日沒有。女子道。你今夜戲散了。合班同行。大約脫身不得。明日到日落時候。你到我家花園後門外等着。我出來接你。那是沒人的地方。只管放心。又道。恐你疑惑。這是姨娘送你的表記。你可收了。遞到他手中。又一把將嬴陽摟得緊緊的。道。親親。你怎這等愛人。我姨娘生得玉美人一般。我總成了你。你不要忘了我。嬴陽還是個十五歲的小孩子。知道甚麼利害。少年心性。以爲奇遇。喜得話都說不出來。只點頭道。我定來。我定來。你務必出來接我。不可誤了。那女子道。不用多說。看有人來。抽身去了\footnote{先見此婢名垂絲。以爲隨手謅一名字耳。至此方知有雙關二意焉。垂絲者。海棠也。故用之爲婢名。二者謂以此婢爲香餌。垂於絲綸之上。以釣嬴陽上鈎耳。書中此類甚多。不能盡爲指出。惟觀者留意焉。}。嬴陽不便打開。將那包兒裝入鈔袋中。又來唱戲。散了回家。已將五鼓。到了家中。取出包兒。燈下打開一看。一隻大紅緞子睡鞋。滿幫白梅花。豆綠拽拔。白綾底兒尖上釘着黃豆大的珍珠。長僅三寸。裡面一個紅紙包兒。打開是一個噴鼻馨香的香囊。上繡着交頸鴛鴦。還有一根金並頭蓮。一根金雙頭如意簪。四個連環戒指。十個滾圓雪白珍珠。嬴陽喜得心窩亂癢。將那鞋親了幾個嘴。叫了幾聲心肝。仍包好放在鈔袋內。脫衣上床。把那鈔袋摟在懷中而睡。矇矓之際。到了聶家與那女子相會之處。那女子一見。喜笑道。好信實人兒。我等了好一會了。上前拉着手道。我們進去。嬴陽猛省。站住脚道。倘遇見你家老爺怎麼處\footnote{此一頓有理。嬴陽雖係無知小孩子。却是個聰明少年。焉敢孟浪無忌憚至此。有此一想。方見彼未嘗不省得。特爲迷魂陣困住耳。}。那女子道。我家姨娘們多。每夜輪着陪老爺的。各人各屋不妨事。難道你怕。我們是不怕的麼\footnote{有此一轉。更自放心。即他人亦必墮其術中。何況嬴陽。}。因接着笑道。小寃家。你這樣多心膽小。嬴陽此時精魂俱失。雖刀鋸在前也不顧了。仗膽同他進來。到了一間齊整屋內。燈下一個美貌婦人。笑吟吟上前拉住道。小寃家。想殺我了。拿臉兒偎倚着。嬴陽見這光景。興不可遏。不暇開言。攜手上床。脫衣解帶。見那婦女柔軟如綿。淫樂了一度。還想要敍敍情意。只見那女子揭開帳子。道。天大亮了。快走罷。嬴陽見日光果然射入。忙穿衣同他往外飛跑。不防被門檻一絆。幾乎跌倒。一驚醒來。原是一個大夢。鈔袋還抱在懷中。淫精已溢於被褥。看窗上時已日上三竿。定神自思。夢境宛然。暗喜道。今晚必定成就佳期。這夢兆大祥可喜\footnote{眞是癡人說夢。誰知此夢不應在婦人之前面。而應在自己之後面。好說夢者謂之反圓夢。如夢哭得笑。夢笑得哭之類。亦是此意。}。他那包兒不敢與娘知道。仍帶在身邊。慢慢起來梳洗。吃罷飯。步到聶家後園門口一看。果是一條死巷。無人來往一塊空地。更自放心\footnote{精細。}。又走了轉來。坐了一會又去。天色尚早。只得又回。眼巴巴再不見晚。急得來回只是走。看看日色銜山。心中大喜。到了園門時。已東方月出。正在遲疑。猛聽得園門呀的一聲。嬴陽心下一驚\footnote{寫少年心虛膽怯。情景逼眞。}。仔細看時。正是那女子。心纔放下。那女子道。趁沒人。快進去罷。嬴陽隨了進來。丫頭關上了門。兩人攜手進入園中。互相摟抱。親嘴咂舌。調笑了一會\footnote{以前則聶變豹所定之計。此處之親愛。定是垂絲自添者。}。纔又同行。轉彎抹角。走了好一會\footnote{照前房屋深邃句。}。到了一間房內。尚未點燈。月光照着。甚是富麗。以爲應夢。心下私喜。那女子低聲道。你等一等。我去看看。若老爺睡了。我接了姨娘同來。徉徜去了。嬴陽等了多時。尚不見來。心中也有些懊悔疑慮。怕有人來看見。要想出去。旣不認得路。又恐遇着人。又轉念道。昨夜夢兆好。料不妨事\footnote{處處拿定好夢二字。後來應得好夢。活是癡心少年自哄自語。}。大約是那裡脫身不得。況且這女子有這樣情意到我。決無他故。正凝眸注目的盼望。忽見兩個大亮燈籠。一陣人走來。嬴陽舉目看時。正是聶變豹。那魂錚的一聲。已不知何往。嚇得跌倒在地。聶變豹進門一見。大喝道。有賊。快拿住。不要放走了。兩三個家人上前拎起跪下\footnote{拎起。妙。是嚇癱了的樣子。}。聶變豹看了一看。問家人道。這不是嬴旦麼。家人道。正是他。聶變豹坐下。大怒道。好大膽的奴才。你夤夜直入我內室。非奸即盜。小廝們。剝了這廝上下衣服。緊緊的綁起來。明早送到縣裡處死這奴才。家人上前正剝了衣服。褪了褲子。聶變豹道。他那帶子上是甚麼。家人道。是一個鈔袋。聶變豹道。拿來我看。家人遞上。他一打開。假意吃驚道。我當是他纔進來。原來把鞋同首飾都偷到手了。明明是盜。又想借此鞋囮奸。好惡人。明日到衙門夾打着。追他的餘黨。嬴陽被綑得如一個粽子相似。精光着睡在地板上。疼痛難忍。流淚哀吿道。老爺天恩。我怎敢私自入來。是老爺府中一個女子昨夜約小的來的。這東西也是他給我的。並非敢偷\footnote{的少年無知之語。此話可是做得辯辭的。}。聶變豹道。這女子姓什麼。如今在那裡。嬴陽又吿道。小的不知他的姓。是他帶我到這裡。他就去了。聶變豹更怒道。這奴才胡說。你連人的姓都不知道。就敢跟他進來\footnote{計雖毒而言有理。}。旣來做賊。又誣賴我家的人。汚衊我家。益發可恨。就算眞有其事。明是奸了。罪更重些。小廝們。出去把衆丫頭都叫來與他認認。若是沒有。也叫他死而無怨。衆人答應一聲去了。少刻。有數個丫頭各拿着一個燭臺。都點着明晃大燭進來。房中照得雪亮。聶變豹道。他說是你們那一個帶進來的。可到他面前叫他認。衆丫頭上前齊道。你認眞了。自作孽自當。不要混賴無辜。嬴陽一個個看了總不是。他也還有些良心。不肯寃人。哭說道。都不是。是一個瓜子臉。雪白的面龐兒。穿着靑衫白裙。腰裡繫着一條紅汗巾。聶變豹道。這奴才信口胡說。我家並沒有這個人。正說着。只見一個美婦走進來。在傍邊椅上坐下。聶變豹向他道。這就是嬴旦。我纔回來就到你屋裡。看見他正在此做賊。叫小廝們拿住綁了。還只當不曾偷得東西。誰知把首飾並一隻鞋都偷了藏在身邊。反誣賴我家有個女子誘他來的。你說可惡不可惡。明早送官夾打死了。方除我恨。那美妾道。老爺不消動怒。丫頭們。取酒替老爺消氣。丫頭答應。去不多時。捧了酒肴來擺下。擡過桌子。斟上酒。美妾在傍陪飮。那嬴陽又是疼。又是怕。哼一會。哭一會。說道。你哄了我進來。這會兒你不知躱在那裡去了。叫我受罪。又叫一會寃枉。聶變豹怒道。這奴才還敢胡說叫寃枉。丫頭們打嘴。那些丫頭看見這樣粉團般一個標致男子。光光的綁在地下。好不心中又憐又愛。誰還認(忍)來打他。因主人吩咐。不敢不遵。一個大丫頭走近前。背着身子。手拍手響了兩下\footnote{妙極。寫出憐愛。}。低聲道。不要嘖聲了。何苦捱打。嬴陽到此時以死自聽。見那丫頭說。也不叫了。只得閉着眼哼。那美妾心中老大不忍。斟了一杯酒。站起敬與聶變豹道。我乞老爺一個恩。聶變豹道。甚麼事。那妾道。這小子罪雖該死。不過是明日到官自有官法處治。此時饒了他。綁拴在這裡。料他也飛不出去。聶變豹還不肯。那妾再三哀求。纔依了。那妾叫丫頭放了他。丫頭忙都上前。七手八脚替他解了。嬴陽渾身綑麻了。這一放。更疼得動不得。睡在地下哼。那妾見他嫩白皮膚綑得一道紅一道紫。更覺慘然。又道。拿他件衣服與他遮着身子。一個丫頭忙拿衣服替他蓋上。只見又走進一個丫頭。到聶變豹面前道。奶奶叫來請老爺。有要緊話說。聶變豹躊躇道。這昝晚有甚麼話說。你去說。有話明日說罷。那妾慫恿道。奶奶旣來請。必定有要緊的話。老爺去去再來何妨。那聶變豹站起道。也罷。我走走就來。兩個丫頭忙點燈籠照着去了。且說這聶〔變〕豹。他雖惡甚。他的個正妻單氏甚是賢慧慈仁。他待這些妾婢不但不醋。且個個加恩。聶變豹甚是敬他。他每每但知丈夫做人那惡事。亦更苦口相勸。聶變豹雖不能全聽。十分中也還聽他一二。那垂絲去哄嬴陽。因奉主人之命。不敢不遵。大非本願。他哄嬴陽到了閔氏房中去。回復了聶變豹。見他去了。忙來向閔氏道。嬴陽已哄到姨娘屋裡。老爺去了。不知他死活何如。姨娘快去解勸解勸。救他的命要緊。不然這個罪是姨娘同我造的。閔氏道。我先去。但恐我的面皮小。救不下來。你可悄悄去稟上奶奶。求奶奶力量。或者還有幾分指望。閔氏來後。垂絲忙到單氏房中。將主人叫哄誘嬴旦的話詳細稟上。求奶奶力勸。救他的性命。又道。奶奶只說聽見傳說。千萬不要說是我來稟奶奶的。恐怕老爺嗔怪。那單氏聽了。嘆了兩聲。念了幾聲佛。忙叫丫頭去請聶變豹。他一去後。那妾立起。走來嬴陽面前蹲下。用手撫摩他的身上。道。我看你也是個伶俐人。怎麼大膽到這裡來。嬴陽先見他求情放了綁。此時又如許見憐。感激不盡。哭訴道。實是有個女子約我進來的。奶奶救救我的命罷。那妾道。人約你進來的話並無見證。就到了官。這句沒指實的話也不信。況你人贓現獲。一陣夾打再不能免。總是你自己的錯。怨不得人。我同這些丫頭那一個不可憐你\footnote{有此一句。使聶變豹之惡愈著。}。你看老爺那性子可是勸得的。叫我如何救你。嬴陽道。奶奶的恩典。我死了也是感激的。我死。怨命也罷了。但只一個寡婦娘。又沒有兄弟姊妹。可惜白養我一場。就嗚嗚的哭起來。那妾也滴了兩點淚。附在他耳上道。只有一件可以救你。你可依得。嬴陽聽得救他。就住了哭聲。道。奶奶肯救我。就是我重生父母了。有甚麼不依的。那妾道。我家老爺酷愛小官。你捨着同他睡睡。救了命罷。嬴陽癡了一癡。也悄說道。外人傳說老爺的東西連婦人還禁不得。我們如何承受。那妾悄悄又道。你依了罷。大約受些狠苦。也還未必就傷命。因爲他愛你。你屢屢不肯。纔下這毒計。你再不依。他不但強弄了。還白白送了性命。送官是假。此時他要害你。性命値甚麼。你難道還不知他平常的狠毒麼。嬴陽方恍然大悟。儘着叩頭道。奶奶你是我救命的恩人。我要不死。後來報你的恩罷。嘆了一口氣。道。罷了。料道跳不出去。捨着身子。性命交與他罷\footnote{太史公曰。怨毒之於人大矣哉。嬴陽此數語。今日不死於聶變豹之手。異日聶變豹必死於他矣。}。那妾道。旣如此說。等他來。我救你。說了。仍回位坐下。只見聶變豹來了。那妾道。我有一句話。老爺肯聽麼。聶變豹道。甚麼話。那妾道。這小子雖來做賊。贓物旣不曾拿去。又不曾有奸淫的事。恕他年小無知。他纔突(哭)訴家中只有一個寡母。並無親人。他也還生得好。叫他拿身子替老爺陪罪。也可出得氣了。不必再深究了。我纔問他。他也情願。聶變豹道。旣你說情。我依了你。因向嬴陽道。我看他面上。饒你一條狗命。你須順順的。若拗手拗脚。我却不算。叫丫頭們擡過一條春凳。鋪上褥子。地板鋪了紅氈。叫他扶起爬在春凳上。站在氈上。嬴陽此時身不由主。憑他們擺佈停當了。聶變豹渾身脫光。笑對那妾同衆丫頭道。你們都不許去。在這裡看我老爺試新。他走近前。摸着嬴陽的屁股道。你不許動。嬴陽知道有個性命相關的局面。也不看他的大小。低頭閉目。伏在凳上。那聶變豹吐了一口唾沫。抹在糞門上。又自己擦些。垂着首。捏着陽物。對準糞門就頂。那裡進得去。還不曾進得些須。嬴陽已覺火燒火辣。那聶變豹不得其門而入。發起性來。憑身用力往裡一下。攮進去了一個龜頭。只聽得嬴陽大叫一聲。哎呀。我死。就不做聲\footnote{應得好夢。}。那聶變豹那管他死活。幾送到根。任意抽送起來。半晌。只見嬴陽透過一口氣來。渾身亂顫。聲氣也顫篤疏的哭道。不得活了。不得活了\footnote{女色男風雖是一件樂事。然必須兩情相洽方有趣味。而有強奸婦女及此。聶變豹所爲有何樂處。予不知此輩是何肺肝。}。那妾同丫頭們看得毛髮都豎起來。替他害疼。又不敢上前來勸。那聶變豹笑嘻嘻只是搗。一面說道。你只當在衙門裡捱夾捱打。那難道是不疼的麼。他本有半夜的本事。喜得是初試此竅。只耍了半個更次就完了。他把陽物拔出大半截來。猛然一攮到根。忽一下拔出那個大喇叭頭子。將他臟頭帶出有五六寸來。鮮血長淌。那嬴陽先已被他搗得一陣陣發昏。眼中金蒼蠅亂冒。被這一下。疼得迷了過去。跌倒在氈上。聲氣全無。聶變豹哈哈大笑。一個丫頭忙將一塊細帕。替他把陽物拭淨。他就精赤條條坐在椅上。說道。這沒福的奴才。當日要好好的依從我。何等不妙。今日一半的也被我弄了。那妾心甚不忍。也顧不得聶變豹在面前。忙上前抱住他的頭。叫道。快取開水來。丫頭們忙忙碌碌倒了一甌水來。灌了好一會。纔聽得他哼了幾聲。微微醒轉。轟(聶)變豹道。不要管他死活。叫小廝們拉出去。撂在空處去罷。那妾道。這小子罪不至於死地。況救人一命勝造七級浮屠。老爺請安歇去。我同衆丫頭們慢慢救他。明日天不亮叫人送他回去。也是老爺的一點陰隲。聶變豹呵呵笑道。憑你。憑你。披上衣服。也不穿褲子\footnote{此時只披衣不穿褲。是樂極。異日被訪拿時。衙役只許披衣不容穿褲。是悲生。福兮禍相倚。遙遙一點(對)。}。一隻手摟着個丫頭。兩個丫頭提着燈籠。纔要走。那妾又道。老爺且請住着。這小子夠他受的了。那包東西只把鞋留下。那些首飾賞了他罷。聶變豹恨了一聲。道。便宜這奴才。說罷去了\footnote{此一段雖是寫閔氏慈心。然恃是愛妾。方敢乃爾。不然豈不懼聶變豹疑忌。}。兩句俗語說得好。

\begin{quotation}

常將冷眼觀螃蟹。看你橫行到幾時。

\end{quotation}

聶變豹初意要置嬴陽於死地以雪恨。今竟寬放了他。一來是看閔氏之面。二來實虧單氏請他去時。苦口力勸數番。故只淫毒一場。了其宿願。便寬怒了。再說嬴陽此時心中也明白了些。見這美妾如此憐惜他。心中想道。我是那裡造化。遇見這位恩人。不然這性命完了。那妾見聶變豹已去。叫丫頭將嬴陽扶到凳上睡下。叫拿個枕頭與他枕着。拿燈照他的肛門。裂做數瓣。大腸拖着。一面叫拿塊舊紬帕把血拭了。叫丫頭們替他往裡揉。又親撫摩他身上繩痕。又叫拿了杯熱酒來叫他吃。嬴陽喫不下。那妾道。你勉強吃些熱酒活活血。嬴陽却不過他的情。強呷了一口。又閉下眼。迷迷的不做聲。那妾叫拿床被來替他蓋上。約到三鼓時分。嬴陽已大明白了。只是肛門疼得受不得。身子痛得動不得。舉目看見兩三個丫頭。東倒西歪的睡着了。只那美人還坐在傍邊替他抹身上。他掉淚道。蒙奶奶救命之恩。我殺身難報了。那美人將口附在他耳上道。我與你同病相憐。我家姓閔。也是好人家女兒。已許過人家。不知甚麼人說我生得標致。他叫人到我家。說要娶我做妾。我父母不肯。他竟差許多家人搶了我來。也似你一般將我淫毒。我是個少年女兒。幾乎喪命。後來聽得我夫家同我父親吿狀。他假揑我父親賣女文書。反說我父親同夫家串通。夥騙官處。俱受重責。我今日在他家雖算第一個寵愛的。但我恨毒在心。因是女子。不能報仇。他愛你久了。幾次叫人去說。你不肯依。他恨極了。故下此毒計。前同我商議。我再三勸他不可。他大怒說。若不依他。就要拿我替你。你想這可行得。我還疑你乖覺。未必就上他的美人計。誰知你竟投在他羅網中。今逃出命來。就算造化了。又道。他家這些惡奴才。沒有一個不是幫主人作惡的。我明早叫他們送你到家。你把這個包兒還帶去變賣了將息。因拔下一根金耳挖。揷在他頭上。道。家中人若送你到了家。不曾拿你的東西去。你到家時。拿這耳挖來回覆我。若不曾送你到家。或拿了你的東西去。切不可與了來。我好追究\footnote{眞所謂救人救徹者。有智婦人勝於無能男子。}。嬴陽感恩無地。只叫恩人。閔氏起身。開了櫃子。在一個皮匣內。拿出有十多兩一封銀子過來。說道。我雖得寵。不管銀錢。頭面雖有。都有數目。給不得你。這幾兩銀子你帶去盤纏。又拿着那隻鞋道。這就是我的鞋。他前要了去哄你的。我今贈你\footnote{先向聶變豹說留下者。欲免其疑。今竟贈與嬴陽者。欲記其恨。此婦眞一斤(個)有心人也。}。嬴陽道。我怎敢要。閔氏道。我贈你。不是私情。有個緣故。你切記着。一來你今日之事。因此鞋而起。見此鞋就想今日。再不可如此孟浪了。二者你這一去。不要癡心想要吿他。我對你說的。千萬緊密。一露風聲。他知道了。你我都是死數。你做戲的人見大官府處多。看有風勵官府。將你我二人的毒害呈上。千萬救拔出我去\footnote{一片施恩熱腸。只重在此一句。}。恐你日久忘却。故贈此鞋。要你覩物思人之意。也不枉我救你一場。這樣惡人自有天報。但恐一旦玉石俱焚。連我也不能免了。說着。不覺悲慟流淚。嬴陽只在枕上叩頭道。奶奶天恩。我若敢忘了。死於千萬刃之下。正是\footnote{因嬴陽年少。不知計策。拆橋蓋房。那曉川流之過。以色成心。不知利害。}。

\begin{quotation}

惟有感恩並積恨。萬年千載不生塵。

\end{quotation}

閔氏聽聽外面已五鼓盡了。說道。你去罷。恐他醒來又有他變。遂叫醒丫頭。扶他起來。替他穿衣着褲。那嬴陽彎着腰。直不起來。站不住。閔氏叫丫頭指名叫了兩個老成些的家人進來。吩咐道。老爺吩咐叫你兩個攙扶嬴旦。送他到家。要一個憑據來回我話。那嬴陽見有人。不敢多說。跪下去要叩頭。方要跪。一交跌倒。閔氏道。不消不消。叫家人快扶起他去。那兩個人上前扶了出來。因係得寵的姨娘吩咐。不敢怠慢。問了住處。送到他門口。天已大明。二人道。送你到家了。有甚麼並(憑)據與我們拿去。嬴陽拔下耳挖。遞與道。有勞二位大爺遠來。回去時替我叩謝奶奶罷。二人接過去了。嬴陽敲門。他母親出來開了。一見兒子爬在地下。面如靑紙。嚇了一跳。儘力扶起。跌跌撞撞攙了進來。放他床上睡下。嬴陽一把抱着娘痛哭道。我同娘娘見面時再世了。若非恩人救我。也不能生回了。養氏也哭着問他緣故。他把始末原由細細說知。又在身邊取出銀子同那個包兒交與娘看。養氏忙把他褲子褪下。見他通紅的腸頭拖着。肛門裂腫。好不難看\footnote{不看東西。先看他屁股。是娘愛子之心。有先後輕重也。}。心疼得要死。一面哭一面咒。又一面感念閔氏。忙去弄了湯水來與他吃。又煩人請了外科來看。用藥調敷。足足有一個多月纔下得床。那臟頭只上去了寸餘。還有三四寸來長不得上去。醫生說。若是趁熱當時整治還收得進去。因是冷了治不得了。遂成了一個殘疾。一辛苦勞碌便淌血水。腰就疼得彎着。戲也不能常唱。只好偶一爲之。至於後庭主顧。不但新孤老不能相與。連那些舊相知看見糞門如此。但道可惜而已。掩鼻而避\footnote{聚寶盆雖然壞了。他腎運却也退了。}。且按下不題。再說嬴陽住的這一條街上。有一家姓陰的。門前開着個小雜貨鋪。夫妻二人只得一個女兒。三口過日。這女兒到了十二歲。因他長得高。像個十五六歲的身髓。就留了頭。嬌模嬌樣。甚是聰明。他隔壁一家姓關。是個住閒的小鄕宦。有兩個兒子。一個十五。一個十一。請了個先生在家敎書。這鄕宦因家寒不能獨舉。遂將左右鄰舍有子弟要念書的約了同出束脩。他家收拾了三間書館。拿家中舊隔(槅)扇\footnote{槅扇二字須記着。}。隔了一間做先生臥室。共總有七個學生。四個大的。三個小的。大的都不過十五六歲。小的也有九歲十歲。這陰老兒忽然高興。向婆子道。我家女兒生得甚好。又伶俐。何不送他隔壁關老爺學中去念書。識得幾個字。就是個全人了。你道好麼。那婆子倒知事。說道。一群男學生。把女兒送去。恐怕不便。陰老兒道。我難道不知道。女兒纔十二歲。怕甚麼。若是十四五歲。我自然不肯了。何待你說。那婆子也就不阻他。這關鄕宦時常到門口走走。間或也到他鋪中來閒談。恰好這日走來。陰老兒連忙讓坐。篩茶送上。說了些閒話。因說道。一句話正要請問老爺呢。鄕宦道。有甚麼話只管請說。陰老兒道。我有個小女。生得也還伶俐。今年十二歲。我的意思托老爺的福。想送到府上學館中。多少學兩個字兒。先生的束脩不過是意思而已。老爺說可行得麼。關鄕宦道。這是極好的事。有甚麼行不得。添一個女孩子。先生能費多少心。束脩任你。我去說。再沒有不依的。因見黃曆掛在壁上。取下來翻開看。道。好。明日就是入學的日子。你趕得及麼。陰老兒道。沒有甚麼不及的。只用買本女兒經。紙墨筆硯是小鋪中有的。明日便好了。那關鄕宦坐了一會去了。晌午時。關家一個小廝來說道。我家老爺對先生說了。叫我來說。你家姑娘只管請去。陰老兒笑道。煩你去多謝老爺。那小廝去了。陰老兒忙去買了一本女兒經。封了一錢銀做贄見茶。拿出紙墨筆硯。叫婆子拿個拜匣盛了。就把桌椅先送了過去。次早。把女兒收拾停當。親送到關家來。拜了先生。與衆學生都相見了。又煩館童帶上去見關〔鄕〕宦夫婦。那關奶奶倒愛這孩子。與了幾枝絨花。一條湖縐汗巾。然後出來念書。衆學生見這女子妖妖嬈嬈。雪白的嫩臉。鮮紅的嘴唇。黑髮披肩。好生消(俏)麗。這一個向着那個努個嘴。那個望着這個擠眼。各各含笑。他這先生三六九要去會文。又時常要去料理家務。一月只好半月在館。次日。先生不在。四個大學生同到一處商議道。這樣一塊好肥肉放在嘴跟前。要不嘗他一嘗。不可惜麼。一個道。他小呢。恐怕不知道情趣。一時喊叫起來怎處。一個道。慢慢的說法引誘他。可是硬開得弓的。須是如此如此。或者可以引得動他。一個又道。他三個小的須瞞不得。怕他們吿訴人就不好了。一個道。叫他們來。我們同他商議。遂把三個小學生也叫了來。道。陰家這女兒你們可愛麼。一個笑道。怎麼不愛。我方纔見他彎着腰在地下拾筆帽兒。一個滾圓的屁股。衣服凹進去一條溝。好不有趣的呢。一個大學生道。我們算計要弄他一弄。想看看他的是個甚麼樣子。但你們小呢。還不會幹。我們大家湊些錢與你三個。你不要對人說。等你們大些。少不得給你們嘗嘗。關二道。他方纔到後院裡去溺尿。我悄悄跟了去張。想看看他的屁股。誰知他拿裙子遮得嚴嚴的。一些也看不見。他起來了。我去看看。他的尿把地下衝了個窩兒。好不有趣。我不要錢也要看看。弄得弄不得不要管我。不然我就吿訴先生。那兩個小的也道。我同他一樣。也是要看看的。那一個大學生道。旣如此說。也罷了。須是如此去做。衆人商議定了。各回位坐下。一會兒這個去買些糖來請他。一會兒那個去買些果子來讓他。到底是女孩兒家。害羞不吃。這個道。我們同學念書。就是親兄妹一樣。怕甚麼。那個道。休說兄妹。連夫妻還沒有這樣親熱呢。他也知瞅人一眼笑笑。人兜他說話他也不答。過了幾日。熟了。也就說說笑笑。再三讓他東西也就吃些。一日。先生又出門。衆學生頑了一會。看見院子裡兩個雞打〖月戎〗。一個指着笑道。這雞打〖月戎〗。屁股對屁股。努一下子。那有什麼趣。難道也快活麼。一個道。他這樣不快活。你看那母雞把渾身的毛鬆了。那一抖。大約也像人兩口子弄酥了的樣子。一個說。到底是有㞠子的好。你看那鴨子。他有個㞠子。弄得那母鴨子快活得鴨鴨的亂叫。一個道。甚麼相干。你看狗㞠子上那樣個大疙瘩。難道弄得那母狗不快活。怎不見他叫呢。一個道。大約是快活得很纔叫。你看驢子那樣個大㞠子。弄得那草驢把嘴巴答巴答的響麼。又一個道。倒不知人弄着可叫不叫。一個道。怎麼不叫。我家隔壁的裘老大。一個金剛也似的大漢。娶了一個老婆。指着道。也只好有陰姑娘這樣大。那一日我聽見隔壁哼。我當是有人害病。在板縫裡一張。原來是裘老大把他老婆按在床沿子上弄呢。是那老婆哼。我張見他的㞠子又粗又大長。疑他老婆是害疼哼。誰知看了一會。他老婆叫道。快活死我了。哥。你再狠些快些。裘老大像搗碓似的又狠狠的弄了一會。那老婆嘴裡混哼亂叫。那個快活的樣子那裡看得。我也沒有打手銃。就把〖月戎〗冒掉了。一個笑道。我不信這話。像陰姑娘這樣大。只好同我們這樣大的人小㞠子弄。那裡禁得那大㞠子。一下子不肏壞了麼。一個道。甚麼相干。女人們生了這個屄來給人弄。那怕甚麼大。越大他越弄得快活呢。衆人哈哈的大笑。那女子也側着耳朶聽他們說。臉紅着。也不住的笑。一個道。說了這一會。好不難過。㞠子怪脹得慌。要是陰姑娘在這裡。我們大家打個手銃。賽個遠近。又一個道。陰姑娘他後來嫁了人。還見的是大的呢。稀罕我們的多大一點子。怕甚麼。一個道。不是這話。他的捨不得給我們看。我們的爲甚麼給他看。一個道。也罷。我們到屋裡去打罷。遂大家笑着一轟到屋裡去。把門掩上。這女子雖年小。心性伶俐。聽這些人說得村淫如此。他就情竇大開了。也覺得津津有味。但不好問得。見他們說去打手銃。不知怎個打法。心中想看看這物件是怎個形狀。遂悄悄走到槅子眼裡去張\footnote{先寫隔先生的臥室用舊槅扇。我疑是寫學房處多用板隔。怕太重出。故特一改。看至此。始知留爲此女張人之地。幾被作者瞞却。}。見他幾個人臉向着門外。用手勒那東西呢\footnote{畫出衆人有心引誘。}。這四個大學生裡面。有一個的竟有四寸多長。那三個都只有三寸的光景。那三個小的只得指頭大。他看得好不動火。想道。可惜人多了。要是一兩個。我就同他試試看是怎樣。他們纔說快活得很。不知是怎樣快活法兒。也將一隻手縮進袖子去。伸入褲襠中。將小牝摸摸。又拿指頭探探。不知是怎樣局面。只見那幾個勒了一會。這個冒出點漿子來。那一個冒出點淸水來了。忽然悟道。我聽見人說〖月戎〗。想就是這東西了。雞蛋黃上那一點子不是他麼。想出了神。眼定定的望着這屋內。這起小子他們雖然在屋裡打手銃。原想誘他去看。手裡打着。眼睛却射在槅子外邊。影影的見他在那裡張。忽一齊跑出來。見了他。笑道。陰姑娘偷看我們的呢。嘻嘻哈哈的大笑。那女子羞得臉緋紅。笑嘻嘻跑上位坐着去了。衆人道。我們吃午飯去。快些來。來遲了的罰五個錢。那女子先去了。衆人商議道。看這丫頭也已動心了。怎麼個弄法。關大道。人多。若齊上手。他必定不肯。等我若哄上了。你們一個個陸續上。就不怕他不依了。你們吃飯遲些來。我等他來調戲他。肯不肯。大家的造化。衆人笑嘻嘻答應去了。關大忙忙吃了飯。先來學中。那女子緊鄰也來得快。這關大安心要哄誘他。坐在一張椅子上。將陽物拿出。用手攥着。眼睛向外看着窗子。只見一個女子的影兒。知是他來了。遂口中叫道。我的好陰姑娘。弄得我好快活。好心肝。好寶貝。好嫩屄。那女子正要進門。聽得他說。打窗洞一張。見他嘴裡叫着。用手勒那㞠子。忍不住嘻嘻一笑。關大忙跑出來。一把抱住道。姐姐你救我一救罷。趁沒有人在這裡。那女子也不狠拒。被他抱到房中先生的床上。就扯他褲子。那女子道。我怕疼。關大道。不怕的。那個女子不同人弄。要疼誰還肯呢\footnote{哄得有理。}。那女子也動心久了。任他脫去。他乍見這條細縫。不知從何處弄起。低下頭用指頭〖扌扉〗門。看明了穴道。那女子閉了眼睛只是笑。他用上許多唾沫。然後對上了。向內一塞。女子道。哎呀。疼得很呢。關大道。頭一次乍弄。規矩有些疼。你忍一忍兒就好了。弄過這一次。下回就只有快活的了。我聽見人說。頭一回刀割。二回鎗戳。三回快活。你疼過這一回就好了。那女子只皺着眉。也就不嘖聲。弄了一會。關大覺得內中一噏。噏得無比受用。頃刻完帳。那女子用手一摸。看了看。說道。被你弄出血來了。關大掏出塊汗巾。替他拭了猩紅點點。那女子拿過來塞在褲帶上。正穿完了衣褲。衆學生一齊跑進來。道。你兩個幹的好事。一樣的人。爲甚麼偏一個向一個。除非都給我們嘗嘗。不然等先生來稟了。大家弄不成。那女子羞得徹耳通紅。背着臉坐在床上。關大道。你們不要着急。事好商量。衆人道。有什麼商量的。大家弄弄就完了。不然我們去吿訴陰老爺。你兩個了不成。關大道。你們且出去。我同陰姑娘計較。衆人出去了。關大摟着他道。這怎處。你除非同他們大家弄弄纔好。不然這一鬧開了。怎麼了得。女子道。都是你引的頭。關大道。生米已成熟飯。抱怨也沒用。你同他們弄弄罷。一來壓口風。二來纔得長久。這女子一來恐怕鬧得先生父母知道。二來初次乍弄。也不覺十分苦楚。後來或有樂處。也戀戀不捨。遂道。我的還疼呢。關大道。你若肯了。那裡定在今日。明日何妨。女子道。人多得很。那裡行得。關大道。豈有一齊同來的理。輪流着。或一個或兩個。憑你心裡就是了。那女子低了頭不做聲。關大道。你們來。衆人進來道。怎麼說。關大道。陰姑娘肯了。但你們不許亂來。從明日起。一日一個輪流着。或是一爭吵。陰姑娘不肯。我就不管了。笑道。你們還不謝賞呢。衆人齊笑着跪下叩頭道。謝姑娘賞了\footnote{活是一群頑皮。}。關大拉他轉過臉來。笑着道。你受他們的。他也紅着臉低着頭笑。那女子年小。到底羞愧。向關大道。我回家去着。下床來就走。關大見他害羞。也不留他。囑咐道。明日等你呢。他也不答\footnote{寫女孩。却是個女孩又羞又喜的樣子。}。回到家中。他娘問道。今日如何老早回來了。他沒得對。說道。我身上有些不自在。那娘見他頭髮亂了。問道。你頭怎的了。他拿鏡子一照。是方纔在枕上揉的\footnote{細極。此等處亦不漏。}。說道。我在先生床上睡了一會就散了。他娘也不疑他。他這一夜又喜又愧。到次早。已梳洗了要去。忽又愧心一萌。道。這麼些人。我怎麼好同他們弄\footnote{這一轉念妙甚。是個初被(破)身的女兒心事。若淫婦則不然矣。}。況且今日不知疼不疼。要只是這樣疼起來。有甚麼趣。他們都說快活。不知是眞是假。又將個指頭將小牝挖挖。與前原封大不相同。塞些進去也不知不覺。笑道。疼是大約不疼了。到底不好意思。還推不好。不肯去。他娘也不強他。早飯後。先生又出門去了。衆學生道。他今日不來。有些古怪。要是羞了不肯來。只便宜了關老大了。關大道。都是你們這些冒失鬼。捱兩日等他熟滑了。又得了些趣。再大家上就好了。纔頭一次就想都要到手。他一個小女孩子不害羞麼。這一弄塌了。大家沒戲唱。一個道。都不消埋怨。他要不來。他老官就取桌椅來了。多半是害羞。等我去說先生叫他。看他可來。衆人道。有理有理。他遂到陰家來叫。那女子想道。先生旣在學裡。就不怕他們囉唣了。遂往學房裡來。一進門。見衆人在地下頑跳。不見先生。抽身就要回去。衆人上前攔住。道。我們昨日頭都叩過。賞也謝了。你如何翻悔得。他紅着臉笑着。到位上坐下。關大走近前。附着耳上道。昨日已說明白了。講不得。你同他們弄弄。堵堵他們的嘴。後來肯不肯就憑你了。那女子此時也有些情願。但不好答得。只低了頭。關大捏了他一下。道。你依我好呢。遂向衆人道。我再三求陰姑娘。他依了。但你們怎麼個輪法。今日該誰。這個道。是我。那個道。讓我。爭個不住。關大道。你們這麼鬧就成不得了。依我一句話。我做長草兒。你們抽。長的在先。短的在後。不許再爭。若再妙(吵)鬧。我就不管了。衆人道。依你依你。關大做了草叫他們抽。那個九歳的抽了一根長的。關二是第二個。㞠子大的是第三。別的都抽定了。衆人道。還到屋裡床上去。那女子坐着不肯動。關大上前抱起他來。道。都是成日會的熟人。怕甚麼羞。將他抱到裡間床上。女子說道。不好。昨日回去娘娘問我頭髮怎麼散了。我說謊哄過了。今日頭髮再一亂。回去怎麼答應。關大道。那不是先生的梳鏡麼。再梳梳就是了。又道。我先替你脫了褲子着。那小人兒不會弄。那女子笑着。關大替他脫了。放他睡好。將他牝戶看了看。又拿指頭探探。笑道。與昨日大不相同。包管你不疼了。那女子只是笑。兩人又親嘴咂舌。頑戲了一會。出來叫那小的道。你去。那小學生進来。見女子仰臥着。忙爬上床來。把褲子褪了。那小㞠子纔有小拇指大。爬上肚子。向腿縫中戳了幾下。說道。我不會。換他們來罷。就下來出去。道。還給我幾個錢罷。我不會弄那東西。一個道。你都弄了還要錢。他急了。道。你去問問。看我弄了沒有。關二拿了五文錢給他。道。給你罷。等我去。遂進來上床。就爬上身。他却伶俐在行。用手摸着了孔竅。然後捏着陽物送入\footnote{三個小的中。寫關二却是一個尖酸伶俐小孩子。可也是壞透了的人。看他先去張這女子溺尿。並此時的弄法。便知其人。}。覺得甚是有趣。不住道。快活快活。不幾下就冒點淸水完了。那女子不但不疼。反被他戳得癢酥酥的難過。想道。這不濟。到底是大些的好。那關二爬起出來。那大學生道。你這樣快。該我了。走進房。見那女坐起要穿褲子。他上前摟住。道。且不要穿了。他們不濟。你要不棄嫌。我同你試試。那女子正未盡興。就住了手不穿。那學生知他心肯。將他放倒。取出肉具。那女子昨日張見過他。是頭一個大物。說道。你的大。比不得他們。不要冒失。他笑道。這還要你說。把龜頭上抹了些唾沫。將他兩條小腿架起。往裡輕輕一送。他那小牝纔被關二弄濕透了的。一滑就進去一半。問他道。可疼麼。女子道。影影的有些。他道。不妨事。又幾送到根。女子道。脹疼呢。他一抽一拽了一會。見那女子屁股扭呀扭的。知道有了些好光景。向他道。你要覺裡頭有些癢癢的。你拿手把我腰抱着。我好用力。又抽了幾十下。見那女子兩眼水汪汪。漸漸餳了。伸手將他抱住。知是火候到了。一陣亂抽。只見那女子面上通紅。打了一個寒噤。知他丢了。又狠抽幾下。也就大洩。那女子將嫩股向上就了兩就。他伏在身上笑問道。可快活。那女子微笑點頭。他抱着親了個嘴。要舌頭。那女子扭頸笑着不肯。他道。你不伸過來。我也不放你起來。那女子只得伸出些。被他緊緊含住了咂。那裡肯放。那女子將他一擰。他纔吐出。道。好甜舌頭。又笑問道。那小的怎麼樣來。他笑道。在腿縫裡戳了兩下跑掉了。兩人笑了一陣。纔下肚子來穿褲。那女子也起來穿了。到桌子跟前拿鏡子照着攏頭。抿完了。衆人都進來望着他笑。他低着頭也笑。那大學生對着那小學生道。陰姑娘惱你呢。說你把他的腿縫都戳腫了。衆人哈哈大笑。那小的羞得臉通紅。跑出去了。關大道。飯時了。姑娘吃飯去罷。那女子就走出來。關大送他。他道。先生不在。我不來罷\footnote{口說不來。却是要來的話。妙。}。關大道。家裡坐着也悶。不如來。大家說頑話熱鬧。弄是說定明日。今日弄不弄由你。又問道。方纔弄得好麼。那女子含羞不語。關大笑道。我同你還怕甚麼羞。他微笑着點了點頭。到家吃了飯。心裡想不來。却有些像放不下甚麼一般。由不得那兩隻脚又走了來。剛坐下。只見那兩個不曾弄的大學生近前低聲道。我們雖派定該是明日。但都是一樣同學的朋友。他們都占了先。把我兩個熬着。姑娘也心忍麼。況那大的也不該是今日的。姑娘方纔也肯依了他。爲甚麼在我們身上又薄些。我們也不敢強。憑姑娘的情罷。因叫那個十歳的道。你來。我們大家跪着求。看姑娘怎麼吩咐。那女子此時也不覺得羞了。又先得了些甜頭。想道。這事也沒有甚麼苦處。那個小的算不得數。只他兩個也不害甚麼怕。那頂大的都弄過了。何況於此。索怪(性)也弄弄。看看是怎麼樣。況且那幾個弄過的倒罷了。這三個不曾弄。相對着倒不好意思。大家弄了熟了倒好。見他三個跪着。也不答應。立起身竟往屋裡走。這三個知他肯了。滿心歡喜。輪着的這一個笑嘻嘻跟着進來了。見他坐在床沿上。一把抱了上床。替他脫了裙褲。二人就弄起來。陽物雖不甚小。只二三十抽就完了事。那女子將有些好意思。見他已不動。甚不樂意。推他道。你這個樣兒也想幹這事。那學生羞得忙忙下床。那一個來道。你這樣不濟。等我來服事姑娘。遂上床來就弄。這一個甚是在行。工夫也久。竟將女子弄丢了兩次。然後纔洩。還伏在他身上捨不得拔出。只見那個小的在床前站着道。好新鮮東西。大家嘗嘗新罷了。你一個人儘着獨自受用起來了。那個笑着拔出下來。道。讓你。他爬上來。將牝戶一摸。見濕〖氵韲〗〖氵韲〗的。笑道。哎呀。姑娘。他弄出你的尿來了\footnote{妙極。是個從未見過的小孩子。}。那女子笑笑。拿帕子揩了揩。那小的弄了進去。陰戶裡面被兩人的陽精塞滿。但覺黏黏滑滑。總不得個邊岸。那女子也毫不知覺。他亂戳了幾下。爬起道。我當是怎樣有趣。還不如打手銃受用。早知道是這樣。我也不如要幾個錢便宜些\footnote{更妙。此一段雖寫淫褻事。七個學生是七個上法。七個弄法。無一重者。}。女子也起來拭淨了。又梳了梳頭。天色將晚。大家散去。這女子得趣之後。大淸早就到學堂來。只巴先生出去。那兩個小的不算數。就是關二到底年小不堪大用。這四個大的。一日內定要輪過。念了兩年。交十四歲。陰老兒道。女兒大了。叫他不去念罷。他不肯。道。旣讀一場。索性念得多識幾個字。我便大了。怕人敢把我怎麼的\footnote{是極。難道弄得鋊不成。}。定要去。他父母拗他不過。只得由他。這兩年來。那幾個大學生都長成大漢。陽物都發了些。連那三個小的。自經破身之後。那小㞠子也都改頭換面。漸漸大些。他一遇着先生不在。任他的意思。要張就張。要李就李。一日四五次取樂。他有一種絕技。又無人傳授。是他自己悟出來的。那陰中一鎖一收。好不利害。遇着歡喜那一個。憑他多弄一會。要不樂意。只幾鎖就請他下馬。這幾個又愛他又怕他。奉承恐後。他這個快活如主母一般。豈肯撇了回去。又痛弄了一年。到了十五歲。長成一個大婆娘。不但父母阻攔。自己也覺得不好再去。只得在家。他一連熱鬧了三年。乍乍的冷淸淸獨自在家高坐。不勝苦惱。却說不出口。他生性聰明。雖同人混弄了幾千次。三年來也還識了許多字。再說這起惡少夥同奸騙了這女子。先因有利於己。故互相囑咐秘密其事。以圖久遠。所以三年之久。竟未洩露\footnote{注得好。}。今日見他不來了。知道已無所望。常於談笑之間向人道出陰家姑娘之美行。而久之傳得前後右左街坊無一不知。聞其名者。以爲如此年小便淫穢至此。皆掩耳趨避。所以擔遲到十九歲尚無人議親。陰老兒也頗有所聞。悄悄吿訴婆子。那婆子怨罵了老兒數日。道。我當日不肯。是你定要叫去。弄出這樣好名來。將來如何嫁人。此後那婆子留心。恐怕女兒在家又弄出笑話來。行監坐守。時刻相伴。夜間叫老兒在鋪子裡睡。他便同女兒睡\footnote{賊去了。關門何益。}。那女子不但被娘監住。況且淺房窄屋。便有情人也無地可做。無可奈何。日間惟有長吁短嘆。夜間則槌床搗枕。那娘明知他的心事是想女婿。不好說得。十分聽不過。數說幾句。那女子只當耳邊風。不曾聽見。再說那嬴陽自受創之後。那掙錢的臉雖仍舊貫。但那掙錢的糞門是沒用了\footnote{大約是他家風水不好。他老子有好糞門而無好臉。他有好臉又無好糞門。豈非祖宗積德未全。}。他因臟頭長拖。走路兩腿楂着。腰又有些彎。如何還做得正旦。只好在班中裝小軍打雜。或打打鑼鼓。間或分得幾分銀子。尚不足家中日食。十七歲上。他娘又死了。向來所積已幾幾將罄。三年孝滿。要想娶個妻子看家\footnote{余閱至此。不覺掩卷長嘆。嬴陽何物。尚至三年孝滿方想娶妻。世上詩禮之家。竟有父母喪中完姻者。是何心哉。}。他因自己標致。一心要娶美婦。常想道。我這樣個面孔。弄個醜婆娘來。如何相對。萬不可冒失。除非自己看中再講。偶然一日到陰老兒鋪中來買些東西。只看見一個標致女子。掀着半邊布簾同陰老兒講話。見了他。忙把簾子放下。却還拿雪白的手攥着。掀開一縫。兩隻俊眼釘釘望着他。嬴陽嘴中雖對陰老兒說話。兩眼不住睃着簾內。陰老兒把東西查了付與他。他不好再站住。只得出來。還不住回頭望。那女子也露出臉來。目不轉睛的望。看去遠了。問他父親道。這是個甚麼人。爹爹怎認得。陰老兒道。街上的娃娃。怎麼不認得。他在西頭住。唱戲旦的嬴大官。那女子就想道。好個淸秀男子。比當日那起學生強多了。我若嫁得他。夜裡摟着睡覺。便不怎麼也是快活\footnote{怕未必然。得隴之後。恐有望蜀。}。那嬴陽一頭走着。一頭想道。常聽見陰家有個好女兒。也不過說是看得過罷了。誰知這樣標致。只恐怕不是。又想道。他家並無多人。不是他是誰。他方纔不轉睛的看我。也有愛我的意思。我得恁個老婆也罷了。又轉念道。不好。我聽得人說他十二三歲就同六七個學生們混弄。是個大破罐子了。要他做什麼。又回想道。那裡有這樣的事。大約是有人惱陰老兒的。贓埋他的女兒。就是破的。怕甚麼。人家還有娶婊子的呢。我煩個人說說看。到家。過了兩日。請了街上陰老兒的一個厚朋友到酒館中飮了兩壺。煩他到陰家去求親。那人擾了他的酒。只得去說。到鋪中向陰老兒說了嬴陽求親的話。這老兒把女兒養到十九歲。從沒有人來說親。今忽聽這話。心中也喜。暗道。可惜是個戲旦。隨道。你請坐。我同老妻商量商量。去到裡邊向婆子說知。又道。論人物倒也罷了。同女兒配得過。但我家雖窮。把女兒嫁個戲旦。恐人笑話。那婆子見女兒長得大。又從沒人題。日夜見他怨天恨地。知他是想嫁人。況且自己已有年紀了。養他到那一日。說道。女兒大了。果然人品好。許了他罷。如今時年。戲子還有做官的呢。那老兒道。且不要急。事從緩來。那女子在內聽得老子向娘說嬴家來求親。喜得了不得。見老子說他是戲子不肯。心中發急。就要發話。聽得娘勸的話甚是入耳。以爲老子必依了。誰知還是活落話。不由得心裡的話從口裡攻出來。道。每當沒人來說。又抱怨養老女兒在家了。旣有人來說。又嫌好道歹的。戲子罷怎麼的。難道戲子人家是不吃飯的麼。我們崑山有一半戲子呢。難道都是沒有老婆的。我知道安心要養我做老女兒了。嗚嗚的就哭起來。婆子道。你聽麼。他旣情願。就允了罷。那老兒瞪了一瞪。心裡道。我活了一把年紀\footnote{一把年〔紀〕四字。天下皆有此口聲。余雖聞之甚熟。却不知作何解說。遇(愚)意度之。一把者。五指也。或謂五十歲之外乎。}。從不見這等老臉女兒\footnote{容或有之。}。嘆了一聲。道。看這樣子。當日人的傳言大約也有幾分\footnote{豈止幾分而已哉。}。罷。料道也沒有人要。將錯就錯。與了他去罷。遂出來道。纔與老妻商議了。旣是老兄金面來說。許了他罷。都是過日子人家。我也沒得賠送。他家也不必費事。兒大女大。將就完成了罷。那來人道。兩家體貼。這便更好了。回了嬴陽的信。喜之異常。他要圖好看。將家中所有私囊盡行取出。把閔氏與他的簪子並珍珠鑲了對冠簪墜子。換了幾件首飾。做了兩套衣服。雖不甚豐。樣樣都有\footnote{江南謂人家借債要娶妻者曰屄寬債緊。嬴陽幸未蹈此。}。擇日送了過來。那女子見了那好珠子金簪。心中暗喜道。嫌他戲子呢。只怕不是戲子還未必跟得上他家呢。到了吉期。頭一日陰老兒也還有些妝奩送去。次晚娶了來。兩人覿面。互相心愛。夜間成親。這嬴陽的厥物也還成文。工夫也還支持得住。陰氏嫁來時。以爲外貌雖佳。內才未必甚妙。只求及得上那個學生就心滿意足了。孰意更有勝焉。眞出望外。他久矣不知道羞字是怎樣解說。今得了這表裡如一的個丈夫。喜得心花俱開。這一個同嬴陽千般恩愛。萬種溫存。嬴陽原是拿定娶破罐子的。他的陽具魁偉。那陰氏當日也不過經的是輕風薄浪。又不曾生育。故四五年來身子又胖了些。他此內也甚豐盈。嬴陽只覺其緊美。不覺其寬深。見他偶然鎖上幾下。更覺有趣。又見他旖旎溫柔。足足愛到百分\footnote{男名陽而女牲(姓)陰。自然陰陽相得。如魚似水矣。}。次日起來。有許多同行中人來賀喜。又收了許多分子。請了好幾日酒。陰氏在家時。因陰老兒做人孤介。從沒親友往來。今見他家如此熱鬧。更自歡喜。夜間倍加恩愛\footnote{古云。勢利起於家庭。此更勢利起於床幃矣。}。嬴陽一連數日日間辛苦。夜間斲喪。舊病發起來了。腰疼得彎着。大腸中不住流血。動不得了。陰氏好生心疼。殷勤服事。問起得病之源。嬴陽細說前事。他感激閔氏。不消說得。把聶變豹足足咒了四五日。嬴陽過了十多日纔好了些。他這病。當日因無妻室。故不甚舉發。今娶了妻子。且又是少而美。美而淫的。可忍得住。十日半月三二日定要高興一番。高興之後。次次定要睡倒。一日。陰氏因愛他得很。違着心苦勸他\footnote{妙。在心雖違着苦勸。此道却不肯違他也。}。他那裡捨得。定要常常鑽硏。不上個把月。把一個美小官弄成個黃皮寡瘦。又睡到將及一月。纔起得來。此時方知道本草上不曾載的這種發物如此利害。纔稍減了些。我因此在本草上後添了一段。使後人見之好知避忌。

\begin{quotation}

婦人陰物一名曰牝。通稱曰屄。北人名曰巴子。閩人呼曰喞歪。川人謂之批。形如淡菜。有肥瘦大小毛光不等。雖微有小異。其形總一。性鹹有微毒。少服令人陽不亢。常服則多嗽。多服則體弱成虛怯症不治。家產者良。衏中產者雖比家產較美。然多毒。誤服有毒者。生楊梅下疳諸惡瘡。野產者味極佳。有大毒。恐有殺身之禍。病人不宜服。一切病後尤忌。服之必發。名曰色復。醉飽後服之。傷五臟。生怪病。每服後忌一切冷物。恐成陰症。反涼水。

\end{quotation}

這種物件。自古及今以至萬國九州。無人不把他當做家常茶飯。見了我這話。大約沒有一個不笑其迂者。但要明白內中的道理。自然有益而無損。譬如人參。偶然服些。自有補益。若把他當做飯吃將起來。可有不傷命者。豈是人參之過。乃服參人之過耳。此猶是藥餌。即如絕精的白米飯。噴香的細點心。以至珍饈海味。何嘗不美。一日或三次兩次。每日八分飽。自然養人。若因其好吃可口。無日無夜。時時刻刻往肚中強咽。定然要撑出病來。如酒多了害酒。茶過了害茶。飮食尚還如此。何況婦人的這件東西。世間事總不可過。同一理耳。且說嬴陽自娶了陰氏來家。捨不得撤他出門。又常有病。連戲班中都不去了。在家無事。見陰氏識字。更加歡喜。敎他念角本。他念三五遍就會。又敎他腔口。也只敎幾遍便熟。嬴陽吹笛子合他得一板不走。喜得嬴陽抓耳撓腮。陰氏也因無事。覺得唱曲甚是有趣。將丈夫旦脚風流的戲學會了許多。嬴陽向他道。我雖是正旦。那小旦貼旦的曲子我都會。就是男脚色我也會。我同你一齣齣的串了頑。遂把小旦貼旦的曲子也敎會了他好些。又將關目科白都傳授了。兩人同串。有不是處。嬴陽一指撥。他就明白。他到底是婦人的身段風流。語音嬌媚。不假造作。更自有一種可愛。嬴陽覺有珠玉在前。自視以爲不及。有幾句贊那陰氏道。

\begin{quotation}

額裹包頭。霏霏黑霧。面擦鉛粉。點點新霜。脂添唇艷。引商刻羽。啓口處。香滿人前。黛然修眉。含笑徉嬌。上場時。翠迎人面。眞可壓倒喬扮卵孫。實要妒殺時興兔子。

\end{quotation}

他夫妻也快樂了多半年。嬴陽娶他時也就囊罄了。又因害病服藥。坐食山崩。這些時陰氏的首飾衣服也陸續當了許多。漸漸不繼起來。陰氏心疼丈夫。倒也賢慧。當他東西。一絲不惜。寧甘淡薄。並無怨辭。嬴陽一日向他說道。這日子看看過不得了。說不得我還往戲班裡去混。多寡掙些回來添補。陰氏道。我難道不知道。只是你多病。如何去得。總是還有些須東西。且當着過罷。嬴陽道。不是常法。只有出沒有進。當完了怎麼樣處。還是去的是。陰氏見他說得有理。不好再阻。他從此又到班中。南邊的戲多是夜坐。常常夜間不歸。陰氏獨自。好不孤悽。一日。嬴陽出去兩夜未歸。陰氏到門口來望他。只見一個少年。也只好二十年紀。是個貴介行藏。風流瀟灑。甚是華麗。心中道。我只說我家丈夫算標致的了。誰知男子中還有這樣人物。心作此想。那眼睛由不得就到那人臉上去了\footnote{化工之筆。必至之情。}。那少年猛見一個美婦頻頻顧盼。他那眼光也釘在陰氏臉上。陰氏忽然想起在門口。恐有人看見不雅。將身子縮進些。禁不得那人十步九回頭的望。由不得身子又探了出去\footnote{寫兩人俱着魔光景。甚妙。陰氏未嫁時因望嬴陽。嬴陽回望而遂成眞夫妻。此時又望金鑛。金鑛回望而遂成假夫妻。婦人必心邪而後望人。望而兩心相同。再無不成肝(奸)者。甚矣。婦人靜坐深閨始得爲良婦也。}。那人去遠了。他纔進來。坐不多時。坐不穩。覺得那人還在街上一般。那兩隻脚不知不覺又走了出去。說也甚奇。他纔到門口。恰好那人也走到面前。陰氏心中暗道。我覺得像他來了。無心出來看看。誰知果然來了。不覺啞然一笑。他這一笑。倒也非有心勾引。是笑自己的癡情。那少年以爲他是情笑\footnote{字新。}。也笑着回頭回腦的望。一步做兩三步。慢慢走去。陰氏又回房坐了一會。嬴陽回來了。愁着眉。只是嘆氣。陰氏道。你怎的了。嬴陽道。辛苦了兩夜。掙了錢數銀子。想拿回來買些柴米。今日一個朋友家有喜事。合班邀我出分子。我娶你時又接過他的禮。臉面錢不得不出。怕你盼望。只得回來和你說聲。晚間還要去。明日又有戲。不得回來。家中柴米俱無。一個銅錢也沒有。怎麼處。陰氏道。呆子。你急急就有得錢來麼。分子是該出的。沒有柴米罷。我餓一頓甚麼要緊。嬴陽笑道。第二頓呢。我後日纔得回來。你難道就餓兩日不成。陰氏道。不論拿些甚麼。且押幾十文錢來買點柴米着。遂將頭上一枝銀耳挖拔了遞與他。嬴陽接着。嘆了口氣。去了一會。買了二升米兩束柴回來。道。押了八十文銅錢。除買柴米。這是剩的。留着你買小菜。陰氏接過收了。嬴陽道。我去了。你關門罷。明日不必望我了。陰氏關了門上床。尋思道。我家丈夫病病痛痛的。日夜辛苦掙來的錢還不夠盤纏。倘累倒了怎麼處。那眞正就要餓死了。看他時時焦愁。又可憐見的。實在也沒法。胡思亂想。忽然覺得那少年又像站在面前一般。他笑道。有了。我看那人定是個富貴人家子弟。他那個樣子倒也有心在我。我若勾上了他。倒還不愁穿吃。況且未必就把我弄壞了些兒。但丈夫恐怕嗔怪。又想。他如今也窮極了。又勞苦得很。若有碗現成飯吃。\endnotemark[1]他也落得閒閒。我看他自己多病動不得。見我靑春年少。孤眠獨宿。他也有些過不得意。我就走走邪路。諒也還不怪我。我要瞞着他做。就是我沒良心了。竟同他商議。看他如何說。他若肯依。豈不是一舉兩得。又暗笑道。我癡心妄想是這打算。那人心裡不知如何呢。且看機緣再講。想着就睡着了。到天明起來。梳洗罷。吃了飯。信步到門口看看。只見那人又來了。望着他出了神。袖子中一把扇子掉落地上。陰氏見他呆着臉望。掉了扇子都不知道。又不好說得。不由得笑着用手往地上指。那人一面回頭忙拾起扇子。左右望望無人。便走近前深深一揖。道。多謝娘娘指與我。不然掉去可惜了。陰氏忙將身子閃在門後。回了一福。那人嘻着臉問道。府上貴姓。看官且住。天地間可有無原故的一婦人一男子忽然作揖扳談起來。有個緣故。這婦人是有他的心了。故不覺望着他笑。又指扇子。明明是開門揖盜。那人姓金名鑛。他父親是科甲出身。現任知縣。家中有萬金之富。專一吟風弄月。何所不知。見這婦人兩次三番望他留情。知他心中已判了肯字。他昨日見了兩次。後來訪問人。知是嬴旦的妻子。聽說他家近來着實艱難。故今日帶了些銀子。安心來想乘釁而入。以利動他。恰有此機緣。可還有不近身的。若是婦人正顏厲色。他就膽包了身。可敢無忌憚至此\footnote{這一頓坐(挫)。妙極。不解說明白。豈不是老大落空。}陰氏答道。寒家姓嬴。那人道。我們縣中此姓甚少。有一個嬴大官是戲班中朋友。可是一家麼。陰氏道。那就是我家丈夫。那人道。我賤姓金。知縣就是我家父。嬴大官常在我家唱戲。是認得的。何必(不)請他出來會會。陰氏道。有生意去了。那人道。府上還有甚人。陰氏道。就是我一個。那人意思還要說甚麼。陰氏問道。門口恐人看見不雅。大爺請回罷。金鑛聽得他家沒人。放大了膽。便道。得遇娘娘。千載難逢的事。如何就去。外邊不雅。裡面說說兒罷。就跨進門來。陰氏抽身往後走\footnote{當說引道了。}。他回身將門閂上。隨後跟了進來。陰氏假作怒容道。我們雖係小戶人家。有個內外。大爺進來做甚麼。他上前一把抱住。道。我那前世的娘。這兩日把我的魂都被你勾掉了。來成就了好事罷。陰氏故意發惱。道。靑天白日強奸起良家婦女來。不看你是個貴公子。我〖口么〗喝起來。就了不得。還不放手。金鑛見他亂(辭)勵而意不峻。雙膝跪下。道。你若不可憐見我。我定然要思想死了。倘蒙娘娘見愛。我不敢輕慢了你。你一家衣食盤費我都供得起。陰氏一來愛了他。見他這句話正撞在心坎上。便道。我見你這樣多情。我依了你。你後來不可負心。金鑛見他肯了。忙說誓道。我若負了你。天誅地滅。陰氏伸手來扶他。他就着那一扶裡。雙手連腰抱住。到屋裡床上。先替陰氏脫了褲子。看了看。摸了摸。嘖嘖贊道。好個寶貝。又替他解上衣。陰氏道。大白日裡。穿着罷。他道。你家又沒人來。穿着衣服雷雷堆堆的。那有甚趣。陰氏只得任他脫光。他然後自己也脫了。陰氏見他陽物粗不過一圍。倒有七寸來長。送了進去。他誇道。好東西。向陰氏道。我也見了些婦人。沒有見你這又緊又乾的美物。陰氏笑笑。也不答應。原來金鑛極會鏖戰。他這陽具長而活泛無比。在婦人陰中東一鑽。西一戳。無微不到。凡婦女遇他。眞有無窮之樂。陰氏見他幹法在行。心中暗喜道。我所遇算他第一了。他與嬴陽久濶了。不多時便丢了一度。那金鍍(鑛)要逞本事。不歇氣又有千餘。陰氏久曠的人。見他陽物堅硬。幹法又強。要圖快活。不肯鎖他。一任他弄。連丢三次。意思要歇歇再來之意。說道。且歇歇着。金鑛賣嘴道。還不曾頑了一半工夫。你就想歇。等你吿饒的時候我纔歇呢。陰氏笑道。當眞麼。他道。怎麼不眞。陰氏笑道。我是不吿饒的。你不要吿饒。金曠(鑛)笑道。你要我吿饒。除非把你的這東西加些鋼來。陰氏又笑道。話要應口。嘴裡說着。兩隻手將他兩股扳緊。金鑛覺那龜頭不似先任意了。且又扳緊了。不得抽動。戳到這邊。一夾一夾的。像人拿嘴含着咂的一般。戳到那邊。亦\endnotemark[2]是如此快活難當。不到一盞茶時。一洩如注。他一把抱住陰氏。道。親親。你原來有恁個寶貝。我何福遇你。此後與你開交不得了。陰氏笑道。你還敢來麼。他道。你放鬆了。我還可以來個連拳。陰氏放手。道。你來。果然那金鑛少年精壯。雖然洩過。陽物還是鐵硬。他又如前那樣亂戳猛破。陰氏一把摟住。又是一陣鎖。不由得又洩了。陰氏笑道。說嘴的郞中沒好藥。可還敢不敢。金鑛親着嘴。道。心肝。我知道你的本事了。我吿饒罷。陰氏還摟住不放。道。我也要你丢三次纔罷。金鑛道。要說再來。我也還來得。後面日子長着呢。我有話同你商量。陰氏見說。放了手。他道。你家的今晚可回來。陰氏道。不來了。他道。這更好。我今日在這裡過夜罷。陰氏道。你是貴人。我家沒有好床鋪你睏。他笑摟住。道。天下還尋得出你這個好褥子來麼。又道。我且家去。叫小廝們送些酒菜來。我們晚上好談談。遂起來。兩個拭抹了。各人穿衣。他在袖中掏出個包兒來。道。這是十兩銀子。你且留着盤纏。陰氏接了。暗喜道。倒是個肯出手的。他道。我去了就來。陰氏送到大門內。看他去了。把門虛掩。進來坐下。暗笑道。天無絕人之路。得遇這樣個在行的人兒。已是遂心。況又多情。若得他時常照看。便是造化。又想着笑道。他的本事。要不是我。別的婦人實在要吿饒呢。知道今晚要來過夜。燒了些水。將牝戶洗得乾乾淨淨。床鋪拂拭拂拭。取出個新枕頭來\footnote{的是。新嫁未久的人。不然家中何得有此便宜之物。}。\endnotemark[3]剛收拾完。聽得外邊門響。正要去瞧。已進來了兩個小子。擡着個食盒。上面放着一罎惠泉酒。又一個小子背着一個大包袱。他進來笑道。都放下。揭開盒蓋。是十二個菓碟。六大碗菜。一對通宵大燭。都掇出來放在桌上。吩咐道。兩個擡了食盒回去。這一個留在這裡伺候。那兩個小子去了。叫這一個去關門\footnote{叫這小子去關。妙甚。筆墨毫無痕跡。不然小子在傍。二人如何調笑。粗心人不(可)看得出否。}。他笑對陰氏道。這是合巹的筵席。忙了。不要嫌不堪。指着燭道。這是花燭。不用花罷。把那包袱打開。是一床嘉錦被。一床閃緞褥子。四疋色紬。指着一個紅一個綠的道。這兩個你做小衫子褲子穿。陰氏道。多謝你的美情。留着做上蓋罷。他笑指着陰戶同乳頭。道。我怕布磨壞了這兩件寶貝。纔拿來你穿的。要上蓋。我還不會再做與你麼。陰氏笑着抖開被褥去鋪。他一眼看見枕頭。笑道。好好。我要拿個來的。不好拿得。好(拿)了草來再裝又費事。誰知你先備下了。因摟着親了個嘴。道。人說夫妻有同心。一點弗錯。又笑道。枕頭原該是女家備的。他道。還忘了一件。除下巾頭。上拔下一根金豆瓣簪兒。一根金如意。替他關在頭上。笑道。人家是先揷戴後成親。我同你是成過親纔揷戴的。陰氏笑道。你太過費了。我怎麼當得起。他捧着陰氏的臉。道。親親。我同你還要說客套話麼。陰氏也感激他了不得。也將他一抱抱住。忙伸舌頭到他口中。互相咂了一會。金鑛叫那小子來道\footnote{調笑已畢。方叫小子。妙。}。你去熱菜煮飯來我們吃。陰氏道。等我去。他那裡會。金鑛不肯。陰氏道。他小孩子家那裡摸得着。須我去照看。金鑛也隨同着到廚房相幫。舀水添柴。拿這樣遞那樣。陰氏道。你是貴人。不敢勞你。請坐着去。他道。你在這裡收拾。我也忍心去坐麼。陰氏暗喜道。倒是個多情的人。但得長久就好了。收拾完。二人攜手同到房中坐下。小子斟上酒來。掇肴上桌。不必細說。到晚。掌上雙燭。陰氏見他情厚。一心要籠絡他。歌喉婉轉。唱了一隻曲子侑酒。金鑛喜得話都說不出來。只叫。活寶活寶。囑道。你必須想法。要得長久相與纔好。坐飮了一會。金鑛情興復濃。叫撤了要睡。陰氏叫那小子在西間廚房裡睡。二人脫衣上床。這一夜雲情雨意。不消說得。次早起來。梳洗了。他問陰氏道。我這去幾時可來。陰氏道。你的厚情。我巴不得時刻相聚談。但這件事瞞不得我丈夫。遂將丈夫有病。受不得辛苦。故捨身養活他的話說了。又道。不想有緣遇着你這多情多義的人。你午後着這小人兒來討信。金鑛見他說捨身養夫。慘然道。你原來有這番好心。難得難得。同你丈夫說明白。我情願養活你夫妻二人到老。就帶着小子去了。已飯時。嬴陽回來。陰氏迎着道。今日來家早。嬴陽嘆了口氣。又笑道。命該餓死了。陰氏道。甚麼緣故。嬴陽道。今日分得錢數銀子。又扣了一個分資去了。我連辛苦了幾日。又有些腰疼。又有幾日去不得。明日定下了又不得不去。這不該死麼。陰氏道。且不要焦。你坐着再商議。嬴陽一到房中。看見床上的被褥。大驚道\footnote{此書無纖芥滲漏處。先云陰氏迎着道一句。是陰氏迎到堂屋中與嬴陽說話也。不然嬴陽入門便到房中。即看見矣。何暇更有閒談。此等處。非作者細心不能到。非我心不能看出也。}。這是你的。陰氏笑着把紬子銀子簪子都拿與他看。嬴陽道。這奇了。果是那裡的。陰氏笑道。你每常唱一夜戲。只掙得幾分銀子。我只串了一齣戲\footnote{日間夜裡該算兩齣。}。得了這些東西。嬴陽變色道。哦。是了。你見我家日子過不得了。敢串的是崔氏逼嫁麼\footnote{不得不疑到此。}。陰氏笑道\footnote{婦人偷漢。雖知無恥者。相對丈夫。暗中再無不萌愧心。今陰氏對嬴陽一連幾個笑道。身雖與人有染。此心是實爲養夫。故於心無愧。與他偷漢者不同。}。你好呆。我同你是何等恩愛夫妻。怎說這話。我串通的是曠野奇逢。嬴陽見妻子不是要棄他的話。也疑他三分是走邪路。又想道。他要做壞事。如何肯向我說。又正正經經的問道。不要說頑話。端的是甚麼緣故。陰氏一把拉着他的手。紛紛墮淚。就把如何見他多病。枉受辛苦。掙錢又不多。不足用度。恐一時累倒。兩口都要餓死。故捨身救他。又把如何得遇金公子。昨日來得一夜。給了若許東西。還許養活他兩口子的話說了。又道。你今後也不必進班去了。養養身子罷。哥哥。我實心爲你。你不要疑我是偷漢。說這好看的話欺你。我若是圖己快樂。你多在外。少在家。我豈不會瞞着你做。又肯吿訴你麼。嬴陽先也艴然。聽他說到這裡。點頭沈思道。果然。他若瞞着我偷漢。那裡去查帳。自己實在也動不得。無吃少穿。其然沒法。便道。你旣一片好心。任你罷。他還說來麼。陰氏道。他午間着小子來討信。嬴陽道。事已至此。說不得了。他若要來。我出去讓他。你對他說。但是來時。先着人來說一聲。不然兩下相遇。到底不好意思\footnote{婦女偷漢。男子當龜。初破臉時。再無沒有羞愧之心者。久之則不覺矣。但看嬴陽此時之言。並後請金鑛作別。便可知之。}。陰氏去熱了昨晚剩的酒肴來與他吃了。臨去。陰氏囑道。哥。你明日早些歸來。今日就辭辭他們班中朋友罷。嬴陽應諾去了。午後。金家小子來討信。陰氏叫請了金鑛來。把丈夫的話向他說了。金鑛心喜非常。又宿了一夜。次日回去。送了幾疋尺頭來給他做衣服。又送幾擔白米。許多柴炭之類。陰氏收了。也將前日的碗碟器皿付他拿去\footnote{細。}。此後金鑛常常來往。不必繁敍。過了數月。陰氏竟得了孕。二人便加親厚。半年有餘。陰氏陸續得過他有百餘金。還有許多衣服首飾。街坊上的人漸漸知覺。有多事的人就編出謠言歌語來唱道。

\begin{quotation}

陰家姐兒忒子個騷。嫁子個男兒又挑子個槽。

金家公子來同他子個睏。把嬴小官變子個大龜老。

\end{quotation}

數日之間。大街小巷都唱起來。向日同陰氏相厚的那些學生聽見了。氣不忿。聚在一處商議道。陰家女兒同我們相厚了幾年。嫁了嬴家。那也罷了。旣然養漢。放着我們舊情人不相與。倒去相與別處的新人。如何氣得他過。我們大家拿他一拿。就不怎麼的。且斷了他這條路。纔出得這口氣。那關二也長成了一條大漢。內中惟有他便不服氣。便在嬴陽左右人家放謠言。又約了幾個地棍不住來踩看。兩下就隔絕了。嬴陽也知道街談巷論。同陰氏商議道。這個光景。我們此處住不得了。我閒養了大半年。覺得病比當日倒好些。我又不老。還可以入班子。南京大去處。我夫妻同往那裡去。你正在靑年。又會許多曲子。要遇着個好大老官。不怕不弄他一大塊銀子到腰。說了笑起來。那陰氏也笑了笑。忽又慘然道。金大爺這一番好情。今日搬(撇)了他去。心裡覺難過些。嬴陽道。外邊些光棍踩得緊。他也來不得了。瞞了他就是我們沒良心。收拾桌菜。我去明公正氣請了他來謝他。並辭辭他罷。陰氏無奈。只得依允。嬴陽把房子先賣了。添着金鑛歷來所贈。除半年來所費之外。還將百金。算了算。儘夠途費。並到彼可以安家。把傢伙什物全寄在丈人家。陰老兒風聞得他令愛所行。也不好相留。嬴陽諸事完了。那日家中收拾下酒菜。他親自去請金鑛。金鑛有一個多月不會陰氏。正在想念。今日見他丈夫來請。坐了轎。跟了幾個家人來。嬴陽讓了進去。金鑛因他丈夫在前。不好深敍。說了幾句閒話。送上酒來。他夫妻二人滿斟一杯敬上。金鑛接了。他二人一齊跪下。金鑛忙道。請起來。我領就是了。嬴陽道。小人夫婦蒙大爺向來恩典照看。但近日街坊上口聲不好。此處住不得了。要往南京去。今日備一杯水酒。一來叩謝大爺。二來辭別。求大爺上過一杯。金鑛聽見他要去。竟癡了。兩眼望着陰氏。只見陰氏淚如雨滴。並無一言。金鑛忍不住也掉下淚來。滴在杯中\footnote{一對情種。比別奸夫淫婦一絕貪淫者。大相懸絕。}。忙把眼拭拭。一口乾了。道。你夫妻請起來。他二人叩了個頭爬起。金鑛讓他夫妻兩傍坐下。問道。路費有了麼。陰氏道。向蒙你給。還有些。昨日房子又賣了二三十兩。夠了。又問道。你們幾時起身。嬴陽道。船已雇了。準在後日早行。金鑛道。我到家就叫人送些路費來。你買小菜吃。他夫婦道。蒙大爺的恩多了。也不敢叨賞。又讓他吃酒。他道。此時心已碎了。一滴也下不去。你倒撤了開。說說話罷。嬴陽見他不用。掇到那邊屋內。陪他家人吃。明騰個空兒讓他兩人作別。陰氏見丈夫去了。忙把門掩上。一把拉着金鑛。低聲哭道。你不要怨我薄情抛你。我就在此。你也來不得了。我們且去幾年。或有相逢日子。你不要惱恨我。金鑛抱他在懷。也哭道。只恨這些奴才壞了我二人的好事。我怎肯怨你。別了你多日。我一肚子話此時一句也說不出了\footnote{至情語。非情深者不知此語之味。}。二人攜着到床上餞了餞別。多悲樂少。不能盡興而止。起來依依不捨。只得要別。金鑛悽惶上轎而去。陰氏掩門而入。這正是。

\begin{quotation}

流淚眼觀流淚眼。斷腸人送斷腸人。

\end{quotation}

金鑛次早着小廝送了十兩路費。兩隻金華火腿。十尾松門白鮝。並兩瓶醬小菜來。又送陰氏八兩別敬。夫妻二人千恩萬謝的收了。他夫妻二人又同到丈人丈母家來辭別。大家痛別一場。回家打疊行囊。次早上船而去。一路無話。到了南京店中住下。要尋個有勢要的鄕宦。投在門下做靠主。聞得阮大誠(鋮)酷喜女旦的這件道地行貨。遂送了一分蘇州土儀。拜在門下走動。就在他家左近租了兩間房子住下。過了三四個月。陰氏生了這個女兒。因他潔白如玉。故此小名皎皎\footnote{古詩云。皎皎河漢女。此名皎皎者。謂嬴陽與金鑛所生。不知何漢子之女耳。}。閒過了年餘。資囊坐食將罄。嬴陽只得入了一個蘇州班內做戲。南京城中戲班更多。生意更有限。掙不出錢來。夫妻商議。陰氏竟入班做了一個雜旦。他不唱正本。只做些雜齣。他姿色旣好。唱得更好。又風流又騷浪。還有一種驚人的技藝。專會替這些公子們或財主大老官箍肉棒槌。因他這種絕技着實動人。人贈了他一個雅號。叫做滿床飛\footnote{滿床舞或可比。曰飛。不知如何飛法。}。嬴陽也不做戲了。只帶領皎皎或班中相幫打雜。阮大鋮酷愛陰氏。白擾了他胯下那件美物也不計其次。一文纏頭之費也捨不得相贈。自己過意不去。他雖品行不端。却有些才名。又相與的人多。替他四處推揚。逢人說項。所以不幾年就掙了二千餘金。他做了戲子中一個暴發戶財主。有些體面。就不肯做這兩樁舊買賣了。置了百餘金一所小房。小小一間。大門進來。前院正房三間。一間堂屋。東一間收拾做客座。西一間做臥室。後院中一間廚房。收拾得十分潔淨。他學做淸客。琵琶絃子。笙簫管笛。掛了滿壁。牆上貼了許多蘇畫。桌上擺設些蘇鑄香爐宜興壺。建窰瓶揷些花。宣磁盤放幾個香櫞佛手木瓜之類。雖是不甚値錢的玩器。倒也熱熱鬧鬧。半雅半俗。他做戲的人。喫慣了這家茶飯。却不會做別的生意。恐坐食山崩。想了一個妙策。請向來同他阿政相契厚的這些公子財主們。內中有好賭者來家中賭博。他在傍拈頭。那陰氏會整理得上好肴饌。絕精蘇碟。款待來客。甚是豐盛。時常他也在傍揷趣。那些嫖過他的人。背了他丈夫的眼。也還親嘴摸胸的頑耍。又還有很親厚的。就是那要緊去處也許撫摩撫摩。但只輸嘴不輸身。故此引得這些人眼中火出。不住時常來往。頗不寂寞。所獲之錢。除日用之外。尚有餘剩。因家中無人買辦物事。央了隔壁姓龍的人家一個兒子名叫龍颺。來家中使用。認做乾兒。每常也幫貼他些須衣服盤費之類。那小廝的父母貧窮愛小。得他些週濟。也落得叫兒子相幫。這猴子不但希圖替他家買辦可以落錢。且日日可以肥嘴吃。連夜間就在廚房裡打個鋪睡。竟常在他家不回。混了幾年。他這女兒皎皎不覺年已十五。打扮得花枝一般。兩道水鬢描得長長的。一雙金蓮裹得小小的。粉森森一個白臉。紅通通一個嘴唇\footnote{先贊陰氏也是此二句。可謂是母是女。}。好不俏麗。戲子人家女兒何所不知。況他幼小時。母親時常同人肉麻。間或落在他眼裡。如今大了。漸漸知覺。夜間知覺他父母的床鋪在前邊。他另鋪一張小床。做丁字樣在床後。他父母夜間或有動作。以爲兩床相隔。又都有帳子。不甚防他。孰不知他父母的床在外。迎着南窗上的亮。他在黑處。又隔不遠。且又都是夏布帳。他父母雖看不見他。他却看得明明白白。徐疾動止。抽拽簸攧。一目了然。且高興中那一種聲息。他父母恐女兒聽見。自然要忍住些。孰不知到忘情之際。男子喉中之喘。婦人鼻內之哼。不知不覺就露出來了。自己反不覺得。却被這妮子聽了個滿耳。看了個滿眼。到了下邊那漰湃乒乓之聲。那時連忍也忍不得的。皎皎聽了。雖不知何故有此響聲。他自己聽得難忍。那不曾開闢的牝中也有些淸水流出。他也就悟到了幾分。請想。這樣聰明女子。又十五歲了。情竇大開。可有個不動心的。他在夜間或一時聽得很難過。也拿個指頭在小牝中摳摳挖挖。並不覺有甚妙處。他暗想了一個主意。相機而行。他父母因有了幾個錢。要圖臉面。倒也拘管得甚嚴。十二歲時。陰氏便不許他見人\footnote{妙。回憶當年。自己是十二歲被衆生引誘也。}。但有人到他家來頑錢。都在東屋。叫他倒關着房門坐在西屋裡。人雖知他有個女兒。却不得見面。皎皎因不得見人。不過時常在窗洞中往外張張而已。要往後邊去。他屋後還有一小門可通。連堂屋都不消走得。皎皎他久矣看上了這龍家小子。要想同他權且暫爲夫婦\footnote{奇談異想。}。以免怨女曠夫之急。只因不得其便。有其心而無其地。時常對着那小子瞟眉撂眼。犯嘴撩牙。做出那些假笑眞顰的浪態。那小子十三四歲時就被人騙做龍陽。如今十七八歲。何事不知。他也想算計這女子。因恐他爹娘知道。打脫了這肥主顧。不但蛤蜊肉不曾嘗得。反把這現在的殘湯剩水。雞鴨脚。魚頭肉屑。都不得吃了。豈不可惜。二來年幼。到底膽小。不敢下手。恐設或變卦。叫喊起來怎處。無巧不成話。一日。他家中無人來賭。他父親出門去了。他母親閒着無事。在房中睡午覺。皎皎偶到後院中來走走。也未必出於無心\footnote{誅心之論。此即前所想的主意。相幾而行者也。}。見那小子背着臉小解。他明明知道。一心要看看這金剛鑽的形狀。故做不知。忙走上前用手搭着他肩頭。笑着道。龍家哥。你做甚麼呢。那小子回頭一看。見是他。因尿尚未完。只得一把攥住。笑嘻嘻把嘴挨着他嫩面上。道。你猜猜看。皎皎笑道。你拿着甚麼東西。與我看看罷了。猜的是甚麼\footnote{騷極淫極。}。這小子是靈透心的。見他撑岸來就船。可還有推辭的理。放了手。將兩個指頭捏着向他。道。請看。是這麼個活寶貝。他因尿未撒完。脹得挺硬。一跳一跳的。又冒了一股尿。皎皎笑嘻嘻的道\footnote{樂哉。虛度十五。今日方得細覩此異物。}。好個硶東西。光頭光腦。又紫又黑得難看。這小子道。我的硶。你的必定好看。我旣與你看了。你的也與我看看。拉住了他。伸手就扯他的褲子。皎皎假做不肯。道。我叫喊呢。看我娘娘來看見。只是口說。却也手不推。脚不走。那小子知道他父親不在家。母親睡覺。那裡聽他。拉開了褲腰。一伸手下去。摸着了又光又嫩的東西。鼓蓬蓬的。上面一條細縫兒。低頭一看。不覺魂消。有一個黃鶯兒贊他道。

\begin{quotation}

兩片肉蓮蓬。小花心吐縫中。光光乍乍形如蚌。奇珍易逢。名花易逢。羨他此竅誠難夢。鼓鬔鬔。想嘗異味。須得入其中。

\end{quotation}

他情急了。摟着親了兩個嘴。道。親親。你不嫌棄。我們到廚房中我的鋪上試試看去\footnote{到廚房裡去。一個嘗蛤蜊。一個嘗棍子魚。正是地方。}。皎皎道。不好。恐一時娘娘醒來怎處。倒在夜間。我將後門虛掩着等你。等爹娘睡着了。我開了放你進來。兩人約定。又親嘴咂舌。肉麻了一會。方纔走開。到夜間。皎皎果然悄悄的把他引進房來。上床弄起。一則龍陽此物甚微。二則皎皎雖未經弄過。却時常摳摳挖挖。也非原封的了。雖微覺有些疼痛。恐父母驚醒。只得隱忍。事完了。又悄悄出去。二人得了這甜頭。遇便就偷。却提心吊膽。再不能暢快。他二人暗地商量道。我們夜裡做這件事。就像做賊一般。心是拎着的。一點趣也沒有。設或被爹媽知道。弄得就不好了。此後等有人在家耍錢。爹爹拈頭服事是時刻不離。娘娘在廚下收拾酒飯。你悄悄到房中來。方可放心取樂。約明了。但是夜間有人來賭。就把小子約了進房。關了門。方得放心大弄。他母親若敲門。他故意遲延。假做睡醒的模樣。半晌纔來開門。那小子已悄悄開了前門去了好一會了。偷得次數也多。不必細說\footnote{此書寫婦人淫亂之罪。報應俱有輕重。即如此二回內。陰氏之淫。初爲衆學生所誘。後因嬴陽有病。捨身養夫。其罪可原。故始終未遭淫毒。至於皎皎。是他先誘龍颺。設計私與相合。皆出自他。後又與鄔合。跟了緣逃走。故受創幾斃也。}。又過了年餘。嬴陽見女子大了。央媒要尋女婿。他因有幾個臭錢。就忘了是戲子出身。且不止於戲子。便出了個大題目\footnote{題目雖大。不意後來文章竟是小作。配了一個幫閒。}。說道。我如今相與來往的都是財主公子\footnote{此語可謂借光。是令政的後(厚)友。何嘗是你的。}。有體面的人。白衣人如何做得親家。須要宦家門第。或詩禮人家。又要家當過得。可來說合。你想這正經人家子孫可肯與他做女婿。小戶人家來求。他又做身分不肯。因因循循。又過了年把。皎皎已十八歲了。他母親忽然見他胸高腹大。吃了一驚。關上房門。拉到床上。解開胸膛一摸。將圍腰扯開。只見兩枚滾圓的大乳突的跳將出來\footnote{語趣。}。倒嚇了陰氏一跳。再用手一捋。乳汁直冒。又伸手將肚子一摸。已鼓蓬蓬的墜了下去。將近要生外孫了\footnote{更趣。}。急得那陰氏將女兒擰了幾把。問他緣由。他倒反使性子哭道。你問我。我知道嗎\footnote{答得妙極。楚人云。昭王南征而不復。君請問諸水濱。却得甚是乾淨。今皎皎道。我知道嗎。意思爲腹中之物你可問之陰戶。何得問於我。也同一意。}。陰氏怒道。沒廉恥的小騷奴。你還強嘴。你不知道你肚子裡的私鹽包是那裡的。追逼得沒奈何。他纔細細供出。陰氏方知女兒腹中是龍家小子的種。氣了一個發昏。料瞞不得。只得吿訴丈夫。那嬴陽第一是怕張揚出醜。二來恐傳了出去女兒不好嫁人。忍了一口氣。尋了個事故。將龍颺好好辭了他去。急急贖了兩劑打胎藥與女兒吃下。誰知這野種比家種分外下得堅固。輕易不肯下來。沒奈何。等到月分滿足。肚疼了一兩陣。呱的一聲。養了一個白胖兒子。人家正經妻子坐產好不煩難。惟有這樣娃娃生得好不順溜。那陰氏忙忙把小孩子撂在淨桶中蓋上。同丈夫到後院暗暗埋了。推說女兒有病。臥了一月。方纔起來。這回嬴陽見女兒做出恁場把戲。再遲不得了。又叫將媒人來說。但是略斯文些。有碗飯吃的人家。也就罷了。也不爭一絲財禮。事成厚謝。恰好鄔合也\endnotemark[4]央媒人尋親事。媒人就提起他來。嬴陽素常在大老們家走動。也見過他。人物也還乾淨。年紀又不多。連鬍影還沒有\footnote{此一件大有可疑。}。一說便允。媒人向鄔合說了。鄔合一個做幫閒的人。比戲子也高貴不多。那管這些。見不爭財禮。且有賠事。歡喜非常。將就行財下聘。擇日娶了來家。他家住在一條死巷內。甚是淸靜。左右不過三五家。那鄰舍都是小買賣老實人。他家有間獨院。二間房子。一間隔做兩截。前半做客位。後半做廚房。有一個小門。後邊一個小院做毛廝。那一間做了臥房。做幫閒的人連衣帽都要用香薰透了的。何況房中不乾淨。雖沒甚富麗。床帳却也收拾得一塵不染。嬴陽因有心病。賠女兒也還豐麗。床帳箱櫃。樣樣俱有。且又是個獨女兒。內囊中衣服首飾也都有些。鄔合喜出望外。娶了嬴氏進門。丈人是外鄕人。無甚親戚。他自己也沒甚親友。淡然而已。這嬴氏正同龍小官打得火熱。忽然被母親識破分開了。如小孩子乍斷了奶。好不難過。沒奈何。淹心的苦咽在心裡。今聽得嫁人。這場喜歡不小。只望那一晚到了他家。安排一場潑戰。又聽得媒人說新郞是三十來歲的人了。自然比龍家小子二十來歲的分外雄壯在行。且另試新物。以廣見識。以暢心胸。不想到了夜間。那新郞官至誠得很。只把上蓋衣服替他寬了。放他睡下。還等他來解帶子脫褲。少不得要假做些新娘腔調。誰知新郞竟不動手。也自脫衣而睡。心中還疑是今日辛苦了。必定稍俟養精蓄銳。大動干戈。心下慮着。恐不能支敵。爲新郞所笑。竟有三分畏怯。等到半夜。孰意這新郞是讀盡魏史的。學羊祜陸凱守邊之法。各保疆界。不但不來交鋒。且並不來答話。只急得眼耳口鼻中慾火直冒。幾乎有個焚了祅廟的樣子。下面淸水長流。恐怕把新褲新褥濕得斑斑點點。不好意思。死命夾緊。那知這個作怪的眼子越夾得緊。越喞出來的更多。竟像黃河倒了壩。輕易再堵他不住\footnote{自從娶了嬴氏進門起至此。無一處一字不令人笑倒。我亦不能贊。只拍案叫絕。大笑而已。}。一夜到明。目未交睫。新來乍到。又不好問得。次夜仍復如此。是不知黑洞洞葫蘆中賣的是甚麼藥。猜詳不出。過了數日。顧不得羞了。盤問起來。只落一聲長嘆。兩淚交流。你道是何緣故。原來這鄔合是個天閹。沒有陽物的。有調黃鶯兒贈他道。

\begin{quotation}

這物太稀奇。體雖雄却是雌。腰中並沒風流具。腎囊太巍。玉莖太微。怨爹娘少下些兒費。漫驚是天閹是號。上下兩枚臍。

\end{quotation}

就如太監一般。他本來不打帳娶妻。所以獨處到三十來歲。因他數年來做這幫閒買賣。不費本錢。只用屈身利口奉承得大老官歡喜。不但有吃有穿。銀子還成大塊掙了來。蒙他有了這小小家業。終日在外無人照管。旣無親人可托。要約個人來做伴又不放心。他要尋個妻子。初意如搭伴修行一樣。若人家有嫁不出的石女兒更妙。倘尋得着。這就是天賜姻緣了\footnote{天生石女焉知不是與天掩(閹)若配者。但恨不能恰巧相女(合)耳。}。萬不能有這般巧事。就是年大些的寡婦也罷。他是嘗過滋味的人。或不在此道上做工夫\footnote{或恐不然。}。便是四五十歲的也情願。要不過借個夫妻名色。原不求生兒育女。只煩他看家而已。或是窮家小戶女兒。他在家無穿少吃。娶了他來。拚着費幾個錢給他好的穿好的吃。他尚未經歷其中滋味。即如在家老女兒一樣。或可相安。起初原不過是這幾個主意。都對媒人說過的。不想媒人只圖兩家成事。好二姓索謝。那管男女死活。就總成了他這個奇貨。是個久在行。連娃娃都養過的後婚女兒\footnote{後婚女兒。奇稱。}。他先也只說一個戲子的女兒。不過是將就人物。誰知是這樣個花朶般俊龐。他一見時。心中也有些老大愧悔。暗暗跌脚。將來這一頂簇新時款的綠頭巾。此頭恐不能免。却沒有多送回去的道理。又不好先呈履歷。今見嬴氏問他。這可是瞞得過去的。赧顏假笑。只得合盤托出。滿心以爲他是個女孩家。還未必在此事上做工夫。豈知這嬴氏就如一個善啖的大肚漢。餓了許久。今日滿擬來吃飽飯的。不想倒從新絕起他的飮食來\footnote{妙譬。}。你道苦惱不苦惱。他聽了這話。不便高聲。暗暗哭了兩三日。那鄔合自知不是。他是奉承人的慣家。百般溫存。十分愛惜。嬴氏雖然下口沒得繕(鱔)魚吃。上口却每日有肥雞臘肉。美酒佳肴的受用。況且吊桶已落在井中。無可奈何。又見鄔合趨奉得十分到家。不但連馬桶替他去倒。他蘇州人最愛乾淨。每晚定要洗洗下身纔上床。鄔合一到日黑。就去掇一脚盆水來。只等他一褪了褲子。蹲下連忙就替他洗淨。用塊舊紬帕輕輕揩拭。猶恐重了擦得他疼。間或天冷。嬴氏夜間要小解。他怕淨桶冰了。他忙先下去坐在上面。等溫暖了。纔扶嬴氏下床。又怕他熱身子冒了風。把脊心替他拍拍。等尿完了。方扶上床\footnote{荀奉倩的門生。}。至於日間掃地鋪床。燒飯煮茶。像活菩薩一般供養。除非有事出外方罷。嬴氏見他這樣週到相憐。倒也換出一點好心。過了幾日。性氣癱了。也好好起來。恩恩愛愛過日子。把個鄔合喜得屁滾尿流。別人看着他是一對好夫妻。誰知竟是兩個乾兄妹。且按下一邊。却說那嬴陽自從女兒嫁出。兩口子捏了一把汗。他的着數都已排定。若是女婿試出女兒是個破罐子。有甚口角。拚着與他兩百銀子討小買和。不想女兒嫁出。女婿文雅溫柔得很。竟無一言半語。他夫妻不勝歡喜。兩口子暗地猜詳不出。\endnotemark[5]陰氏說。想是女婿的陽物大得過當。故不覺女兒是已經風雨的。嬴陽說。各人的毛病各人知道。大約是女兒伶俐。善於遮飾得好。故此不曾露出馬脚來。再不然。女婿雖然年老。於此道中或者不曾歷練。被他瞞過了。總想不到這位佳婿雖是男子。下邊是替女兒一樣的毫無陽氣。竟不曾試得。再說這龍家小子自從在龍(嬴)家出來之後。也知是皎皎露了破綻。攆他出來。敢怒而不敢言。先還癡心妄想。他女兒肚中現懷着我的種。就盤問出來。怕有醜聲。或者就嫁與他也不可知。每日呆着望信。打點好做他家的嬌客。不想隔了些時。竟嫁與鄔家去了。一腔悶氣如何出得。眞叫做老羞變怒。這小子十三四歲時曾跟着游混公念過書。游混過(公)自宦蕚家出來。開了個散學館。那個無品的人。他愛這小子生得乾淨。背不得書也不打。寫不得字也不罵。他倒暗地與他錢買果子吃。把他吃厚了。就把他一個囫囫圇圇的後庭。替他開出一條大路。後來有幾個大學生知道了。吿訴他父母。打鬧了一場。將兒子叫回。游混公的館也就自此散了。這名一出。誰家的父母肯把孩子送來從他。這小子自下了學就在嬴家幫了這幾年。不曾去看得這位大花子先生\footnote{自有先生以來。未聞〔有〕此〈有〉奇稱也。}。他在嬴家時。每日有得好的吃。又有錢落。七八年來受用慣了。今回到家中。頓頓一碗糙米飯。熬靑受淡起來。心中如滾油燒的一般難過。要想修修舊業。因想。雖有幾個孤老。總沒有先生當日這一番相愛。因此到游混公家訪故。游混公鰥居久了。正用得着他。且是故人故物。更加親厚。這小子常到他家以股換食。這叫做以其所不愛易其所愛。這日。偶然在路上遇着了游混公。撒嬌撒癡。拉着問他要酒肉吃。游混公正同他相厚。推却不得。同他到了一個賣肝板腸的鋪子裡。又粗又肥的腸子。炒了一大碗。要了兩壺燒酒。痛飮了一番。費了游混公靑銅百文。這游混公怎肯容他白擾了輕輕的放他去。帶他到一個荒園中一間毛廝房。將他後庭着實盤弄一番\footnote{毛廝房內正是做此事的去處。}。纔放他回來。這小子上下都飽足了。欣欣得意而歸。剛到嬴家門口。有幾個街坊上的閒人站在那裡說白話。衆人見他醉醺醺走來。問道。龍小官。今日在那裡吃得這樣春色滿面。他倚酒三分醉。答道。今日人請我吃酒消氣。故此多了幾杯。內中一個笑道。騷鬍子膀胱氣。你有甚麼氣消得。他道。一個老婆被人占了去。還不氣麼。衆人都只當他說笑話。又一個合他笑着頑道。你的老婆在丈母娘腿肚子裡轉筋。還不知養了沒有。如何就被人占人(了)去。衆人都笑了。他又道。我的老婆連孩子都養了。還說養了不曾。又一個道。你的孩子呢。他道。我的孩子被丈人丈母弄死了。又一個笑道。你丈人姓甚麼。在那裡住。爲何弄死你的兒子。把你老婆怎樣了。他就指着嬴家的門道。這不是我的丈人家。他嫌我窮。故把我老婆嫁與鄔家去了。內中一個老成些的人喝道。小孩子家吃了兩杯酒。嘴中胡說亂道的。他道。老爹。我酒在肚裡。事在心裡。怎麼胡說。一個酒吃在人肚裡。難道吃在狗肚裡不成。你老人家不知道我們的這些彎兒帳。他從小認我做乾兒子。就是要我做女婿的。親口把女兒許過我。他女兒知道同我終就(究)要做夫妻。就預先合我好了這三四年。今年有了七八個月肚子。見我家窮。倒把我攆了出來。把女兒另嫁了人家。衆位老爺如果要不信。問那忘八可敢出來說話。我有本事到他後院裡挖出小孩子來。若沒有眞贓實據。把我舌頭割下來。再不然。我把他女兒渾身的上下是怎個樣兒。同屄是怎樣的。我說了。叫他當着人把女兒剝光了。看我有一句說的不對。憑着把我怎麼處治。這沒良心的女兒的屄。差不多被我肏鋊了。從新反悔起來。衆位老爹請想。叫我氣不氣。衆人見他說得鑿鑿可據。倒不好意思。大家含笑散去。這小子也回去了。衆人說話的時候。那嬴陽正開門出來。要往別處去。聽得有人大〖口么〗小喝的高談。他且不開門。站住了聽。原來是龍家小子述他女兒的美行。氣了一個直挺。本要出來打他。恐怕小子越發胡言亂語。更不好意思。要經官動府。又怕牽連着女兒。忍着氣回到房中。細細吿與陰氏。夫妻商良(量)道。這個醜名一張。此處如何還住得。有甚臉面見人。不如作速搬回家鄕。我們有這些家私。儘可過日子了。女兒不成器的東西。撇了他罷。倘或偷雞的貓兒性不改。在人家再做這些醜事出來。越沒顏面了。趁早去的是\footnote{在家因有妻子醜聲而來。在此復因女兒醜聲而返。絕妙文法。}。商量已定。把房子並器皿傢伙全賣了。雇了船。臨行時方來辭阮大鋮。到了他家門首。看門人傳了進去。出來叫他入見。嬴陽見大廳上結綵懸花。肆筵設席。鼓樂梨園。許多人在那裡。阮大鋮正在支派家人收拾。嬴陽上前叩頭。稟道。門下離鄕久了。如今要回家去。特來叩辭老爺。門下荷蒙天恩護庇十數年。今來叩謝。後來稍有長進。再圖報大恩罷。阮大鋮向日白受用了陰氏的美牝多次。歷來四時八節。又常受他些孝敬。今聽得他要回鄕。要賞他路費。少了拿不出。多了又不捨。若一毛不拔。又覺過意不去。躊躇了一回。猛然想起。說道。你回去。我一點東西沒得賞你\footnote{先說這一句。妙極。含蓄着下意。}。你向日求我說那姓聶的話。我常常在心。遇不着一個好可托之人。我今日請新按院鐵老爺。他是個鐵面無私。敢做敢爲的漢子。又是我同年。你在這裡伺候着。說話中得便。我托托他看。他若肯替你報了這個仇。也不枉你在我門下一場。他依不依。這看你的造化了\footnote{說此者。鐵按院准了。自然自己居功。設或不依。非我不盡情。你沒造化耳。將奸巧人心腸描盡。}。嬴陽忙又跪下叩頭。道。門下蒙恩多了。要再蒙老爺替門下報了仇。門下粉身碎骨也報答不盡也。阮大鋮道。你起來。這鐵老爺他衙門事多。不得來。我再三去請。他却不過。纔允了。大約也就到。你等着。原來這鐵按院。他雙名鎭惡。乃建文時忠臣鐵鉉之後。燕王大殺靖難諸臣時。鐵公有一妾。腹中懷孕。他夫人托這妾的父母帶他遠逃。後來把鐵公二女發了敎坊。查拿他家屬甚緊。他父女逃到陝西延安府住下\footnote{延安府。妙。謂到此方安然得延忠臣之嗣。}。後生了一子女。鐵鎭惡就是他嫡派子孫。他生性忠直。大有祖風。不避權貴。眞是個鐵面御史。姓鐵。他那性情也就是一塊生鐵。他素鄙阮大鋮爲人。故辭席不赴。因他再三敦請。却不過年誼。只得來走走。來到阮家。阮大鋮冠帶出迎。嬴陽遠遠看他烏紗豸服。一臉殺氣。令人望而起畏。到廳禮畢坐下。阮大鋮道。多承老年臺不棄。弟叨光多矣。鐵按院道。弟非敢過辭。實因敝衙門事繁。承老年臺厚意殷殷。不得不撥冗赴召。看見戲子桌席。說道。弟先吿罪。實不能久坐。梨園\endnotemark[6]可以不必。也不消在此坐。移一席到書房中。我二人促膝談一談濶悰倒妙。阮大鋮道。一巵魯酒。原不足敬老年臺的。久不相晤。奉屈少敍。以盡弟之鄙敬耳。鐵按院道。不敢。承老年臺如此過愛。弟心領就是了。你我年家至契。何必拘此客套。弟之鄙性。薄奢華而敦儉素。老年臺所洞悉者。在書房中知己談心。還可多坐一會。若必欲在此。弟先吿過三杯之後即吿別了。阮大鋮知他是個拗性的人。只得道。旣承尊諭。敢不如命。旣然如此。就請到書房中寬坐罷。讓到書房中。請他寬去官服。然後安坐。二人飮酒。閒談了一會。阮大鋮道。老年臺按臨南直。這些黎庶皆得蒙覆載之恩了。鐵按院道。弟雖不敢自謂欲澤民爲堯舜之民。然一片鉏惡之心。欲爲民除害。雖夢寐不忘。即權貴之家。弟亦不懼。拚此一官以救百姓。捨此一身以報朝延。上不愧祖宗。下不負所學。此弟之素志\footnote{這些話向〈旣〉阮大鋮說。眞如冰炭。}。弟辭朝之時。把功名二字已付於度外了。但恐耳目不廣。或有漏網吞舟者。則負弟之初心耳。阮大鋮乘機道。這是實言。如大奸大惡。他上下皆有線索。互相蒙蔽。代爲隱瞞。一時如何查訪得出。即如蘇州府崑山縣巨惡聶變豹。戕殺人命爲兒戲。奪人妻女。占人田產。無惡不作。且大肆淫毒。一縣之民爲所魚肉幾盡。歷過多少代巡。他尚安然無恙。即此一端。便可槪見了。鐵按院道。老年臺何以知之甚詳。阮大鋮道。受害之人屈指難數。因指着嬴陽。道。此人即其一也。鐵按院道。此是貴紀綱麼。阮大鋮道。不是。他夫婦受害。幾至喪身。避難到此。猶恐他追求。投在弟門下爲之護庇。今十數年了。他思鄕念切。欲返故園。適間來辭。弟因老年臺談及奸惡。弟偶然想起他來耳\footnote{其人則非。其言甚巧。說得毫無痕跡。}。鐵按院問嬴陽道。你受過他甚麼害。他作惡如何。你不可妄爲加減其辭。若果情眞。本院自有公道。嬴陽忙走過。叩了頭。跪稟道。蒙老爺下問。小的敢有一字涉虛就是欺天了。小的名叫嬴陽。祖籍崑山。小的有一個表姐關(閔)氏。生得頗有幾分姿色。自幼曾許過人家。聶變豹他家這些惡僕。專一在外替主人探聽得美男俊女。肥產良田。就去報知主人。以圖功賞。就將小的表姐報他知道。他着人來說要了做妾。小的母舅不肯。又不敢得罪他。婉回已許過人家了。不然敢不遵命。他遣了二三十個惡奴。公然搶去。小的母舅約同親家吿到縣中。他反假寫小的母舅賣女文書。買出硬保。說小的母舅串同光棍誣吿圖騙。反受重責枷號。至於小的受害。事屬鄙穢。不敢上稟。恐汚老爺金耳。鐵按院搖頭道。不妨。只管說。他又叩了一個頭。哭起來道。小的今日得在老爺臺下訴寃。也是再生了。小的少年時生得略似人形。他不知如何知道。忽然一日。他家着了一個人來對小的說。你家姐姐約你去說話。恐你不信。這是你姐姐頭上的簪子爲據。此時小的又不知道表姐的死活存亡。聽得有信來叫。歡喜不盡。那裡還思前想後。二來少年孟浪。就跟了他去。領進內室。叫小的等着。他說去叫小的的表姐來。等了片刻。聶變豹帶領多人將小的拿住。搜出簪子。說小的是賊。剝光綑縛在一間屋中。小的的表姐聞得。奔了來哭救。悄向小的說。這惡人想漁男色。昨日他家人說你標致。故設此計騙你來。你若不從。就不能生出此門了。你忍受他一場淫毒。或天可憐見。逃得性命。我姐弟二人將來此仇或可有報復之日。倘你不幸而死。我報仇無日。你此來因我而死。我決不偷生負你。鐵按院笑道。這件事南人皆以爲常。爲何你說得如此利害。這就是挾仇的誑語了。嬴陽又叩頭道。小的敢有一字欺天。罪該萬死。他有名叫做聶驢子。這些娼妓不幸遇他尚還啼哭不禁。少年女子爲他所淫者。十存四五。還俱帶疾。何況男人。小的那時不能自主。尚圖一線之生。只得依允。他好狠。將小的綁在凳上淫媾。將小的臟頭帶出尺餘。至今尚拖數寸。老爺不信。求差人驗着。彼時小的已經死了。虧小的姐姐救了半夜始得復生。小的醒後。姐姐哭說小的死了的時候。他叫家人拉出去撂。是小的姐姐再三求吿。纔留得性命。次早買囑他兩個家人。送了小的回家。鐵按院問道。你表姐在他家作何項下。就可以自主救得了。嬴陽道。小的表姐悄吿小的說。初到他家時。聶變豹恨小的母舅抗拒。將小的表姐淫毒。也意欲主意死地。徼倖不死。又幸虧有幾分姿色。他還有絲毫憐惜。命人扶養數月纔好。後來竟得他專房之寵。所以纔救得小的。小的姐姐又说。我之不死者。豈貪他之豪富寵愛。他拆我父子。分我夫婦。且我父翁皆被他陷受官刑。我與他之仇不共戴天。養此身。忍辱報仇耳。鐵按院點頭道。果如你說。這閔氏也還算個好婦人。嬴陽又道。小的表姐又囑小的道。你逃出命去。萬不可想要吿理。不要講府縣。雖撫按衙門也是無用。倘有不妥。我姐弟二人命都不保。皆做負屈之鬼了。你可到南京去。或遇有鐵面無私的上臺哭吿。或可除恨。小的含忍多年。今得見靑天老爺金顏。是小的姐弟之萬幸了。按院想了一想。問道。這是你多大的事。答道。那時小的纔十五歲。又問道。如今呢。答道。小的今年三十八歲了。又問道。你到這裡幾年了。答道。小的到此十八年了。又問道。你那幾年在那裡。答道。小的逃得性命歸家。病倒一年有餘。小的並無兄弟姐妹。只一寡母。又因家寒。心旣疼兒。又加紡績勞苦。及到小的病好。小的老母又病倒了。臥病數月故了。此時小的家無一文。力不能葬。小的不忍遠離。苦掙數載纔葬了。又問道。你旣如此貧窮。你妻子如何娶。又如何來\footnote{反復駁問。足見細心。恐仇口有寃民也。}。嬴陽見他駁問得利害。心下倒吃起驚來。又答道。小的自幼父親在日。定下陰家女兒。後來小的丈人見小的力不能娶。那時小的二十歲。他女兒十九歲了。小的丈人也只兩口一女。家道也甚寒薄。無可奈何。贅了小的入去的。按院點了點頭。他又稟道。小的幼時曾附搭在金知縣家館中念書。他的兒子同小的着實契厚。他憐小的寃苦。贈了幾兩路費。纔到了這裡。投在阮老爺門下。蒙恩護庇。直至今日。按院微笑道。你也讀過書。怪道你話語中也還明白。又問。你會做何事業。答道。小的因無資本。自幼學得些吹唱。在大人們門下做幫閒。按院笑道。這是你蘇州人的長技。又道。他還有何過惡。把你知道的說上來。稟道。小的離家年幼。不知其詳。不敢妄對。大約合縣之內。無不欲食其肉。就是招吿。人懼他的積威。寧負屈也不敢伸理。要是先拿役放吿。若無多人伸寃。小的領誑言之罪。願死臺下。按院又問道。難道地方上就沒一個好官。容他如此放肆麼。稟道。小的每遇鄕人問故鄕之事。聽得說當日有兩位刑廳老爺。訪問得他的罪惡。也要拿他。但他是皇親的瓜葛。但是來的欽差太監。那皇親諄托護庇。他上下大小各衙門書吏又俱情熟。事未舉行就有人報知。太監在撫按上邊就挽回過了。有此手段。故爾橫行無忌。按院怒道。俟本院再訪。只你姐弟二人的事。要果情實。這奴才就該一死了。何況於他。把你名字開來。嬴陽叩了個頭。起來寫了。跪呈上。按院接着。上寫嬴陽崑山縣民。表姐閔氏。遂遞與他家人。道。等到蘇州稟我。家人答應接過。又向嬴陽道。本院方纔駁問你者。你若有虛情就答應不來了。屢問屢答如流。其寃苦或者似實。你幾時回去。嬴陽跪稟道。小的兩三日內就行。按院道。你到家不可露出風聲。打聽本院按臨蘇州。你到衙門裡來投狀就是了。嬴陽叩頭道。小的謹遵。按院吩咐道。起去罷。嬴陽道。叩謝老爺大恩。叩了四個頭起來。按院也就吿辭。阮大鋮款留不住。衣冠送出。上轎而去。回到廳上。嬴陽又叩謝了回家。阮大鋮將酒席差人送了一桌與陰氏作別。嬴陽把前話向陰氏說了。夫妻好生歡喜。要起身這一日來辭女兒女婿。鄔合不在家。對女兒說了要回蘇州的話。嬴氏吃了一驚。流淚道。我嫁了不上一個月。爹娘爲甚麼好端端起這意思。撇了我去\footnote{妙。因未滿月未曾回家。故嬴陽夫婦不得知女婿是天閹也。若住久。也(豈)有不知之理。此雖未明明補出。却是不補之補也。}。他老子不好說得。只嘆了一口氣。道。都是你替娘老子添的光彩。你攆了我們去。倒說我們撇你。嬴氏不解其意。問母親這話緣故。陰氏遂將龍家小子在街坊上怎樣放屁辣騷說你的話。砢磣死了。令人聽不上耳。將醜名哄揚得鄰舍全知。如何還住得。所以要回去的話。說了一遍。嬴氏面赤低頭。無言可答。只痛哭了一場。嬴陽留了五十兩銀子與他兩口子。也哭了一會去了。嬴氏坐在房中。心中悲慘了一回。又想起龍家小子。切齒恨道。我一朶鮮花被你採去。和你相好了三四年。懷了肚子。爲你出乖露醜。你倒如此花敗我。就不顧我一點臉面。又把我父子都弄得分散了。無情無義。我有日相遇。把他的肉咬下一塊來吃了\footnote{與肉何干。當咬去他的陽物。}。纔出得我的恨\footnote{有此數語。故後日死龍家小子。毫無戀惜也。}。且說鄔合歸來。嬴氏拿銀子給他看。說父母要搬回故鄕。鄔合趕了去送。方知已去久了。回來問嬴氏丈人搬去之故。他如何好說自己偷漢出醜的話。只說父母想念家鄕。因此回去了。再說這嬴氏自到鄔家。雖無房慾遂心。却衣食件件如意。那鄔合又十分疼愛他。有好東西。鑽頭覓縫弄來奉承。要是出去幫閒。必定將家中肉菜果品各樣買些。知道嬴氏能飮一杯好酒。也成大罎擡放在家裡纔去。嬴氏倒也安心樂意。不想久而久之。他飽暖又思起人肉來了\footnote{此句話雖舊。換二字覺新。}。因鄔合在外的日子多。他家中從沒有個親友往來。只有個送水的王老兒。綽號王酒鬼。有七十歲了。在巷盡頭住\footnote{下此一句有因。}。只他每日早間送擔水頭到他家裡。除外別無一人。他是常到門口站。站半日不見一個人過。如此多次。一日。王老兒送水來。嬴氏問他道。我們這條巷通那裡的。怎不見有人走。王酒鬼道。這是條死巷。那裡有人走。街坊又不多幾家。都是外邊做生意的。每日早去晚歸。如何得有人來往。這嬴氏聽了。心中一把火被冷水一澆。先還妄想。或者遇巧相與個把趣人兒解饞。誰知連看的人都沒有。這個老兒又是過了時用不得的了。只得死心塌地。夜間同鄔合也臉兒廝貼。口兒相親。摟抱着親親熱熱的睡着。只是下邊少安上了那一點筍兒。也竟是一對恩愛夫妻。日間但是鄔合不在家。他便揷了門坐在屋裡。睏了睡一覺。悶來飮幾杯。即如長齋吃久了也就不大想葷腥吃。那鄔合十回九次來家。見嬴氏閉戶而坐。心中暗喜。以爲這樣貞靜女子可以牌坊都建得起的。那裡還疑\footnote{有此一句。後面方引出許多疑字來也。}心他。是以更加恩愛。就知這女子如窮漢。手中無錢食肉。苦捱淡薄而已。光陰撚指。不覺就是二年有餘。他家這條巷口。有一個土地廟。向日原有個老和尚看守香火。因這巷內人家少。沒得養贍。別處去了。空了許久。忽然來了一個和尚。叫做了緣。生得濃眉暴眼。力壯身強。有三十多年紀。要來此廟中修行。來拜衆人。衆人就說。我們這巷內只有四五家人住。都是小本經紀。供給不起。只好各家每日出一碗盞飯燈油。佈施一些沒有。所以前時的師傅住不住方去了。怎好留你。了緣道。阿彌陀佛。出家人原是苦行修行。捱餓也不妨。何況有飯吃。這就是列位的慈悲了。衆人說。你旣願看守香火。是極好的了。我們有個不依的麼。你只管來住。了緣聽說。遂來住下。前後打掃潔淨。這座廟是個大門進去一個院子。三間小房。供着本坊土地。還有個土地奶奶\footnote{泥土地還要奶奶。活和尚焉得不要婦人。}。後面一道牆。又一個小門。也是一個小院。兩間西廂房。一間做臥房。一間做廚房。這和尚原來是江洋大盜。事犯收監。越獄出來。他向來所蓄的財物約有千金。埋藏在地。逃出時起了出來。藏在身邊。剃了頭髮。做了和尚。護住身子。逃走在外。因想南京繁盛之地。四方人煙湊雜。可以混跡。故雲遊到京城來。又怕熱鬧處不便安身。被人識破。尋了多日。剛剛尋着這僻靜巷內這座小廟。得意之甚。每日只往各家去收盞飯。回來便在廟中高坐。從不出門。衆人都說他是一位有德行的高僧。他原來掛名出家。如何斷得葷酒。手中有的是金銀。只是不肯自己買來受用。這個王酒鬼每日來替他送水時。常坐了閒話。了緣知他好飮。拿錢煩他去買來。二人共酌。又常把脚步錢與他。這老兒喜得沒入脚處。一日。王老兒送了水來。閒話中。他道。我蒙老師傅這樣厚情。恨我沒錢。要有錢。買些甚麼來孝敬你。出家人的東西不是常常白擾得的。了緣笑着道。你要請我。是殺雞是宰鵝。王老兒也笑道。你出家人也用起葷來了。了緣道。狗肉我也吃。你不聽得人說。心好不用齋麼。王老兒只當他說頑話。笑答道。等我有錢着。買狗肉來請師傅。了緣笑道。只要你肯買。我出錢買來同享。如何。遂向房中取了三百文錢遞與他。道。不要買生的。或熟雞鵝鴨。或熟牛羊狗肉。不拘甚麼。買來都可。那老兒嘴笑得咧着。眼白瞪着。撅着幾根白鬍子。看着他道。師傅可是當眞的麼。了緣道。不當眞難道是假。那老兒每日挑水掙幾個錢。沽飮之餘買米還不夠。成年不見葷面。今聽見買肉來與他同享。那饞蟲已爬到喉嚨上來了。嚥了兩口唾。拿着錢往外走。了緣又叫了他回來。他倒猴急起來。道。我說你是哄我。了緣道。不是哄你。你明明的拿着。人看見了不好意思。取了個筐子遞與他。道。買了放在這裡面。上邊不論甚麼菠菜白菜。買些蓋得嚴嚴的。不可與人看見要緊。那老兒笑着一面走。道。不勞吩咐。我知道了。去了不多一會。且是來得快。笑嘻嘻的拎着筐子來了。買了大塊熱牛肉。兩隻燻雞來。了緣又取了二百文錢。一個大瓦罐與他。道。我切着菜。你可去把上好乾燒酒不拘多少。只打滿了來。沒有人看見便罷。有人見了若問。只說是你買的。他聽得打酒。更跑得快。頃刻而回。他二人關起大門來。大斟大嚼。直吃到天晚。那老兒酒醉肉飽。千恩萬謝。起身要回。了緣道。我還有話說。你每日早間往人家送水不得閒。到午後你閒了。到我處來。替我買東西。我還請你。又與他一百文錢。道。這與你買雙鞋穿。你千萬酒後不要對人說。若人知道。我住不住。你就沒得吃了。那老兒喜出望外。連忙答道。我的頭毛都白了。難道還不知好歹。師傅這樣好情待我。就殺了我。也是不吿訴人的。作別而去。此後習以爲常。每日將午就來。替他打酒買肉。二人受用。這王酒鬼生平也沒有過這樣好日子。快活不過。再說了緣。每日往這幾家收盞飯。從不曾到鄔合家中來。他也從未見這嬴氏。嬴氏也並不曾看見他。這是何緣故。鄔合因多在外少在家。只一個少年婦女在家中。恐怕不便。先對他說過。我家無人。不必來收飯。每月送他五升米。到日來取。做定了規矩。先來過兩次。皆値鄔合在家。街上去買了米就送與他去了。那日又到日子。鄔合偶忘了這日。夜間天氣甚熱。蚊子又多。這嬴氏一夜沒睡。次早天涼。方矇矓睡着。鄔合要出門去。叫婦人道。我要出去。你起來關門。嬴氏睏得很。說道。我要睡睡。關了門。停會老王送水來又要開。我不耐煩。你帶上去罷。那鄔合也就依他。把門帶上去了。恰好這了緣是收月米的日子。他也知鄔合常不在家。故淸早來尋他。走到門口。見門還關着。只他還未起來。等了一會。不見開門。用手一推。原來是虛掩着的。他叫道。鄔大爺可在家。叫了兩聲。不見答應。走進來伸頭往客坐內一張\footnote{強盜行徑。}。不見有人。到臥房窗眼中往裡一看。只見一個婦人精赤條條。上下無一遮蓋。仰〖扌扉〗着睡在床上。一身雪白淨肉。一雙小脚穿着大紅睡鞋。因怕蒼蠅。用芭蕉扇將臉蓋着。雖隔着一頂冰紗帳子。看得明明白白。眞可愛也。眞如。

\begin{quotation}

竹絲蓆上。橫堆着一段羊脂白玉。

冰紗帳裡。煙籠着一簇芍藥嬌花。

\end{quotation}

他打頭頂心上一麻。直酥到脚底。這個賊禿四顧無人。此時性命都不要了。那裡忍得住。悄悄將房門推開。脫了衣服。揭開帳子。輕輕爬上床來。再一細看。這婦人因怕熱。將兩條腿揸得大開。一條還擱在竹夫人上邊。那件寶貝雖然生產過。因兩年多不曾弄聳。長得飽飽滿滿。他身子比當日又發胖了些。此物越發滾圓。竟像放光的一般。只露一條細縫。微微張開。紫巍巍一個小花心吐出。上面又光又滑。並無毳毛。那賊禿淫興大發。那個小和尚直豎豎在腰中混挑起來。足有七寸餘長。鍾口粗細。他也不敢造次。吐了一口唾沫。抹在龜頭上。又擦些在他陰戶門首。低頭看準。往那縫裡一頂。早把個小和尚的腦袋鑽進紅門裡去了。那婦人夢中驚醒。把扇子揭開。眼睛一看。原來是一個和尚。驚問道\footnote{不怒而驚。可見情願。}。你是那裡來的。這麼大膽。那賊禿將他抱得緊緊的。道。女菩薩。小僧是來化緣的。一面說着。下邊亂抽。那婦人久曠的了。忽然嘗新。已美不可言。又從未經過這樣驢大的行貨。覺得內中滿滿塞住。無微不到。下下皆中癢筋。話也說不出來。任他橫衝直闖。這賊禿身體強壯。力氣粗雄。極力衝突。把個嬴氏弄得面紅耳赤。骨軟筋酥。受用不過。但恐樂極悲生。命因奸喪。要知二人後事如何。須將下回接看。

姑妄言六卷終



\endnotetext[1]{此句原書右有夾批「可傷」二字。}

\endnotetext[2]{以下有錯簡。下文自「這個女兒」至「桌上擺設些蘇」爲一葉,原置於此,書眉註明「此篇在後九篇接筍」,亦即當接九葉之後「就在他家左近租了兩間房子住下,過了三四個月,陰氏生了」句下(此行書眉註明「後一篇」、「前一篇」。)今據文義改正。}

\endnotetext[3]{此段批註,「的是」原作「是的」,「何得有此」原作「何此有得」,據文義改。}

\endnotetext[4]{「也」字原置「合」字之上,據文義改。}

\endnotetext[5]{「不出」原作「出不」,據文義改。}

\endnotetext[6]{「梨園」原作「園梨」,據文義改。}

\setcounter{footnote}{0}

\theendnotes

\part*{姑妄言第七卷}
\addcontentsline{toc}{part}{姑妄言第七卷}
\markboth{姑妄言第七卷}{姑妄言第七卷}

鈍翁曰。嬴氏受了緣色癆錢癖之創。雖是寫賊禿獄卒之惡。然不有此一番荼毒。後來嬴氏仍回鄔室。不能悔心相安也。

捕快之獲了緣。足見此輩之能。亦顯此輩之惡尚過於盜也。寫了緣避難之盜心虛如見。

王酒鬼之懷恨。因了緣先親後疏之故。所謂遠之則怨是也。足見人之處世。待小人不囗(可)不留一番心思。

忙敍事中夾寫知縣去接旨。爲魏忠賢建坊。筆力何等矯健。

世間之惡。到了獄卒。再無過於此輩者。漢周勃云。吾曾將十萬兵。身爲大帥。不知獄吏之尊若此也。千古皆然。爲官者能禁其惡。犯罪者得稍甦其苦。自當獲福無量。于公治獄。大興駟馬之門。豈非前轍。

鐵按院之誅聶變豹。鉏凶去暴。雖是警醒惡人。乃是了結嬴陽報閔氏一番公案。閔氏嫁金鑛。亦是趁此完結二人。省得後來累筆。

龍家小子事中。隨筆即出楊爲英。充好古。郗氏。何等筆力。且無痕跡。龍颺來尋嬴氏。欲續舊好。情雖可惡。鄔合夫婦處以此法。似乎太過。然不如此。將來終不能斷絕也。又要累筆。如此結去。何等乾淨。

牛質之好淫。即有苟氏好淫之妻。牛質喜胡旦之後。苟氏即喜胡旦之前。己與紅梅所生之子反棄之。胡旦與苟氏奸空(宿)之子反留之。貪淫之人。神鬼奪其魂魄處。香姑更不知爲誰氏之兒。彼自欺之。夫復誰尤。其報應之說。正文已見。茲不再贅。

此一部書中。婦女貞烈賢淑者少。淫濫潑悍者多。或謂將婦人貶之太過。此一回內有三奇女焉。閔氏忍辱報仇。高女矢貞死節。單氏善賢預化。亦足以揚婦女之至矣。

這兩回書中。陰氏有二奇焉。前一回。他自幼淫蕩。到後來竟能潔身自處。一奇也。此一回內。他與金鑛可謂厚之至矣。且金鑛又長於戰法。而彼竟辭之。不復與淫。又一奇也。以陰氏所爲言之。淫只可謂之三。而情有七。較諸婦淫濫不堪者。高出許多頭地。宜乎後有好處也。

\chapter*{姑妄言卷之七\\
第七回 凶淫獄卒斃官刑 奸險龍陽遭暗害\\
附 嬴陽報舊恩 苟氏私新寵}
\addcontentsline{toc}{chapter}{第七回 凶淫獄卒斃官刑 奸險龍陽遭暗害}
\markboth{第七回 凶淫獄卒斃官刑 奸險龍陽遭暗害}{第七回 凶淫獄卒斃官刑 奸險龍陽遭暗害}

話說那嬴氏正在睡中。做那巫山之夢。不想被這賊禿一陣衝突醒了。那賊禿也是熬久了的。只耍了不多工夫也就洩了。方伏着不動。婦人甦醒了好一會。才喘過氣來。問他來歷。賊禿道。我在巷口土地廟中住。來了兩三個月了。並不曾見你的嬌容。若早知道。我也來親近久了。說着。那小賊禿又硬起來。他又要弄。婦人被他這一陣弄丢了數次。渾身酥軟。又怕王老兒送水來。推住他。道。你旣住的不遠。我們有日子頑呢\footnote{反是婦人先說。寫盡淫婦之淫。}。此時怕老王送水來撞見怎了。你快穿衣服出去。賊禿聽了。滿心歡喜。親了幾個嘴。纔洩出那話來。還是硬幫幫直豎着一條紫皮甘蔗。婦人看見。倒反吃了一嚇。暗想道。我說裡邊怎麼這樣有趣。原來這等粗大。比小龍的竟有兩個還旺些。虧這裡頭怎麼容得下他。兩人拭抹了。一齊穿衣下床。那賊禿捧着婦人的臉。又親了幾個嘴。要他約個日子好來。婦人道。我家的或在家或不在家。日子定不得。你留心。但看見他出去。左右無人。你來輕輕敲門。我便放你進來。這裡鄰居稀少。你只管放心。賊禿歡喜得了不得。兩個人笑嘻嘻的攜手同出房來。不想王老兒送了水來。撞了個滿懷\footnote{先嬴氏說怕王老兒來。此時偏就撞着。天地間有此等巧事。}。笑問道。老師傅來作甚麼。賊禿忙答道。我來收月米。低着頭忙忙的走出去了。這婦人也急忙縮回身來。那王老兒只當鄔合在家。也不管閒事。倒了水自去。婦人出來關上了門。進房坐在一張杌子上。沈思道。不想今日無意中遇着這件活寶。不但粗大。而且又長久。不枉我胯中生了這件東西來。蹺開腿。伸手把陰戶一摸。還像個沒牙的嘴一般大張着。尚未閉嚴。心中又喜又是好笑。且說那賊禿回到廟中。想道。我也遇過好些婦人。總沒有他這種標致風流。看他又騷淫得有趣。得這個妙人兒長遠守着。隨早隨晚的高興便弄。方纔暢快。也不枉我出家一場\footnote{不是強盜。算計不到此。不是強盜做了和尚。亦算計不到此。若在家人。雖有壞者。或尚無此等惡腸算計。}。須設個法子騙了他來。想了一會。道。有了。須如此如此。方才便得動他。這賊禿留心在廟門口守了一日。不見鄔合回來。捱到掌燈時候。知他家無人。走來輕輕敲門。這婦人二十多歲。今日乍經了這番快樂。秋淸氣旺。此時正小飮了幾杯。正等鄔合回來好去睡覺。忽聽得門響。即走來開門。原來是和尚。笑吟吟放了進來。隨把門閂上。到了房中。那賊禿假作驚慌。道。不好了。早間我兩人出去。被老王看見。他午間吃醉了。到我那裡發話。說我來同你私偷。我再三分說我來收月米。他說我明明看見你兩個人手拉着手走出去。難道他家沒男人。你拉着婦人的手笑嘻嘻的。化(普)天下化米化緣的也多。我七八十歲了。從沒有聽見這個化法\footnote{說得活像。不由婦人不信。}。兩人明明是通奸。還要胡賴。被他拿住筋節。我沒得說了。只得軟求他。他說要不張揚。須送他一百兩銀子。方買住口聲。不然要吿訴你鄔大爺。還合同衆街坊送你我到官處治。我哀求了半日。求他寬我十天。我湊銀子給他。他才依了。他說明日還要來向你講話。如今是那裡這些銀子給他。這怎麼處。那婦人那裡知道賊禿是詭計騙他。也着了急。哭道。這是你做的事。就到官。我也實供是你偷我的。賊禿道。這如何辯得淸。兩人做的事。官府也不肯偏信。我怕甚麼。就是問了和尚的奸。不過打頓板子。枷號還俗。只是你也要褪褲子打光屁股。枷號官賣。我一個出家人那裡怕他。佛家弟子隻身一口。何處不去。但恐連累了你。心中不忍\footnote{好慈悲。不枉是和尚。}。特的來同你商議。那婦人聽了這些利害話。越發哭起來。道。我一個婦道家有甚麼主意。人家說一夜夫妻百夜恩。我的身子也與你睡了。你可想一個主意救我才好。賊禿道。可不是呢。我要不爲你。我就悄悄去了。他往那裡去尋。我因放你不下。纔來和你說。我倒想了一個主意。只怕你不肯依。婦人道。你說了看。賊禿道。千着萬着。走爲上着。除非你同我逃走。方免得這禍。婦人道。逃往那裡去。賊禿道。我原是好人家的兒女。也做過一任官來\footnote{強盜也。而云曾做過官。是盜而官乎。官而盜乎。令人笑道。或少年時做過小官。則未可知。}。因看破世情。出家也不久。我家還有大房產地土。你同我去。我留起頭髮來。作個長遠夫妻。你還是一位夫人呢\footnote{眞是壓寨夫人。}。我的家私儘夠受用一輩子。你依不依。憑你酌量。不然我明日獨自逃去了。等他來同你吵鬧。婦人也沒了主意。雖不知他這些話是眞是假。實在有幾分戀着這和尚的本事。問道。依你說。要走幾時走呢。賊禿趁機道。安心走。今晚就走。若到明日。露了風聲。人防範起來。就走不脫了。婦人只得依他。那賊禿滿心只想騙這婦人。他銀錢自有。不稀罕他家的東西。婦人趕忙只收拾了他行經的絹帕睡鞋。又拿了兩把梳子。拿塊布包了。塞在裙腰上\footnote{精細之甚。此數件物是婦人萬不可少者。}。此時已將起更。街上靜悄悄的。他同了婦人出來。反帶上門。往廟中來。那婦人與鄔合二載有餘乾夫妻。雖無實事。也感他那相愛的恩情。雖然有些捨不得他。到了此時。也顧不得了。到了廟中。將兩層門都關上。進房坐下。他有現成的酒肉。取出來讓婦人吃了些。他自己呷了幾碗燒酒。見婦人不用了。將傢伙撤去。撥明了燈。替婦人脫衣上床。他也脫去衣服。然後擺開陣勢殺將起來。怎見得。

\begin{quotation}

一個光頭元帥。一個豎嘴將軍。那光頭元帥仗着黑纓鎗分心直攮。那立嘴將軍忙持紅邊劍向腦就吞。這元帥連珠炮一出二子。那將軍皮擋牌兩瓣雙迎。那元帥怒豎倒生鬚。這將軍笑張無齒口。那元帥鎗鎗單刺紅心。這將軍劍劍只含紫腦。那元帥越加梟勇。戰多時。光頭上爆火出來。這將軍漸覺酥麻。敵不住。豎口中流水氾濫。

\end{quotation}

這賊禿眞有不歇不洩的本事。日間因是久不見婦人。故此易洩。這一回上手就是幾千抽。弄得這淫婦心花內都是快活。欣欣暗喜。誰知他只管弄將起來。有一個更次。那嬴氏丢了數遍。有些受不住了。說道。歇歇罷。讓我透透氣兒。那賊禿那裡聽。便道。早呢。倒從新鼓起威來。自首至尾搗了有幾千下。搗得個女人氣都接不上來。大張着嘴。白瞪着眼睛。兩個鼻孔一張一張的。賊禿看見他這個樣子。略慢了些。婦人纔回過氣來。哀求道。我受不得了。明日再弄罷。這裡邊有些疼了\footnote{可謂甘盡苦來。}。賊禿親了個嘴。道。你略忍忍兒。我丢了就好了。一面說着。又一陣亂抽亂搗。這一陣也不計其數。更加凶猛。一陣緊似一陣。起先婦人陰中有些滑溜還自不覺。此時他拿出那做強盜的本事來。如扯風箱一般。陰中淫水被他扯乾。一出一進。連皮帶肉。扯得火燒火辣生疼。婦人苦苦哀求。他那裡肯聽。抽一抽。那婦人疼得哎喲叫一聲。他也只當不曾聽見。那賊禿覺得裡面乾乾緊緊的。箍着陽物。如口裹着一般。快活不過。又弄了有一個更次。忽然像瘋了似的。極力亂搗了一陣。也覺樂極。方一洩如注。纔肯歇手。外面已交五鼓。這婦人被他弄得七死八活。眼淚也流了不知多少\footnote{下眼之水流盡。上眼之水又流。何此婦水之若是其多也。}。見他歇了。如放赦一般。痛得哼個不住。側身而臥。這賊禿先飮酒時也有八九分醉了。乘着酒興。不管人死活。足足搗了一夜。也乏倦了。倒下頭。鼻息如雷。鼾鼾睡去。這婦人那裡睡得着。覺得陰中疼痛難忍。伸手摸摸。原來裡外都腫了。裡邊因乾的緣故。被他一陣蠻扯。皮都扯塌。所以這般疼痛。這婦人雖好飮一杯。不過三更鍾的量。適興而已。那裡禁得拿大碗如長流水一般灌起來。自然要吃到潦倒不堪\footnote{妙譬。甚趣。}。況他這樣一個嬌怯怯身子。可經得這等狂風大浪。他經了這一番。反拗(懊)悔起來。暗想道。當初幼年雖行得不是。同龍家大小子私偷。彼此還有些情意。後來嫁了鄔家。雖然是乾夫妻。他這種恩情實令人感激不盡。今日遇了這和尚。只說也必定有些恩愛。跟了他來。誰知這樣狠毒。將來定然死在他手中。如今旣走了出來。料道又回去不得。左思右想。忍不住嗚嗚咽咽哭將起來\footnote{應前欣欣暗喜。可謂喜極生悲。}。此時夜短。天已大明。和尚也睡醒了。看見他哭。說道。你哭甚麼。摟過脖子來親了個嘴。爬起來道。我還有些餘興。再弄弄着。那婦人把腿夾得死緊。用手推着。道。還弄呢。被你弄得稀爛的了。且說正經話。你昨日說要走。今日爲何還在這裡住着。此處近着家。不是當頑的。和尚原是要騙他來。何嘗有心要走呢。哄他道。我般(船)還沒有雇停當呢。等停妥了再走。又對婦人道。你日間只在這屋裡。關着門窗坐着。若外邊有人敲門。你躱在這口大櫃子裡面。鎖了櫃門。神鬼不知。櫃子裡屜兒我已去掉了。後邊的板也打下來了。坐在裡頭。一些也不悶氣\footnote{不做過強盜決想不到此點。強盜可謂點矣。其如滑番子更滑。奈何。}。且躱兩日再走。我這裡也從沒人到來。你只管放心\footnote{妙。有此句。方見他才敢拐婦人來也。不然離家咫尺。也非愚呆者。何敢大膽至此也。}。那婦人只得依他。賊禿說着。又扳婦人的腿要弄。婦人死也不肯。他笑道。也罷。讓你養了精神。夜裡再弄罷。說罷。穿衣下床。婦人只得也起來關着門窗。坐地又是間西廂房。天氣炎熱。幾乎悶死。到晚來。他吃一個飽燒酒。抵死要弄。他力氣又大。婦人又拗不過他。又不敢叫喊。但弄一遭定弄得死而復甦者數次。你想一個作強盜的人。殺人不眨眼的魔君。可還有甚麼情意。那婦人陰中腫破。又是汗螫着。痛不可忍。一日到晚只得揸開了腿坐着。透些涼氣還略好些。兩邊膫骨又被他撞傷了。兩隻腿如折了的一般。又揸不得多工夫。捱到下晚。天氣略涼。痛才稍止。他又要弄起。這婦人此時求生不得。求死不得。一連過了四五日。並不見他提起走的話。再三問他。只含糊答應。又聽得王老兒每日送水來。歡歡喜喜替他買東買西。並無話說。方悟到是被他所騙。說不出口。只是暗暗的哭泣。再說鄔合那一日從淸早出了大門。到宦家去幫閒。遇有酒席。晚了未能回家。次日一早回來。恐家中少長缺短。沒有嬴氏的食用。到了門口。方要敲門。那門隨手而開。他道。娘子今日如何起得這樣早。倒開了門了。或者是昨晚忘了關。走進來。見臥房門也開着呢。他道。原來起來了。走進房來。却不見有人。一眼望到床上。被疊得好好的。這是昨日疊的。未曾動。他疑是今日早疊\footnote{一疑。看他下邊寫許多疑字。都有層次。}。疑他在廚房燒火洗臉。走去一看。淸鍋冷竈。不但沒煙火。連人都不見\footnote{二疑。}。疑是在後院上毛廝。走去一看。也沒有\footnote{三疑。此必無之事。不得不疑到此。}。心中動疑道。想是家中沒了火種。往鄰居去討火去了。但他從不出大門\footnote{四疑。疑到無可疑處。只得疑到此。}。忙走到鄰舍家去問。都答道。你家娘子這兩三年了。從不曾到我們家來。我們還不曾見他的面目是甚麼模樣呢。大淸早夕他來做甚麼\footnote{是寫兩年多靜處之嬴氏。故愈動鄔合之疑。}。鄔合聽了。心中疑他逃走。忙回家來查點。東西一絲不少\footnote{五疑。眞令人猜疑不出。}。復疑道。要是同人逃走。有個不拿東西的。難道是投井去了不成。但井在盡頭。他也認不得。又沒有吵鬧拌嘴。如何尋死\footnote{六疑。到水窮山盡無可疑處。不得〔不〕疑到此。}。又疑道。莫不是爲我沒這東西。急了去尋死。但也過了這幾年。又不見有甚聲色。眞急得沒法。就走到井邊一看。那是個石頭井欄。只有盤子口大。僅容得個小竹吊桶。跳不下人去\footnote{七疑。這一疑更不可少。寫鄔合總猜疑不着。走投沒路處。眞妙筆也。}。疑他還是逃了。復來問這些鄰居。此時男人都出去了。只有婦女在家。他問道。我家女人不見了。大嫂們素常可曾看見有甚麼人在我家走動。或者同人逃走了\footnote{八疑。自然還要疑到此。}。那些鄰舍婦女們道。你家娘子極賢慧。不但從不見面。這幾年連大聲氣也不曾聽見他的。他輕易門邊也不出。又沒個人到你家來。如何會走\footnote{鄰家如此說。鄔合越猜疑不着。連逃走尚疑在是否之間。}。正說着。王老兒送了水來。問道。鄔大哥。你在這裡說甚麼呢。鄔合將不見了妻子話吿訴他。他也吃了一驚。放下桶。道。你娘子終日在屋裡坐着。如何會不見了。我成年家送水。十回還有四五回不見他的面呢。又想了一想。道。我昨日送水還看見他呢。往那裡去了\footnote{和尚拐去嬴氏才一夜。此時王老兒與他尚無恨。故不肯說出和尚。妙。}。鄔合道。正是此說。不知何故不見他。四處訪問了一日。全無影像。次日只得到兵馬司去遞失呈。求他緝捕。竟數日杳無踪跡。這一日對宦蕚說了。宦蕚發了名帖。差長班雇人替他寫了張失呈。送到縣中。煩他上緊緝拿。這知縣是宦實的門生。見兄來托這點小事。敢不遵命。即刻傳馬快來吩咐了。發了捕批。立了限期。過期不獲。定行責處。這幾個快手領了批出來。到鄔合家中問了詳細。鄔合又送了一個東道。折乾的封兒。捕快們拘齊了鄰舍來問。衆人同答道。他娘子從來門邊兒也不出。他家又從沒個人來往。這不見得蹊蹺得很。我們如何得知道呢。差人道。你們都是緊鄰。這地方又沒多人。推不得乾淨的。大家都有干係。若拿不着人。少不得你們都要到官。衆鄰居見說。都是膽小的人。從沒有見過官府。聽見了這話。有些着忙。大家背後商議。一家拿出一百文錢來。共湊了五百文。向捕快道。師傅們到這裡來。我們應該備一杯淸茶奉敬。窮家小戶不便宜。我衆人湊了個薄禮。衆位師傅請茶館中坐了罷。衆捕快道。我們怎敢受你們的禮。衆鄰舍陪着笑。道。原輕微得很。不是敬師傅的。但我們都是窮漢。可是人說的。顯道神跳井。盡盡心兒罷了。一個捕快道。旣承你們的情。我們領你們的了。你們有甚麼話說麼\footnote{五百靑蚨說話。可嘆。}。衆人聽見他口氣鬆了些。就借因兒推說道。鄔家這件事。要求衆位師傅照看。我們都是做小買賣的人。早出晚歸。從來都不到他家走動。只有王酒鬼與他家送水。是每日到他家去的。有人來往沒有。或者他還知道。捕快道。王酒鬼在那裡住。答道。他住在盡頭那一家。門口有井的就是。捕快道。你衆人同我們去找他。衆人只得跟了同去。却說這王老兒每日大酒大肉。擾繞吃了兩個多月。好生的快樂。又間或得他些資助。替他買東西。又可賺錢肥家。正然吃得興頭。自從他拐了婦人到廟中之後。再也不留他吃酒吃肉了。把房門關着。也不容他進去。每日還托他買東西。買得比先前更多。却沒得與他到口。雖然給他幾文脚步錢。但他這些時好東西吃慣了。這幾文錢只好買酒呷。那得有肉來吃。喉中的饞蟲都爬將出來。心中恨道。這禿驢好可惡。你一日買這些東西。一個人也吃不了。天熱又放不得。與我些吃吃何妨。就這樣吝嗇起來。待我這樣刻薄。幾時我故意給人看見。弄個大家吃不成。心雖如此想。還貪他的錢文吃酒。尚捨不得洩露。這日正在井上打水。只見一夥人走來。他不知是做甚事。方才要問。內中一個鄰居叫道。王老爹快過來。這是衙門中的捕快師傅們來問你話。那王老兒連忙把桶放下。走近前來。笑着道。衆位老爹叫我說甚麼。捕快們就說。鄔家的妻子不見了。定是跟人逃走。道你在他家常常送水。你可曾看見有甚麼人在他家走動。那酒鬼正恨賊禿。這一問。正中心懷。即答道。我在他家送了幾年的水。不曾見有個人影兒。就是他妻子不見的頭一日我送水去。遇見巷口土地廟中的和尚在他家來。我問他做甚麼。他說收月米。別的却不曾見。是他拐不是他拐。我也不知道。他這些話答應。原不曾疑心和尚拐婆娘。不過總成\endnotemark[1]捕快們到他廟中看見了酒肉。詐出他些錢來。出出自己的氣。且又不曾破臉。後來還可以替他買東西賺錢作酒資。誰知這賊禿惡貫滿盈。應該敗露。捕快們聽了王老兒話。向衆人道。這和尚是那裡來的。住了多少時候。做人如何。現今可還在廟中。衆人道。這座廟因沒養贍。空久了沒有人住。他是個雲遊的和尚。是上江人。才來了有兩三個月。情願苦修。每日只是收了盞飯就關了廟門。從不出來化緣。是位有德行老實的和尚。他老在這廟中修行了。作馬快的人比伶賊還透露三分。王老兒雖是無心說話。他却有心。聽婦人不見這一日恰恰的和尚就在他家。十分中就有五六分動疑是他拐去。便道。你們且散了罷。我們往別處去訪問訪問。衆鄰舍散去。幾個捕快同到一個僻靜的小冷酒鋪中坐下。商議道。聽那老兒口氣。多半是這個禿驢。一個道。若是他拐了婦人。這幾日爲何不逃走。還肯在這眼皮子底下住着。一個道。也定不得是不是。咱們到廟中踩踩看。又一個道。衆人都說他是有德行的高僧。若是踩不着。傳到官府耳朶裡。還說我們借端生事。詐騙好和尚。不是當耍的。內有一個老捕快姓計名德。他想了一想。道。不然。多應是他。他裝老實慣了。說沒人疑他。定然藏在屋裡。況且光着個腦袋。帶着個婦人。怎個逃法\footnote{一語道破。眞是老奸。非此輩不能知強盜的心腹。}。我有主意了。等我吃幾杯酒。裝作醉了的樣子。敲開門嚇他一嚇。他若不動一些聲色。你們上前來拉開。替他陪禮。只說是我們是上司差來替鄔家拿人的。他請我們吃酒。天熱。要到廟中歇歇涼。要碗水吃。我有兩歲年紀了。多吃了幾杯醉了。和他頑耍。他也只得依了。若是心虛。形色一變。必定是他。再行拷問。你們說好不好\footnote{此計眞妙。不愧名爲計德。}。衆人笑道。琉璃簪不錯。你到底是東方朔。好個老賊。叫掌櫃的打了幾壺酒來。又煩他去買了一大盤稀爛的狗肉。鹽醋蘸着。大家吃畢。會了賬。一齊走到土地廟前。天色將晚。這個計德將腰中的鐵線取出。提在手內。把廟門乒乒乓乓亂敲。這和尚正赤剝着。抱着嬴氏在懷中吃酒。這婦人頭不梳。臉不洗。面色焦黃。眼眶通紅。愁眉苦臉。一點東西也不吃。賊禿把婦人的胸前坦開。摸着奶頭耍笑。強讓着婦人吃酒。忽聽得打門。他道。沒有別人。這又是王酒鬼來想酒吃。不要理他。聽得打得甚凶。有些疑影\footnote{這一疑。心先虛了幾分。妙。}。忙把婦人藏在櫃中鎖好。將酒肉都藏過了。披了衫子。一路問出來道。是誰打門。外面也不答應。只是敲打。心中甚疑。不得不開。纔拔了閂。只見一個人一手拿着鐵線。一手推開門。進來就劈胸揪住。大喝道。你這個禿驢藏得好。一般的被我拿住了\footnote{一語雙關。妙甚。若果是好和尚。便謂。敲門多時不開。你藏在裡面何事。一般要出來拿住了你。一也。或是拐了婦人。便謂。你藏得好。一般被我拿住。不必定心虛。二也。至於竟是強盜。彼心懼自首。則非計德之料矣。}。這賊禿原是有心病的人。看見許多人進來。並不想到是爲婦人。只當緝着前案盜情來拿他的\footnote{錯認得妙。映前心疑。}。不由得撲的跪倒。衆位爺。我前案的事結過三四年了。又不是本地方的事。若饒我的狗命。我重重的酬謝衆位爺。衆人原是來試探他。不想弄假成眞。聽了這話。就知是逃盜。遂順着他說道。果然不是我們地方上的事。但有廣捕文書來。方來拿你。果然重謝我們。自然護庇下你來。衆人也並不知是那一案的事。不過是想詐他一主財香。也就撒了手了。於是把大門關了。同到房中來。那賊禿見事體不妙。強盜的事都犯了。還怕和尚吃酒肉的罪不成。遂將酒肉搬將出來。衆人也就吃。只留心看守着他。不多時吃完了。問他道。許我們的東西拿出來罷。我弟兄們人多。不要一點點子。打水不渾的。一個姓滑的叫作滑游。道。他走江湖的人。自然在行。何用我們說呢。倒像我們小器。他這是買命的錢。少了他也拿不出來。我們還替他擔着天大的干係呢。那賊禿此時也軟了。戰戰兢兢的將床底下一個掛箱取出來。道。小僧的家當全在這裡頭呢。將鎖開了。衆人一看。內中黃白之物約有六七百金。他只留下了一大包銀子。有四五十兩。吿道。這些須留下與小僧做個盤費。別的都孝敬衆位爺罷。衆人見了這些東西。已是快活得很。但貪心再是不足。見他出手又大又快。疑他別有所蓄。說道。這點子就要買一條命。有再拿出些來。我們好放你。那賊禿何嘗是捨得。也並不見出手大。只因急了。顧命要緊。況且東西原放在一處。一時又藏不及。所以全箱送上。留這幾十兩銀子。好想方法帶着婦人逃走\footnote{此一句妙。他未嘗不想逃走。在此久住。因未曾想出好方法來。}。別尋安身之路的意思。聽見衆人說他還有。急了道。衆位爺在上。銀錢是人掙的。自家的性命要緊還是錢要緊。這是我一生的積蓄。因感衆位爺活命之恩。故都送上。留這一封做盤費。不然叫小僧餓死了不成。屋裡空空的。別處也沒藏放的地方。況小僧才來不久。難道埋在地下。他這些話說得盡情。衆人道。也罷了。那滑游見了這口大櫃子大鎖鎖着。心中一疑。道。這禿騙旣做強盜。焉知不拐婦人藏在這裡面亦未可知\footnote{眞滑。}。就是裡面沒人。雖未必有銀子。或有衣服紬緞之類。也可分惠些。遂指着櫃子道。這裡面是甚麼東西。開了我們看看。這賊禿見事體有幾分妥了。正陪着笑臉說長道短的哀求。忽聽得要開櫃子。面色頓改。答應不出。半晌道。是是空櫃櫃子。裝着些破爛東西。並沒一個値錢的物件。滑游見他顏色有些古怪。走起來相了相。用手把櫃子推了兩推。覺得裡面沈重。上前將鎖一扭。那什件是朽了的。一下就斷了。雙手將兩扇櫃門豁剌一開。定睛一看。原來是一個蓬頭撒髮的活寶。大笑道。在這裡了。遂喝問道。你可是鄔家逃走出來的麼。那婦人初關在櫃中。已是熱悶得心慌。尚無懼。後來聽得進來吃酒說買命討饒的這些話。已知道這賊禿是強盜了。不由得心中撲撲的跳起來。後來又聽得問道櫃子。他渾身都抖。上下牙齒逐對撕(廝)打。及至聽見擰鎖。開了櫃門。已嚇得在裡面着急。雖聽見問他的話。那裡還答應得出來。只是戰呵呵的哭。那滑游又問了一聲。不見答應。一把抓着。拎將出來\footnote{拎字趣。是嚇癱了的。}。劈面一掌。打得一交跌倒在地。一個道。不用打他。明日到堂上拶起來。怕他不說麼。此時賊禿已嚇昏了。跪在地下。一個捕快腰間抽出鐵尺。照膀子上儘力兩下。喝道。賊禿。細細的說如何拐出來的。免得老爺們動手。賊禿被打得頭渾眼花。哀吿道。爺們不要動手。我實供罷。此時見婦人也跪在傍邊。人贓現獲。料推不掉。不如實招。免受他的拷打。遂將如何收月米。如何看見婦人獨臥。如何奸他。如何設計騙他出來。這賊禿活該倒運。從頭至尾細細說出。那捕役聽了。切齒恨道。你這個禿奴。人家好好的婦女。活活坑在你手裡。你暗暗的奸他就該死了。又設計騙他逃走。到衙門一陣拶打是不消說的。還要官賣。若賣下水去。這婦人一條性命不是你送了他的。說着。又狠狠打了幾下。計德\endnotemark[2]道。且不要打。問他當日是何處的強盜。逃到我們地方上來。問明白了。明日好稟官。這賊禿聽了此話。問(不)知所措。方知他們剛纔不是來拿他的。悔之無及。不肯實供。一個發怒道。這樣惡人。不下手打他。他肯好好的說出麼。遂大家動起手來。番子們收拾強盜的非刑。說起令人寒心。先吊打了無數。和尚死握不招。計德將他兩隻膀子用鐵線拴在一處。取出一根數寸長的檀木棒來。有大指粗細。揷在鐵線中。用力絞起來。勒得深入半寸。皮開肉烈(裂)。他咬牙死受不說。衆人就拿他作蝦蟆曬背。兩手兩足用繩拴了。背向上臉朝下。懸空弔在(住)。衆人又背上放一大盆滾水。他尚不肯招。又將大石壓上。渾身骨縫皆開。這賊禿眞是個頑皮鐵骨。他猶然堅忍。計德恨怒極了。將他放下綑好。腰間取出一個包兒。打開。原來是一包硬豬鬃。扯開賊禿的褲子。拿豬鬃通他的馬口眼\footnote{這小和尚也受用了幾日。今受此刑。也可謂樂極生悲。}。這是番子處強盜的頭一件惡刑。那禿奴不是鐵人。如何禁受得起。他雖然性惡。也是父娘生的皮肉。被這些捕役們收拾得他就像他弄嬴氏的一般。死去活來數次。忍不得了。方纔實供他是江西鄱陽湖的江洋大盜。越獄逃走。出家避難。始末原由備細說明。衆人方放了他。看那婦人時。嚇得渾身戰得要死。坐到天色微明。將和尚綁起。婦人鎖着。帶到衙門中來。這日北京有欽差官齎旨意到來。諭各府州縣替魏忠賢起蓋生祠。縣官隨上司去接旨。不得審理。吩咐一應事務都等回衙發落。衆捕役將和尚婦人墩鎖在鋪內。交付人看守着。知道官府不得就回。大家去分用和尚的金銀。還有些零星什物。蓆捲分之。每人約得百餘金。心中暗喜。付(復)又都到衙門口來伺候。將有午刻。官才回衙。因辛苦了。進內歇息。直到晚堂。方升公座審事。頭一起就是衆番役跪上堂繳捕批。將和尚拐去婦人拿獲到案。細細稟了。知縣先叫帶婦人上來。問他從何時通奸起。如和(何)跟和尚逃走。把驚堂一拍。衆衙役喝了一聲。如轟雷一般。這婦人小小的年紀。何嘗見過這樣威嚴。也顧不得羞恥了。二來心恨和尚。添了些話。就將他如何睡覺。和尚進來強奸。若不依從。便要殺害。又如何哄他逃走。藏在櫃中。不許聲張。不然也要殺。小婦人怕死貪生。纔作了這醜事。知縣喝過一邊。帶上和尚來審問。賊禿見活口質誣在傍。無可辯得。也就直招了。知縣大怒道。和奸罪只擬杖。和尚應加一等。況且這一個淸白婦人被你坑陷。死有餘辜。吩咐夾起來。衆衙役〖口么〗喝了一聲。動手夾起。夾得那賊禿叫苦連天。收緊了。又吩咐敲二十槓子。然後撂下六根籤。〖口么〗喝叫重積(責)。衆衙役聽見這婦人的口供。生生被這禿驢坑害。況他又不曾用錢\footnote{上一句是賓。此一句是主。}。揀上好頭號大板。儘力斫了個足數。已是打昏在地。知縣命人也拶起那婦人來。衆衙役將婦人拶了一拶。堂上〖口么〗喝道。再敲三十下。命帶到衙門外褪衣打他十五板。這十五板比和尚的輕了許多。一則人可憐他被和尚坑騙。二則見這嬌嫩少婦粉團似的屁股。存了一點愛惜的心。三則官府又遠。不過打個數兒罷了。就是先拶敲時也留了些情。不然這樣個嬌怯怯的人兒。早已嗚呼尚饗了。雖說是輕。他那細皮嫩肉已打得血肉分飛。打畢繳籤。有一首花心動的詞兒。說這賊禿拐騙嬴氏奸淫。道。

\begin{quotation}

此恨無人共說。逢賊禿粗雄。心竟飛越。竊負相逃。掩上禪扉。枕簟忙忙鋪設。夜恣淫毒得天曙。怯身兒經他磨滅。孽緣滿。公堂臺下。又遭笞責。

\end{quotation}

知縣吩咐衙役去傳他丈夫鄔合。一面又審別件。那衙役去了。不多一會。來稟道。鄔合家中鎖着門呢。問他鄰居。說他時常出門。不知何往。無從尋覓。知縣道。料道這樣婦人。他丈夫那裡還要。他情有可原。免枷。今晚暫收監。明早傳官媒領賣。衆衙役答應了一聲。將婦人帶去送監。知縣又吩咐將和尚枷號一月示衆。再行發放。一面兩個就去擡枷。衆捕役又上前跪稟道。這和尚原係江西鄱陽湖江洋大盜。已經拿獲。越獄在逃。爲僧避難。到此潛躱的話。說了一遍。又道。限滿之後。或解回本地。或申報上司。若放了出去。恐將來貽害地方。知縣大怒道。奴才。不知被他殺害了多少人的性命。又坑了地方官的功名。陷害禁子拷役的幾個身家。我也沒力氣費紙筆。吩咐衆皀隸着實打。以打死爲度。衆役見本官發怒。吩咐打死。五板一換。兩膀加勁。竭力奉承。那賊禿大喊道。老爺天恩。他衆人得了我千金東西。原說是放我。此時倒求害我。我死固當。求老爺將這項銀子追了入官。小僧死而無怨。知縣問衆捕役。衆人見活口質證。不敢隱瞞。都招承了。知縣道。今日奉旨與魏上公修祠建坊。正愁沒有錢糧。可取來供用。衆役面面廝覷。只得去取。那賊禿先已打得發昏些須。此時打不到五十。已斃杖下。知縣怒猶未息。吩咐攙出去抛於郊外。這賊禿作了一生惡人。今日零星葬於豬犬鳶鳥之腹。這就是。

\begin{quotation}

善惡到頭終有報。只爭來早與來遲。

\end{quotation}

衆捕役取了贓物來呈上。知縣看了。道。方才和尚供稱有千金。如何只有這些。衆人跪稟道。實在只有這些。怎敢欺瞞老爺。那是和尚恨小的們。多說些。好叫小的們賠補。知縣笑道。贓物應當入官。和尚若不供出。你們也就瞞下了。本當重責。因你們有獲盜之功。准折了罷。下次再敢如此。定然重處不貸。衆捕役眞是狗咬尿脬。空歡喜了一場。知縣命庫吏將金銀兌明收了。留爲建坊之用\footnote{以盜贓建逆璫之坊。用得妥當之極。}。且說那衙役將嬴氏帶到監門外交與禁子。討了收管。這監中有兩個窮凶極惡。貪財好色的禁子\footnote{天下之禁卒未有不窮凶極惡。貪財好色者。}。闔衙門中慶(送)了他兩個雅號。一個叫色癆。一個叫錢癖。這錢癖遇有犯人進監。不管罪輕罪重。有幾文淹心錢給他。雖是犯剮斬的重罪。他也不怕干係。鬆放着他。還滿臉是笑。爺長爺短的奉承\footnote{大盜越獄的多由於此。爲官者不可不嚴察矣。}。若沒錢與他。就是鬪毆的小事寄監。他拿出那惡狠狠的一副面孔。白日裡手杻脚鐐兩副傢伙與人戴着。到晚來。像強盜似的上了押床。弄得人七死八活。一日到黑。嘴中〖口根〗〖口根〗嘟嘟的罵個不休。人沒奈何。連衣服都脫了送他纔罷。他得了。同色癆罷力\footnote{說盡此輩之惡。一絲也不謬。}。這色癆錢還在次。若見有婦人下監。就如蒼蠅見了血的一般。定同錢癖作好作歹的騙上了手。他二人輪流着受用。他與刑房的書辦串通了。時常的有些須小的孝敬。故爾如此大膽。闔衙門都知他二人的惡處\footnote{闔衙門皆知。其〈惡〉凶惡可知。}。有一位知縣不知\footnote{知縣者。謂要知一縣小民之疾苦。知吏胥之善矣。知風倦(俗)之厚薄也。今衙中咫尺。禁卒之愿(惡)尚不知。所知者何事。豈朝廷用一知縣。只知要火耗受私賄而已哉。然而此類知縣不少。}。這日正該他二人當値。這婦人晦氣。剛剛撞到他兩個手裡。他兩個收了婦人。與了收管。帶進女監來。那女監中空撈撈的。只有兩張矮板床。連破蓆也沒有一塊。將婦人推進裡面。把門倒拽上出來。那色癆見這婦人生得有幾分姿色。心中無限歡喜。拉了錢癖到僻處商議去了。那嬴氏自從昨晚拿獲。一日一夜。連黃湯辣水也沒有嘗着。已餓得腰酸肚痛。適纔一頓拶打。已昏暈過去。倒也不知疼痛。此時帶來收監。先是帶到衙門外照壁下去打。打完了帶進來繳籤。監在大門內右首。又帶出來。帶出帶進兩三次。也有幾百步遠。雖那衙役憐惜。扶着他些。却要自己的脚走。心裡一來害怕。二來那高底的板子在脚下拐呀拐得吃力。這一走。血脈走開。到了監中。反疼得要死。八個指頭。皮都塌了。揸着腫疼非常。到了這間黑魆魆的屋裡。越發害怕了。屁股疼得坐也坐不得。將身子斜歪在板凳上啼哭。忽聽門響了一聲。急擡頭看時。只見那錢癖手中拎着幾條絕大的鐵鏈鐐杻。豁剌的往地下一摜。喝道。起來。這個地方是許你睡着哭的麼。那婦人吃了一驚。忙要起來。渾身疼得爬不動。掙了一會。方才站起。那錢癖圓彪彪睜着兩隻眼睛。惡狠狠的道\footnote{畫出禁子小像。}。監中規矩。是女犯追來要鎖靠(銬)了。吊在梁上的\footnote{此一款。大約是大明律之外添者。}。一面拿起鎖來。道。伸過脖子來。那婦人慌得跪下。道。爺開恩罷。我這個樣子已是要死的了。這一吊起來。實實的活不成了。求爺積陰隲罷。錢癖喝道。放屁的話。朝廷的王法。積甚麼陰隲。實對你說。我這裡但是人進監都有個常例。叫做發油錢。要送得厚呢。便搭些干係鬆放他些。要沒有錢。是定要吊起來的。你一個錢也沒有。還說甚麼。難道我們在這裡喝風\footnote{是禁子聲口。此輩索錢。皆作此話。}。拿過脖子來罷。說着。理起鐵繩要往脖子上套。那婦人知道是要錢。料沒得與他。只得任其所爲。把脖子伸着。那色癆在傍邊道。哥。他也是好人家的兒女。一時被禿驢哄騙了。受了這一番苦。我怪可憐見他。哥。你饒了他這點情兒罷。錢癖道。他有甚麼情到我。叫我留情與他\footnote{話囗(中)有刺。}。色癆道。哥。你息息怒。且出去走走。讓我和他慢慢商量出個法兒來。用手推着他。那錢癖也就轉身。故意狠狠的道。兄弟。看你的面。且鬆他一會兒。我看他有甚麼法。要沒有常例錢。我今夜收拾得他不死也塌層皮。說着。忿忿而去。色癆向着婦人道。可憐可憐。你起來說話。嬴氏掙着要站起。那裡起得來。他昨晚拿來時。因天氣熱。只穿了一件夏布對衿衫兒。色癆見他胸前露出一條白肉。影影的兩枚乳峯。好生動火。站起來上前做做扶他。將他胸前接住。抱將起來。也就幾乎做了個呂字\footnote{是個色癆。}。扶他站住。道。你看見他那個樣子了。這一吊起來。你怎麼受得。你又沒錢與他。這怎麼處。叫我看着怪可憐的。婦人道。我昨夜空着身子拿了來。頭上有兩根銀簪子。耳朶上的一副金丁香。才在衙門口。不知被甚麼人拔了去。我丈夫又不知道。就是知道。他見我做了這番事。也未必肯來救我了\footnote{嬴氏作此語者。疑鄔合怨彼所爲定棄之耳。不意後來反救他憐愛他。眞是出於意想之外。故感之深且切也。}。公門裡面好修行。爺你救救我罷。色癆道。我心裡巴不得要救你呢。叫我也沒法。只是我那哥從來極愛小。你若沒些甜頭到他。他如何肯罷。停會他再發起性子來。連我也就難勸了。婦人哭着道。爺你看我就是這一件衫子。一條褲子。還有一個光身子。別的還有甚麼。拿甚麼送他。死活只得憑他罷了。色癆笑道。衫褲不留着遮肉麼。他也不稀罕。倒是身子也還使得\footnote{眞是乘機而入。}。婦人也懂了三分。不好答應。色癆又逼一句道。你怎麼不作聲。遲會子他再來。我就不管了。婦人道。爺的意思怎麼樣。色癆笑嘻嘻的摟着脖子到懷中。將嘴對着他耳朶上道。你旣沒錢。捨着身子給他睡睡罷。你也不是怕羞的。況且和尚的手段是有名的利害。你都見過了。還倒怕他麼。這比那吊着還好捱些。這是我愛你的話。憑你的主意。還不知他肯不肯呢。那婦人已是渾身疼得難受。怕他果然吊起來。如何禁當得起。此時屁股疼得很。陰中昨夜得空了一宵。倒覺得略好些。沒奈何。只得道。憑爺們罷。色癆道。你旣這樣說。就好講了。叫道。哥。你來。那錢癖走進來說道。怎麼說。色癆道。哥罷。我和他商議了這一會。實在一絲沒有。吊又禁不得。他情願把身子謝你。你好歹看我的面上。將就些罷。錢癖假作不肯。道。我只要錢。沒有錢。吊起來就是了。誰頑那和尚肏剩下的騷屄\footnote{此物也有剩的。奇語。豈未剩時又另有一味耶。}。色癆道。哥。他實實的沒有。你就處死他也沒用。不過臭這塊地。凡事看我兄弟的薄面罷。遂看着婦人道。還不脫了褲子睡着呢。那婦人只因一時之錯。到了這個地步。沒奈何了。含羞忍恥。只得將褲子褪下。爬在光板床上臥着。色癆帶着笑將那錢癖推進前。道。哥請受用罷。他走出去了。那錢癖急急忙忙扯下褲子。也不暇脫\footnote{也是個色癆。}。跨上身來。挺着一根鐵硬的孽具。亂搗亂戳。尋不着路頭。急得他低頭一看。因那陰門腫得翻着。故此門都沒了。他忙用手送了進去。如乞兒打肋磚一般。死力一場混弄。嬴氏起先覺得好些。此時被他拿出築牆的手段來一陣混搗。搗得那床板亂動亂響。倒反又疼得難受。屁股是打破了的。在光板子上一頓亂揉。疼得眞個要死。只得含着淚。將衫子衿兒咬着死捱。正在難受的時候。忽見上邊不動了。知是了。心中暗道。造化。逃得命了。因天熱。那錢癖弄了一身臭汗。拔將出來。提着褲子走出去納涼。那婦人定了一定。捱着疼慢慢的掙起來。歪着屁股坐着。用手一摸。兩腿鮮血淋漓。窗上月光有些微微亮影。看得明白。陰中黏黏達達淌了滿股。又沒有個甚麼擦。只得將鞋脫下。把裹脚打開。扯下些來。將股上的血擦抹乾淨。將陰戶也擦了。手指又疼。勉強着剛收拾完。才待穿褲子。只見色癆跑進來。向前摟住。親了個嘴。道。你這人好沒良心。若不虧我。此時不知如何受苦呢。就不謝謝我。不由分說。將他放倒。那婦人疼得動不得。又不敢強。只是憑他。那色癆忙自己脫了褲子。弄將起來。因有餘精在內。滑順得比先略可忍些。那色癆在門外看他兩個弄了一會。火動久矣。不多幾下。那婦人覺得那牝戶中跳了幾跳。就不見動了。暗道。這還好些。色癆把褲子也不穿。只圍在腰中。起身出去。那婦人纔要掙起來。見錢癖拿着個大土碗。點了個明晃晃的燈進來。道。住着。我還要弄弄呢。忙把燈放在牆洞內。爬上身去。不管三七二十一。儘着搗個不歇。弄夠多時。方才完了出去。色癆又要來弄。婦人哀求道。爺。你先前可憐我。討情救我。你此時如何下這狠心。我實在的受不得了。色癆道。我救了你。你就不救我了。我方才弄了不多幾下。沒有盡興。你再與我弄弄就罷了。那顧他生死。上身就弄。這一弄。足足弄了半夜。緊抽慢扯。再不肯歇。任那婦人求吿。他總不理。只見錢癖進來道。你還讓讓我呢。只管獨吃起來了。色癆道。好哥。你在外邊涼快涼快。略等一等兒。我也快了。不住的又抽。錢癖急了。上前要拉下他來。他緊緊抱着婦人。死也不放。拉得狠他弄得狠。口中只叫。好哥。你只當積陰隲。再容我一會兒\footnote{這色癆二字纔寫得滿足。}。這一拉一掙。用力分外猛大。揉得那婦人屁股疼得列(到)心裡去。身子又被他壓緊。氣也出不來。婦人氣恨沖心。方知道他二人通同做的圈套。料道哀求也是無益。就是手好也推他不動。何況手又疼。氣迷了。就像死人一般。憑他肏搗。那色癆緊抽了一陣丢了罷。爬起來。錢癖又上。幹訖一度下來。乏倦了。對色癆道。兄弟。我夠了。讓你受用罷。我睡覺去了。走倒在那一張床榻上呼呼的睡了。那色癆滿心歡喜。道。他睡了。讓我來獨享。又爬上身來弄聳。此時婦人迷一會醒一會。也疼木了。眼淚也流乾了。醒轉來。他還在上邊弄呢。把身子直挺挺的。動也不動。撂了憑他。暗恨道。小時做了不長進的事。以致爹娘撇了去。嫁到鄔家。好端端的過日子。被這賊禿奸騙。到今日受這樣的荼毒。況官府說還要官賣。不知此身落在何處。待要尋死。諒又不能夠。千思萬慮。甚是傷心。又想起鄔合的情愛來。難抛難捨。又悔又恨\footnote{後得爲良婦者。此一悔一恨之力也。}。嗚嗚的哭。却沒眼淚了。看看天已大明。聽見外面敲門來帶人犯。色癆還在高興。緊一下慢一下的弄呢\footnote{眞是色癆。}。聽見了。忙忙下身來。自己穿上了褲子。替婦人也把褲子穿好。又替他趕忙繫了褲脚帶。看見他頭髮揉散了。披散了一臉。慌忙替他亂挽上。扶將起來。推醒了錢癖。扶着婦人出去。開門交人明白。他二人關了門。欣欣得意。重復大睡。不題。他二人可謂。

\begin{quotation}

此時關門監裡睡。少刻禍從淫上來。

\end{quotation}

這衙役將婦人扶着。剛走到儀門外。一眼看見那鄔合同一個人站在那裡。他又羞又怕。羞的是沒臉見他。怕的是他心中懷恨。恐稟官加責。眼淚汪汪。低頭含愧。鄔合見嬴氏臉如菜葉。髮似蓬鬆。人形都脫了。只見他。

\begin{quotation}

面容灰黑。喉間嘶隱痛之聲。頭髮蓬鬆。眼內滴傷心之淚。一雙手血跡模糊。兩隻脚拖鞋拽帶。因同那大小兩和尚做了幾夜快活道場。却被那色錢二禁子弄成這般狼狽形妝(狀)。

\end{quotation}

鄔合看見他這個樣子。心不忍見。點了點頭。嘆了兩聲。你道鄔合爲何來得這樣早。因昨晚知縣審事時。他有個朋友叫做鮑信之\footnote{此處出鮑信之。第一次報信與鄔合。}。他在縣中也有些勾當來。親眼看見事完。回家路上恰好遇見鄔合。把嬴氏的事說與他知道。又道。官府傳你。回說不知你的去向。明早傳官媒領賣。鄔合這兩日因宦蕚同賈童正在初交之時。終日會席。他在兩三家幫閒。兩日未回。竟不知道。今聽說妻子已經拿獲。明早官媒領賣。忙別了鮑信之。如飛跑到宦家。將關大門。煩人進去說了。宦蕚發了名帖。明早着長班去說情。將婦人給原夫領回。鄔合就在他家住下。天未大明。就約長班同往。到縣中時。知縣尚未上堂。他拿了錢煩代書寫了張領呈拿着。同長班在儀門口等候。不多時。堂府升堂。喊堂開門。長班看見帶進婦人。他同鄔合也就跟了進去。衙役認得是宦府中的人。誰敢擱阻。只見衙役上前跪稟道。犯婦到。那婦人跪在丹墀之下。又見一個人跪上去道。官媒伺候。官兒正要吩咐。那長班忙將名帖雙手高呈。走到公座傍邊遞上。將家主來意說了。知縣自然肯做分上。問道。他丈夫在這裡麼。長班道。在這裡伺候。遂叫鄔合。那鄔合聽叫。走到丹墀中間跪下。雙手舉着呈子。門子接了上來。鋪在公案上。官府看了。問道。你還情願要這妻子麼。鄔合叩頭道。老爺天恩。小人情願領回。知縣道。旣如此。你帶了去罷。那鄔合又叩了個頭起來。方要去扶那婦人。只見嬴氏高聲喊道。靑天爺爺救命。這一聲叫。把鄔合嚇了一驚。恐他妻子不願回去。別有甚話。怕官府見罪。那官兒見他喊叫。疑鄔合是假冒來領\footnote{掩卷試猜。鄔合疑的是。還是知縣疑的是。}。忙叫。將那婦人帶上來。衙役將他帶到滴水簷下。問道。你喊甚麼寃。那嬴氏忿恨塡胸。雖有多人。也顧不得羞恥了。遂將昨夜兩個禁子怎樣夥同奸騙。直到天明。幸得老爺天差提人。方才歇手。不然小婦人的命都被他二人送了。哭訴了一遍\footnote{看至此。方知前二疑都猜不着。原來爲此。}。這獄卒奸淫犯婦是官府極痛恨的事。聽了大怒。喝叫。快拿了來。這兩個凶徒風流了一夜。正在高臥養神。他二人昨晚商議騙這婦人。只說他到底是少年嫩婦。就吃此虧。當堂怕羞。決不肯說出\footnote{二人這一想。可見此事行過多次。受其荼毒者。非嬴氏一人而已。}。據我做書的人料着。大約要是高興一兩次。這婦人沒有吃大虧。他也就忍過去了。這兩個惡奴太刻薄了些。誰知這婦人恨毒在心。不顧羞了。細細供出。不想被官拿來。上前跪下。官府怒容滿面。鼻中冷笑道。你兩個做得好事。又叫那婦人說了一遍。二人情眞罪當。大張着嘴。無可回答。官府切齒甚怒。將滿筒籤全摜下來。吩咐二人齊打。一邊一個。每人重斫四十。徇情者同罪。官府動怒。誰敢徇私。況這兩個惡奴。就是本衙門人也惱他淫惡。下下着肉。打完革役。命托(拖)了出去。這二人吃了一夜的扁食大空心。昏頭昏腦。又吃了這一頓毛竹筍湯\footnote{吃扁食時是婦人的股痛。此時吃竹湯時是自己的股痛。可(何)報應如此之速快耶。}。已是發昏。雇人擡到家中。血奔了心。都做了風流之鬼了。這也是他兩個人凶淫之報。正是。

\begin{quotation}

地獄新添貪色鬼。監中少了愛錢人。

\end{quotation}

知縣吩咐禮房\footnote{細。}。拿帖子回復宦公子。交與長班。又命鄔合帶出婦人。鄔合又叩了個頭。上前扶起了嬴氏。攙着打西角門出來。到大門外扶他站住。央煩長班馬頭去雇了一頂轎子來。將婦人扶上了轎。忙向長班作揖道。有勞衛下。我改日酬勞。相煩先回謝老爺。我送妻子到家。就來叩謝。說畢。跟着轎子去了。頃刻間到了自家門首。開了門。將嬴氏扶出轎來。攙他進去。到房中床上睡下。取錢打發了轎夫。忙忙進來熱了一壺滾酒。整了些菜來替婦人暖疼。婦人吃不下去。他再三勸着。勉強呷了幾口酒。不吃了。他又取了些錢出門。忙到宦蕚處謝了。到藥鋪中買了一大包甘草並幾個貼棒瘡的膏藥。又往香蠟鋪裡買了銀硃。如飛而回。到家。將銀硃調了些。替嬴氏將指頭傷處都擦了。又到廚下熱了一鍋甘草湯。舀在坐盆內掇進來。替他脫了褲子。扶下床來洗瘡。嬴氏手又動不得。鄔合替他洗。低頭一看。見他的陰戶腫大如桃。破爛得像翻花石榴一般\footnote{桃實中開了一朶石榴花。倒也是一種異本。}。他嫁了二三年。鄔合雖不曾嘗着他這東西的滋味。却是常常撫摩愛惜。相會過無數的。今日忽然看見了這個怪樣。驚問是何緣故。婦人流着淚道那和尚狠毒的話說了。又被昨夜那二人作賤得如此。鄔合恨了兩聲。將一塊舊紬帕替他把臀上的血蘸着水拭淨。又將陰戶內外輕輕用指頭掏着洗了揩乾。扶他爬在床沿上。貼上膏藥。抽(抱)他上床。換水替他擦了擦身上。換了件小汗衫。又替他洗了洗臉。把頭髮梳梳\footnote{梳匣如故。旦(但)只少了兩把梳子。丟在土地廟中。}。挽了個髻兒。放他睡下。把夾被蓋上。然後坐在床沿上守着他。那婦人得這一番的收拾。渾身爽利了許多。因想自己作了壞事。以爲丈夫不知如何懷恨。今見他反加恩愛\footnote{鄔合之不恨嬴氏而反憐情(惜)者。亦猶曹孟德見張魯不焚倉庫憐而厚待之意也。嬴氏旣背夫逃走而不拐帶絲毫之物。揆其心。不過因己之無陽而貪和尚之具耳。所謂罪雖重而情可原者。律之以婦道。其罪自不必言。以此拿人此等事論之。或可寬一籌耳。}。十分感激。況連日遇的都是凶徒。那裡有他這種恩情。悔恨從前。反放聲哭將起來\footnote{這一哭。是良心發現處。}。鄔合道。你哭甚麼。你自己做的事。難道倒恨我不成。那婦人道。哥哥。我負了你。我實該死的了。你不恨我。倒這樣疼我。我今生報你不盡。來世變馬變狗報你的恩罷。鄔合道。我同你雖是乾夫妻。數年的恩愛怎麼忘得了\footnote{嬴氏聽得此話。更自抱愧。}。況原是我不是。我一個廢人。把你一個花枝般的少婦躭擱着。我何嘗不悔。這是你被人坑陷說不出來。我也不要你補報。從今後一心一意。安心樂業過日子就夠了。苦楚你也都嘗過了。再不要妄想了\footnote{鄔合這一番的話。眞可死婦人之淫心也。}。嬴氏道。我經過這一番。又蒙你這樣恩情。再生他想。眞是豬狗不如了。這婦人伏養了幾日。陰戶痊癒。棒瘡也好了。他這棒瘡原打得輕。皮打破了。肉未傷重。所以好得快。倒是手指頭有一個來〈的〉月纔好了。此後果然這婦人的慾念全消。就是一時偶動淫心。想起這和尚的狠毒。兩個禁子的凶惡。一點高興樂趣也沒有。又想在衙門中那一番苦楚。任你一丈高的慾火。想到此處。一星也無。他疼愛這丈夫。比那有的更甚。一心一意。十分的和美。話分兩頭。且說那嬴陽同陰氏自南京起身。坐船到了家鄕。雇了乘轎子擡着陰氏。許多人搬着行李。徑到陰老兒家來。此時陰老兒夫婦都是七旬外的人了。忽見女婿女兒歸來。且氣槪軒昻。行李甚富。悲喜交集。忙收拾房子與他住下。過了數日。嬴陽用了二三百金買了一所住宅。把向年寄在丈人家的器皿傢伙搬了去。又添了許多金漆床桌。斑竹椅凳之類。擺設得好不富麗。典了一房男婦使用。買了一個小廝聽叫。一個丫頭服事陰氏。他見丈人丈母年老。就接來同住。那陰老兒見女婿如此體面。竟像是作了顯官榮歸的一般。十分的快樂。那老婆子向老兒誇口道。你當日嫌他是戲子呢。你看看今日這個光景。窮鄕紳還趕不上他家呢。女兒該是享福的人。當日一聽見他家來提。就一心要嫁他。怪不得他今日有這個造化\footnote{有得他誇口。寫盡淺見婦人。還不知他令愛是如何得來的銀錢。}。那陰老兒別無子女。將所有些須的積蓄並房子賣了。都交與女兒女婿。爲養老送終之費\footnote{甚矣。人情之可嘆也。陰老夫婦別無子女。當日嬴陽貧窮時。何不以私蓄付之。靠其養老。今見彼頗豐而反資之。可見非親親之誼。乃勢利二字起見耳。}。後來老兩口皆是嬴陽夫妻發送殯葬。不在話下。嬴陽把門面收拾出三間來。拿出數百金。雇了個夥計。開了個香蠟鋪。俱料理完畢。然後去拜望舊日那些朋友。盡都來回。看見他這個局勢。無不致敬\footnote{千古固然。只見人有了錢就尊敬。更不問是爲何如人。世情可嘆。}。盡來溫房洗塵。熱鬧了幾日。一日陰氏向他道。金大爺我們當日着實承他的厚情。我的意思要備桌酒。你去看他一看。請他來家坐坐。也見我們的情長。嬴陽笑道。你的意思要想他來敍敍舊了。陰氏也笑着啐了一口。道。受了人的情都不想着感謝感謝麼。嬴陽笑道。他的情固然厚。自從他同你往來多半年。我覺你那蓋子上也被他磨厚了好些。可以扯直了\footnote{果如嬴陽所言。大約這十數年來比牛皮還厚了。}。陰氏笑罵道。沒良心的忘八。先的銀子東西算是爲我了。臨起身的時候他送的盤費呢。那時我們要去的人。他還圖的是甚麼。那難道不是他的情。嬴陽道。我同你說頑話。你就發急了。你收拾下。我就去請。嬴陽到了金家。金鑛會着。知他夫妻回來。甚是歡喜。聽得他來請。便道。你請先回去。我隨後就到。嬴陽道。舍下新買的房子恐大爺不認得。請同去罷。金鑛就同他步了來。行至門首。讓進內室。陰氏接着。二人各滴了兩點相思淚。金鑛當他還是當日的樣子。圖來續未了之緣。不想高房大廈。呼奴喚婢起來。肅然起敬\footnote{說盡人情。}。就不像當日相得。嬴陽夫婦讓他上坐。決然不肯。定要分賓主之禮。嬴陽自覺不好意思。讓之再三。不得已。金鑛客位。嬴陽叫陰氏對面相陪。自己打橫坐下了\footnote{這一坐妙極。}。嬴陽道。向蒙大爺厚恩。臨行又蒙厚賜。至今不敢稍忘。金鑛不好稱他嬴大官的了。說道。臺見言重\footnote{稱呼更妙。}。些須微物。何足掛齒。在南京去了這些年。作何貴幹。嬴陽道。不敢。也不過在列位大人門下走動。深承重愛。故戀住了。所以直至今日才回。說着話。丫頭送上菓仁泡的茶來。陰氏拿了一盅奉與金鑛吃了。他〔此〕時一看。陰氏的年紀雖將四旬。丯韻不減昔日。打扮得滿頭珠翠。更覺可人。心愛得了不得。回想起當年去時懷着孕。問道。我記得那年別時。娘子有孕來。後來生了個甚麼。陰氏道。到那裡三四個月。生了個女兒。今年十八歲。已出嫁了。金鑛道。光陰好快。不覺一別十八年了。陰氏問道。府上都好麼。他慘然道。都好。就是賤荆前歲不在了\footnote{爲閔氏做〇地。}。陰氏又道。還不曾續娶奶奶麼。他道。先妻在日頗稱賢慧。也還有幾分姿色。今日也想要娶。但我身邊有幾個人。娘娘是知道的\footnote{是舊相(知)之語。}。倘娶一個醜而潑悍的來怎麼處。只好慢慢看緣法罷了。說着。丫頭僕婦送上酒來。他夫妻要斟鍾。金鑛再三不肯。坐定。不過說些閒話。換席後。陰氏又讓着飮了幾杯。嬴陽知他是陰氏心上的密友。恐他要敍敍舊情。不敢久坐\footnote{韓熙載猶爾。何況嬴陽。}。遂道。大爺請寬坐一坐。我在前邊小鋪中照看照看。就來奉陪。嬴陽去了。陰氏就到嬴陽的位上坐了。與他相近。見丫頭執壺在傍。說道。把壺放在桌上。你吃飯去罷。那丫頭去了。金鑛見他支出丫頭。上前一把抱住。就親了個嘴。道。親親。自你去後。我的魂靈兒隨你去了幾個月才回來。眠思夢想。廢寢忘餐。今日纔得重會。掀起衣裙。伸手入褲中去摸牝戶。陰氏也就欠起屁股來讓他摸。他道。親親。別了你這些年。你這件寶貝還仍然如舊。你可肯賜我一刻歡娛。以消十數年之相思麼。陰氏笑道。我承你深情。還何所顧惜。但我年將四十。半老的婦人。女兒都嫁了人家了。不堪再荐枕蓆。舊情未已。不過是這樣戲耍就罷了。況恐家下人看見。何以爲顏。因反摟過他來送嘴遞舌。與他道。親親。你須諒我。不要怪我。金鑛只顧咂舌\footnote{咂着舌。婦人如何說話。}。且不答應。又將懷解開。把雙乳摸弄了一會。方答道。別的話都是你的謙辭。至於怕你家人看見。這是實情。是我一時情之所鍾。見不及此。如何怪你。陰氏也伸手摸了摸他的陽物。堅硬如鐵。笑道。你可謂老當益壯了。二人笑了一回。怕有人來。各自坐了。陰氏斟了一杯酒。手拿着。敬了他半鍾。剩了半杯。自己\endnotemark[3]吃了。金鑛回敬。讓陰氏先吃了半杯。他吃了半杯\footnote{二人不善學古。啖我以餘桃。當自己見(先)飮。然後再敬方是。}。然後低訴一會離情。講一會相思。少刻。嬴陽進來。金鑛起身謝別。夫婦二人挽留不住。去了。嬴陽回到房中。笑問道。幾千抽。陰氏笑道。放你的屁。這幾年來你看我還同人做這事麼。嬴陽道。舊情人相遇。他如何放得過你。我不信。陰氏道。我實感他舊情。相會訴訴衷曲罷了。果然有事。瞞你作甚麼。他自己扯開褲子。护(拉)嬴陽的手摸道。你看這是弄來沒有。嬴陽摸着笑道。這又奇了。這東西吃了多少野食。今日又禁起口來\footnote{果然奇。實非嬴陽所料。}。陰氏笑道。不虧他吃野食。你如今不知怎個光景呢\footnote{眞使嬴陽沒得答。}。夫妻笑了一會。次日。金鑛送了一分厚下程來。陰氏也送了他許多南京人事。此後像是親戚般常來走動。或遇沒人時。不過頑笑而已。竟不曾雀入大蛤化爲水。過了數月。嬴陽聽得按院將到蘇州。他同陰氏商議要去投狀。陰氏道。你何不尋訪了閔家父親同去。嬴陽道。我也想來不好。倘露風聲。那惡人必殺閔奶奶以滅口舌。不是我救他。反是我害他了。陰氏道。你說的是。遂收拾行囊起身。到了府城。方知巡按已到衙門行過香了。聽得次日放吿。要倩人去寫狀子。因想恐漏風聲不便。他自己也還動得筆。却寫得累累贅贅。照他前在京面稟的話寫了一大篇。次日淸晨到衙門首。遂放吿牌進去。許多人都跪倒高舉呈狀。書辦接了上去呈上。鐵按院取頭一張一看。滿滿一紙。從不曾見此款式。一看名字是嬴陽。忽然想起。也不看了。就把呈子摺了。收入袖中。吩咐道。叫嬴陽上來。衆人接聲如轟雷一般叫嬴陽。嬴陽答應了一聲。在丹墀下忙忙叩頭。按院道。上來。他膝行到滴水簷下。按院又道。你到公座前來。他匍匐到案前。問他道。這狀子是甚麼人寫的。他叩頭道。小的不敢托人。是自己寫的。按院點頭道。好。吩咐道。衆人明日早堂再聽發落。嬴陽在此伺候。掩門。衙役齊聲〖口么〗喝出去。衆人向外飛跑。衆役呐喊。放炮關門。打點退堂。鐵按院叫嬴陽跟着進到後堂坐下。吩咐傳推官刑廳。早在大門首官廳中伺候隨傳。打躬已畢。按院叫放了一張椅子在傍命坐。吿了坐然後坐下。按院問道。貴廳職司風憲。鉏強去惡。職所當爲。如何地方上元凶巨惡也曾訪拿一二麼。刑廳深深一恭。道。卑廳也曾拿過幾名。案牘具在。按院道。捨豺狼而問狐狸。非本院之意也。本院所說者。大奸巨惡耳。豈立豪鼠賊類耶。左右一顧道。廻避。衆人答應一聲。遠遠躱開。嬴陽跟着也走。按院道。嬴陽過來。嬴陽忙走回跪下。按院袖中取出一張狀子。遞與刑廳。刑廳忙立身接過。坐下打開。見一大篇。不知是甚麼東西。從頭細看。方知是一張新樣的狀子。看了一會。看完了。起身雙手繳上。就站在傍邊。按院便不讓坐。滿面怒容道。該廳一府理刑。容此淫惡魚肉無辜。此奴凶惡至此。該廳竟無所聞。也可謂聾瞽之甚了。若有所聞而不敢舉。畏其勢耶。慕其賄耶。不但難免尸位素餐之誚。豈不愧民之父母四個字麼。本院白簡從事。該廳難免居首了。刑廳見按院動怒。上前搶一跪。道。卑職有下情上稟。按院道。起來講。刑廳站起。道。此惡卑職知之久矣。屢欲舉行而不果。皆爲上臺掣肘。時時切齒痛恨。卑職素仰老大人世秉忠貞。不避權貴。昨聞得老大人按臨此地。私心竊喜。以爲定可爲民鋤害。使此一縣人得生。因老大人憲駕纔臨。不敢驟稟。欲候公務消(稍)閒。卑職方敢細呈始末。因向公服內胸前取出一個招文袋。檢出一紙呈上。道。此係卑職訪得此奴惡款。求老大人賜覽。足見卑職非敢欺老大人之語也。按院接過。一面看着。只是點頭。落後看得一款道。

\begin{quotation}

農夫高鳳之女\footnote{烈女。}。年十二時即擅儀容。性端莊。言笑不苟。里中每有春秋社會之聚。鄰家姊妹莫不明妝艷服。趨觀恐後。女則閉戶紡績。未嘗履閾一窺。於是閭巷老幼男女皆目之爲迂。號曰腐頭巾阿姐。不二年。腐頭巾阿姐之名之貌共聞於一邑。求之者卜皆不兆。惟南鄙寠人子朱鑲筮吉焉。時高族有名世勳者。世爲狙獪。工\endnotemark[4]於謟笑。與聶變豹友善。因變豹爲鄕人多怨苦。世勳謀輸粟入太學。又敎其重賄各衙門胥吏。又勸以妹獻京中張皇親。於\endnotemark[5]是變豹出入乘輿\endnotemark[6]張蓋。交結官吏。聲勢傾一方。而人莫敢仰視。每見其冠蓋相望。無不搖首咋舌。世勳鄕居。現充撫軍門胥。變豹常至其家。共謀害人利己之事。久之。窺見烈女美而艷。欲圖爲小星。世勳乃勒朱家退婚狀而強委禽焉。其父畏勢唯命。女聞之即不食。其母患之。倩鄰嫗相勸。女曰。爲儂語朱郞。儂不活矣。誓無二心焉。母泣曰。人盡夫也。父一而已。若\endnotemark[7]之勢焰。夫誰不懼。殺人多矣。未嘗服刑也。兒死。爾父亡無日矣。哀哉。奈何速禍我老㹀。烈女聞之乃食。變豹擇吉來迎。里中姊妹相愛者多泣送之。女則欣然登車。毫無悲戚之容。觀者無不異之。以其先欲覓死。而此時樂往也。阿母哭之慟。或誚之曰。去貧就富。女喜可知也。汝奚泣爲。母哭道。我深知兒心。彼決不苟活。必無生望。我與之永別。焉得不慟。諸人聞之不信。猶有腹悱之者。鄰嫗亦賤之。心鄙其曩\endnotemark[8]者不食之詐。女旣抵變豹家。下車入室。呼世勳曰。役夫。爾則禽獸其行而盜賊其心。夫何使我至於此極也。我生不能食其肉。願做厲鬼以求爾之魂魄也。卒抽衣襟\endnotemark[9]中預伏之利刃。自刺而死。年十有五歲。變豹懼。燬其屍。投之江中。以滅其跡。

\end{quotation}

鐵按院看完。大怒道。據該廳所訪數款。若始末無差。此奴不可一刻留於世者。該廳今日暗暗帶領捕役。都陸續四散起行。途中且不必指出名姓。恐此惡知風逃竄。若到彼拿獲時。即着那崑山知縣嚴解前來。婦女俟放吿後。有親人者。皆着領去。其餘看守。再聽發落。家私查明封貯。其田產有占人者亦並給還。向嬴陽道。你跟了同去。該廳查出閔氏。即付他領回。刑廳打一恭。道。是。嬴陽也叩了個頭起來。只見那刑廳站着不走。按院道。該廳還有所說麼。刑廳一恭。道。職有一鄙言。怒(恐)觸老大人尊怒。故不敢啓齒。按院道。何妨。刑廳道。這兩個太監他毫不知道理。倚欽差二字。妄自尊大。他若知道了。只管在老大人面前來纏繞起來。何以處之。按院大怒。立起身來。將紗帽往上一挺。道。該廳視本院爲懦夫了。本院不但姓鐵。連心膽都是鐵的。本院旣一心癉惡救民。此時就是朝廷有特旨到來赦他。本院捨此官。棄此身。以爲衆民雪恨。也決不肯奉詔。何況於閹狗乎\footnote{好按院。不愧鐵公之後。}。刑廳深深一恭。道。卑職失言了。後到正中。向上一揖。道。卑職吿辭。按院一拱手。刑廳抽身就走。嬴陽也跟了出來。回到衙門。打點的當。連夜悄悄去了。過了兩三日。鐵按院差人去請那兩個太監。那太監以爲是新按院定是奉承他。請他吃酒。還笑道。怎不下個請帖兒呢。初風初水就差人口請。這光景倒也托契。隨即吩咐鳴鑼喝道。乘輿張蓋而來。按院迎着到堂上。分賓主。禮畢坐下。這兩個太監看見又無席又無戲。惟見他一臉怒色。甚是疑惑\footnote{眞不可解。}。問道。老先兒請咱們來。有甚麼見敎的。按院道。有一段奇聞。特請二位老太監來奉吿。他二人呵呵笑道。老先兒是大通的人還不知道。我們知道甚麼奇事。咱們只知服侍萬歲爺。還會穿衣吃飯。說了。又呵呵大笑。按院道。本院未出京時就聞知崑山縣有一個大惡叫作聶變豹。萬惡滔天。昨日沿路來吿他的狀子就有幾百張。內中竟有說二位老太監是他的座主。殺人害人皆二位老太監所使。求本院題奏。本院見了大怒。開諭他們道。二位太監是朝廷家的內臣。豈不知國家法度。況荷蒙皇上天恩。今日欽差到此。焉有不愛百姓的。但非刑名衙門不能爲民除害。安有護庇惡人之理。爾等不許聽人妄言。他衆人執定是眞。且說得鑿鑿可據。本院皆怒責逐去。這豈非奇聞麼。本院料二位老太監決不肯爲此。或有無知小人借老太監的聲名做此犯法之事。但此口碎(碑)一揚。恐皇上聞知不便。故請二位來奉吿。還該出張吿示。曉諭百姓不可妄聽無稽之言纔好。本院也還要差人查訪。有做老太監之名在外生事的。定要拿處。那兩個太監面容失色。你望我。我望你。有話說不出來。掙了一會。道。多承老先兒見愛。咱們回去就出吿示曉諭。他坐不住。吿辭去了\footnote{這兩個太監大約生平來初次方領這樣盛情。}。再說那刑廳先差人密打一角釘封公文與崑山知縣。上批該縣密拆。知縣接着。親自拆開。看了內中事體。他雖素常與聶變豹有首尾。但這是按臺訪犯。可敢護庇洩漏。即吩咐典史暗傳捕快衙役弓兵百餘名伺候。遵奉來文。不敢出迎。將黑。刑廳一乘小轎擡到縣衙穿堂下轎。坐下。略敍寒溫。用畢酒飯。次日五鼓。率領多人到了聶家門口。四面圍住。刑廳吩咐知縣典史進前門。縣承(丞)同嬴陽進後門。又吩咐道。無論男婦大小。見一個鎖一個。不許走脫一名。着縣丞隨將門戶箱櫃皆即封固。俟再淸查。衆人領命。呐一聲喊。打開大門而入。縣丞同嬴陽領着多人從後門打入\footnote{嬴陽可稍洩當年之恨。}。此時都還未起。如甕中捉鱉。手到禽(擒)來。一家大小不曾走脫一個。只他妻子單氏。自從見他哄騙嬴陽之後更加凶惡。屢屢苦勸不聽。後又見他逼死了烈女高氏。他合掌道。天地鬼神亦可畏也。遂長齋繡佛。每日高聲朗誦大慈大悲救苦救難觀世音菩薩寶號。決不肯與聶變豹同床。聶變豹也強過他數次。見他執意不從。只得罷了。數年來。他終日趺坐念佛。虔誠無比。一毫外事不問。數月前一夜。睡夢中忽然驚醒。道。大難到了。我要先去。遂沐浴更衣。坐化而逝。聶變豹念經出殯。不用細說。剛纔葬了。未及百日。便遭此事。聶變豹因淫毒太甚。他妾婢雖多。並無兒女。只他一身。他正同着一個妾精赤條條高臥。衆人掀開被。一伸手。用鎖套上。只許那妾穿了衫褲。也不曾容聶變豹穿褲子。只拿一件長衣與他披上\footnote{衙役亦妙。}。帶了出來。那刑廳在廳上正中坐着。知縣傍坐。捕快帶他到廳前。喝叫他跪。他氣昻昻的道。我又不犯法。我是一個大監生\footnote{眞大。}。我爲甚麼跪。我有甚麼罪。敢來拿我。冷笑道\footnote{冷笑。妙。滿肚皮捂着兩太監也。}。你拿我也罷了。我看你明日怎麼放我。刑廳大怒道。本廳久要拿你。恨我官微力薄。爲人掣肘。今你係按臺訪犯。尚敢如此無狀。左右掌嘴。衙役上前。幾個嘴巴。打得鼻口冒血。他才不敢作聲。刑廳向知縣道。男犯都拿齊了麼。知縣道。都齊了。刑廳道。將幼小者留下。同婦女從妾。命典史看守。衆犯貴縣連夜解往按臺發落。此係憲件。不可稍遲。勿得疏虞獲罪。知縣打恭領出。此時轟動了合縣男女。都來聚觀。看見聶變豹蓬頭赤足杻鎖着。鼻口津津淌血。他家那些助惡家奴。都連連牽牽杻鎖在後。皆合掌道。阿彌陀佛。他也有今日這一日。有的道。他叫做聶驢子。不知他的㞠子有多大呢。衙役中也有恨他的。見他沒穿着褲子。將他衣服前衿拽起。露出那驢腎樣的陽物。一摔一摔的走。他到此時也沒法了。只低着頭。兩邊看的人無不暢快喜笑。小孩子個個拍手打掌的笑道。都快些來看大雞巴耶。婦人們見了他那東西。彼此相顧。盡皆咬唇嚙指。張目吐舌\footnote{這日街上好熱鬧。}。到了縣中。吩咐且下了監。知縣收拾完備。連夜解了去了。且說那刑廳見許多婦女皆鎖繫在廳下。問道。內中那一個是閔氏。那閔氏見衆人中單問他一個。恐說他是寵妾罪重。不敢答應。刑廳又問了一聲。衆役喝問衆婦女道。誰是閔氏。別的婦女指着道。他就是。衙役帶到面前跪下。刑廳問道。你如何到他家來的。閔氏戰兢兢的哭稟道。小婦人原是好人家兒女。被他搶來做。那個妾字還不曾說出口來。刑廳道。不消了。叫嬴陽。嬴陽忙上前跪下。刑廳問道。你看這是你姐姐麼。嬴陽時刻念他在心。雖隔多年。面龐兒彷彿認得。答道。正是小的姐姐。刑廳吩咐道。開了刑具。衙役將鎖開了。那刑廳不知嬴陽的來歷。見按臺諄諄吩咐。可有不作情的。便向閔氏道。你可將你的衣服之類進去拿了出來。跟你兄弟去罷。閔氏先聽說他是那人的姐姐。定睛一看。並不認得。但嬴陽當日是個小孩子。如今將四十歲了。又多年不唱戲了。長了一嘴的鬍子。正在疑心。猛然想起方才叫他的名字嬴陽。疑是嬴旦。心中暗喜。遂叩了個頭。爬起纔要走。只見衆人中一個小女孩痛哭道。娘娘你去了。就不救我一救。閔氏也掉淚道。我蒙老爺天恩開釋。如何還救得你呢。刑廳問道。閔氏。這是你甚麼人。閔氏復回跪稟道。他六歲時沒了父母。小婦人憐他。當義女養了這幾年。今年十三歲了。刑廳道。與這小刻(孩)子何干。旣是你的義女。你帶了走罷。吩咐道。放了他。衙役與他開了鎖。那孩子同閔氏歡喜叩頭謝恩。刑廳道。閔氏。帶這孩子進去。把他的衣服之類也查了去。這明是刑廳作情。叫他拿東西的話\footnote{寫此一女子豈非蛇足。不過特做一勢利之嘆耳。嬴陽候(係)按院所托之人。刑廳不但恩待閔氏。即閔氏之義女尚蒙寬宥。可見勢利二字到處無不可行也。}。閔氏到了房內。將所有頭面盡行包了。繫在腰中。將上好的衣服包了一大包。背了出來。刑廳看見。對嬴陽道。你領了去罷。嬴陽閔氏同那孩子都叩了頭。嬴陽拿着那包袱。歡歡喜喜出了門來。叫了兩乘轎子。閔氏坐了一乘。那孩子坐了一乘。將包袱塞入轎櫃下。一直來家。到了家中。下轎讓入。那陰氏迎進。嬴陽叫鋪子裡打發了轎錢。他到了裡邊。將一張椅子放在上面讓閔氏坐。向閔氏道。奶奶你不認得我了麼。兩眼掉淚。道。若非奶奶救我。安得尚有今日。奶奶請坐了。我好拜謝。撲的跪倒。閔氏也忙跪下。道。我當日救你。你今日救我。我也該謝的。嬴陽再三的讓他。他決不肯起來。嬴陽叫陰氏攙扶。他也不肯。讓了許久。閔氏道。方纔在官衙中旣說是姐弟。你若不棄。我們認作姐弟罷。嬴陽大喜。問了年紀。他比嬴陽大三歲。四十一歲了。讓閔氏受了兩禮。陰氏也拜見了。那孩子拜了舅舅舅母。嬴陽將他那鞋取出繳還。閔氏收了。擺上酒來飮着。閔氏問道歷年境況。今日如何吿理報仇。嬴陽把他家事略敍。把吿狀的話細訴了一遍。又問閔氏的父母住處。閔氏說了。嬴陽去尋了他父母來相會了。相隔了二十餘年始得重逢。痛哭了一場。閔氏對父母說嬴陽救他的事。老夫妻深感不盡。向嬴陽夫妻再三道謝了。接了他母女二人家去。再說那刑廳招吿。那吿聶變豹的狀子有數百張。有白占人家的婦女田產。皆給原主領去\footnote{好。}。餘者候按臺發落。又淸查了他的家私。造了册子。諸事完畢。起身回蘇報院。嬴陽也隨了去叩謝。鐵按院將聶變豹並首惡家奴並皆處死。其餘男婦隨輕重發落。合縣之人無論受害與不受害者。無不歡欣鼓舞。感恩戴德。又差役\endnotemark[10]去拿高世勳。回稱烈女死之次日。即嘔血而死。按臺深以爲異。大書廩(凜)然千古四個大字。勒名於烈女之門。把聶變豹的銀子給一百兩與烈女之父高鳳。爲烈女建祠\footnote{此一事不可少。若漏去。則只能鉏惡不能旌善矣。}。這年正値蘇州一府六縣荒歉。按院委刑廳將聶變豹現存的銀兩。並將家產變賣。賑濟窮民。受恩之民家家尸祝。嬴陽辭了回來。同陰氏商議。請了金鑛來家。陰氏向他說閔氏與他同歲\footnote{此處方出金鑛年紀。}。相貌端莊。生性賢淑。勸他續絃。他見情人說合。必然不錯。就煩嬴陽做媒。閔氏聽說與公子做正妻。又是富家。況係恩弟做媒。焉有不肯之理。金家下禮迎娶。都不消細說。閔氏到了金家。他當日雖係聶變豹寵妾。因胸中有父翁之仇。不過勉強從順。今嫁了金鑛。不但年齒相當。且內才甚妙。恩情甚篤。金鑛見閔氏之姿不下陰氏。覺潘(端)莊過之。又見他相夫以禮。待妾以和。處家之道無不盡善盡美。十分相敬相愛。那嬴陽同這姐姐彼此有相救之恩。金鑛同這小舅姆又有相知之素。惟這門親戚更覺得親厚。不必煩敍。嬴陽這麼個旦而兼龜的人。有這一點報恩的好處。不但成了個好人家。後來竟還做了官。焉知非冥冥之中報之耶。足見人生何不學好。這是後話。再說那鄔合的家事。古語有兩句說得好。道是。

\begin{quotation}

好事不出門。惡事傳千里。

\end{quotation}

嬴氏被和尚拐去拿到衙門的這一段新聞。不幾日。合城皆知。那龍颺也聞得了這話。心中暗想道。這婦人和我好了三四年。生生被他爹娘拆散了。心裡久要想看看他去。替他敍敍舊。恐他夫妻和美。不肯認賬。反弄出是非來。他今旣肯跟人逃走。定然是不喜他的丈夫。聽得說他丈夫成日不在家。我何不踅了去見見他。若有舊情勾搭上了。強似把自己的後竅只管與別人弄。我也弄弄他的前孔何妨。想定了主意\footnote{主意雖不錯。孰不知大謬不然。}。打扮起來。他雖二十多歲。還做賣圈兒肉大臟頭的生意。他年紀大了。比當日更覺在行。會奉承湊趣。所以倒興旺起來。他當日跟着游混公混了兩年。游混公見他長成了一個大漢。嘴上鬍子察(渣)兒也有了。屁股溝子裡的毛也老長的。就把他撇開了。有他兒子游夏流相厚的一個初出世時興的小兔子。叫做楊爲英\footnote{隨手就帶出楊爲英游夏流。何等的省筆。}。他也揷上一脚。父子兩個合包着這個小。這龍小官見游混公另敍上了少年。冷淡了他。他賭氣把嘴上的鬍子撏得乾乾淨淨。屁股溝子裡的毛也拔得光光撻撻。也另相與了個孤老。叫做充好古。原也是好人家的兒孫。自幼酷好小官。他的妻子郗氏\footnote{在龍家小子事中帶出充好古郗氏。甚妙。後來再說他夫婦的事。便不是劈室(空)揑出的姓名。}。生得也甚有姿色。他總棄而不顧。在這一件事上。把個小小家業花得精光。如今手頭短促。不能相〔與〕那時興的兔子了。恰遇了龍颺這一位老小官。他是新出陽關無故人的時候。賤價就售。雖無銀錢。或有酒食。他也就樂從。充好古見他的這種貨物雖不工巧。却甚價廉。不但他摜(慣)拾爛棗。而且想道。俗語說。會嫖的嫖婆兒。會騎的騎騾兒。取他個在行受用之意。他這老小官定與初出世的兔羔子不同。自相與了他。果然枕蓆之間歷練無比。充好古的三魂七魄都落在他身上。把家中無所不賣。替他製了幾件紬絹衣服。龍颺辭了他回來。把他掙的這幾件時樣蜘蛛絲織的衣服此時穿起。你道何爲蜘蛛絲。因他是屁眼中抽出來的。故有此美名\footnote{近來穿蜘蛛絲的人甚多。}。這小子搖搖着一路問到鄔家來。見門關着。只說鄔合不在家。就去敲門。誰知鄔合正在家中。聽見了。開門問道。是誰。却不認得。便道。是那裡來的。那小子見了鄔合。吃了一個定心拳。虧他會隨機應變。答道。我姓龍。原是嬴老爹的緊鄰。他有信來。我來對他姑娘說。鄔合才要讓他進去。聽後面有人叫道。鄔大哥且站着。我有話和你說。鄔合站住了看時。是他一個相熟的朋友到跟前。讓他同入。那人見龍颺在那裡。便道。我不進去了。有句話同你商議。鄔合道。你請站一站。我送這位朋友進去就來。同龍小官進來。叫婦人你出來。你家老爹煩人送信來了。說完。他便轉身同那人說話去了。這嬴氏忽聽見爹娘有信來。滿心歡喜。忙走出來。見是龍家的小子。舊恨在心。忽然變下臉來。因他是寄信來的。不好發作。含怒問道。我爹的信呢。那小子這兩三年沒見他。見他的身子發胖了許多。越發白淨標致。魂都沒了。也不看他的臉色勢頭。恃着宿好。笑嘻嘻的道。沒有甚麼信。婦人道。沒有信。你來做甚麼。那小子笑道。我當日同你甚麼樣的恩情。忽然分開了。我日夜想你。這幾年我要來看你。不得個空兒。每日心裡惦着。近來又知你爲了官事。甚是放心不下。故此特來看看。那婦人聽了。又羞又惱。變了臉。道。各家門各家戶。你非親非故。到我家來放屁辣騷的是甚麼。那小子一團的高興。被他這一掃。也放下臉來。道。你這沒良心的淫婦。從小兒是我破的身。肏了三四年。孩子都養過了。我是你的原夫。你老子嫌我窮。把你另嫁了人。我聽見你跟和尚逃走。捱了拶打。我好意來看你。你不認我。這個樣兒待承我。我到衙門中吿你一狀。說你背夫改嫁。拿了你爹娘來。大家弄到了官。我不圖打魚。只圖渾水。那會你求我就遲了。我還未必肯饒你呢\footnote{話未嘗不妙。眞使嬴氏無可答者。}。這婦人聽了。羞氣得了不得。果然怕弄出事來。又出乖露醜。眉頭一蹙。心生一計。走進房中。招他道。你進來。那小子見叫他進房。必有好處。忙跨入來。婦人低聲道。我同你的情還有甚麼說的。我丈夫在門口。你說話不防頭腦。我怕他聽見。故拿搡話回你。是瞞他的。你怎就惱了。今日他在家。不中用了。你明日還是這個時候來。我和你說話\footnote{寫嬴氏機變之巧。活跳一個伶俐的婦人。}。那小子聽見這話。眉開眼笑。抱着親了個嘴。伸手就要掏褲子。婦人道。看我男人進來看見。那小子道。不妨。我望外望着呢。婦人攔他不住。被他扯開褲子摸着了陰戶。用指頭挖挖。笑道。當日和你弄時。只一條縫兒。如今竟像個大漿口了。婦人笑着推他的手。道。你快去罷。後來有日子頑呢。那小子討了個實話。也就往外走。鄔合還同那人在門口說話。他出來拱了拱手去了。少刻。那人也吿別去了。鄔合進來問道。你爹的信呢。婦人道。那裡有甚麼信呢。鄔合道。沒有信。他來做甚麼。那婦人紅着臉。掉了兩點淚。道。我當日小時在家做了件醜事。要吿訴你。恐怕你惱。鄔合道。你在我家做出這番事來。我還不惱。何況你在家裡做的事。那是隔(個)過去的賬。我惱的是甚麼\footnote{好大肚皮。}。你只管說。那婦人把他當日先要去看小子的陽物。並後來養孩子的話不說。剪頭去尾。只說。我當日年小在家。這個人姓龍。是我家雇了使用的。三番五次哄我奸了。後來爹娘看得有些破綻。把他攆了。我才嫁到你家來。他氣不憤。在大街小巷敗壞我。我爹娘住不住。方搬回家鄕去了。我恨到如今。不好對你說得。今日瞞不得了。實情向你說了。你恕過我罷。鄔合方悟道丈人丈母去的緣故。問道。他無故今日來做甚麼。婦人道。他今日又想來奸騙我。我變了臉罵他。他要往衙門去吿的話也說了個盡情。又道。我哄他明日來。我同你商議。等他來時。你躱在後院裡。他要奸我的時候。我叫喊起來。你拿住他。或打個臭死。或送他到官。纔出得我這口惡氣。鄔合搖頭道。使不得。這一鬧起來。私休不得。一到了當官。你少不得也要出去。他當堂說出舊話。又添一個醜名。婦人道。據你這樣說。明日他來。拿甚麼話回他呢。鄔合見嬴氏這一篇言詞。也知他有了幾分的烈性。還要試他一試。便道。你旣和他有舊情。他來也沒有甚麼歹意。不過想同你敍敍舊情。你和他弄弄。了了他的心願。好好打發他去。也就罷了。何苦又多事。惹是招非的。你要瞞着我做。就是你的不是了。你旣對我說了。我已知道你的心。你只管同他弄。我不惱的。我明日出去讓他。那嬴氏的臉通紅。發急道。哥哥。你把我眞看得豬狗不如了。我做了不肖的事。你還這樣恩情待我。如今就殺了我。我也不肯依從了。鄔合聽說。知婦人是實心改過從善。心中暗喜。又道。你果然恨他麼。恐怕到底有絲毫的情分。婦人道。他奸了我幾年。還負心揚我的醜呢。弄得我父南女北。我恨他深入骨髓。還有甚麼情意。鄔合道。我想在這裡了。倒有一條好計。纔除得這個禍根。不然。你終久被他纏繞不妙。只怕你下不得毒心。嬴氏道。若有妙法敢自好。就是殺了他叫我去償命。我也情願。有甚麼毒心下不得。鄔合見他是眞心。遂向他道。也不用殺他。也不用與他償命。只如此如此\footnote{兩個如此。送了一個小官。}。這般這般。可不出了你的氣。把這禍根就拔掉了。你說可行得麼。嬴氏歡喜得了不得。說道。好好。明日就是這樣行。一宿晚景休提。次日起來。早飯後鄔合要出門。婦人叮囑道。外邊憑着有甚麼要緊的事。今日千萬可要回來。鄔合道。我知道。不用你囑咐。去了。午間。婦人把大門閂拔了虛掩着。坐在房中等他。這小子活該倒運。走將來了。這正是。

\begin{quotation}

豬羊走入屠家。一步步來尋死路。

\end{quotation}

這小子死到臨頭尚不自知。喜喜歡歡走來赴約。到了門口。見門是虛掩着呢。推開走了進來。婦人也笑臉相迎\footnote{這才眞是笑裡藏刀。}。他一把抱住。就要求歡。那婦人道。使不得。我家的今日還在家。纔出去買東西去了。就回來的。你不見我開着門等他呢。撞來看見怎麼了。那小子急了。道。這怎麼樣處。你哄了我來。叫我空空的回去。婦人道。我怎麼肯哄你。今日早間有人來約他今夜吃戲酒。有一夜不得回家。你到日落掌燈後來。我等着你。你輕輕的敲門。不要叫別人聽見\footnote{心深而且毒。}我接你進來。你在我這裡過了夜。明日五鼓再回去。神不知鬼不覺。你道可好麼。那小子當婦人是眞心。他歡喜非常。摟着道。親親。久不見你那寶貝了。我急得很了。將就且見見意兒罷。那婦人道。不好。你留此精神。夜裡憑你弄罷。這會子怕他回來遇見。問你來做甚麼。你怎樣答應他。你快些回去。那小子捨不得。定還要扯開了褲子摸了摸。親了個嘴。他也怕鄔合來撞見。無言回答。只得忙忙去了。日色啣山時分。鄔合來家。手裡拿着幾個紙包兒。又拎着些銀錁白錢\footnote{說猜要做甚事。}。敲門進來。問道。他來了不曾。婦人笑道。來了。就把先的話向他說了。兩個人笑着。將包兒打開。一包是靛花。一包煙子。一包瀝靑。又把前次婦人擦指頭剩的銀硃\footnote{好記性。}也取出來。拿幾個碗裝了。都用香油調好。尋出幾枝舊筆來洗淨。都放在那邊客座桌子抽屜內。又尋出一根曬衣裳的細長繩子來。也放在客座屋裡。找了個棒槌放在手邊。安排停當。專等他來行事。看看天晚。夫妻飽餐了夜飯。點上燈來。約將起更時候。聽得外邊輕輕敲門。知是他來了。鄔合拿着棒槌躱在廚房裡去。那婦人出去開門。放那小子進來。忙把門揷上。走進房來。那色鬼把婦人抱在床上。不暇言就替他褪褲子\footnote{眞可有趣。}。婦人總不推辭。他自己脫得精光。也沒工夫上床。就站在床前。扛起婦人的兩條腿來。將屁股拉出床沿外。燈光下照得甚明。對着一攮到根。一個其大無外。一個其小無內。那小子如渴龍見水。命也不顧。下死力一陣亂抽。不多幾下就完了事了。正在麻歡的時候。被婦人伸手將他的脖子摟過來。把舌頭遞在他的口內。這小子快活得了不得。咂了幾下。那婦人也叫他伸過來\footnote{不意嬴氏竟善兵法。這叫做將欲取之必姑與之。}。那小子忙把舌頭伸出。恨不得連舌根都吐出來送入他口中。被婦人緊緊含住。猛的下力一口。趷噔\endnotemark[11]一聲。齊齊咬下\footnote{也可爲揚他的醜聲之報。}。那小子疼得喊叫不出。一跤跌在地下。婦人忙把斷舌頭吐出。叫道。有賊了。快些來\footnote{果是個偷花賊。}。只聽得房門外喝道。賊在那裡。拿住。不要放他走了。那小子正疼得發昏。耳中忽聽得這話。曉得是被他暗算。也顧不得衣服了。爬起來。精光着就往外跑。那鄔合嘴裡〖口么〗喝。却不進來。他有心算計無心。在房門外等着。說時遲那時快。他纔一隻脚跨出房門檻。屋內有燈。外面黑。看不眞切。被鄔合下死力對準踝子骨一下打得哼的一聲。一交跌倒。鄔合上前按住。坐在脊背上\footnote{鄔合會降龍。}。那婦人也忙穿上褲子。拿出燈來\footnote{細。此等句極易於漏者。}。取過繩子來。同鄔合將他緊緊的背綁起來。那小子舌頭沒了。疼得一聲也無。腿又打傷。又跌得昏頭暈腦。動也不能一動。況這小官只會屁眼中捱那挺硬的㞠子。棒槌打踝子骨上。從不曾嘗過這橫量的木棒槌。他掙挫不得。任他夫妻二人舞弄。鄔合把他綁得定定的。然後起來把他的頭髮打開。婦人已將日間預備的寶貨都搬了出來。鄔合用瀝靑將頭髮替他刷得直竪竪的。然後將油調的紅黑藍三樣顏色。從頭至脚。二人用筆一陣混塗亂抹。彩畫了個花花綠綠\footnote{畫工畫龍原是五彩的。}。將銀錁紙錢替他渾身掛上。婦人向那小子道。你奸了我幾年。我那些兒虧了你。你還四處花敗\endnotemark[12]我。你今日又想來奸我。我且出出氣着。拾起棒槌來。拿那一頭細些的把兒。對準他的糞門。儘力往裡一揷。竟進去了有四五寸。疼得那小子把屁股只是扭。又拿了一根細繩。將棒槌紮緊。繫在他腰間。一頭在糞門內。一頭托在外邊。又找出幾根舊頭繩來。拿了些爛紙拴在棒槌上。像個大尾巴\footnote{這棒椎可名曰華韻。何以言之。曰龍尾。}。纔提將起來。開門放他。那小子得了命。一瘸一跛的纔要走。他夫妻二人各拿了一把錐子。照屁股肉厚處戳了兩下\footnote{此股向日也曾有人愛惜過。今日何不幸一至於此。}。那小子疼得又叫不出來。屁眼內又是棒槌塞着難走。戳得沒奈何。只得瘸着腿一拐一拐的沒命往外跑。鄔合還恐他躱在僻靜處。故意的大〖口么〗小喝。後面攆着。那小子怕錐子利害。直往前奔。鄔合一直送他出了大街。見去遠了。方才回家關門。夫妻笑了一場。上床而臥。他這條巷內竟無一人得知。再說那龍颺跑到街上。已有二更天氣。人都盡了。靜悄悄的。雖有微月。昏頭昏腦。連路都認不淸白。拐呀拐的亂跑。遠遠看見一簇人拿着燈籠。知是巡夜的官來了。轉身往回裡就跑。那官同衆人已經看見。說道。那是個甚麼東西。快快的趕上。衆人一轟趕來。那小子被趕急了。腿瘸着也跑不動。倒站住了脚。有個要人救他的意思。却說不出話來。衆人離得不遠。見他不動。反吃了一驚。仔細定睛一看。從不曾見過這麼個怪物。衆人心裡都有些發毛。膽小的退在人背後躱着看。有幾個膽子大的。高聲喝問。又不見他答應。那小子分明是說甚麼。因舌頭全沒。說不明白。只聽嘴中嗚嚕嗚嚕的叫。那官兒仗着膽子。說道。要是人必定會說話。他只會叫。不是鬼定是妖怪。我們人多。陽氣盛。逼住了他脫不得形。你們快動手打。不要被他走了。那小子也聽見了。着了急。越發奔了人來。要人看看的意思。嘴裡更叫得凶。衆人見他撲了來。心中大慌。想是本官說的有理。到底是讀書的人不同。又恐他先下手傷了人。仗着膽。一齊上前。一頓亂棍。打得腦漿直流。渾身骨折。方敢近前。將燈籠照着細看。方知不是鬼怪。倒是個人怪。吃了一驚。道。這官兒因太通格物。格錯了。默無一言。次日報了察院。差人驗看。唇外血汚。口中無舌。肛門內有棒槌一根。備圖了一個形狀呈上官府。就知是這人定是因奸被人暗算。究無謀主。又無屍親。吩咐地方掩埋。這小子奸了人家閨女。這原是女子先去就他。還情有可恕。世上有幾個魯男子柳下惠。但只後來揚他的醜名。無情負義。他已有了丈夫。今日又想來奸他。其情原自可惡。一死也不爲屈。但這鄔合夫妻也算下得毒手。這個小子的父母見兒子數日不歸。四處尋覓了幾日。杳無踪影。只疑他跟了好龍陽的大花子去了。再也想不到他這一着。這小子也只算個無主的孤魂罷了。再說那鄔合次日到街上。紛紛聽得人說昨夜有一樁奇事。一個人不知作了甚麼壞事。被誰人弄得如此如此形狀\footnote{先鄔合的如此如此是起。這兩個如此如此是結。}。下此毒手。送了一條性命。聽了。回家吿訴嬴氏。除了你的病根了。夫妻笑了一場。有一首詞兒說這獄卒凶淫並龍颺的愚呆。道。

\begin{quotation}

惡毒從無過禁卒。逞凶那懼遭刑朴。嘆嬴氏雖淫。坑他機穽。幾乎就木。堪笑龍颺愚滿腹。想當年風流再續。似投火飛蛾。猶欣欣的。反被情仇戮。

\begin{flushright}右調雨中花\end{flushright}

\end{quotation}

此後這嬴氏同鄔合過得好不和美。鄔合也疼愛他至極。一日。鄔合因有事到城外。忽然聽得一個墳圈內有小孩子啼哭。忙走去一看。却是個一歲來的男孩子。一臉的痘瘡。原來這孩子出的是火症痘兒死了。他父母怕狗吃他。撂在人家墳圈內。這一夜得了露氣。又沾了土氣。復又活了。故此啼哭。鄔合滿心歡喜。抱了回來。叫嬴氏好生養着。過了幾日。痘兒好了。好個白淨的孩子。他夫妻二人知道自己不能生育的了。待這孩子比親生的兒還疼。雖才一歲。也會吃了。買那各樣的糕點餵他。漸漸長大。起了個名字。叫作鄔繼祖。這孩子只知他夫妻二人是他的爹娘。並不知別有父母。連鄔合還不知他是甚麼人家的。何況於那小孩子。後來撫養成人。承繼了他的宗祀。這婦人幼雖淫蕩。到後來改過自新。竟做了一個賢妻慈母。壽考善終。那鄔合眞是。

\begin{quotation}

乾妻反勝實妻。無子公然有子。

\end{quotation}

也受用了下半世。此係後話。不題。再說鄔合那一日領了宦蕚之命邀賈童相會。回家歇宿。這話還在嬴氏被和尚拐去未曾拿獲之時。因一枝筆寫不得兩處的事。此時方又續出。他次日大淸晨起身要往他兩家去。剛出門。遇見縣裡差來的捕快替他拿人。他送了個封兒。又同衆鄰居問了王酒鬼。衆役去後。他方得脫身前\footnote{分得淸楚。接得明白。}去。正然走着。到了一個人家的大門口。看那個門第。若非仕宦門楣。定是富翁的華宅。只見有十來多歲的一個標致後生。身穿得十分的華麗。打着一個小廝。也只有十來歲。打得哭喊連天。滿地下亂滾。足足打了有百數。怒猶未息。氣狠狠罵着。走了進去。鄔合嘆道。一個下人就有過犯。將就打幾下罷了。何苦打到這個地位。做主人的恩寬些也好。傍邊一個老兒笑道。兄當是主子打奴才麼。這是奴才打主子。眞是天翻地覆。有寃沒處訴的賬。鄔合驚問道。請敎老爹。這話是怎麼說。我不明白。那老兒笑道。牆有風。壁有耳。這話對兄說不得。兄也不必問。他說着。就走了開去。鄔合聽了。心中胡胡塗塗。猜測不出。也就去了。你道這老兒說的是甚麼緣故。原來這個體面的後生。姓牛名耕。字希冉\footnote{稀軟的耕牛濟得甚事。不過殺才而已。}。他父親叫作牛質。這牛質有個堂兄。現做顯官。名爲牛解\footnote{弟爲牛之軀質。而兄爲牛心。牛在一戾。可謂牛兄牛弟。}。這牛質家中有數萬之富\footnote{諺云。鄕下人不識麒麟。是個有錢的牛。大的(約)此語因他而起。}。他自幼酷好的是一個色字\footnote{牛也好色。出奇。}。除妻子苟氏之外\footnote{古謂龍交牛生麒麟。他牛與狗交。不知生個甚麼異獸。}。妾婢約有數十。他的房子最大而且富麗\footnote{好個牛圈。}。臥房之後還有一處小園\footnote{閱此偶記起一笑談。一人死去。閻君命其托生。彼云。若依得我的願方去。閻君問。你願去何。彼云。我要萬頃良田一湖水。小小花園在屋裡。一妻二妾和氣美。父做高官子甲科。年終百歲而已矣。閻君道。世間若有此等人。你做閻王我做你。牛質臥房之後有小園。兄(又)有美妻妾。又兄是做高官。竟將同此願矣。}。內中有亭有塘。有樓有閣。曲曲折折。甚有幽致。各處供(俱)鋪設床榻。隨處興到。便同妻婢們高興一番\footnote{各處俱設床榻。本爲自己縱淫之地。孰不知亦爲令政幸奴之所。自作之。妻法之。又何足怪矣。}。他這園中果然收拾得好。但見那。

\begin{quotation}

瀟灑旁軒。高明戶牖。畫貼春宮滿壁。書堆淫艷連床。庭前院內。碧桃相間海棠紅。廊下堦前。芍藥並參枚(玫)瑰柴(紫)。夏月荷花映日。秋來桂蕊飄香。遶屋梅花三十樹。垣牆翠竹幾千竿。欄杆〇字斜連。窗槅衢花掩映。樓開(閣)俱鋪床榻\footnote{一篇贊誥。只爲這一個閣字。爲苟氏幸胡旦之地。}。庭軒盡設枕衾。淫情一動。不拘何處便行。騷興但濃。那管妾鬟混幹。

\end{quotation}

園後還有個小便門通着外邊。時常叫家人們打掃出那些汚穢之物\footnote{小便門是作後來胡旦的入路。欲瞞觀者眼目。假說出汚穢之所。巧甚。}。就不從內室中走。這牛質雖有許多妻妾。總無兒女。他這個好淫。不但這些妾婢是他分中應樂之物。至於家中僕婦。不論精粗美惡。他總放不過一個。都要賞鑒賞鑒他們的光毛肥瘦。又好南風。龍陽戲子也養着許多。眞是一個色精。然而以實論之。是登徒子的傳流。只算得好淫。却算不得好色。他這妻子苟氏。坐(生)得風騷俊美。老(是)個綿裡針笑裡刀的婦人。任憑丈夫娶妾納婢。他談笑自如。毫無慍色。心中雖然醋氣薰蒸。面上從不露一絲形跡\footnote{俗語。咬人的狗兒不露齒。此婦人謂。}。他內中又別有一番心事。待這些妾婢們不但和和氣氣。而且都施些小惠\footnote{他雖有一片狗心腸。若以那一種潑悍之婦較之。連此苟婦不若矣。}。牛質誇他賢德。畏敬他是不消說了。這些婢妾也沒一個不感他的恩私。牛質心愛的一個戲旦。叫個胡可慥\footnote{天下的戲旦皆可用這個名字。}。是蘇州人\footnote{也是蘇州人。蘇州人的祖父多係水葬。生的子孫多與人走旱路。這風水另(令)人不解。}。生得嬌媚如婦人一般。有十七八歲。他不在戲班中算的。只自己家宴。偶然叫他唱幾句。養在內書房中。竟作個婦人粧束。金簪珠墜。儼然一個女子\footnote{此輩即不女粧。枕蓆之間何嘗不儼然一女子。}。苟氏時常見他唱戲。恨不得摟到懷中。一口水吞他下肚\footnote{他之孽具送入肚去還有妙處。吞他下肚何益。}。雖然愛到十分。礙着人多眼衆。無可奈何。只好眼飽肚飢而已。苟氏有一個丫頭叫做紅梅。有二十歲了。生得紅白麻子。着實俏浪。那牛質自然是饒不過他的。但這丫頭年長而騷。主人公的內寵多。雨露之恩不能常波及到他。時常牛質叫他往書房中取東西。他也看上了胡旦。反拿話兒勾他。他一個做戲子的人。這風月調情是他的拿手。況恃着主人公的疼愛。未免膽大。也就想同他做個串字。兩下裡都有心久了。但因未得其便。這一日早晨。牛質叫紅梅到書房中去取健陽固本丹。紅梅到了書房。見胡旦上身脫剝在那裡洗臉抹身。露出一身白肉。下穿一條大紅縐紗單褲。白紬褲腰畫着許多人物。紅梅心愛得了不得。笑嘻嘻的道。小廝家也穿條大紅褲子。你那褲腰上畫的是甚麼。那胡旦正想要調戲他。便把褲腰扯開。拎着那個道。你看看這樣的好故事。紅梅一看。原來畫的都是春宮。他笑得了不得。說道。不害羞的。一個褲腰上畫這東西做甚麼。胡旦笑着。故意把手一鬆。褲子掉了下去。一個㞠子直豎豎。硬而且大。紅梅笑着打他一下。道。好大膽子。我在這裡。怎把你老子的頭露了出來。胡旦就這意兒對面一下抱住。那挺硬的陽物向他亂聳。笑道。你旣不待見他。拿你的皮套子把他裝起來罷。紅梅笑着亂捽。道。你看我可吿訴老爺。胡旦道。你不要假做撇淸了。我兩個今日完了這心願罷。紅梅被他調戲得心花撩亂。做作不得了。說道。這會兒來不得。老爺等着要藥呢。過會兒你等着我。我有空就偷着出來。胡旦摟他親了個嘴。定叫他吐過舌頭來咂了咂。才放了手。取了藥付他拿去。到了午後。紅梅果然偷空溜了出來。他二人成了好事。如此者多次。久而久之。人也就有些知覺。傳到苟氏耳中。苟氏正想個人通線。聽了這話。不但不怒。而反暗喜。一日。帶了這丫頭到了後園一個小閣上坐下。他做了一個笑容。問那丫頭道。我聽得人說你同胡旦私偷。可是眞的。你實說。不要瞞我。那丫頭見針着了他的實病。臉色緋紅。毛骨悚然。不敢答應。把頭低着。苟氏笑着道。這呆丫頭。這件事是人的常情。怕的是甚麼。你實說了。我倒不惱。我要是怪你。肯在這沒人處問你麼。你只管放心的說。那丫頭見主母這樣的開恩。感刻(激)入骨。況且每常主母待人性極寬厚。從不施打罵於奴輩。就說了。諒也不妨。遂跪下道。奶奶天恩。我怎敢欺瞞。事是眞有的。苟氏笑道。你起來。我有話問你。那丫頭叩了個頭。站起。苟氏道。你也同他偷過有多少回數。丫頭道。像有十來次了罷。苟氏笑着道。他年紀小呢。也會弄麼。他的東西也有多大。紅梅含羞笑着。不好答應。苟氏道。你還是才見男人的女孩子麼。怕甚麼羞。你說給我聽。那丫頭紅着臉含着笑。道。他年紀雖小。那個東西比老爺的還粗大些。會弄多着呢\footnote{此段雖與裘氏問春花相彷彿。却無一字雷同。說話行事俱個自別。}。苟氏聽了這話。渾身麻了一下。心窩裡亂癢\footnote{腹內之心乎。胯下之花心乎。}。不由得臉上發起燒來。笑嘻嘻的說道。當眞的。我信不過\footnote{眞不眞。何預奶奶事。可笑。}。丫頭道。奶奶這樣大恩。我敢說謊麼。苟氏附着他的耳朶。道。我同你商議。我今晚借個因頭到這裡來睡。你到書房裡去約下他。晚上叫他在後門口等着。你開門接他進來。我同他試試。看你的話可眞\footnote{凡事何消叫奶奶如此費心。}。你要做的穩秀(妥)。我不但重重的擡舉你。我後來看巧就把你配了他。那丫頭聽見這話。笑容滿面。忙跪下叩頭。道。謝奶奶的恩典。我此時就去對他說。連忙的推了個事故。出去約了胡旦。俟晚行事\footnote{可謂重賞之下必有勇夫。}。苟氏也滿心歡喜。回到房中。打點夜赴佳期。且說天地間造化弄人\footnote{看官拭目看着。}。眞正奇絕。他要總成人做個好人。定有好些湊巧的奇遇。要總成人做個壞人。也使他有個湊巧的機緣。古今來事也多端。不能盡述。即如這個苟氏。忽然一點淫心按納不住。叫丫頭去約了這胡旦。若是不能湊巧。他脫身不得。過了後。或者一回想。自己是主母。那到底是家奴。如何竟鶼鶼比翼。燕燕于飛。做起這樣反常的事來。愧心一萌。翻然自愧悔。豈不使他做了一個良婦。不想剛剛有個空兒。成全了他這淫行。豈非造化弄人。然而又有說者。那勸善錄上有十個大字道得好。他說是。

\begin{quotation}

我不淫人妻。誰肯淫我婦。

\end{quotation}

那太上感應篇上也有兩句說得好。

\begin{quotation}

善惡之報。如影隨形。

\end{quotation}

眞是絲毫不爽。如這牛質貪淫好色。蓄了許多婢妾。雖然也是大過。這還罷了。至於家中的僕婦。雖然都是主人銀錢買來。但他各有丈夫。豈無臉面。豈無恩愛。以主人之勢壓而淫之。內中雖有無恥之流。以賤人之陰得沾尊貴之卵。欣欣以爲榮樂者。然而內中已傷了一點陰隲。或有身居下賤。雖有貞操之心。而爲勢之所凌。不敢不從。你說他這一腔的怨憤可還了得。舉頭三尺有神靈。冥冥之中自然有個乘除加減。折算到他的妻女身上。古語有兩句更道得好。他說是。

\begin{quotation}

淫人妻女。妻女人淫。

\end{quotation}

雖然是八個字。却只四個字。上面的四字。何等之樂。下面只轉換一轉換。何等之苦。仔細一想。這個淫字就可化爲烏有了\footnote{這一段才是作書者之本意。那許多淫穢的事。千言萬語不過是這幾句的引頭。看者須知作者之心。}。閒話少敍\footnote{又瞞人。明是要緊的話。反說閒言少敍。}。且說苟氏得了個甚麼空兒。你道是何緣故。這日晚間。牛質家宴。他夫妻二人上坐。衆妾團團圍繞坐着。歡飮說笑。或彈絲或品竹。或歌或唱。好不熱鬧。這些婦人一個個逞能獻媚。容悅丈夫。那牛質有了幾分醉意。這些妾婢如花團錦簇。他見了這些光景。那裡還把持得住。把這個摟過來親個嘴。那個拉過來咬咬腕。或拿出這個的酥乳來捏捏。或伸手到那個褲襠中去摸摸。這些妾婢見大奶奶在上面。雖知他不吃醋。到底畏畏縮縮。跼跼踖踖的。苟氏見了這個機括。心中暗喜。便立起身來。說道。我在這裡。你們未免拘來(束)。我的酒也夠了。我到後邊小閣上去睡。讓你們暢快頑耍罷。只叫紅梅同我作伴去。別的丫頭都在這裡伺候。牛質大喜。吩咐點燈。衆人恐怕他是心懷醋念。還再三勸留。他決定不肯。牛質道。奶奶是極賢慧的。倒讓他隨意罷。衆妾要送。他也止住了。只同紅梅點上燈籠而去。這牛質以爲苟氏去了。省得衆妾婢礙眼。且痛樂一番。那裡知道他賢妻也去別尋樂境\footnote{他同這些婦女取樂。不過是個蠢牛。那苟氏只同胡旦一個取樂。牛便化而爲龜矣。}。苟氏的一隻小脚只剛三寸。每常自臥房中到堂屋內。不過數尺之地。必須要扶着個丫頭。一步挪不得幾寸。略跨遠些就像要跌倒的一般\footnote{寫盡浪婦嬌態。世上貴(眞)有此類。}。此時園中係鵝卵石鑲的路。七高八低。雖有燈籠照着。到底有些黑影。只聽得他兩個高底板兒趷噔趷噔的響。走得飛快。紅梅穿着平底鞋。反落在後邊。趕不上他。由不得心中暗暗失笑。到了閣上。紅梅忙點上大燭。爐中爇上香。繡帳高懸。錦衾鋪設停當\footnote{這等華麗鋪設。却與狗睡。殊覺可惜。}。苟氏心忙意急。催他快去接胡旦進來。紅梅也不拿燈。黑影中悄悄的去了。這苟氏雖然淫興發作。但自己是主母。且年紀尚未三十。未免有些含愧。心中暗想。若對了面。到底不好意思。兼之又無寒溫可敍。不如先脫了衣裳睡下。等他弄過之後就罷了\footnote{奇想。先脫了衣裳便好意思乎。}。脫衣睡下。不多時。只見紅梅來說道。他來了。苟氏道。叫他上床來罷。那胡旦忙脫光了上床。也無可說者。鑽進被中。見他已是精光。就上肚子弄將起來。胡旦先見紅梅約他時。聽得主母有這樣大恩。拿臍下的這件美物賞他。無可報恩之處。就把主人公放在書房中的春藥酒吃了許多在肚裡。安心來奉承奶奶。那苟氏也有些醺醺醉意。酒興正濃。色興大熾。見他弄了進去。十分高興。覺得比牛質的果粗大些。幹法也甚得竅\footnote{這是自然。蠢牛如何及得媚狐。}。而且工夫更久\footnote{藥酒之力也。牛質泡的酒。不想反作成了胡旦。}。欣喜非常。一連丢了兩度。叫他暫歇。苟氏見他年甚靑春。身材小巧。心愛不過。就馱在他腹上\footnote{只聞得狼與狽彼此相依。這狗腹上馱着個狐。倒是新聞。}。摟着親了他一個嘴。舌吐丁香。彼此含咂了一會。苟氏心愛他不過。隨口編了個駐云飛贈他。道。

\begin{quotation}

你二九靑春。嬌媚嫣然美婦形。你把紅褲褪。好個風流棍。噫粗大勝良人。堅而且硬。直搗紅心。如入迷魂陣。把你做異寶奇珍並看成。

\end{quotation}

胡旦感主母相愛之甚。也就接口編了一個。在苟氏耳畔嬌聲低唱。道。

\begin{quotation}

恩似天高。賞我褲中這美槽。緊暖香乾妙。絕世風流竅。嗏情愛盛而饒。將何爲報。惟有竭力驅馳。稍盡忠和孝。但願你日久天長永不囂。

\end{quotation}

苟氏聽了。愈加歡喜。緊緊的摟了他兩摟。笑說道。你年紀小小的。被窩中的事倒這樣在行。不枉我失身一場。你若如我的心。我就地久天長的同你作樂。後來但是有空。我就叫紅梅來叫你。你要始終心不改變。我久知紅梅同你有私情。我就把他配與你作妻子。那胡旦聽了。感恩無地。他長了十來多歲。只遇過紅梅一個。在書房中做那私偷的事。急忙急促。不過苟且適興而已。今見苟氏千般嫵媚。萬種風騷。吟吟笑語。不覺魂消。且要博主母的歡心。圖賞妻子。又竭力奉承了一陣。苟氏覺比每常同牛質交合賽過許多。樂到十二分地位。又伸舌頭叫他咂了一會。那胡旦鼻口聞得他脂香滿唇。口中嘗得他甜唾融心。在肚子上又抽抽扯扯的動作。苟氏心疼他年幼。怕他弄傷了。便道。你也丢了兩次了。且下來養息養息着。那胡旦也就依他下來。苟氏拿隻左臂與他枕着。用右手將他渾身撫摩。遍身光膩異常。十分心愛。又用指頭探探他的後庭。笑道。老爺每常同你弄弄這個。你也有趣麼。胡旦也笑了笑。也拿手摸他身上。滑溜如脂。先摸了摸酥胸嫩乳。漸次摸到那又肥又凸的那件妙物。他掀開被。縮身下去一看。只見又紅又紫。小小的一個花心。燈光射着微微的幾根毳毛。興又大勳(動)。就側身摟抱。兩個人挺觸了一會。都乏倦了。互相摟抱着睡了一覺。醒來又親嘴咂舌。兩人調笑上興來。又上身弄起。正然兩下綢繆。看看天色漸漸微明。苟氏只得歇住\footnote{狗極降狐。狐極畏狗。不意他兩個竟如此親熱。古謂貓鼠同眠。不足異也。}。叫他起來穿衣。着紅梅悄悄送他出去。有一個詞兒述他二人這一番幽會。道。

\begin{quotation}

幽房寂寂簾幃靜。擁香衾。歡心稱。金爐麝裊靑煙。鳳帳燭搖紅影。無限狂心乘酒興。這歡娛。漸入佳境。猶自怨鄰雞。道今宵不永。

\begin{flushright}右調晝夜樂\end{flushright}

\end{quotation}

還有一首花心動的詞說苟氏。道。

\begin{quotation}

仙苑春濃小桃開。枝枝已堪攀折。乍雨乍晴。輕寒輕暖。最動芳心時節。狡童嬌秀堪相亡。且偷閒相攜。調舌弄圈套。同衾共枕。鴛鴦帶詰(結)。

\end{quotation}

此後苟氏但是有空。就叫胡旦進來取樂。一日。苟氏行經之後。正値同胡旦弄了一夜。竟受了胎。到了四五月上。那牛質知道。喜得非常。那知是個野種。不意那紅梅也是月事淨時。牛質偶然同他高興了高興。誤打誤撞。也竟得了孕。自從胡旦被苟氏占去。他一副精神心力全注在主母身上。並未曾與紅梅沾身。這明明是主人公的嫡種了。不想苟氏已知道這丫頭肚中有了丈夫的根芽。他因自己腹中有了寶貨。明日生下來豈不是個異寶。設或紅梅也結了子來。不免分了些寶氣去。心生一計。這一日。苟氏的生辰。家宴唱戲。飮夠多時。正本完了。苟氏點了一齣必正偷詞。一齣西廂上的書館佳期。叫胡旦唱。胡旦先裝鶯鶯會張生的那種嬌羞。看得好不動人憐愛。後來又裝陳妙常。那番浪態沒一個不動起火來。那牛質歡喜得只是笑。連飮了十數觥。也有幾分醉意了。苟氏留心看他那樣子有些模模糊糊的了。忽然指着胡旦。向他道。這小廝倒唱得好。他伺候你一場。我賞他個老婆。你說可行得麼。牛質不但心愛苟氏。要遵他的言語。且又愛胡旦。聽了這話。笑着道。這是你的恩典了。苟氏道。這樣個好標致小廝。醜丫頭也配他不上。就指着紅梅道。我這丫頭也還生得端正。好配他做個賀新郞罷。倒是一對好夫妻。牛質並不知紅梅腹中有物\footnote{物者。犢也。}。何況且配了胡旦。寄之外府如收之內庫是一樣。何礙於時常取用。便道。你旣念奴嬌。賞他個好姐姐。有何不可。點頭應允。苟氏恐怕他的酒醒後有變。說道。今日趁着我的好日子。就在內書〔房〕裡權做他二人的洞房。改日再撥房子與他。遂吩咐管家婆與他收拾。他是財主人家。何所沒有。衣裳被褥首飾盆鏡之類查些賞他。把個紅梅打扮得花花綠綠。當夜配了下去。即成好事。他夫妻二人。在紅梅是久旱逢甘雨。在胡旦是床中遇故知。一上床就殢雨尤雲起來。感念奶奶不失前信。抽一抽。齊叫一聲奶奶。那紅梅到了樂極的時候。他一連叫了十數聲我那知疼着熱的好奶奶喲。那知是奶奶的一條妙計。過了數月。苟氏生了一子。合家歡喜。牛質是不必說。三朝滿月。那親友都來慶賀。熱鬧非常。那紅梅配了胡旦。只五個月。也就生了個兒子。他夫妻兩個都知是主人的親種。但怎敢送了上來。少不得認爲己子。牛質算了算。也知是他的骨血。此時若苟氏無子。他也就只得認了。今見苟氏已產了麟兒。況丫頭又配下去將半年。這孩子也有了些雜氣。故此就不要他。那裡知那丫頭生的雖染了些兔子雜氣。還是眞正牛種。這正夫人生的毫無牛氣。純乎兔而且雜。這雜種就是牛耕了。打的這個小廝就是紅梅所生。與他同歲。豈非奴打主子乎。還有一件異處。這牛耕生得嬌嬌媚媚。與胡旦的模樣竟相彷彿。那個小廝粗粗實實。行動言笑與牛質一般。這牛質心中也常想。奶奶所生之子雖類胡旦。但苟氏極美。母美兒子亦美。自然之理。他並不疑有別故。但紅梅之子全像自己。旣從小不認。大了如何相認。只得罷了。這小子就服侍牛耕。每每他主僕在一處。這家中的妾婢以及下人。無不暗暗指指搠搠的談笑。他一家皆知。街鄰因而知道。所不知者。就是牛質與牛耕假爺兒倆人耳。這妾婢們都感苟氏相待之恩。且事關重大。誰肯做寃家說破。這日。你道牛耕爲何毒打這小廝。牛耕向人家尋了一個小哈叭狗兒。每日叫這小廝抱着。此日偶到大門外。不防那狗一下跳在地下亂跑。恰値街上一條大狗趕上。一口咬死了。所以牛耕怒恨打他。且說牛質自苟氏得子之後。他常常得意。念那兩句古語道。

\begin{quotation}

無官一身輕。有子萬事足。

\end{quotation}

向妾婢們道。這兩句剛剛合着我了。喜樂非常。又過了二三年。有一個私窠子計氏。生得甚美而騷\footnote{大約婦人美者無有不騷。若醜婦雖騷。有誰愛之。}。他有一個癖好。凡與男人交合時。單要弄他的後庭。不喜幹他的前面\footnote{這才眞是雌兔不懼。龍陽妒殺耶。}。他有一番的講究。道。男子中堅久長大者少。短\endnotemark[13]小不濟者多。果然陽物大。本事好。在前面盤桓。自然有許多的妙境。若遇那短小而不濟者。不但弄在內中全無知覺。且正在興濃之時。他忽然中止。眞使人心中十分難過。至於後路。男子中大也可。小也可。長久固佳。快亦無礙。那快的。他縱完了。我心中亦不覺怎麼。遇着長久而大者。不但其樂無窮。即前面亦有樂趣。因此十次之中倒有八九次是走後路。他又有許多的妙想。恐有愛潔淨的人嫌此地穢汚。設或有糞屑帶出。豈不爲人憎惡。臨弄時。他將紫菜木耳用水泡軟\footnote{這兩種是和尚家的菜蔬。不想他拿了款待小和尚。}。拌上許多的香末。先塡入後庭中。同人弄時。不但一點穢物帶不出。且抽得有許多香氣撲鼻。有一個賞鑒家取遼懿德皇后的十香詞內一首道。

\begin{quotation}

解帶色已戰。觸手心愈忙。

那識羅裙內。消魂別有香。

\end{quotation}

就把別有香三字做了計氏的雅號。牛質聞得別有香的盛名。去嫖了他幾夜。果然枕蓆之上妙技超群。心愛至極。用了將千金弄了他來作妾。以供後庭之樂\footnote{胡旦奈何。}。只交七個月。便生下一個女兒。牛質暗想道。我自得了他。只在陸路驅馳。從不曾水門來往。何得忽生此女。雖知這娃娃來路有些不明。因沒有多的兒女。也就葫蘆提認了。反向人拿話掩飾道。人說七成八敗。七個月生的頗多。倒是八個月的養不大\footnote{一笑談云。有數人閒話。因講起婦人生子女。七個月的養得。八個月的養不大。內中一個道。也沒有這話。我家祖當年就是八個月生的。那人道。令祖旣是八個月生的。到底可養大了沒有。}。因計氏叫做別有香。這女兒是他生下來的。可接了下一字。乳名香姐。家人都稱爲香姑。可笑這牛質自己的親骨血明知不認。倒作了家奴的兒子。却拿這一男一女兩個雜種當作親生。豈非天斬其嗣。以償貪淫之報耶\footnote{又是一番喝棒。}。且按過一邊。不知鄔合如何去邀賈進士童財主。與宦公子如何相會。這賈進士與童財主叫甚麼名字。是何出處。要知道他的事蹟。再聽下回分解。

姑妄言七卷終



\endnotetext[1]{「總成」原作「成總」,據文義改。}

\endnotetext[2]{「計德」原作「計得」,據上文改,下文或同。}

\endnotetext[3]{「自己」原作「巳自」,據文義改。}

\endnotetext[4]{「工」原作「上」,據陳鼎《留溪外傳》卷十五《高烈女傳》改。}

\endnotetext[5]{「於」原作「提」,據陳鼎《留溪外傳》卷十五《高烈女傳》改。}

\endnotetext[6]{「輿」原作「與」,據陳鼎《留溪外傳》卷十五《高烈女傳》改。}

\endnotetext[7]{「若」原作「茗」,據陳鼎《留溪外傳》卷十五《高烈女傳》改。}

\endnotetext[8]{「曩」字原缺,據陳鼎《留溪外傳》卷十五《高烈女傳》補。}

\endnotetext[9]{「襟」原作「禁」,據陳鼎《留溪外傳》卷十五《高烈女傳》改。}

\endnotetext[10]{「差役」原作「羞沒」,據文義改。}

\endnotetext[11]{「趷噔」原作「吃噔」,據下文改。}

\endnotetext[12]{「花敗」原作「化敗」,據上文(第六回)改。}

\endnotetext[13]{「短」字原置「少」字之上,據文義改。}

\setcounter{footnote}{0}

\theendnotes

\part*{姑妄言第八卷}
\addcontentsline{toc}{part}{姑妄言第八卷}
\markboth{姑妄言第八卷}{姑妄言第八卷}

鈍翁曰。寫賈明之䑛犢。莫氏之姑息。曲盡老人愛子。繼室疼兒。說透人情。至於賈文物之囗食(含)香。皆宦家少年所必有之事。寫得逼眞。

富氏一驕暴女子。却是個大家風味。並不是窮家小戶潑婦的樣子。富氏舉動行事。帶着富宦之女驕傲的體段。與侯氏鐵氏毫無一同。所以爲妙。

魏忠賢之來歷。祖孫父子如此家世。竟得居一人之下。肆其凶惡。罪忠賢者十之三。而罪任忠賢之天啓則十之七。其五虎五彪。及舉朝之乾兒廝養。皆天啓之過。其意何居。若天啓不任忠賢。此輩安能流毒於正人君子。幾幾一網打盡也。

阮大鋮父子聚麀。無嬌嬌焉得有此事。無嬌嬌又焉得有寶姑。無他母子二人。又焉得有家門之醜。郟氏之私愛奴。寶兒之私阮優禿小廝馬兒騾之輩。阮最阮優之私嬌嬌。雖寫衆人奸淫之惡。實總歸現報於阮大鋮一人而已。這叫做君子惡居下流。一家之惡皆歸焉。

這一回內通篇都是淫艷之事。從中有楊公劾魏忠賢一疏。被這些淫惡的人一襯。更覺忠義凜然。許多淫褻之語。不但不玷楊公之疏。反足以更顯其辭。壞人壞事亦有可用之處。

世之惡淫書者。恐導人以淫耳。此書可謂淫笑(乎)。須看他淫之報應爲何如耳。此一回內阮最淫庶母。郟氏便私愛奴。嬌嬌叫阮最偷己之婢。欲塞其口。以便同他往來。郟氏便叫愛奴偷己之婢。以便往來。針鋒相對。有絲毫舛錯否。更有妙者。嬌嬌愛阮最未必到十分地位。郟氏之愛愛奴竟到十二分。此有深意。謂淫婦之罪雖一。無足重輕。以男子言之。愛奴一無知之僕耳。僕奸主母。罪固應死。而較之阮最。讀書子弟奸淫庶母。其罪更浮於愛奴矣。故寫其妻之淫濫更勝於嬌嬌也。看到此等處。即有生性極淫之人。亦當心悸毛豎。尚敢起一點淫念否。余謂即作勸世文。未必有此等說得令人可畏。尚可作淫書觀耶。

阮最之私嬌嬌。尚足以情求。以情合。阮優之所爲。嬌嬌雖未必屈心相就。然而竟是以強上。所以後來二人之死有輕重之分。

或謂阮最阮優二名儼然音似聚麀。太覺顯而易見。阮大鋮豈不知二字非佳耶。爲子命名。決不如是。余曰。不然。王安石生封荆公。死贈舒王。豈彼之黨羽竟不知此二字之不佳。而竟全然不悟。且永樂竟用方臘年號。豈當時在朝諸公皆不讀書者耶。此乃天惡惡人。使當局者盡迷耳。

阮最阮優爭鋒一段。必寫賽紅張見者。好做嬌嬌阮最死時。毛氏審問賽紅。他便和盤托出。使人人皆知。不然。彼自爲之。孰知之。不如此寫。焉得知阮氏之門風若此。罵大鋮如何罵得盡情。

金瓶梅一書可稱小說之祖。有等一竅不通之輩。謂是西門慶家一本大帳簿。又指摘內中有年月不合。事有相左者爲謬。誠爲可笑。眞所謂目中無珠者。何足與言看書也。如此書中說阮大鋮家事。大鋮逢迎逆璫。僅七年耳。今自彼得嬌嬌起。至後嬌嬌死。將二十年。屈指所差多矣。此不過欲極辱大鋮。以雪衆忿。不如此寫。不足以盡其惡。倘又有聖嘆所謂冬烘之流見之。又必摘其謬處。但作小說者。不過因人言事。隨筆成又(文)。豈定要學太史公作史記用年月表耶。大凡書遇此等不通人持看。亦書之一厄。誠所謂如之何者。吾莫如之何也已。

\chapter*{姑妄言卷之八\\
第八回 賈文物借富丈人力竟得甲科 鄔幫閒迎宦公子意走邀富貴\\
附 魏忠賢履歷新奇 阮大鋮家庭特異}
\addcontentsline{toc}{chapter}{第八回 賈文物借富丈人力竟得甲科 鄔幫閒迎宦公子意走邀富貴}
\markboth{第八回 賈文物借富丈人力竟得甲科 鄔幫閒迎宦公子意走邀富貴}{第八回 賈文物借富丈人力竟得甲科 鄔幫閒迎宦公子意走邀富貴}

話說那鄔合見那少年打小子。雖聽那老兒說了兩句胡塗的話。心中猜詳不出。也不便再問。就順便先到賈進士家來。這賈進士名文物。乃賈翰林之子。賈翰林名字叫做賈明\footnote{名字旣假。其人非眞可知。}。做過一任主考。年老無子。致任(仕)家居。前妻王氏早故。後娶了一個莫氏續絃。到七十歲上邊才生了這賈文物。正合着蘇東坡的二句道。聖善方當而立歲。頑尊已及古稀年。他這樣年紀纔得了這個命根。夫妻愛這兒子視同至寶。自不必說。七八歲上筵(延)師敎他。倒也聰明。只是一心務外。不肯讀書。他父母又恐拘管懷(壞)了兒子。凡事只假推不知\footnote{方合尊姓。}。賈文物到十歲上就會作怪。看見家中婦女。無人處就去摳摳挖挖。丫頭僕婦們去溺尿。他就躱着張看。人見他年小。也不理論他。莫氏知道了。恐他年幼。一時間有無恥的婦女破了他的童身。以致生疾。況那個賈老兒也是個掛名丈夫。八十歲的人了。起坐還要人扶。那裡還有風流的興致。遂留了兩個大丫頭服侍他。只帶了一個十二三歲的小丫頭叫做含香。搬到西屋另住。帶着兒子。每夜臥在身傍。又過了二三年。此時賈文物交十三歲了。竟知識大開。這含香丫頭也十六歲。生得嬌模嬌樣。頗有幾分姿色。他背了母親的眼。就皮着臉同這丫頭打牙撩嘴的頑戲。那丫頭也是有知覺的了。起先還怕主母知道。後也就漸漸膽大起來。也回嘴回舌的調笑。那賈文物久要下手。他想嘗嘗這蛤蚌的滋味。怕他不從。故不敢輕動。今見他說說笑笑的回言。乘機就摟過脖子來親個嘴。那丫頭也不嘖聲。只把頭扭着笑笑走開\footnote{大約也想嘗鰻魚滋味了。}。或把他手上輕輕擰一下推開了。並不言語。總是那半推半就的光景。心中已判了個肯字。賈文物知道好事可成。一日晚間。因他父親痰火上來。他母親照看着。却三更時好了些。方才就寢。熬了夜的人上床睡着猶如小死。他却留心靜聽。見母親睡熟。悄悄褪出被來。爬下床。摸到床後一張矮榻上。那丫頭也因辛苦了。沈沈睡熟。他上床將被揭開。替他輕輕脫了褲子。摸着了此物。光光滑滑一條細縫。用指頭挖挖。緊緊揪揪。他此時雖然愛極了。那心中却撲撲的跳\footnote{理所必有。寫得眞好。}。還怕他或一時叫喊。母親聽見。又將指頭往裡塞塞。那丫頭睡得總不知覺。此時也顧不得了。那小陽物也挺硬起來。他也用些唾沫替他擦在牝中。把自己小㞠子上也抹了些。輕輕分開兩腿。爬上身。用手摸得眞切。\endnotemark[1]將陽物揷了進去。內中其熱如火。那丫頭雖是個處子。但他比賈文物大了三歲。又生得胖壯。所以輕易便弄了入去。毫不煩難。此時丫頭也驚醒了。明知是小主。故意道。是誰\footnote{誅心之言。然而再無有不問者。意雖假而理眞。}。賈文物忙向耳邊道。親親。是我。丫頭道。你還不下去。看我叫起來。那賈文物道。心肝。我想你久了。你救救我罷。說着。忙忙亂抽。那丫頭也是巴不得的事。因主人是賈文物。他少不得也要假惺惺。抽了一會。那小卵中也冒了些淸水出來。他牝戶內不知是血是水。也有些黏涎流出\footnote{妙。黑地有不見也。}。都是初次開葷。不得其中深趣。也覺得比別的東西有些美味。賈文物得了手。仍舊回到母親床上睡下。他二人嘗着了這甜頭。得空就做。就是日間或在無人處遇着。兩人扯開褲子站着。摟得緊緊的抽幾下。親兩個嘴才罷。晚間但是他母親睡熟。就悄悄去舞弄一回。也都漸知其中樂趣。那一夜。他又摸了去同丫頭弄聳。弄得倦了。互相摟抱。不覺睡去。那莫氏一覺醒來。恐兒子蹬了被\footnote{慈母愛子之心。寫得實然。}。摸了摸。却是一床空被堆在一傍。兒子不知何處去了。吃了一驚\footnote{有趣。好防閒。}。還只道他下地小解。等了一會不見上床。就猜料了其中原故。忙下床撥開爐內的火。點上燈。拿了走到床後邊來。只見兒子與丫頭嘴對嘴。四隻膀子摟得緊緊的睡呢。捨不得打兒子\footnote{實情。}。只把丫頭擰了兩把。那丫頭驚醒。明燈之下見主母站在傍邊。忙將賈文物推醒。睜開眼見了母親。又羞又怕。赤條條跳下來爬到床上。鑽入被中而臥。他母親也跟了來。熄燈而睡。到次日。要罵兒子打丫頭。又恐老兒知道氣了他。只得忍住。又防範不得許多。叫兒子到前邊書房去睡。那賈文物這一下來雖不得與丫頭私偷。倒覺比跟娘睡時散誕。瞞着外邊去嫖婊子弄龍陽。無所不爲。他母親也漸漸知道了。生怕他一時弄出瘡來怎處。思量要替他娶房媳婦。方可管他。那時有個戶部郞中姓富。他收過兩次稅。家私巨萬\footnote{旣做過戶部。又收過兩次稅。自然巨富矣。理應姓富。}。久已喪偶。〇〇〇〇〇〇〇〇〇〇〇〇〇無子息。這個女兒却生得醜\endnotemark[2]〇〇〇〇〇〇〇〇〇〇〇〇〇〇〇〇〇〇〇〇〇〇〇〇〇〇〇〇〇〇〇〇〇〇〇〇〇〇〇〇〇〇〇〇〇〇〇〇〇〇〇〇〇〇〇〇〇〇〇〇〇〇〇〇〇〇〇〇〇〇〇〇〇〇〇〇〇〇〇〇〇〇〇〇〇〇〇〇〇〇〇〇〇〇〇〇〇〇〇〇〇〇〇〇〇〇〇〇〇〇〇〇〇〇〇〇〇〇〇〇〇〇〇〇〇〇〇〇〇〇〇〇〇〇〇〇〇〇〇〇〇〇〇〇〇〇〇〇〇〇〇〇〇〇〇〇〇〇〇〇〇〇〇〇〇〇〇〇〇〇〇〇〇〇〇〇〇〇〇〇〇〇〇〇〇〇〇〇〇〇〇〇〇〇〇〇〇〇〇〇〇〇〇〇〇〇〇〇〇〇〇〇〇〇〇〇人雖慕他〇〇〇〇〇〇〇〇〇〇〇〇〇〇〇〇〇〇〇〇〇親生兒子送入虎〇〇〇〇〇〇〇〇〇〇〇〇〇〇〇〇富戶部暗急托媒人。只要〇〇〇〇〇〇〇〇〇〇〇〇〇〇富。都肯與他。恰好這莫氏要替兒子尋親事。對老兒說道。你也有年紀了。兒子也大了。替他娶個媳婦。若生得個孫兒你見見。也不枉養兒一場。你心下如何。老兒道。我年老多病。諸事管不得了。你是他親娘。有個不愛惜兒子的。凡事你就作主罷。那莫氏就叫了媒人來轉尋親事。媒人就將富戶部家中如何富厚。沒有兒子。只有一個小姐。生得人物又齊整。性格又溫柔。又賢又孝。只要尋個有福的好女婿。如今賠的嫁事是不消說。將來這幾十萬家私房產地土。都是女婿承受\footnote{媒人說了許多話。只這兩句是眞。}。這小相公天生的正是那位姑娘的對子。莫氏滿心願意。問他年庚。媒人知他兒子纔十三歲。不肯說富小姐二十多歲了。只說纔交十八歲。因揀選女婿。纔遲到如今。那莫氏算他大着五歲。又想兒子會作怪。媳婦年長些也好管他\footnote{娶了一場媳婦。只這一件合了婆婆的心。}。遂滿口許媒重謝。托他去求。那媒人久受富戶部之托。人但聽見富小姐尊名。便搖頭閉目。富戶部催過多次。俱回沒有售主。今日見莫氏願求。知他必允。走到富家。把賈翰林兒子求親的話說了一遍。又道。不但這小相公生得人品淸秀。且又是獨生兒子。富戶部也知女婿小了十來歲不能相配。只是如今女兒大了。又因醜惡沒人來求。只取他門第並一個好女婿罷了。只得將錯就錯。許了他家。莫氏知道他家富足。將來都是我家之物。竭力鋪排。行聘納采。着實體面。過禮之後不多時。就擇吉日與兒子完姻。那賈文物正與含香恩愛得好。忽然分開。雖在外邊尋些野食。一來年小不老到。二來手中無錢。又不敢問父母要。如何得像意。今聽見替他定了富戶部之女爲妻。不但媒人說他標致。又將來得他家私可做財主。眞喜得打跌。巴到娶親的頭一日。見丈人家過了嫁妝來。富盛至極。無所不備。莫氏將他住房後一進三間收拾了。與他做洞房。富家來的東西將三間屋塡塞得滿滿當當。賈文物這喜那裡說得出來。連莫氏滿心也說他的主意尋得這樣好親家。暗暗歡喜。賈文物又見陪了四房下人。四個小廝。又是四個好標致丫頭。都與含香不相上下。其婢如此。姑娘之美可知。心窩裡喜得亂癢。巴到天晚。過了一宿。次日親迎娶了來家。急得要看看這小姐是怎麼樣一個天仙容貌。不想揭去蓋頭。坐下含巹。定睛一看。嚇得幾乎跌下床來。你道他是怎個嬌容。

\begin{quotation}

面雖不肥。而團團一枚大臉。身雖不胖。而偉偉數尺長軀。兩眼圓睜似杏。雙眉濃掃如鈎。指雖不糙。却短短粗粗如虎掌。足雖不長。却圓圓滾滾似擂槌。項短如虎。聲雄若牛。雖不發怒。而臉上常露凶光。即是喜時。而胸中每存潑味。

\end{quotation}

賈文物暗暗跌脚。幸喜他家陪的妝奩果然富麗。且有這四個齊整丫鬟。一名玉簪。一名金桂。一名銀杏。一名珠蘭。都有幾分姿色。回想道。妻子雖然醜陋。若是賢慧。這幾個丫鬟還可盤桓取樂。想到此處。也就不惱了\footnote{你心如此如此。他心未然未然。}。晚間上了床。雖然有些怕他。如餓鷹見食。也顧不得了。只得替他脫衣解帶。那富小姐也是久矣待字久渴的女兒。況他的性格也不是怕人的。而且見丈夫又是個小孩子。任他做作。賈文物替他脫光了。爬上身。也用些津唾往內直攻。賈文物到底年小。物件不甚雄壯。只微微聽得他口中噯呀一聲。已弄了一個直竅到底。次日。夫妻起來廟見拜堂。那賈文物尚幼。身材小巧。富氏誰(雖)只二十多歲。長成門扇般一個大婆娘。賈文物剛剛只有他多半長。有四句古話改一改。甚合他夫妻兩個。

\begin{quotation}

賈家新郞罔談彼。富氏新娘靡恃己。

兩人堂前相並立。剛剛撞着果珍李。

\end{quotation}

他二人雖是夫妻。究(宛)如母子。賈老兒見了。暗暗搥胸跌足。那賈文物自此以後。他心中雖有些憎嫌。晚來却得像意做事。強似與含香私偷膽怯。這富小姐他做女兒時等了二十多歲。滿擬嫁個魁偉丈夫。做一番大事業。不想嫁了這樣個小孩子。心中甚是不樂。看他也還生得淸秀。自然有大了的日子。又見他每晚定要點點卯。甚是慇懃。倒也罷了。那知這賈文物過了些時。小姐的這件新物吃厭了。又想嘗幾個丫鬟的起來。背了富氏。就望着這幾個丫頭調戲說笑。這些丫頭雖未嘗不想見見姑爺的這個異物。但都知道姑娘的尊性。一些不到。還要打個半死。這個醋甕可是開得他的。那漏脯救飢。鴆酒止渴的事。如何做得。又不敢得罪姑爺。都悄悄來稟命(明)姑娘。富氏想了一想。道。你們聽憑他取笑。不必聲張。只他要動手動脚的時候。就着一個來對我說。此後那賈文物對着丫頭要說些趣話。那丫頭們也笑笑。只不答他。他以爲有情相愛。又拿出那調含香的手段來。漸漸摸手捏脚。親嘴摟頸的。丫頭們也不瞅睬。就是偶然在胯襠中一掏。或在股縫中一挖。那丫頭們也只笑笑。把手推開。並不嘖聲。也無一毫羞怒之色。他以爲都是契厚的了。只等偷空行事。那一日。珠蘭在後院中彎着腰摘花。他悄悄隨去。從後面把屁股一把抱住。要做些風流的勾當。那丫頭只是亂掙。却也不做聲。他口中不住的道。好姐姐。趁着沒人。我們在這靑草地上了了心願罷。抵死不放。正然熱鬧。誰知別的丫鬟已報知了小姐。那富氏悄悄走來。到了後面。夾耳帶腮一個大巴掌。喝道。靑天白日在這裡做甚麼。那賈文物自出娘胎。腦彈也沒人挨他一下。今被這一掌。耳朶中磬響了一聲。打個發昏。急回頭看時。原來是他的令政。又羞又痛。摀着臉往外飛跑。躱到娘房中來。莫氏忽然見兒子面目更色。看臉上紅紫了半邊。嚇了一跳。急問緣故。賈文物先不肯說。盤問急了。方含淚直訴。莫氏纔知是媳婦見敎的。這莫氏當初誤聽媒婆之言。貪他豪富。也不想媳婦醜到這個地位。娶進門來。懊悔無及。又被老兒背地埋怨。說他不打聽明白。娶了恁樣個媳婦來。可惜了我個好兒子。被你作娘的坑了。但已生米做成熟飯。無可奈何。今日見兒子把臉打腫了。要去說媳婦。又恐老兒知道抱怨。況又是兒子做得不是。心中暗急暗疼。只得撫慰兒子道。誰叫你做這樣不長進的事來。叫他打你。你要正經。他敢打你麼。他若無故欺負你。我也好說話。好好的去罷。那賈文物捱到晚間。只得進房。不想被他這一掌把魂都打走了。見了他。不由得心中凜凜害怕。富氏不許他同臥。叫丫頭擡了條春凳。放在床傍與他睡。賈文物不敢違他法度。竟自欽此欽遵。過了數日。莫氏知道了。心疼兒子。反來替媳婦陪話。說兒子年小不知事。你年紀大些。凡事要你照看他。你小夫小妻爲何分開了睡。看我臉面。今晚好好的在一塊罷。那富氏雖然性凶。旣打了丈夫。婆婆還說一篇好話。也就說道。奶奶的話我有不聽的麼。果然晚間仍叫賈文物同臥。那賈文物也知修飾。在被窩中盡力賠了個禮。過了多日。舊性復萌。把前次那一巴掌竟忘了。又是前番那種光景。仍對着這些丫頭胡鬧。他見這些丫頭總不推阻。以爲幾個人都有意於他。決想不到是妻子的一黨。要拿他獻功。連富氏前日撞見。他還說是無心之遇。那裡疑是活耳報神去報的。一日。天氣炎熱。午間富氏洗了澡上床去睡。丫頭打着扇。那金桂丫頭因接着姑娘洗的殘水。也在那裡洗澡。不想賈文物進來。向房中張了張。見富氏正睡。又到後邊房內窗洞中往裡一張。原來是金桂洗完了澡。坐在一張椅子上蹺着腿。揩那腿上的水。露着一身白肉。下面一道溝兒。火齊內吐。豆蔻含葩。那裡還忍得住。將門一推。却是虛掩着的。他跑將進去。就勢將那丫頭兩腿直扛起來。倒在椅上。那丫頭只顧揩澡。並不防他來。無心被他扛起兩足。跌倒椅上。一個光臀正正對着他臍下。那賈文物也顧不得褪褲子。一個硬邦邦的陽物向他腿縫中混戳。丫頭用手混推混搡。那丫頭本待要叫。一來姑娘吩咐過不必聲張。二來知姑娘睡覺或者不知道。就趁此機會且嘗嘗這肉滋味的意思。就是姑娘知道了。原是吩咐過的。況且賈文物穿的是葛布褲子。雖然隔着弄不進去。却下下戳的是那個地方。被葛布擦得陰門癢癢酥酥。也有幾分動火。所以此時他也不甚十分推辭。那賈文物是急了的。兩隻膀子扛着他兩條腿。要騰出一隻手來扯褲子。怕放鬆了他一條腿。恐他掙了起來。只是隔着褲子混戳。如何弄得進去。那葛布又硬又癩。連門邊兒進不得。弄了一會。還是一個門外漢。正在用力的時候。那知富氏已走到後面。賈文物進來時不曾顧得關門。他心中以爲。就是別的丫頭來看見。都是素常調戲熟了的人。把他看看這個款式。使他也好動情。誰知道那些丫頭未來。反是丫頭的姑娘來了。富氏是有心的人。輕輕走到身後還不知覺。手中拿着條門閂。那金桂早已看見。急得要掙起來。富氏搖了搖手。雙手舉起閂來。連腰帶股儘力打了一下。打得賈文物哼的一聲。一交跌在地下。擡頭一看。原來是母大蟲。顧不得疼。想掙起來跑。那裡掙得起來。被富氏連肩帶脊又是幾下。那賈文物嬌嫩皮膚。何曾嘗過這種惡味。且只穿着一件單衫。痛得滿地打滾。高聲喊叫救命。那金桂却笑嘻嘻背着臉穿衣服。他母親莫氏正在廊簷下納涼。只見含香忙忙的走來。道\footnote{到底是他。}。奶奶。不好了。相公不知甚麼緣故。大〖口么〗喝叫救命呢。莫氏聽得。撂了手中扇子。慌得兩步做一步跑到後邊。只見媳婦拿着一條門閂。兒子在地下哭喊。那地下因洗澡濺了一地的水。被他滾得一件雪白紗衫葛褲就像泥漿的一般。媳婦還在那裡惡狠狠的要打。那莫氏又氣惱又心疼。上前奪住門閂。變下臉來發話道。你也是宦家小姐。那裡有這個道理。就是丈夫有不是。好好的勸。他再不聽。吿訴公婆。有你動手就打的麼。我養他這麼大。還不曾盪他一下。你看打得恁個模樣。你也忍心。少年婦女這樣不賢慧。那富氏從小無娘。被他爹嬌慣了。任性橫行。大氣也不敢呵他。今見婆婆來屬(數)落。如何受得。他就回話道。你養的兒子不長進。還來護短。誰叫他偷丫頭來。不說你兒子沒廉恥。倒來說我。你說我不賢慧。誰叫你家娶我來。嫌不好。休了我去。你旣護短。我偏要打。看把我怎麼的。此時門閂被莫氏奪住。他搶不下來。就丢手撲了賈文物去。莫氏恐怕他難爲了兒子。丢了閂。死命將他抱住。連忙〖口么〗喝兒子道。你還不走麼。那賈文物見勢頭凶惡得很。也顧不得疼了。掙起來就往外跑。正走不動。幸得含香也跟了莫氏來的。看見打得恁個樣子。好不心疼。說不出口。見他跑出來。連忙將他扶住。往前邊去了。莫氏見兒子已去。纔放了媳婦。那富氏見賈文物走去。一口氣不得出。自己一頭撞倒。儻(躺)在地下\footnote{寫出一個活跳潑婦的樣子來。}。大哭大叫道。你家娶我來做媳婦。是娶我來受氣的麼。我爹爹也不曾說我一句。你倒來罵我。撞頭磕腦。虧得丫頭多。將他扶住。不曾着傷。莫氏見這個樣子。再要說他。料道也不肯服順。且恐親家知道。他是溺愛的人。不說女兒不賢。反說婆婆嘴碎。只得忍了口氣回去。走到房中。只見兒子睡在床上哼哈。含香替他身上揉摩。莫氏叫兒子脫了衫子一看。十數處打得烏紫。心裡疼得要死。嘆了一口氣。道。寃家。那丫頭有甚麼到你。你到了這個田地。不由得放聲大哭。含香也忍不住墮淚。賈翰林聽見着。驚忙叫了莫氏過去問他緣故。莫氏隱瞞不住。把打兒子的話說了。那老兒別無他言。只把脚跌了幾跌。咬牙恨了幾聲\footnote{不知者以爲恨媳婦。知者乃恨莫氏也。}。嘆了兩口氣。落了兩點淚。睡倒床上。那富氏賴在地下,被衆丫頭擡到房中\footnote{擡字趣甚。畫也畫不出。}。直哭到掌燈時方住。一口氣塞在胸中。無處發洩。將金桂打了個半死纔罷\footnote{金桂造化低。姑爺的肉棍不曾嘗得。姑娘的木棍反領了無數。}。那夜莫氏叫兒子休要往媳婦處去。留在自己房中養息。那含香好不疼他。一夜也不睡。替他揉搓。時刻不離服事。次日。莫氏坐在床沿上看賈文物。只見含香走到跟前。道。奶奶。我纔到後邊去。見大娘的幾個丫頭在那裡說說笑笑。原來兩次三番都是大娘同他們弄的圈套。因金桂昨日被大娘幾乎打死了。他〔們〕都抱怨說大娘當日定的主意。今日又拿他出氣\footnote{虧這一打。不然他們尚不肯說出。賈文物還在夢井中。}。吿訴了我。大相公還呆着當他們同他有情。睜着眼往火坑裡跳。吃了這兩場囗。賈文物如夢方覺。醒悟道。我同丫頭調笑。他便不知。剛要動囗。囗知道了。原來有這些機關。悔恨無及。那莫氏聽了。嘆道。小小年紀。這樣狠心。夫妻間一點情義都沒有。只恐我老夫妻死囗囗還不知怎樣受他的罪呢。落了幾點眼淚。因對含香道。我看你倒還疼他。我的眼睛看不到。你留心打聽他們有甚麼機謀見識。你敎他防備防備。含香道。不用奶奶吩咐。我自然留心。莫氏聽得甚喜。賈文物也心中感激。又過了幾日。賈文物身子漸漸好了。起得來。莫氏想媳婦兒子兩處分着不是常法。把惡氣放下。掏出好氣來。將兒子拉着到媳婦房中來。道。我前日一時心疼兒子。勸了你幾句。你就惱了。我今日送了他來。你夫妻和和美美的。前話總不須提起\footnote{媳婦潑惡至此。莫氏兩次反向彼說好話者。一則雖是爲兒子。二則到底是看富字面上也。}。那富氏前日把丈夫打得太毒。自己後來也覺過意不去。撒了一場潑。公婆也沒有甚話。心裡也有些不好意思。且這幾日獨臥。甚是冷淸。有他在床上。雖不能大暢所懷。也還拱拱聳聳。在肚皮子上熱熱鬧鬧的\footnote{後富氏也說好話者。因此二句耳。前所云過意不去。不好意思。都未必眞。}。今見婆婆來說好話。他道。我一時失錯。奶奶不要怪我。那莫氏見媳婦也說好話。纔放心去了。正是。

\begin{quotation}

婉轉和兒媳。慇懃做老娘。

\end{quotation}

賈文物此後見他就怕。只是到床上那一會工夫還可以見他個好臉。閒常就如小鬼見了閻王一般。隔了些時。富氏偶然回去看父親。留他住了十數日。那賈文物是閒不住的人。獨自一個又想胡做起來。富氏的丫頭是不敢惹他了。這個含香旣是舊交。又甚有恩情。思想溫溫舊帳。那日趁着母親在父親房中看着熬藥。這丫頭因夜間服侍老主病症。不曾得睡。此時偷空在他床上睡覺。賈文物悄悄進來。左張右望不見丫頭。走到娘房內又不見。到床後一望。見他睡着。滿心歡喜。忙上前親了個嘴。推醒了他。要同他高興高興。那丫頭也久別此道。正在企慕之時。欣然笑納。二人如久渴得漿。那裡就肯便住。莫氏一時要丫頭拿東西。叫了兩聲。不見答應。也疑他偷睡。走了來床後一看。見兒子正同他弄呢。莫氏知兒子同他有舊帳。又見這丫頭甚有情到兒子。也不動怒。只嘆了一聲。罵道。孽障。你還不怕。又做甚麼呢。他二人正弄得高興。融融笑語。曲盡于飛之樂。並不知道娘來。聽見說話。那賈文物連忙穿衣往外去了。丫頭也繫了褲子出來服侍。這賈文物覺得同含香幹事甚有情趣。不像同富氏。下邊雖然也一般幹着。上面心裡到底膽怯。況這丫頭比富氏模樣又標致些。且娘又不十分嚴緊。兩人偷工摸夫。得便就做一齣。若要人不知。除非自莫爲。不想被富家陪嫁的家人媳婦們知道了。要在姑娘跟前討好。等得富氏來家。一五一十。全全奉吿。富氏惱在心頭。因不曾拿着贓犯。聲揚不起。又恨婆婆縱容兒子。每日留心看他破綻。又吩咐家人丫頭細心打聽。一日。也是合當有事。莫氏叫含香到他房中來叫賈文物。這富氏是眼中放不下砂子的人。一見了他。眼中火冒。醋氣直噴。罵道。你這小騷奴。到這裡來尋漢子麼。含香道。奶奶叫我來叫相公。無緣無故爲甚麼罵我。富氏道。你來尋他肏搗罷了。說奶奶來叫他。我不在家。你們肏搗夠了。我來了。你還浪着尋了來。沒廉恥的臭娼根。養漢精的淫婦。你熬不得了。脫了褲子到街上尋人肏搗去不是。你到我屋裡來怎麼。那丫頭也回言道。我是奶奶的丫頭。不到你罵。我同相公怎麼樣你見來麼。小小年紀。肏搗不離口。倒說我沒廉恥。那富氏可是容得下人頂嘴的。幾句說急了。跳起身撲了他來。一把抓着頭髮。罵道。你偷漢子可不是沒廉恥。還敢強嘴。就夾臉打了個嘴巴。那含香那裡依得。雖不敢還手。把他兩隻手揝得死緊。說道。你是官宦人家的小姐。連一點禮性也不知道。婆婆的丫頭到你打。你說我偷漢子。奶奶不管我要你管。富氏罵道。你那奶奶也算得人麼。白披着張人皮。連畜生還不如呢。他要是有人氣兒的。肯容兒子偷丫頭。許丫頭偷漢子麼。兩下爭持着。衆丫頭旣不敢勸姑娘。又不敢幫打含香。正急得沒法。原來富氏先同丫頭拌嘴時。賈文物已進來聽見了。忙報知莫氏。道。媳婦同含香鬧呢。莫氏急忙走來。到了門外。聽得媳婦罵丫頭偷漢子。知道是爲兒子起見。反不好意思進去。聽到後來連他也傷犯起來。如何忍得住。進門嚷道。好媳婦。好媳婦。連婆婆都罵起來了。我的丫頭是你打的麼。還不放手。上前撥他的手。富氏也不叫奶奶了。嚷道。你爲丫頭難道打我麼。丫頭偷你兒子。你還來護他。你旣然有這樣好媳婦。當初又娶我做甚麼。莫氏見他不遜。也怒極了。便道。我早知你這樣不賢良的東西。我兒子就一世沒老婆。我瞎了眼也不娶你這樣媳婦。見他還抓住含香的頭髮不放。將他手背上下力一擰。那富氏從來線疙瘩挨着都叫疼。何曾經過這辣味。只得放手。那丫頭如飛跑去了。他嚎咷大哭道。原來你娘兒們捎成幫兒來算計我。我還不如一個丫頭。要這命做甚麼。正哭着。一眼看見賈文物在門外。便惡狠狠的撲了去。莫氏正然氣得發昏。忽見他去撲兒子。生怕被他拿住吃他的虧。忙奔了出來。拉着兒子往上飛跑。到房中坐下。看那賈文物臉都嚇白了。丫頭在那裡梳着頭。淌眼淚縮鼻子。紅着半邊臉。幾條指印。一抽一吸的哭。莫氏見了這個樣子。因想媳婦如此不賢。兒子將來不知怎麼樣結局。又是自己做的事。怨不得人。不由得傷心哭將起來。聲雖不高。那一種怨恨之氣未免露出。那老兒聽得聲息異常。叫丫頭請了莫氏過去問他。莫氏正一腔忿恨。把媳婦不知事的話盡情吿訴。老兒只是嘆氣。且說那富氏哭了一會。晚飯也不吃。睡在床上。到了夜間。又哭了一場。拿了根帶子。在床欄杆上上吊。\endnotemark[3]幸得丫頭們聽得他哭。都還未睡。忽然不見聲息。走來看看。要是睡着了。他們好睡。猛然看見打鞦韆呢。嚇得大叫道。姑娘不好了。在這裡上吊呢。你們快來。四個丫頭慌的一齊推進門來。忙忙解下。一面救着。一面着一個上去說信。那富氏因方纔上去不多的工夫。不曾着傷。撅了一會。一口痰湧出。又重新哭將起來。那丫頭飛跑去與莫氏報信。莫氏方纔睡下。聽得打門。說媳婦上吊。這一驚不小。望着兒子道。這是你前世的寃家。不知弄的怎樣個下場頭呢。一面說着。一面忙穿了衣服。叫一個大丫頭拿着燈\footnote{此等沒要緊處。亦必留心寫出。云大丫頭者。不好帶含香來也。}。開了院子門。一直前來。看見媳婦已救醒了。睡在床上哭。心中一塊石頭方纔落地。只得好言撫慰道。癡孩子。小小年紀。怎尋這短見。我婆婆勸你是好話。肯爲丫頭說你不成。好好的快不要胡思亂想。富氏總不理他。只是哭。莫氏見他如此。又羞又惱。坐不住起身。又勉強安撫了幾句上去。此時老兒也知道了。起來靠着枕頭坐着。只是長吁短嘆。莫氏回來。到他房中坐下。老兒道。媳婦這樣潑悍。不是小可的事。明日請了親家來。等我說明了。後來就有一差二錯。我有話在前。也好分說。莫氏連稱有理。看着老兒睡下。也自去睡了。到了次日。果然請了富戶部來。那老兒一肚子鬱氣脹得久了。從始至末。將他女兒怎樣打女婿。同丫頭通同害丈夫。又怎樣罵婆婆。昨日又怎樣打婆婆的丫頭。並夜間上吊的話。盡情吿訴了一遍。又道。我一生只有這一點骨血。我將九十幾的人了。將來小兒不知做何光景。不覺揮下淚來。那富戶部惶愧至極。心中想女兒如此凌虐丈夫。不孝公婆。十分過意不去。又見親家年老。說得如此傷心。更覺惻然。只得說道。親家。你年尊了。不必着惱。小女自幼無母敎訓。不知人事。凡事不要理他。你但放心。我又無兒。女婿我自然竭力照看。成就他的功名。老兒見親家說得甚好。深謝了。那富戶部辭了出來。到女兒房中。見他也不梳頭洗臉。睡在床上哭泣。便說道。我兒。你如今在人家做媳婦。比不得在家做女兒。賢名難得。公婆可是得罪得的。就是女婿年小。有不知事。只勸得他。一個丈夫可是打得的。富氏滿胸惡氣。聽得他父親來了。只道是來替他出氣。誰知反說起他來。遂大嚷道。我不賢。當初誰叫你養我來。我今日在他家。不要你來做喬家長管閒事。不怕他家有鍋煮吃了我。就是我死了。也不稀罕你來替我要命。那富戶部見女兒如此無知。出嫁的女兒又不好罵他。又恨了一聲道。玷辱家門的孽帳(障)。遂忿忿的出來。賈文物不敢進房。在廳上候着丈人。那富戶部見了。一把拉着他的手。道。小女無知。賢婿不必記懷。諸凡看我面罷。有我丈人在。你只管放心。賈文物作揖謝了丈人。那富戶部上轎回去。這一場鬧。富氏氣不得出。成日打丫頭罵僕婦。摔碟摜碗的使性子。足足有一個多月。方纔氣消。那賈老兒見親家說了許多好話。又見兒子媳婦兩下隔了月餘。不是常法。只得吩咐治了一席酒。叫了媳婦到跟前。說了些好話。勸了幾句。叫莫氏領了兒子媳婦回房飮酒和事。事雖和了。這賈文物的膽也碎了。從今後在家中不敢起一毫妄念。這些時在母親房中睡。因他娶過媳婦。不便帶他一床睡。床橫頭安了個鋪給他。與含香相離咫尺。無夜不兩人在一處。莫氏惱恨媳婦。明知道也不禁他。他兩個百般恩愛。雖心中難割難捨。因富氏法度利害。也只得割恩斷愛。循規蹈知。不敢再尋舊好。只好得空到外面去混混罷了。富氏見他守了法度。倒也相安無事。那富戶部自從許了親家成就女婿。每日以此事爲念。一年値文宗科考。這宗師當日與他做過同寅。甚是契厚。再三請托。要替女婿進學。那文宗也自依情。府縣考的名字容易。不消說得。到了道考。也進了學。熱鬧了一番。上秋鄕試。這主考又是富戶部同年同門。一出京就備了一分厚禮。半情半賄。求一關節要中女婿。那主考自然肯做分上\footnote{這一句非爲同門同年四字。乃爲厚禮二字也。}。他進了三場。那文章不知從何而來。放榜之日。又輕輕巧巧中了一名舉人。再說這江南三學中有一種學覇。自己從不讀書。遇歲考時用銀子老保一個三等。他一年的買賣。惟以把持衙門爲事。議論風生。是非蠭起。專一羅織管事騙錢入(而)已。今見賈文物中了。知他是新進小子。一竅不通。又知他丈人豪富。遂買謠言說富戶部替女婿買的舉人。希圖馬扁。執(孰)不知他翁婿二人學了兩句古語。叫做。

\begin{quotation}

任他風浪起。穩坐釣魚船。

\end{quotation}

且在家中擺酒唱戲。賀喜熱鬧。竟不理他。這幾個學覇老羞變怒。遂一唱百和起來。說某人是某宦兒子。某人是財主賢郞。都是買的舉人。爲頭的雖不多幾個。有好生事的秀才就跟上數百。同去文廟中哭廟。又蜂擁着打到主考公館門首。那主考知道了。不勝大怒。傳地方官擒拿。江南人稱爲呆鵝頭。那鵝見人走着。他却伸着大長脖子來嚇人。被人一脚踢去。他反嚇得跑得老遠。江南人就是這個樣子。無事之時。一人首唱。就有許多人幫襯。及至弄出事來。一哄跑個乾淨。起先有幾百秀才。戴着方巾。穿雙紅鞋。手中拿把扇子。口中之乎者也的亂嘈胡鬧。後來聽得拿人。這些人誰知都是屬屁的。一喞就不見了。跑得一個皆無。只剩得爲頭的七八個。主考將這幾個人交與地方官。他連夜上本。別話一槪不題。只說惡衿不中。欺凌主考。這主考是魏璫門下。遣人預先賄通。不消說得。這富戶部見風聲不好。恐連累了自己。叫女婿收拾了往京中去。一者躱是非。二者尋門路。備了有三千金的一分禮物。叫他到京送與阮大鋮。這阮大鋮是同鄕同里的人。又素常相識。因他是魏忠賢第一個用事的門下。在京做官。轟揚天下。故去托他。又備了萬餘金厚禮。托阮大鋮轉送魏忠賢。要領賈文物拜他門下做個孫子。以爲靠山。還求擡舉。賈文物到京。見了阮大鋮。送上書信。交了禮物。阮大鋮好生歡喜。次日即同去見了魏忠賢。送上厚禮。都是黃烘烘杯盤壺碗。金晃晃錦緞紗羅。卷軸盡唐詩宋畫。骨董悉周鼎商彝。玉帶犀杯。珍珠寶石。魏忠賢收了。賈文物又拜了門下做孫兒。魏忠賢先見了禮物。毫不介意。見賈文物認了孫子。倒覺歡喜。阮大鋮將賈文物中了舉。衆人見他家殷實。想耍詐騙。要求上公照看。又把江南秀才哭廟的話。大槪說了數句。魏忠賢怒道。前日我見本來。深恨這些秀才可惡。已批了旨。皆着責革問罪了。這賈孫兒中一個舉多大事。明年咱偏中他個進士。看人怎樣的。阮大鋮道。這是上公天恩。他翁婿自圖厚報。忙叫賈文物叩謝。魏忠賢笑道。你有咱這樣個爺。連孫兒的進士也不能中一個。把咱的體面都沒了。向阮大鋮道。阮官兒。你同他去罷。叫他等着。二人拜辭出來。果然次年春榜。賈文物又搭了一名進士。正是。

\begin{quotation}

胸中何用書千卷。只要生來福運齊\footnote{二語慟殺老儒。}。

\end{quotation}

你道這魏忠賢一個沒卵袋的太監。怎麼就大到這樣地位。是個甚麼來歷出身。聽我細細講來。便知詳細。他祖籍直隷河間府肅寧縣人。他父親是屬兔的。自幼小名叫做魏卯兒。人都順口叫熟了。倒不知他的正經名字是甚麼。這魏卯兒生得着實標致。在縣中當了一名門役。雖伺候過一個知縣。却不曾作興到他。這六房書辦。無一個不同他契厚。穿的吃的用的倒都不愁。後來一個新任知縣。係福建人。酷好男風。又因路遠不曾帶家眷赴任。就寵幸起他來。竟如伉儷一般。言聽計從。那六房書吏都是他親密極了的人。表裡通連。替他在外邊招攬過付(府)。數年間他也弄有二三千金之物。知縣因此聲名大壞。被上司揭參了。革職回去。那時魏卯兒也有二十多歲了。不但腰中厚實。而且唇上漸漸長出那不情的鬍子來。況且縣官之壞因他而起。恐再來的官府若是知道。倘一責革。更覺無顏。就退了役回家。想要娶房妻小。浼托媒人替他尋一個標致女子。那媒婆道。眼面前這些人家女兒我都見過。人物都只中中。沒有甚麼上樣的。只有臭水溝住的賣扁食的邊家女兒\footnote{臭水溝賣扁食的邊家多甚。}。他雖是個小戶人家。那女子眞有十分姿色。但聽得人說未必是個眞女兒了。你若不計較。這倒是現成的。一說就穩。你要嫌他。只好別處慢慢打聽。魏卯兒聽得這女子有十分姿色。動了火。想道。管他是整是破。若錯過了。焉知將來可還遇得着這樣人物。因對媒人道。我不論這些甚麼眞女兒假女兒的。他就是眞正黃花女兒。到我跟前。第二日依舊是個破了的。這有何妨。只要模樣兒好就罷了。媒婆道。旣如此說。我包管你必成。只要謝禮從厚。說完。去了。你道這媒婆如何拿得這樣穩。原來這女子瞞着父母。相與了個趣人兒。是在他家每常走動的一個花(化)緣和尚。邊老兒常常捨塊豆腐給他。不住來往。同這女兒就暗暗的偷上了\footnote{邊老兒可謂薄往而厚來。他只常捨和尚一塊豆腐。和尚便回達(答)他女兒一條肉棍。}。有一年光景。那女兒已有了半肚身孕。想要同逃。不得其便。他父母知道了。要急急遣嫁。料瞞不得。倒將不是原封的話吿訴媒人。預先說明。願者成交。所以媒人知道必肯。走來一說。果然兩口子不但肯而已矣。聽得是致仕的門官。且又家中殷實。眞算攀高結貴了。欣喜非常。媒人復了信。魏卯兒行茶下禮。不須煩說。到娶的這一日。他舊日相厚的這些書辦並衙門中人。都送分資來賀喜留酒。他因見新人果然美貌。心中十分歡喜。衆人敬他喜酒。他鍾鍾不辭。到人散時。他的酒也有了十分。進到房中。那新人早已睡下。他忙脫衣上床。鑽入被中。摸那新人時。也脫得一絲不剩\footnote{這新人來得脫套。}。他大醉了的人。忘其所以。將屁股往新人胯下亂拱。那邊氏忍不住笑問道。你這是做甚麼。他道。我同你成親。邊氏道。你成親如何是這樣的。你錯了。他模模糊糊的道。我從小就是這樣。成過幾千次了。如何得錯。邊氏笑道。我也曾成過。是對面來。却不是這樣的。魏卯兒被他提醒。方想起是娶老婆。不是伴孤老。纔轉過臉來。爬上肚皮。做了一齣武戲\footnote{虧娶了這個老作家將他提醒。若娶了個眞女兒。豈不須(虛)度此良宵。}。過了兩日。他偶然見邊氏一個大肚子。腆着問道。你腹中有疾患麼。爲何肚子這樣大。邊氏笑而不答。剛剛到了一百日。就生了一個肥頭大臉滿抱的兒子\footnote{前屠家之通氏七八十個月始生兒。今邊氏只一百日便產兒。何多者太多。而少者太少。}。魏卯兒知這娃娃來得有些古怪。意欲抛棄。邊氏執意不肯。道。你要棄了這孩子。我也就尋個死路。我嫁了你三個多月。就是你的骨血了。爲何要撂他。魏卯兒疼這邊氏過甚。不得不依他留下。這娃娃就是魏忠賢了。起初他也上過學念過書。他原舊日的名字叫做魏進忠。忠賢是後來御賜的名字。魏忠賢到了十七歲上。他老子娶了個媳婦與他。一年後生了一個兒子。起名魏良卿\footnote{這却是個眞種。}。他娶的這媳婦姓薊。也生得有幾分顏色。魏忠賢却不十分相愛。反愛在外宿娼。再說這魏卯兒十多歲時因後庭主顧太多。得了楊梅瘡。他正在當時的時候。怎容他發了出來。一陣輕粉頂藥頂了回去。如今四十開外的人了。又被邊氏淘虛。舊瘡發將起來。成了翻花楊梅。醫治不效。先將鼻子陽物蝕去。後漸漸遍身腐爛而亡。這邊氏每常在被窩中。一夜也不許丈夫躱懶。自魏卯兒害瘡起。有半年多不曾高興。急得要死。要尋個朋友。又有兒子媳婦礙眼。虧得向年相與的那和尚。假說是表兄。來家走動。暗暗同他解饞。今見丈夫死了。忙忙下葬。剛過了三七。捲了些私房。同那和尚相約逃走。一日黑早。不知往那裡去了。這魏忠賢自幼好賭好嫖。因家私是父母管着。不得到手。他只好零碎偷些東西出去當賣了頑耍。再不能像意。見老子死了。心中暗喜可以自由。猶恐娘不肯。到底掣肘。今見他做了柳翠。被月明和尚度了去。歡喜無限。那裡還肯去尋問。遂將他老子少年糞門中掙來的這分家私任他揮霍。不但大嫖。而且大賭。不數年而罄。房子也賣了。租了兩間破屋棲身。不但人見他沒錢不同他賭。連日食都斷絕起來。因叫他妻子薊氏做個私窠接客\footnote{他嫖過了。又該他妻子來嫖。}。賺錢度日。誰知這薊氏因丈夫同他不甚相愛。在外貪嫖貪賭。徹夜不歸。他這數年來。不等丈夫吩咐。早已相與過許多朋友。自做久了。魏良卿承爺爺的舊業。也被人誘去做了小官\footnote{此時做小官。誰知後來竟封了侯。}。十日半月常不歸家。魏忠賢明知放縱。並不查問他來去。這薊氏自從做了這樁買賣。倒也大(在)行。魏忠賢除去家中柴米之費。餘者仍拿去做賭本。但零零星星。不得個爽快。過了一二年。偶遇着一個山東的水客要買婊子。魏忠賢帶他相了薊氏。講明身價五十兩。竟賣與他去了。他欣欣然把銀子揣到賭場同人大擲。人只知他賣老婆。不知是多少身價。都想大贏他。一日一夜。贏了他一百餘兩。到開發時。只得五十金。兩下就爭鬧起來。三個人打他一個。魏忠賢此時也急了。又一無所戀。思以性命圖賴。恰好傍邊有把刀子。他搶過來。衆人當他行凶戳人。倒都躱開。不想他將褲子扯下。揝着㞠子。狠命一刀割去\footnote{他想賴。也是老婆賣去了。此物留之何用。}。血暈倒了。一交跌在地下。血如湧泉。衆人想要跑。那開賭的窩家道。列位去不得。這場人命官司要打大家打。推不在我一個人身上的。且救了看。救活了是大家的造化。救不活再商議。衆人知道脫不得干係。只得上前來救。一面燒綿花灰替他摀住刀口。一面用薑湯灌餵。多時甦醒過來。衆人商量了一番。向他道。這是你自己割的。並非我們害你。你就死了。我們也到不得償命。如今這五十兩銀子還你。我們還大家湊錢養活你。你好了呢。不消說。設或你要不好。身後我們發送埋葬你。這五十兩銀子給你兒子安家。你的意思怎麼樣。你必欲要到官。這銀子我們且留着打官司。魏忠賢自想。自己一貧如洗。此時旣疼得動不得。又無親人\footnote{因有此想。所以後來舉朝臣宰皆要了做乾兒孫也。}。兒子又小。先是拚着一死。不想又活了轉來。且落得得這五十兩。後來還好做賭資\footnote{却不想嫖了。}。也就肯了。衆人見他依允。圖脫禍患。上好飮食供給他。又去尋了他兒子來。把前話向他說了。這賭錢的內中有一個是他的大花子。拿好話兒給他吃。他也喜諾。就留他日裡服事他老子。夜間那人帶他去幹此竅。這河間府閹割的娃娃甚多。有絕妙的藥方。請人來替他醫。就將他㞠子煅灰配藥。給他吃了一個多月。瘡口痊癒。一嘴鬍子也掉了。宛然一個內監。這種人地方上私容不得。就去報了官。官府問起緣故。他稟稱情願自宮。希圖進用。地方官具文差役送到京中司禮監衙門交割。他就帶了兒子魏良卿一同前去。到了京中。那時正是王安掌司禮監事。差役投到。王安撥他到東京皇長孫處給使。這皇長孫就是天啓皇帝。那時天啓正在童年。左右伺候的全是些小內監。又都在宮中長大。還不知道甚麼。這魏忠賢在外邊過了三十多年。何事不知。他身邊還有那五十兩銀。將外面所賣頑戲的物件無不買來哄誘天啓。天啓這疼愛他眞如至寶。一刻也離他不得。天啓的乳母客氏。係定興縣民侯二的妻子。生得模樣甚好。選入時只得二十來歲。他乳大了天啓到了十三歲。這客氏也是個騷淫婦人。沒奈何被選了進宮。十多年無非同些宮女內監爲伍。今見天啓大了。一來圖解讒。二來圖固寵。竟被他引誘。破了天啓的童身。他當日拿小奶頭的奶給天啓上嘴吃。到今日他又拿〈天啓〉大奶頭的奶〔給〕他下嘴吃。天啓自嘗了這種佳品。覺御廚中的供膳無一可及。竟同他同起同臥。如夫婦一般。魏忠賢知道了。以客氏將來可做內中一個大援。遂同他結拜了做兄妹。魏忠賢自割勢進宮之後。隔了一二年。得了個異人傳授。常服丸藥。生啗小兒腦髓。他那陽物竟長出有四五寸長一段來。却是沒頭沒腦的一件東西。客氏心喜。魏忠賢狡黠。兩人暗暗私通。成其夫婦。天啓尚幼。客氏不過要哄他親厚。那根御屌原不足大用。得遇忠賢。眞是意外奇逢。忠賢又引魏朝與之私。客氏愈喜。二人千般海誓。萬種山盟。但他這種盟誓與別的男婦不同。那講的是情。他講的是利。無非是皇孫登極之後。如何內援。如何外應而已。到了萬曆四十八年。神宗崩遐。泰昌登基。一月。龍馭上昇。天啓即位。忠賢得賜今名。命爲司禮監太監。總督東廠官旗辦事。客氏當日在宮中人皆稱爲客巴巴。到今封爲奉聖夫人。出外乘八人大輿。內官錦衣花帽執棒前驅。聲位與皇后等。天啓又特給客氏與忠賢爲妻。到二年九月。賜魏忠賢客氏各金印一顆。方二寸。四爪龍鈕玉筯篆文。每印九字。作三行。一曰。欽賜顧命元臣忠賢印。一曰。欽賜奉聖夫人客氏印。每顆二百兩。御用監製造中書篆文。內官監置金龍印盒。一時伺旨獻諛。靡費數萬金。他二人一內一外。漸執朝政。招權納賄。熒惑聖聽。那個罪惡也不能盡述。直是罄南山之竹。書罪無窮。決東海之波。流惡難盡。那時有一位忠義大臣參了他一本。但看此本。便知魏忠賢同客氏之惡了。也不用我細說。那本上道。

\begin{quotation}

左副都御史臣楊漣\endnotemark[4]題。爲逆璫怙勢作威。專權亂政。欺君藐法。無日無天。大負聖恩。大干祖制。乞大奮乾斷。立賜究問。以早救宗社事。如東廠太監魏忠賢其人者。舉朝盡爲威制。無敢指名糾參。臣實痛之。今若畏禍不言。是臣負忠義初心。以負皇上起臣特恩。他日有何面目以見先帝〈於〉在天之靈。謹撮其罪之大者二十四款。爲皇上陳之。

忠賢原一市井無賴人耳。中年淨身。夤入內地。拔之幽賤。寵以恩禮。原名進忠。改命今名。豈非欲顧名思義。忠不敢爲奸。賢不敢爲惡哉。乃初謬\endnotemark[5]爲小忠小信\endnotemark[6]以倖恩。旣而敢爲大奸大惡以亂政。自忠賢擅權。旨意皆出傳奉。假若夜半出片紙殺人。皇上不得知。閣臣不及問。害豈渺小。壞祖宗二百年來之政體。大罪一也。

舊閣臣劉一燝。冢臣周嘉謨。同受顧命之大臣也。忠賢交通孫杰論去。急於剪己之忌。不容皇上不改父之臣。大罪二也。先帝壯年登極。一月賓天。執春秋討賊之義者。禮臣孫愼行也。明萬古綱常之重者。憲臣鄒元標也。忠賢一則逼之吿病去。一則嗾言官論劾去。何親於亂賊。仇於忠義。大罪三也。

王紀爲司寇。執法如山。鍾羽正\endnotemark[7]爲司徒。淸修如鶴。忠賢一則辱而迫之去。一則陷之削籍去。不容盛世有正色立朝之直臣。大罪四也。

國家最重無如枚卜。忠賢一手握定。是眞欲門生宰相乎。大罪五也。

爵人於朝。莫重廷推。忠賢用羽翼之奸。致一時名賢不安位去。顚倒有常之銓政。掉弄不測之機權。大罪六也。

聖政初新。正資忠直。乃滿朝薦等九人。抗論稍忤。忠賢傳春(旨)盡令降斥。屢經恩典。竟阻賜環。長安謂皇上之怒易解。忠賢之怒難饒。大罪七也。

然猶曰外廷之臣子也。傳聞宮中有一舊貴人。以德性貞靜。荷上寵注。忠賢恐其露己驕橫權謀之私。託言急病。立刻掩殺。是皇上且不能保其貴幸矣。大罪八也。

猶曰無名封也。裕妃以有喜得封。中外欣欣相吿矣。忠賢以抗不附己。矯旨特令自盡。不令一見皇上之面。皇上又不能保其妃嬪矣。大罪九也。

猶曰在妃嬪也。中宮有慶。已經成男。乃繞電流虹之祥。忽化爲飛星墜月之慘。傳聞忠賢與奉聖夫人實有謀焉。是皇上不能自保第一子矣。大罪十也。

先帝在靑宮四十年。護持孤危者。僅王安一人耳。而忠賢以私忿。矯旨掩殺於南海子。身首異處。肉飽狗彘。是不但仇王安。而實敢仇先帝之老奴與皇上之老犬馬。略無顧忌也。其餘內臣擅殺擅逐者。不知數百千也。大罪十一也。

今日討獎賞。明日討祠額。又於河間府毀人房屋。以建牌坊。鏤鳳雕龍。干雲揷漢。又不止於塋地擅用朝囗(廷)規制。僭擬陵寢而已。大罪十二也。

今日廕中書。明日廕錦衣。金吾之堂口皆乳臭。誥勅之館目不識丁。如魏良卿等。及外甥野子傅應星等。五侯七貴。何以加茲。不知忠賢有何軍功。有何相業。甚褻朝廷之名器矣。大罪十三也。

用立枷之法以示威。枷死皇親數命矣。其扳陷皇親者。欲動搖三宮也。若非閣臣力有護持。言官急爲糾正。椒房之戚。又興大獄矣。大罪十四也。

良鄕生員章士魁。以爭煤窰傷其墳脈。托言開礦而死矣。假令盜長陵一抔土。何以處之。趙高鹿可爲馬。忠賢煤可爲礦。大罪十五也。

王\endnotemark[8]思敬胡遵道侵占牧地果眞。小則付之有司。大則付之撫按學院足矣。而徑拿黑獄。三次拷掠。身無完膚。以皇上右文重道。而忠賢草菅士命。大罪十六也。

未也。科臣周士樸執糾織監一事。原是在工言工。忠賢竟停其陞遷。致士樸卒困頓以去。以中官之尊大得矣。而朝廷何可有此名也。大罪十七也。

未也。且將開羅織之毒於縉紳矣。北鎭撫臣劉僑不肯殺人媚人。自是在刑言刑也。忠賢竟逐之去。於是張忠賢之威焰得矣。而國脈何可崇此蘊毒。大罪十八也。

未也。且示移天障日之手於絲綸矣。科臣魏大忠到任。已奉明旨。忽傳旨詰責。\endnotemark[9]及科臣回話。無論玩弄言官於股掌。又煌煌天語。提起放倒。令天下後世視皇上何如主。大罪十九也。

最可異者。東廠原以查奸細。非擾平民也。自忠賢受事。雞犬不寧。快恩仇。行傾陷。片語違忤。則駕帖立下。如近日之拿中書汪文言。不從閣票。不令閣知。不理閣救。當年西廠汪直之僭。恐未足語此。大罪二十也。

尤可駭者。東廠緝訪何事。前韓宗功潛入長安打點。實往來忠賢司房之家。事露始令避去。假令天不悔禍。宗功奸細事成。不知九門內外生靈安頓何地。大罪二十一也。

祖制不蓄內兵。原有深意。忠賢謀同奸細。創立內操。復傾財厚與之交納。不知意欲何爲。大罪二十二也。

近日忠賢進香涿州。鐵騎之簇擁如雲。蟒玉之趨隨耀日。警蹕傳呼。淸塵墊道。人人以爲駕幸涿州。及其歸也。以輿夫爲遲。改\endnotemark[10]駕駟馬。羽幢靑蓋。夾護環遮。則已儼然乘輿矣。想亦恨在一人下耳。大罪二十三也。

忠賢走馬御前。皇上曾射殺其馬。忠賢不自畏罪請死。且聞進有傲色。退有怨言。從來亂臣賊子。只爭一念。放肆遂致收拾不住。奈何尚養虎兕於肘腋間乎。大罪二十四也。

凡此逆跡。皆得之邸報招案。與長安共傳共見。非出於風影意度者。忠賢之二十四大罪。內有受而外發之。外有呼而內應之。又有奉聖客氏爲之彌縫其罪戾。故掖廷之內知有忠賢。不知有皇上。都城之內知有忠賢。不知有皇上。即有大小臣工。又積重之所移。積勢之所趨。亦不覺不知有皇上。而只知忠賢。宮中府中。大事小事。無一不是忠賢專擅。即奉奏之旨。反覺皇上爲名。忠賢爲實。且如忠賢前往涿州矣。一切事情必星夜馳請意旨票擬。嗟嗟。天顏咫尺之間。忽慢\endnotemark[11]不請裁決。而馳候忠賢意旨於百里之外。事勢至此。皇上之威靈尚尊於忠賢\endnotemark[12]否耶。每思至此。尚爲有天日耶。無天日耶。忠賢狼子野心養成。今日客氏又從傍巧爲營解。忠賢欺君無上。罪著惡盈。豈容當斷不斷。伏乞皇上大奮雷霆正法。以快神人公忿。其奉聖客氏亦幷勅令出外。無復令其厚毒於宮中。其傅應星等着法司責問。然後布吿天下。暴其罪狀。如此。

天意勿回。人心勿悅。內治外安。不新開太平氣象者。請斬臣以謝忠賢。知此言一出。忠賢之黨斷不能容臣。然臣不懼也。但得去一忠賢。以不誤皇上堯舜之名。即可以報命先帝。可以見二祖十宗之靈。一生忠義之心事。兩朝特達之恩知。予願已畢。死且不恨。惟鑒臣一點血誠。即賜施行。

\end{quotation}

他這一本上去。在廷忠義之臣皆以爲天啓必定震怒。將忠賢滅族。客氏貶開。盡洗耳以聽。不想魏忠賢積威所致。天啓久矣拱手服降。且天性愚騃。見了這本。不但不怒。反恐忠賢遷怒到他。滿臉陪着笑說道。這本上說的話。那外邊的事。說我不知道還罷了。這些宮中的事。我尚不知道。他外廷臣子何由得知。我有些信不過。忠賢道。上位說得是。只這麼一想。就知是假話了。他們見上位托我掌管朝政。他外邊官兒不得弄權。想要觸了上位的怒將我貶開。好讓他們大家弄鬼。客氏扭頭捏頸的道。這些嚼舌根的。連我也拉在裡頭。他們不過怕我在爺跟前說他們的不是。都想擠撮我。我出去就是了。就往外走。天啓忙親自跑去拉住。說道。你不要着惱。我自有處治。因怒向魏忠賢道。你把這樣多事的人重重的處了。別的才不敢學樣兒。忠賢道。上位不知道。他們這一黨的人多着呢。就處一兩個。他們也不怕。天啓道。不拘他多少。你都盡情重處就是了\footnote{忠賢之肆毒。若非天啓主意。安敢大膽乃爾。後人但歸罪忠賢而不責天啓。是捨本而求末矣。即如秦檜之殺岳飛。若無高宗之意。彼亦焉敢下手。凡看書者。當於言外會意方妙。}。忠賢客氏聽了這話。心中暗喜。出來就批嚴旨切責。忠賢知道皇帝是他治服的了。何得尚容臣子嘵舌。遂弄了個東林黨。大戮忠良。把些正人君子盡行殺逐。所留合朝文武皆是他的乾兒。自首相魏廣微起。五府六部。大小九卿。以至科道。無非兒而已矣。那時有個禮部尚書將八十歲了。向忠賢說道。我本意要與上公做個兒子。因年紀太大了。不好認得。叫我兒子與上公做個孫子罷\footnote{此尚書知禮。不愧爲禮部。}。你看那時士大夫無恥至此。可還成個世界。此時魏忠賢已建府第在外。客氏亦起大業。各家有數千奴僕。每出朝到家。千歲之聲震耳。那時有奉承忠賢者。尊呼爲九千九百九十九歲。他二人互相表裡。忠賢出。則客氏在內。客氏出。則忠賢在內。把一個天啓竟被他二人監官(管)得定定的。一毫不能自主。忠賢因與客氏輪流出入。不能常伴他同宿。挑選了四個貌美陽大的小廝送客氏爲小夫。笑說道。我不得常常奉陪。送這幾個小廝與你服事。料道你家侯爺也不敢管你。你可留下用罷。客氏也就笑納。客氏住在大宅。在隔壁又蓋了一處房子。與丈夫侯二同兒子侯興國住。他但是出宮。便叫這四個小廝一床同睡。大暢所懷。所以越發感激忠賢。更加親厚。表裡爲奸。忠賢一手握定生死權柄。在廷衆臣工非乾兒即廝養。吩咐一語。雷令風行。他要放個宰相。還易如反掌。何況要中個進士。那賈文物也不知有文章沒文章。不過說了名字與主考。竟中而已矣。再說那賈文物中後。捷報到家。那賈老兒聽得兒子中了進士。年老病久的人喜極。一笑而逝。莫氏忙差人往京去報喪。賈文物辭了魏忠賢阮大鋮。星夜奔回。他家弔賀同時熱鬧了一番。開喪出殯十分華彩。自不必說。不想次年他母親莫氏也病故了。又忙了一場。殯葬之後。賈文物恐富氏懷恨含香。難爲他。偷空向丫頭說要設計救他出去嫁個單夫獨妻。以報他向日之情。商議了主意。那丫頭雖然心中不捨。也怕富氏利害。十分感激。落了幾點淚。那賈文物到丈人家來。將這丫頭的事不敢欺瞞。從頭一一說了。求岳父如此如此設法救他。那富戶部旣疼女婿。又怕女兒果然送了那丫頭性命。次日就到賈文物家來。婿女迎入。他要到親家靈前看看。他夫妻陪了上去。富戶部靈前作了揖。見一個丫頭在傍站着。故意問賈文物道。這女子當日服侍誰的。賈文物道。是先母的侍婢。富戶部回頭問女兒道。這可是當日同你嚷鬧的那人麼。富氏道。就是他。當日倚着奶奶的勢兒。他膽子大多着呢。且等我慢慢的拆洗他。富戶部變下臉來向賈文物道。你府上是詩禮人家。母親的使女。兒子都是要得的麼。賈文物假做惶恐道。這是小婿年幼無知。悔之無及。富戶部道。令堂老親母縱容得他這樣無狀。還不打發了他。留在家做甚麼。賈文物道。先母骨肉未寒。心有不忍。富戶部笑道。你捨不得罷。故如此假說。我却容不得。賢婿就怪我些也罷。吩咐家人道。把這女子帶到家去。叫媒婆替我即刻賣了。此時就行。那丫頭明知是賈文物好情救他。但在此多年。臨去未免傷心。收拾了東西。叩辭主人靈位。大哭了一場。他這哭。三分戀故主。七分感情人。富戶部叫人領了去了。他恐女兒疑心。望着富氏道。向日親家請我來說那些閒話。受了一肚子的氣。我因見他年高了。故此忍住。只得昧着心說了你幾句與他壓氣。我忍到如今。今日才出了我父女的這一口惡氣。這富氏聽見父親說這樣疼愛他的話。好生歡喜。那裡知是他翁婿二人弄偷天撫(換)日的鬼。富戶部回家。吩咐尋個好人家與他去嫁。家人舉薦了一個買賣本分的人。叫做鮑信之。有三十來歲。富戶部一文不要。仍看女婿的面上。反與了丫頭十數金的妝奩。又與些衣服首飾之類。那丫頭千恩萬謝而去\footnote{含香之配鮑信之者。取梅蕊含香以報春信之故。二人合而爲一耳。}。賈文物知道含香得其所天。也感丈人不盡。過有二年。那富戶部也是花甲外的人了。偶染時疫。大勢已危。女兒女婿都在跟前。呼了過來。說道。我死之後。把我跟前的婢妾都揀個好人家打發嫁了去。其餘家中人口房產。內囊細軟。一並付與你夫婦。又囑女兒同女婿道。你們都大了。不比當日幼小。好好的和美過日子。再三說了。瞑目而逝。這個喪事都是賈文物治辦。也着實熱鬧。事完之後。把些妾婢都嫁了人。然後兩處併做一家。這賈翰林家中房產地土家私原有萬餘金之物。今又得了富家這分家產。約有十數萬了。將房屋收拾得華麗之極。僮僕數十。婢婦多人。比賈翰林當日反熱鬧了許多。他如今是個進士。又算巨富之家。自然有人來親近他。就是文人墨士也都相與起來。人雖知他舉人進士來得曖昧。不過背地談論。誰敢當面說他不通。明知他一竅不通。又誰敢出個題目考他一篇不成。況且勢力(利)二字是人人所不能免者。就是有一種假豪傑。嘴雖鄙薄他。不由得身子走來親近\footnote{說盡小人肺腑。}。古語二句說得好。一絲也不差。他道是。

\begin{quotation}

時來誰不來。時不來誰來。

\end{quotation}

正是此謂。那賈文物他也因自己是科甲中人了。雖是趕(擀)麪杖吹火。一線不通。也勉強學些文人的體段。凡說話定要帶些之乎者也的文腔。引用些書語。却是不通得可笑。他到服滿之後也二十多歲了。比當年舉止大不相同。體統雖然尊重。只是怕夫人的心腸分外勝前了。權且按下。且把賈文物向日去投托的那阮大鋮家世細表一番。他係兩榜出身。雖宦居淸要。却屈體求榮。做了魏璫的第一個心腹。他生母貝氏。先是他父親的通房之婢。他腹中懷着阮大鋮。臨分娩時。夢見一個官兒向他道。我唐朝李林甫也。十世爲牛。九世爲娼。皆遭雷擊。今罪限已滿。來與夫人爲子。貝氏驚醒。忽然肚痛。生下了一個兒子。貝氏不知李林甫是甚麼人。過後吿訴夫主。他父親暗想道。此子將來必貴。但恐奸惡不端耳。遂將貝氏陞而爲妾。後來阮大鋮中了舉。他嫡母故後。他父親因貝氏當年夢中有夫人之稱。將就貝氏立爲正室。不久他父親死了。只有貝氏在堂。他丁憂滿了。中了進士。入了詞林。投在魏忠賢門下。做了個走狗。他同時文臣中魏璫已有五個爲首的乾兒。崔呈秀。吳淳夫。倪文煥。田吉。李夔龍。時人稱爲五虎。又有武臣中爲首乾兒五個。舉朝稱爲五彪。田爾耕。許顯純。崔應元。楊寰。孫雲鶴。這十個陷害多人有同梟獍。殘害忠良如豺狼。貪婪淫穢如狗彘。阮大鋮在他衆人中分外又惡幾分。那魏璫也比別的兒子更親厚幾分。你道何故。他知道魏璫惱東林諸公。編了一本點將錄。把一時賢臣搜羅殆盡。如水滸傳名色。天魁星呼保義左都御史高攀龍。天罡星玉麒麟應天巡按周起元。天機星智多星吏科給事中魏大中。天勇星大刀左副都御史楊漣。其周順昌。萬璟。周宗建。黃壽素。李應昇。繆昌期等三十六人爲首。其次地煞七十二人。則周嘉謨。崔景榮。余茂衡。陳于達。周希聖。申用懋等。臨了一個地賊星鼓上蚤中書汪文言。共一百餘人。呈與魏璫\footnote{如此奉承。只落得一走狗之稱。求爲一乾兒猶不可得。何苦乃爾。}。魏璫大喜。按名挨次殺害。此時他又丁了母憂回南京。買了剪子巷一所大宅居住\footnote{剪子巷妙。謂作惡太甚。自剪滅其子也。然而他家實在剪子巷。非作書者謅出。}。他或在家或往北。替魏忠賢探訪事情。生事害人。居止不定。他生平有一戲癖。不但愛看戲。而且好編戲。他在家時。常到牛首祖堂寺呈劍堂作寓。每夕與狎客飮。以三鼓爲率。客倦罷去。他挑燈作傳奇。達旦不寐。他若見了戲班中有個好旦脚。就愛之不置。定要同他相厚一番。要是見了個女旦。竟連性命都不顧了。不弄到手不已。他先遇陰氏時。雖然心中十分相愛。他怕陰氏被窩中利害。故不敢要他。不然他夫妻也不能保全回去了。此時南京有一個小財主姓白。他祖籍原是蘇州。故此人都稱他做白舍\footnote{白舍者。白捨也。謂其白捨嬌嬌與阮大鋮也。}。他家中養了一班戲子。內中有一個女旦。名字叫做嬌嬌。生得模樣俏麗。嬌媚是誇獎不盡。且八脚俱會。那腔口板眼吞吐淸楚。都從牙縫中一字字逼將出來。音韻悠揚。眞似一管簫聲。令人聽得魂消心醉。又只得二十歲。阮大鋮一見了。骨軟筋酥。千方百計要弄他回來。這嬌嬌果然生得好。怎見得。

\begin{quotation}

亭亭如玉。更饒繞梁之音。楚楚如花。時做風騷之態。媚眼中善引淫人之魄。纖腰下慣消浪子之魂。賽過煙花妓女。勝似喬扮孌童。美哉絕世梨園。允矣無雙雌兔\footnote{雌兔二字甚新。如前之別有香。偶然有一或可。若世間果又有此一種。龍陽輩定然痛哭流涕而長太息矣。}。

\end{quotation}

那嬌嬌是一班之冠。起初他主人如何捨得放他。後來虧那有見識的親友提醒了他。道。戲旦固可愛。自身尤爲可愛。他是魏上公頭一個心腹。東林多少大老被他害得家破身亡。何況你一個白衣財主。若惱了他。把你竄入東林黨籍。輕則蕩產破家。重則叫你死無葬地。連正經妻孥皆不能保。依舊人還白白拿去。這豈不是爲惜一指。連肩臂都不顧了。不若趁早送與他去。不但免禍。或者他歡喜了。還可得幾兩銀子。再去買個人來敎罷。那白舍聽了這話。深爲有理。且素常也知他的利害。遂送了與他。阮大鋮得遂了心。大出手。竟送了二十四兩身價。那白舍爲這一個人費半千金還不止。還費了幾年心力敎成。可稀罕他這幾兩銀子。推辭不受。寧可白送。阮大鋮只說了兩聲多謝。莞然笑納。他自從得了這嬌嬌。眞如獲了至寶。要他的心肝五臟煮湯吃。他也情願掏出奉承。另收拾了三間精緻房子與他住。買了個丫頭叫賽紅服事他。做衣服製首飾那不用說得。不但把別的姬妾視同糞土。連他嫡妻毛氏也如同陌路。這嬌嬌善於音律。阮大鋮向來塡的詞。內中或有差謬不合板眼處。他都一一指出。阮大鋮又得了一個良師。更加鍾愛。此時阮大鋮已四十多歲了。俗語說月裡嫦娥愛少年。阮大鋮雖然十分愛他。他在矮簷之下不得不假喜假笑。與他假親厚。倒眞心眞愛看上了他長子阮最。這阮最才二十一歲。一則年紀與他彷彿。二則生得眉淸目秀。齒白唇紅。又輕輕薄薄。渾身骨頭沒有四兩重。就像戲上的一個風流生一般。嬌嬌在戲場上看慣了這般人物。所以心中私愛。就不知這阮最也早已看上了嬌嬌。阮最的妻子郟氏雖然貌也美。心甚淫。却像個泥美人。一點風韻也沒有。所以阮最常道。與他行房。竟是弄死人一樣。有何趣味。他倒愛一個龍陽小子。叫做愛奴。時常幹他的後庭。自從見了嬌嬌之後。精魂俱失。一心一意魂夢顚倒的想念着他。但他係老子的愛寵。可敢輕易動手動脚。只好無人處撂一半句俏話兒勾引。那知嬌嬌愛他比他相愛還勝數倍。男去偷女甚是艱難。女要偷男易如反掌。只消眼角微微留情。話語暗暗遞春。不知不覺就相合而爲一了。你道爲何如此容易。他二人旣兩情相愛。彼此笑語中就有許多勾引的話頭。那阮最旣是拿雲捉雨的班頭。竊玉偷香的領袖。這嬌嬌又是四海納賢的女旦。況又是多多益善的淫娃。還是顧甚麼羞恥。惜甚麼名節的不成。但嬌嬌儼然有庶母之尊。不便俯身下就。然那一種相親相愛之情。自然各別。阮最心雖默會。但不敢輕易下手。或恐忽然有變。如何了得。故此但見他父親一出門。就到嬌嬌房中。姨娘長姨娘短喁喁笑語。奉承得那嬌嬌連心眼裡都快活。他也和顏悅色。大相公長大相公短的相答。阮最有心要下手。他恐老子一時回來撞見了。只得權且納住。一日。嬌嬌鬪着毛氏所生次子阮優頑耍。恰値阮最走來。那阮優纔五六歲。甚是乖巧。嬌嬌笑向阮最道。你兄弟好乖。我心裡很疼他。阮最就遞進一句道。他小呢。知道甚麼。一樣的兒子。姨娘就不疼愛我。不怕人說你偏心麼。嬌嬌笑着。也不答他。抱着阮優在懷中親嘴。阮最也來親那阮優的嘴。幾幾同嬌嬌的嘴三個合在一處做了個品字。他笑着瞅了一眼。又一日。嬌嬌正在吹簫。阮最走來笑道。姨娘。古人說吹簫引鳳。你把我引了來了。嬌嬌住了。笑道。我引的來不是鳳。是一隻狗。阮最笑道。姨娘把我比做狗。那狗是連娘都要跳的呢。嬌嬌也不惱。只笑了笑。阮最見有八九分光景。只等老子遠出。便想着實調戲他一番。好做圓滿功德。一日。春景融和。天氣晴爽。阮大鋮被一個好朋友請了出游燕子磯。阮最知有竟日之空。滿擬今朝要完成好事。早飯後便到嬌嬌房中來。嬌嬌正在那裡看阮大鋮編的春燈謎。阮最笑向他道。姨娘。我父親編的這戲。我細看來。那裡及得古人作的風流。笑嘻嘻向他做着那戲上的關模。道。像那西湘(廂)記中的軟玉溫香抱滿懷呀。劉阮入天臺。又道。你那裡半推半就。我這裡乍驚乍愛。又道。你軟腰款擺。我花心輕摘。露滴肚(牡)丹開。蘸着些兒麻上來。那活捉裡頭的那幾句也好。他道。銀釭下和你鸞交鳳滾。向紗窗重擁麝蘭衾。又道。聽你嬌吐依然舊聲音。打動我往常時逸興。動了我往常時興。也就是那後誘上的白也好。張三郞說。公明兄旣是通家。尊嫂也就可以通一通了。姨娘。你說這樣的曲白何等有趣\footnote{因嬌嬌是小旦。即以戲中之曲白誘之。省用別的口舌。妙。}。那嬌嬌也不回言。微微笑着斜瞅了他一眼\footnote{騷態撩人。}。阮最想道。今番好事就在此一刻了。\endnotemark[13]趁丫頭不在跟前。再着實調戲他一番。便可上手。又笑嘻嘻的道。你這個姨娘的姨字不好。嬌嬌道。怎見得不好。阮最道。一個學生念詩經。念到委蛇委蛇。他照着本音讀。先生說。這念做威移威移。你念錯了。那學生後來但是蛇字他就念做移。一日。吃飯來遲。先生要打他。問他往那裡逃學去來。那學生哭道。我並不敢逃學。方纔在街上看見幾個花子在那裡弄移來。弄了半日。把那移弄得稀軟動不得。纔歇了。我故此來遲。那嬌嬌忍不住笑罵道。促恰短命鬼。旣這麼說。你明日不許叫姨。就單叫我娘。那阮最跑去把門關上。到他面前雙膝跪下。一把抱住他下身。道。我就學蘇州人罵的。做個肏娘賊罷。伸手就去扯他的褲子。嬌嬌道。好大膽。我是你的庶母。都是這樣得的。還不放手。看我叫起來就了不得。若撞了老爺回來。你就該萬死了。那阮最見他話雖如此說。却滿臉是笑。知他心是肯\footnote{不但見笑方知其肯。他說要叫者。夫不叫也已矣。心肯矣。}。說。到此時。就是天雷打我。也顧不得了。我那親親的娘。你慈悲成了好事罷。不然我就要死了。那嬌嬌也不十分堅拒。他不用手捍禦。只拿嘴說。被阮最扯開褲子。嬌嬌假意要拿手掩時。早已被他摸着那又肥又美的妙物。此時嬌嬌已被他調弄得情興如火。任他行事。阮最即將他抱到床上。褪去紅裩。自己忙脫了褲子撲上身。挺着陽物向胯中亂搗。嬌嬌一面把屁股扭着。拿陰戶就他的陽物。一面說道。我當你是頑\footnote{何嘗不是頑。}。你竟當眞弄起我來。一個庶娘母。都許這樣麼。阮最笑道。小娘兒原是混弄得的。一下弄了進去。二人痛弄了一陣。方纔住手。嬌嬌笑道。你這惡強盜。我生生被你強奸了。我今早月事纔淨。若這一下被你弄得了胎。後日若生下來。還是算你的兒女。算你的弟妹呢。阮最也笑道。俗語說的。穿靑衣帶孝。死鬼肚裡明白就罷了。二人說說笑笑。嬌嬌笑道。你也是個不知足的饞狗。你的娘子也就算標致的了。放家食不吃。倒來算計我。你一個人想占便宜弄兩個。太覺沒良心些。譬如你老子此時要想你的娘子。你敢就捨不得了。阮最道。我家的雖然標致。死死板板。一點風韻也沒有。你想。同一個死人幹事有何樂處呢。若只圖模樣。難道雕一個木頭美人也可行樂麼。你道我家食不吃吃野食。你不聽得說。野花偏有色。又道家花不及野花香麼。要說我想占便宜。老子要想我家的我捨不得。那倒不相干。若他老人肯換。我就情願將媳婦洗得乾乾淨淨的孝敬\footnote{終有日遂你這一點孝心。}。把你與了我。我同你做一對恩愛夫妻。同生同死。也是願意的\footnote{古云。出口有讖。將來兩件事都遂了心。}。你說我算計你。這就辜負我的好心了。我見老爺將五十歲的人了。一來恐傷了他老人家。二來恐誤了你靑春年少。故此來同你做伴。不過是替他老人家代勞同孝敬你的意思。嬌嬌笑道。好孝子好孝子。又肯把媳婦孝敬公公。又替老子代勞。又孝敬庶母。眞是難得。二十四孝上又添你這一個。成二十五孝了。媳婦再來孝敬公公。就是二十六個。二人說笑了一會。阮最摸乳咂舌。勃然淫興大起。二人又竭力盤桓了一度。看時日已過午。二人方起身整衣。嬌嬌道。我們的事瞞不得丫頭。恐有洩漏。你須把他也弄上了。堵住了他的嘴。纔好放心來往。阮最笑諾。一日。嬌嬌往毛氏上邊去。阮最走來。把賽紅哄騙着奸了。做了一路。過了數月。嬌嬌有了身孕。他初遇阮最的那日。正値經路淨。日間同阮最弄了一次。夜間又同阮大鋮高興了一番。連他自己也不能辨下種之人是子是父。肚中之物是子是孫了。到了月分滿日。分娩了一個女兒。長到四五歲。眞好一個孩子。形狀似母。神情同類阮最。阮大鋮也只說嫡親兄妹。雖係隔母。到底同老子的骨血。那裡疑到是兒子替他代勞所生。那孩子容顏秀美。生性聰明。沒一個人不疼愛他。阮大鋮同嬌嬌竟疼得如掌上明珠。因起個小名叫做寶姑。阮最知嬌嬌受胎先他起而父後斷(繼)。且模樣又相似。明知是自己所生。雖不敢明認。却也暗暗疼這寶兒了不得。且說那阮最的妻子郟氏。他身子雖不善流動。心性却十分流動。他是宦家之女。從小父母管敎。習成個端莊樣子。他並不是一塊木頭一般的人。只因阮最自己性情輕佻。在外邊花柳叢中混慣了。見的都是戲旦淫娼那種舉動。後來又每日見嬌嬌的態度風騷。語言俏利。眞個引魂勾魄。與郟氏兩下相形起來。越覺得他死板了。所以不甚相愛。旣不相愛。到夜間偶然做那一番事。也不能十分鼓舞豪興。只算做虛應故事的一樣。那郟氏雖有千萬分的興頭也不能施展。況是丈夫同他就淡淡交合。再要做出那淫腔浪態來。又恐丈夫嫌他鄙賤。所以他一身的騷淫技倆。未得展出十分之一。他見丈夫旣同嬌嬌打得火熱。就得空時。再不於他身上用工。反去用工在愛奴身上。那愛奴有十五六歲。雖不爲美色妖童。也還生得白白淨淨。頗有可愛。一日。想道。他旣寵幸得小子。我也可以寵幸得。此處無人敢來。除此小子之外。也再無可幸之人。他旣偷得庶母。我便幸幸小子也無妨。況幸上了他。不但可以聊且解饞。且俗語說得好。溺愛者不明。他主人旣一心愛這小子。諒不疑惑。但恐年幼無濟於事。又想道。人說短棍撥火。強如用手。且救目前。再作養他二三年。自有長大的日子。強似如今下邊這張嘴長吃月齋。弄得望梅止渴。饞眼嚥唾。心中旣注意於他。自然又另是一種顏色。笑面常施。恩波屢及。不拘做甚事。便不甚防閒他。那小子做了龍陽數年。豈止阮最一個。或以此窟爲覓利之藪。或與同類彼此交易。爲取樂之竅。他却不曾遇過婦人。因時常進來。見郟氏不在面前。就同那丫頭打牙犯嘴的調笑。那丫頭也被阮最開闢過。一月之內還不得一場快活處。也是久違渴慕的了。就是逆來也情願順受。而況乎順來者。可肯逆拒。一日。阮最出門去了。郟氏有事往婆婆上邊去。那小子進來。見只那丫頭在房。便上前抱住。要同他如此如此。丫頭道。怕奶奶撞了來不好。相公不在家。我同你到書房裡去。二人遂到書房中。借主人的閒榻。成就了鸞交鳳友。恐有人來。苟且了事而已。也弄過多次。促促忙忙。總不像意。況那丫頭只筋臍下有件婦人之物。他那面上雖不十分醜陋。也毫無可愛之姿。愛奴旣得了隴。又望起蜀來了。看見郟氏生得甚美。時妄想他胯下之穴。暗暗尋思道。婦人此竅津津有味。覺比我們臀後的窟味似甚美好\footnote{好男風者則非此想。}。若美人的。自然更佳了。怎得嘗一嘗奶奶的妙味。也不枉一場相遇。雖有此心。但有主奴之分。豈敢妄動。古語說。日近日親。他每日在房中出出進進。那郟氏或早間坐床上裹脚。露着白森森的腿兒。因不防他。常被他瞥見一眼。或臨聰(窗)梳頭。遇天暑穿着對衿小衫兒。揚起兩手理髮。袖手捲下。影影露出乳峯。嫩藕般兩隻玉臂。或着紗褲。偶然在日影之下微微照見雙股。他好生動火。只好在無人處閉目存想。打個手銃。借此當彼。後來見郟氏在無人處和顏悅色。間或向他吟語說笑。他雖不敢答應。也做個笑臉相迎。這小子是滑透心的人。何事不知。也就心照了幾分。故意時常在房中不住來回的走。一日。郟氏在房中洗澡。叫丫頭拿換下的衫褲到後邊去洗。把房門虛掩着。這小子恰巧進來。聽得房中水響。在門縫中一張。見郟氏赤身坐在盆中。上下無一點瑕庇(疵)。猶如一個玉人。兩個小小嫩乳圓緊得有趣。但他那妙物浸在水內看不見。悄悄蹲下。要等他起來。做個一覽無餘的意思。屛息以俟。那郟氏先聽得有脚步響。忽然住了。還當是丫頭。問了一聲是誰。不見答應。他就知是愛奴。故意道。我洗澡呢。是誰。不許在外頭張。此時已洗完了。站起來。倒把臉朝着門外揩抹。又蹺起一隻腿來。踩在盆沿上揩下身。那又肥又美的一條細縫。正對着愛奴的那隻眼睛\footnote{那隻妙。在門縫中張。只得一隻眼見。}。愛奴一見。渾身一酥。那厥物突然跳起。忙用手攆(攢)住。郟氏雖揩着身上。眼光却射着門外。見有個人影兒。猛然把門一開。那愛奴躱不迭。撞了個滿懷。郟氏笑罵道。好大膽的奴才。你敢來張我。那小子跪着叩頭。道。小的怎麼敢張。一時無心進來。並不曾看見甚麼。郟氏也不穿衣。精着身子。只用手掩着下身。問道。\endnotemark[14]相公呢。愛奴道。出門去了。郟氏暗想道。趁此不做。還等幾時。走到床上坐下。道。你來。我問你。那愛奴進來又跪下。郟氏笑罵道。你這大膽的奴才。你常常同你相公幹那齷齪沒廉恥的事\footnote{罵得是。何不來同我幹這乾淨事。而同他幹那事也。}。我倒不管你。你今日公然偷張我洗澡。你端的起的是甚麼心。你就說你該甚麼罪。愛奴見他色旣不怒。語又和而帶戲。也就放了膽。說道。小的實出無心。憑奶奶恩典處治罷。郟氏道。看有人來。你且去關了門。再來問你。那小子知有好處。忙去關上門。過來時。郟氏已仰臥在床上。側過臉來向他道。你這樣大膽。我如今睡在這裡。看你敢把我怎麼樣的。愛奴知是此處無銀之意。取出肉具。如飛上床。一翻上身。就往臍下直攻。剛剛湊巧對着。一個是鐵硬的陽物。一個是水浸透的陰戶。一下到底。就抽起來。郟氏先以爲小子未必懂局。那裡就敢動手。等他求饒。還想用些話開釋他放了心。然後使他感恩。好來賠罪。雖然在此候敎。少不得還有些須做作。不想他竟突然而來。一下竟直搗至根。亂衝亂突。那些虛文套數半點也用不着。覺得小子的陽物雖不及阮最的大。而堅勇過之。一面笑。一面罵道。好奴才。公然大膽。竟弄起我來。我也強不過你。憑你弄。等相公回來。看我可吿訴。那小子得遂素願。下力死弄。也笑着說道。奶奶的恩典。就對相公說。小的不過是個死。不如此時死在奶奶肚子上罷。說着。越弄得狠。郟氏覺有妙境。不必再說。雙手堅勾。往上亂就。那小子弄了一度。洩訖一度。陽物尚堅。他初嘗美味。不捨得就歇。定了一定。又復弄起。兩度之後。還不肯住。有個要三度春風之意。郟氏起先以爲這小子初出茅廬。不過拿他來暫且解饞。以待將來或有妙處。不意如此雄壯。他也丢了兩次。實出望外。見他還不肯歇。遂道。恐丫頭來。你且去着。你常常進來。等有空時。我同你商議個長久之策。那就可放心了。那小子也是意外奇逢。已遂心滿意。便歇住。雙手捧着他臉。道。奶奶下邊的寶貝賞小的嘗過了。求把寶貝舌頭也賞小的嘗嘗。郟氏笑着也便吐出些。那小子含住咂了幾下。下面又狠狠的搗了幾搗。那郟氏也往上湊了幾湊。小子纔起來下床。拽上褲子。忙出去了。郟氏也爬起。重在洛(浴)盆中將牝戶掏洗淨\footnote{縱然掏盡湘江水。也不能再洗此軀淸白矣。}。然後穿衣。睡在床上。要想長策。想了一會。道。別無可慮。只怕丫頭礙眼。況丫頭又是他主子收用過的。倘或落在他眼中。暗向他主子說。就不好了。須得叫愛奴把丫頭也弄上。事就好處。\endnotemark[15]一日。阮最到嬌嬌房中敍濶去了。郟氏在房中正望愛奴來。見他走到面前。忙摟在懷中親了幾個嘴\footnote{反是郟氏親愛奴的嘴。寫出淫之至。愛之極也。}。商議這話。愛奴笑道。奶奶不說到這裡。我也不敢說。要怕別的。我就沒法。若單怕這丫頭。不瞞奶奶說。我同他弄過多次了。郟氏笑着在他頰上輕輕咬了一下。道\footnote{郟氏此時可謂風騷極了。流動極了。却不死板了。若阮最見之。喜乎。怒乎。}。你這小奴才。我還當你是個雛兒。原來竟是個老賊。旣如此。就好處了。今日老爺不在家。相公在嬌嬌那淫婦房裡去。有一會肏搗呢\footnote{只許自己同奴才肏搗便罷了。丈夫同嬌嬌肏搗便氣不忿。眞是淫婦心腸。又自各別。然而郟氏之私奴。亦由於阮最之烝母。不然。何一變淫騷至於此。}。趁這空。你可如此如此。我衝破就好做了。愛奴應諾。郟氏出來對丫頭道。你看家。我到嬌姨處走走來。方纔出去。愛奴摟住着丫頭。道。每常在書房裡。怕有人遇見。再不得快心。奶奶這一去。有一會纔得來。今日在這裡做個快活的。那丫頭有何不肯。二人脫了褲子。就在堂屋椅子上扛起腿來就弄。那郟氏是個商量定的。只在門口站了一會。就輕輕推門進來。見他兩人正弄得好呢。假意喝道。好奴才。幹得好事。愛奴假做吃驚。忙撇了丫頭。跪下哀求。那丫頭又羞又怕。褲子也穿不及。光屁股跪着。只是低着頭。郟氏道。我此時也不同你們講。等相公來着。看他怎麼發放。遂把兩條褲子拿着。道。這個就是證見。遂走進房中去了。那丫頭急得只是哭。抱怨愛奴道。正經到書房裡去罷了。怕人看見。要在這裡。我看在書房裡弄了這麼些回數。也沒有遇見人。纔在這裡。就被奶奶拿着了。都是你帶累我。若吿訴了相公。怕不有個半死麼。愛奴道。哭也沒用。抱怨也沒用。想個法兒救命要緊。丫頭道。你就想。我是不會想的。我又不圖你的銀子錢。白白給你弄了多少回數。前日間我要根糖吃。你還捨不得買給我\footnote{以此物換一糖而不得。其情曷苦矣。}。你今日要帶累我捱打。我看你良心也過得去麼。愛奴故意想了一想。道。你悄悄去看看他可做甚麼呢。那丫頭輕手輕脚去了來。道。放着帳子。在床上睡呢。小子道。我兩個有命了。等我去看。他要睡沈了。我也偷他一下子。偷上了。不消說。大家造化。若偷不上。那就是命了。丫頭道。不好。若不肯。越發不好了。小子道。總破着我的命。若弄犯了。不過我是個死。你也推是我強奸的。你也就沒事了。丫頭含淚道。除了這個。實在也再沒法了。你可輕輕的去。那小子進去多會。不見動靜。那丫頭走來張時。見帳子亂動。就知道事妥。心中暗喜。纔不慌了。張了一會。只見愛奴先下床來。然後郟氏掛起半幅帳子。叫丫頭。他忙走進去。郟氏也不說別的。便道。看愛奴的面。饒了你。把褲子撂與他。道。穿起來罷。但下次不許瞞我私偷。那丫頭臉上纔有了些笑容。忙把褲子穿了。此後打成一家。郟氏同愛奴三五次中也分惠他一次。郟氏又吩咐愛奴同丫頭打聽。老爺若出門。相公若到嬌姨房中去行樂。你便到我房中來行樂\footnote{針針相對。這才叫做疏而不漏。}。再說那寶兒到了八九歲。聽他母親唱曲。不但一字不得遺忘。還唱得一腔一板不走。到了十四歲。出落得像個燈人兒似的。比他娘還覺風流。女工針指雖一絲不通。淫詞艷曲却記了滿肚。阮大鋮的次子叫做阮優。正纔十八歲。人稱他阮二郞。雖然輕佻與乃兄無異。却生得精精壯壯一條健漢。不像阮最柔弱。他愛這個妹子眞出尋常。要一奉十。百依百隨。只要圖妹子歡喜。別人看着。只說他心疼妹子。誰知他存了一肚狼心狗肺。要把妹子哄厚了。想採他胯下的那朶鮮花。那寶姑時常見他老子不在家。他母親與大哥哥嘲風弄月。眼去眉來。常常做些不尶不〖兀介〗的事。也都落在他眼中。他心中道。我母親放着有爹爹。他還同大哥偷情。我二哥這樣疼愛我。我何不同他也厚上了。料母親也管我不得。他旣有了這一點私心。那阮二又是素常有邪念的。何消費力。一日。阮大鋮偶然高興。要同嬌嬌打個白仗。因他房中怕女兒看見。同他到一間密室去了。恰好阮二走到妹子房中坐下。寶兒見左右無人。笑着對阮優道。哥哥。你今年十八歲了。我前日聽得爹爹說。今年上冬替你娶嫂子。說這花家的女兒標致得很。還有大半年。你心裡不急麼\footnote{反是他先勾。諺云。上樑不正下樑歪。其母之淫若彼。無怪乎女之不方也。}。阮優也皮着臉道。急也沒法。誰肯可憐我。妹子。你明年也十五了。別人家十五歲養娃娃的不少。但是你還沒有許妹夫。大約比我還暗急呢。我倒好不可憐你的。你嫂子雖然說標致。料道那裡如得你。我要娶了像你這樣人兒。我就把他頂在頭上過一世。寶兒笑着斜溜了他一眼。道。我就這樣好麼。是你疼我。所以這樣說罷了。阮優道。我同你也是前緣。我心裡疼你。眞是說不出的。偏生生在一家。若是兩姓。我憑着怎樣也要娶你做妻子。寶兒道。我也是這樣想。就是夫妻也沒有像你這樣疼愛我的。大哥哥同嫂子不是樣兒麼。我看他倆個還言和意不和的呢。我也感激你不盡。願來生同你做個夫妻罷。阮優見他是開門揖盜倒勾情的話。諒無更變。大着膽。上前捧着臉親了個嘴。道。你旣這樣好情。那裡等得到來生。我們雖做不得眞夫妻。權做一對露水夫妻。你心下如何。寶兒道。哥哥你旣愛我。我還有個不肯的麼。阮優忙關上門。怕嬌嬌回來。不敢脫上衣。只把褲子卸下。上床動作。他雖憐憐愛愛。款款輕輕。但阮二的陽具甚雄。寶兒又甚年幼。痛楚難禁。阮二甚是憐惜。意欲中止。倒是寶兒不肯。道。你只管來。說不得我忍着些。阮優也不敢大張旗鼓。只微微見意而已。便收兵罷戰。兄妹奸淫。行同禽獸。有個黃鶯兒贈他兩人道。

\begin{quotation}

伶倒(俐)小寃家。俏身材。面貌佳。情深願與鸞鳳跨。輕開玉葩。忙舒肉芽\footnote{肉芽二字新奇。}。兄奸親妹眞堪詫。但嗟呀。何生禽獸。父母行事差。

\end{quotation}

臨了這一句歸罪於他父母者。謂阮大鋮不強占了嬌嬌來。何得有這樣辱門敗戶的女兒。嬌嬌若不偷阮最。寶兒也不敢這般大膽。豈非父母行差乎。此後他兩個親兄妹竟做了一對暗夫妻。也偷過幾次了。寶兒的一個丫頭叫做待月。阮優也奸上了。以便往來。過了些時。寶兒眉散胸高。與做女兒時光景各別。那嬌嬌兩隻眼睛如琉璃葫蘆一般。如何瞞得。他早看得有些蹺蹊。把寶兒叫到房中。摸了摸他的下體。那寶兒已成兩瓣了。便追問所以。寶兒隱瞞不住。方說這寶貝是他二哥用金剛鑽打的小小個眼兒。嬌嬌一腔怒恨。不敢吿訴阮大鋮。只背地將阮優痛數說了一場。把女兒羞辱了幾次。這寶兒不責備自己不是。反心中暗恨母親。道。你現同大哥通奸。還來管我。我看個巧。叫二哥拿住。把他也弄在網裡。看還說甚麼。遂暗地與阮優商議停妥。一日。阮大鋮外出。嬌嬌趁空。大白晝約了阮最在房中高興。寶兒冷眼見了。他那個心腹丫頭待月是他的一個紅娘\footnote{待月者。取待月西廂下。已比做紅娘矣。恐或有看不出者。此處故提起紅娘二字。}。這丫頭已是阮二串熟厚了的。寶兒叫他忙去叫了阮優來。對他說了。叫他在母親房門外等着多時。阮最事畢。穿衣開門出來。一眼見了兄弟。臉俳(緋)紅。低着頭。忙出去了。阮優跑進房中。見嬌嬌光着屁股坐在床上。正纔拿着褲子要穿。阮二劈手搶下。一把抱住。道。你同大哥好弄。一起手我就在門外聽着這半日了。你同我弄弄就罷。不然我就聲張起來。妹子就是證見。嬌嬌知爲他同女兒所算。遂道。你同妹子做那樣的事。我倒忍了。你倒來拿我的短。阮優道。那沒有憑據。你此時的眞贓現被我拿住。你還說甚麼。說着。便一手伸到胯中去摸。嬌嬌去推他的手。他便伸了個指頭到他牝中勾住。道。你推推。我就摳個大窟窿。嬌嬌一來推辭不得。二來他也不是怕此道的。就不嘖聲。阮優便將他按倒。自己扯開褲子。取出陽物。弄將起來\footnote{阮最還是以情求。阮優竟是以強上。所以二人死時各有重輕。}。原來阮優的陽物比他父親哥哥的強壯許多。把個嬌嬌弄得心迷意亂。騷態百出。弄了多時。方纔歇手。這阮優向來雖愛妹子。但他是個雛兒。枕蓆上風流一毫不知。只好仰〖扌扉〗着揸開腿憑人弄而已。這嬌嬌是個老作家。攧搖哼喞夾五個字無不精通。把個阮二喜得魂飛。以爲奇遇。至於嬌嬌。他當年就嫌阮大鋮老了。何況到今。他愛阮最年少風流。但本事原自有限。今日遇了阮二。陽大力強。又頂提擎捎刮五個字件件知曉。正配着他的五件。弄得遂心滿意\footnote{可謂後來者居上。}。深恨相遇之晚。阮二自遇他之後。魂夢都落在他身上。想道。我看他弄得那樣子。也就算騷淫極了的。哥哥久是他的厚友。除非我極力弄得他十分痛快。纔可奪他的歡心。弄下了許多好春藥。安心來同他取樂。有那日。阮大鋮同阮最到一個朋友家去拜壽吃酒。阮優托故不去。打聽父親哥哥去了。忙把春藥服下。又擦些在玉脛(莖)上。就到嬌嬌這裡來。頂頭遇見寶姑。那寶姑見哥哥這幾日忽然疏淡了他。心中也正想高興高興。遂一把拉着他的手到房中。並肩坐下。偎偎倚倚。嘴中不好說得。心中有十分要弄的光景。說道。今日爹爹同大哥哥都不在家。此時母親又睡覺。你同我在這裡大坐坐。不要去\footnote{巧言不如直道。此時竟拉他要弄。他也沒法推。}。那阮優知他是要如此的意思。因一心想着嬌嬌。假說道。我好幾日沒同你頑頑了。不知你母親睡着了沒有。我看看去。若得空。我就來陪伴你。寶兒以爲實話。放他去了。他走過嬌嬌房中。嬌嬌只當他父子三人同出門去了。無所指望。在床上睡覺。阮優忙把門揷上。揭開帳子。見他睡思正濃。輕輕褪下褲子。分開兩腿。弄將起來。嬌嬌矇矓星眼見是他。笑罵道。賊短命。我當你出門去了纔睡睡。大靑天白日來做賊。看你妹子撞來看見。阮優說謊道。我纔看見妹子也睡呢。房門我也揷上了。一面說。一面扛起他雙足。竭力大弄。帳鈎搖得叮咚亂響。陰戶中水聲震耳。嬌嬌覺得他的陽物如一塊燒紅了的生鐵一般。又熱又硬。弄得爽快不過。哼個不住。他二人正在發狂。那寶兒只說阮優就來。撫摩小牝等候。不想等了一會不來。悄悄到母親房門口竊聽。聽得兩人正在高興。聽得他娘的那個哼聲十分難聽。又聽得一陣響聲更凶。響過了一陣。忽聽見阮優道。親親。我同你情孚意合。我有句話問你。你要說眞話。我比老爹同哥哥的本事何如。又聽他娘笑道。你爹有年紀了。有其名而無其實。他雖然離不得我。實在事有限。況且身邊人多。那裡還有本事支應得過來。別人還罷了。馬六姨那騷奴。他哄得你爹滴溜溜的轉。會哄漢子多着呢。你爹倒同他弄得多。你哥哥身子軟弱。力量單微。心有餘而力不足。心肝。實不瞞你。我也遇過幾個人。像你的就少了。實實可我的心。阮優連親了幾個響嘴。道。親親。承你這樣愛我。我也沒得別的報你。只有竭力報答你罷。又聽得他娘道。你心上有你妹子。他年紀又小。臉又嬌嫩。又是你從小心愛的。況且他那個東西又是你破的。自然緊。就我比你大着十來歲。臉上也老了。我自己也知道。我生產過的東西。自然寬鬆。你不愛他倒肯愛我。你是初同我相交。少不得拿甜話兒哄我。過後頑厭了。敢就嫌我老。就要變心。你上冬再娶了花家娘子。他又生得好。想就不理我了。親親。那就把我要想死了呢。阮優見他說這話。便發誓道。我若負了心棄了你。後來粉身碎骨。不得好死。就是花家女兒生得好。料道也沒有你這樣風流知趣。你自己假意說老。我看你還一指甲掐得出水來\footnote{未必然。那沒指甲的大指頭向此道中通得出水來則有之。}。嫩得很呢。至於妹子。我當日不曾遇你。故同他相好。他是個雛兒。一點情趣不知道的。況他終久要嫁人。也不得長遠。說着。又聽得響起來。比先更凶。那寶兒聽得淫水淋了兩腿。用手揉着花心。心中大恨道。這負心的短命。我朶鮮花付了你。況且母親還是我總成你的。原圖堵了他的嘴。我兩個好作樂。你今日倒負起心來棄了我。這沒良心的負了我也罷了。恨我錯認了人。母親恁大年紀還不識羞。旣有爹爹。又養着大哥。還來爭我的鋒。忿忿的回房。倒在床上睡下暗泣。那阮二弄夠多時。兩下興足。穿衣開門出來。忽然想起妹子相約的話。也覺得心上過不去。張了一張。見他面朝裡臥着。便一溜煙出去了\footnote{眞負心。}。此後二人如膠和漆。如糖拌蜜。反把寶兒撇開。這寶兒原圖捉了母親的破綻好同哥哥痛樂一番。不想反被娘占了去。即如一個大酒量的人。到一個極吝嗇的東家去。知道他家的酒再不能足興的。拿話譏誚他道。府上的酒從不能醉人。倒不如學古人醴酒不設的爲妙。這話本要激出酒來痛飮。孰〈不〉知那主人竟恭敬不如從命。只待飯而已。連那不盡興的酒都不得沾唇。你道可惱不可惱\footnote{譬喩得甚趣。}。寶兒的心腸即此一理。不由得那醋味自丹田直衝至泥丸宮。被天庭閉塞住了。從口中發洩出來。時常拿冷話譏誚母親。道。一之爲甚。其可再乎。或又道。兄終而弟繼矣。或又道。父子連科。兄弟同門\footnote{寶兒怨不得母親哥哥。只怨自己爲法自斃。}。那嬌嬌却不好認他話頭。也常拿話敲打他。道。齊襄公通妹。後爲稱連管至父所弑。鼓兒詞上說。隋煬帝奸妹。所以被五花棒打死。如今的春牛就是他。因爲阮二的這根肉棒槌。他母子竟成仇敵一般。那寶兒待阮優也就情意淡淡。不似向日親熱。但他終嘗得這一宗甜頭。忽然離開。心中時刻難過。一日。嬌嬌不在房中。他偶然過去。見有許多黃蠟。是阮大鋮買來熬燰臍膏用的。他心有所觸。拿了一塊到自己房中。用火烤軟。搓了一根圓棍。如阮優肉具大小。晚間睡下拿來消遣。過了幾日。覺得短細。遂漸加添。極粗極大。儘陰門容得下而後止。把一個嫩而且緊的物件。弄成了個寬大無〔比〕的東西。雖覺出進有些意思。但他生得嬌軟。手腕未免酸痛。不能長持。那待月是他貼心的牽頭。竟叫他同臥。將蠟棍用帶子束住。繫在腰間。同他交媾。他也繫了。同待月戲耍。兩人也不像主婢。竟似一對雌夫妻一般恩愛。阮二良心難昧。間或要同他溫溫舊。不但強而後可。寶兒毫無當日情愛。阮二亦中輟而止。從此益發淡了。那時有一個勞御史在北京做官。也是魏璫黨羽。同阮大鋮都是一類。他兒子勞正。在南京家中養病。因年紀大了。他寫書托了個親厚朋友到阮家來求親。嬌嬌嫌女兒爭鋒礙眼。巴不得把他送出。百般慫慂着阮大鋮。久了。行茶下禮。豐富不消說得。擇日來娶。阮大鋮陪的妝奩也從厚。一則是獨女。二則看嬌嬌面上。三則奉承親家。還陪了三個丫頭帶待月四個。那寶兒因同母親爭鋒成了寃家。見哥哥又變了心腸把他撇開。聽得出嫁。打點去大大的快樂一番。不但一點眼淚不落。連一毫留戀之意皆無。欣欣然上轎而去。這勞正年紀二十五六。他自十二三歲就水旱齊行。幼年作喪太過。所以成了癆症。他父親因他怯弱。故延到此時纔替他完姻。他是閱歷多了婦女的。何所不知。成親之時。寶兒雖百般做作。兩腿夾得死緊掩飾。但他那已經開闢的物件如何哄得那過來人。勞正早已知覺不是處子。未及盡興而止。因兩家俱是仕宦門第。怕張揚醜聲。只得耐住。到次夜即推有病到書房去睡。總不進來同床。有一調搗練子說那寶兒道。

\begin{quotation}

假裝緊。寶(實)寬鬆。但聽檀郞任意攻。做作料難欺識者。元紅久矣屬親兄。

\end{quotation}

這寶兒心中滿擬嫁了丈夫。明公正氣得一番大弄。強似同哥哥做那鼠竊狗偷的事。況且聽得新郞大着十一二歲。必定更老成歷練。今嫁了來。不但一次快樂不曾經着。連新郞的那物件滋味也不曾深嘗。仍舊是在家做女兒一樣形單影孤的。當日還間或嘗嘗哥哥的陽味。如今連這味都不能得了。但這話說不出來。眞如啞巴吃黃連。只好苦在心裡。過了滿月之後。回到家中暗暗哭訴與母親。嬌嬌也只說女婿是個癆病鬼。心中懊悔。那知嫌他女兒是個破罐。寶兒這一個月熬狠了。同阮二時常大弄。嬌嬌一來到底疼女兒。二來不過一個月他就要去。況自己還有夫主同阮最可以行樂。何妨暫讓寶兒。住了些時。少不得要回去。到了勞門。仍舊孤幃獨守。終日短嘆長吁。以淚洗面。一日。待月做了一根蠟棍送與他。道。姑娘。你日夜愁煩。何時是了。還是拿這個解解悶罷。寶兒接過。擲之於地。道。當日在家無可奈何。借此解饞。今已嫁人。不能同丈夫如此。豈有終身用一蠟夫哉\footnote{蠟夫。奇聞。}。待月見他不要。拾起留爲自用。過了月餘。待月說道。姑娘。你這一寸眉尖怎經得千層顰皺。成日這樣熬煎。豈不苦壞了身子。我聽見姑爺今是(日)不在家。何不到書房裡去走走。推解一時之悶。寶兒先還不耐煩去。被待月苦苦相勸。他主婢二人纔走了出去。他這書房後邊有個小園。有一小圈門可通上房。他遂(從)此門入去。悄無人聲。園中幾缸蓮花開得正盛。內中有一盆開了一朶並頭蓮。待月笑着道。姑娘。你看這枝並頭蓮正向着你。大約今夜定然有喜事了。寶兒先把眉一愁。後微微一笑道。得應你的話就好了\footnote{古云。庭前生瑞草。好事不如無。不想並頭蓮應在禿小廝身上。}。看了一回。走進書房。果是明窗淨几。前院門閂着。院中盡是梧桐芭蕉。遮得並無日影。淸風徐來。着實涼爽。西牆角一間茶室\footnote{待月之西廂。原來在此。}。也走去看看。見那個看園的禿小廝姓張。有二十多歲\footnote{不意是這樣一位好張生。}。天熱無事。他地下鋪了一床竹蓆。上身赤露。一身黑肉。把布衫捲成一團做枕頭高臥。有一調駐云飛僧(贈)他道。

\begin{quotation}

腦袋稀奇。不長頭毛只長皮。裹不得天羅地。挽不得風流髻。嗏瘡滿鬢毛稀。黃膿如涕。走到人前。一陣乾蝦氣。偶爾鬆頭似雪飛。

\end{quotation}

這小廝是個雞屎禿。滿頭瘡蓋。遍頂黃膿。兩隻毛腿。脚上皴泥大厚。仰面睡得正濃\footnote{非極寫禿小廝之不堪。乃寫寶兒不堪之甚也。}。穿着一條破麻布褲子。襠上一個窟窿。那小禿子想是要乘涼。剛剛在那洞中舒了出來。直豎豎粗而且硬。寶兒暗吃一驚。道。這樣個蠢人。倒有這等個妙具。淫情一動。不由得意亂心迷。因愛上了那小禿子。也顧不得那大禿子穢惡了。待月正要叫那小廝。寶兒連忙扯住。拿袖子掩口笑着\footnote{騷態可掬。}。悄向他道。你去看看後門。不要放人進來\footnote{此處方見先寫院門閂着四字省筆之妙。}。待月知他看上了那物件。也笑着〈情〉向他道。姑娘要應並頭蓮了。含笑而去。寶兒慾火大發。那管他醜俊。忙褪去裙褲。輕輕跨上身來。對準了。用力往下一坐。就進去了一半。又一連兩坐。把個小禿子全身鑽入。那小廝驚醒。見是主母。打扮得嬌滴滴俏生生。玉天仙一般。把他做了坐具。一個嫩汪汪軟秋秋的白屁股騎在他身上呢。那裡還顧得甚麼主僕名分。用手搊扶着他一起一坐。自己的黑股向上一迎一落。寶兒別了阮二一月有餘。枯渴久了。不多時便丢了。那小廝道。奶奶。這樣弄。你吃力。請下來睡着。等小人來服事。寶兒依他。就在光蓆上睡倒。禿小廝就拿他枕頭的衫子替他墊在股下。他爬上身好弄。拿出那吃奶種園的力氣。命都不要。死弄了一場。弄得寶兒丢了又丢。渾身通暢。遍體酥麻。也不管家奴小廝。心肝親哥叫得震耳。多時罷戰。寶兒穿了裙褲。拉着小廝的手到書房內。他在一張圈椅上坐下。將小廝摟在懷中\footnote{親愛至此。阮大鋮奈何。}。說道。晚上你在上房院子門外等着。我叫待月出來接你。黑了進去。天不亮出來。每夜不可誤了。禿小廝連聲答應。欣喜欲狂。寶兒又道。那丫頭你也同他弄弄。好叫他做牽頭。那小廝豈有不願。寶兒到後窗跟前點手喚待月。他把門閂上。笑嘻嘻走了來。寶兒道。我約下他了。你晚上開院子門帶他到屋裡去。看不出他恁個人兒。倒着實在行。你也同他試試看。待月假意道。我不消。叫他留着力氣晚上服事姑娘罷。寶兒向小廝努了個嘴。那小廝上前將他一把抱住放倒。就將天地借爲衾枕。他褲子是破的。不用脫。陽物出來得好不便宜。只扯去了待月的褲子。那待月口中還說。我不我不。已被他塞了進去。一陣亂抽亂搗。待月屁股往上混攧迎。口裡戰篤篤的。我不哦我不哦儘着叫。把個寶兒笑得幾乎笑倒。他我不了好了一會。兩下俱丢。直等小廝拔了出來。他纔不說我不了。主婢二人無心得了奇遇。暗暗歡喜回房。禿小廝喜得咧着嘴只是笑。還疑是做了一場好夢。想着夜間定有一場大弄。趁主人未回。且去睡睡養力。又到茶室中來。將那衫子捲兒推到一頭枕着要睡覺。滿臉滿脖子黏〖氵韲〗〖氵韲〗的。只當是方纔使力禿頭上掙破了淌的膿。拿過一看。方知是墊在主母股下淌的淫精。歡喜得他把那衫子緊緊抱在懷中。叫了幾聲心肝寶貝。他纔睡了。那寶兒心中快活。每常那些愁悶都抛到九霄雲外。到了房中也睡了一覺。天纔一黑。把丫頭們攆開。悄悄叫待月去開了院子門。香(禿)小廝正蹲在門底下等。見待月來開門。忙抱着親了幾個嘴。扯開褲子。站着抽了幾下。纔同了進來。脫衣上床。無話可說。扛起腿就幹訖一度。寶兒叫待月也上床來。三人一床混滾。這禿廝兒竟成了彩蝶兒。纔向東又向西。亂採花心。直到天色微明。方纔送他出去。有幾句說他主僕二人道。

\begin{quotation}

那大禿頭拿他的小禿頭。直鑽寶兒胯下寶眼。這騷寶兒將他那扁寶兒。含吞禿子腰中禿頭。禿子拿力。掙得大頭上膿流。小頭上膿也冒出。寶兒遂心。樂得上嘴中涎淌。下嘴中涎更漰湃。將多時。只呼得出好一個爽心樂意的禿心肝。罷戰後。頻呼幾聲好一個風流騷浪的嬌寶貝。那還知主僕尊卑。怎逃得輪迴報應\footnote{看書者到此須着眼。不可作淫書觀。}。

\end{quotation}

不到一個月。把個烏黑的壯健小廝。弄得面色萎黃。成了個黃病鬼。閉眉合眼。大白日不拘到那裡就打盹。支撐不住了。陽物也不似先堅久。屢屢求饒乞命。寶兒道。你要我放你。除非尋一個替身來就罷。不然你就死。我也顧你不得。那小廝忙應道。有有有。這容易在我。在我容易。包管比我強十倍的送上。你道這小廝如何應得這等爽快。一則他圖饒命。二來他自私通寶兒之後。寶兒常與他些銀錢。他不敢做衣服穿。怕起主人之疑。却終日肥肉大酒買來受用。他同夥的家人姓馬。也是個沒妻小的。因他陽物過大。人起他混名叫馬兒騾。他冷眼看見多次。疑心道。他是何物得來的錢。這樣大吃大用。一日。馬兒騾掏出幾十文錢。打了三四斤燒酒。買了幾塊豆腐乾。請這禿小廝。吃醉了。儘着拿話套他。這小廝一者有了幾分酒意。二者正要顯主母有這一番垂靑格外。他有這一段僥倖奇遇。盡情奉吿。馬兒騾聽得津津有味。甚是垂涎。也想揷上一脚。同他商議。求他周旋。倘得嘗主母的美味。若得了賞賜。定然買美味還加美酒酬謝。禿小子道。這事不可造次。弄得不好。大家都沒戲唱。等看機緣。纔可行事。不想寶兒叫他尋替身。不但不負馬兒騾之托。擾他美物美酒還是小事。且可救了自己。遂極力舉薦。誇馬兒騾的陽物怎樣大怎樣雄。並說了他的混名。把個寶兒聽得那慾火打十萬八千毛孔中冒將出來。恨不得即刻就叫他來救火。反將禿小子脖子摟過來咬了兩口\footnote{寫出急態。}。再三托他。附耳唱了一句。道。你叫他明朝千萬早些來\footnote{寫一部書中淫婦之醜。未有如寶兒之不堪者也。}。此時心中難忍。又叫禿小子強掙挫着餞了餞別。次夜。寶兒叫待月暗暗將馬兒騾接了進來。寶兒又試新物。那馬兒騾想念主母久了。呷了一飽老燒酒。仗着酒興。爬上肚子。便奮勇前驅。竟三戰三捷。弄得寶兒心滿意足。方信禿小子果然言如其實。自嘗了這可心可口的妙物。越發夜夜不肯放空。此後他父母想他。差人來接。他也不肯回去。接過兩次己(幾)番。只得去走走。決不肯過夜。就是阮優苦留。他也不肯。嬌嬌阮優私議。只說他心懷舊恨。或是女婿同他弄得好得很了。不稀罕舊物。那知他是有了可意新奴。馬兒騾同寶兒夜夜風騷。過了些時。神疲力倦。懨懨欲斃。把一個千里駚(駃)騠弄成了駑駘賤騎了。連鞭笞都不能動。只得又轉薦他僕。求饒草命。寶兒還恐他是躱避差使。不肯寬假。那一夜他爬上身。纔抽了幾下。叫腰酸腿疼。跌下肚子來。寶兒還疑他裝假。叫他仰臥。跨到他身上。做個倒騎驢勢子。那馬兒騾的陽具先還有些硬氣。被寶兒蹲了幾下。縮軟如綿。知他實在不能了。只得允他保舉替代。他經過馬兒騾又大又久的物事。別人弄的總不像意。心中想到物小以多爲勝。況他旣破了臉。偷過三姓家奴。還知甚麼叫做羞。遂叫這個家奴將家中精壯小夥子。每夜約三四個進來。不管長大短小。他仰睡着。只叫輪流上身。一夜弄到天明方罷。家中二三十個下人。除了幾個年老的不要。別的都叫來嘗過。有的弄受用了。都有賞賜。激勵衆人。那些不濟的。旣要博主母歡心。又希圖重賞。也都下死力捨命去弄。年餘光景。這寶兒竟成了色癆。遍身虛火炎燒。越發要弄。陰中一時空了。便熱癢難過。這是下體受了陽毒的過失。尋了幾個舂蒜的石杵。用涼水浸得冰冷。輪替放在陰中纔過得。一到晚。就四五個家人輪流到曉。日漸羸瘦。又過了幾月。日間飮食俱廢。每夜還不肯放鬆。不幾時。竟乾枯而死。他嫁了首尾不足二年。如此終於內寢。這勞正只在書房養病起臥。一時虛火動了。有兩個心愛的小子取樂一番。這寶兒是他棄了置之於肚外的。也決想不到他無恥到這個地位。同家奴淫亂。這些下人淫了主母。都是不赦的死罪。互相隱瞞。四個丫頭又同在渾水裡。皆被衆人弄過。所以二年來瞞得風聲不漏。竟未曾傳出醜名。那勞正見他死了。心中暗喜。將他殯葬之後。見了他陪嫁的丫頭。就想起寶兒這樣個齊整女子。却是破瓜。心中就惱。盡行遣嫁。暗暗囑托媒人。他要續絃。不拘門第。只要標致。眞正處子就娶。此乃後事。且說待月嫁了人家。他丈夫雖是個小買賣人。倒有三分骨氣。那待月偶然一晚多飮了幾杯。又同丈夫高興了一度。因說起當年閒話。俗語說。兔兒是狗趕出來的。話兒是酒趕出來的。不因不由。把他家姑娘在家做女兒並嫁後的美事以爲笑談。詳詳細細向丈夫說了。雖然他不肯說出自己做牽頭通同作弊。那男子可有不想到姑娘如此。其婢可知。自然也是個淫物了。想想這頂綠頭巾不是好戴的。暗暗把他賣下水去了。有親友見他。責他負心。問其原故。他實言所以。三人口濶一尺。故此阮寶姑這些美處。互相傳爲笑談。沸揚通國皆知。阮大鋮一家也都有些風聲吹入耳中。只好推聾裝啞。阮大鋮做了一生壞人。子烝其妻。兄淫其妹。女私其僕。媳寵其奴。也就是天公暗暗的報應他了。尚不止此。因他害了多少忠良。作惡太甚。後來還有惡報。人生何不學好。那待月替姑娘做了牽頭。又洩了他的穢行。墮了淫孽。被丈夫賣入煙花。使他。

\begin{quotation}

生爲萬人妻。死做無夫鬼。

\end{quotation}

也就夠酬其罪了。豈有那些淫毒的惡奴反倒漏網。聞得那一年\footnote{聞得二字妙。不然誰見耶。}。勞家看園的禿小廝害了瘟病。嘴中胡說亂道。說主母領了許多惡鬼來打他。要拿他陰司去對理。就是馬兒騾衆人。也都要拿去。日夜求饒喊叫。不數日而亡。但是奸過寶兒的家人。疑心生暗鬼起來。心中都有些害怕。不上一月。盡皆傳染而死。其餘不曾同他弄過的下人。並皆無恙。這也甚奇。

\begin{quotation}

人間私語。天聞若雷。暗室虧心。神目如電。

\end{quotation}

這樣看起來。天道可不畏哉。且按不題。一日。五月中旬。阮大鋮被朋友約去遊榴園。那時天長無事。嬌嬌正睡午覺。阮最悄步進來。到了房中。見他放着帳子濃睡。向賽紅道。你看着門。不要放人進來。把門掩上。揭帳一看。見嬌嬌光着上身兩枚酥乳。下穿紅紗小衣。不曾繫帶。上床自己脫光。就替嬌嬌脫褲子。他醒來道。我纔睡得甜甜的。你就來混我。阮最笑道。沒良心的。這個混法。得每日有人同你混混也罷了。嬌嬌笑着任他脫去。兩人就弄起來。這阮優也知父親不在家。偷空來與嬌嬌高興一番。張得院中無人。一溜煙到他房中來。就要推門。賽紅忙攔住。道。二相公不要進去。姨娘屋裡有事呢。阮優也同這丫頭弄過多次。摟過脖子親了個嘴。笑道。癡丫頭。你姨娘慿着做甚麼事。那一日又躱起我來了\footnote{是相厚的口聲。}。推門徑入。那丫頭又不敢拉他。阮優見帳子放着。只說嬌嬌睡覺。順手一掀。不想哥哥同他兩個精光的弄呢。阮優素常性子極壞。一見了。也不像弄他老子的妾。竟像弄他的妻子一般。急得火星亂冒。道。做得好事。做得好事。向哥哥道。你可成個人。幹這樣的事。那阮最正同嬌嬌弄得將入佳境。不防兄弟揭開帳子。倒覺羞愧難當。又不好拔出來。只好伏下身子。把臉朝着床裡。忽聽得兄弟說了這幾句話。也就忍不住仰起頭來。說道。我倒不管你。你倒要管起我來。我做這樣事。你難道是沒有做的。阮優道。是你先做。我後學你的樣子。就講到老爺跟前。罪也有個先後輕重。阮最大怒道。你說我在先。我且問你。你同寶姑通奸。我也有來麼。我忍着不曾說破。就夠你的了。你倒還這等放肆。阮優道。你看見來麼。你那時爲甚麼不早說。如今寶妹子也死了。沒得對證。你此時現在他肚子上呢。那阮最愈怒道。這麼說。你明明來拿我的囮頭。我同你拚了罷。阮優道。你不要唬我。這個我倒不怕。你拿大奶頭唬我小孩子呢。那阮最也忍不過了。也顧不得羞。一下抽出那話。起身就要往床下跳。嬌嬌先被他壓在肚子上動不得。只好用嘴勸。此時見阮最起來要下床。恐怕二人弄出禍來。連忙爬起。抱着阮最。道。你們一個親弟兄。爲甚麼這樣的\footnote{正要問你。}。這一鬧起來還了得麼。我們三個都是要死的。大家忍一句就完了。阮最被他這一句提醒了。心中想道。是呀。這一鬧得老爺知道。可還饒得過我們。心中只一怕死。那氣就息了幾分。說道。我每常待他極好。你是知道的。你同我相厚是多少年。後來聽見說他也把你囮上了。我說他小人兒家。憑他去罷。不同他一般見識\footnote{己先不正。同他一般見識。也無其奈何。}。他今日倒管起我來。要是好的。知道我在這裡。就不該進來。難道沒有日子了。就安心來同我爭鋒相鬧。阮優道。我倒是無心撞來的。並不知你在這裡。你拿話壓我。說我不該來。難道只許你弄。我就弄不得。那阮最纔要開口。被嬌嬌把他一捏。道\footnote{寫得入神。}。大哥。你大幾歲年紀。兄弟小。就讓他句兒罷了。等我勸二哥。那阮最拉過褲子蓋着下身。就不做聲。嬌嬌一手拉阮優坐在床沿上。拿嘴對着他耳朶道。二哥。你怎這麼個性子。一個哥哥也該讓他些\footnote{這話出之他人便是良言。出自淫婦口中便成穢語。}。鬧起來。有甚麼好處。你一個伶俐人。還用我說。我說句話。你不要惱。笑嘻嘻一手摟着他脖子。道。我若當初先有了你。後來他又揷上來。你該惱。他先有了我十多年。你是後來的。如何爭得他。況且都還爭不着呢。我要是你跟前的。被他占了。那你自然應當發怒。我是你老子的人。你也有得。他也有得。又親了個嘴。道。我說的可是麼\footnote{淫婦生花之舌。兩邊都周旋到。說得固然有理。但虧他說得出口耳。}。那阮優先也是一沖性子。此時也就回了些。想着果然鬧將起來。大家不好。但他不肯服軟。聽了嬌嬌這話。又硬一句道。你旣這樣說。放了手我去。讓你們受用。只要受用得長久就好。就要起身\footnote{就要二字妙絕。}。嬌嬌那裡肯放。這隻手摟得更緊。那隻手抱着他。道。二哥。我難道就沒一點情兒到你。我勸你。你就不依。這麼樣強。那阮最道。不消不消。我去罷。讓你兩個如何。就穿褲子。嬌嬌要拉這個又怕那個走了。拉那個又怕這個走了。一則是怕鬧得阮大鋮知道不好。二則怕他二人這今日一變了臉。彼此拿囮。就不能來往了。急出個苦肉計來。道。你兄弟兩個我勸着都不依。何苦爲我一個人叫你弟兄成仇。不如我死了。慿你們去罷。遂回手拿了一根褲帶下床來。鼻涕眼淚的\footnote{何處得此一副急淚。}。就往欄杆上拴\footnote{將來的先兆。}。那阮最阮優見他雪白的個身子。臍下一條細縫。兩個圓圓的奶頭。好不動火。又見他哭得三行鼻涕兩行眼淚。心中又憐。見他拴帶子要上吊。忙上前。一個人拉着他一隻膀子。道。姨娘。快不要這樣。我弟兄不惱了。嬌嬌道。你兩個旣捨不得叫我死。過後你們又鬧起來呢。二人齊道。我們要再鬧。都不逢好死。嬌嬌道。旣然不惱。兩個都不去。阮最道。不叫我們去。儘着坐着做甚麼。嬌嬌揩了眼淚。復上床來。向阮最道。大哥。你還上來。此時阮最只穿着褲子。尚未穿衫。嬌嬌道。你還脫了。笑嘻嘻向阮優道\footnote{倏啼倏笑。寫出個活跳的淫婦來。}。二哥。你也上床來。脫了衣服。阮二道。怎麼的。嬌嬌笑道。爲我叫你兄弟兩個生氣。說不得苦我身子不着。替你弟兄和和事。把你兩個。一個在前。一個在後。只要你弟兄和好了。我受些苦也說不得。又笑向他二人道。我這屁股正經。我跟你爹一場。他要弄過幾百回。我決不肯依他。今日便宜你兩個短命的受用。說着。一手去扯阮大褲子。那阮最就勢脫了。又轉身拉阮二上來。替他寬衣解帶。因天熱。他也只穿着一衫一褲。阮二也不等他動手。忙忙的自己脫光了。嬌嬌一看兩人的陽物都直豎在那裡。便道。你兩個誰在前。誰在後。他二人同聲道。憑姨娘吩咐。嬌嬌說道。不好。我要說。又像我有偏心一般。遂將頭上的茉莉花拔下來。攄去了花。把那棒兒分做兩半\footnote{奇想。}。擫做一長一短。攥在手中。道。你兩個齊抽。抽着長的先弄前頭。短的弄後頭。弄過一會再換轉過來。大家都嘗滋味。就不偏了。他二人笑着伸手來抽。却是阮優抽着了長的。嬌嬌叫他仰睡着。把夾被疊了幾層墊在他股下。然後跨上身來。對準套入。蹲了幾蹲。已沒盡根。嬌嬌把身子伏在阮優肚子上。屁股蹶着。回顧阮最道。你也來罷。阮最此時魂飛骨醉。忙用唾津潤了。雙手捧着他雪白的屁股。款款送入。嬌嬌顫着聲兒道。好脹得慌。阮最忙往外縮縮。就不敢動。嬌嬌道。這住着不是事。說不得我忍着些。你上下一齊動罷。那阮優往上一攧。嬌嬌的屁股往下一坐。阮最向下一聳。嬌嬌把屁股向上一迎。二人樂不可言。嬌嬌的快活更不消說得\footnote{此一段方把他父子聚麀說明。將他弟兄兩人名字方寫得暢快。}。弄了有一頓飯時。嬌嬌吁吁氣喘。香汗浸肌。顫着聲兒說道。且歇一歇着。我一點力氣星兒也沒有了。你兩個換換罷。阮最下來。嬌嬌也爬下來。睡在枕頭上喘息養力。阮優也起來。嬌嬌道。熱得很。把帳子掛起來透透氣。料道沒有人來。他二人忙把帳子掛起\footnote{不知者以爲他眞怕熱。知者不掛帳子賽紅何由得見也。}。嬌嬌歇了一會。道。再來該大哥在底下了。阮最忙睡倒。嬌嬌便上身套進。此時他前後都已潤濕。一坐到根。阮優也挺陽物一送。直進後庭深處。嬌嬌道。你兩個一齊用力。弄丢了。大家歇歇吧。恐怕你爹來家。二人聽說。一齊奮力。弄夠多時。那嬌嬌的身子也被他二人弄得動不得了。嘴中哼聲不絕。任他二人翻騰了一會。阮最忍不住也洩了不動。那阮二初嘗這件美味。況他精力原強。加勁直搗。嬌嬌被他弄得後庭中酸麻樂極。四肢都軟了。渾身癱在阮最身上。阮優也伏在嬌嬌背上大弄。阮最叫道。我禁不得你兩個人壓着。快放我起來。讓你們弄。阮優聽說。把嬌嬌的兩胯扳住。用力抵到了根。身子往後一仰。把嬌嬌的腰抱了起來。嬌嬌兩手也用力一拄。胸脯懸空。那阮最得鬆。掙了過去。嬌嬌道。心肝。你把枕頭與我墊着肚子。阮最忙拿過來替他墊好。他伏在上面。屁股高蹶。阮優興不可遏。自首至尾出沒數百。方纔洩了。又往內狠狠送了幾下。然後拔出。那嬌嬌好生受用。有幾句說話贈他三個道。

\begin{quotation}

前後夾攻。腹背受敵。上邊的向下一聳。後庭內已自酥麻。前面的往上一迎。牝戶中更覺爽利。二筋鎗攮得一庶母。魄散魂飛。兩肉孔夾得二賢郞。汗流精洩。

\end{quotation}

那嬌嬌透了一會氣。笑道。你兩個和好了。我也被你兩個天殺的弄癱了。今日當面說定你兩個後來是怎麼個來法。省得爭講。阮二道。憑哥哥的意思。我再沒有不遵的\footnote{眞悌弟。}。阮最道。據你的意思怎麼說\footnote{眞友兄。}。阮優道。我兩個分了罷。哥哥要前頭。把後頭讓我。或哥哥要後頭。把前頭讓我。嬌嬌笑道。短命的。這不過是我替你們取和的意思。難道這是常弄得的麼。阮最道。二哥。你這主意不好。弄前弄後。兩不照面。誰得知道。又是爭端。還是慿姨娘主意。嬌嬌道。要我的主意。你兩個輪班。遇有空。大哥先來。再有空。二哥再來。如此輪着可好麼。這可沒得爭的了\footnote{大公無私。眞好慈母。}。阮優道。這主意越發不好。老爹時常在家。間或有空。哥哥來了。或半日半月沒空。我怎麼等得。嬌嬌笑道。我不管。憑你弟兄兩個商議去。阮最道。我想了個大公的妙法。蒙姨娘這樣好情在我們身上。我們再有一點爭論就不是人了。今日大家說定。我們兩個或有一個往那裡遠去不在家。若那一個不許來。難道忍叫姨娘孤孤悽悽的等着\footnote{眞孝子。}。那就只管來陪他。不必論次數了。若我兩個都在家。要來便一齊來。那纔沒有厚薄。阮優道。妙呀。哥哥說得是極。就是這樣行。嬌嬌笑道。寃家。你兩個一齊來也罷了。不難爲了我些\footnote{喜極而做還(違)心語。}。阮二笑道。姨娘。拿出良心來。這苦你也還樂得呢\footnote{知心之言。}。嬌嬌笑罵道。怪短命。我給你弄了。還說這樣燥皮的話。他兄弟二人穿了衫褲。笑向嬌嬌道。姨娘。多擾你的肝板腸同扁食了。我們去了。兩個笑嘻嘻拉着手開門出來。那賽紅坐在門檻上望着他兩個。忍不住格格的笑。阮最道。這瘋丫頭。你笑甚麼。賽紅道。我瘋麼。我看你們三個方纔舞獅子壓灰堆。纔像瘋了的呢。那阮優把他腮上擰了一下。笑着去了。原來他弟兄兩個拌嘴並三人後來和事。這丫頭先在門口聽了個滿耳。後又在窗洞中看見這一副新款嬲字春宮。\endnotemark[16]故此忍不住笑。阮最阮二這一場公弄。他弟兄二人此後果然和好非常。阮最打聽得有空。就去約兄弟。阮二看得他老子出門。便去約哥哥。再不肯瞞着獨往。一日。他弟兄同在嬌嬌床上。嬌嬌仰臥。叫阮最上身先弄。叫阮優等着再上。阮最道。你回回儘着叫我在你肚子上也絮煩了。今日你上我身來。也新鮮些。嬌嬌就爬在他身上。兩手拄定屁股。用力一起一落。阮優看得十分興動。爬上他脊背。道。姨娘。我忍不得了。把後頭與我弄弄罷。說着。就往糞門內頂。嬌嬌忙道。哎呀。行不得。我瀉肚呢。阮優連忙抽出。道。這怎麼處。想了想。笑道。有了。放着東西不會吃。看準他陰戶。就往裡一塞。進去了半截。嬌嬌笑罵道。短命鬼。你穿破了我的呢。他又狠狠的往裡送了送。那阮最往上頂。阮優向下搗。他兄弟兩個覺緊箍箍的有趣。那嬌嬌也被他兩個塞得內中滿滿。更無一毫罅〖阝少日小〗。樂不容言。兩條陽物在內中彼此相擦。又被陰戶箍緊。不多時。二人就洩了。阮優下來。嬌嬌也睡下。將帕揩了。用手一摸。笑道。你這兩個促恰癆。把我無樣不弄到。你看弄成這麼個大洞。你爹要試出來。看怎麼答應。阮大笑道。你夾緊着些。就試不出了。嬌嬌笑道。這也是夾得緊的麼。阮優道。我有個妙法傳你。你用手從後邊捏着一半。那一半就緊了。說得三人大笑了一陣。大家散去。又一日。他三人又在一處。阮優向嬌嬌道。我想了個新樣子。是二十四解裡雖(頭)沒有的。我們試試看。且脫了衣服着。嬌嬌是騷淫極了的。聽得好不歡喜。忙脫光了。阮最阮優也脫盡。阮優叫嬌嬌把脊背合着他的脊背。他反過手來摟着嬌嬌肚子背將起來。叫道。哥哥。你把他兩條腿夾在肋下。弄上了。你往前推。我背着走着。可好。阮最就把嬌嬌的腿夾住。弄將進去。向前推。阮優背着。在房中團團的走。把個嬌嬌笑得了不得。弄了一會。又換阮最背着。阮優弄。他三人這個弄法。無樣不想出來。不能細說。嬌嬌從得他兩個齊來。惟憑自己高興。或叫他弟兄輪流弄陰戶。或是一個弄前。一個弄後。或是兩個同門。日裡興已飽足了。夜間又有阮大成(鋮)補空。他却也得了個快足。但恐興盡悲來。冥冥中未必肯久留此輩淫汚世界。後來自有分曉。且把那宦賈童三人如何相會。如何結盟。聽我下回細說。



\endnotetext[1]{此句原書右有夾批「年歲小,此事老在行」七字。}

\endnotetext[2]{以下原書(第九葉)殘破,缺文字數以半葉八行、每行二十四字加以估算。}

\endnotetext[3]{此段原有眉批「必是成一吊」五字。}

\endnotetext[4]{「漣」原作「璉」,據《明史》卷二四四《楊漣傳》及下文改。}

\endnotetext[5]{「謬」原作「璆」,據《明史》卷二四四《楊漣傳》改。}

\endnotetext[6]{「信」原作「佞」,據《明史》卷二四四《楊漣傳》改。}

\endnotetext[7]{「正」原作「政」,據《明史》卷二四四《楊漣傳》及卷二四一《鍾羽正傳》改。}

\endnotetext[8]{「王」原作「伍」,據《明史》卷二四四《楊漣傳》改。}

\endnotetext[9]{「傳旨詰責」原作「傳語責」三字,據《明史》卷二四四《楊漣傳》加改。}

\endnotetext[10]{「改」原作「故」,據《明史》卷二四四《楊漣傳》改。}

\endnotetext[11]{「慢」原作「漫」,據《明史》卷二四四《楊漣傳》改。}

\endnotetext[12]{「賢」下原衍一「耶」字,據《明史》卷二四四《楊漣傳》刪。}

\endnotetext[13]{以上正文原作「阮最想道今番好事就在此一瞅了他一眼那嬌嬌也不回言微微笑着斜刻了」三十一字,據文義改。}

\endnotetext[14]{以下原書中空八格,書「連上寫」三字。}

\endnotetext[15]{此段原有眉批「寫丫頭爲了字時留心」九字,文義不通。}

\endnotetext[16]{以下原書抄有批註「嬲字奇想,此囗」六字,後又塗去,中空近二行,書「接着寫,不要空了」七字。}

\setcounter{footnote}{0}

\theendnotes

\part*{姑妄言第九卷}
\addcontentsline{toc}{part}{姑妄言第九卷}
\markboth{姑妄言第九卷}{姑妄言第九卷}

鈍翁曰。寫賈文物咬文嚼字。滿嘴之乎者也。一片假斯文身分。不過供人一笑。其待鄔合也。富中帶酸。寫童自大呆財主的身分。尚不足爲妙。只看他廳上的一番擺設。俗氣衝人。眞是財主家款式。其待鄔合也。吝而臭。令人幾乎笑得腸斷。寫宦蕚自是驕奢公子狂妄的才分。別是一樣。三人迥不相合掌。

李太孰謂其不通。他竟是東方曼倩淳于髠黃幡綽一流人物。不然何以開口便是趣話。無一字一句不令人解頤。李太之延師干生。與之不相合者。干生之過。非李太之過也。何以言之。天下之東家多半有李太之習。干生若向游混公卜通二人求其爲先生五字之秘訣。決如膠投漆。必不至於冰炭矣。

百家姓直解爲千古第一講章。上大人一封書爲千古第一家信。宦賈童結拜千古第一盟文。不意此一回書內見此三絕。

鍾趨之棄婿。何損於干生。特自害其女耳。眞家訓之嫁女。何榮於干生。乃自成其女耳。二人之心胸眼界。孰優孰劣。孰幸孰不幸。擇婿者請擇其所從。

鍾生救郗氏。資助郗氏。拒李氏。成全李氏。一是鍾生今日得中之因。一是二氏異日報德之果。

鍾生得遇錢貴。梅生之力也。梅生之娶李氏。又鍾生之力也。可謂以德報德。

宦賈童結盟一段。作者非有二十分憤懣。二十分傷心。不能道也。何以見之。但看他三人口中所說的話。無非是富貴他人合。貧窮親戚離之意耳。

\chapter*{姑妄言卷之九\\
第九回 鄔合苦聯勢利友 宦蕚契結酒肉盟\\
附 李都督延師千秋佳話 鍾秀才救溺一片熱腸}
\addcontentsline{toc}{chapter}{第九回 鄔合苦聯勢利友 宦蕚契結酒肉盟}
\markboth{第九回 鄔合苦聯勢利友 宦蕚契結酒肉盟}{第九回 鄔合苦聯勢利友 宦蕚契結酒肉盟}

話說鄔合到了賈進士門首。只見門樓下正中掛着一個門燈。上面賈衙兩個大字。傍邊放着兩條大凳。坐着四個家人。是賈進士得用的管家。名喚賈勢。賈利。賈富。賈貴。鄔合平素都認得。走上前。帶着笑拱手道。久違久違。那四人見了。也起身拱手讓他同在凳上坐下。問道。鄔相公許久不來。今日到此。還是來求我家老爺的詩文。還是要求那衙門說事的名帖。鄔合道。都不是。有句要緊話要見老爺面講。相煩傳報。那賈勢叫管門的賈閽道\footnote{賈閽二字令人放聲一哭。閽者。門也。人生在世豈特勢利富貴爲假。雖此門亦假也。門旣假。此身非眞可知。釋經云。人生如夢幻泡影。如電復如露。人尚不悟此。猶營營於勢利富貴何哉。}。你去稟聲。說鄔相公要見老爺。鄔合接口道。相煩大哥。改日買茶酬勞\footnote{恰是江寧人聲口。}。那賈閽去了多一會。出來說道。老爺在廳上。請鄔相公進去。那鄔合別了四個大管家。隨着賈閽走到廳院中。遠遠望見賈文物在廳中間一張椅子上坐着。鄔合忙跑上前。深深一揖。道。驚動老爺大駕。有罪有罪。賈文物慢條斯理的走下來。把腰略彎了彎。還了半個揖\footnote{彎彎腰。半個揖。是個大走(老)官得篾片身分。}。讓他客位坐下。自己把座兒斜僉了相陪\footnote{斜僉了座兒相陪。是有錢的人妄自尊大的身分。}。把臉仰着道\footnote{仰着臉。是假書呆身分。這幾句活畫出一個假斯文來。}。久別鄔兄。今日何見顧之早也。毋得而有事諸。鄔合打一恭。道。無事不敢造次進謁。今者一來請老爺台安。二來因昨日在宦大老爺處。承他過愛留飮。因提起大名來。宦大老爺甚是渴慕。有個要奉屈結社之意。又不好驟然奉拜。故命晚生先來介紹。不知老爺尊意如何。賈文物道。嘗聞之矣。宦公子富有而驕。貧與賤。彼之所惡也。不有其勢利之不取也。不意竟與兄相識。可見人言之誤。所謂盡信書不如無書者同然耳。由是觀之。宦公子可謂富而好禮者也。又是見鄔兄相識滿天下。知心有一人矣。但所云結社之事。我學生科甲中人。若與公子交。如衣朝衣朝冠坐於塗炭。決乎其不可行者。結社也。兄可善爲我辭焉。如有復我者。予小子必避於箕山之陰矣。鄔合道。老爺尊見固是。但宦老爺一番殷殷美意。老爺不允。未免太覺恝然。且還有一說。老爺若與宦公交結。通家往來一深厚了。也頗有益處。他太老先生也是有名人焉。異日老爺到部榮選。或可稍得其助。老爺請上裁。賈文物聽了。撫掌揶揄道。有心哉。斯言乎。斯人也而有斯言。可謂善談也矣。我不亦樂乎。夫如是。我明早即趨造於府。決不瞰其亡也而往拜之\footnote{世人做了財主。未有不想做官者。賈文物不但財主。而且又是進士。官之一字。自然熱衷。鄔合即以此餌之。彼豈有不樂從者哉。做篾片者亦必有篾片之才始可動得大老。若蠢蠢然唯知舔瘡砥(䑛)痔。只能奉承三家村之豪耳。}。鄔合見他依允。滿心歡喜。即起身作別。賈文物拉住。道。我有酒食請先生饌。鄔合道。晚生怎敢叨擾。賈文物道。聖人云。君子食無求飽。未云不食也。兄以我之食爲不義之粟而弗食乎。鄔合道。晚生怎麼敢。特不敢當耳。賈文物道。我之粟雖非以械器易之者。乃小价輩播種而耰之。又得肥磽雨露之養。然後得倉廩實。皆勞力所致也。何傷乎。且坐小其吃也已。須臾。衆家人擡過桌子來。將肴饌堆了滿案。甚是豐盛。鄔合道。老爺爲何如此盛設。使晚生何以克當。賈文物道。食前方丈。我得志必爲也。食不厭精。鱠不厭細。我非鄕人也。豈可不效聖人之語乎。飯蔬食飮水。此陋巷中之所爲耳。噫。斗筲之人何足算也。此豈我素富貴行乎富貴之人所爲者耶。正食間。他回顧家人道。不撒薑。食小菜何不以薑爲之。不得其醬不食。肉何不以醬熩之。向鄔合道。此鵝非陳戴所畜之鶂。兄何爲不食。此肉非陽貨所饋之豚。兄又何爲不食。兄以此物出三日則不食之乎。未也。我學生雖遠庖廚。若謂小价有校人烹之妄。彼烏敢當欺我之名哉。然而無有乎爾。鄔合道。老爺也請用些。晚生方好動箸。賈文物道。何謂也哉。可以吃則吃。可以止則止。亦各從其志也已。魚我所欲也。故捨肉而取魚者也。兄但正席而先嘗之。鄔合聽了。大嚼大吃。多時食畢。又叫取了酒來。讓鄔合道。惟酒無量。不及亂耳。沽酒則不食。此非沽來者。請飮之。各飮了數杯。鄔合吿止。衆人撤了下去。他起身謝別。臨出門。說道。明日耑候老爺大駕。幸勿爽約。恐宦公加罪晚生。賈文物正色道。鄔是何言也\footnote{此句巧。}。予豈若是小丈夫然哉。民無信不立。前言定之耳。鄔合忙揖道。晚生得罪得罪。又作揖而別。有幾句贊這賈文物寫照道。

\begin{quotation}

形容雖秀。骨格庸愚。滿口詩書。掩不盡白木行踪。萬千做作。裝不出斯文腔調。一身中搖搖擺擺。全無坦坦之容。滿腹內腐腐酸酸。大有花花之態。

\end{quotation}

鄔合別了出來。一路奔到童自大門首。只見兩扇黑漆油的大籬笆門關着。貼着一張吏部候選州左堂的紅封皮\footnote{此等事果有之。勿以爲笑談。}。傍邊貼着兩張街道坊官禁止汚穢的吿條。上寫道。

\begin{quotation}

本廳司示諭。一應閒雜人等。勿得在此汚穢。如違拿究。

\end{quotation}

硃筆大圈\footnote{妙極。江南或監生或財主。十家有七八貼此。}。看了一會。竟不見一個看門的出入。只得推開門走了進去。到大廳上。見有許多人皆在廳內兩邊靠牆大凳上坐着。鄔合近前拱拱手。也隨衆坐下。看他藍粉貼金的屛門上貼着一張紅紙。捷報候選州左堂的報帖。中間懸着一軸紅綾金字的大畫。是夥計們賀他援納的賀軸\footnote{夥計們。妙。大約他除行財夥計之外。未曾相與他人也。}。後面許多名字\footnote{是財主家的堂畫。}。正中間放着一張大公座。擺着筆硯。拴着大紅潞紬桌圍。桌子上放着一架大天平。一個大算盤。傍邊放着一張方桌\footnote{笑倒。是個財主監生。以富翁而效官樣者。趣甚。}。堆着許多帳簿包裹。屛門兩邊放着兩架大揷屛。硃紅漆描金螭虎架子。一邊畫的是虎牢關三戰呂布。一邊畫的是九里山十面埋伏\footnote{這兩架揷屛。非財主家別處再用不得。}。正中放一張椐木金漆大几。几上放着一個紅綠花大磁瓶。黑退光漆座子。內中揷着一枝裁帛做的大牡丹花。還有幾根孔雀尾\footnote{好點綴。不愧是財主。}。廳東南角上放着一面大鎭堂鼓。西邊一頂屯絹圍子五岳朝天錫頂的大轎。一把大雨傘。兩對大〖缶曼〗(幔)燈。一邊是候選州左堂五字。一邊是童衙兩個大紅字\footnote{眞好鋪設。雖與前卷鄔合向宦蕚所說一字不移。他那是口說。這是眼中看見。故不覺其重出。}。中樑懸着一個大匾。紅地金字。題着世富堂\footnote{堂名妙絕。}。兩邊柱子上貼着硃砂箋的對子。一邊是。

\begin{quotation}

但願銀錢湧來。如長江大海。萬載無休。

\end{quotation}

那邊是。

\begin{quotation}

惟求米糧堆積。似峻嶺高山。千年永在\footnote{見此對。偶憶一笑談。有一老人性甚貪。一日於郊外閒步。見一大空地。盤算道。用多少牛力。用多少耕種。開多少田。一年收穫若干。久之。便可爲財主矣。傍有一人笑謂曰。還得數百斤鐵方妙。老人問曰。要鐵何用。其人曰。還鑄一個你。不死才好。此對萬載無休。千年永在。也須鐵鑄一個童自大方妙。}。

\end{quotation}

坐了有兩三頓飯時。只見走出一個家人來說道。等了這半日老爺纔醒了。叫你列位們且等着。衆人應了一聲。鄔合認得他叫童祿\footnote{是個財主家人的名字。銅錢生綠。非財主家焉得有。}。忙向他拱手。道。相煩稟一聲。我在此候老爺有話說。童祿去了一會出來。道。老爺知道了。鄔相公請坐。就來。鄔合只得又等。心都等焦了。將過午時。只見那童自大糟包着一個臉還醉醺醺的。兩隻眼半睜不睜\footnote{是個財翁形狀。}。靸着厚底紅鞋。扶着個蘇州淸秀小廝叫做美郞。慢慢的踱將出來。看那童自大時。

\begin{quotation}

身上一般華服。而呆氣衝人。面上的是財翁。却癡肥可笑。權裝官體。頭上戴一頂軟翅唐巾。假學斯文。脚下靸兩隻三鑲硃履。

\end{quotation}

鄔合見了他。忙上前作了揖。道。老爺好受用。此時還在夢鄕。童自大道。連日這些借銀子的人請我吃戲酒。每日熬夜。又吃得大醉。昨日偏你又多了幾杯。今日這時候還爬不動。若不是他夥計們來算賬交利錢。我正好要睡呢。讓了鄔合坐下。因問衆人道。你們都來齊了麼。衆人都站齊作了揖。答道。都久已到齊。伺候老爺算賬。他聽了。向鄔合道。你且請坐着。有話等我算完了賬再說。就到公座上高坐\footnote{令人笑倒。也不用排衙喊堂便登公座。倒也省事。}。叫衆人一個個將賬簿算起。算完。然後擡過天平來。將銀子兌明。兌畢了。衆人方纔辭去。足足弄了半日。又將賬目叫美郞記淸了。收入書房櫃子裡去。又親自送進銀子交與鐵氏。過了好一會。時已下午。他方出來坐下。纔向鄔合道。久不會你。你竟胖了好些。想是在那個大老官跟前弄得了幾個錢了\footnote{看他開口便是錢。才是眞財主。}。鄔合道。向來只在宦大老爺那邊。承他照拂。並未曾到別處去。童自大道。我每常聽得人說他家銀子多得很呢\footnote{頭一句是錢。第二句便是銀子。非財主決無此等寒溫。}。你旣常在他家走動。看他比我何如。鄔合道。他家雖富到極處。大約也與府上不相上下。童自大嘆了一口氣。道。我只說京城裡算我是個頂瓜瓜的財主了。誰知又有他家。我從今後。拚着幾年不吃飯。定要把銀子積得比他家多些。做了第一個財主。方纔遂我心願\footnote{七日不食則餓死矣。幾年不吃飯已成枯骨。還用那財主之名何用。較那得做半日神仙死了也快活者更愚。}。說話間。那童祿走來說道。請老爺用飯。童自大道。有客在這裡。且慢些\footnote{看他第一次是如此請。如此答。}。那童祿去了。鄔合道。晚生昨日在宦大老爺處。他說要結交幾個朋友。俱要出色的人物。晚生因題起大名來。宦大老爺甚是歡喜。故命晚生來奉問老爺可有此高興麼。童自大把嘴一努。道。唔\footnote{描寫入神。}。他們一個做公子的。老子做着官。銀錢來得容易\footnote{此語却不呆。}。我雖然是個財主老爺。都是牙齒上刮下來的。心血上掙下來的\footnote{老爺是牙齒上刮下來。心血上掙了來。奇聞。}。怎肯拚他。鄔合道。雖如此說。宦公子在今日也是叫第一家有勢利的呢。老爺與他做朋友也不得錯。就是費了幾個錢。等相交厚了。尋件把人情煩他那衙門說說。怕那個官府敢不依他。那時連本利都有了。正說時。只見先那童祿又出來。在耳朶底下道。裡面奶奶罵呢。說放着飯不吃。少刻冷了又要費錢炒。童自大道。你對奶奶說。有人在這裡說話。不然我先就進去吃了。就冷了也不妨。天氣正暖。叫留些熱茶。我停會泡了吃罷\footnote{二次請是如此答。}。童祿又去了。他因對鄔合道。我去年做了一件倒運的事。到如今還悔恨。但提起來。我渾身的肉都噶達達亂顫。牙根咬得格支支的響。鄔合道。是甚麼大事。老爺就氣到這等樣的田地。童自大道。我也因一時這兩隻牢耳朶軟。聽了人的話。說納甚麼他娘大屄的監生\footnote{監生二字之上。從未見此奇稱。閱此。因記一舊事。有數人閒話。偶及拔納一事。一人曰。世間納監之流。他前生係拖欠錢糧之頭戶。今生以納監爲名。特來補正身。不然。天下之監生不下數萬。有幾人得叨一命之榮者。彼豈不知而向爲此耶。一曰。不然。他非圖做官。不過借此名色抵擋門戶耳。但此輩目不識丁者多。濫廁衣冠。殊褻大禮。還該考一考。稍有文墨者方可以准入太學。似乎得體。又一人笑曰。他原怕如此。却才如此。若還如此。他何苦如此。附此以作一笑。}。戴頂紗帽。威勢好看。老來畫影。穿着大紅圓領又官冕。他說到這裡。嘆了口氣。把牙咬了一咬。道。哏\footnote{形容得有趣。}。悔不聽奶奶的話。說了這一句。靠在椅背上。道。哎喲。我肚子都氣脹了。鄔合道。奶奶說甚麼來。他又嘆了一聲。道。我奶奶倒說得好。他說我。你癩蝦蟆跳在三絃上。好個綳綳綳兒。你不要鑽在陰溝洞裡想天鵝肉吃了。勸你多吃幾個荸薺。把妄想心打掉罷。就沒有鏡子。你自己撒脬尿照照。你那個賊樣。你也想做官。不如安分守己的好。我雖然不敢做聲。我還暗恨他貶別得我這樣刻毒。連半個紙錢也不値。我竟趁着高興。又是賭那口氣。就去做了。以爲做了監生回來。便是朝廷家的大官了。就可以發財\footnote{想頭奇甚。做了監生便是大官已奇。而且就可以發財更奇。}。要我收了許多家人。做了一頂大轎。指着那轎子。道。這不是麼\footnote{畫也畫不出。}。我的勞骨屍又沈\footnote{所以有福。}。因轎大了。出動門定要三四個轎夫纔肯擡出城。略遠些定要六個人輪班纔肯去。多費了多少瞎錢。你不見我如今出門只是走麼。除非人家有轎馬的封兒。我纔坐了轎去。那時趁着一時倒運的興\footnote{倒運的興。也是奇聞乍見。}。請官府。拜當道。白花了幾百兩。把舌頭一伸。道。你當少麼。白晃晃的好幾大包呢。誰知一毫利益也沒有。雖弄了張國子監的敕書\footnote{奇談。}。供在家堂上。又吃不得。又穿不得。揩屁股又有字。糊窗戶又花裡胡哨的。我聽得人說。那東西看了消災。你長了這樣大。可曾看見過。我取出來你看看\footnote{看了消災。想頭眞愈出愈奇。}。鄔合忍住笑。說道。不消罷。那是老爺鎭家之寶。恐汚損了了不得。童自大連連點頭。道。也是也是。又道。人因我是監生。又有幾個錢。都假意奉承我\footnote{此句話却甚乖。}。雖然是當面叫聲老爺\footnote{要知叫聲老爺還是看家兄的體面。並非有監生的體面。}。背地還是老童童臭的叫\footnote{這倒不足責。河南歸德府雖位至宰相尚書者。人在背地直呼其名。風俗之惡薄至此。}。究竟往人家去弔紙。我也體體面面的。還只打兩下鼓。吃戲酒戲子還不來參場。只不過晚上那裡赴席回來。打個候選州左堂的體面燈籠。初一十五家堂燒香。穿穿鷺鷥補服。淸明十月朝上墳去。戴頂紗帽嚇嚇鄕下人\footnote{穿這補服。戴這紗帽。只好嚇鬼。還未必嚇得動鄕人。}。上秋到莊子上收租。擡頂大四轎。門上貼個大紅封皮。除此以外再沒有燥皮處。在衙官求個分上。還千難萬難的不依。他把脚跌了兩跌。發恨了一聲。道\footnote{這一段紙上寫出一個活童自大來。}。把我整整氣了這兩年。如今把些家人都攆到莊子上種地去了。也不相與人了。一日該用十個。省下五個。要補起這些數來纔罷。搖着手道\footnote{描寫呆態。妙至於此。}。如今我乖了。不上你的當\footnote{恐未必然。}。我現鐘不打反去煉銅。還想甚麼說人情翻本呢。是人說的。不願柴開。只求斧脫。把鄔合笑道。大老爺也說得是。但宦公子家中銀子現堆在家中無數。他做公子的人又肯撒漫。若相與下來。問他借幾萬銀子。老爺拿來生利錢用。不過後來還他本錢。他難道好問老爺要利不成。這豈不便宜。童自大站起來。滿地跳了幾跳。復坐下。用手向空連圈。道\footnote{起先跌着脚蛟(咬)着牙恨。此時亂跳。用手連向空圈。寫出喜極的樣。眞活潑。}。妙哉乎也。妙哉乎也\footnote{還不曾會着賈文物。便過了文氣來了。}。你說了半日的話。就是這一句妙絕。古今通道那沒道理的地位。說得我連心眼兒裡都覺得快活。正誇獎着。見那童祿一路喃嘟出來。道。兩次三番請吃飯不肯去。帶累我捱罵。不知那裡這些沒要緊的話講。到童自大傍邊。扯他的衣襟。道。茶都冷了。請吃飯去吧。奶奶說有話且吃了來再講。不要討沒趣。快去罷。又附在耳上道\footnote{扯主人衣襟。附在耳上說話。畫出個不知規矩的蠢僕來。}。奶奶還罵呢。說嚼蛆嚼舌根。有話留兩句。臨死打發勾使鬼。如今是那裡有這些說的。童自大正說得高興。旣丢不下。又陪人坐着。怎好進去獨吃。只得說道\footnote{只得二字。見他着實爲難。}。你去回奶奶。說我有個朋友鄔相公在這裡說要緊的話呢。我怎好撇了。自己進去吃的。你進去把飯拿出來。我同鄔相公吃罷\footnote{三次是如此請。說得快活歇不得。才叫拿飯來吃。寫銅臭人刻骨。請吃飯作三段寫。妙極。}。鄔相公是自家人。便飯就好。不必費事\footnote{不意中饋有人意(竟)還費事。}。你照着我說。不要說錯了。惹奶奶生氣。童祿應諾而去。童自大道。你雖然說得好。不知他端的可肯借銀子給我。鄔合道。古語說。小本不去。大利不來。老爺也要破費幾文。與他相與得情孚意合。做呆公子的人慣好小搊。況又見府上家私富厚。豈有借不動之理。老爺雖然用去幾個。到後來生起利錢來。自有多的。豈止一本十利。童自大聽得快活起來。只是點頭\footnote{先說我不上你的當。此時却上當了。}。嘻嘻的笑個不住\footnote{鄔合之說辭甚妙。此是因人而施。說賈文物也以功名。說童自大也以財利。正觸二人之慾。改(故)此樂從。方我(符)淸(其)苦聯勢利友題面。}。只見那童祿拿方盤托了兩碗菜。兩個小菜碟。擺下說道。只留了老爺一個人的飯。沒有多的。將就拿茶泡泡。同鄔相公勻着吃罷。鄔合看時。一碗中是四五塊臭醃魚鋪在碗底上。一碗中是一塊冷豆腐。面上放着一撮鹽。一碟是數十粒炒鹽豆\footnote{徽人上品與餐只用數粒。今他家竟用數十粒。可謂大費盛設。}。一碟是十數根醃韮菜。童自大道。這白豆腐只好自用。如何待客。向童祿道。你拿一個錢。到香蠟鋪中買些香油來拌拌。千萬饒兩張草紙幾根燈草來。不要便宜了他。你到當鋪裡要個錢去買。不要上去要。好惹奶奶說破費。那童祿就拿着那盛豆腐的碗走。童自大道。客在這裡就拿着碗跑。成個甚麼規矩。拿個別的傢伙買了來。童祿道。拿個傢伙去買。倒沾掉了一半。還當是我落了半個錢去的樣子呢\footnote{眞奇。半個錢不知如何落法。落去半個錢又何所用。可謂主僕相得。}。放在這裡頭還見眼些。童自大連連點頭。道。好好\footnote{諺云。養兒不要屙金溺銀。只要見景生情。童祿能體貼主人心腹。眞可謂幹僕。}。倒也是當家心。童祿去了。童自大對鄔合道。兄每日在宦公子處。自然吃的是大酒大肉。我每日家常吃飯只是一品鹽豆。隔得三五日買塊豆腐拌拌。今因兄在此。奶奶替我做人。不但有豆腐。又且有醃魚。這魚是他留着自己受用的。我每常摸還不敢摸他的呢\footnote{明是不要他吃。妙極。}。鄔合道。賢慧的奶奶。支人待客眞是難得。古人食不兼味。豆腐一味就儘夠了。何必要魚。老爺這就算太過費了。過日子的人家當省儉爲妙。童自大道。兄可謂知心之言。然而待客不可不豐\footnote{昔有一人請客。並無一物。只自己亂舞亂叫。客驚問其故。答云。待客不可不瘋。童自大若效此。豈不省了魚腐二品。}。說話間。童祿買了油來。拌了豆腐。每人吃了一碗多些茶泡飯\footnote{一個人的飯勻做兩人吃。每人不過一碗多些。細極。}。那幾塊魚鄔合也沒敢動他的\footnote{知局。不愧是老篾片。}。他也不讓。吃畢。吩咐童祿道。剩的豆腐賞你吃了罷。把這碗魚同這兩張紙燈草送與奶奶去。魚是有塊數的。要交明白了。那童祿骨都着嘴。鼻子孔裡笑着收了去了。鄔合道。明日早間老爺可到宦老爺處一拜。晚生在彼拱候。立起身來。童自大道。我明日去是走還是坐轎。鄔合道。自然是坐轎纔成體統。童自大道。他家若沒有轎馬封兒。豈不白折了轎錢。鄔合道。適纔所說的這還無片時。老爺倒忘了。童自大道。我因算現的。故此忘了賒的那一宗了。千萬留神。凡事我要占些便宜纔使得。若同他們一樣行就做不來了。鄔合道。知道知道。纔要走。他一把拉着。道。我明日是吃了飯去是不吃飯去。鄔合道。他那裡自然有酒飯。家中不必用罷。遂別而去。此時天色已暮。想道。此時不能往宦府去了。況且家中無人\footnote{細。此時嬴氏尚未獲者也。}。今且回家。明日早些去罷。回家不題。却說那宦蕚。那日早間捱了兩棒槌。跑出來同鄔合飮了一日。晚間只得進去。被侯氏又罵了一場。不敢出一聲。睡了一夜。次早又躱了出來。等鄔合回信。午後還不見他來。仍叫宦鷹道。你可到老鄔家去看他可在家。叫了他來。宦鷹去了。一會來稟道。鄔相公家鎖着門。不知往那裡去了。宦蕚等至晚尚不見到。遂大怒道。這廝可惡。敢欺誑我。因吩咐家人道。明日老鄔若來。着實打一頓。攆了他去。再不許他上門。衆人答應了一聲。原來宦家這些鷹犬都是與鄔合相厚的。次日見他來了。因對他道。昨日老爺見你不來。惱得了不得。吩咐說等你來時。叫我們打你一頓。還要攆你呢。鄔合聽了。吃了一大驚。因連連作揖。道。煩諸兄想一妙計。爲弟挽回一二。容圖後報。內中一個叫宦計道。他呆公子狗頭性兒。過了一夜想已忘記了。我替你進去回一回看。走了進去。只見宦蕚正在不足堂上獨坐。你道何爲不足堂。他取王安石天道不足畏。人言不足恤。祖宗之法不足守的意思。故匾題此名。那宦蕚高高坐在上面。還有許多不足的模樣。宦計上前稟道。今早鄔相公來的。小的們因老爺昨日吩咐。着實打了他一頓。要攆他回去。他定死不肯。說恐老爺惱他就當不起。跪在門口要求寬恕。宦蕚笑道。打了就罷。又還惱他做甚麼。着他進來。那宦計出到門首。對鄔合道。恭喜。老爺請你呢。鄔合聽見。如鬼門關放赦一般。忙忙走到廳上。跪下道。晚生負不可赦之罪。竟蒙原宥。實出望外。特此叩謝。宦蕚叫人扶起他來。說道。我不過一時之高興耳。已不罪你。你可坐了。鄔合方敢坐下。宦蕚道。昨日因你不來。我故此動怒。今日你來了。我的怒都趕到東洋大海不知往那裡去了。還惱甚麼。你昨日往那裡去來。他打了個哈哈。笑了兩聲。道。難道你又有個老婆不見了去尋麼。我雖不惱你。也要罰你個失信。叫小廝取一盤糖食來。頃刻。一個家人拿了一銀盤天茄門冬橘餅靑梅之類。送到跟前。宦蕚笑向鄔合道。罰你吃。你道這是何故。原來宦蕚生平不吃這甜物。一嘗着便惡心嘔吐。他以爲人人皆然。鄔合知他有這毛病。假意哀求道。旣蒙大老爺寬恩饒恕了。這東西晚生如何吃得。宦蕚笑道。那顧你不得。定要你吃。鄔合大空心走了來。正有些肚餓。故做艱難之態。一面吃着。一面說道。晚生蒙罰。不敢不領。有茶求一碗。不然這甜味就膩死了。宦蕚吩咐倒了碗茶給他。鄔合就着吃了有一半。那東西甜得實在有些吃不得了。便說道。晚生實實的下不去了。求天恩饒了罷。又假做惡心。背過臉去嘔了幾聲。宦蕚大笑道。夠他受的了。饒了他罷。叫小廝們收了下去。然後問他道。你前日說往賈童兩家去。你昨日可曾去麼。鄔合道。奉老大爺鈞旨。晚生若不曾去。就該萬死了。昨日淸早小人剛要出門。前日蒙老爺天恩。對縣中說了。差了幾名捕快到晚生家下來問詳細。晚生同他們說了一會話。方纔去了。晚生隨就到賈老爺那邊的。因那求詩字的求文稿的絡繹不絕。等他打發完了。纔得說話\footnote{說謊者世不乏人。然而鄔合向宦蕚謊言者。不如此不足以聳動宦蕚。非比他人誠心以說謊爲事者。然他向宦蕚力贊賈童兩人。也是我(爲)完那苦聯二字餘意。}。晚生因說起大老爺有下交之意。他再三謙說不敢克當。是晚生說恭敬不如從命。不可負了大老爺禮賢下士之意。他纔肯了。說今日定來晉謁。又承他賜飯。那富麗是不消說。只那些精肴美饌都是生平不曾看見。眞是富貴才子呢。宦蕚嘖嘖贊道。好人家。因向鄔合道。你這一篇說我下交的話眞講得妙。雖戲上六國封相的那個蘇秦。還有他一個朋友姓張的。叫做張甚麼呢\footnote{六國封相唱得多。他常見。故記得是蘇秦。張儀的戲唱得少。故記不得名字。畫出愚呆公子形像。妙。}。他兩個也不能賽你。你可曾到那個童大財主家去呢\footnote{財主之上加一大字者。是素常聞他百萬之名耳。}。鄔合道。晚生別了賈老先生。就到童府的。他因終日在人家吃戲酒。熬夜醉了。那時還未曾起來。等了好大一會。他纔出來。他又要收利錢。不得說話。有許多夥計在傍伺候。一個衣架大的天平放在中間。兌了又兌。足足兌了不知幾千。都是十足的細絲。晚生看得好不動火。等他事完。衆人都去了。纔得閒說話。宦蕚點頭道。眞財主眞財主\footnote{連贊。妙。可見自以爲不及。}。鄔合又道。晚生說起大老爺這邊來。他也着實渴慕。也說今日定來拜的。他定要留晚生吃飯。決不肯放。將黑方散。恐老爺安歇了。因此不敢來驚動。故此今早來稟。晚生焉敢在老爺尊前失信。求開恩鑒察。宦蕚道。原來有這些緣故。方纔白白的寃屈。罰你吃了那些糖食。旣說明白。我一些惱意都沒有了。但我每常只說我算第一個無對的門第富翁了。誰知道又有老賈老童\footnote{雖是其心折。却難爲他竟還服善。}。鄔合道。他兩家不過富而已矣。怎及得大老爺富貴雙全。天下第一\footnote{先極誇賈童兩家之富者。一欲實己之前言。二欲宦蕚起敬起愛。其交方固。今二者已定矣。仍抑他兩家奉承宦蕚。眞好篾片的老手。}。宦蕚摸着肚子。大笑了一回。因吩咐家人道。我今日要待大賓。伺候兩席酒。要齊整些。作速預備。不可怠慢。正說着。只見家人跑進來。道。賈老爺來拜。遞上一個名帖。鄔合接過。念道\footnote{他恐宦蕚認不得數字也。}同學里年世通家眷小弟賈文物拜\footnote{千古第一奇絕拜帖。}幾個大字。鄔合忙忙放下。跑出大門外接着。道。宦大老爺在廳上拱候了久矣。賈文物方下轎踱將進來。到廳院門口。宦蕚迎了出來。拱讓進廳。揖罷坐下。宦蕚看他時。模樣頗還淸秀。雙眼有些微眊\footnote{近日假斯文皆裝眊眼。不知起自何時。豈古名士之目皆眊耶。}。身上穿得甚是華麗。脚上穿一雙硃履。拿着一把雕邊寫畫的金扇。扇上拴着一個眼鏡。跟着十數個齊整家奴。須臾捧上茶來。吃罷。賈文物道。久慕老兄臺宗族稱富焉。鄕黨稱貴焉。自有生民以來未有之佳公子也。昨聆鄔兄所云。老兄臺不恥下問。予小子何以克當。老兄臺已莫如爵。又齒德俱尊。可謂有達尊三矣。而猶殷殷愛士。雖吐哺握髮之周公。甘拜下風矣。我小弟非妄談。從來行不由徑。雖公事不至於顯者之室也。因鄔兄舉爾所知。聞老兄臺喜朋自遠方來。又善與人交。久而敬之。弟敢不入公門鞠躬如也。宦蕚道。久仰賈兄大名。今承光顧。弟不勝欣躍。賈文物道。承老兄臺汎愛衆。可謂好客也矣。弟其舍諸。宦蕚道。老鄔說賈兄才富雙全。故此弟企慕之甚。賈文物道。小弟得之不得有命。不義而富且貴。於我如浮雲。至於才不才。亦各言其志也。小弟曾記幼年時。小弟敝業師贊小弟說。汝。器也。瑚璉也。賢乎哉。我得天下英才而敎育之。一樂也。汝人不知而不慍。不亦君子乎。然而小弟雖聖則吾不能。但所學不倦而敎不厭也。正在高談。家人進來稟道。童老爺到。宦蕚纔起身要迎。那童自大頭戴唐巾。身穿麗服。搖搖擺擺的。一個家人夾着個描金護書跟隨。早已走到廳門首。宦蕚忙讓了進來。從(彼)此都作了揖。相遜坐下。童自大向宦蕚舉手道。素常聞得公子的財勢怕人\footnote{看他開口頭一句便是財字。}。不敢輕易來親近。雖然渴想。要會無由。今有鄔哥的這條門路引進。纔來奉拜。因叫家人在護書中取出個沒字的紅單帖。雙手拿着。打了一恭。親自遞與宦蕚。道。本要寫幾個字的。一來不知該怎樣稱呼。二來我要煩人去寫。恐公子也要煩人去看。故此不曾寫得。公子留着改日拜人也好\footnote{只聞文(古)有沒字牌。不意今有沒字拜帖。又可以長一番見識。賈文物之拜帖已奇。童自大之拜帖更奇。此一日內見了許多奇處。令人樂極。}。宦蕚道。我們旣然要做相與。何必還行此客套。尊帖仍請收回罷。童自大道。當眞麼。旣如此說。小弟竟遵命了。就遞與家人。道。收好了。又省兩文錢。宦蕚道。弟嘗聽得老鄔說。童兄府上在京城中算第一殷實之家。故此奉約了來。大家同結個社。朝夕相聚頑耍頑耍之意。今承不棄。感甚感甚。童自大道。豈敢豈敢。因指着賈文物問鄔合道。此位兄可是有杆子的那大門樓內三個金字有錢的賈進士兄麼\footnote{他見鄔合時開口便是錢。乍會宦蕚開口便是財字。此問賈文物又是錢。非錢字再不開口。古時和嶠人謂之錢癖。童自大或是其後身耶。}。鄔合道。正是當今馳名。天下第一的才子。童自大因拱手道。久想。忽笑道。我前日看戲。唱賈至誠嫖衏。他見那婊子。說了句歇後語。正合我今日見賈兄。他說十八個銅錢放兩處。久聞又久聞。賈文物道。此位童兄尊姓得非童子六七人之童。夫人自稱曰小童之童乎。鄔合答道。正是有名的百萬童老爺。賈文物道。富矣哉富矣哉。旣富矣又何加焉。童自大道。小弟這富翁老爺也不是容易做的呢。富翁是日夜盤算出來的。老爺是大塊銀子買來的\footnote{富翁是日夜盤算出來的。是自然之理。老爺是大塊銀子買來的。雖然體面。但臭味難聞。}。兄不要看輕了。比不得你二位公子。進士是不費本錢的\footnote{賈文物當道。我費的本錢更大。}。賈文物道。富人之所欲也。不以其道得之不處也。若果誠然富而可求也。雖執鞭之事吾亦爲之。但恐爲富則不仁矣。說畢。即欲起身作別。宦蕚道。承二兄光降。豈有空坐之理。備有便飯。奉屈稍坐。賈文物道。飮食之人則父母國人皆賤之矣。小弟決不敢再拜而受。童自大道。小弟是極托實的。還不曾吃飯來的。旣承公子留飯。何不擾他一碗。家裡也可以省些柴米。弟生平自知有兩件好處。一留就坐。一請便往。從不叫主人難心\footnote{雖不足爲好處。然較之裝腔做勢可厭之物稍強耳。}。賈兄不可裝假。賈文物仰天道。嗚呼。我不意子學古之道而以餔啜也。寧不懼其爲士者笑之。童自大道。我好意替主人留你。不聽就罷。何必咬文嚼字。兄要去只管請行。我可是不去的。宦蕚道。還是童兄托契。兄不可固執。鄔合又在傍苦留。他纔肯坐下。笑道。童也慾。焉得剛。因四顧屋宇宏敞。嘆道。山櫛藻梲。何如其居也邦君樹塞門。官府亦樹塞門。可見宦公之位不爲小矣。焉得儉。擡頭看見不足堂三個字。點頭咨嗟道。美哉此堂名也。百姓足。君孰與不足。百姓不足。君孰與足。此之謂也。看見董其昌畫的一軸山水大畫懸在中間。贊道。此非思白玄宰其昌大宗伯董老先生之作者乎。此山乃譬如爲山之山。登東山而小魯之山。登泰山而小天下之山也。此水乃溝澮皆盈之積水也。氾濫天下之洪水也。原泉混混。不舍晝夜之長水也。知者樂水。仁者樂山。賢者而後樂此。不賢者雖有此不樂也。童自大對鄔合皺着眉。道。我也去罷。是還坐坐呢\footnote{自去自留。妙極。}。宦蕚道。兄方纔還勸賈兄。如何此時也說要去。童自大道。小弟實不相瞞。自昨日陪鄔哥吃飯。直到此時。連點心也不曾吃就來奉拜。我昨日曾問過鄔哥吃了飯來是不吃飯來。他叫我不用吃東西罷。我就依實。此時有些餓得很了。肚子裡骨碌碌的亂響。腸子疼得就起來了。若有飯。求快些纔好\footnote{他雖臭吝。倒是個實心人。故有大福。徽州人枵腹嫖妓。正高興時。肚中因空。骨碌碌響聲若雷。妓駭問之故。彼無可答。但曰。這是賊行。童自大或亦是賊形也。蓋江南罵人不堪曰賊形耳。}。宦蕚因催酒。不一時擺下兩張桌子。分賓主坐下。那些家奴一碗碗捧將上來。無非是膾鯉炰羔。山珍海味。杯盤羅列。堆設滿案。賈文物道。我讀書人二簋可用享。何必若是乎饌者之豐也。有盛饌必變色而作。宦蕚道。不過便飯而已。猶恐褻尊兄。何必過譽。賈文物道。狗彘食人。食而不知檢。民有飢色。野有餓莩。可謂率獸而食人也。童自大道。放着這樣香噴噴的好東西不吃。只管說閒話。冷了豈不可惜。我可不能奉候。因低頭大啖。賈文物淡笑道。小人哉。童兄也。鮮矣仁。左丘明恥之。某亦恥之。少刻食畢。賈文物又要起身。宦蕚道。我舍下有一個絕妙的斐園。請二兄同去看看。且還有小酌。尚請寬坐。賈文物道。此非東郭墦間之祭者。何故乞其餘不足又顧而之他乎。恐妻妾相泣於中庭也。然而兄賜食。斯受之而已矣。宦蕚留住二人。同到斐園中四處游賞。童自大道。公子。你這園却也收拾得好。也要好些銀子用呢。叫我就捨不得拿了。開個當鋪。一年不生許多利錢麼\footnote{如此想頭。焉得不做財主。}。鄔合道。大老爺這園也要算京城中第一了。賈文物道。然。誠哉是言也。你看麀鹿濯濯。白鳥鶴鶴。山淥雌雉。烏牣魚躍。當今之囿。舍此其誰也。想經之營之之時。必庶民子來。不日成之。若民欲與之偕亡。雖有臺池鳥獸。豈能獨樂哉。因回顧家人道。此雖非爲阱於宅中。爾等有殺其麋鹿者。如殺人之罪。吾力猶能肆諸市朝。戒之戒之。賞玩了一會。同到一個居蔡軒中坐了。賈文物道。軒乎。吾道體而面之人不得則非其上矣。不得不可以爲悅。得之而不與人同樂。亦非也。今兄與朋友共。其肥馬輕裘之子路何足道耶。不一時。掇上絕精的果品醃臘下酒之物擺下。斟上酒來。大家吃了個落花流水。天色將暮。賈文物道。旣醉以酒。吾飽矣。不能用也。以其時考之則可矣。當咏而歸。款留不住。大家都吿辭起身。賈文物臨行。顧他三人道。三人行必有我師焉。明日行至於我之室也。雖不能以季孟之間待之。然當前以三鼎而後以五鼎爲敬也。宦蕚道。明日自當奉拜。到了次日。宦蕚童自大到賈文物家拜望。鄔合自然是跟去幫閒。賈文物留飮。果然豐盛。飮酒中間。宦蕚向童自大道。我們明早同到兄府上奉拜去。童自大紅着臉不嘖聲。半晌答道。弟家沒人。就弄點東西。恐不中口。也不敢勞拜。改日再請罷\footnote{童自大壞了。也竟會說謊。有一財主吝甚。生平從未請客。一日。其僕在市買菜。有一鄰人問道。你家主人今日請客麼。買這許多東西。其僕道。我家主人要請客。除非來世罷。主人聞之大怒。罵道。我不請只是不請。你怎麼許他個日子。童自大竟許請。還算大方。}。宦蕚是公子性兒。見他那個樣子。知是吝嗇。笑着道。拜是再沒有不拜之理。對賈文物道。我們明日到童兄府上。拜過之後同到我舍下。我替童兄代東。次日。大家到他家拜了。宦蕚把他們約到家中共樂。彼此來往。連聚飮了幾日。童自大自覺過不去。也約他們到家。牽葷帶蔬六碗菜。三杯之後一飯而已。鄔合幾乎吃得快活。連夜間都不歸家。此時嬴氏已獲。家中有人。故此他放心在外。不必多敍。過了幾日。又都在宦蕚家中聚飮。宦蕚對衆人道。如今雖日日飮酒食肉。到底不甚親切。須結拜個弟兄。纔覺親熱些。二兄以爲何如。鄔合接口道。還是大老爺學問深。見得到。想當日劉關張桃園三結義。千載馳名。如今三位老爺這一結義了。後來也是要傳的呢。賈文物撫掌道。妙哉。兄弟怡怡戚之也。倘二兄不幸短命死矣。則二嫂使治朕棲我。豈不勝齊人之有一妻一妾哉。童自大道。要結拜弟兄。我做老三纔來。不然我是不來的。賈文物道。先生何爲出此言也。童自大道。若論起時勢來。公子勢利雙全。該做大哥。賈兄有勢。做二哥。我有利。做老三。這是從古來的一團大道理\footnote{他這一團大道理。不知向何處學來。}。賈文物道。朝廷莫如爵。鄕黨莫如齒。公子一位。今世所頒之次序也無移。至於兄丈夫也。我丈夫也。兄何畏我哉。君子愛人也以德。爲何要居小弟之下乎。且君子惡居下流。兄當效君子上達也。童自大道。還有一說。南京風俗。但是結拜。老兄弟是不出錢的。我故此要占這些便宜\footnote{這就是他的一團大道理了。}。這是實話奉吿。若不依我。就散了桃園。賈文物道。兄一介不與。居簡而行簡。無乃太簡乎。宦蕚道。也罷。他旣如此說。不要強他。就叫他做老三罷。鄔合道。三位老爺結義也是一件驚天動地的事。還要烏牛白馬。殺牲歃血。作篇盟文祭吿天地鬼神纔是。童自大道。費這些錢做甚麼。買半斤燒酒來。弄個小公雞滴點血。大家吃些生雞血酒。鬼混鬼混罷了。何苦多事。宦蕚道。豈有這個此理\footnote{這個二字。甚妙。極寫其學文話而不通也。}。我們紗帽人家做事。要不離紗帽氣纔好。不然就不成體統了\footnote{童自大之紗帽氣定是臭。賈文物之紗帽氣定是酸。他的紗帽氣倒不知是甚味。}。那雞血可是行得的。牛馬雖不必。豬羊定要。遂叫過家人宦畎來。吩咐去製辦稿(犒)物。因想道。別的都容易。但這篇盟文那裡去尋人作。躊躇再四。童自大忽然笑道。公子。你眞是騎着驢子找驢子。現有賈兄這樣才子\footnote{要知賈兄也只算得驢子。算不得才子。}。一篇盟文値甚麼。還要去尋別人。宦蕚喜道。虧你想。我一時倒也忘記了。賈兄可快作起文來。今日就要結拜\footnote{眞是呆公子火燎性兒。}。賈文物正在說得高興之際。忽聽得要他當面作文\footnote{二人結訟。內一理曲者當受責。彼云。我是生員。官不知眞僞。云。說係生員。可作一篇文章來看。其人云。生員罪不至此。賈文物亦當云。我罪不至此。}。如靑天霹靂。掙得滿臉通紅。說道。兄謬矣。祭神在。祭神如神在。今者薄暮。豈結盟之時哉。況齋戒沐浴。然後可以祝上帝。欲禱爾於上下神祇。請緩之。以待來日然後可。宦蕚道。也說得是。老兄今晚回府作了寫好。明早來我家中做個斐園三結義。不可誤了。二人應諾。又吃了一回酒。方纔辭去。這賈文物到得家中。一下轎就慌忙吩咐家人。快去請干先生來。我有要緊話說。就不在家。隨早隨晚。務必要等了來的。那人飛跑而去。他到書房中。忙叫小廝將紙墨筆硯擺下。又吩咐人去買黃紙。叫烹了一壺好茶。放在桌上。又叫預備酒果伺候。不多時。干生早到。你道這干生是何等人也。他是學中一個知名之士。名壹字不驕。生得相貌頗淸。準頭微赤。些微幾莖髭鬚。二旬以外年紀。他父親在日也是個有名的秀才。與鍾趨同窗同學。猶如骨肉。他二人指腹爲婚。後干家生了干壹。鍾家生了一女。彌月時就聘下了。干生八歲時。他父親便病故。只寡母在堂。又過了幾年。他母親也歿了。服滿後。二十歲上纔進了學。他生性放達不羈。惟以詩酒爲事。又平素好結交朋友。所以家道漸漸蕭索了。他讀書的人。又別無營運。纔(終)年守困而已。那時府學中有個敎官。姓廣名聞思\footnote{看官記得此人否。即前童自宏贈金之社友也。}。他愛干生人品才調。甚是契厚。一日。打發個老門斗\footnote{老門斗有所本而來。牡丹亭內云。學中門子老成精。}來請他去講話。干生見學中老師來請。就同門斗來到宅內相見了。廣敎官讓了坐下。說道。我素知年兄近來着實守困。奈我鱣堂俸薄。愛莫能助。心甚歉然。昨日都督李公請了我去。托我要請個西席。愚意要奉薦年兄。我素知年兄豪放不羈。恐不屑爲此。但聖人云。素貧賤行乎貧賤。君子無入而不自得。況設帳一事。也是讀書人所爲。不知年兄的意思若何。可肯屈就麼。若謂可。我當奉薦。干生一來家中寒薄。二來身閒無事。又承老師殷殷見愛。便道。旣蒙老師見愛。敢不遵命。廣敎官見他肯去。心中甚喜。叫門斗沽了一壺。內邊要了兩碟小菜來。一碟炒苜蓿。一碟酸虀\footnote{雖是寫廣文寒酸。到底是寫徽人吝嗇也。}。二人對飮\footnote{到底古人不同。順着厚道。今之求人薦館者。非有封儀不行。廣敎官爲干生之館反破費己鈔。沽酒求之。今日大的(約)難得。}。談了半日近來月課的時文。干生纔辭了回來。你道要請先生的這個李都督是何處人氏。怎麼出身。他祖籍山西大同府人\footnote{大同人。妙。謂今日延師之東家大約皆同也。}。代代俱當丘八。他父親叫做李之富\footnote{他父親叫做李之父。他母親定是母氏了。}。母親早亡了。他妻子滑氏\footnote{人家妻子似此姓者極多。}。也是個一個字的鄕紳兵的乃愛。他有四個兒子。七八個孫子。他單名一個太字。他吃糧時原名李大。他一字不識。粗鹵至極。這待人接物禮貌上的儀文。一毫不知。他當日隨着主帥去征流賊。他心雄膽大。膂力過人。該他的命好\footnote{蘇東坡云。但願生兒愚且鹵。無災無難到公卿。李太之謂也。只要生來命好。要識字做甚麼。}。遣他去禦敵。無敵不摧。着他去攻城。無城不克。他也並不是甚麼勇冠三軍。力雄萬夫的好漢。該有他官星照命。自有機會來湊他。一日。他跟着主帥同流賊對敵。他騎的那馬被賊的馬鎗子打着了耳朶。忽然在陣中驚跳起來。控勒不住。李大用力打了幾鞭。那馬性起。自本陣上直衝入賊陣中去。他着了急。怕賊來殺他。他舉起刀來。橫七豎八。亂砍亂刴。一來古語說。一人拚命。萬夫難敵。二來賊隊中不防他這一衝。竟有些亂了。官兵也不知他是馬驚。只當他奮勇衝鋒。見賊亂了陣勢。誰不望殺賊建功。大家訥(呐)一聲喊。齊奮力殺將上去。賊兵大敗。誅殺殆盡。論功行賞。他獨得了頭功。又一回。飛報到來。流賊據了蔚州。主帥連夜發兵救援。他跟了同去。到了城下。流賊固守甚嚴。攻了幾日。城不得下。主帥大怒。命造了雲梯。令衆兵爬城。也虧他膽大。就往上爬。衆人隨後。離城垜不遠。城上一個賊一鎗攮來。他是仰面看着的。一下閃過。右手攀住雲梯。左手一把將鎗桿攥住。那賊若往下一送。他便不死也要跌傷。該他的造化。那賊反往上一提。他趁勢向上一躍。跳上了城。綽起右腕上刀來。順手一刀。把那賊刴倒。便舉刀混砍。衆賊見有人上城已自驚慌。又見後面的人魚貫而上。喊了一聲。各自逃生。他同人砍開城門。放官兵入城。衆賊殺的殺了。逃的逃了。論得城之功。他又是頭一個。如此巧事也不能盡述。因他屢立軍功。漸次陞遷。做到了副總。他有一個小舅子。名字叫做滑稽。他父親雖也是兵。却是個識字的。接交官府衙門書辦之類。這滑稽也讀過幾日書。心下倒還明白。李大做了副將。署中公事多了。他捨不得銀子請幕賓。就約小舅子替他主文。撥了分馬糧與他。後來李大陞了南京後軍都督府都督同知。單騎赴任。將父親妻子兒媳孫兒俱留在故鄕。他做副將的時候。又娶了四五個妾。臨行再三托滑氏留心照看。千萬嚴緊。不要叫他們弄出醜來。我到任後。等尋了房子。慢慢來接你們。滑氏應諾。他仍帶着小舅子並十數個家人去了。到了南京上過任。不必細說。他此時的名字還叫李大。他因自己是大了。他的四個兒子就叫李二。李三。李四。李五。一日。那滑稽因勸他道。你今日做到都督。是朝廷大臣了。你這名字甚是不雅。還得改一改纔妙。李大道。我自娘肚裡掉下來就是這個名字。今日做了這麼大官。那些兒不好。滑稽道。這個大那裡是名字。因你是大兒子。所以就叫大了。後來當兵就不曾改。今日做了顯職。還用這個字。不怕人笑話麼。李大道。這個大字我認熟了。要另改一個。不但別人不認得是我。連我也不認得是我了\footnote{他這奇談可笑。然而世上我認得我者誰耶。}。滑稽想了想。笑着拿筆寫了個大字。內中點了一點。問道。這個字你可認得。就改做他罷。李大道。我嘗見一塊字底下點一點。我問書辦。他們說上頭的一塊字是菩薩。底下這一點就是那塊字。你叫我改做李大大的意思了\footnote{辱翁曰。此說竟是極。太字原係大字下兩點。篆書作〖介介〗。所謂複篆也。孰謂此老兵不識字。}。忽大笑。罵道。你這騾騾(㞠)子攮的。你同我頑罵我咧。連你姐姐都罵上了。滑稽道。我好意替你改名字。忽(怎)然(麼)是罵你。你倒罵起我來。他笑道。我前日差了幾個兵到後湖裡去打魚。魚沒有打得。拿着了許多烏龜。他們打了報單來。說烏龜有大大的多少。小小的多少。那個大字底下也是一點。你罵我是大烏龜。可不連你姐姐也罵了。滑稽道。不是這話。那一點是在底下。這一點是在內中的。他又道。旣不是大大。大字胯襠裡墜着個東西。大的是大毬了\footnote{奇想。然而他竟叫大毬亦可。}。滑稽笑道。這是個太字。人稱太爺太太就是這個字了。怕你不認得別的。這個太字你這(還)容易認。雖不甚佳。比那個大字還像個名字。他大笑道。好得很。我叫做李太。妳姐姐叫李太太。他比我大些些不得。我有些怕他呢。你就吩咐閤衙門的人。我的名字叫李太了。滑稽道。這如何吩咐人。你如今是官。改名字要上本的。上邊准了。有小抄到各處。人就都知道了。何用吩咐。李太依他。題了一本。准了下來。纔改了今名。一日。李太向滑稽道。我這些日子細想起來。你勸我改名字。是你哄我。明是拿着我奉承你姐姐。滑稽不懂他的意思。說道。你這話我就不解了。李太道。你姐姐是我的老婆。倒叫李太太。我倒叫李太。明明的說你姐姐大似我。把我怕老婆的招牌替我擺了出去。不是你拿我奉承你姐姐麼。還有一說。人叫你姐姐一聲李太太。倒把我的名字叫了兩聲去了。滑稽道。豈有此理。字雖一樣。有兩個講法。原該用那丕泰的泰字。因這個太字你好認。借音取那個泰字之意。是極好的。你不用多疑。要說叫我姐姐一聲李太太。把你名字叫了兩聲。那還是叫我姐姐。你前日沒有改名字的時候。人叫你李大老爺。難道也是叫你的名字不成。他想了一會。道。你的嘴能幹。我說不過你。我到底心裡信不過。可恨前日冒失上過了本。不然還是我的大字好。我做着個大官。名字自然該是大\footnote{愈想愈奇。豈當日在下位時。爾名李小耶。}。滑稽道。不但你的名字該改。就是四個外甥的也該改。那裡有個老子叫李大。兒子同着二三四五排行的理。我如今也替他們改改。當日岳少保說。行兵之道。智信仁勇嚴五字闕一不可。李嚴三國時已有了。況你也只四個兒子。就把智信仁勇排去。你又是武將。恰合道妙。他道。偏你會這麼瞎〖扌扉〗。你在那裡又認得個甚麼岳少保。聽見他說的。我如今還聽你的話呢。我也不懂得甚麼叫做智的信的。況且我纔上本改了名字。又替娃娃們去上本。囉囉娑娑的。滑稽道。你是官。故要上本。他們又上甚麼。李太道。旣如此說。改改也好。他們如今都是公子了。若單叫李二李三的。實在也不好聽。我前日點兵。這樣名字多得很。我先還疑惑。我家的娃娃怎麼又在這裡當起兵來。細看看又不是。我也覺得不好\footnote{李太正當名滑稽。他無一語不全又失笑。}。我怕又要上本。故此罷了。旣不費事。等我替他們改。但他們這二三四五幾個字我叫慣了。萬萬去不得。一個人添一個奇字就好了。我聽得人說。人生在世。要妻財子祿壽俱全就是好的。他們的婆子都有了。那個妻字不用了。叫做李二財。李三子。李四祿。李五壽罷。你說這幾個字我想得奇不奇\footnote{眞奇。虧他想。}。又明白好懂。可不強似你謅的那幾個字麼。滑稽見他不通得可笑。也不同他爭講。任他自己去改。過了些時。他叫滑稽寫了封家信。與他老子說。南京房子甚貴。還不曾買。目今權借衙門暫住。等買了房子。再來搬接家眷。又把自己改名。兒子們添名的話。詳細寫了。差了個大管家叫做李得用回去。過了兩個來月。李得用回來了。投上老主的家書。他問了家中大小平安。心中甚喜。叫家人道。快請舅爺來念。家人道。舅爺往雨花臺耍看去了。李太道。這怎麼處。也罷。叫個書辦來念罷。頃刻叫了個書辦進來。他把那家信拆開。遞與他。道。這是太爺帶與我的稟帖。你念與我聽。那書辦接過。打開一看。不敢做聲。李太道。你爲甚麼不念。是我家太爺給我的。又不是給你的。你看看自己知道就罷了麼。書辦道。並不是家信。叫書辦怎麼念。他大怒道。這是我家人纔帶來的。怎麼說不是。忘八肏的。老子肏你的奶。你當一個書辦。連一塊稟帖也念不來。要你做甚麼。要你弄鳥。喝道。攆出去。再另叫一個來。家人去了來說道。別的書辦都回家吃飯去了。不在這裡。別的書辦何嘗回去。因這個書辦向衆人說道。並不是家書。是一小學生的倣。怎麼個念法。白白的捱了一頓罵。衆人聽說。誰還肯進來。故此都推吃飯去了。李太見沒人念。急得罵滑稽道。這個瞎毬攮的。在家坐坐罷了。偏偏今日他又去耍甚麼臺臺的。吩咐道。等舅爺回來。就叫他到上邊來。家人答應了。你道這封字那書辦果然連家信都不會念麼。原來這李得用沿路呷酒嫖妓。把封家信不知如何失落了。着了急。因想主人不識字。又一竅不通。到了一個鄕學館中問那先生要了一張小學生的倣。封了來哄主人。那書辦雖不知這些情弊。但看見這個字。疑必有故。不肯說破。恐得罪了帶書的管家爺。白受了一場大罵。午後滑稽回來了。李得用恐他說出。再三央吿求他遮蓋。滑稽因他是姐夫的大管家。況他們素常又極其相厚。滿口答應。到了上房。李太道。等你這半日纔來。俺爺帶了塊稟帖來。那書辦又不認得。你念念與我聽。滑稽接過來。笑着念道。

\begin{quotation}

上大人。某乙己。化三千。七十士。爾小生八九子。佳作仁。可知禮也。學生李彬習字。

\end{quotation}

念完了。他滿臉慍色。道。一塊老子與兒子的稟帖。寫得明明白白的也好懂。這是些甚麼文話。我一句也不知道。問那李得用道。太爺的才學當日也比我高不多。如今爲何這樣文起來。難道老都老了。從新又上學念書去麼。李得用先還恐他知覺。捏了兩把汗。今見他問這話。心中暗喜。忙跪稟道。太爺雖不曾上學。因老爺官尊了。近日同這些鄕紳舉監文人們來往。大約是講學講道了的\footnote{辱翁曰。如此趣話却好。}。李太搖頭道。就是同文人講講。那裡就文到這個地位。眞是迂夫子的卵袋。文縐縐的。大約還是煩了甚麼不通的才子寫的\footnote{不通的才子。奇聞。}。又向滑稽道。你可懂得。你要懂。細細講與我聽。我叫買辦打燒刀子同牛羓請你。滑稽笑道。你聽着我講。頭一句上大人。說你如今做了大官是個大人了。上覆你這大人。是問你好的話。李太喜道。明白明白。講得好。滑稽又道。某乙己。某就是我字。你不見戲上都自己稱某家。這某字是太爺自己稱呼。說你在任上。只某一個在家。李太道。越發明白。滑稽又念道。化三千。七十士。太爺有三千句話要對你說。內中有七十件事。李太道。我的爺爺喲。你老也老了。省些心罷了。那裡就有這麼些事。虧他老人家記得。滑稽不往下念。李太道。你怎麼念了這幾句。底下不講了。滑稽笑着向他戲說道。我講了怕你要惱。李太道。這纔說的是沒來頭的話。這是俺老子與我的字兒。你不過講與我聽。有甚麼話得罪了我。我就惱。只惱我老子。你又不是俺老子。爲甚麼惱你\footnote{一竅不通的人亦有趣。他之趣語不少。只此數句。到不通可笑之至。非此人不能有此話也。}。滑稽笑着念道。爾小生八九子。爾字就是你字。說你的幾個小婆子生了八九個兒子。李太大驚道。我不在家。是那裡來的這些娃娃。滑稽道。書上寫得明白。佳作仁。說是家裡做出來的人。李太怒道。你那姐姐也不是個人娘養的。我臨起身再三托他照管。他們如何就做出這些娃娃來。我想來別人也不敢。不要就是俺那爺老沒廉恥做的事罷。滑稽笑道。你好想。所以臨了說可知禮也。說你要猜到這上頭。可就是知禮的了。李太大怒。搶過字來扯得粉碎\footnote{李太則大怒。看書者則笑倒也。此一封書。眞千古家信絕唱。見此而不大笑者。必李太之儔也。}。面紅頸赤。低頭無語。半晌。忽又問道。從(後)頭還有甚麼李彬習的又是怎麼說。滑稽道。他說學生李彬。人家老子稱兒子做學生。這也是文話。因你做了大官。要叫你名字不好意思的。要稱你老爺又無此理。你原當過兵。要稱你做李兵。習字。媳是太爺稱呼媳婦。就是我姐姐了。說媳婦不另寫字了。同這一封字。所以說學生李彬習字。講完了。滑稽忍不住哈哈大笑。道。你快叫人去打酒買牛羓來請我。李太道。大毬的牛羓。把些小婆子的巴子還不知弄成個甚麼樣兒了。你還想吃牛羓子呢。滑稽笑了出去。李得用向他感謝了又感謝。忙去買了許多佳肴。沽了一瓶美酒來奉敬。不題。再說李太一腔怒恨。徹夜無眠。次日即打發李得用帶了四五個家人。回去接滑氏同幾個小老婆並兒子媳婦孫子來京。單不接他老子。也不寫家信。衆家人到了家。李之富聽得兒子來接家眷。獨不接他。問家人是何緣故。家人雖有知道的。都懼李得用。俱不敢說。只答應不知道。李之富恨了兩聲。復又笑道。我知這奴才的心了。他如今做了大官。說我原是個兵。恐怕我玷辱了他。故不來接我。連字也沒一封問問安。眞畜生。眞畜生。那李太做了多年的官。俗語說。官久自富。他家中也置了許多田產佃房。李之富儘夠受用。也就在家。並不管媳婦孫子去不去。滑氏臨行。帶了衆人到公公處辭行。那老兒也無多話。只道。你對那奴才說。叫他長遠在外做官。就死在外邊。總不要回來見我。那滑氏見公公動怒。也不知是那裡賬。起身去了。一路無話。到了南京。他夫妻父子相會了。李太見了這幾個小老婆。睜圓了眼瞅着。咬牙切齒。不交一言。晚上他夫妻上床幹了一次接風的事。完了睡下。李太埋怨滑氏道。我臨來那樣托你管着這幾個小婆子。不要弄出醜來。你應滿了的。怎麼這一二年裡頭就叫他們養了八九塊娃娃。滑氏驚道。你聽人胡說。這是那裡的話。李太道。你還瞞我。是俺那老沒廉恥的爺帶來的信說的。還說就是他在家裡做的人呢。我所以纔不接他。這滑氏當日見他娶這些小。心中未嘗不惱。但他是個兵的小姐。家世寒微。今日見丈夫做了大官。攜帶他做了夫人。享榮華。受富貴。揷金戴銀。呼奴使婢。未免有些勢利。敢怒而不敢言。今聽見他這話。雖不明白內中的細故。知他是誤聽了。方悟到不接公公之故。遂借他的話因答道。誰叫你當日尋這些浪貨來\footnote{先責丈夫之不是。}。那時我要阻你。倒像我吃醋一般。只得任憑你胡做\footnote{次表自己之賢慧。}。你托我照管他。我只管得他們的身。管不得他們的心。沒有個拿封皮長遠封着他們那騷東西的道理\footnote{再責諸妾之無恥。}。況又是你老子做的事。叫我一個媳婦如何管得。只怨你自己不是。怎麼倒反怨我\footnote{終歸不是於公公。且明己之不得已。此婦眞滑。不但姓滑而已。}。李太怒道。明日我把這幾個淫婦全殺掉了。纔出得這口惡氣。滑氏知他是誤聽。故此謅出些話來。激他打發了這幾個妾。他好獨樂樂之意。忽見他說要殺。恐他鹵夫性兒誤害無辜\footnote{還算賢婦。}。忙道。還虧你做着個官。王法都不知道。人都是輕易殺得的。養漢拿雙。你又不曾拿着他。這一殺了他們。倘被人知道參了。不但壞了官。連命都送了呢。就算着不到這地位。如今這醜事人都不知道。若無緣無故殺了這幾個浪肉。不明明尋頂綠帽子戴麼。你只把他們攆了出去配了人。眼不見爲淨就罷了。李太生來粗蠢。滑氏乖巧。凡說話行事。李太都在他籠絡中。素常有些懼怕他。故此極肯聽他言語。次早起來。並無別話。把衙門中沒有老婆的兵叫了幾個來。將幾個小老婆即刻驅出\footnote{辱翁曰。大陰德。}。每人配了一個去了\footnote{這回得自在。}。這幾個妾也不知是甚緣故。還以爲主人開籠放鳥。得配一夫一妻。好生歡喜感激。滑稽背地私問姐姐是爲甚麼。滑氏把李太誤聽話詳細吿訴了他。滑稽不禁失笑。也把假書並自己同他講頑的話也向姐姐說了。笑道。不想這草包弄假成眞。滑氏纔知內中的這些緣故。心裡感激兄弟同李得用了不得。偶然一日。李太叫了兒子們到跟前。說道。我常聽見人說甚麼文武世家。我自從七八代前的爺爺當兵起。傳流到我。我如今又做了這樣大武官。這個武世家是不用說了。我看你們都大了。筆拿不動。弓拉不開。是俗語說的。毛坑裡拾得一桿鎗。聞也聞不得。舞也舞不得的了\footnote{文不得。武不得。此類人多甚。不獨李太諸子。}。如今我要雇個敎書的來。把孫子們叫他識幾個字兒。可不就是文武世家了\footnote{好想頭。眞是文武世家。}。前日俺爺帶了那封稟帖來。你舅舅又不在家。叫了個書辦來又不認得。好不爲難。若孫子們後來認得幾個字。何必求人。兒子們見老子這樣說。不敢阻他的興。李太因此請了廣敎官來。托他要請個大通的好先生。廣敎官因想干生寒苦。又素相厚敎。要薦他。問明了他肯去。親到李太家來。說先生請下了。是個名士。幾時進館。李太道。且商量明白了着。一個月只好一兩工銀\footnote{近來就算是好館了。}。飯是自己回去吃\footnote{近來亦多有之。}。廣敎官笑道。束脩多寡倒也罷了。府上這樣門第。那裡有先生回去吃飯的理。若是住得近還罷了。要住得遠。一日回家吃兩遍飯就晚了。還讀甚麼書。他想了一會。又囗(皺)着眉曲指頭算了算。說道。供給他吃飯。一日只算五分銀子。一年倒要十八兩。比工銀還多。這是買馬的錢少。製鞍的錢多了。成不得。廣敎官道。讀書的人飮食倒不責備。就是家常茶飯也可款待。只要潔淨應時。李太道。旣如此說。一日兩頓。就是隨常茶飯。只好初一十五吃個犒勞有些肉。閒常是沒有的\footnote{可謂待先生如此其豐且敬也。}。至於要吃點心吃酒是他自買。老敎先\footnote{奇稱。大約他聽得人說敎官先生。他減去二字好稱呼。眞好(妙)人。}。你對他說明白了就叫了他來。我還要親自考他一考。果然通纔要。廣敎官道。那裡有這個禮。還差人去請纔是。辭了出來。親到干生家。向他道。館事雖明白了。但只脩金太薄。年兄將就負屈一年罷。只當借館中讀書。就是供給不堪。也免得自己心操薪水。年兄可肯去麼。干生見老師情意殷殷。也還以爲他雖是武弁。已是個顯官了。必定還知些人理。就應允了。廣敎官又復了李太。叫他差人拿帖去請。李太道。雇他敎書。又不是請他吃酒。用甚麼帖\footnote{李太的話也有長人見識處。我今日方知帖子是請人吃酒才用。}。叫人口說罷。廣敎官見他如此粗俗。也不與他爭講。叫門斗帶那衙役同到干生家來請。干生見沒有名帖。雖心中怪他無禮。然却不過老師面皮。只得同往。到了後堂。見他在正中一張虎皮交椅上坐着。動也不動。看他那形狀。令人絕倒。有幾句寫他的行樂。道。

\begin{quotation}

形容鹵夯。相貌猙獰。話語多粗俗。儀文沒半分。心如頑石無微竅。腹內稠糊有一盆。巍巍高坐墊臯比。却是當年一老兵。吁嗟乎。果是沐猴而冠。誠然哉。不謬獸性人形。

\end{quotation}

干生先還想與他講些揖讓之禮。見他這個蠢牛樣子。一肚子沒好氣。連手也不同他拱。見傍邊列着幾張椅子。也就昻然坐下。只見他問道。你就是先生麼。干生忿然答道。正是。他說道。我這樣人家的先生。要會講書的纔要呢。你可會講麼。干生又是那惱。又是那好笑。說道。我們一個做秀才的。甚麼書不會講\footnote{近日做先生者竟大不然。}。你要講甚麼。他道。別的我不懂。百家姓我還知道兩句兒。你就講講我聽。干生笑道。你要一句一句的講。還是要一個字一個字的講。他道。自然是一塊塊一塊塊字兒講得纔明白。干生笑着道。你聽我講。趙錢孫李這百家姓是當年宋朝的人作的。那宋朝的皇帝姓趙。所以趙字就放了頭一個。世上除了皇帝。就算有錢的大了。故此第二個就是錢。這個孫字你當是誰。就是那大鬧天宮的齊天大聖孫猴兒。只因要讓皇帝。又要讓有錢的。沒奈何。屈了他在第三。干生復大笑道。這個李字就是你了。除了這三個。還有大似你的麼。故把你放做第四\footnote{有一海南先生講子曰予欲無言一章書道。夫子說。俺不說舍兒咧。子貢說。夫子不說舍。叫俺們說舍呢。夫子說。天說舍兒來。春兒夏兒秋兒冬兒的過。葱兒韭兒韮兒蒜兒的生。天可賣(曾)說舍兒咧。予以爲此講可冠絕千古。不意干生之講百家姓更妙。又高出其上。}。\endnotemark[1]那李太大喜。大笑道。講得好。講得好。這叫做上堂三下鼓。通通通\footnote{這一篇講章。不但李太叫通。我亦謂之通。}。干生又笑道。這一講還不足爲奇。我還會倒過來講呢。李太愈喜道。我雖然這樣大年紀。從沒有聽見倒講書。煩你再講講我聽。干生笑道。你姓李的穿上幾件猴兒皮。再有幾個錢。除了皇帝。倒過來就算你大了。他聽了。仰在交椅上哈哈大笑。道。好先生。好先生。這纔是個眞才子。講得有理得很\footnote{他並不是謬獎。}。因四顧家人。道。我果然這樣大麼。先生講得可是。衆人道。先生講得是得很。他笑着向干生道。我又沒有讀過書。知道甚麼叫做百家姓上有趙錢孫李這兩句。我當年跟着主帥時。外頭報流賊犯邊。主帥差了個周守備吳千總去征勦。他去了些日子。總不見回報。那一夜主帥做了一個夢。夢見竈跟前生了一棵李樹。第二日叫人圓夢。他衙門裡有個大通的主文相公姓鄒。說道。這個夢有些不祥。多管應在周守備吳千總兩個身上。主帥問他怎麼見得。鄒相公說。天機不可預洩。等應過了再講。又過了兩日。探馬來報。說周守備吳千總都被流賊殺了。主帥問鄒相公前日的夢怎麼應在他二人。鄒相公說總是讀的書多了就無所不知。百家姓上說竈前生李。周吳陣亡。故此就先知了\footnote{世上偏是善謅的人專謅得着。}。我聽了記在心裡。今日考考你。誰知你比他講得更通。眞是名公。忙吩咐家人將馬房隔壁打掃了兩間做學房\footnote{幸喜先生通。才在馬房隔壁。若稍次。定在東厮中做館地矣。}。大大小小的七八個學生來拜了先生。不但沒有贄見禮。連進館的酒都沒有\footnote{近來竟以爲例。行之者十僅二三耳。}。干生知他是個不知禮的人。也不與較量。過了幾日。這學生中那三四個小的還知些怕懼。但他那父母又溺愛得很。一會叫人來說。孩子小呢。不要拘管壞了。放他去走走。干生見東家來說。只得依。去了一會又來。坐不上半個時辰。又來說道。恐怕孩子餓了。叫他進去吃些點心。一日到晚。如走馬燈一般。不住的來來去去。到了這幾個大學生。甚是頑劣。內中一個居長的。名叫李蓀。是李三子的兒子\footnote{李三子之子自然是李孫了。妙妙。}。頑劣更甚。又刁鑽心壞\footnote{此類學生多甚。}。內中也獨他打得更多。他父母叫人來說。都是一樣的學生。先生要打一齊打\footnote{奇談。只聞得有陪綁的囚犯。從不曾聽得有陪打的學生。}。怎麼偏心單打他的兒子\footnote{宦家子弟成器者少。豈朱門皆生餓莩耶。皆緣姑息之過耳。}。干生聽了。一肚氣惱說不出來。打得更狠。這幾個學生一日到晚書背不得。字寫不來還在次之。干生但低頭看書。那大的中就不見了兩個。叫人去尋了來。每人打了幾下。還不曾打完。那兩個又不見了。及至拿了來。纔打着。回過頭來。先那兩個眼淚還不曾乾。又不知去向。只得拿來罰跪。他便謊說要出大恭。干生以爲實話。況且沒有等他撒在褲子中的理。只得放去。他人不知跑到何處頑跳去了\footnote{非做過不知斯文宦家之先生者。不得其詳。}。干生每日氣也淘盡。他家那供給的飮食更爲可笑。他山西邊外的人不吃粳米。叫人到山東買來的小米蕎麪。他每頓都是這兩樣在一處。倒上許多的醋。或切上許多醃菜。還着上一大把秦椒。又不像粥。又不像〖飠強〗餬。又酸又鹹又辣。進不得嘴間。或漆黑的麥麪打那一寸厚的鍋盔。挺幫鐵硬。嚼也嚼不動。他家中吃的都是酸菜水。從不知吃茶。干生如何吃得慣。要鍾茶千難萬難。那鍋盔又容易吞不下去。餓得沒奈何了。只得伸着脖子乾咽。又不好在飮食上講論。只得捏着鼻子拿來充飢。天氣漸漸炎熱。隔壁馬房中那馬糞臭得薰得要死。那紅頭大金綠蒼蠅滿屋都是。在頭臉上混撞。先也甚是難過。久之。如入鮑魚之肆。也就不覺得十分嗆鼻。也耐過了。但只是每頓送一大碗翻滾熱的蕎麪湯來。天氣又熱。如何進嘴。放在桌上晾了一會。等溫些好吃。那大金蒼蠅就撲上幾個。在碗內燙得稀爛。一肚子子飄得滿碗全是蛆。忍不住惡心。只得倒去餵狗。再要添時又沒有了。只得忍餓。深悔當日不該輕諾。一日大雨。滿屋皆漏。如篩子一般往下淌水。那些學生妙極。恐濕了衣服。也不等先生吩咐。如同躱大兵的一般。轟的一聲跪(跑)個乾淨。把書橫三豎四撂的滿桌。干生恐滴濕了。倒替他們一本一本的去收。雨略止了。外面雖然小下。學房裡倒還大下。四處滴水。竟無一處可以容身坐得。干生叫人對李二財說要回去躱雨。叫個人打傘送他家去。李二財吩咐了一個官轎夫拿傘相送。干生走到途中。見濛濛細雨猶然未止。信口念一句道。

\begin{quotation}

濛濛細雨潤如酥。

\end{quotation}

那轎夫忽說道。相公好詩。我續一句罷。干生驚異道。你一個擡轎的人。如何會作詩。他笑道。我難道娘胎裡生下來就是擡轎的麼。不瞞相公說。我當日也敎過書。因東家相待得十分刻薄。遂賭了一口氣。想道。人生天地間。何事不可爲。爲甚麼受這個罪。身爲無罪之囚。妻守有夫之寡。況古人說。寧爲轎夫長。莫做一先生\footnote{此人竟善於套古。}。我因此纔到都督府營謀捐納了一名轎夫頭兒的。干生笑道。旣是你能續。你續一句看。他朗吟道。我

\begin{quotation}

夫師持傘送師夫。

\end{quotation}

干生訝道。你這句令我不明。何以謂夫師。又何謂師夫。只有人稱師傅的。從未見師夫兩個奇字眼。他笑道。夫師者。我今是轎夫。昔日曾爲過師。故稱夫師。師夫者。相公不要見罪焉。知今日之師。異日不爲轎夫耶\footnote{辱翁曰。此轎夫眞正大通。不塊(愧)爲人師。}。師也轎夫也。轎夫也師也。其間不能以寸去也。不是我斗膽說。我與相公還算同寅呢。干生也笑道。你雖當日敎過書。但今日旣爲轎夫。我是他家西賓。大不同了。我與你。

\begin{quotation}

堂前坐立分高下。

\end{quotation}

他大笑道。據我看來。相公雖然自譽。吾語汝弗如也。

\begin{quotation}

若論工銀君尚輸。

\end{quotation}

干生道。這又怎麼講。他笑道。我一年十二兩銀子。還有三擔六斗米。相公你只得十二兩工銀。尚還無粟與爾之鄰里鄕黨。豈不輸我一籌。說話之間。干生已到了家。他說道。相公。大家說頑話。千萬不要介懷。拿着傘去了。干生想他說的話。倒也好笑了一會。過了兩日。天大晴了。干生只得又到館中。每日只同這幾個頑童淘氣。又是那氣。又是那好笑。道。這幾個也不是學生。竟是一群野牛。我也不是他家請來的先生。是他家雇來做牧童的。干生在他家坐了半年館。李太同幾個兒子連學房門也不曾進。並不知道陪先生坐一坐。惟有滑稽曾讀過書。還知些人文道理。常到館中陪先生坐談。講講閒話。倒也還相投\footnote{有此一線。故後來好到干生任上也逍遙。伏下。}。干生偶然一日心有所觸。向衆學生道。你爺爺雖是行伍出身。在官場中也混久了。別的不知道也罷了。難道連天地君親師五個字都不知的麼。我是你家的先生。就是師了。你爺爺待我。一點禮貌也不知。成何道理\footnote{這竟大不然。我常見非行伍出身者亦多如此。}。學生們回去吃飯時。那李蓀就把先生的話向他爺爺說。李太笑道。這個書呆子好不知事。他不見多少的官兒在我跟前磕頭禮拜的。我還不理。那些衛所的指揮千百戶在我面前。不要講坐。連站的地方還沒有。他一個精窮的秀才。我待他坐着就算我敬重斯文得很了。他還想爭甚麼\footnote{奇談。}。不說他秀才們不知官體。反說我不知禮貌。況他敎的是我孫子。就同我兒子是一輩了。叫我如何敬他\footnote{更奇。千古未聞之奇語。}。你就把我這話敎導他。李蓀到館中又把這話說了。干生大笑道。蠢牛蠢牛。幸喜我敎的是他孫子。若是敎他的曾孫。竟把我當他的孫子相待了。干生一心要辭了回去。又因廣敎官囑託。諄諄勸他了此一年之局。彼此存個體面。只得耐住。因長嘆道。大丈夫不能奮飛。餬口靑氈。受此小人下賤。我見有人尚鑽刺爲西席者欣欣爲榮。是何心耶\footnote{游混公卜通輩處此。自然爲榮矣。}。因信筆題了一調靑衫濕的詞。道。

\begin{quotation}

靑氈第一低微事。腆面向人誇。拘囚無罪。奴顏婢膝。依傍東家。措身無地。蒙羞忍恥。乞食爭差。斯文掃地。逢人羞道。心愧無涯。

\end{quotation}

纔寫完。那廣敎官偶來相探。干生忙接着進來。讓他坐下。他一眼看見桌上那詞。取過一看。笑道。年兄此言必有所謂。干生細將館中這些妙處並李太所說的話。低低相吿。那廣敎官不禁大笑。道。是我屈了年兄了。也不想一至於此。又道。我之大賢與於人何所不容。況宰相肚裡好撐船。年兄且耐這幾個月罷。干生笑道。那船直撐了來還可容得。他竟橫撐了來。叫門生如何能容。說罷。二人大笑。又閒談了一會。干生要了七八回茶。只見答應。並不見到。廣敎官道。不消了。就立起作別。干生送他出去。那李蓀見那張詞在桌上。悄悄偷了。藏在身邊。干生進來。見那張詞不見。因沒要緊。也不尋覓。到午間放吃飯。這李蓀到他爺爺處來。這日李太的一個大肥騾子病死了。他叫人開剝煮熟。切做大臠。同着幾個兒子在那裡痛吃。正吃得大飽。忽李蓀走到跟前。將那首詞拿出來。道。這是先生寫了罵爺爺的。方纔同那個敎官看了大笑。又低低的向那敎官罵了爺爺好些話。我也記不得那許多。李太怒道。他爲甚麼好好的罵我。叫兒子們道。你們大家看看。看罵的是甚麼話。原來他這幾個乃郞都不願兒子讀書。因是老子的主意。不敢違拗。又見先生常打他們的兒子。心疼得說不出來。那幾個婦人又護短。常嘓噥丈夫道。一個孩子們好容易養大了。憑他們頑頑罷。好好的叫他們念甚麼書。受這樣的罪。時常打得喞嘛喊叫的。你們也忍心麼。我見你們沒有念過書。一般也過日子穿衣吃飯的。他們聽了老婆的話。巴不得攆了先生去。讓他兒子好快樂。他四個人本不認得字。見老子叫看。假意接過來。看了一會。那李二財認得一個奴字。指着說道。這不是個奴才的奴字麼。他罵爺是奴才呢。好罵好罵。又道。我前日在學房門口過。也不知他罵那一個孩子。甚麼狗肏心。肏肏心。又肏心。做先生的人這樣話都罵出來。又咒孩子們短命死矣。眞野賊奴。罵得這麼刻毒\footnote{他雖不識字。記性却好。竟能過耳不忘。}。我氣得了不得。要吿訴爺。恐怕爺嗔。說請個先生敎孫子。我們護短擠撮他。今日連爺都罵起來了。李四祿瞎指着一句。道。罵爺奴才値甚麼。這一句纔罵得狠呢。我也不敢說。李五壽又指一句。道。你說那一句狠。我看還輕。這一句纔利害呢。李三子道。你們不通文理。都是混說。我看這紙上東一道西一道畫的。那一句不狠。一大些黑字。都是人罵不出來的話。他都罵出來了。不要說是爺。叫我也受不得這些惡話。就敎出個狀元來也有限。這樣的壞人不攆掉他。還留他做甚麼。被他轟揚出去。爺倒罷了。叫我們拿甚麼臉面見人。他弟兄幾個。你一嘴我一舌。把李太激得一腔怒氣。拍着胸叫道。氣殺俺咧。氣殺俺咧。一沖性走到學房。干生正在看書。忽見他氣忿忿走來。尚不知何故。還笑着站起相迎。他指着干生罵道。你這驢毬毬攮的。我管下多少兵丁。一年只關十二兩銀子。還當多少差事。稍誤了還要打狗腿。你自己摸摸良心想一想。我一年十二兩銀子雇你來家。成日高高的坐着。你做些甚麼重活來。一日兩頓小米飯蕎麪湯供給着你受用。你吃得肥瘋了。反罵起我來。走你奶的村路。我的孫子就不念書也不怕沒有飯吃。他們蹺起腿來比你窮秀才的頭還高些。干生也不知是因甚事。見他無狀。也大怒道。我還愛在你家麼。因却不過廣老師的面皮。纔在這裡忍受。君子絕交。不出惡聲。你滿嘴噴的是甚麼糞。因大笑道。仰天大笑出門去。我輩豈是蓬蒿人。恨道。畜生畜生。殺才殺才。忿然去了。李三子向他老子道。爺聽見沒有。他罵爺畜生。還說殺來殺來。還要來殺爺呢。李太愈怒道。他想殺我。你們跟了我去殺了他。纔除得這恨。就叫人鞲(備)馬拿腰刀來。那滑稽聽得。忙來勸止。他那裡肯聽。急得暴跳如雷。嘴中的白沫都泛了出來。滑稽暗叫人上去忙對滑氏說了。滑氏叫人下來請他上去。說道。皇帝老兒人背地下還要說長道短呢。他罵你。你親耳朶聽見了麼。你信孫子們胡說。就要去殺他。他一個窮秀才你同他拚甚麼。這殺了他。你不償命的麼。況這南京的秀才有幾千。他們要齊了心。可就是西遊記上說的。男人們到了女兒國。一個人掐一下。就只剩個骷髏了。我說的是好話。快不許去胡做。不然我就了不得。你不要疑惑我心疼那先生。我却是爲你的好意\footnote{妙。此等蠢物。不得不分剖明白與他聽。}。那李太見夫人說了。不敢不遵。忍了一口暗氣。他一肚子的騾子肉因氣一裹。不能消尅。漸漸飮食不下。成了噎食。百般醫治不能痊可。他一日睡着。總不見醒。滑氏心疑。上前摸了一摸。手足冰冷。只口中微有溫氣。不住墮淚。坐在傍邊守着。到了三鼓。聽他連嘆了幾口氣。道。悔遲了。悔遲了。滑氏忙問他。他也不答。只兩目直視。淚下如雨。過了半晌。叫把兒子媳婦孫子都叫到面前。道。我纔到陰司去來。閻王怪我疑老子。不孝。待先生無禮。拿糞淸灌了我好幾碗\footnote{果如所言。世間之人該灌糞淸者大半矣。}。哭道。暫放我回來說與你們知道。勸世人不要像我。都要孝敬父母。尊敬師長。我這去。聽得說還要變隻夯狗\footnote{何必要變狗。何嘗是人來。}。日日要囔糞的呢\footnote{今生糞噴多了。後世囔些也該。}。好苦呵。哭了幾聲。做狗噑而死\footnote{在生噑了一輩子。臨死還要噑。趣甚。}。他妻子少不得裝臉(殮)搬喪回家。他老子見了也不哭。也不問他因何而死。心懷前恨。但罵道。這奴才死遲了。此時李得用見主人已死。他囊中已厚了。又恐當日假書的事或有人洩漏與老主知道。不能免罪。他帶着老婆兒子已桃(逃)之夭夭了。過後衆家人方把李得用帶假信並後來請先生的這些話。吿訴了李之富。李之富倒反慟哭道。我那不通的兒囉\footnote{世上人家不通的兒多極。老子也哭不得許多。}。你聽奴才的假書。疑我老子。又聽孫子的讒言。罵逐先生。你死何足惜。但苦我老年人將來入土。不見貴兒子。只有些壞孫子了。後來不知他家下落。亦不復再贅。再說那干生自李太家出來。逕到廣敎官處。將前事說了。廣敎官自愧不該薦他這館。再三自認不是。干生竟毫不介懷。付之一笑而已。鍾趨知他貧寒。久矣萌悔親之念。他兩個賢郞鍾吾仁鍾吾義又常力勸父親道。古云相女配夫。我家雖不算大富。也還是有碗飯吃的人家。妹子甚麼豪門巨族嫁不得。爲何配他一個窮酸。雖然說當年曾指腹爲婚。那不過是兒戲的事。如何做得準。鍾趨原有此心。又聽兩個兒子這一番話。遂拿定主意反悔。因聽得他在李都督家坐館。尚不敢造次。今聞得他賓主不合出來了。料道他力不能娶。算計了一番。先不好就露其意。恐親友談論\footnote{人初起壞念未嘗不有些良心。一過後便喪盡矣。}。一面托人來催他行聘迎娶。一行(面)又出一個難題目。要多少頭面。要多少尺頭。多少羊酒。多少果餅。不然如何進得我家的門。干生聽了這話。笑道。旣然如此。等我有僥倖之時。然後再議。那人復了鍾趨。鍾趨便發話道。放他的狗屁。他若一百年不得中。我女兒留一百年不成。他旣不能娶。他若情願退婚。叫我女兒另嫁。我還與他幾兩銀子度日。那人又來會干生。就直言拜上。干生大笑道。老殺才見我貧欲悔盟耳。何多言。我豈屑要他分文。竟寫了一張退婚文書與他。鍾趨喜不勝言。干生的業師眞佳訓知道了。大怒。要約些朋友。叫干生遞公呈在學院處吿他。反是干生勸道。老師盛情。門生深感。人生但患不能功名成立耳。何患無妻。以門生嫌他家之女則不可。彼嫌貧棄婿。我就爭來。亦無顏矣。眞佳訓見他志氣可嘉。又平素愛他抱負不凡。便道。賢契旣不屑要他。我有一小女。作賢契之配何如。干生辭謝道。老師雲天高誼。門生銘感五內。但門生今日一貧徹骨。豈敢辱老師門楣。眞佳訓正色道。賢契以鍾趨視我耶\footnote{好先生。不愧爲人之師表。此一語。視鍾趨爲狗彘矣。}。若恐我小女愚陋。不足爲賢契之匹則止。至於其他。我不較也。干生道。蒙老師如此錯愛。門生豈不願爲門下婿。還拜謝道。門生愧無寸絲之聘。奈何。眞佳訓笑道。何必拘此世俗之套。我前得了徽州府祁門縣敎官。數日內就要起身。小女旣許奉箕箒。若帶了去。將來婚娶便費事了。因在袖中取出一封銀子來。道。我適間問一敝友貸得五十金做途費。今以二十兩贈與賢婿。明日就是良辰。我同老妻送小女來。你們完成之後。我也就要起程。但事在倉卒。小女的妝奩絲毫未備。寒家所有者皆送了來。餘俟後補\footnote{雖是好丈人。却是好父親。雖疼愛女婿。正是疼愛女兒。眞佳訓不但眞會做先生。且眞會做岳丈。}。干生見他這樣一片熱腸。惟有再三稱謝而已。眞佳訓回去。只與老妻說了。連女兒也不說知。次日只說親戚家請餞行。叫了三頂轎子。竟送到干家來。干生也備了桌酒款待岳父岳母。他老夫妻看着女兒女婿合了巹。抵暮回家。他是要上任去的。將家中所有器皿什物盡行贈了女兒女婿。孟夫子云。女子生而願爲之有家。他那令愛在閨中待字。信都不知。忽然間得了個女婿。大約也沒有甚麼抱怨父母處。他見干生相貌魁吾(梧)。胸懷磊落。干生旣感岳父高情。又見新人態美。夫妻甚是相敬相愛。那眞佳訓把他的那間書室典與鍾生。所得典價十兩。也贈與女婿爲讀書燈火之費。數日內也就上任去了。鍾趨自得了那張退婚文書。先還恐有後話。過了幾日。聽得眞敎官把女兒嫁與他了。遂放了心\footnote{不但放心。再無不笑眞敎官呆者。}。托媒人妻(要)尋個富貴女婿。誰知他嫌貧棄婿的這個美名傳出。那正經人家都鄙他爲人。誰還肯要他的女兒。因循了幾年。他女兒年已二十五歲。恰逢勞正因寶姑死了要續絃。媒人說起鍾趨的女兒生得甚是標致。但只是年紀太大些。勞正也是將三十歲的人。這女子年紀尚還小着兩歲。這有何礙。就煩人去求親。鍾趨聽得是御史公的公子。求之不得。兩個兒子又十分慫慂。因圖奉承豪婿。賠了有千金妝奩嫁與他\footnote{世人因自己豪富而嫌貧棄婿者。不知是何肺腸。即如鍾趨因干生之貧而棄之。却賠千金嫁女于勞宅。若以此千金贈干生。利(則)不爲貧矣。歸之以女。豈不爲慈父賢岳。奈何溺於勢利場中而不悟。惜哉。}。勞正迎娶過門。成親之夕。不但貌美。而且果是處子。不勝恩愛。誰知後來事敗。魏璫磔後株連。勞御史是他二等用事。黨逆人犯。本身伏法。妻子一家發陝西邊衛充軍。連鍾趨的乃愛。也同着鐵甲將軍去了。干生同鍾生同年中了舉。次年又同中了進士。做了一任知縣。行取後又做了推官。鍾趨悔恨無及。把女兒的一位推官奶奶白撂掉了。還去做了軍妻\footnote{可惜他死早了。不曾見他令愛後來做澤國公的權夫人。}。李自成在陝西猖㙭(獗)。音信杳無。死活存亡都不知道。他每每欲自抉其目。以恨不識人。還被親友在背後不知笑罵了多少。因此抱恨成了蠱脹而亡。這是後話。且說這干生住處與賈文物相近。賈文物因有個假文名在外。人見他又是科甲。或有求他作詩的。求他作文的。他又不好推辭不會。自己却又弄不來。他與干生自幼相識。知道他有些才學。時常請他來代庖。這日因要作盟文。故又去請他。一見他來。大喜道。弟候久了。忙迎着讓坐。也不暇敍寒溫。就把宦公子要結盟幷要作一篇文。故請他來代筆的話。說了一遍。隨自己斟了一杯茶送過去。即將筆遞上。將紙鋪下。干不驕與賈文物因同里巷。素常又杯酒往來。賈文物因常要求他。每遇節令定有些食物餽送。又常送些柴米。干生雖推辭不受。賈文物決定不肯。干生因見他情意諄切。只得笑納。今日他請了來。見他一番慇懃。十分奉承。況只要代作幾句盟友(文)。又甚是易事。雖知他與宦蕚童自大結盟。不過是膏粱子弟。狐群狗黨。一夥酒肉之朋。信筆作了一篇譏誚戲謔的話。作完。隨又將黃紙謄淸。遞與賈文物。賈文物看了一遍。贊道。非長兄大才。何以得此。替小弟生輝多矣。留他小飮了幾杯。干生辭別。賈文物深深作揖道謝。送他出門而去\footnote{賈文物見人說話無一不文。惟見了干生。半個文字也不敢說。不但是小巫見了大巫。正是他純是以做文欺局外之人也。}。回到內室。富氏問道。你今日往那裡去的。此時纔回來。又請那姓干的寫甚麼。賈文物鞠躬道。有政故晏也。予久矣升堂矣。未入於室耳。富氏怒道。你向別人文縐縐的罷了。在我跟前也是如此。問着話。不明白說。甚麼叫做有政晏也。賈文物道。予豈多文哉。久假而不知其非也。幸恕之。富氏反笑起來。道。我看你眞是迂夫子。倒埋着文屁沖天\footnote{的評。}。到底是甚麼事。說來我聽。賈文物道。有一宦公子。居氣養體。大哉居乎。翩翩之佳公子也。欲與拙夫同氣相求。爲朋友共。其臭如蘭。故歸來不覺日之夕矣。富氏道。啐。你嚼蛆。便上床脫衣而睡。賈文物也便上床。臥了片刻。爬起來。道。不孝有三。無後爲大。不吿而娶爲無後也。況男女居室乎。奶奶雖未學養子而嫁。我拙夫恐廢人之大倫。不敢不勉請搗之。富氏也不理他。他將富氏放得睡正了。他站起。向陰門深深一恭。道。得罪了。予日日新。又日新矣。然後爬上肚皮。雲雨起來。斯斯文文。慢慢一下一下的抽扯。富氏急得叫道。你到這個要緊的時候。怎還這樣慢條斯理的。賈文物道。好勇鬪狠。以危父母。不孝也。況古云。文質彬彬。然後君子乎。富氏怒道。你旣然做這麼個樣子。你掙這個命做甚麼。賈文物道。此孝當竭力。忠則盡命之時。況與夫人交。敢不興乎。不能也。非不爲也。頃刻氣喘吁吁。伏於枕上。富氏道。你怎麼越發不動了。賈文物道。吾了矣。不能動也。非敢住也。力不進也。富氏又恨又怒。將他一搡。跌下身來睡倒。嘆道。血氣方剛。戒之在鬪。而今而後。吾知免夫小子。富氏聽得恨極了。下力將他掙(擰)了幾把。他道。身體髮膚。受之父母。不敢毀傷。夫人不自苦。然而我苦之。何若是乎擰之之也。富氏恨恨而睡。一宿晚景已過。次早賈文物起來。梳洗穿衣。袖了盟文。坐轎往宦家來。進到園中。童鄔二人早已在彼。宦蕚迎着問道。兄的文曾作了麼。賈文物道。予歸而來之有餘師。焉得無\footnote{這一句文袋掉得是實。}。遂在袖中取出遞過。宦蕚接了。打開叫鄔合念。大家上前同聽他念道。

\begin{quotation}

維南贍部洲大明國南京應天府居住信官宦蕚賈文物童自大。謹以烏豬白羊。香花紙燭。致獻於天地三界十方萬靈眞宰。初封三界伏魔大帝神威遠鎭天尊關聖帝君之前曰。

\end{quotation}

宦蕚道。這信官兩個字下得妥當之極。好想頭。鄔合道。就是烏豬白羊四個字也對得工得緊。童自大道。寫上關老爺眞好。我見人家結拜都寫上他老人家的。鄔合又念道。

\begin{quotation}

某等向係異姓。今結同盟。只願同年同日生。不願同年同日死。

\end{quotation}

鄔合道。這生死兩個字轉換轉換。多了許多學問。不是賈老爺這樣名公。誰能想得到此。童自大道。這兩句話原是古人不通。如今人家的親弟兄爲幾個錢還像生死寃家\footnote{乍看似呆話。細思之。眞至言也。}。況結拜的酒肉弟兄。不過圖些東西肥嘴\footnote{近之結盟。不過爲此。}。無原無故。同起甚麼生死來。這樣沒道理的胡話豈不可笑。宦蕚道。果然。你這話說得有理之極。向鄔合道。你再念。他念道。

\begin{quotation}

自今設誓之後。某等三人輪流做主。或以酒肉開筵。或向煙花訪妓。倘負斯盟。人神共殛。

\end{quotation}

童自大伸了伸舌頭。道。旣這樣說。你把我的名字摳掉罷。我是不來的了。宦蕚道。旣已講定。爲何又變起卦來了。童自大道。賈兄是個送人的棺材座子。他同我頑呢。他上頭說輪流做東。我如何來得起。我一個經紀人家。那裡經得這等大費。若是我家奶奶知道了。我這條打板算就送在你們手裡了。賈文物道。送爲賓主禮也。旣如此說。你竟二而一。我們一而二。何如。童自大搖頭道。也做不來。我前日聽見個人念書。甚麼二十而取一。就依着書上說。你每位當十回我當一回罷。宦蕚道。太無此理。我們兩個當十回東擾你一回。何如。他聽了纔不做聲。鄔合道。二位老爺請聽着念完了罷。又念道。

\begin{quotation}

某等今日富貴相當。故結弟兄之社。他年豪華不敵。定散手足之盟。上吿蒼穹。願鑒同志。天啓 年 月 日謹疏。

\end{quotation}

讀畢。童自大道。一篇文我只喜這兩句。鄔合道。通篇都是妙的。如何只說這兩句好。童自大道。他說有錢相聚。無錢散夥。可不妙哉乎也。我因二位哥有錢勢纔來拜把子。若是兩位兄倒了運。我還同你作甚弟兄。同胞骨肉尚如此。何況區區酒肉盟\footnote{朋友已是五倫中之一。果能敦友道。患難死生可以相共。何待結盟而始原也。近之結盟者。皆不過是酒肉社。特美其名爲結盟耳。昔人曾有兩句道。最好笑的世情。朋友們結盟。童自大這幾句話。與之持合今人。多少譏貶。多少傷心。孰謂之呆哉。}。宦蕚對賈文物道。人不可不弄個進士做。賢弟這篇文都是我心眼兒裡的話。却說不出來。都被你說出來了。眞不愧才子二字\footnote{宦蕚這幾句話初看不覺。細思之。眞不通到趣極。他並不知進士是因自文才而得。以爲中了進士自然有才。匪夷所思。令人笑倒。}。賈文物道。愚弟此文乃雞鳴而起。孳孳爲之者。雖小套。必有可觀者焉。說話間。衆家人已將各項擺列停當。叫鄔合念盟文。他三人焚香歃血畢。然後交拜過。擺上酒來。大家散福痛飮。狂呼哥哥弟弟。眞比親手足還覺親熱。有幾句道他三人道。

\begin{quotation}

臭味相投。同盟共好。弟弟兄兄。酒肴列遶。若問義氣有無。這却不能分曉。

\end{quotation}

飮到更闌。方纔分手。宦蕚回到房中。侯氏問道。你今日前邊殺豬宰羊做甚麼事。宦蕚將同賈童結拜的話說了。侯氏道。我同你夫妻多年。不見你一些親熱。每日歇店也似的。晚上進來睡一覺。淸早就鑽了出去。成日在外邊不知做些甚事。又同外人結拜甚麼弟兄。可不是親的倒疏。疏的倒親了\footnote{此類人多甚。}。宦蕚道。我豈不要親熱你。只是見了你努(怒)目金剛似的那一種相貌。一點喜容也沒有。我的魂都不在身上。怕還怕不過來。還敢來同你親熱呢。侯氏此時偶然有些高興。正想他來親熱親熱。遂密縫着兩隻紅眼。齜着嘴\footnote{要是我。更害忙(怕)。}。故做嘻嘻的笑道。我如今這個喜笑的面龐。難道你還怕麼。看你怎麼個親熱的法見(兒)。宦蕚已有半酣。見他滿面春風。一時膽壯起來。也笑嘻嘻走上前抱住。親了兩個嘴。道。我的娘。你若日日有這個喜容。我便夜夜同你親熱。我同你到床上親熱去。把侯氏抱上床來。替他寬衣褪褲。二人脫得精光。宦蕚腹中雖然不濟。腰中這一副本錢倒甚濟。有一調西江月贊他道。

\begin{quotation}

堅舉長餘六寸。生來能軟能剛。軟如醉漢倒郞當。剛似瘋僧狂樣。出牝入陰本事。腰州臍下家鄕。天生二子在身傍。慣與佳人打仗。

\end{quotation}

那侯氏貌雖不揚。倒好一個陰戶。也有個西江月贈他道。

緊暖香乾俱備。光光滑滑堪憐。有時吐舌笑開顏。睏便懶張兩片。淸水池邊故土。褲襠縣裡家園。有時忽動興緯綿。戰鬪千回不倦\footnote{前贊宦蕚鄔合是兩調西江月。此處贊他二人也是兩調西江月。前後一映。}。

他兩個一時弄將起來。只見。

\begin{quotation}

一個兩足高蹺。一個單鎗直刺。一個柳腰款擺。一個玉杆(杵)忙舂。一個笑吟吟把腰肢緊摟。一個喜孜孜將兩股頻搖。這一個面似火燒。那一個舌如冰冷。一個喉內哼哼。如小兒睡夢頻啼。一個鼻中喘喘。似老牛耕田力乏。下一個矇矇星眼。心窩內樂極魂飛。上一個汗流浹背。遍身中酥麻精洩。

\end{quotation}

幹夠多時。雲收雨散。那侯氏得了這一番樂趣。也與每常大不相同。二人四臂交加。兩胸相貼。眞個親親熱熱睡了一夜。此後侯氏圖他這種親熱。也就常與他個笑臉。宦蕚也就漸漸膽子略壯了些。雖不敢犯他的法度。也不似先那樣畏縮了。且說那鍾生一日在梅生家會文。作完之後。互相評論了一番。鍾生見案頭有一册手抄。便拿過來翻閱。梅生道。這是個姓郭的敝友。他與點(黔)寧侯沐國公有些瓜葛。往雲南去相探。沐公留他住了年餘。他將滇中風景作了三十餘首竹枝詞。昨日回來。他送來與弟看。雖不爲佳。然而看看。知那地方的風俗。不無開卷有益。鍾生翻開看道。

\begin{quotation}

朱樓繡戶鬪年光。綵勝新花八寶妝。上客登堂來拜歲。金盤十只送檳榔。

三冬雷雨兩交加。但到立春桃已花。正月盡頭梅子大。嘗新二月有黃瓜。

簾外春風初淡蕩。梁頭燕語已呢喃。獨有鴻飛曾不到。長空耿氣鎖煙嵐。

花朝時節女成行。攜榼城東坐小莊。石子爭拈打石臼。中時應產好兒耶。

楊花歷亂下秋千。趁着淸明無雨天。金汁河邊桃李陌。稠人堆裡狡風鳶。

頭上靑梭布一幅。防風行動手親扶。歸來不見新娘面。嚼碎檳榔罵濫奴。

柳葉桃花日夜開。靑樓小妓踏歌回。閒情解釋愁多少。帶得春風滿面來。

一隻金釵十萬貲。霍家小玉傾城姿。好花纔吐新鶯滑。妒殺姝姝打棗詞。

圓通勝蹟小蓬萊。樓觀金銀厓上開。磴道盤空直到頂。可憐羅襪半塵埃。

肉身金像古庭龕。銅殿新修鸚鵡灘。出門試請朝東看。山頭坐破女和男。

夏木千章祈雨壇。鳥龍潭繞碧欄杆。神魚隊隊皆龍種。誰敢吟風下釣竿。

金馬山前金馬寺。碧雞關外碧雞祠。王褒祀後南雲嘆。猶道昆明鑿漢時。

大理黑龍憶白龍。傳聞人說是雌雄。如今一歲一相見。飛雹寒冰帶滿空。

白塔街前嶽廟開。血池賺得婦人來。半空蝴蝶飛灰盡。獨坐西廊苦不回。

蜀梁自古產銅山。九府官開寶貨泉。一月一緡收子母。人人爭放排〖貝巴〗錢\footnote{滇中皆以海〖貝巴〗貿易。至今呼錢猶曰〖貝巴〗兒。}。

小兒好事日千端。甘蔗性寒梅子酸。買得燒鵝還未請。索錢又換米花團\footnote{滇中小兒謂炒煳蠶豆爲燒鵝也。}。

〖口么〗〖口么〗喝喝百般腔。魚市街連羊市長。聽去綿蠻渾不解。螺螄䐗兒螺螄黃\footnote{滇中螺螄甚大。賣者分頭䐗黃三等。}。

雲焼星回六月天。食生人競共嘗鮮。不知五詔同焚死。直似〖月麗〗(驪)山舉火年\footnote{六月二十四日爲火把節。土人皆食生肉。}。

矗空兩塔望巍巍。西寺人從東寺歸。崢嶸五百阿羅漢。一時齊着錦襴衣。

太華山上白雲秋。太華山下水長流。彈詞唱罷歷朝事。不見當年楊用修。

晏公海口混茫茫。昆明池水接昆陽。舟船何事行深夜。白日風波不可當。

鐘聲鳴咽梵王秋。歸化千年大路頭。莫道西南通漢使。滇池不肯向東流。

誰家少婦挽雙鬟。拜掃淸明哭百蠻。自道良人中國子。可憐死葬梁王山。

白日狂飆十丈高。拔山盪海怒奔號。勞勞亭外重關道。劈面塵沙無處逃。

寶石陸離出永昌。黃金照耀產麗江。傾囊猶恐公家罪。百姓何人敢自藏。

近城風脈祖墳山。盡日堪輿馬上看。俱道來龍埋處好。不知何代始高官。

進耳山中祈夢人。事誇一夢覺先因。不知人事渾皆夢。獨自殷勤夜問神。

高樹茶花如火屯。千紅萬紫似兒孫。三春景色眞眞好。一片花聲賣過門。

二忠木上照滇雲。太史聲名動海濱。生滴(謫)死歸皆是義。南中稱有此雙仁。

黔寧開第五華東。珠樹繁花照眼紅。鸚鵡西飛芳草暮。桂枝獨自唱春風。

玉樹後庭花已殘。梁王山下鳥飛寒。民間不解傷心事。一夜月明打棗竿。

\end{quotation}

看完了。梅生又留鍾生小飮了數杯。鍾生見日色將暮。作別歸家。正走時。紛紛落下雨來。正無處躱避。遙見一個菜園中搭着一個蓆棚。係種園之人午間陰涼之所。只得急走到底下暫避。不想一陣陣只管大下起來。竟如瓢傾一般。頃刻間。平地水深數寸。一個聚水灌園的塘子都漲滿了。幸得這個棚上豆葉遮滿。又在一棵大槐樹之下。雖然身上略沾濕了些。還不至十分狼狽。等到將起更時分。淙淙猶尚未止。鍾生因離家尚遠。泥濘難行。且又下個不住。到一更之後。雨纔止了。黑雲中微微有些月光。此時雖然晴了。却夜深歸去不得。心中好生着急。忽隱隱聽得有哭泣之聲。朦朧月下四處一望。恍恍惚惚見水塘邊有個人影。哭聲雖不高。却甚是悲切。像有個投水之意。鍾生悄步走近前去。原來是個婦人。那婦人哭着。不曾看見。聽得脚步響。忽回頭一看。見有人來。忙攛入水中。鍾生眼疾。見婦人下水。趕上一步。一把拉住衣服。儘力拖了上來。那婦人還往下掙。鍾生顧不得嫌疑。也不惜泥汚了自己的衣服。拉住他膀子。道。你是誰家宅眷。有甚麼寃苦的事。尋此短見。那婦人掙不脫。只是嗚嗚的哭。鍾生道。你有甚麼萬不得已的事。何妨吿訴我。我或者可以救得你也不可知。你家住在那裡。那婦人方住了哭。指着個小門兒。道。那就是我家的後門。此時婦人自頭至足。渾身都是泥水。鍾生用力扶起他來。道。你且請回去。萬不可如此。那婦人微亮之下見鍾生儒巾儒服。是個讀書人的樣子。又哭着道。相公。你救我也無益。我始終是不能活的。倒不如趁這深深的水。讓我死了罷。鍾生道。我不見就罷了。可有見而不救之理。且回去有話說了。我若力量可行。定然相救。那婦人見他苦勸。只得回家。鍾生也隨在後面。那婦人一身拖泥帶水沈重了。地下泥深路滑。他鞋弓足小。一步一跌。鍾生看得心中過不去。只得上去扶着他走。婦人怕又滑倒。將兩隻手把鍾生肩膀緊緊扳住。把個鍾生也弄了一身泥水。扶他到了房內。你道鍾生一個讀書人。豈肯夤夜到一個孤身婦人室中。因恐無人。他又去尋死。豈不辜了救他的一片熱腸。二來要問他詳細。有可救他處。好設法相援。做個救人救徹之意。到了房中。燈火也沒有。月又不明。黑魆魆伸掌不見。那婦人摸了條板凳讓鍾生坐下。他在床沿上坐着。那婦人一身雖然濕透。幸得七月初頭。天氣正熱。鍾生問他投水的緣故。丈夫何在。他重新哭起來。道。我姓郗。我丈夫姓充。名好古。當日也是好人家子孫。因不成器。成日在外拐騙小官。做那下流的事。把個小小家業都花盡了。如今手頭沒錢。舊日相厚的那些人都撇開了他。他還不死心。三日前又引了個小夥兒到家中來。說到這裡。越哭得悲慟。鍾生道。不用傷心。你說完了再做商議。婦人止住哭。含羞道。他因沒錢與那小夥子。要叫我同那小夥子睡。他捷他的屁股。我也是好人家兒女。怎肯做這樣無恥下流的事。被我同他大鬧了一場。他賭氣出去。三日不歸。家中當賣俱無。柴米油鹽一樣沒有。大長的天氣。我整整餓了三日。米星兒也沒有沾牙。相公請想。我這樣苦命還活着做甚麼。螻蟻尚且貪生。我難道就不愛命。我餓得受不得了。纔去投水。先要上吊。又下不得手。想着深深的水往下一跳就罷了。不想又遇着相公救起我來。我也想來。嫁了這樣不成材的丈夫。他圖風流快樂。妻子餓着都不管。我就做些不長進的事。他也怨不得。相好個正經人也還罷了。怎肯把身子同兔子小廝去睡。婦人的這幾句話來得有意。他雖黑影裡未見鍾生容貌。見他文文雅雅。是個正經人。又有救他的這番好情。且又不顧泥汚。竭力扶持。又還說要救他。大凡人猛性尋死。死了就罷了。被人救轉。誰不惜命。這郗氏不但要捨身報他相救之恩。且有個要結交他。圖他照顧之意\footnote{非寫郗氏一段貞性忽爾變爲淫心。此乃是他一片報恩之念。因今日不曾捨身以報。故後日念念不忘。終必報之也。}。鍾生是個誠實君子。那裡認他話頭。便問他道。你難道沒有父母兄弟麼。郗氏道。要有父母倒好了。只有個哥哥。嫂子前年又死了。也是個孤身。見妹夫不成人。也嚷鬧過幾回。不大上門。他往外邊做生意去了。原說八月裡纔回來。鍾生道。事也好處。你不必胡思亂想\footnote{這一句妙極。鍾生是個聰明人。豈不料郗氏前言之味。今云你不必胡思亂想。淺人看去。謂是不可再尋死了。深味之。暗言切不可因貞而失身也。}。你一個人。一月有一兩銀子就夠將就盤纏了。我雖是個貧士。我明日去替你設處。郗氏道。相公貴姓。我蒙相公這樣大恩。怎麼報答。鍾生道。我賤姓鍾。救人之難。理所當爲。何必講報答的話。說話時。外面又大下起來。鍾生初意說完了話。安撫了婦人。還要到棚下去。不意下得越大。只得閉目凝神坐着。郗氏見鍾生這等好情。心中感他不盡。又想。孤男寡女黑影裡共坐一室。可有不動心之理。恐他先動起手來。反不見了情面。我旣欲以身相酬。不如先去就他\footnote{又寫此數語者。非謂郗氏之淫濫。特更顯鍾生之難得耳。}。遂走近前。道。夜深了。相公不棄。請在床上去睡睡。我在板凳上坐着罷。鍾生道。你請自便。我坐坐好。郗氏見他推辭。只得仍到床沿上坐下。那雨足足下了一夜。他二人也就坐了一夜。鍾生對着那婦人。毫不動念。有四句贊他道。

\begin{quotation}

空房雨夜對嬋娟。正直心腸鐵石堅。

寂寂通宵能遏慾。坐懷端可繼前賢。

\end{quotation}

東方亮了。天色方晴。郗氏把鍾生一看。好個標致少年。心愛無比\footnote{有此一句。相隔數年。故一見即識也。}。起身向鍾生道。泥深路爛。相公怎麼回去。寒家柴也沒一根。茶也沒一鍾敬相公的。鍾生看那郗氏也大有幾分姿色。雖然是裙布荆

釵。却掩不得他的花容月貌。古人有幾句話道。

\begin{quotation}

好好好。不必綾羅襖。靑衫白練裙。好的只是好。

\end{quotation}

還有幾句贊他道。

\begin{quotation}

冰肌藏玉骨。衫領露酥胸。柳眉積翠黛。杏眼閃銀星。月樣儀容俏。天然性格淸。體似燕藏柳。聲如鶯囀林。看他渾身水濕。似帶雨海棠籠曉日。遍體泥淤。如經霜黃菊弄秋晴。雖不及瑤臺仙子。也算個窈窕佳人\footnote{此極贊郗氏之美者。若非有殊也。後來焉能動榮公之目。一見師保愛也。}。

\end{quotation}

這郗氏渾身還是精濕。鍾生答道。顧不得泥濘。我此時回去設處盤費送來。你不可又尋短見了。換換濕衣裳。養息養息。我就來的\footnote{二語足見鍾生相愛之甚。情不敢越禮耳。}。郗氏道。我就是身上這件衫子。可憐那裡還有得換。鍾坐(生)點了點頭。嘆了一聲。拖泥帶水而去。到了家中。將錢貴贈他的銀子稱了三兩。帶了一百文錢。把舊衫舊褲拿了兩件。捲緊籠在袖中。復到郗氏家來。那婦人正倚門盼望。見了他。忙側身讓入。鍾生先把衫褲取出。放在桌子上。道。這兩件舊衣。你將就換換身上。又將銀子遞與他。道。你昨日說令兄八月來家。如今已是七月初了。到八月盡。只兩個月。但出門的人定不得歸期。這是三兩銀子。夠你三個月用度。等你令兄回來。就有接應了。又取了一百文錢與他。道。恐一時沒人與你換錢。你餓了三四日。且買個點心充飢。郗氏見他如此週到。相愛之切。滴了幾點淚。道。相公這樣深情。我無報答之處。若不嫌我醜陋。願以此身相報\footnote{此非謂郗氏之水性。乃贊其受恩必報之堅心。正反槪(襯)世之鬚眉者。今日受人之德。明日即掉臂不顧之流耳。}。鍾生正色道。我一番救你的熱心腸。豈有不肖的念頭。你快不要妄說這話。錯會了主意。郗氏見他說得如此斬截。知道他不是個好色悖禮的人。忙忙拜謝。鍾生也頂禮相還。辭別而回。離家有百步之遙。一家門口站着一個老婦同一個年少婦人在那裡閒望。見了鍾生。那少婦失口贊道。好一位俊俏郞君。有甚麼要緊的事。弄了滿身兩足的汚泥。鍾生〇〇看見。雖然淡妝素服。竟是國色天姿。也有古人的幾句贊他道。

\begin{quotation}

俏俏俏。不用菱花照。淸水淡梳妝。俏的只是俏。

\end{quotation}

鍾生見了。忙低頭而過。只聽得那一個半老婦人道。這就是前面那園子裡住的鍾相公。是個才貌雙全。有名的小秀才\footnote{只離百步之遠。老婦已知鍾生之姓之名的。鍾生反不知其爲何如人。足見他不務外事。閉戶潛修也。}。鍾生到家。換了衣服鞋襪。因一夜無眠。睡了一覺。然後起來讀書。天色晴了。過了兩日。因家中缺少些動用之物。打發那雇的小子上街去買。他獨坐看書。忽聽得敲門甚急。疑是那小子忘了甚麼東西回來取。忙來開門。原來是前日那家門口站着的那美婦。鍾生道。尊駕到這裡來。有何貴幹。那美婦笑着道。我來看看相公的書室。說着。就走了進來。鍾生又不好推他。只得也跟着走入。前日不過瞥見一眼。未曾看明。此時將他一看。却好一個女子\footnote{腐頭巾謂。看人婦女。大損陰德。此迂腐不通之論也。人非瞽目。見美色焉不一看。即如走馬看花。過眼即了。勿介在胸中。有何妨礙。若見了美色。時刻不忘。且又不住口提。則不但損德。乃眞正小人矣。}。有幾句贊語道。

\begin{quotation}

月掛雙眉。霞蒸兩靨。膚凝瑞雪。鬢挽祥雲。輕盈綽約不爲奇。妙在無心入畫。嬝娜端莊。皆可咏絕。非有意成詩。誠哉絕世佳人。允矣出塵仙子\footnote{寫李氏如此美麗。非正筆。特謂如此艷質於無人處來奔。而鍾生毫不動念。眞奇男子耳。}。

\end{quotation}

他到了房中。道。好一間潔淨臥室。眞是瀟灑書齋了。不愧才人所居。鍾生站在窗外。道。男女授受不親。請回罷。恐一時有朋友撞來。見之不雅。那美婦道。相公請進來。妾有心腹之言奉吿。鍾生道。豈不聞瓜田李下之嫌乎。有話但請見敎。我在此聽着是一樣的。那美婦道。妾家姓李。我父親是黌門老儒。我向日爲媒所誤。誤適匪人。先夫桑姓。自不知書。惟以嫖賭爲事。妾今孀居三載。賤庚二十有一。自先夫亡後。妾即歸於母家。我父母公姑憫我年幼無出。叫我改適。我恐又嫁一庸奴。豈不誤了終身。要圖覓一良偶。故爾不敢輕托。晚見相公丯儀出衆。又聞知學富五車。妾私心欣慶。不自揣鄙陋。願侍箕箒。妾此來。非爲淫奔之事。欲以終身相托耳。昨遇相公的那家是我姨父。姓陶。姨母柳氏。係家慈之親妹。今日他老夫妻都往親戚家去了。妾偷空到此。不惜慚顏自媒。相公肯俯允否。鍾生道。多承厚意。但我已定過荆妻了。有辜盛情。不敢從命。那婦人想了一想。又道。我想寧爲讀書郞之妾。不願做賣菜傭之妻。相公旣聘過夫人。願留一小星之位以處我。尊意如何。鍾生道。尊翁旣係前輩先生。你是儒門閨秀。可有與人做妾之理。令尊自然愛女。爲擇佳配。古云。寧爲雞口。勿爲牛後。不要錯想了。恐有人來。快請回步罷。那李氏聽了這話。眞個是。

\begin{quotation}

只道桃源路已通。豈知猶在夢魂中。

靑鳥浪傳雲外信。錯將心事語東風。

\end{quotation}

不覺滴下淚來。道。昨見郞君之後。私心以爲終身有托。不意相公如此拒絕。我亦聞之。寧甘玉碎。不肯瓦全。一生事一誤。寧堪再誤。命薄如斯。我從此投入空門。長齋繡佛。今生不復再嫁矣。掩袂悲啼。鍾生聽他說得慘然。心中着實動憐。想了一想。道。不必傷心。我替你做個伐罷。我有個契友梅兄。今年二十三歲了。相貌瑰異。才學天成。將來必成大器也。前歲斷絃。家頗充足。較勝我多矣。若肯嫁他。必不失所。那李氏道。相公尊諭固是良言。但不知果如相公之說否。鍾生道。承你這一番見愛。我已銘刻肺腑。好色人之所慕。我若不曾聘過。豈不願得你這樣佳人。要說我不相愛。便是矯情之語。我雖有十分憐愛之心。但於禮有萬不可行者\footnote{此數語見鍾生才是眞豪傑。才是眞情種。}。我爲作伐者相報你這種深情耳。豈肯誤你終身之事。李氏聽他說這話。眞出肝膈之言。深深斂袵而拜。鍾生還了一揖。道。我今日就去對梅兄說了。擇日到府奉求。不知令尊府上在那裡住。李氏道。若貴友不鄙寒門。不必遣媒。如不吝玉。就到家姨父處。煩我姨母去說。更爲省事\footnote{有心哉。斯女也欲梅生來。自己偷相耳。}。鍾生道。這更妙了。那婦人喜笑盈腮。欣欣而去。鍾生等了小子回來。就親去到梅生家。不好說這婦人來奔的話。只說昨日偶然看見。眞是麗人。訪問鄰舍。方知姓李。是儒家之女。聞得孀居。纔二十一歲。正在選擇佳婿。弟見吾兄鰥居。特來奉吿。佳人難得。吾兄萬不可錯過。若親去煩他姨母作伐。事在必成。梅生大喜。再三稱謝。次日。備了一分禮。親同鍾生來央陶老夫婦做媒。他老兩口見梅生少年英俊。滿口應允。那李氏暗地偷覷梅生。果然一表非俗。心中私喜。感激鍾生不盡。陶老向李老說了。接了女兒回去。問女兒主意。那李氏自然願意。李老就許了\footnote{錢貴與鍾生。梅生之媒也。廣氏與劉顯。梅生之媒也。成全了兩對好夫妻。今李氏與梅生雖緣陶老說合。實起於鍾生之媒也。亦成全了他一對好夫妻。做良媒者自有好報。世間之媒專誤人家子女。何也。}。梅生擇吉行聘。也甚齊整。選了八月初四日親迎。娶過門來。梅生看那李氏。果然美艷無比。與當年雪氏可相伯仲。李氏也偷眼看梅生。比前番私窺時丯韵更佳。有四句說他兩人道。

\begin{quotation}

郞顏敷粉女容嬌。角枕橫陳粲此宵。

兩兩情投如鼓瑟。千金良夜實難消。

\end{quotation}

他二人這一夜的恩情賽過百年歡好。到了三日之期。請丈人李老。丈母柳氏。姨丈陶老。姨丈母。舅丈李老。舅丈母楊氏。幷桑老夫婦。又有丈人家的親戚桂老栢老多人。到家喜筵。鍾生臨場。不得來赴席。親朋熱鬧了數日。他夫妻如魚似水。深感鍾生這個月老。梅生得了佳偶。竟連場期都不去赴。眞是。

\begin{quotation}

得成比翼何須貴。願做鴛鴦不羨仙。

\end{quotation}

暫且放下。再說那宦蕚賈文物童自大三個自結盟之後。無比親厚。朝聚暮散。十日有七八日在宦家。有兩三日在賈文物處。他們知道童自大吝嗇。總不到他家去\footnote{眞好弟兄。童自大當云。生我者父母。知我者二兄也。}。一日。又在宦蕚家中來。要知在何處共坐。做些甚事。〇〇下回分剖。\endnotemark[2]

姑妄言九卷終



\endnotetext[1]{此段原有眉批註云:「說:山東土音,念做上聲,五音皆無字可釋。舍:沙去聲。」}

\endnotetext[2]{卷末空白處原有「較了」二字,今刪。}

\setcounter{footnote}{0}

\theendnotes

\part*{姑妄言第十卷}
\addcontentsline{toc}{part}{姑妄言第十卷}
\markboth{姑妄言第十卷}{姑妄言第十卷}

鈍翁曰。宦賈童三人雖爲同類。然氣質各別。一個人是一個身段。一番談吐。毫無相似。宦蕚之呆也狂。賈文物之呆也假。童自大之呆也則眞呆矣。即鄔合之奉承三人亦是三等。宦蕚爲重。賈文物次之。童自大爲輕。可見利字又遜勢字一頭。看他三人說頑說笑。純然一夥不經世事膏粱癡頑子弟。

宦賈童之遇錢貴。乃鍾生錢貴之幸也。如錢貴不遇他三人。不顯錢貴之貞。不見鍾生之雅。作者之筆。正如畫石畫三面之法。

游混公幹子後庭。雖是楊爲英之惡計。然而世上酷好龍陽之人。皆當以此法處之。

前數回內雖夾寫游混公之不堪。尚未見其不堪之所以然處。這一回內是他的小傳。細閱之。不但不堪。而且不堪之至。

後半冊極力寫多銀之淫賤。游夏流之下流。借子罵父。游混公卜通輩自思之。料亦無辭可解。

或有迂叟見游夏流一事。必勃然曰。有是哉。此物奚可舔哉。彼不知借這一個下流。罵盡古今多少下流也。有勢之股旣可舔。多銀之陰獨不可舔耶。勢與利等耳。多銀之陰猶可鮝魚香。恐有勢之股純乎狗屎臭也。且游夏流舔這妻子之陰。尚有曖昧。他人彰明較著。竟舔外人之股。以此較之。游夏流尚高一籌。

\chapter*{姑妄言卷之十\\
第十回 狂且乘狂興憶高官 美妓具美心譏俗客\\
附 卜氏女奇淫出奇思 游家兒妙舔眞妙想}
\addcontentsline{toc}{chapter}{第十回 狂且乘狂興憶高官 美妓具美心譏俗客}
\markboth{第十回 狂且乘狂興憶高官 美妓具美心譏俗客}{第十回 狂且乘狂興憶高官 美妓具美心譏俗客}

話說宦賈童三人自結盟之後。終日相聚。比同胞兄弟還覺親熱幾分\footnote{此所謂小人之交甘如醴也。}。一日。同在宦蕚家中斐園內一個吞萍閣上乘涼。你道何爲吞萍閣。這是夏天避暑的一座涼廳。四圍是水。此閣在內獨峙。塘沿四週都是參天垂柳。遮得那閣上一〖阝少日小〗日光皆無。水內荇藻鋪滿。那龜鼈魚蝦往來游戲不絕。皆浮於水面。吞吐浮萍。景甚可觀。故此取名爲吞萍。他衆人坐在閣上。散髮披襟。呼盧痛飮了一會。宦蕚道。我們只是這樣蠻吃。一點趣味也沒有。不若大家淸談淸談。還覺快活些。鄔合道。大老爺若發一言。出一想。就都絕妙。淸談高雅。可是俗人能及。眞高出尋常萬倍。童自大道。鄔哥。你好搊。你拿花盆兒給哥頂呢。據我說。說那鬼話不過聽得耳朶快活。不如吃酒吃菜。嘴同肚子兩處快活。倒不好麼。賈文物道。賢弟失矣。子貢方人。夫子但曰。夫我則不暇。何面叱鄔兄之短。而負惡訐以爲直者之名乎。童自大道。我也是同鄔哥頑呢。不消多講。就依着哥說鬼話罷。宦蕚道。我們如〇〇〇〇〇笑話兒頑耍。要有親眼見的更妙。不然就是〇〇〇〇〇〇〇罷。說得不好的罰一杯。賈文物道。妙矣〇〇〇〇〇〇〇〇宦蕚道。我前年在京中的時候。有門下走〇〇〇〇〇〇〇〇〇永平府去有事。去了些日子回來。他說〇〇〇〇〇〇〇〇歲的一個漢子趕着一輛車車。上坐着一〇〇〇〇〇〇〇十來歲。生得很好。就是這個漢子的老婆。有一個標致小夥子。也纔二十多歲。前前後後。總不離那車。同那婦人眉來眼去的調情。二和尚覺得有些古怪。留心冷眼看他\footnote{非和尚決不如此留心。}。或是那漢子略離遠些。他兩個就打牙犯嘴。說頑說笑。午間打中火。也定在一處鋪子裡吃飯。晚上也同在一個店裡歇。北邊的店比不得我們南邊。一間一間的都是敞着的多。那一晚歇了店。二和尚也在這個店裡。是對面兩鋪炕。這個婦人靠着牆睡。他漢子挨着他。一個白鬍子老頭子也在那炕頭上。別的人因有小媳婦子在那炕上。都擠在這邊一炕睡。二和尚就挨着這小夥子在一處。夜裡那婦人的漢子起去上馬草料。這小夥子忙跳下炕。鑽在那婦人被裡去了。一會聽得那漢子要進來了。他忙又跑了回來睡下。衆人都醒着。誰肯管這閒事。那漢子剛睡下。想是摸着了那婦人的下身。不知怎樣的。忙坐起來。叫道。不好。有了壞人了。一屋子的人。不知是那一個。他疑是同炕睡的那個老兒。他下炕舀了一瓢涼水。推那老兒。道。起來喝水。那老兒睡在熱炕頭上正在發渴。接過來。就一氣喝完了。那漢子沒得說。也就睡了。天亮時。那漢子同婦人先去了。衆人也都起來。這小夥子向那老兒作揖。道。多謝太爺替我喝那一瓢水。那老兒笑道。我的哥。是你老嗎。我要知道是你。還替你喝兩瓢。把一店的人都大笑起來。這豈不是個眞笑話。童自大笑道。這想就是二和尚做的事罷。他不好說是自己。推在別人身上\footnote{他這想豈但不呆。而且乖甚。}。賈文物點頭道。有理哉。賢弟之言如見其肺肝然矣。我有目覩之一事焉。前偶到鍾山之上去玩。觀象之台有四五婦人焉。亦在其上。憩於山之麓。其同行之男子皆四散而游之。突有一壯年之狂且至諸婦之前。解其裩而出。其厥物大而且剛。置之於石上。奮拳以搥之。諸婦有赧而避者。有嘻而笑者。疾呼其男子而擒之。及衆人趨至之時。此狂且則自後山而奔矣。豈不亦可笑乎。鄔合道。晚生也眼見一個笑話。旱西門大街上住的康爸爸。他是個財主。那一日他家大約有甚麼喜事。有七八個女孩子。大的不過十四五歲。小的也有十二三。都打扮得齊齊整整。在門口站着說笑。一個老頭子有七十多歲了。手裡拿着個筐子遠遠站着。兩隻眼睛定定的看了一會。忽然跑上去抱着一個大女孩子。一連親了幾個嘴。脖子上腮頰上一陣混咬。把那女孩子嚇得亂叫。別的跌跌滾滾往裡跑。他家男子們聽見。跑了出來。看見那個老兒還抱住不肯放。衆人打了一頓。見他有年紀。不敢狠打。拉到上元縣稟了官。官也見他老了。薄責十五板。打完了。那老頭子跪稟道。蒙老爺天恩賞責。小的却寃屈得很。縣裡老爺大怒道。你這老奴才這樣可惡。做出這等事來。本當重處的。姑念你年老。薄責示罰。還說本縣寃枉了你。那老頭子叩了個頭。道。小的活了這樣大年紀。難道王法都不知道。敢去做這樣的事。不知怎樣。一時看昏了。跑了去抱着親嘴。小的自己並不知道。後來衆人拿住了打。小的才醒過來。方知是錯。小的說的是這個寃枉。那裡敢說老爺。那縣裡老爺倒反大笑。命攆了出來。這樣事豈不是個眞笑話。童自大笑道。這看昏了的事你當假麼。我就幹過一回。吃了一個大虧。宦蕚向他道。賢弟也說一個。童自大道。我也沒有聽見過。也沒有看見過。沒得說。就說我自己看昏了的這個笑話罷。我家奶奶的一個丫頭叫做仙桃。生得好不標致。那一日我無心看了他一看。他望着我一笑。我從頭頂心上一酥就到脚底板上。便昏了過去。被我家奶奶看見了。拿擔箒把兒好打。把我光脖子上打了十來多下。幾乎把脖梁骨打斷了。即刻把丫頭賣掉。你說這事寃枉不寃枉。好笑不好笑。衆人聽了。倒大笑了一回。童自大見賈文物眼有些眊。笑着向他道。我聽見人說一個眊子的笑話。我說與哥聽。哥不要惱。賈文物道。無傷也。是乃笑話也。何以惱爲。童自大道。哥不惱。我就說了。

\begin{quotation}

一個人專好弄屁股。同他老婆高興。十回倒有七八回弄後頭。他老婆說。你旣這樣愛他。該替他起個名字。那男人說。這個眼子極有趣。就叫他做趣眼罷。他老婆又指着陰門。道。這個東西你也間或還用他。也該起個名字。男人說。他同趣眼相近。就叫他做近趣眼。

\end{quotation}

宦蕚大笑。賈文物見童自大傷了他。因看他有些呆氣。便道。我也有一笑談。說與諸位聽之。

\begin{quotation}

一男子呆人也。其妻陰戶之內生其瘡焉。呼其夫而吿之曰。我此物之內癢痛不可忍也。子可呼醫而治之。厥夫延醫至。命婦人裸而視之。吿其患。醫曰。此非湯丸力所能及。當以殺癢止痛之末藥敷於龜頭之上。送入癢痛之處而擦之即癒矣。其呆夫曰。我不知病在何所。汝醫也。可自行之。醫聞而喜甚。即以藥用唾調之敷其龜。送入其妻之陰。來往抽拽不止。呆夫大詫曰。汝擦藥耳。何故動之不休。醫曰。龜頭無目者也。安能入便見其病之處。須探得要害處而後可擦。來回抽拽愈急。其妻樂甚。連呼曰。好太醫。好太醫。其醫亦樂極而洩。伏於婦人之腹。大叫曰。吾得其病處矣。呆夫在傍注視良久。點頭曰。汝二人若非用藥。看此舉動。吾疑之甚矣。

\end{quotation}

宦蕚笑得一仰一合。連酒杯都打翻了。童自大脹紅了臉。道。哥。你罵我是呆子罷了。如何說我家奶奶與醫生弄。說別的頑話還行得。一個老婆那是混說了頑得的。賈文物道。言悖而出者亦悖而入。前言戲之耳。何慍耶。童自大只管爭競起來。宦蕚道。好弟兄。說笑話如何認得眞。每人罰了一大杯酒。纔不言語了。宦蕚道。我也有個笑話說與你衆位聽。

\begin{quotation}

一家弟兄兩個。有一個嫂子。他哥哥出門去做買賣時。許下了一個願心。若賺錢回來償還。果然出去得利。回家買了幾斤肉。煮了還願。那嫂子在廚房裡燒火。他弟兄兩個收拾供桌。香蠟紙馬停當了。哥哥叫兄弟。你看肉要好了。拿來燒紙。兄弟到了廚房裡。見嫂子彎着腰撅着屁股燒火。褲襠破了。剛剛把陰戶露出來。那兄弟忍不住伸手去一摸。那嫂子嚇了一跳。回頭一看。見是小叔。笑罵道。斫千刀的。你肥肉能吃得幾塊。他哥哥聽見了。只當兄弟偷肉吃。罵道。你害了饞癆了。還沒有敬神。你就想受用。

\end{quotation}

原來婦人的這件東西都是敬得神的。衆人大笑了一場。鄔合道。老爺說的固然是笑話。然而竟實有這樣的事。晚生前日往北門橋去。見一家門口圍着許多人。晚生也擠了進去看看。原來是弟兄兩個。有一個老娘。還有一個嫂子。他娘晌午些睏了。在堂屋裡春凳上睡覺。怕蒼蠅。拿一隻袖子蓋着臉。這小兒子打外邊進來。只當是嫂子。輕輕的爬上身。拿挺硬的㞠子向胯襠中狠狠的一戳。他娘驚醒了。見是兒子。罵道。要死的奴才。你做甚麼。他見是娘。忙跳下來。說道。哎呀。我看錯了。他娘道。一家只有我同你嫂子。你又沒媳婦。你旣說是錯了。這明明是要偷嫂子了。要送他到官。拉到街上。衆街鄰問知了緣故。勸了回來。只叫哥哥打了他十扁擔。攆了出來。這是晚生親眼看見。也可當個笑話。童自大道。你說這嫂子的事。我也想起個笑話來。

\begin{quotation}

一個揚州人托個朋友做件事。說道。你要替我做成了。把我家嫂子把你熱一下子。他哥哥聽見了。罵道。蠟花。你個嫂子怎混許別人熱。他兄弟道。我是哄他的。嫂子的屄放着。我不會熱。肯把他熱。

\end{quotation}

衆人也笑了一陣。宦蕚道。我還有個笑話。

\begin{quotation}

一個大老官帶了個篾片去嫖婊子。叫婊子睡在床沿上。這大老官站在地下弄。說道。我們弄着。要編隻曲子唱着弄。纔有興頭。遂扛起那婊子的腿來。唱道。小脚兒高高豎了。然後把㞠子弄了進去。一抽一抽的唱道。卵子兒緊緊撞着。却謅不出來了。唱不下去。誰知那個篾片躱在床底下聽他們動作。見大老官編不出來了。忙伸出頭來接腔。道。俺呵。

\end{quotation}

大家大笑。連鄔合也笑了一會。道。大老爺道出晚生的本像來了。童自大笑道。鄔哥。你呵。鄔合道。晚生也有個笑話。呵一呵三位老爺罷。

\begin{quotation}

一個大老官陪客坐着。忽然放了一個響屁。那客道。是誰放屁。那篾片知道是大老官。忙道。不是屁。是蝦蟇叫。少刻臭將起來。那客白篾片道。你說蝦蟇叫。如何會臭。那篾片沒得答。說道。像是死蝦蟇叫囉。

\end{quotation}

衆人笑了一回。宦蕚向賈文物道。老鄔我們幾時替他起個號好叫些。儘着老鄔鄔哥的不好聽。賈文物道。兄之言是也。何不即爲起之。童自大哈哈大笑。望着鄔合道。大哥二哥罵你呢。賈文物道。三弟何哂兄也。此何言哉。童自大道。這也是個笑話。

\begin{quotation}

一個人到熟驢肉鋪子裡買肉吃。見一根熟驢㞠子。問道。你那驢雞巴怎麼賣。那掌櫃的道。你這人好村。一個驢鞭子。甚麼雞巴。叫得好醜聽。那人笑道。怎麼一個雞巴你也替他起個號。

\end{quotation}

大哥二哥要替你起號。不把你比做雞巴了麼。就叫鄔鞭子罷。倒都大笑了一陣。又各飮了幾杯。童自大向鄔合道。我聽見人說做篾片的人是曲蟮托生的。又會呵脬。又會唱曲。你算會呵了。難道就不會唱曲子。你唱一個我們聽聽。大家吃一大杯。鄔合道。晚生曲子倒記得幾個。因爲喉嚨不濟。所以不曾習學。宦蕚道。甚麼相干。不過大家取樂。亂唱一個頑頑。管他好不好。賈文物道。昔者王豹處於淇而河西善謳。你岳翁岳母皆以歌名。你豈有不能者耶。蓋不爲也。非不能也。童自大道。可又來。老子娘會唱。女兒再沒有不會唱的。女兒會唱。女婿自然就會唱了。人說。若要會。同着師傅一頭睡。你同着母師傅睡。自然會唱。買個驢子拉尾巴。不是這個謙法。不要謙了。唱罷。鄔合被他們逼着。只得說道。晚生不會大套。只知道幾句小曲。宦蕚道。管他小呀大的。是個曲兒就罷了。鄔合要奉承他衆位。說道。晚生唱個劈破玉帶三掉灣兒罷。以箸代拍。就唱起來。道。

\begin{quotation}

靑山在。綠水在。我那寃家不在。風常來。雨常來。你的書信兒不來。災不害。病不害。我的相思常害。春去愁不去。花開悶不開。小小的魚兒粉紅腮。上江游到下江來。頭動尾巴擺。頭動尾巴擺。小小的金鈎掛着你腮。小乖乖。你淸水不去渾水裡來。紗窗外月影兒白。小乖乖。你換睡鞋。哎喲你手拿睡鞋把相思相思害。相思病。實難捱。倒在牙床起不來。翻來覆去流淸淚。好傷懷。淚珠淚珠兒汪汪也。寃家滴濕滴濕了胸前的奶。

\end{quotation}

他因是天閹。還是纖纖的童音。唱得竟覺好聽。宦蕚喜道。你原來會。我竟不知道。該罰不該罰。大家都吃了一本(大)杯。鄔合道。晚生唱得不中聽。汚衆位老爺的尊耳。賈文物道。鄔兄之歌。雖楚狂接輿歌而過孔子之歌大約亦不過是矣。童自大道。鄔哥將廟的會接着上。再來一個。要騷騷的纔有趣。鄔合又唱道。

\begin{quotation}

俏寃家。這兩日你待我的情兒淡淡。言語中屢屢的不似了先前。你忽然來忽然去。我看你精神恍亂。寃家想必是那人待你的恩情好。你向我跟前假惺惺。左右難。寃家你不必強支吾。畫虎畫皮難畫骨。我悔恨當初。悔恨當初。有眼不識薄倖徒。薄倖徒。把海誓山盟一旦無。我搥搥胸。跌跌足。老天生我不如無。癡心無有癡心報。好命孤。我一心也不怨你這麼樣無情也。怨只怨我八個字兒生來的苦。

\end{quotation}

童自大笑道。鄔哥。你唱的眞是土地老兒沒兒子。宦蕚道。這是怎麼說。童自大道。廟絕了。又普席吃了一杯。宦蕚道。罷了。大家吃酒頑笑。叫他一個人唱就不公道了。我們一家唱一個。唱不來的拿兩根筷子豎在耳朶上。學三聲老驢子叫。童自大道。哥。你不是剃頭。竟是殺人了。我知道甚麼叫曲子。聽着還不懂得呢。宦蕚道。不會唱就學驢子叫。誰是會唱的麼。不過頑意而已。混哼哼就是了。我就先唱個占花魁上萬俟公子游湖的幾句罷。唱道。

\begin{quotation}

沒頭角。少問學。打雄吃飯酒量濶。倚着區區家父勢。橫行到處慣作惡。

\end{quotation}

唱了。向賈文物道。二弟來。鄔合道。從沒有聽見過大老爺的妙腔。這個腔口板眼。大約閤城的名班也沒有勝得過的了。賈文物道。長兄旣歌而善。弟敢不而後和之。幸勿哂焉。我唱琵琶記考試中一曲可乎。宦蕚道。管他甚麼。是個曲子就罷了。他唱道。

\begin{quotation}

看你腹中何所有。一肚腌臟臭。若還放出來。見者都奔走。把與試官來下酒\footnote{他二人各道本色。}。

\end{quotation}

童自大道。二位哥倒都還來得呢。叫我就不會這幾句。宦蕚道。顧你不得。快些唱。童自大道。憑哥怎麼處治罷。唱是不會的。宦蕚道。先說過不會唱學驢子叫。童自大笑着拿起一雙筷子豎在耳朶傍。呼兒呼兒叫了三聲\footnote{也是自道本色。}。衆人無不大笑。又飮了數杯。宦蕚道。我行個令。先說的笑話都不甚好笑。如今拿一個骰子。從我滴一家擲一擲。點到誰誰就說。滴着么說一個。滴着二說兩個。童自大道。譬如滴個六。把我肚子翻過來也沒有這六個笑話。這就活殺人了。宦蕚道。你聽我說完了着。說得好惹人笑。衆人吃一杯。說的不好不笑。本人罰一杯。不會說一個笑話罰一大鍾。童自大道。這就難爲死我了。我知道今日這個酒全要灌到我肚裡了。宦蕚叫取了骰盆來。先吃了一鍾。道。令酒乾。拈起一個骰子擲將下去。是個四。數到鄔合\footnote{看他擲骰數點的坐位。宦蕚以大哥自居。係坐東面。西賈文物。對坐童自大面南。鄔合下陪。寫得一絲不紊。此等處亦必細心不苟。}。宦蕚道。你說四個。鄔合道。晚生有僭了。說道。

\begin{quotation}

一個人窮得很。每日虔誠禱吿。求一位眞仙救度他的苦難。一日。感動了一位神仙降凡。賜他一枚金錢。道。你到大海沿上。拿着這錢。炸炸炸大叫三聲。那海水就乾幾丈。龍王急了。自然來求你。任你要甚麼寶貝怕沒有麼。他叩謝了。走到海邊。大叫了三聲炸。果然水乾數丈。一個巡海夜叉爬上來道。上仙有甚麼事撤我的海水。他想道。若說要寶貝。多了我一個人拿不去。少了不濟事。何不要他的女兒做老婆。有了海龍王做丈人。還愁沒有寶貝麼。遂道。我因沒有妻子。要來求你龍王的公主作配。若不依從。我有這個金錢。只用叫幾聲炸。你海水就乾到徹底。你龍王一家連存身的地方都沒有。你快去說了來回報。那夜叉慌忙跳了下海。到水晶宮把他這話報知龍王。龍王着急。忙傳鯉丞相鯾軍師衆臣來商議。鯾軍師道。須如此如此。就不怕他了。龍王大喜。就差鯉丞相快去。到了岸上。向那人道。方纔夜叉報說上仙要公主爲婚。龍王焉敢不遵。但我家公主是個貴人。上仙須下一個厚聘。纔成禮數。那人道。我空身到此。那裡有甚麼東西可做聘禮的。鯉丞相道。何必要別物。仙翁的這枚金錢就可做聘禮了。公主少不得還帶了來。那人欣然就遞了與他。鯉丞相接過。就下海去了。半日不見動靜。那人又炸炸炸的大叫。那夜叉在海中望着他笑道。你先有個浪錢炸着人怕你。你如今沒了錢了。還炸些甚麼。

\end{quotation}

宦蕚賈文物都笑了。童自大道。好罵好罵。罵我有錢的炸呢。鄔合道。晚生怎麼敢。老爺不用多心。宦蕚道。無心說笑話兒頑。那裡認得眞。向鄔合道。你再說。鄔合又道。

\begin{quotation}

一個秀才做文章。哼哼喞喞。千難萬難。總做不出來。他妻子笑道。你們做文章難道比我們養孩子還難麼。那秀才道。難難難。你們是有在肚裡不得出來還容易。我是沒有在肚裡的要他出來。豈有不難的。

\end{quotation}

衆人都大笑。童自大笑着向賈文物道。哥。他打趣你呢。你做文章可是這樣難。賈文物道。難矣哉。難矣哉。彼之言是也。非戲我者耳。宦蕚道。我們一家吃一杯。叫他也吃一杯。潤潤喉嚨好說。大家都飮了一杯。鄔合說道。

\begin{quotation}

一個鄕下人。他家的房子無處不漏。一下雨竟無棲身之地。他村中又有虎又有賊。他家裡有一條牛。因不放心賣掉了。一夜天又下雨。他睡着說道。我如今也不怕賊來偷我的牛。也不怕虎來吃我的牛。我只怕漏。儘着念個不住。一個虎正來要吃他的牛。聽見了這話。想道。我會吃他的牛。賊會偷他的牛。他倒不怕。反怕甚麼漏。這個漏是個甚麼東西。這樣利害。我不要冒失。且等等着。不要遇見了漏。就在牛欄門口伏着。不覺就睡着了。恰好有一個賊。只當他的牛還在。想來偷他的。也聽見他說這話。心裡忖道。我同虎他都不怕。單怕漏。這漏端的是個甚麼。又想了想。道。管他漏不漏的。且趁早偷了牛去着。走到牛欄門口。黑影裡見那黃虎睡着。只當是牛。輕輕的跨上。要打他起來。那虎猛然驚醒。慌道。不好了。這定然是漏了。馱着往山上沒命亂跑。這賊見那虎一跑。也慌道。這就是他說的甚麼漏了。忙把他脖子抱緊。任他混跑。天色黎明。這賊一看。原來是一隻大錦毛老虎。心中正然着急。那虎也跑乏了。靠着一棵大樹喘息。這賊忙爬上樹去。那虎見身上的漏去了。歡喜非常。又往前跑。遇着個猴子。問道。虎哥。你爲甚麼跑得恁個樣兒。虎道。不要說起。我去偷一家的牛。遇見了一個漏。我馱着跑了半夜。他爬上一棵樹去了。我纔脫身跑了來。猴子道。從來沒有聽見甚麼叫做漏。大約還是個人。那虎同他商議道。你拿一條葛藤。一頭拴在我的脖子上。一頭拴在你的脖子上。我同你去看。你上樹去。要是個人。你推下來我吃了。改日我尋些鮮桃美果謝你。若是漏。你望我擠擠眼。我好拖着你跑。兩個同到樹下。那猴子往上爬。那賊着了急。扯開褲子溺下尿來。正撒在那猴子的臉上。猴子低下頭。把眼一陣擠。那虎正仰着臉望他。一見他擠眼。大駭道。不好。是漏了。拖着就跑。跑了幾里。回頭看那猴子。那猴子已拖死了。把嘴齜着。那虎道。猴兒猴兒。我這樣費力。你齜着牙望着笑呢。

\end{quotation}

說得大家大笑。童自大忽道。一棒打着了三個。把我們都罵着了。說我們齜着牙望着他笑呢。還不該罰。鄔合道。晚生是無心。老爺要這樣計較。就不敢再說了。宦蕚道。免你罰。你說個篾片的笑話兒罷。鄔合道。有有。

\begin{quotation}

大老官放了個屁。傍邊一個小孩子道。是那裡鬼叫。那篾片喝道。胡說。放狗屁。

\end{quotation}

宦蕚大笑道。這該罰。這該罰。鄔合道。晚生本是奉承的話。說扠了些。晚生領罰。吃了一大鍾。宦蕚將骰盆送與童自大。道。該你擲。他捻起來。道。菩薩。不要擲着我自己纔好呢。擲將下去。是個么。他道。還好還好。要是五就坑人了。想了想。道。我想起一個來了。我前日聽見人說個笑話。打趣那好打馬吊的。

\begin{quotation}

一個怕老婆的人好打馬吊。一日輸了錢。人上門來要。他老婆惱了。叫他頭頂馬桶跪着。他說。奶奶。你看我頂着這東西可像個空湯。那老婆大怒。拿起馬桶蓋。劈臉一下打去。他笑道。奶奶。你打的謅得很。一文錢怎打得空湯。

\end{quotation}

齊笑了一陣。賈文物心有所觸。嘆道。滔滔者天下皆是也。吾非斯人之徒與而誰與。衆人也不懂得他說甚麼。童自大送盆與他。他也擲了個么。笑說道。

\begin{quotation}

有一文人娶其妻焉。晚間向妻子深深一揖。道。周公之禮不可不達。其妻不知何謂。默而不答。彼即趨而出。如是者一月矣。妻歸而吿諸母。母曰。爾但云。旣侍君子。任君所欲。妻記其言。他日歸。其夫又如前揖而言之。妻以母敎之言相答。遂如此云云。久之。妻得其樂趣。不待其夫來揖。便道。旣侍君子。任君所欲。其夫則交垢(媾)之。如是者屢屢。其夫力不能矣。對陰戶一揖而吿之曰。非敢後也。馬不進也。

\end{quotation}

衆人見他講得文縐縐的。倒都大笑了一場。遞盆與鄔合。鄔合忙站起接過。拈起骰子。道。尊令了\footnote{寫篾片是個活篾片的身分話語。一毫不肯苟下筆。}。擲了個二。該是宦蕚。他說道。

\begin{quotation}

一個人出門回來。見床上睡着個漢子。問老婆道。這人是那裡來的。老婆說。他家因妻子狠打了攆出來。沒處安身。借我家睡睡。男人說。我回來了。他在那裡睡。老婆說。他是客。自然讓他床上睡。你將就在地板上睡睡罷。男人說。你呢。老婆說。你是自家。我自然是陪客睡。那男人想了想。忽然大笑。老婆問道。你笑甚麼。男人道。我想這人被老婆打了出來到我家來睡。恐怕後來要當忘八呢。

\end{quotation}

衆人正笑着。童自大道。哥罰一鍾。宦蕚道。爲甚麼罰我。童自大道。人說對着和尚不要罵禿子。你方纔這個笑話。不怕鄔哥多心。說你打趣他麼。鄔合被他提破。臉脖子徹耳通紅。宦蕚笑道。多嘴的。我倒是無心。罰了一鍾吃了。又說道。

\begin{quotation}

一個人做官胡胡塗塗。不論原吿被吿。拖番就是二十板。他女人道。一個犯人也有該打多打少。怎麼一例混打。今後你審事。我在暖閣後邊聽。該打該放。你回頭看我做手勢。次日上堂。審了一件事。回頭望望。他女人伸了五個指頭。又做手勢叫打。他吩咐道。拉下去打五板。打完了又回頭望望。那女人搖手叫不要打了。他錯會了意。吩咐道。你們推他地下滾。那人是褪了褲子打的。滾翻了過來。一個軟叮噹的大㞠子拖着。那女人見了。把個指頭咬在嘴裡。他又回頭看見。〖口么〗喝皀隸道。把他的㞠子咬掉了。

\end{quotation}

大家笑了一會。又重新添上佳肴美果。一面吃酒說笑。宦蕚笑向童自大道。令舅是敎門。我有個回子的笑話。說了你不要見怪。童自大道。他是回子。我又不是回子。與我甚麼相干。宦蕚笑着說道。

\begin{quotation}

回回家女人的陰毛是要剃盡了的。一個老回婆叫了個待詔到房去剃。那待詔見他的陰戶也還飽滿可愛。不覺興動。陽物大舉。取出來。一下頂進。一陣亂抽。那回婆假意道。哎呀。你這是怎麼說。待詔道。奶奶的癟了不好下刀。我楦起來好剃。說着。越弄得利害。那回婆受用得很了。哼着說道。我的哥。你不用剃了。就是這等楦罷。

\end{quotation}

說了。衆人笑了一陣。賈文物問童自大道。賢弟必知其詳。有婦人焉果若是乎。童自大道。那裡有這話。那東西怎好叫人剃。自己用鑷子拔是有的。賈文物道。此嬌嫩之處也。拔之豈不痛乎。童自大道。譬如人拔鬍子。慣了也就不覺。宦蕚笑着套他一句道。回子家的女兒嫁到我們家來還拔不拔呢。他道。怎麼不拔。自覺失口。笑道。不知道。不知道。不要管閒事。我們且吃酒\footnote{閱此方見童自大眞呆而老實。宦蕚則呆中猶帶滑也。}。宦蕚賈文物哈哈大笑。他也紅了臉。嘻嘻的笑。大家又飮了幾鍾。宦蕚對童自大道。我們結拜過。就是親弟兄一樣子。我與二弟一個是廕生。一個是進士。都算是現任官。賢弟雖然是個加納的老爺。算不得現任。還得弄一個現任的纔妙。童自大道。愚弟也有此興。但細想來。哥做官有老子做主。人不敢欺。二哥做官有同年相爲\footnote{這幾句話却是乖。}。我若做了官。上司說我是個財主老爺。張着大嘴要吃起來。我的銀錢是性命一樣的。怎肯白送給人。想到這裡。就一點興頭氣兒也沒有了。宦蕚道。你想的固然是。難道今生就是這樣罷了麼。童自大道。可不是甚麼。我如今把個兒子眼都盼穿了也沒有。趕着養個兒子。大了送他去讀書。像二哥似的。買個舉人進士給他。也就算得現任了\footnote{好想頭。}。宦蕚道。賢弟。你這話叫做整韮菜包餃餌。好長饀。兒子還不知在那個腿肚子裡轉筋。就想做封君。就是做了封君。也算不得現任。童自大道。我就是這個想頭。別的再沒法。古語說得好。只愁不養。不愁不長。只要有本事。養下個兒子來。長得快多着呢。我記得當日六七歲的時候。我的哺還抱着我吃奶\footnote{徽州呼母爲哺。}。就像幾日的事。我如今就這樣大了。但只是沒本事。養兒子就沒法。宦蕚笑道。你旣這樣巴兒子。多娶幾個妾。自然就會生了。童自大把脖子縮了縮。舌頭伸了伸。回頭四處看看。叫了兩聲童祿。宦家的人答應道。他纔出去了。童自大向着宦蕚道。哥。說正經話。像這樣兒戲的話不要說他。造化方纔童祿不在這裡。牆有風。壁有耳的。設或傳得我家奶奶知道。不說哥說頑話。還疑是我說的。那就叫做竹管煨鰍。直死了。宦蕚笑了笑。道。你如今旣沒有兒子。到底另想個主意出來纔好。童自大道。實在不會想。但恨我生的不是時了。若生在一千多年前。可不好來。却生在如今這時候。只好怨命罷了。宦蕚道。這是甚麼緣故。童自大道。我聽得人說。當初漢朝有個姓崔的。說他拿了幾百萬錢。買了一個甚麼司徒。說這司徒大得很呢。只有他吃人的。再沒人敢吃他。我若生在那時候。拚着家私不着。也買一個做做。只當開了個大當鋪。利錢還用不了呢\footnote{古今貨郞皆不過是此想頭。}。豈不燥脾。却生在如今。怎不怨命。宦蕚道。我一團做官的興被你說得冰冷。但天生我財必有我用。不然生我們這些財子做甚麼\footnote{不須着急。此等財子萬無不做官之理。}。或者等着賣司徒的時候也不可知。若有這時候呢。愚兄與賢弟大大的兩位司徒自不必說。若不能遇。我二人優遊林下。做個山中宰相罷。賈文物道。長兄之志則大矣。獨不思莫之爲而爲者天也。莫之至而至者命也乎。鄔合贊道。好個山中宰相。異想異想。童自大道。哥的想頭雖然甚好。只山字不合。我們現住在城心兒裡。怎說得個山中。還是城字是理。宦蕚道。城字罷。是也罷了。只是俗得很。不如村字還雅。童自大道。村字好是好。只是太下賤了。村裡可是容得我們這樣大老官的。得一個半俗半雅的字纔好。宦蕚道。賢弟旣如此說。就請想這麼個奇妙字眼。童自大想了一會。道。我當鋪隔壁有個學館。我聽見那先生敎學生的詩。有一句甚麼甚麼落御溝呢。一時再想不起來。鄔合道。晚生倒記得句把。不知可是。童自大道。你說了看。鄔合道。可是一葉隨風落御溝麼。童自大道。是極是極。這也奇了。你竟是個順風耳。怎麼我家隔壁先生敎詩。你就聽見了。向宦蕚道。我聽見那先生說。御者。朝廷之御內也。溝者。御內之溝也。這兩個字豈不又富麗又新鮮。豈不妙之乎。我三個人同做個御溝中宰相罷。鄔哥同我們日日相聚。不要偏了他。也叫他到溝中來。日逐同樂。哥。我這個想頭。可是山頂上一連三座觀音堂。宦蕚道。這是怎麼說。童自大笑道。這叫高廟高廟高廟。宦蕚大喜道。虧你想。果然好新奇字眼。可謂妙極而無以復加乎也。賈文物道。長兄賢弟雖願爲小相焉。但愚意不在斯耳。宦蕚道。我們好弟兄。有官同做。有馬同騎。自然該同心纔是。賢弟怎麼又有別意。賈文物道。小弟已是發甲之人矣。後來倘有徼倖鼎甲之時焉。豈不榮耀而之乎也哉。童自大道。哥。這算計果然好。我明日也像哥買個舉人進士做。好陞鼎甲。狀而元之。燥其皮也。大約也與那甚麼司徒差不多了。賈文物道。賢弟之言謬矣哉。舉人進士乃博學而成名者。豈能沽之哉沽之哉之所得也。童自大笑道。哥。我們好弟兄。你還瞞我。你那年中舉。多少人還打榜哭廟。又打到那個官兒門口去了。我也跟了去看來。那官兒惱了叫拿人。我穿着一雙紅鞋。人把我當做秀才。幾乎把我捉了去。虧傍邊有人認得我。說這是童百萬。一個字也不認得的大白丁。你拿他做甚麼。纔放了我跑了回家。你道我怎麼記得這樣淸。我因着了慌跑急了。掉了一隻鞋。到了家裡。奶奶疑我在外邊做甚麼偷甚麼的壞事。被人攆急了纔掉了鞋。要拿棒槌打我的踝子骨。是我再三哀求纔分辯淸了。饒了打。還罵了好幾日呢。是我親眼見的事。如何哄得我。哥。你當日買這舉人也費了幾個錢。要是價錢賤。今年倒是科舉年。要有賣的。你是老在行。總成替我買一個。我兄弟體面起來。也替哥爭些光。鄔合道。童老爺聽錯了。那一年有個姓家的舉人說是買的。非賈老爺也。以賈老爺之大才。取狀元如拾芥。何況舉人進士。人之打榜哭廟。並非爲賈老爺而起也。賈文物笑道。有是哉。童之迂也。即有如杞梁之妻善哭其夫之哭。非因我也。爲二三子也。宦蕚道。你們大家不要爭。眞也是進士。假也是進士\footnote{二語妙極。}。爭破了網巾邊兒沒得戴。我們閒話休題。且歸正傳。古時不知是那個說的一句話好。他說。無紅裙俗了人。像這酒席間。須得個名妓頑笑頑笑。纔可以醒脾。不然拿着酒。像灌老鼠洞似的一味蠻呷。總沒一點興趣。因向鄔合道。只有那肉夾剪夏錦兒還好。我摸他身上。有幾個楊梅豆兒。不敢惹他。童自大道。哥。怎麼叫作肉夾剪。宦蕚笑道。他的那件東西緊就得有趣。又會收鎖。故此人起他這個混名。童自大道。我也沒有多見婦人的這件傢伙。我覺得爛鬆得像個皮口袋一般。怎得有這樣緊東西。不怕他夾成兩截子麼。宦蕚笑道。是這麼說。那裡就緊得這樣利害。因聽見他說話有因。問他一句道。你遇見那個婦人的傢伙像皮口袋一般。童自大生平只見過他尊夫人那肥牝。一時無心說出。笑道。我是這樣猜。不要管他。大家都笑了。鄔合道。江西來的姓嚴的那婦人生得還好。大老爺只頑過一次。怎麼再不會他了。宦蕚道。那老婆的根子大着呢。他是當年嘉靖明閣老嚴嵩的兒子嚴世蕃的孫女兒\footnote{此二妓。前姚澤民所嫖者。先以爲不過隨手謅出二名耳。此處又還照應到。作書者好記性。看書者想已忘了矣。更有妙者。借此又將嚴氏父子羅龍文一辱。所謂筆劍誅奸者耳。}。他漢子姓羅。是羅龍文的孫子。因家道窮了纔出來接客。在家鄕怕人笑話纔到這裡來的。他好是好。有個血崩的病。時常要發。我有些嫌他。故此就撂開了。除了這兩個。別的都看不上眼。問家人道。你們可知道近來可有甚麼出名的婊子麼。一個家人叫做多嗣\footnote{宦家之僕無有不多事者。}。說道。外邊這些婊子並沒有聽見一個出色的。那裡入得衆位老爺的眼。倒有一個瞎姑叫做錢貴。生得十分標致。又有才學。近日合城聞名。同他相與的都是公子財主。些把些的人\footnote{眞正江南聲口。}也到不得他家。但他從來不肯出門。或者衆位老爺到他家去頑頑。他家中也還乾淨。賈文物道。然有是言也。吾嘗聞其語矣。未見其人耳。鄔合道。這錢貴晚生也知道。果然有才學又美貌。算得第一個名妓。可以陪得衆位老爺。賈文物道。子不過道聽而塗說耳。其然豈其然乎。鄔合道。果然不錯。晚生怎敢在衆位老爺跟前說謊。宦蕚道。旣果然好。我們幾時接他來頑頑。雖然說他從不出門。料道聽見我們去接。他不敢不來。要做一點身分。我吩咐了敎坊司差人去拿毛鏈鎖套了他來。這倒是容易的事。但有一件不瞞二位賢弟說。你嫂子雖然着實有些賢慧。只是性子利害些。我不敢輕易惹他。我這樣頂天立地的好漢是懼內的人不成。三人擡不過一個理字。他樁樁件件都合理。我不得不遵他。倘或冒冒失失接了人來。一時他發起怒來。如何了得。等我慢慢的同他商量明白了。再做區處。飮夠多時。夜闌方散。宦蕚乘着一團高興走到內室。那侯氏獨坐無事。小飮多了幾杯。已經睡下。正有些慾火炎蒸。宦蕚見他已睡。也慌忙脫衣鑽入被內。輕啓兩股。盡根揷入。十分努力抽提。要博他的歡喜。那侯氏果然喜孜孜笑着。兩手勾定他的腰往下直搗。做得正在得意。宦蕚乘他歡喜。一面抽送。一面說道。今日老賈老童說外頭有一個馳名的瞎姑兒。生得模樣又好。各樣的曲子都會唱。他們說明日接到我家來頑頑。我問你一聲可行得。侯氏聽了大怒。擰了幾把。將他一掀。跌下肚子。侯氏一骨碌爬起。揪着他耳朶。赤條條叫他下床地下跪着。罵道。你這天殺的。我說你今日爲何這般着力。原來圖我歡喜。想做這樣大膽的事。你有我這樣的妻子。也就儘夠你受用了。還想吃野食。惱了我。性子狠一狠。把你的㞠子生生的咬了下來。我這兩日纔與你三分顏色。你公然就想開起染房來了。宦蕚哭喪着個臉。道。你知我素常守你的家法。對着丫頭們連笑也不敢一笑。看也不敢多看。何嘗有一點私心欺你。我要是欺你。就是欺天了。這是他兩個的好意。說同我結拜一場。無可奉承長嫂。要叫個瞎姑來唱與你解悶。我怕你多心。不敢應承。他們叫我來預先和你說明白了。纔好去接。一團敬你的美意。爲何倒疑心起來。反這樣發怒。我要有這樣驢心狗肺。憑你叫我說甚麼咒我就說。你前日怪我不親熱你。纔親熱得幾日。你又放出這樣嚇人的面孔來。叫我怎麼不怕。不要說我嚇軟了。你看連這樣個鐵一般挺硬的東西也被你嚇得鼻涕似的。好像一條大蚰蜒蟲了。侯氏聽了。回嗔作喜。將他拉起來。道。你不曾說明白。幾乎沒錯屈了。你這樣個大漢子。說話到三不着兩的。笑嘻嘻一把攥着陽物。道。你不會說話。怪不得我\footnote{不撫慰宦蕚而撫慰此物。視人爲輕此爲重耳。淫心有趣。}。快些上來罷。你明日對他們說。雖是他們的好情。這樣事萬萬行不得。若是男瞎子。不要說一個。便是十個一百個叫了來也不妨。一個女瞎姑同婊子兩種人。都是撩漢精。可是容得上門的。斷斷行不得。我連聽見說還惱得慌。不要說眼睛看見。宦蕚爬上床來。恐他尚有餘怒。只得搓捏了一會。又弄硬了。儘力奉承一度。然後並肩交股而睡。次日起來。飯後賈童鄔三人齊到。吃酒之間。宦蕚道。接錢貴的事。我昨晚與你嫂子說了。倒被他正言厲色說了一頓好的。他說我家老父現做着大亨兒八的顯官\footnote{此乃江南市井之語。亨兒八三字却不解何意。}。如何接妓者進門。雖然說是瞎子。到底人說的不好聽。恐外人談論不雅。他的話眞是頭髮牽着老虎走。理能服人。純說的是些大道理。令我毛骨悚然。無言可答。不然。接到二弟家中。我們大家一樂何如。賈文物正拿着酒杯吃酒。聽他說這話。心下一驚。渾身打了個寒噤。把個杯子掉下地去。跌得粉碎\footnote{前魏如虎嚇掉茶杯。此處賈文物跌掉酒杯。先後遙遙一對。}。忙說道。西子蒙不潔。則人皆掩鼻而過之。見冕者與瞽者。雖褻必以貌。彼無目者也。可相親乎。且賤閫之政如嚴君焉。若知之。弟雖死而無悔。且恐獲罪於兄。慮彼亦必自經於溝瀆矣。宦蕚道。一團高興。我兩家都行不得。難道就罷了。這樣罷。我兩個出東道銀子。不要破費三弟一文。接到他家去頑頑罷。這可行得。童自大聽了。希圖內中有得羨餘。滿口應允。道。今日遲了。又都吃得酒醉飯飽。就接了他來。我們也吃不得甚麼東西了。不如明日罷。大家又說笑了一會。宦蕚向賈文物道。旣說這錢貴有才學。二弟明日作幾首詩嚇他一嚇\footnote{作詩何以嚇人。奇談奇想。}。賈文物道。一瞽者何以文爲。只弟數語之下。彼必瞠乎其後矣。鄔合道。他若聽了賈老爺這文才。自然害怕的。大家又坐了多時。約定明日取齊同到童自大家去。然後方散。那童自大利令智昏。不記得他夫人的利害了。到了家中。歸到內室。做個笑嘻嘻的臉。走到鐵氏面前站着。將宦賈二人出銀子要接瞎姑錢貴到他家中來頑\footnote{前宦蕚對侯氏所言詳。此處童自大之言略。}。還不曾說完。不隄防被鐵氏夾臉一掌。一個滿臉花。連耳根稍帶了一下。誰知鐵氏這手比鐵還硬。打得童自大滿目生花。耳中如磬。鼻血直冒。他潑聲罵道。你這囔死飯無用的殺材。好飮貪杯。終日吃得爛醉。一倒下頭。如死人一般。夜間一些正經事也不能幹\footnote{此等說。眞該打。}。反要接瞎婆子來頑。我知你眞活得不耐煩了。童自大昏了半晌。一手摀着臉。一手捏着鼻子。道。我何嘗要接了頑。是他們的意思。我不過想賺些酒食肥嘴。家裡又可以省些柴米。我可敢要做這樣壞事。我要有這樣爛心爛肝又可敢來。還望着你說\footnote{事雖與前宦蕚侯氏相同。却是兩人兩樣。毫不相同。}。鐵氏還喃喃嘟嘟罵了一會。方纔去睡。童自大不敢嘖聲。洗淨了鼻血。也悄悄睡了。次日淸早。先到宦蕚家中。他恐遲了。衆人到他家去。剛坐下。適賈文物也攜了分金來。鄔合亦到。宦蕚問童自大道。昨晚說接錢貴來頑的話何如了。我等二弟來。正要同到你家去。你怎麼倒又來了。笑道。像是有人不許麼。他脹紅了臉。惱都都的也不嘖聲。賈文物笑道。此樂事也。賢弟何怒之甚乎焉。必有故也而勿隱。童自大氣憤憤的道。你們兩個怕嫂子都不敢做。就總成我這個老呆。你們也心忍。叫我昨晚回去纔說得一句。被我家奶奶一掌幾乎把我打死。今日已是兩世人了。還說接甚錢貴呢。指着臉道。你們看看這腫的。我方纔照照鏡子。還靑了半邊呢。這是二位哥的擡愛。我昨晚的鼻血淌了有兩碗。這會子還暈刀刀的。鄔合咂着嘴贊道。三位奶奶都這樣善於持家。不許老爺們外務。有此賢內助眞是難得。多嗣在傍揷嘴道。旣是家裡做不得。三位老爺何不瞞了奶奶們。還是到他家去。又便宜又放心。宦蕚道。有理。我做東替三弟暖疼壓驚。童自大道。承哥的情。去是去。要有人問我的臉。不要說奶奶打的。只說我昨日吃醉了。打轎子裡栽出來跌成這個樣子。衆人笑諾。遂大家整衣冠。乘肥馬。僕從跟隨。到錢家來。且說那錢貴自與鍾生定盟之後。並不接客。郝氏逼他數次。他尋死覓活。誓死不從。又經了姚澤民那一番。頭面俱傷。實在有個要尋死的樣子。郝氏雖然以錢爲寶。到底是他親生女兒。恐怕逼出人命來。只得由他。凡有客來。都推有病回了去。錢貴每夜焚香祝天。願鍾生秋闈得意。早諧連理。一日。飯後倦臥在床。忽郝氏走來。道。兒呀。有個宦公子同了兩個人。他像是富豪鄕宦。因慕你的名。特來訪你。我回他說。你有病在床。久不會客。他定要會你。坐在客座內呢。錢貴道。兒已矢身。雖死不能從命。郝氏道。兒呀。你不知道這宦公子是京城中第一個有勢利慣作惡的。同來的那兩個。我看他裝腔做勢。也不是良善好人。你若不肯出去。他一時使出宦勢來。我這老性命就送在你身上了。且還有一說。他若動了那呆公子性兒。把你凌辱一場。又奈何他。且又低了聲價。你今就說有病。他們料不留宿。不過陪他坐坐。吃幾杯酒。一來免得有禍。二來又作成老娘賺他幾個錢。豈不兩得。這也是替我母子解紛的意思。再三說勸他。那錢貴思忖了一番。素常聽得這宦公子的呆惡。恐拒絕狠了弄出事來。不但貽累母親。而且辱了自己。況只相陪坐坐。也還無害於禮。沒奈何。長嘆了一聲。只得起來。那虔婆見女兒肯了。不勝歡喜。出來道。小女因病睡在床上。纔勉強叫了他起來。待梳洗了。就出來陪衆位老爺。說罷。便安排酒飯去了。那錢貴叫代目替他掠掠鬢。將隨身衣服理了理。代目因說道。我纔張見那三個人\footnote{張字妙。若是出去看見。童自大豈不認得。}。一個是我舊姑爺。姓童。那兩個不認得。都生得癡肥可笑。若同鍾相公比並起來。眞是神仙小鬼呢。我不扶姑娘出去罷。怕他認得。叫了財香來罷。錢貴點頭。代目去叫了財香來。錢貴裝個病態。財香扶了出來。朝上拜了幾拜。衆人讓他坐下。鄔合先說道。三位老爺。一位是有名的宦大老爺。一位是進士才子賈老爺。一位是百萬童老爺。都是本地有名的大官府。因慕錢娘。特來相訪。宦蕚道。老鄔。他果然生得好。比那大衏裡的婊子果然好些。名不虛傳。鄔合道。晚生怎敢說謊。誇獎錢娘的人也不是一個。人人見了沒有一個不道好。晚生兩耳也聽久。今日托三位老爺的福攜帶來。得見嬌容。眞是三生有幸。童自大笑道。沒眼兒的珍珠。我那瞎寶眞好標致。我的虛火都看動了。臉上都發起燒來了。賈文物道。君子不重則不威。吾弟何匪之至此也。然而不知錢姑之姣者無目者也。無怪乎賢弟若此耳。宦蕚吩咐家人道。拿錠銀子賞那老鴇。叫他快收拾酒肴來我們吃。那錢貴先聽得代目說他三人形容醜陋。今又聽宦童二人談吐粗俗。賈進士假裝文墨。滿口之乎者也。因想起鍾生風流蘊藉。愈加不樂。只不做聲。有四句話兒描寫他的心事。道。

\begin{quotation}

雅意遇眞才。偏偏逢俗子。

傷心淚暗流。愁恨何能已。

\end{quotation}

不多時。就捧出酒肴來。那郝氏出來替衆人安了席坐下。各敬了兩杯進去。賈文物見錢貴雙眉緊鎖。低頭不語。因說道。久聞錢娘色藝雙絕。眞異人也。特來訪之。何不一假色笑耶。所謂一人向隅。滿座不樂也。童自大叫家人道。把錢姑面前那碗魚撤去了。宦蕚道。這是爲何。童自大道。二哥說。一人向魚。滿座不樂。何不撤去。大家樂一樂呢。賈文物笑道。愚兄所云乃方隅之隅。豈魚肉之魚哉。吾弟過矣。鄔合道。賈老爺可謂童老爺一字之師了。童自大道。鄔哥。我說錯了。你又更錯。我錯說的是魚字。你怎說一字之師。難道人說魚肉叫做一肉麼。宦蕚道。你們把閒話收拾起來。且說正經的。我久聞錢姑彈的琵琶絕精。曲子更妙。請敎這樣一曲。以伸渴想之私。錢貴道。多承過獎。但病軀氣弱。不能服事。鄔合道。錢娘不要過謙。辜負了大老爺相愛美意。因要了琵琶。送了過來。錢貴推辭不脫。沒奈何。道。不要琵琶。我淸歌一調。衆位老爺聽罷。此時一來想念鍾生。二來厭惡他三人。心有所觸。隨口編了一調醜奴兒令。歌道\footnote{曲牌名甚妙。}。

\begin{quotation}

香閨對飮知心聚。幽韻歌詩。低唱新詞。骰子拈來催玉巵。遭逢俗子驕人態。滿口胡支。裝盡呆癡。跌綻雙彎悔是遲。

\end{quotation}

音韻悠揚。以箸代拍。歌完。他們三人並不懂詞中意味。宦蕚不住攧頭播腦。口中連贊道。唱得好唱得好。那童自大靠在椅背上。道。噯呀噯呀。我渾身都酥了。賈文物道。觀三弟之態。可謂鄭聲淫矣。雖然我大賢歟。亦當三月不知肉味。賢弟聆音一至於此。定高山流水之知音矣。亦識此歌之妙乎。童自大笑道。我聽錢姑唱得這樣嬌聲嬌氣的。故此心眼裡快活。我却一個字也不懂得。那裡叫做甚麼知音。我在家常在大門口站站。聽那些小孩子們唱的幾句。那我倒是知音。聽得稀熟的。記在心裡。宦蕚道。賢弟旣學會了。何不唱給錢姑聽聽。做個抛磚引玉呢。童自大笑道。怕唱得不好他笑話。宦蕚道。不妨事。大家頑意。他笑甚麼。童自大道。哥旣這樣說。我就從鼓樓上一交栽下來。直滾到北門橋。臉上的油皮兒也沒有塌一點。還拾了一個大錢。宦蕚道。這話是怎麼講。童自大笑道。哥不懂這市語麼。這叫做老臉大發財。你們聽我唱。

\begin{quotation}

姑娘姑娘生得俏。頭戴骨姑帽。腰裡拽把草。肚裡娃娃叫。遇着大雞巴。肏得他兩頭蹺。

\end{quotation}

衆人聽了。哈哈大笑。錢貴倒也被他引得破顏一笑。鄔合道。錢娘旣然身子不快。倒是請行個令。吃杯酒罷。宦蕚道。說得通。錢姑請行令。錢貴道。從不知行令。還是衆位老爺請。賈文物道。不知令。無以爲君子也。其身症無令而行可乎。王速出令。還是錢姑而行始妙哉。錢貴推之再三。宦蕚道。你若要我行。可要遵的呢。不遵。罰一百杯。我的令。大家脫得精光。一個人一碗酒。輪流着吃。你可遵得遵不得。要遵不得還是你行。童自大道。倒是哥這個令有趣呢。錢姑你照着行罷。賈文物命衆人篩了一杯酒。遞與錢貴。道。不則不可以爲悅。無才不足以爲悅。可興於詩。否則下而飮。錢貴見他們體段談吐甚覺可笑。因道。旣承尊命。有僭了。遂說道。此令要古詩一句。頭一個要一洞字。便道。洞口桃花也笑人。童自大聽了。伸着舌頭。道。活殺人。好狠令。這都是二哥起的禍。好好的吃幾杯罷了。甚麼興於詩。詩出這麼個令來。我看那裡去尋這個洞。因笑道。錢姑眞是三句話不離本行。你家忘八便會鑽洞。我們是那裡來的洞。鄔合道。先吿過。晚生不在令內的。衆位老爺有酒。晚生情願陪飮罷。宦蕚道。這也罷了。只是不許賴酒。要賴酒就是錢姑家的老忘八。賈文物道。不拘次序之先後而可說之乎。吾恐先進而說者。野人也。錢貴道。這有何妨。賈文物道。旣如此。吾即言之矣。洞裡神仙下象棋。宦蕚道。你把我一句好的說了去了。鄔合贊道。好個洞裡神仙下象棋。好想頭。好高雅。錢貴道。請問這句詩是何出處。賈文物道。是古也。非今也。錢姑你乃通文墨者。此詩豈今之人而能作出者耶。錢貴道。旣是古詩。是那一個作的。在那一部詩上。賈文物道。古自唐宋以來稱詩伯者多多矣。此一人則予忘之矣。若謂係那一部所載之詩。愈問得而可哂也。我一個科甲之家。如千家之詩。神童之詩。唐詩古詩。還有許多無名之詩。堆之數樓焉。安能記憶載在何本哉。錢貴聽他滿口胡柴。也沒力氣同他班駁。遂道。旣說是古人中有這一種詩。姑准免飮。宦蕚道。我也有了。只是五個字。可使得麼。錢貴道。只要有典。倒不拘五言七言。宦蕚道。洞洞洞洞洞。這一句如何\footnote{蠢哉宦蕚。何不再添上兩個字。便是七言。}。鄔合道。古人疊字詩最少。晚生記得有解學士的兩句道。

\begin{quotation}

泉泉泉泉泉泉泉。飛岩石〖阝少日小〗噴龍涎。

\end{quotation}

以爲是從來沒有再見的了。今日大老爺倒記得這句好的。宦蕚道。這倒不是假話。果然也虧我想。錢貴道。這句詩從何處來的。宦蕚道。是我肚子裡想出來的。錢貴道。原說要古詩。這是杜選(撰)。罰一巨觥。宦蕚發急道。這句詩古得很。盤古沒有分天地就有的。解學士那七個泉就是我這五個洞裡淌出來的了。因望着賈文物道。賢弟你可記得。這句詩就是你先下象棋那個人作的。是我那一日在你那詩樓上翻見過。因見他作得出奇。故此記在肚裡。方纔偶然想起來。錢姑不信。改日在那本詩上翻着了送來你看。我要說謊就發個大誓。錢貴見他發急。也就笑笑道。旣是古作。也免飮。宦蕚問童自大道。賢弟快些說。不論甚麼古詩。說一句就是了。爲何如此作難。童自大道。我腸子想斷了。也沒有這個洞。求錢姑從寬。不拘甚麼話。只要說得通罷。鄔合道。吃酒原是適興。令要苛刻就沒趣了。求錢娘通融些罷。錢貴道。旣如此。聽憑尊意。童自大又想了一會。喜笑道。一般也想出來了。說道。行不動的哥哥。這一句可妙。難道又是沒有典的。我聽見鷓鴣是這樣叫。錢貴笑道。典是有典了。只是洞不在頭上。罰一杯。若論起。動字錯了。該罰三杯。也只罰一杯罷。共兩杯。請用。家人把酒斟上。童自大吃着酒。說道。錢姑你說洞字不在頭上。罰我吃了這杯酒也罷了。我請問你。頭上有個洞是件甚麼東西。笑了一會。又道。若說動字錯了。難道有兩個動字。罰便罰了。吃得有些屈得很。說着。把杯酒向口中一倒。忽然一笑。把酒嗆了出來。噴得衆人滿臉滿身。連桌子上無處不是。宦蕚道。你想起甚麼來。這樣好笑。把酒噴得滿處。童自大咳了一陣。方笑着道。方纔錢姑說洞字有兩個。我還不信。吃着酒想起來。一點不錯。婦人家屁股底下那兩個洞。一扁一圓。可不是兩樣麼。故此好笑。倒把衆人引得大笑了一場。連錢貴見他這等村俗。忍不住也笑了。他吃了二杯。鄔合也陪飮了\footnote{不漏。}。令完。宦蕚道。錢姑再來。錢貴道。先已占過。自然是老爺們請行。宦蕚道。你先已做過令尊。\endnotemark[1]何必又謙。好事成雙。只求容易些的。錢貴也就說道。這回要兩句詩。落脚要一東字。便道。螻蟻也知春意好。倒拖花瓣過牆東。宦蕚搖着頭道。這越發難了。賈文物道。此等詩多乎哉多乎哉。兄試思之。宦蕚道。賢弟有了麼。賈文物道。予腹中久記之。我言之而兄聽之。看妙乎否也。因說道。文昌八座同。鳳臺陸起東。宦蕚笑道。妙妙。好促才。鄔合道。賈老爺毫不假思索。竟同宿構。接得這等快。眞天才呢。錢貴道。請問這詩來歷。賈文物聽了。放下臉來。道。錢姑。勿謂我輕薄爾也。你能記幾許之詩。我輩做名公之人。何處不記些詩文於腹中。此二句者。乃一舍親之家堂畫臨了之結句也。我滿腹之詩何止五車。豈肯以無指實者誑爾也。苟不我信乎。我借來你試看之。我非古人之詩不敢呈於人前也。錢貴道。這鳳臺陸起東五個字。大約是落款的地名人名。決乎不是詩內的。賈文物道。嗟乎。錢姑。爾知之爲知之。不知爲不知。是知也。予嘗聞古之稱詩伯皆曰李杜。汝不聞李白譏杜甫之詩乎。有云。

\begin{quotation}

飯顆山前逢杜甫。頭戴笠子日卓午。

何處行來太瘦生。只爲從前作詩苦。

\end{quotation}

此首句豈非地名人名乎。然此亦係落款而非詩耶。你旣不知之。何必強爲知乎。鄔合道。記得詩已奇了。又記得許多的出處故事。更爲奇絕。聽當日宋朝有一個王荆公好記性。想來也未必能加於賈老爺之上。錢貴聽賈文物說得妄誕不通可笑。也再不駁。原來賈文物說的這兩句有個緣故。他曾見過一個親戚家掛着一軸大字。係南京名士陸晉公名起東所書。詩是七言律。末句都與文昌八座同。他家住鳳凰臺。故云鳳臺陸起東。因紙短。此五字與上詩相連。賈文物把這五字認做結句。反把上句去了二字。念做文昌八座同。鳳臺陸起東。倒非謅出來的。只見宦蕚笑道。造化造化。我也想出來了。賈文物道。何如。弟所謂多者豈謬言耶。宦蕚道。曰南北。曰西東。鄔合贊道。眞愈出愈奇了。賈老爺的已妙極。大老爺的更妙。只六個字。把四面八方都包藏在內。含蓄了多少文章。錢貴笑着問道。雖不違令。但這兩句如何當得詩。宦蕚道。這也怪你不得。雖然不是詩。這是我府中收藏傳家的本經上的。我聽見人說。孔夫子刪的有一部詩經。這兩個字連在一處。可見詩就是經。經就是詩了。如今在朝中做尚書。我家太老爺當初中舉中進士。都是這本經。我自幼一上學就請了一個名公特來敎我。這經我讀了七八年纔讀熟了。這經上天下的事。以至古往今來。無所沒有。也說不了那些。我自讀了此經。就不覺大通。以後再讀別的書。覺得文理就都淺薄了。童自大道。好哥哥呀。有這樣好書。就不借我兄弟看看。宦蕚道。這經是留着傳代的寶貝。原不給人看的。旣賢弟要看。改日借你看看。萬不可再傳別人。童自大道。我從小讀過半本百家姓。做了家藏的秘寶。就不知道還有這個奇書。承哥擡舉肯借我。我難道當眞是呆子\footnote{當眞二字妙。尚不自信以爲呆也。}。肯借別人。那鄔合要奉承宦蕚。假做不知。故意嘆口氣。道。這樣好書。我們小戶人家今生料不能見了。錢貴忍不住含笑問宦蕚道。請問府上這經是何名。宦蕚低頭想了一會。屈指自數道。金剛經。觀音經。女兒經。嫖經。賭經。促織經都不是。這經兩個字名古怪得很。每常熟極。偏今日就想不起來。又想道。我隱隱的記得頭兩個是人之二字。想是人之經罷。因問賈文物道。你是才子。可曾見過這經。賈文物道。此乃三字之經也\footnote{若是三字經。開蒙小兒無不讀過。若果又有三字之經。我亦不曾見過。宜乎宦蕚以爲秘寶也。}。宦蕚聽了喜極。拍案大叫道。是是是。極好記性。難道你家也有這樣好書。賈文物道。有諸。宦蕚道。我想這樣密寶。自然是我大官府同你才子纔有。料別人家沒有的。錢貴笑道。這樣奇書。天下或者儘多。旣說是府上秘寶。只得要算做奇書了。但到底非詩。該罰一杯。宦蕚道。先說過的。詩經雖不是詩。却是經。也就算得詩了。看這奇書分上。免了罷。鄔合道。大老爺說了這一番奇話。錢姑也長了許多奇學問。姑准了罷。錢貴也就笑笑罷了。因道。此位童老爺請說。童自大道。我倒有一句。恐怕不好。你又要罰。錢貴道。請說了看。合式便罷。不合式免罰另說。如何。童自大道。你往西來我往東。可合式。錢貴道。字倒不錯。這是油言。算不得。況且該兩句纔是。怎麼只得一句。免罰另說。童自大道。你殺了我也罷。東是今生不能有。要罰幾杯。情願領罰。錢貴道。無詩應罰三杯。因來得眞率。用一杯罷。童自大一氣吃了。宦蕚道。賢弟大才。平常肚子裡詩極多的。爲何不說。倒情願吃酒。童自大道。詩是有多少在肚子裡呢。只是一時輕易出不來。況且放着不要錢的酒不吃。倒滿肚裡去尋東\footnote{辱翁曰。大通大通。}。鄔合道。老爺說的是飮酒說詩。各人適興。何必拘呢。宦蕚道。錢姑再起令。錢貴道。豈有一人行三令之理。宦蕚道。你不行就遵我先的那令了。童自大笑道。麻雀的雜碎。你只當可憐見。我行個容易些的罷。宦蕚道。怎麼叫做麻雀的雜碎。童自大笑道。這是我親熱奉承錢姑的意思。麻雀的雜碎者。小心肝也。衆人大笑。錢貴道。童老爺竟是麒麟了。童自大道。你這是怎麼說。鄔合恐怕言語參差。忙揷口道。麒麟是多寶的。這也是錢娘奉承老爺是財主之意。因道。錢娘請行令罷。衆位老爺候着呢\footnote{眞好篾片。個個奉承到。即錢貴亦必周旋到。}。錢貴也會意。更不再講。說道。就依童老爺說。容易些罷。只說五個字。不拘上下。只要白丁二字在內。因道。往來無白丁。大家想了一回。賈文物也想不出來。恐人笑他。因說道。樂不可窮。慾不可極。酒止矣夫。兄請在此留宿。弟輩可以去則去矣。童自大道。今日是大哥睡。明日是二哥睡。後日纔輪到我。這兩夜叫我怎熬。我們兄弟同門做一個三戰呂布罷\footnote{這是他家揷屛上所畫者。故此記得耳。}。錢貴道。本當奉留。但身抱微恙不潔淨。得罪衆位老爺。宦蕚道。旣然如此。我們且回去。改日再來相訪。童自大道。哥。你竟是狗咬尿脬空歡喜。倒是大家同回的好。省得我眼睛出火。賈文物道。吾未見好德噫如好色者也。盍去諸。說了一齊大笑。家人點上燈籠。一哄而去。正是。

\begin{quotation}

仙花遙望莫能攀。可笑狂奴空腆顏。

自是靑蓮泥不染。何妨嬌慧對癡頑。

\end{quotation}

他衆人歸去如何。權且按下。且說那游混公自宦家出來。失了肥館。又開了一個散學胡混。因把龍家小子騙做了龍陽。被他父親打散之後。品行全無。人都知道他的心是通了六竅的。却是一竅不通。那裡還有宦家掛名讀書的學生來請他。他沒事做了。恃着一頂硬邦邦的頭巾。武斷鄕曲。把持衙門。凡是可以弄錢的去處。任你甚麼凶惡無恥的事。他無不踴躍爲之。他妻子花氏早亡。這花氏原是個團頭的乃愛。團頭者。即花子頭兒之尊稱也。他父親原也是個小花子。後來因積攢了幾文錢。他算計却好租了三間房子。收留那無歸着的乞丐在家中存宿。每日一個人交他三文做房錢。又積了幾年。囊中竟有了餘資。他買了幾間房子。到各雞鵝鋪中收了毛來曬乾。鋪在屋內有尺許厚。招攬各處花子來他家住。每夜鑽在那毛裡睡覺。比睡床鋪還受用。但偶天陰下雨。出去討飯不得。便吃他家的飯。每日要交他幾文錢。名曰雞毛錢。今日不足。明日定要補上。不敢少欠一文。俗語說。掇他的碗服他管。這些花子都仰仗着他。任他頤指氣使。不敢稍忤。他竟儼然有個主人公之勢。日積月累。十餘年竟積有數百金。公然穿起細布直裰。吃起肉糜來。做了一個花子中的財主。衆花子就尊他做了團頭。他沒有兒子。只得一個女兒。說也甚奇。他這樣個瘸腿弓腰。眇目撆手的形狀。生的這女兒並非花子之花。宛如花木之花。頗有幾分姿色。他是花子中的鄕紳了。要擇一個讀書人家的子弟做女婿。廣托媒人。事成厚謝。請敎是那個正經人家肯扳這花老親翁。他見無人肯就。便以利餌之。托媒人道。如有願成交者。除妝奩之外。還以二百金爲壓箱之資。游混公聽得此信。他那時年已三十。小兒尚還無母。他父母是早故了。是自己做主。情願爲這位花翁的門下婿。媒人去說。那老花反疑心未必是正經人家。細細訪問。知他祖父原都是秀才。他也還曾讀過書。遂許了他。這花翁着實體貼女婿。知他貧寒。不但不要他行聘。反先送銀二十兩爲製衣裳酒水之費。嫁過來時。妝奩雖不爲大麗。而箱櫃床桌之類。件件俱備。果有細絲二百兩在箱中。把個游混公喜得屁滾尿流。不但白得了一個紅顏。且又獲了許多白鏹。但只是一件。晚夕成親之時。游混公還以他是個處子。白費了許多津唾。誰知他那件東西不是含葩之花。已是大放之花了。游混公雖不曾娶過妻。也因同妓女們釘打過無數。他見花氏之物與那妓女們相彷彿。口中不住咨嗟道。噯呀噯呀。怎是這樣的。那知那花氏更老辣。聽了這話。反怒起來道。你嫌我是破罐子麼。你不要我。送我回去就是了。有我這樣個人並這些嫁妝。不怕嫁不出漢子來。游混公忙陪笑道。我誇你的這件寶貝怎是這樣的有趣。話沒有說完。你就多心起來。竭力奉承了他一度。方纔睡下。原來花氏在家時。他一個花子的府上知道甚麼叫做閨門嚴肅。有他舅舅的個兒子常到他家。十日半月的住。他兩人相厚久了。他的父母並不知禁忌。幸喜腹中還未曾結子。還是游混公的造化。游混公因囊中有鈔了。不但圖榮耀門閭。且又要與丈人爭光。那時正有捐納秀才的例。他費了百餘金納了一名。公然頭巾藍衫到丈人家去威武。那花老見此乘龍佳婿。敬之如神明。又贈了數十金爲喜筵之費。過了年餘。花氏生了一子。游混公替他起了個名字。叫做游夏流。取個與子游子夏一流人物之意。這花氏嫁了游混公剛只五年。便一病而歿。游夏流尚幼。家中無人照看。他送到花老岳翁家去撫養。到了十三歲。那花老夫婦也故了。他已過繼了那內姪承嗣。游混公方把兒子帶回。這游混公久要想續絃。因恐費鈔。希圖又有花子家的寡婦。一文不費。白白的嫁他。如何有此等巧事。所以鰥居了十餘年。年已五十來歲。性又好淫。還時常去做那鑽穴踰牆的勾當。往往爲人所辱。他恬不知恥。還道。投梭折齒不失爲名士風流。此何傷乎。南京衏中妓女們的市語。白晝有人會房名曰打釘。他無事時常在衏中閒蕩。見有略像樣些的妓女們。他定要去釘一釘。釘了問他要錢時。他道。我生員也。奉太祖皇帝制例。免我一丁。這樣不通得可笑。這些龜子們素常知道他是一個生事的秀才。誰敢惹他。況且又不曾釘壞了甚麼。只得忍氣吞聲。白白被他釘去。後來這些妓女們見了他。都稱他爲白丁生員。他不但不自己羞愧。猶欣欣得意。向人前自述。以爲樂趣。他更有一件可笑之事。出人意表。他一夜到一妓家去嫖。上床之時。他到那妓女身上交媾一次。歇了片時。叫那妓女到他身上倒澆了一番。又過了一會。他同那妓女側身對面摟抱着。又幹起一度。睡不多時。又叫那妓女到他身上舞弄了一回。天明起來時。向他要嫖金。他道。初次我弄你。二次你弄我。三次平交不算。四次又是你弄我。論理你還該給我一次的嫖錢。我因你是個小人。不問你要罷了。你怎麼反倒問我要。那龜子有些怕他。讓他白嫖而去。却也在背後彰揚咒罵了個夠。所以他的美名。人人皆知。後來他這些劣行被文宗訪着了。拿去打了一頓板子。把衣巾褫革。他羞辱還在次之。把一個騙人的本錢沒了。着了一口重氣。疽發於背。睡倒在床。他那個賢郞游夏流也二十歲了。看慣了他父親所作所爲的事。更比他乃尊加倍。凡係下流的事。無所不做。遇錢就賭。有鈔即嫖。見龍陽便愛。若沒得錢了。情願拿他的尊臀兌換。却又奸詐百出。而且一張好嘴。他那三寸巧妙之舌。一副伶牙俐齒。人再說他不過。明明別人有理的事。到他嘴中一說。不但一毫理氣皆無。還連一點人味兒也沒有。到他自己做了那萬分下流的勾當。他誇得亂墜天花。竟到了希聖希賢的地位。如他要用了人的錢。人向他索取時。他反責備人道。銀錢如糞土。仁義値千金。朋友有通財之義。肥馬輕裘還可與朋友相共。而況於些微之物。我不是不還你。正是試你爲人何如。果然小人不失爲小人。及至別人少他一文。便拚命拚死。必定要來纔罷。他又有一番妙論掩飾。道。我豈稀罕這一文錢。這正是敎你做好人處。古人說。財帛分明大丈夫。況誰無急處。你此時還了我。不失了信。下次又還可以通融。如我是生平再不失信的。聖人說。民無信不立。這是第一件要緊的事。如他用人的錢。那人說。人淸財不淸。你到底記個數目。省得後來混賴。他責那人道。能幾個錢。你便如此小器。朋友家就差了。也是有限的事。人要借他的。定要當面記淸。有的說道。怎麼你用人的便不記。人用你的便記。他道。我並非爲你而記。我記個數目。以便查算耳。凡事翻來覆去。總是他的是。全是別人的不是。或有人說及龍陽一道。他便正顏厲色的道。以鬚眉丈夫而效淫娼之事。不要說爲親友所恥。即在家庭中。今日何以對父母兄弟。將來何以對妻子兒女。勿謂爲人所知。即人不知。寧不內愧。此輩狗彘之不若。言之猶恐汚吾頰。有人知道他也是卯字號的朋友。不好明明搶白他。或用隱語譏諷。他又有一番侃侃議論道。慕容沖以龍陽而爲帝。董賢以龍陽而爲相。陳子高以龍陽而爲男皇后。彌子瑕乃子路先賢之內戚。而尚爲衛君之嬖臣。今日衣冠中人爲之者衆矣。此皆遊戲三昧耳。庸何傷乎。他這一種飾非之巧言也不能盡述。眞是個口是心非。人質獸行的下流。他四五歲時。游混公就替他定了卜通之女爲媳。他二人聯這一門親。說起來倒也是個笑話。他二人雖同城居住。同在黌門。又都出入衙門。却從未曾會見。那時有個富翁同人打官事。約了幾十個慣走衙門在庠的朋友做硬證。官事完了。設席相謝。上座之時。恰好游混公卜通兩人同一個姓計名德淸的三人同在一席。這計德淸便是鍾趨之子鍾吾仁的內兄\footnote{計德淸的名字是如此出法。妙。}。他三人坐着飮酒。都各問了姓名。卜通不住的看游混公。那游混公也不住的看着卜通。各看了一會。游混公忍不住問道。弟同兄雖俱在學。却不曾會過。却又面熟得很。像在那裡見過一般。一時再想不起來。卜通也道。正是呢。老兄也着實面善得很。再想不起何處會過。所以適纔弟不住端詳尊面。想是我兩個素常彼此聞名神交的緣故罷。計德淸笑道。二兄相會的去處。弟倒記得。二人忙問道。請敎長兄。我兩個在何處會過來。計德淸道。說了恐二兄見怪。故不敢啓齒。二人同道。這有何妨。望兄見敎。計德淸笑道。前次宗師發落時。二兄同時被屈。大約是在那裡見過一面。原來游混公同卜通前日都考了個四等。同時被責。偶然相遇。故一時想不起來。今被計德淸提醒。忽然憶起。游混公道。噯。卜通也道。噯。彼此嘆了兩聲。又都微笑了笑。卜通道。弟是罷了。兄是文場中久擅名的。前日的尊作爲何就受屈。游混公道。不要說起。弟前日臨場病目。又不得不進去。兩眼昏花。把字寫得太大了。宗師說我字在格外。故放了個四等。請敎兄的佳作却是爲何。卜通道。弟聞得新宗師是少年科甲。極喜新奇文字。我將題目用偏鋒作了。圖一篇新奇文章。掙一個案首。不想反爲所害。宗師說弟的文章。文在題外。也放了個老四。因長嘆道。哎。

\begin{quotation}

早知不入時人眼。多買胭脂畫牡丹。

\end{quotation}

兩人又閒話了一會。彼此問問家常。契厚得了不得。計德淸聽他二人說各有子女。便道。二兄可謂一見如故。游兄的令郞。卜兄的令愛。你二位何不結一門親家。豈不更爲親厚。游混公道。這是極妙的了。但不知卜兄尊意如何。卜通道。兄旣不棄。弟還有不願的麼。計德淸便做保親。二人就在席上交換了酒杯定下。過了十餘年。兒女都大了。游混公因捨不得費錢。尚還未娶。游混公的意思。把卜通的女兒只管躭延着。他父母見女兒大了。着了急。自然白白送來。豈不省事。這游夏流成日在外邊同着個小官。叫做楊爲英。朝夕相隨。這小官生得模樣雖不爲十分美麗。他那眉目之間有一種媚態動人。他還有一件絕技。枕蓆之上。舔咂迎送。比那淫極的婦人還騷浪幾分。游夏流愛他如命。却沒有許多錢使。他二人時常兌換做那翻燒餅的勾當。所以十分親熱。這游夏流十一二歲時。在他花外祖家便同那些小花子換弄屁股。無日不幹幾次。小孩子家作喪過了。弄成個精滑的毛病。望門流涕。陽具但挨着陰門或糞門。就轅門拜倒。汨汨流出。雖是他拿錢包着楊爲英。却倒是楊爲英弄得他工夫多。游混公也同他有一手兒。你道他兩個怎麼弄上的。一日。游夏流不在家。楊爲英來尋他。游混公看見過這小子多次。久已想他。因沒有機會。今見兒子不在家。趁此留他坐下。打了幾壺酒。買了兩樣菜請這小子。甜言蜜語哄他。要幹他的後庭。這小子起先不肯。游混公許他做衣裳送錢鈔。這小子就依了。與他弄了一下。過後不但衣服不做。連低錢也不見一文。楊爲英問他要過多次。他只口中答應。總捨不得拿出來。楊爲英恨他如醋。心中算計道。這個天殺的原來這樣壞。等我哄他父子兩個弄一下。一來出我的氣。二來好囮着他要錢。一日。他問游混公要錢使。游混公道。你再給我弄一下着。我纔給你。楊爲英道。罷了。今日夜裡我到前邊客坐裡春凳上睡去。你到那裡來。游混公道。你何不到這裡來。他道。你屋裡熱。那裡還涼快些。到時候我來叫你。到那裡不要說話。恐怕你兒子在隔壁聽見。不好意思。你只啞幹就是了。游混公滿心歡喜。答應不迭。這小子晚間問游夏流要酒吃。游夏流去打了兩斤燒酒來同他共飮。那小子做出許多騷模騷樣。不住勸他吃。游夏流心中快活。吃了個大醉。他又說熱得很。拉着游夏流同到客屋裡春凳上睡着乘涼。游夏流乘着酒興要同他高興高興。那小子欣然攤股。游夏流剛送了進去。抽了沒有三下。已算春風一度。楊爲英爬起來就弄他。儘着弄個不歇。游夏流道。我這會子有些酒泛上來了。你歇歇着。等我睡一覺。醒了再給你弄。我方纔只弄了你兩三下。你弄了這一會也該罷了。楊爲英也就拔出。不多時。聽得他呼聲大響。推了推。不見他動。知他睡熟。楊爲英抽身出來。到游混公窗下。低聲叫道。你來罷。游混公正等得心焦。聽得是他聲音。一骨碌爬起。赤着身子開門出來。原來楊爲英躱在那倒座內呢。游混公輕輕走到前邊屋裡。往春凳上一摸。一個人精光着。臉朝裡睡。屁股向外。以爲是楊爲英候他來弄。爬上去就幹起來。一陣混抽混搗。游夏流被他弄醒了。還以爲是楊爲英。說道。叫你等一等。你就這樣急。把我混死了。游混公正在高興之時。聽得是兒子的聲音。又不好問。心中一疑。就慢了些。忽見楊爲英點了個燈進來。笑道。你爺兒兩個好弄。游混公見弄的果是兒子。羞得連忙拔出。跑回房中去了。次日抱怨楊爲英耍弄他。楊爲英道。你抱怨我。你若不正正經經給我幾個錢。我四處替你一張揚。看你可見得人。游混公被他拿住囮頭。只得常常送他幾文。游夏流被老子弄了一下。不知內中的這些彎兒帳。又不好問老子的。私問楊爲英。楊爲英哄他道。他來想弄我的。不意錯弄了你。游夏流也就信以爲實。楊爲英雖貪了游混公幾個錢。却也回不得他。時常被他弄弄。這小子却同他錢親意不親。倒同游夏流相厚。他父子爲這小子吃醋拈酸。時常吵鬧。游混公但罵兒子一句。他睜着眼道。你想想你做的是甚麼事。你還管我。不要討我吿訴人。你纔下了地獄呢。游混公無言可答。只暗暗恨楊爲英而已。游夏流自從他老子疽潰了睡在床上。疼得一陣陣發昏。晝夜喊叫。他與楊爲英飮酒作樂。不但竟到了老僧不覩不聞的地位。而且嫌呼號之聲聒耳。偶然見他老子一個匣子中有幾兩散碎銀子。他趁老子昏迷之際偷了出來。同楊爲英不知何處去作樂。也不管老子的死活。那游混公病久了的人。瘡旣疼痛難忍。兒子又不在跟前。要口湯水也沒人與他。不知幾時死在床上。他家又沒有第二個人。誰得知道。一日。他那花大舅來看他的病。推開門入來。不見一人。走到臥房門前。聞得屍臭。進內一看。見他妹丈的那個樣子。是作過好幾日的。竟幾幾乎似齊桓公。將及屍蟲出戶了。忙各處去尋游夏流。這游夏流自從偷了幾兩銀子出來。同楊爲英各處混了幾日。一日。他向楊爲英道。我有年把不見婦人的那東西了。我到南市樓打個釘去。你在陡門橋上坐着等。我就來。楊爲英笑道。你吃麻油上腦箍。受罪也不覺得。你想想你那本事。討那罪受做甚麼。游夏流也笑道。香油炒韭菜。各人心裡愛。不要管我閒事。你等着我就是了。遂走到樓內。到一家去打釘。他同妓女上床。褪下褲子。兩物方接。他不知不覺就冒了出來。他忙跳下床。拽着褲子就往外跑。那妓女也忙穿上褲攆出來。向忘八道。這人沒有給錢就跑掉了。忘八就往外攆。趕到評事街大街上。方纔攆上。拉住道。有個打白釘的麼。你錢不給就想跑。游夏流道。我纔挨着就完了。還不曾嘗着是甚麼味道。你要的是甚麼錢。那忘八道。放着屄誰不叫你肏來麼。你自己沒本事怪得誰。你不給錢。也別想放你。兩下正在那爭持。恰好宦蕚騎着馬。幾個家人跟隨着走來。看見游夏流被一人拉住了爭講。傍邊圍着許多人看。宦蕚素常認得他。也便下了馬。問那忘八道。那拉着這游相公做甚麼。那忘八認得宦蕚。見問他話。忙放了手。跪下叩了頭。將前事稟吿。宦蕚大笑。向游夏流道。他一個小人。快給他錢去。那游夏流雖然無恥。到此時也自羞愧難當。腰中取出銀包。捻了有錢數銀子給那忘八去了。宦蕚正要上馬。只見一個人跑來叫游夏流。道。我纔到你家看你父親去。已死在床上不知幾日了。你快些回去。游夏流別了宦蕚。他聽見老子歿了。毫無悲切之容。還到陡門橋上帶了楊爲英來家。他倒也托實得很。並不裝假。進門也不看看老子的屍骸是怎樣。也並不號哭。忙忙把他老子一生坑騙人的私囊傾箱拿出。約有數百金。好生歡喜。買了一口薄皮棺材。就是那隨身膿血的衣服被褥裝殮了他。圖省錢。說出一番大道理來。道。我們儒家當遵文公家禮。不用僧道念經。信那異端邪敎。這說的還有理也罷了。棺材囂薄。又未經灰漆。那一股臭氣衝人。他因嫌惡味。却說不出口。又恐放久了。親友聞知。若來弔送。未免費事。他又有一番話說道。古禮天子九月而葬。諸侯五月而葬。士大夫三月而葬。我父親已革去靑衿。與庶人等禮。當三日而葬。況死者見土如見金。久放着何爲。剛過了三日。就擡去與他母親一處埋葬。及至他家的親友聞知走來弔唁時。孰知他令尊已出過殯了。有他的長親父執責備他死不報喪。葬不通信。他道。我抱終天之恨。擗踴苫塊泣血之時。恨不欲生。況只孑然一身。那裡還能及此。我今在哀慟迫切之秋。衆位不憐而唁我。反責我以細故。情何以堪。衆人也沒得責備他。反覺失言。各自去了。無人之時。他美酒肥肴。同楊爲英快樂。宦蕚那日聽見先生死了。也沒有見他家報喪。也不知幾時出殯。過四五日了。還不見來報。他念先生當日相待之情。封了二兩奠儀到他家來。先生的靈柩不曾見。倒見了個騷模騷樣的少年。把銀子付與游夏流。辭了出來。路上問家人道。方纔他家那標致小子。你們可有認得的。宦畎道。那小子姓楊。是個兔子\footnote{惟犬慣能識兔。}。宦蕚聽了。記在心裡。且說那卜通在鄕間敎學。聽得親家病故。上城來弔紙。入見靈柩已出。神主也沒一個。把女婿大發作了一場\footnote{卜通此却通。}。見一個小後生在他家。知道是不正氣的事。恐他把家私胡花了。催着他七內完親。不由他做主。擇了吉日。硬叫他把女兒娶去\footnote{此舉雖不通。因人而施。也還算通。}。游夏流知道這件事是終始要做的。也就尊命奉行。且又賞鑒賞鑒新人的妙容。嘗嘗臍下的鮮物。且說卜通的妻子水氏是二婚嫁他的。他前夫姓王。是個小兒科醫生。他婆婆寇氏。慣會替婦人收生。也知用藥。又給小孩子治病。水氏在他家時。跟着婆婆也就學會了這兩樁手藝。寇氏的兒子死後。見媳婦年小且又無子女。先只說等他守過週年令他改嫁。不想纔過了百日。水氏便同人作些不三不四的勾當。寇氏知道了。忙忙叫他另嫁。卜通正托媒人尋親事。聽見水氏有些帶頭。就娶了他。頭一胎生了個兒子。第二胎就生這個女兒。初生他時。卜通道。我們這樣貧寒。如今兒子也有了。女兒也有了。所少者。銀子而已。銀子又要多纔妙。就取他個名字叫做多銀\footnote{辱翁曰。此名幸而他的姓好。}。他自幼就舉止可笑。他哥哥叫做卜之仕。有三分儍氣。他父親在外邊敎學。常不在家。他母親就帶他兄妹二人同睡。間或卜通歸來。夫妻在被窩中。再沒有不做些正務的。又不好因父母要做事。把兒子女兒攆開。少不得先睡一會。叫兒子女兒幾聲。不答應了。知道睡着。方纔放心行事。後來又過了兩年。卜之仕已十三歲。他雖有些儍。也便有三分知覺。多銀那時纔九歲。兒子放在脚頭。女兒一頭同臥。又一日。卜通回來。睡了一時。叫了他兄妹數聲。總不答應。夫妻動起手來。古語說。新娶不如遠歸。都是別久了的。少不得竭力大做一番。不但要補以前之欠帳。還要預支後來的虧空。豈肯輕易草草完事。一度不已。兩次不休。弄得那水氏陰中之水漰湃大響。屁股亂攧亂簸。口中先還哼哼喞喞。弄到後來。水氏大叫道。哎呀我死了。哎喲我死了。那卜之仕忍不住嘻嘻的笑。這卜通聽見兒子醒着。忙爬下肚子來睡着。那水氏阻了高興。又羞又怒。一骨碌起來。掀開兒子的被。把光屁股上打了幾掌。打得那卜之仕大哭大叫道。我各人笑。你爲甚麼打我。只聽得多銀說道。該打。打的還少。聽見媽說要死了。你不哭倒還笑。打了你。你還叫呢。他兩口見女兒兒子都是假推睡。甚不好意思。過後把兒子分開了另睡。以爲女兒還小。不甚防他。仍帶在身邊。這丫頭醜則醜。一肚子的心。他但見父母同臥。他上床就假做打呼。及至他父母放心高興。他却將被蓋着臉。露出眼睛來觀陣。見的也多次了。心中想道。我看爹媽做這事。想是快活得很。我幾時也弄個人試試看。雖如此想。他一來年小。不知招攬來試之人。他母親替人家收生。又會給小孩子整治病。生意大行。時常不在家。卜之仕十六七歲了。終日在外閒蕩。游手好閒。做那些不知事的事。常常只留他一個在家中看家。他到了十三歲。長大了。不但他生性淫蕩。且生得醜到十分。大約世間也就無對。臉上的疙瘩麻子有指頂大。還不足爲異。都是連環圈兒。一個套着一個。活像螞蝗絆。兩隻眼中兩個大蘿蔔白花配着。那眼睛邊週圍如大紅線鎖了的。眞也異樣。那臉上的雀斑。黃的黑的堆了一臉。厚厚的抹上一層粉。襯得斑斑點點。與那芥末拌的片粉無二\footnote{異想奇比。}。頭上吊着五六寸高的一個桃兒。歪在頂上。走路一摔一摔的。四面短髮蓬鬆。金絲般披得滿臉滿項。一口烏黑的豬屎牙。牙黃也不知有多厚。兩隻大扁脚有七八寸長。一個碗口大的高底板墊在脚心上。專好穿雙大紅花鞋。竟像娃娃們頑的兩隻小船。他自己猶以爲是絕色佳人。走動定要扭頭捏頸。說話必定要抿嘴咬唇。做那風流的騷態。古人有幾句道。

\begin{quotation}

醜醜醜。只把腰肢扭。扭斷脊梁筋。醜的只是醜。

\end{quotation}

這就是他了。他還有幾件妙處。又饞又懶。又惡又淫。眞是個四德俱無。七出咸備的醜美人。有個西江月贊他的形容道。

\begin{quotation}

面似羊肝紫漆。肌生冰裂花紋。腮邊頰上滿奇痕。眼內珠中有眚。指露幾條墨玉。牙排兩片烏銀。身軀扭捏更驚人。活跳妖魔形徑。

\end{quotation}

又有兩調黃鶯兒贊他的手足妙處。贊手道。

\begin{quotation}

十指似擂搥。光溜溜如帽盔。彈箏鼓瑟渾無濟。身癢難推。血淚怎揮。欲剝靑葱倚靠誰。好傷悲。諸般果殼。全仗嘴施爲。贊他的足道。

金蓮三寸長。看他的要橫量。扁鋪在地鯿魚樣。白花滿牆。紅細做幫。高底碗大奇形狀。響噹噹。房中舉步。戶外已聲揚。

\end{quotation}

他家後門外是一塊大空地。來往的人常在那裡解手。他無事就在門縫中往外張。那陽物大小長短他倒見了許多。一日。天氣甚熱。他母親哥哥都出去了。午後熱了一鍋水。洗了個澡。因怕熱。褲子也不穿。只繫了一條夏布羅漢裙。上身穿一件小汗衫。坐着乘涼。偶然事上心來。揸開腿彎着腰。低頭看了看牝戶。道。我娘弄的時候那樣快活。且趁他今日不在家。尋個人弄弄。定然有趣。正想着。聽見門外叫賣茉莉花。他跑了出去。叫道。賣花的來。那個賣花的小子走近跟前。他一看。約有十七八歲。生得也還白淨可喜。他想了想。道。就同他試試罷。便道。你跟我進來穿花。那小子進門。他把門揷上。引到內裡。講了價錢。叫他穿五十朶一枝大牌樓。那小子放着花籃。在地下蹲着穿花。他也蹲着在傍邊看着。拿一隻手搭在那小子肩膀上。故意把裙子掀開些。露出他小肚之下那個騷物。多銀生得形貌雖醜。臍下那東西竟還可觀。

\begin{quotation}

一條細縫鼓蓬蓬。微吐花心紫更紅。

容貌媸妍雖各異。料來此竅一般同。

\end{quotation}

那小子一眼看見。由不得那陽物在褲襠中一跳一跳。那小子穿的是一條麻布單褲。那多銀看得明明白白。故意笑指着道。哎呀。你褲子裡是個甚麼蟲在那裡跳。你不怕他咬了肉麼。那小子倒紅了臉。笑着忙把兩腿攏來夾住。怕他家中有人來看見。忙忙穿完了遞與他。他揷在頭上嘻嘻的笑。那小子站起來。道。姑娘給我錢去罷。他道。我沒有錢。那小子急了。道。你沒有錢。如何叫我穿花。他笑着道。你要有情。就送給我戴。你要不肯。我給你弄一下算了罷。那小子道。不要說頑話。看人聽見。他笑道。我家裡沒人。你只管放心。說着。解開了衫扣\footnote{此可以不必。不如穿着還可藏拙。}。把裙子脫了。胸脯同下身全全露出。把小肚子往前腆着與他看。道。我這一朶鮮花。難道還不値你這朶殘花麼。他雖貌醜。這件東西西施嫫母都不過大同小異。沒有甚麼醜俊。有幾句道。

\begin{quotation}

褪放鈕扣兒。解開裙帶結。酥胸紫勝檀。玉體色如墨。肘膊熟藕般。香肩糟茄色。肚皮幸軟綿。胯下還光潔。中間一道溝。露出風流穴。今生卜女叫多銀。前世秦妻名長舌\footnote{一部書獨多銀提出前世者。因寫淫濫醜惡特甚。恐看書者非其言之太過。故提明。乃作者特特辱罵不堪。稍洩胸頭恨耳。}。

\end{quotation}

那小子從未曾見過此奇形。嘗過此美味。甚是朶頤。況且腰中那件作怪的東西。雖有一隻眼。却又無珠。不知如何。見了陰戶他就混跳起來。正脹得難過。因初會這張沒牙的癟嘴。反有些羞愧之意\footnote{這小子反不如多銀老練。}。紅着臉道。一時遇見你家的大人來怎麼處。他道。有人敲門。你打後門裡跑。怕甚麼。那小子聽有後門。也放了心。道。在那裡來呢。多銀就仰臥在春凳上。大揸着兩腿耑候。那小子忙脫了褲子。陽物挺硬。又抹些唾沫。頂了進去。他哎了一聲。道。有些疼。那小子就歇着不敢動。他道。我每常見是一抽一抽的。你怎麼不動一動。小子道。你又說疼。他道。這纔沒要緊。要養漢還怕得屄疼麼。我疼我的。你弄你的。怕甚麼。那小子也就抽抽扯扯不歇。他先還哎喲了兩聲。後來也就不做聲。不多時。那小子冒掉了些。怕有人來。忙忙拔出。拿褲子揩了揩穿上。多銀拿塊白布汗巾將牝戶擦擦。看看也有許多〇〇〇着對那小子道。你每日到門口來叫賣花。要沒人〇〇〇〇〇那小子滿口應允。笑嘻嘻提着花籃要走。多銀道。你站着。給你花錢去。小子道。多謝你。不好要錢的。送你戴罷。多銀道。你多大本錢。我要不給你。你下次就不敢來了\footnote{不想雛把勢也會招攬主顧。}。倒多數了幾文給他。那小子旣白弄了。又還多得了花錢。何等快樂。歡歡喜喜的去了。此後但沒有人在家就叫來弄。也弄過好幾次。但那小子的陽物甚微。且又不甚在行。先還將就弄了。後來弄得甚不足興。一日。在門口站着閒望。見一個賣雜貨的擔子。全是小孩子頑戲的物件。他見有許多搬不倒兒。心裡想道。這個比那小子的粗好些。買一個弄弄看。遂買了一個。走進房中。脫了褲子。揸着腿。拿那圓泥底子往裡塞。塞了一會。弄不進去。他着了些唾沫。將牝戶潤了。擦些在泥底子上。用指頭又將陰戶〖扌扉〗得開開的。往裡狠狠的一下。也就塞進去了。他捏着那人頭來回抽送\footnote{酒罐裡伸出頭來是個酒鬼。這搬不倒兒大約是個色鬼。}。正在有趣。不想那紙身子被淫水濕透浸軟了。一下斷了。扯出來。把個泥底子掉在內中。費了許多力。摳得生疼。纔摳了出來。這一下被他楦大了。再同那賣花的小子弄時。毫無趣味。他想道。這個人是沒用了。須揀個大些的纔好。每日在後門張看。或見有陽物大的。無故又不好叫了進來。或有做生意可以叫的。他母親又在家中。總不遇巧。那日家中無人。他守定了。在那裡張看過了幾個。全都細小不像意。正心中發急。一個搖鼓兒賣絨線的。把箱子放在傍邊地下。忙忙去溺尿。大約是尿急久了。陽物脹得挺硬的豎着。甚覺可觀。他一見了這個大物。顧不得了。把門一開。那人一擡頭。見是個女孩子。忙背過身子去。溺完了。背上箱子要走。多銀叫道。我要買你的線。同我進去揀。那人同他到了堂屋內。纔把箱子放下。他一把拉住。變下臉來道。我家一個大人也沒有\footnote{使之聞之。有此慧心巧舌。不可以貌諒人。}。你無緣無故到我屋裡來。要想奸我麼。那人陪笑道。姑娘是甚麼話。我怎敢無故進來。你叫我買線。怎說起這樣戲話。多銀道。我同你戲甚麼。實對你道。你要同我弄弄呢就罷了。不然我就〖口么〗喝起來。那搖鼓的道。這事如何做得。我怎麼敢。多銀急了。道。你當眞不麼。遂大聲叫道。救人。那人急得忙掩住他的嘴。道。姑娘。依你就是了。不要叫。他笑道。我當你不怕。你也怕麼。早這樣說。省了多少事。拉他同到屋裡床上。脫衣睡下\footnote{從來皆是男子強奸婦人。他竟是強奸男子。也是罕見奇聞。}。那人可是甚麼至誠君子。先推辭不過是怕事。況且又是個沒老婆急三鎗的光身漢。今見他騷淫至此。雖然醜陋。叫做飢不擇食。且又不費錢。何樂不爲。豎起一個大陽物。恐他年小當不得。還用了許多津唾。輕輕一頂。孰知是個多見廣識的。一下就進去半截。幾聳盡根。那人知他是個慣家了。遂大弄起來。那多銀樂所未經之樂。也就學他令堂。我死我死的哼個不住。事畢了。穿衣起來。拉住那人再四叮嚀。叫他常到前門來搖鼓。撞巧好約他進來。後門遠。恐一時聽不見。那人一來得了趣。二來別處那裡有這樣捨屄的善女。果然每日來他家門口搖上幾次。遇便就約進來高興一番。有二年光景。這女子腹中竟有了寶貨。他母親在外生意盛行。也竟不知。到了月分滿足。肚裡疼將起來。水氏纔知女兒是要生產。喜得他會收生。不多時。養了一個好白胖娃娃。拿去埋了。也不曾與卜通知道。過後水氏見女兒連外孫都養過了。嚴緊也是無用。任憑他的尊好。這也是甑已破矣。顧之何益之意。況自己外邊生意又撂不下手。也竟由他。多銀一日到後門口去張張。見一個討飯的花子。在對過牆根下脫了破襖蹲在那裡捉蝨子。褲襠稀爛。胯下一根好肉具。軟叮噹大長的拖着。他淫心大動。開門叫他到家中來。捨了他一頓飽飯吃了。又給了他幾十文錢。那花子感恩不盡。正然要走。多銀笑道。你站着。我問你話。你是孤身一個。還有老婆沒有。花子道。自己一身一口還糊不過呢。還禁得有老婆。又問道。你難道這樣大還沒有見女人麼。那花子笑道。當日見過來。多銀道。你如今想女人不想。花子道。我恁個賊樣。誰來愛我\footnote{孰不知竟有愛之者。不愛人而愛此物耳。}。想也是無益。想他做甚麼。多銀道。你難道見了女人就不動一動心兒。那花子見他只管盤問得有因。笑道。人在世上。誰沒個淫心。螞蟻蝨子還知道幹個事呢。沒奈何。只得罷了。料道我們討飯的人。還有這樣慈悲心的人肯施捨這個的麼。多銀笑道。你跟我進來。那花子覺有妙處。竟跟到房裡去。多銀褪下褲子。仰臥在床上。道。看你說得可憐見。我捨你一捨。只當積陰隲罷。那花子見他一個光光肥肥的陰戶。不覺那陽物跳將起來。笑道。姑娘。你果當眞的捨我麼。多銀道。不當眞。我難道哄你不成。那花子把拐棍一略(撂)。笑道。我不要是做夢。他爬上床。扛起腿就弄。把那叫街打磚的力都使出來。把個多銀弄得無窮的受用。多時方歇\footnote{寫此一段。不過極辱多銀之不堪耳。}。多銀約他常常的來。那花子喜得滿臉是笑。連連答應去了。這花子討了半生的飯。忽遇着這樣一位慈悲好善的女裙釵。你道他感激不感激\footnote{繡繻記鄭元和云。那位慈悲好善的女裙釵。與乞兒一頓飽齋。與乞兒繡一副新鋪蓋。與乞兒攜手上陽臺。這才是捨貧的奶奶。多銀大約是讀過此者。}。他無可報恩之處。惟有鞠躬盡瘁。捨命奉陪。把個多銀喜得欲狂。不想施些小惠。竟得他這樣厚報。此後或搖鼓的。或這位丐老。輪次奉承。多銀也算樂意了。不想這個花子有個夥計。名叫褚盈\footnote{褚盈者。謂以巨物杵多銀也。}。混名叫做鑽洞老鼠。當日也是好人家兒女。好在花柳叢中着脚。不但把一分家私花盡了。還落了一身楊梅瘡。後來弄得一貧如洗。只得到這卑田院中來享福。近來見這花子時常腰中帶着幾十文錢。俗語說。小人乍富。腆胸凹肚。這花子得了這番奇遇。面上未免就帶着些驕人之色。說話也不像先那乞乞縮縮的樣子。在衆花子中就拿出他那大老官的身分來\footnote{借這花子罵盡乍富狂炸小人。}。別人還不覺。褚盈是滑油一般的人。着實疑心。也就看破了幾分。屢次套問那花子。他總不肯露一字。褚盈心生一計。數日之中。將叫化的錢積了三四十文。打了斤燒酒。買了兩文錢的鹽豆請這花子。有心算計無心。假做讓他。全全灌在他肚中。褚盈見他有些醉意。笑說道。好老哥。我們不但是同行朋友。且又是好弟兄。你有甚麼好處。攜帶攜帶我做兄弟的。也是你的好情。我敢忘了哥麼。那花子只是笑不做聲。褚盈又套哄他道。哥。你得好處。我兄弟也略知道了些影兒。何必瞞我。我可肯壞你的事麼。你不吿訴我。反失了朋友的情義了。那花子還不肯說。褚盈大怒。把那把缺嘴的瓦酒壺拎起來摜得粉碎。把破線襖一脫。拍着胸。瞪着眼道。肏娘的。擾了我幾百錢的酒肴。問你句話兒。你就千難萬難的。你啞黃脹了心了麼。怎不做聲。你成日大串的錢帶在腰裡。你不是做偷摸便就是剪綹弄來的。不要帶累了我。一把揪住領子。道。我同你到頭兒跟前講講去。那花子也並不是謹言不說。一來奸人家的幼女是有罪的事。故不敢相吿。二來這褚盈生得模樣又比他強。又少年精壯。恐他知道扠了進去。怕撐掉他這好主顧。今見褚盈撒潑。他素常有幾分怕他。還要拉了去吿訴管頭。忙陪笑說道。好哥。我弟兄們也犯着這樣的麼。你問我。我可有個不說的。你何必動怒。放了手。我吿訴你。褚盈也就放手。他只得笑嘻把多銀同他勾搭的帳詳細相吿。褚盈笑道。哥。你是有福的人。俗語說的好。一人有福。拖帶滿屋。哥。你有這樣好處。就不總成我兄弟沾些光麼。我身上還有幾文。再去打半斤來請哥。你再沒有不肯的。他的酒壺摜掉了。拿了個破瓢去。又沽了一斤燒刀子來。二人一遞一口的呷。那花子知他是必於要去的。囑他道。我們好弟兄。我把實話吿訴了你。你不要得了手。把我撐了下來。褚盈道。哥是甚麼話。你竟是以小人之心度君子了。船多不礙港。車多不礙路。我爲甚麼多着你。你若不放心。要是我得了。要去我兩個同去就是了。如今先商量定了主意。你明日先去。也不必向他說。我隨後踩了進去。他正同你弄着。我撞到跟前。就不怕他不給我弄了。省了多少唇舌。那花子應諾。褚盈滿心歡喜。兩人把酒吃完了。打點明日行事。次日早飯後。那花子到卜家後門來高聲〖口么〗喝。姑娘捨些。恰好水氏卜之仕都不在家。多銀聽得。忙出來開門。見左右沒人。他叫進來。隨手把門拴上。相攜而入。這褚盈遠遠的尾在後面。眼睛瞟着。他見那花子進去了。他踅到後門口來。推了推。是拴着門。那門板上罅着一條大縫\footnote{即多銀張陽物處也。}。地下拾起根柴棒兒來。一陣撥便撥開了。挨身而入。仍舊拴好。輕輕的走了進來。不見有人。在房門口聽聽。聽得一個女子的聲音說道。你這幾日比先越發有力氣了。弄得這樣有趣。又聽見那花子說道。我當日有一頓沒一頓的。故此沒力。如今承你的情。給我的錢我時常買些狗肉吃。那狗肉性熱極興陽。所以有力量了。又聽得那女子笑說道。旣這樣說。你每日多吃些。沒有錢我給你。就不聽見再說話。只是哎呀哎呀。快活快活的叫。褚盈自從入了討飯的道路。何處再有婦人同他高興。與此道相別久了。方纔聽得他二人這一番問答。陽物脹硬難過。就走了進去。原來那花子將多銀橫臥在床上。他站在床沿地下。扛着兩條腿弄呢。褚盈走到背後。把那花子肩膀一拍。道。你的福也享夠了。不要折了福。也讓讓我們同行的朋友。那多銀正快活的閉着眼。聽得這話。睁開一看。是一個驀生的花子。吃了一驚。就推那花子要起來。那花子見褚盈來了。他是心照的。就放下他的腿。拔了出來走開。褚盈見多銀纔要翻身想爬起。他上前忙將他的腿擡起。夾在肋下。道。姑娘不要動了。我們都是一樣的品級人物。他服事得。我也服事得。一面說着。扯開破褲。趁那濕〖氵韲〗〖氵韲〗的。一下攮了進去。蠻抽重扯。多銀同那花子正弄着。已被這人看見。推辭不得。況他也是求之不得的事。任他衝突。這褚盈久不會此物了。把多年養蓄的力氣一齊使出。又想奪那花子的愛。儘力舂搗。況他又是風月行中歷過的人。比不得那個花子是夯工。他十分在行。他因熬久。雖然洩過。陽物還堅硬如鐵。把個多銀弄得渾身爽快。遍體酥麻。口中親爹親哥心肝寶貝的無樣不叫出來。足足弄了有一個時辰。方纔歇手。那多銀被他弄酥了。還睡着喘氣。那花子笑向褚盈道。你這拉牢的。我正弄得高興。被你來拆開。我要忍了精。明日淌起淋來。纔同你算帳。說着。上去又抱着多銀。道。姑娘。我先沒有了事。再捨我弄弄着。不由分說。又被他弄了一陣。他看褚盈弄了半日。興致倍豪。比前番更加勇猛。多銀這個樂境眞說不盡。事畢了。多銀穿衣起來。咧着個大嘴嘻嘻的笑個不住。將他娘的錢偷出二百文來分贈二人。囑他們常來走走。他們可有不願意的。連聲答應。也笑着道謝。各人拿着竹杖破瓢就走。多銀送他兩人出了後門。回房坐下。想方纔的樂處。喜不可言。又想道。天色還早。爲甚麼放了他們去。若留住他。此時不還在快活麼。正在懊悔。忽聽得前門外不住鼕鼕的搖鼓兒響。忙跑去開門。看見是兩個。那個舊主顧笑問道。奶奶同大相公可在家。多銀笑道。不在家了。他道。我進去有句話說。就同那個搖鼓的都一齊進去。多銀關了門進來。搖鼓的走進臥房。用手招多銀入內。附着耳朶笑向他道。我承你的厚情。沒得報你。我這個夥計本事高強。帶來同你作樂的。你可要麼。若是要呢。叫進來。你同他試試看。若不要。我就同他出去。多銀恨不得有十個他也不嫌多。何況兩人。又聽得誇他有好本事。笑着不住點頭。搖鼓的知他首肯。出去向那人悄悄說了兩句。那人進房。見多銀坐在床上。笑道。我那夥計舉薦我來服事姑娘呢。就上前替他脫褲。那多銀毫不裝假。任他脫去。那人也褪去了。弄將起來。陽物的大小與那一個雖差不多。被他從午後直弄到日色平西還不肯歇。多銀丢了數次。眞是嘗所未嘗也。顧不得陰戶的皮穿底塌。任他抽搗。正在高興。忽聽得前邊叫門。是水氏的聲音。多銀忙道。不好。我媽回來了。嚇得那人穿褲不迭。跳下床。背上籠子。同那一個搖鼓的如飛跑到後邊。開門跑了。多銀出去開門。水氏問道。怎麼我叫了這一會纔來開。多銀道。我睡着了。不曾聽見。纔醒了就來開的。那水氏也不再問。後來四個人皆源源而來替他應差。數年之中養過了三四胎。但苦了這些娃娃。都是未見天日而亡。水氏見女兒生產過多次。以爲是理所當然。毫不爲異。這年他十九歲了。游混公在日。卜通也催過他多次叫他家來娶。游混公連老婆都捨不得錢娶。如何肯娶媳婦。以爲他家女兒年大等急了。自然送來。今卜通見親家已死。催着女婿娶去。新娘進門。揭去蓋頭。游夏流見了這副嬌容。魂都幾乎嚇走。至晚到床上交合之時。游夏流以爲這件未破的原牝。比那久經的糞門自然緊就許多。用上若干的唾沫。生怕唐突了他。輕輕緩緩的送進。竟汪洋無際。那卜氏以爲嫁了丈夫。或者僥倖有個絕大的物事。一者試試新。二者圖可以供終身之樂。只見他爬上了肚子。聳了兩聳。還不知弄了進去不曾。他早已伏着不動。心中雖然着急。纔頭一夜。不好便問。次夜仍然如是。游夏流已是兩度春風。多銀尚未知何味。忍不得了。起來一看。軟叮噹活像吃醉的和尚纔吐過了。搭拉着個光腦袋。口中還淌黏涎。不要說比那兩個搖鼓的同那一老丐的三分之二。還只得那賣花小子的十分之七。你道像個甚麼。

\begin{quotation}

身微體細。渾如絕大之蟶乾。

頭小腰躬。宛似極粗之蝦米。

\end{quotation}

且又是一把上好的解手刀。又小又快。多銀一見。眞合了古語二句道是。

\begin{quotation}

三尸神暴跳。七竅內生煙。

\end{quotation}

他不責自己的過大。只怪丈夫的忒小。又急又怒。那裡還按納得住。就一頭撞去。混打混咬。大哭大叫道。你這麼個樣子要甚麼老婆。豈不躭誤了我的少年靑春。我這一世怎麼過得。叫我守活寡。還要這命做甚麼。便拿過褲帶。光着屁股。跳下床來。要在床欄杆上上吊。那游夏流起先見他哭罵是羞。後來被他打咬是疼。他的俊龐頭髮撞散了有一扠長。披了一臉。配着那個奇形異狀的臉。純乎一個活鬼。又是那怕。及至見他要上吊尋死。心中着急。忙下床跪在面前。抱住他兩腿。哀求道。你息息怒罷。是我父母不是。從小定了你。怪不得我。雖然我沒本事。我像父母般孝敬你。凡事遵你的法度。你將就過罷。那多銀那裡肯聽他。哭哭啼啼的罵道。你就把我當祖宗供着。也抵得上那個東西麼\footnote{下流人的祖宗不及一個陽物。可嘆。}。還掙着往上拴帶子。游夏流見勸不住。急得沒法了。此時緊緊的抱着他的腿跪着不放。多銀因仰過身子勾着拴帶子。下身就往前腆着。游夏流那張嘴正對着他的陰門。人急智生。也顧不得纔剛洩出那黏達達的陽精。忙把口對了陰門。一口含住花心。咂了幾下。伸出舌頭替他亂舔。又把舌尖伸入戶中絞動。那多銀從未經過此事。覺得癢癢酥酥。有些趣味。就住了聲不哭。也就不掙。兩隻手垂了下來。也不拴帶子了。只見他把兩腿鬆開了些。小肚子往前腆着。拿陰戶就他的嘴。游夏流見事尚可挽。心中暗幸。道。夠了。這就是父母的陰靈。祖宗的保佑。想出這個妙策。忙跳起。把他抱到床上睡下。將兩腿推起直豎。牝戶大張。這一舔將舔有一個更次。比世上無恥的人舔那有錢大老官的屁股溝子還舔得利害。直舔得舌根都酸疼了。弄得滿臉黏涎。然後纔得安然睡了。多銀雖內中深處不得受用。門內却從不曾嘗過這番妙境。也就息了幾分怒氣。每夜游夏流以舌代屌。定要交媾一番。連行經之日還不饒他。弄得那游夏流滿鼻子臉上口中全是鮮血\footnote{巧言利舌之報。}。活像那屎皮無賴的光棍。自己打出鼻血抹上一臉騙詐人的樣子。把他那根有陽物之名。而無陽物之實的匪具。反置之高閣。有一調黃鶯兒說他二人道。

\begin{quotation}

夫婦本前緣。却因何出怨言。聲聲不願成姻眷。因他細纖。將他打撏。哀求那有垂憐念。氣求捐。願常舔絞。雖臭不憎嫌\footnote{有夫婦二人打嚷。値一官過。聞之。怒道。可拿了出來。衙役將他夫婦拿出。官道。我在此過。你夫婦敢如此放肆。是爲何事。婦人道。他要舔我的這東西。我不肯給他舔。是他嚷鬧。與我不相干。官聞之皺眉。向那男子道。你這奴才。那爛臭魚味的鹹東西。你要舔他做甚麼。可見此官亦曾舔過。何況於游夏流也。}。

\end{quotation}

他這懼內的勢子。不但要算一個都元帥。大約天下僅一。古今無二的了。間或卜氏一罵。他就跪下哀求。娘長娘短的叫。寧可我受責。不可你生氣。有時要打。他便匍匐地下。不但小杖甘受。就是大杖也不肯走。人家的兒女要像他這樣孝敬。也就算得個孝子順孫了。多銀也竟被他柔尅其剛。倒也相安無事。至於掃地鋪床。燒茶煮飯。都是他竭力供役。是不用說。連多銀洗脚雞眼兒是他作嘴兒挑。到晚來。馬桶兒是他隨鼻兒倒。但他只一到了外邊。就不是他了。高談濶論。巧言如簧。若聽得有人說某人怕老婆。他便語中帶刺。也不管那人面皮削盡。譏誚得那怕老婆的連地洞都鑽下去。他一日同着五六個朋友說某人怕婆。某人懼內。正說得高興。內中有一兩個也風聞得他家中閫政嚴肅。不好直道。說道。世間怕婆的也甚多。就是兄恐也不能免。他正色道。這是甚麼話。我家內人家中的事敢違我一毫麼。我說往東。不敢向西。設或惱了我。見敎他幾下還不可知。我們堂堂丈夫。可是那種怕老婆的忘八。諸兄敢同我賭個東道。到我家看看我的規矩。他這不過是個好看的話。料着誰同他賭。不想內中一個尖酸的說道。就是如此。我們每人出一錢銀子。到府上去。果然令政凡事聽你指揮。我們算輸了請你。若稍有違拗。你加倍罰出來還席。衆人聽了。道。有理有理。就湊出銀子來遞與他。他沒得推了。只得說道。等我回去製辦了。兄們下午些到我舍下來。衆人散去。他買了些肴饌果品。打了二三十斤酒。拿往家來。多銀一見了這些東西。嚥了幾口唾。不由得口水流出。笑逐顏開的道。這是那裡的美物。他放下了。走近前。雙膝跪倒。道。我的親親的娘。我求你個恩。多銀道。甚麼恩。你說。他道。方纔在外邊有幾個人。這個說那個的女人不賢慧。會欺負丈夫。那個笑這個的老婆很憊賴。專打罵漢子。忽然問到了我。我極口說我家那娘。天下尋遍了也沒有這樣賢慧的第二個了。當家立計是不用說。支人待客其實沒有。衆人不信。要同我賭。出了銀子。叫我備個東。他們來看看可是果然。我想我素常又沒有好東西孝敬你。借着這個意思。買些好物來。你揀好的留下受用。下剩的拿出去給衆人吃。只求你燙酒拿菜。憑我〖口么〗〖口么〗喝喝。你都忍着些。我不過是假意。好騙人的嘴吃。我何敢〖口么〗喝你麼。你又得了一個大賢慧的好名。好娘。你要依了我。我沒得報你的恩。我今日晚上足足替你舔半夜。多銀見有好的吃。又許替他舔半夜。便道。我依你這一回。下次再不許了。游夏流見他肯依。歡喜的叩了個響頭。起來忙將果肴選上好的裝了兩盤。送與多銀。又趕忙燙了一大壺酒。並鍾箸送上。然後自己囗都預備停當。又把客坐內桌椅板凳設下。多銀吃了這一個醉飽。也歡歡喜喜的去燙茶燙酒。午後衆人來了。讓了坐。就〖口么〗喝捧出按酒來。那多銀也就掇出。他出來接了擺上。陪了坐着。道。這都是我賤內收拾的。連桌椅都是他親手抹拭。我買了東西來家。只吩咐了一聲。我就去睡了一覺。衆人也還半信半疑。只見他〖口么〗喝道。送酒來。果然多銀聽見就送了酒來。一會又〖口么〗喝道。酒太熱。溫着些。少刻就送了溫些酒來。衆人都暗道。怪不得他說嘴。果然好家法。那游夏流見應了他的心。越發〖口么〗〖口么〗喝喝個不住。日色已沒。點上燈來。他又〖口么〗喝道。酒太冷了。換熱的來。這樣沒用。那多銀此時有了些厭煩。在廚下燙酒。多飮了幾杯。又懶動了。聽得又叫。怒上心來。拿了一壺熱酒。走到門外邊。道。拿熱酒去。游夏流自不知機。把威風使得太過。竟忘了他的利害。興抖抖的走來接酒。多銀一手遞酒。一手張開。連耳帶腮。一個大漏風嘴巴。打得響聲震耳。淸脆異常\footnote{趣甚。}。游夏流領敎了這一下。頭眼發昏。幸得他口舌伶便。跑出外邊。用手指着內裡道。我就打你個酒冷。故意恨恨的道。打的還少\footnote{好急智。非極伶俐人不能如此。}。然後坐下。衆人聽得。以爲眞是他打老婆。大家尋思道。爲朋友吃酒。叫他打妻子。倒都不好意思。起身作辭。那裡知道是他捱了這樣一下。游夏流見局面有些變了。還有些打得昏頭昏腦的。也不敢留。送了出去。關門進來。忙把傢伙收拾了。到房內看時。那多銀見人去了。還剩得有幾壺酒。不管冷熱。一氣全裝入肚中。已上床脫光睡下。游夏流見了。不敢消停。恐他等急了生怒。忙就上床。多銀酒多興發。正等他來舔。游夏流剛把嘴對了陰門。舌頭還未曾伸出。忽聞得一陣奇臭。一個惡心泛將上來。幾乎吐出。連忙扭轉頭忍了下去。你道游夏流是舔慣了的。今日何故如此。一來多銀每常終日高坐。一應都是游夏流服事。他腿胯中沒有甚麼汗汚。略有些臭。慣了還忍得下去。今日他在廚下燒火燙酒。熱氣薰蒸。又兩頭走着送酒菜。汗透了。況且他生性奇懶。這件東西輕易不見水面。所以臭得利息(害)。二則游夏流每常老早吃了飯。上床時已半飢了。今日不要錢的酒肴。他道主不吃客不飮。雖然不住的讓。人客還未得半飽。他足足飽到十分。因此一聞着就惡心上來。多銀正等着情急如火。見他這個樣子。大怒道。我爲你辛苦了半日。等你來舔。做出這個樣兒來。敢是嫌我的臭麼。他忙道。我的娘。松門鮝一般噴香的好東西。怎得臭。今日飽了些。纔要打飽饐。恐怕酒氣薰了你的香東西。得罪了他。我何敢嫌你麼。便道。我的舌頭怕不乾淨。去漱漱口來。跳下床。摸了團綿花。將兩個鼻孔塞緊。漱了漱口上床。多銀見他塞着鼻孔。罵道。你明明的是嫌我的臭。還敢強嘴強辯。不然你爲甚麼塞着鼻子。游夏流忙道。我的娘。你把我的好心做了驢肝肺了。我這兩日有些傷風。怕一時間打噴涕。嚇了你的寶屄\footnote{昌氏之鐵屄已奇。而多銀又有寶屄尤奇。但彼鐵者尚有鐵意。此寶竟毫非寶也。可爲謬矣。}。那怎麼處。你怎麼倒反怪我不是。多銀信以爲實。纔不嘖聲。游夏流見支吾過去。心中暗喜。道。夠了夠了。忙扶起他兩腿。伏身就舔。多銀兩手抱住他的頭。摟得緊緊的。對着陰門。整舔了半夜。舌根都腫了。第二日連話都說不明白。兩三日後纔好了。權且按下。纔說這位怕老婆的英雄。再講那個懼內的好漢。要知是誰。看\endnotemark[2]下回分解。



\endnotetext[1]{此句原有眉批云:「以妓者行令,說作過令尊,雙關一罵。」}

\endnotetext[2]{以下原缺,據文義補。}

\setcounter{footnote}{0}

\theendnotes

\part*{姑妄言第十一卷}
\addcontentsline{toc}{part}{姑妄言第十一卷}
\markboth{姑妄言第十一卷}{姑妄言第十一卷}

鈍翁曰。宦蕚蠢然一物。惟於此道中頗有機智。侯氏可謂悍而猴者。尚落在他術中。始急之。得嬌花之咂。終服之。得嬌花之替。宦蕚豈非猴而又猴者耶。游混公敎了他幾年。一本三字經不曾讀熟。司傅只數夜。敎得他如此聰明。誠不愧名爲司傅。可見人之不可不擇良師也。

童自大家的春宮未曾試新。反是宦蕚看了先來學樣。眞正奇想。

香姑之嫁馬台。此不幸中之幸也。若嫁了牛耕一類。這等精靈般好淫女子。豈不又是第二個奇姐。至於偷老和尚。似有定數焉。香之一字。豈非禾日二字成之者耶。或曰。香字從甘不從日。此語未免強捏。予笑曰。不但近寫皆從日字。且甘者甜美也。所以更好。

馬台之娶香姑。隱寓巧妻常伴拙夫眠一語。爲千古佳人所配非偶同聲一哭。但馬台太囗。香姑似太難爲情。然而後來又大得其呆利。所謂塞翁失馬。得禍者未必非福。

香姑尋事丫頭僕婦一段。人家主母不悅下人。眞有之事也。寫衆不知死活之惡奴。把小人心腸一筆寫盡。誠所謂之奴才。

馬士英實產於苗婢。非辱之也。至於蹇氏阿呆馬台諸事。雖係作者曲筆。若以馬士英之所爲。雖辱及九祖。猶不足以盡其辜。何況只辱其己身並妻孥耶。亂臣賊子。人人得而誅之。凡有忠義爲心者。閱此定當叫快。

侯捷奉差一段。若不一提。只開首見其一名。此後不知何往。豈非漏處。今一寫出。不但使侯捷不泯滅。且使魏忠賢不冷落也。

天下之遠莫過滇黔。他處人到者尚多。猶能言其民風土俗。至於滇黔。人遠遊者百無一二。即成有之。又未必能紀其事。今詳書之。使看者一開卷如同臥遊。亦一快事也。且夾敍此一段。亦有謂焉。如演戲至半本時。雜以跌打彈唱做一間斷。使眼目略新一新。然後戲子重復上場。亦更有趣味。

劉文韜汪時珍實有其人。眞有之事。雖與本傳無涉。然報應顯著。故引入以作負心人頂門一針。

\chapter*{姑妄言卷之十一\\
第十一回 宦蕚逞淫計降悍妻 侯氏消妒心贈美婢\endnotemark[1]\\
附 奸禿享嬌姿 欽差遊異境}
\addcontentsline{toc}{chapter}{第十一回 宦蕚逞淫計降悍妻 侯氏消妒心贈美婢}
\markboth{第十一回 宦蕚逞淫計降悍妻 侯氏消妒心贈美婢}{第十一回 宦蕚逞淫計降悍妻 侯氏消妒心贈美婢}

話說宦蕚自錢貴家回來。到家下了馬。慌忙走到上房。他因見了錢貴那種風流標致。心下十分動興。見侯氏已脫了衣裳在床上。斜倚着枕頭。一半截蓋着被。宦蕚是(走)到跟前。道。奶奶。你還沒睡呢。侯氏道。你往那裡去了一日。此時纔回來。我等着你呢。宦蕚聽見這話。一面忙脫衣服。一面說道。今日賈兄弟家請吃酒不肯放。此時纔散了回來。便上床摟住。道。我來親熱了。你不要變臉。侯氏笑道。你好情來親熱。我惱的是甚麼。宦彎道。我前日童兄弟請了去吃飯。他書房裡放着個册頁。我揭開看看。原來都是男女幹事的。我記了幾個樣子。來同你做做看可有趣。你要依我擺佈纔好。侯氏笑着點頭。宦蕚將被掀開。把他妙臀墊起。掉轉身做了個倒入翎花。弄了幾下。侯氏道。不能盡根。又抽得不爽利。叫他另換個樣子。宦蕚便將侯氏扶起。伏在枕上。又做個隔山取火之勢。侯氏嫌不着實。遂臥倒。直舒雙足。叫他上身來弄。宦蕚道。這個樣子也不知弄過幾千百回。熟得一點也沒趣了。你上我身來。做個倒澆蠟燭還新鮮些。侯氏此時任他所爲。隨手而轉。一些也不拗他。宦蕚仰臥在下。將屁股墊高。叫侯氏跨上身來。對準幾坐。盡根而入。他又使力墩了兩墩。只剩二卵在外。間不容髮。侯氏覺得頂着裡面花心。酸酸癢癢。從未得此樂境。宦蕚一手扳住他的腰。一手扶着他的股。侯氏也將手兩邊拄定。二人一齊用力。上下衝突。一個下坐。一個上迎。下下不離花心。戰夠多時。侯氏丢了一度。伏下身來。將舌尖伸入宦蕚口中。咂了一會。他得了這場樂趣。歇過片刻。淫興復起。重又大弄。宦蕚因先在錢貴家見了他那段嬌嬈丯韻。厥物已翹然高舉\footnote{此虛(處)寫宦蕚淫心如此之熱。方顯後來能逼慾之奇也。}。到此時火氣已過。把恃(持)不住。一股股冒將出來。其物漸萎。侯氏正在高興。忽然覺下邊不見了妙筍。用手一摸。已軟叮噹如瘋癱一般。問道。你怎麼正弄着。成了這個樣子了。宦蕚道。我已洩了。來不得了。侯氏淫興正濃。如何肯住。只得跨了下來。替他百般摩弄。只是不起。急得侯氏將他項上咬了一口。罵道。你這狠心的忘八。故意使促掐奈何我麼。宦蕚道。我怎敢奈何你。他不肯硬起來。叫我也沒法。侯氏道。這怎麼樣處。你可有個法兒弄硬了他。宦蕚笑道。有一個妙法。須是你去央及央及他。或者起來也不可知。侯氏擰了他一把。笑道。有這些鬼張。你叫我怎樣央及。宦蕚道。我聽得人說。這東西軟了。容易再不得起來。須是拿嘴一咂。纔得硬郞(朗)。侯氏瞅他一眼。道。纔在那裡頭弄了一會。髒巴巴的。怎麼好咂。你不要急惱了我的性子。我一口咬下床呢。宦蕚笑道。你要咬。我也講不得。你旣嫌髒不肯咂。我自己又夠不着咂。叫我也沒奈何。侯氏急了。道。你前日愛嬌花。偷看他溺尿。叫他來替你咂咂罷。宦蕚道。罷罷罷。想前日無心略張了一張。打了一頓棒槌。今日要叫他來咂。連渾身的骨頭都要碎了罷。這事做不成。留着我的命同肩膀骨要緊。侯氏笑嘻嘻將他打了一個巴掌。罵道。怪奴才。偏有這許多鬼話。我不好叫的。你叫了他來。那宦蕚巴不得\endnotemark[2]這一聲。高叫嬌花。嬌花正在要睡。聽得叫他。走到床前。宦蕚一把拉着他的嫩手。要拉他上床來。那丫頭見侯氏醒着。嚇得掙着要跳。宦蕚笑嘻嘻拉住不放。侯氏道。你就上來罷了。浪的是甚麼。那丫頭見主母吩咐。不敢不依。就爬上床來。宦蕚捏着陽物。笑向他道。叫你來嘗新。你替我咂咂這東西。那丫頭還是女兒。從不曾見過這三怪的物件。將頭別轉。但這件東西。但是男子漢個個腰中都有的\footnote{大不然。只此書內魏忠賢鄔合二人便無。}。何爲三怪。他不曾剃過頭。那個腦袋比和尚頭還光得古怪\footnote{和尚頭焉能及此。若如此頭。省了多少剃頭錢。}。一點骨頭也沒有。比有幾個臭錢人的腰還硬得古怪。從來不見天日。比那走長路人的臉還黑得古怪\footnote{尚不止此三怪。一隻無珠的小眼。見了美婦便跳起來。豈非一怪。又沒有鼻子。聞見婦人的陰味。便鼓起戚來。豈非又是一怪。}。那丫頭乍見這個怪物。要說是個禪僧。却又有一部落腮鬍鬚。要說是留鬚的宗門僧。却又無眼耳鼻舌。要說是道士。又光着頭沒道冠。要說是大鼻子回回。却又鬍子不曾剪。羞得滿面通紅。就像惱這怪物似的。背過了臉不理他。侯氏急等着咂硬了好做事。見他不來湊趣。反做出這個樣子。急得罵道。小淫婦。好意叫你來嘗嘗新。倒做出這麼個浪樣兒來。不要討我一頓好皮鞭。這丫頭也巴不得嘗嘗是個甚麼味道\footnote{要識此味須下口嘗。上口嘗之何益。}。細看看是個甚麼款式。先恐主母吃醋。故做羞態。不好就來領情。今見主母固請席不赴反怒起客來。也就低垂粉頸。款啓朱唇。一手捏着陽物。一手攥着腎囊。將龜頭含入口中。如小兒吮乳一般。仔細端詳。方知這管紫肉簫並無孔竅。只品得而無音。兩個毛栗子却是核桃皮包着。正是。

\begin{quotation}

不覩奇形狀。焉知此物新。

\end{quotation}

宦蕚素常被這丫頭引得魂都不在身上。他較侯氏標致了許多。每常連多看兩眼還恐奶奶生疑。欲求親一嘴如登天之難。今番得他如此做作。可有不動興之理。不上一刻。那厥物跳將起來。分外堅硬。那侯氏先正高興之時忽然中止。正在難過的時候。又見他二人如此舉動。越發急得屁股只是亂扭。宦蕚見他急得可憐又可笑。遂道。我來了。侯氏聽得忙忙仰臥。兩足直豎如兩柄雉扇一般。紅溝赤露。候他入來。宦蕚一下揷將進去。緊緊摟定。對嬌花道。你不許去。可在後面推着我的屁股。我好用力。那丫頭也正要賞鑒賞鑒這樁故事。遂依他。儘力前推。那侯氏是熬急了的人。屁股不住亂攧亂簸。又得嬌花在後推着。下下着實。覺得自嫁夫以來。未有如此之樂。抽拽多時。侯氏忽然大叫道。哎喲。罷了我了。只見他面如火熱。鼻靑唇白。眼閉口張。兩腿掉了下來。雙手散於褥上。四肢癱軟。遍體酥麻。呼呼睡去。宦蕚見他如此。知他樂極。輕輕拔了出來。一把摟住嬌花。連親了幾個嘴。悄叫道。心肝。我想你久了。快些來弄弄。那丫頭年已十五六歲。久矣知竅。每常礙着主母。不敢妄想。今見了這種光景。興不可遏。色膽如天。也怕不得許多。雖假意不肯。却笑吟吟憑着主人解帶脫褲。並不推阻。宦蕚將他放倒。上身來摸着關竅。往裡直衝。一下進去了大半。這丫頭還是個處子。宦蕚因是想他久了。此時高興到十分。竟忘了他是未破瓜的女兒。猛然一下。直疼得那丫頭跳將起來。眼淚汪汪。又不敢出聲。恐驚了主母。起初這丫頭先替他咂時。嘴中雖嘗了異味。臍下那竅中尚不知個中是甚滋味。也覺得十分興動。後見同主母這一番大弄。又見主母弄得那樣光景。以爲是一件有樂無苦的事。一來大意。二來不防他竟是這樣勇猛直前。方知這果子見(先)辣而後甜。開首不是好吃的。幸而先見他們高興時。小牝中也有許多淸水流出。宦蕚的陽具又是侯氏陰精泡透。兩家俱還滑溜。所以尚不致狼狽。宦蕚見他吃了虧。又憐又愛。方輕憐重惜。慢慢用水磨工夫做將起來。這丫頭雖還未曾得了樂處。也就不似先那樣苦辣。這正是。

\begin{quotation}

嬌姿未慣風和雨。吩咐東君好護持。

\end{quotation}

抽弄了一會。也覺稍有甜頭。親嘴咂舌。宦蕚的陽物塞在他的陰中。他的舌頭送入宦蕚口內。從此上下互相更摸着交媾。正在綢繆之際。不想那侯氏又好醒來。他原非瞌睡。因弄得渾身通泰。心中快活至極。不覺酥酥睡去。故此不多時就醒轉來。身傍不見了宦蕚。探起身子向脚下一看。見他二人正做得好。臥榻之前。人鼾睡尚然不可。可是人容得大弄的去處。不由得醋氣發將起來。妒心頓起。罵道。好大膽。你們做得好事。一骨碌爬起。將宦蕚打了兩掌。就伸手去抓丫頭。那丫頭見勢頭凶惡。也不暇穿褲子。光着屁股。一滾跌下床去。將膝蓋的皮都跌蹋。猶恐主母拿住他咬他的肉。忍着疼爬將起來。跑出外邊去了。那侯氏精着身子就要下床來趕。宦蕚死抱住。道。奶奶。一來看風吹了你的熱身子。二來不管(關)他事。饒了他罷。侯氏打了他一個嘴巴。道。你瞞了我做這樣欺天大膽的事\footnote{謂之大膽則可。欺天二字太甚。古云夫乃婦之天。當云欺地方合理。}。還敢替丫頭討情。宦蕚道。我怎敢瞞你。又怎敢替丫頭討情。我的東西方纔軟了。一時起不來。是你好情叫他來替我咂。及至咂得硬了。原要同你着實大弄一番。不想你又睡着。我不敢驚動你。我的這東西一時脹得難過。沒奈何。拿他來當災\footnote{侯氏當說。這災我怎肯讓他當。}。你倒打起我來。侯氏怒道。你還同我強嘴。恨恨的又舉掌要打\footnote{妒婦之心於此可見。}。宦蕚陪笑道。這都是我的不是。起初時我要同他弄\footnote{正所謂蜜語。}。他怕你。死也不肯。是我強按着他弄的。果然與他毫不相干。你若打他。可不是我害了他了。你請想。要是我同他有甚私心。還敢在你身子傍邊大膽弄麼。你若疑我是假話。不信你看我這東西。此時還脹得這個樣子。侯氏低頭一看。果然那根厥物還直豎在那裡\footnote{好硬干證。}。笑吟吟一把攥住。罵道。你這作怪的東西。一時軟起來就像綿花。叫你硬硬也不肯。一時硬起來就這樣作怪。就想吃一看二起來。宦蕚見他不十分有怒。覺事尚可回。不由分說。將他放倒。二足分開。從中直搗。又大弄了一場。方纔睡下。這一下弄得侯氏如醉如癡。把先那些怒氣竟不知何處去了。新(鼾)呼睡去。這宦蕚想嬌花方纔那一番興趣。喜道。這幾年的心願。今日方纔酬了。怎得同他大弄一場纔快活。因看看侯氏。見他已睡熟。想道。他只自己要弄。就不與我一些空兒。纔同丫頭親熱一會。將有樂趣。他就吃醋未了。必須治倒了他。纔可同這丫頭做得快暢。主意已定。次早起來。正要出門。只見鄔合走得滿臉汗。唿噓噓的進來。宦蕚見了。問道。你到那裡去來。走得這樣喘急。鄔合道。外邊有一件新聞的事。晚生見了。特來奉吿。大老爺可有高興去看一看。宦蕚道。是甚麼事。鄔合道。是鳳陽馬總督家媳婦爲了奸情。好一個標致婦人。只得十六七歲。被他丈夫拿到中城察院。因官府家中有事。還未上衙門。都還在門口等候。晚生見此離不遠。故來問老爺可有此興一往。宦蕚道。我也正要出門。順路去看看也有趣。叫家人把驢子叫了一匹來與鄔合騎\footnote{江南與北直相同。各巷口皆有雇驢者。頗覺便宜。}。他上了馬到了那裡。果然見圍着許多人。宦蕚打馬上前一看。見了個十六七歲的男子。穿得甚是華麗。形狀癡癡蠢蠢。倒也還不十分醜惡。却兩管鼻涕大長的拖在口辱(唇)上。口吻邊不住淌憨水。宦蕚不由得腹中暗笑\footnote{勸君且勿笑。十二三歲時與彼是一對也。}。只見他呆呆睜着兩個大眼睛。東望西望。坐在那裡。倒有七八個豪僕在身傍貼着。一個少年嫩婦。生得十分美麗。靑布衫藍布裙。一頂包頭齊眉罩住。坐在一乘沒簾子的轎內。低垂粉頸。那臉白嫩得像豆腐腦兒一般。却裡邊襯出胭指(脂)鮮色。羞慚滿面。淡淡雙蛾蹙着。以眼觀鼻。以鼻觀心的坐着。一個老和尚白髮皓鬚。短短的。一頭一嘴。像魚刺一般。繩子綁住。只穿着一件夏布衫。光着屁股。拴在轎槓上。閉眉合眼。蹲在地下。四五個雄糾糾的惡奴守住。宦蕚也是有三分呆氣的。問傍人道。怎麼這樣一個少年婦人同一個光屁股的老和尚在一處。傍邊一個人笑道。他們爲的是奸情事。這個小婦人也是沒有穿褲子的。他們的兩條褲子都在那體面小夥子傍邊放着不是〈甚〉麼。宦蕚一看來。見一個捲兒。是一條大紅縐紗的。一條夏布的\footnote{和尚所穿之衫並此褲。乃香姑做與他裝新者。不想來此出醜。}。宦蕚又問道。這奸情是怎麼樣起來。被他男人拿住。這樣小年紀婦人怎愛這麼個老和尚。是甚緣故。傍人道。誰知他家的詳細。我們也是纔來看看。聽是盡是這麼說。少刻官府到來審問。自然就知道了。宦蕚也就下了馬。同鄔合到個茶館中坐着閒話。等看熱閑(鬧)。且說華麗而癡蠢的這位公子。他是風(鳳)陽總督馬士英的令嗣。名字叫做馬台。天生的一個奇物。一毫人事不知。吱着個舌頭。不但說的話人不十分懂得。而且連說還說不全。吃飯人給他吃就吃。若不叫他住。就儘着吃個不休。要不與他吃。他也就罷。也並不知要\footnote{論起來實在有(是)有福的人。}。總不知甚麼叫做飢飽\footnote{遇荒年。窮漢有此肚腸。眞是大造化。}。譬如吃東西。人一時偶然忘了叫他住。他直吃得肚腹脹得膨脝。定吃完而後已\footnote{此正所謂有大量方有大福。}。穿衣亦是如此。也不知甚麼叫做寒暑\footnote{頗有仙意。}。虧他一個乳媼養氏憐他。到了這樣大還像孩提般看待。早起晚睡。吃飯穿衣。還是他照看。他父親馬士英係貴州人。馬士英之父名喚馬達。也還有(是)個渾厚的老儒。中年乏嗣。要娶妾無力。恰好有人家賣的一個苗婢。有二十多歲。那家因他任甚事都不懂得。又是一個烏黑的醜臉。憎嫌他。拿出來賣。價錢甚廉。馬達要圖他生子。原不取他容貌。遂買了收用在身邊。剛進門一年。遂生了馬士英\footnote{這的的確確是個眞馬駒。}。却有件奇事。這苗婢一般腹中也會懷胎。陰戶也知誕子。也知乳哺。却舉動說話與人大別。是俗語說的。九分銀子打了十個銀娃娃。連一分人氣兒沒有。這樣個蠻物偏有好陰戶。他生的這馬士英。也竟能中舉中進士。又還做官。而且做頭宦。雖如此說。然而與中華婦人所生者到底有些各別。何以見得。馬士英生性奸貪苛刻。那種奇異心腸却大異於人。譬如人說他壞。他知道了。更要壞得盡情。人說他奸。他聽得了。定然奸到至極。當日人說王安石性拗。他生來是這樣個牛性倒還罷了。這馬士英却又有古怪處。他一生難道就無絲毫好處。設或有人稱念他好的。就更該好了。他却決不肯照那好處去行。定要改壞了纔罷。這豈不是異乎於人。這馬士英頗生得聰明。十數歲就在庠。他二旬之外父母皆故。只他同妻蹇氏\footnote{蹇者。驢也。馬與驢交便生騾。原是雜種。馬台之出於阿呆。原無足怪。}。並一老僕婦。同一個小廝阿呆。四口過活。這阿呆就是他生母的姪兒。也是個苗種。同馬士英嫡親姑舅弟兄。因他是異類。便把他一半當僕。說起這阿呆來。比他那姑娘老苗婆更蠢。眞呆得出奇。一些人事也不懂。蹇氏見他呆頭呆腦。耍他道。你拿一塊炭替我洗白了來。他當眞拿一塊去先(洗)。從早至晚。洗得粉碎。拿了些碎末來。向蹇氏道。我洗了一日。都洗破了。也不得白呢。蹇氏忍不住大笑。一日吃雞。剩了些給他吃。蹇氏道。你吃這雞這樣的好東西。仔細掉了耳朶。你拴着再吃。他果然拿根線拴着纔吃。吃完了。摸了摸。向蹇氏道。奶奶。我的耳朶沒有掉。蹇氏笑向馬士英說知。以爲取笑。不意馬士英聽了暗暗歡喜。你道爲何。馬士英在外縣庭(處)館。一年只端陽中秋年下回來數日。這蹇氏生得貌旣妖嬈。性又流(淫)蕩。馬士英常不在家。恐他少年嫩娣(婦)。做出些偷期的事來。見這阿呆呆至於此。尚有何慮。便叫他在堂屋中睡。不但不防他偷這嫂主母\footnote{嫂主母之稱甚奇。或苗子家之禮耶。}。且恐蹇氏偷人。叫他做個隘屄使者之職\footnote{此等官職。不知服何衙門所轄。}。這蹇氏性極好淫。馬士英不但孽具鄙猥。且本事又甚是不濟。他出去處館。三分是爲餬口之計。倒有七分是躱避差徭。故馬士英喜阿呆之呆。可無後患。且說蹇氏向日馬士英未出去敎館之先。雖夜夜在家。也不能滿他之欲。自從他出去之後。創了個新奇異想。雲貴有一種土產的黃蘿蔔。粗細雖與胡蘿蔔相等。却長將一尺。他每日買兩根粗大的。刮得光光滑滑。留爲夜間取樂之具。每到那得趣的時候。呼曰黃心肝。黔中天氣暑熱。這蘿蔔四時不斷。他守着這姓黃的假夫。倒也不生他想。一日。六月炎天。他夜間那老黃消遣了一會。未免自己費力。汗出如漿。嘆道。這黃心肝處權則可。若論守常之道。如何及得那肉寶貝。偶念及此。慾火炎蒸。忽然口渴。要些涼水壓一壓。他住在東邊房內。那老僕婦在西間廚下睡。叫了幾聲。不見答應。夜靜更深。淺房窄屋。又不便大聲喊叫。只得起來。披了一件長衫。拿着燈到西屋去取水。剛出房門。不想那阿呆精赤條條睡在一條春凳。正腰中一個陽物竟有六七寸長。又粗又大。橫拖在腿上。蹇氏一眼瞥見。由不得渾身一麻。輕輕走到跟前。拿燈照着。細細賞鑒。越看越愛。此時口中竟不渴。心裡反劈劈的往上撞起來。臉上如火燒的一般。暗忖道。不想這個呆人竟有如此奇物。我若偷上了他。不但有許多樂境。且夫主決不動疑。豈不大妙。纔要伸手去推他。忽然心中一愧。道。雖如此說。到底不好意思罷。忍忍罷。也不取水。仍走回房。剛要上床。那心中像有根繩子拴在堂屋裡一般放不下。又拿着燈走出來。又照着細看了看。纔要伸手去捏捏。又忙縮住。道。不好。恐或有人知道怎麼處。方纔轉身。忍不得又回頭看了看。此時慾火如焚。也顧不得了。便走到跟前。一把將他陽物攥住。那阿呆猛然驚醒。燈光下見是主母。嚇得幾乎要哭。說道。我再不敢了。奶奶饒我罷。蹇氏笑着。悄聲道。不許做聲。跟我去。有話問你。他爬起來。蹇氏攥着陽物不放\footnote{此所謂不忍釋手也。}。似牽牲口一般。到了房中。將燈放在桌上。到床沿上坐下。笑着問他道。你這東西叫做甚麼。他道。叫做雞巴。奶奶饒我罷。我再不敢了。蹇氏又笑問道。你這樣大了。可看見過屄沒有。他道。我不知道甚麼甚麼叫做屄。雞我倒認得。蹇氏道。你可會肏屄。他道。那一回奶奶賞我雞肉。叫我拴着耳朶吃來。我會吃。蹇氏見他總不懂局。心中火發。把衫子脫了。光着身子。蹺開腿。指着陰戶問他道。你看這是甚麼。他看了看。道。這是相公的嘴。蹇氏笑得了不得。問道。怎麼是相公的嘴。他道。相公的嘴上有鬍子。這個也有鬍子。可不是相公的嘴。蹇氏一面笑着道。你上床來。他道。奶奶饒我罷。我不敢了。蹇氏拉他上床。自己仰臥着。兩腿大蹺。指着陰戶道。拿你那雞巴放在這嘴裡去。他道。奶奶饒我罷。我不敢了。蹇氏淫心大動。急了一身汗。想了想。爬起來。將他按倒。替他把㞠子一陣搓弄。他嘴中雖說奶奶饒我罷。奶奶饒我罷。那㞠子竟挺硬起來。蹇氏大喜。便跨到他身上。將陰戶對準。一連幾坐到根。不住起坐。阿呆口中不住的道。快活快活。蹇氏蹲坐了一會。丢訖一度。自己乏了。下來叫阿呆上他肚子來弄。阿呆道。奶奶饒我罷。再不敢了。蹇氏料道替他說也無用。拉他到身上。拿着他的陽物塞入牝中。他一眼見枕傍蹇氏用的那根蘿蔔。一把拿過來。道。奶奶。賞我吃了罷。便放在口中吃。一面吃着一面道。好吃好吃。蹇氏笑着道。賞你吃了。你要依我\footnote{蹇氏可謂捨黃心肝而取肉寶貝矣。}。遂兩手搊着他。一上一下的抽。抽了一會。他竟自己一出一進的扯拽起來。蹇氏喜得非常。不意阿呆陽物旣大且甚長久。足足不歇氣抽了有一個更次。蹇氏丢了幾度。眞出意想之外。許久他纔完事。就伏在蹇氏肚子上。蹇氏也心愛他。馱了一會。有些壓得慌。叫他下來。他方下來。蹇氏囑他道。這事對人說不得的。連那老婆子也不許對他說。相公要知道了。活活打死你。我的這個也就再不得給你弄了。他也知連連答應。此後蹇氏夜夜叫他相伴\footnote{賣蘿蔔的少了一個主顧。}。馬士英毫無疑忌。後來馬士英連捷做了官。蹇氏房中丫鬟僕婦多了。同阿呆做不得樂事。每向馬士英誇他老實。不但是貧賤時的舊人。且又是婆婆的親姪\footnote{還有一句。而且又是小夫。}。勸馬士英着實擡舉他。馬士英滿心以爲他向日監屄有功。也十分擡舉。鮮衣美食的照顧他。但是要替他娶個妻子。蹇氏便再三阻攔。道。他呆到這樣地位。也會行夫妻的事麼。豈不躭誤了人家女兒。馬士英也就信以爲實。寔(蹇)氏一來是不能忌(忘)情於他。恐替他娶了妻子。他便到(別)戀。二來說他如此呆。便馬士英更不疑。以便別作主意。又勸馬士英在內宅門口另蓋一間小房給他住。叫他守宅門。馬士英因家私漸厚。也便依他。蓋了一間房子叫阿呆在內坐臥。看守宅門。馬士英那知是蹇氏的奸計。叫阿呆在門口住。以便得空好行幸他。但是馬士英出門赴席回遲。僕婦們都下去了。支開丫頭。偷空便到阿呆房中同他高興一度。如此也多次。一日。又同他去舞弄了一回。回到上房。剛纔睡下。馬士英來家。也就脫衣上床。便同蹇氏高興起來。他內中有阿呆所洩之精。馬士英問道。你這裡頭怎粘達達的。蹇氏謊答道。我這兩日淌白帶呢。馬士英道。你怎不早說。這是下寒的病。明日叫醫生來看看。次日。果請了醫生來診了脈。撮了兩劑藥。又還留下一服暖宮丸。蹇氏暗暗的好笑。後來蹇氏竟得了孕。遂生了這馬台。究竟連他自己也不知是誰人之種。要論這馬台之呆。自然是阿呆之子無疑。他生了這樣一位賢郞。無賢無愚。是大是小。都知他是個呆物。惟馬士英以爲是個蓋世的神童。雖唐朝的劉晏李泌囗囗囗囗囗囗囗敦篁都不能及。他還有一個譽兒癖\footnote{諺云。人莫知其子之惡。馬士英之謂也。}。開口便道。這些不知事的人說我兒子癡愚。不但他不癡。就癡也與他人癡得不同。他癡人自有癡福\footnote{諺云。庸人多厚福。未必似令賢郞之癡也。似令賢郞奇癡者。自必有一段奇厚之福矣。}。依我看來。他正合着古人的詩。豈不聞蘇東坡的詩麼。他道是。

\begin{quotation}

但願生兒愚且鹵。無災無難到公卿。

\end{quotation}

他見兒子到了十六七歲。要與他娶媳婦。旣要好門第。又還要十全的閨女。馬公子之呆。迥出尋常。眞是出於其類。拔乎其萃的呆法。是人人盡知。誰家大門第好女兒肯配與他。倒是蹇氏有知兒之明。見他如此之呆。有個要阻攔丈夫不可娶媳之意。偶然想道。阿呆當年何常會來。我敎他敎也就知道了。娶了媳婦進門自然會敎他。諒着這件事再沒有學不會的。且說那時有一個財主。就是那牛質。他的女兒香姑已長大了。出落得好個齊整人物。有幾句贊他道。

\begin{quotation}

此玉香還勝。如花語更眞。

柳眉橫遠岫。檀口綻櫻唇。

金釵翹翡翠。玉趾蹙緗裙。

更有超人處。淫辭滿腹新。

\end{quotation}

他不但生得模樣妖嬈。而且識一肚子好字。就是他母親計氏敎的。他十三四歲時就千伶百俐。也不去看那女史孝經正經書籍。專偷看他母親所蓄在枕蓆上與丈夫助情的樣樣奇淫小說\footnote{看得此書。竟比女史孝經易曉而有趣。後文方應此句。觀書者愼之凜之。}。他記性又好。看過全全記在胸中。這樣鬼精靈也似的女兒。看了這些風流淫話。可還貞靜得住。但無可奈何。只得死忍。巴不得早嫁一刻。早去效一刻之顰。誰知他這個老子是俗語說的。鄕裡人不識麒麟。是個有錢的牛\footnote{有錢之牛可比麒麟。如令麒麟聞之。不亦可嘆乎。}他只知錢財勢利動心。定爲門當戶對。他只圖趨炎附勢。也不管女兒死活。聽見馬家覓媳婦。情願與他。猶恐馬士英棄嫌他不是仕宦。反托牛尚書寫書去轉就。馬士英見是尚書公的族姪女。又聞得美貌聰明。將就配得過賢郞了。雖未必十分中意。但別人家都不肯與他。只得行聘娶了過來。他知道鳳陽先經過流賊之害。他家中銀子已積得無數。怎肯在這險地放着。故在京城買了大宅。留下兒子看家。他見兒子尚年幼。恐人偷他的銀子。將歷來所掙的宦囊。齊出五十萬來。着他的一個大管家其(吳)義。叫銀匠囗做五百兩一個的大錠。共傾一千錠。以爲傳家之物。況且這樣大銀是人再偷不去。即有大夥來打劫。他能拿得幾個。庶幾可以放心。不想與(吳)義串通銀匠。將銀子三百兩鑄成空殼。內中灌上二百兩黑鉛。他知道主人公的銀子。只有聚起來再沒有用出去的。這項大錠是萬不動的了。何妨分惠落得偷下了。他也無從查考。馬士英欺君罔上。刻薄屬吏小民。辛辛苦苦掙了一生。弄了這些賊賊(贓)。却被吳義欺瞞着他。輕輕巧巧。一絲力氣不費。將及分去一半。他將這些大錠都留在家中收貯。他只同妻妾們在任上。他胸中又有個成算。若流賊再來賜顧。十分擋不住。便把任上所蓄棄了。只同妻妾跑回來。還不失爲富家翁。且說馬台娶親之日。養氏把成親的話敎了他有幾千遍。他只翻着白眼。大張着嘴。也不答應。把那養氏急得咬牙切齒。一身一身的汗出。把嘴都說鋊了。他只當不曾聽見。那養氏也沒法。家下這些男婦何嘗不知公子是娶妻沒用的。但主人的意思誰敢阻勸。新人進門。拜堂行禮。交杯合巹。都是乳媼養氏指點。拉他作揖他作揖。按他叩頭他叩頭。就像提偶戲的一般。那香姑以爲自己生得如此才而且美。父親必定替他覓一個風流佳婿。誰知嫁了這樣個人形而獸質的物件。由不得淚如雨下。傍人都看不過意。牛質見親家不在家。要奉承老親翁。親自送女兒過門。惟有他見了這位賢坦。欣欣然道。眞好女婿。不媿爲貴家公子。渾厚有福。與尋常人家子弟自是不同。到了晚間。養氏附着馬台的耳。又囑了他許多話\footnote{諄諄至囑。奈公子聽之眇眇而弗曉。其奈何哉。}。急急又推他坐。替他把上下衣服脫了。放他睡下。然後帶了門出去。這香姑恨塡胸臆。但到了這裡。料跳不出去。或者他略通些枕蓆上之事。也還可聊解數年之鬱。先還坐着等新郞來替他解帶寬衣。做些成親的技倆。不想坐了一會。總不見他動手。看那位新郞時。已鼾呼大睡到華胥國去矣。他沒奈何。自己脫去上蓋。拉過一個枕頭來。在這一頭氣忿忿和衣而臥。千思萬想。一夜無眠。次日淸晨起來。只是痛哭。那養氏再三勸着。纔肯梳頭洗面。一日連茶飯也不吃。淚眼不乾。這養氏甚是不忍。勸他道。你旣嫁了來。哭也無益。雖然不是對頭。也沒奈何了。遂低低向他道。他從幼就呆。到如今這樣大。穿衣吃飯還要我敎。大約成親的事他是一絲不知的。我昨日傳授他幾千百遍。他仍不懂得。你旣是夫妻了。還怕甚麼羞。你到床上敎他。他或者也就會了。這個事是別人替你敎不得的。那香姑聽了這話。也就會意。住了些哭。到了晚間。養氏又替馬台脫光。放他睡下。又向香姑低低的道。你天長地久的夫妻。不要羞了。你放老辣些。敎導敎導他。勸他脫了衣服。也叫他一頭睡了。將被替他二人蓋上。自己纔去了。牛氏想着養氏的話。他雖呆。難道人生在世連這件事都是不省的。就呆到這地位不成。倘或敎會了他這樁本事。夜間也還可以消遣。想替他說。料道不中用。不若我拿手去摸摸是個甚麼樣子。一來見見識面。二者或經我嫩手捏弄。他竟知高興。也不可知。遂伸手到他腰中去摸。他雖然看小說知道此物生在腰下。却不知長在何處。自小腹之下順手摸去。不想他指甲尖長。剛剛把他陽物戳了一下。馬台大叫起來。滾下床去。大叫道。媽媽。不好了喲。快些來救我喲。養氏方纔要睡。聽見了。不知何故。忙跑來一看。見他精光的坐在地下。養氏問他道。你不睡。跑下地來做甚麼。他道。我怕那個銀喲。他要拍(掐)我的雞雞呢。我不同他睡喲。那牛氏先見他滾了下去大叫。不知何故。倒嚇了一跳。及聽見他說這個話。羞愧得無地逢(縫)可入。那養氏聽說。知是他所敎的事了。忙喝道。不要胡說。好好上床去睡吧。他只〖口么〗喝。我怕他喲。我怕他喲。我不同濟銀睡。我到外頭去睡喲。養氏見他嘴裡混說。也怕羞了香姑。只得一面道。不許胡說。一面忙替他穿上衣服。送他往書房中去睡了。忙又進來。又安撫了香姑幾句。那香姑只是墮淚。勉強而臥。有一個黃鶯兒說這香姑道。

\begin{quotation}

忍淚上牙床。擬今宵恣意狂。誰知好事成魔障。來親那樁。先丢面光。羞慚無地難相傍。惱人腸。一團高興。變做淚汪汪。

\end{quotation}

那養氏又往前邊去帶馬台。到了次日。養氏帶他進來。兩口子同吃飯。他一見了香姑。就叫道。我怕他掐我的雞雞喲。掙着要往外跑。許多丫鬟僕婦在傍。那香姑羞得臉緋紅。淚如斷線珍珠。紛紛往下直滾。又有一個黃鶯兒道。

\begin{quotation}

觸目愈悲傷。轉思量惱斷腸。聞言眞使羞難狀。雲鬟懶妝。啼痕漬裳。金蓮跌綻空惆悵。恨穹蒼。怎將織女。生扭配牛郞。

\end{quotation}

那養氏也沒法了。同他出去。再不敢帶他進來。香姑氣忿塡胸。苦無處訴。夜間獨臥在床上。思量道。我這樣的一個人兒。以爲爹爹必然相女配夫。擇一個才貌雙全的嬌婿。怎知把我送在這個地獄中來。我若嫁了個知情識趣的妙人兒。這兩夜的被底恩情也不知怎樣個快樂。恨了兩聲。他閉目凝神。虛空摹擬怎樣個標致的男兒。在被中是怎樣個溫存。這個中是怎個景界。想了一會。不覺矇矓睡去。心有所思。便幻成夢境。仍是在家做女兒的光景。見一個僕婦來說道。我纔在門口見一個齊整不過的少年騎馬過去。人都說是我家的姑爺。正與姑娘是天生的美對兒。明日姑娘嫁了去。不知怎樣享福呢。聽說雖不好細問。也時時歡喜。過了幾日。說是綵轎到來要娶他。忙忙裝飾。雖裝出許多嬌羞的樸(模)樣來。心裡却暗喜得了不得。上了花轎。鼓樂喧天。花燈照耀。三元百子響若春雷。到了那家。攙扶入內。交杯合巹。偷覷新郞。果然丯姿韶美。私喜道。古人說擲果潘安。大約也不過如此了。少刻人散。那新郞緊上床來了。上前摟着香肩。在耳畔低聲道。夜深了。請睡罷。自己身上不由得酥麻了一下。却不好答得。那新郞便動起手來。正是。

\begin{quotation}

寶帶含羞解。銀釭帶笑吹。

\end{quotation}

放了睡下。新郞脫衣鑽入被中。便來解褲帶。少不得佯羞做作。假意用手攥住。那人口搵香腮。說道。今夜我夫妻百年之始。不要躭誤了良宵。低語悄語。軟款溫存。由不得那手就鬆了。被他卸去紅裩。此時心中又喜又快。他剛扶(伏)上身來。正要嘗是如何滋味。怒(忽)聽得大喝了一聲。一個老和尚把帳子一掀。道。這元紅該是我採的。把那人推將下來。抱着省(香)姑就往外走。那人也下床趕來。和尚抵死與他相併(拚)。香姑此時赤着身體。被那和尚抱住。又羞又怒。忽一驚醒來。原來是一場春夢。終日回思夢境。愈覺傷心。整整一個月眼淚未乾。回家住對月。放聲大哭。無人處。把新郞的這些妙處。細細吿訴他娘。他娘是妓女爲妾的。可敢埋怨夫主。不過微微的婉轉達上。那牛質道。你婦人家見識淺薄。知道甚麼。他是貴公子。自然渾厚篤實。他是有福的人。自然與衆不同。這是女兒的造化。難道倒要那輕薄兒郞虛花子牙纔好麼。計氏不敢再言。香姑在家住了些時。又坡(被)他父親送到婿家。一進門就哭起來。如到了愁山苦海一般。無一刻眉頭吾(畧)展。心地稍舒。養氏憐他。怕哭壞了。司(同)他到大門的樓上。垂下斑竹簾來。看看街上散悶。他家十數間門面俱是樓房。惟這一間空着。坐了一會。見那街上的人來來往往不斷。賣東西的〖口么〗〖口么〗喝喝。甚覺熱鬧。正看着。只見一個老和尚敲着一扇鐃鉢宣卷化錢。大大小小圍着許多人聽。香姑也倒(側)耳聽了一會。見他唱得鏗鏗鏘鏘。甚是入耳。便向養氏道。媽媽。這個老和尚倒唱得好聽。叫他進來唱唱。那養氏見是個有年紀的和尚了。有何妨礙。巴不得與他解解悶。就叫看門的人叫他進來。同香姑下樓。一齊到了廳上。叫那和尚唱了一會。音韻悠揚。甚覺可聽。比先遠聽時更是淸楚。牛氏叫收拾些蔬齋與他吃。因問道。我聽你倒說得好。你也記得多少。老和尚道。老僧零碎混記了些。要走(全)說唱。一兩個月也說唱不了。香姑又問道。老師傅。你今年多少歲了。那老和尚道。老僧今年七十八了。香姑道。你老人家倒還健壯。那老和尚道。出家貧苦人。無穿少吃的。也衰朽了。說着。拿飯來與他吃了。又叫丫頭們取了一百文錢來與他。道。師傅。你明日還來。要唱得好。我布施你一件衣服。那和尚忙打了個問訊謝了。拄着拐。牛氏叫人送了他出去。又吩咐門上人。明日老和尚來。可進來說。遂回內室。一宿晚景休題。次日飯後。家人進來說。那老和尚來了。牛氏道。一個八十歲的老僧。叫他進來罷。怡(怕)甚麼。遂叫僕婦們領他到臥室中來\footnote{此所謂別(引)賊入室。}。茶兒飯兒點心果子與他吃着說唱。唱到將晚。和尚要去。牛氏定要他把這一段故事說完了。和尚道。奶奶。後面還有好些。若等說完。老僧就回不去了。牛氏道。不妨事。你回不去。有年紀酌(的)人就在粗(祖)先樓上去睡。那和尚巴不得奶奶歡喜。好騙衣騙錢。可敢違拗。聽了。就依他坐下。此時家中馬士英夫婦不在家。馬台是個呆子。牛氏是一家之主母了。誰不聽命。可敢不遵他。不叫留一(下)。況且也落得大家聽唱。這和尚說到了半夜。方纔完了。牛氏叫僕婦丫頭拿帳子鋪蓋送他到樓上去睡。原來他住的這一間樓。底下做臥房。樓上供的是他祖先香火。和尚在樓上睡了。次日。牛氏吩咐僕婦們替他做了一身新夏布衣褲\footnote{和尚此時應當得意。}。一連說了數日。總不肯放他回去。養氏這幾個月來見香姑終日愁眉苦臉。兩淚汪汪。不住的長吁短嘆。不曾見他一刻舒眉。自從這和尚來說唱了幾日。纔見他笑容。茶飯也吃得些。不像以先那茶墉(慵)飯懶的樣子。也不肯放這和尚去。留着與他解悶。那和尚一日三茶六飯人服事着受用。也巴不得他留着。牛氏憐他年老。叫了個十來歲的小丫頭扶他上樓下樓照看他。那一晚說到有四更方歇。牛氏睡在那大人步床上。因天熱氣悶。大淸早起來。在春凳上歪着乘涼。牛氏叫那小丫頭。你上樓看看那老師傅醒了沒有。此時衆丫頭都因昨夜熬睏了。都還在洗(沈)睡。這小丫頭他不曉得聽甚麼。老早就去睡覺。所以他倒起得早。那丫頭去了一會下來。笑嘻嘻的道。那老師傅還沒有醒呢。我掀開帳子一看。他精光光的睡着。腰裡那個雞雞子。用手比着。道。有這麼長。有這麼粗。硬邦邦一跳一跳的。倒好耍子。牛氏聽得心中頓了一頓。想道。我看小說。但是人年紀一老。血脈衰敗。那東西就沒用了。怎這個老和尚八十來歲還這樣精肚(壯)。我嫁了恁個呆子。可守的是甚麼貞節。難道人生一世就做一輩子女兒不成。若這和尚果還用得。我且同他相與起原(來)。一則嘗嘗這件東西的滋味。二則免守了活寡。且又沒人動疑。何等不妙。況我前番做夢。搶我的那老和尚說我的元紅原該是他得的。焉知不應的是此老僧身上。遂囑咐那小丫頭道。你是個丫頭家。這村話說不得。羞人子。你再要混說或亂吿訴人。我若知道。就要打嘴巴子\footnote{未式(試)新郞。先將小僕嚇住。以免口舌。誰知禍反生於此。預爲後復(伏)線。}。你須要緊記心中。不許胡說。那丫頭嚇得連忙退出。牛氏淫心一起。那裡還按納得住。到四處看看。丫頭們一個個都還鼾睡。他輕輕走上樓去。把帳子一掀。果然那和尚好一個厥物。有五寸來長。直豎在那裡。他雖淫書看得多。不過只言其形狀而已。却從沒有見過。今見這東西光頭滑腦。紫漒光鮮。眞是眼見稀奇物。勝活一千年。那個暗喜那裡還說得出來。一屁股坐在床沿上。彎腰低頭。仔細端詳了一會。窗上亮光射着。見那龜頭紫艷艷的放光一般。由不得胸頭小鹿丕丕的跳。那牝戶中一吸一吸得難過。忍不住了。把他推了兩推。那和尚一睜眼。見是他笑容可抝(掬)。忙〖目止〗(扯)被蓋上。道。一時睡着了。不知奶奶來。赤身露體的。得罪。望奶奶不要見怪。牛氏紅着臉\footnote{淑女之態。}。低聲道。你今夜醒睡些。我晚間在堂屋裡睡。等夜靜了。你悄悄的下樓去。我有話對你說。千萬不要誤了。那和尚連連喜諾。牛氏說了。怕丫頭們醒來。忙忙下樓。回到房中。丫頭們少刻也都起來了。服侍梳頭洗面。不題。你道這和尚果然七十八歲了麼。這賊禿纔年踰五十。他一生好嫖貪賭。未曾娶妻。把家私花盡了。無處歸着。想去做強盜。怕拿住了血殺。不然似這年輕的人精精壯壯去乞化。怕人不肯捨他。沒奈何。纔出了家。他又不會經典。因幼年時讀過書。認得些字。自幼好看說唱本兒。大來游手好閒。無事時常常聽人說唱。他記性頗好。學會了許多宣卷在肚裡。他要出來說唱化緣。料道哄不動男人。只好騙女人們幾個錢用用。一個睜眉豎眼的壯年和尚。少年婦女怎好叫他的。他幸得生來是個少年白。五十多歲上那頭髪鬍子盡已皓然。皤然一個老翁。他又裝做那龍鍾老景。行動都是艱難的樣子。彎着腰。拄着拐。走快些就像要跌倒一般。他鬼說將八十歲了。圖人憐他。多捨幾文。倒也還沒有奸騙婦女的心腸\footnote{和尚萬分之一想不及。不料今日竟逢此女子。實出望外。妙筆。}。只想混數文錢糊〔口〕而已。每常性動。他自己料這樣個白首皓鬚的老物。可還有婦人愛他。他雖有淫心。又可還敢去調戲婦女。自到了此處幾日。見這牛氏體段風流。語言波俏。雖然心愛\footnote{和尚乃色中惡鬼。見此態未有不動念者。況此僧之來歷不端。而又遇此淫婦。未有不愛。}。不過陽物在褲襠中跳跳而已。可敢有大膽調情之念。他見這樣門第。巴不得假裝志誠。求捨他些衣食錢鈔。就是萬幸了。可還敢動絲毫邪念。不想牛氏是愛收老骨董的。對他說了這話。他也久與牝物睽違。忽然在這裡紅鸞照命。眞是喜從天降。牛氏有心夜間有一番公務。又因起得早。日裡睡了半日。那些丫頭僕婦見奶奶睡覺。可敢叫和尚說唱驚動他。那和尚無事。見牛氏睡了。他也心照。也偷空上樓去睡。養息精神。以俟夜間大舉。牛氏一覺直睡到下午方醒。吩咐丫頭僕婦們道。夜間床上甚熱。我睡不着。可將涼床擡到堂屋裡鋪下我睡。衆人七手八脚擡去。掛上冰紗帳子。錦衾緒(繡)褥。涼枕竹簟。鋪設停當。到晚掌燈時分晚餐罷。纔叫和尚。說到有二更時。盡見這些丫鬟婦女不住的呵欠連天。只是舂盹。知他們睏了。繞(纔)吩咐睡罷。打發和尚上樓。他也就上床安歇。吩咐道。你們各自去睡。不必在此。我不須人作伴。那些婢婦正愁來上夜沒處掛帳子。怕蚊子咬。聽見這話。好不感激奶奶恩典。體恤下人。忙各去分頭睡了。牛氏醒在床上。側耳而聽多時。毫無聲息。似乎都睡着了。隱隱像樓梯上微微有淅淅索索聲響。知道是和尚來了。他從未遇過男子的。此時反有些不好意思。二來未曾經過此道。不知是甜是辣。心中反跳將起來。只見帳子一掀。和尚赤條條鑽上床來。一把抱住。摸見牛氏還穿着褲子。忙替他脫下。就上身來。牛氏恐他冒失。忙附耳低聲道。你不要造次。我還是女身呢。和尚驚問道。奶奶。這是甚麼緣故。牛氏道。我嫁了大半年。丈夫還未同床。故此纔來尋你。那和尚越發大喜過望。雖活了五十多歲。未曾遇過處子。何幸今日得採鮮花。輕輕慢慢。款款溫溫。用了許多津唾。費了無限精神。方纔得兩下相合。有幾句話兒寫他二人的行樂。

\begin{quotation}

一個白頭老禿。撒淫心。橫舂玉杵。一個紅粉嬌娃。展弱體。俯貼牙床。一個乍親原牝。分外心歡。一個初遇雄陽。不由膽快。這女子。也顧不得女訓中三從四德。那禿驢。也不管那佛與(典)內五成(戒)三皈。在香姑。從來想不到元紅付與老禿。在和尚。今日眞個是祐(枯)藤纏繞鮮花。下一個。嬌聲怯怯。上一個。樂興濃濃。書房中。空臥蠢新郞。涼床上。共眠老和尚。

\end{quotation}

那和尚的厥物雖不爲十分雄壯。但牛氏年輕。乍經此道。甚覺受創。叫他下來。和尚道。初次破身。自然有些痛楚。你忍着些。不過是這一遭苦。下次就有甜處了。若這一次怯疼中止。下次仍舊還要疼。還不得遂心。是反受苦多了。那牛氏託(記)得小說中也都有此說。只得嚙被隱忍。心中暗想道。我曾見過書上有一個笑話。一家有個小姑私問嫂子道。男女幹事果快活麼。嫂子哄他道。有甚麼趣。幹一次受苦一次。小姑信以爲實。甚是憂愁。及嫁後滿月回家。笑向嫂子道。說囗的。你可羞。這樣〇〇〇〇〇定有妙境。且忍這一回罷。過了一會。那和尚洩〇〇〇〇〇〇拔出來溫養。再四定了多會。他因久別此竅。〇〇〇〇〇〇〇起。牛氏雖未大嘗樂趣。也就竟不似前番〇〇〇〇〇〇〇〇雞鳴。牛氏約他夜間再來。那和尚方輕〇〇〇〇〇〇〇〇〇白紬手帕將牝戶拭〇〇〇〇〇〇〇〇〇〇〇〇〇〇〇〇胭脂鮮色。自己嘆了〇〇〇〇〇〇〇〇〇〇〇〇〇〇〇〇初心以爲自己如此才〇〇〇〇〇〇〇〇〇〇〇〇〇〇〇嫁了這樣個呆子。不得已想到這老〇〇〇〇〇〇〇〇〇〇自愧。故此嘆息墮淚耳。次夜又復與和尚〇〇〇〇〇〇〇〇其中的趣味。此後總不放這和尚出去\footnote{可笑這才〇〇閉着門打和尚。}。〇〇〇事惟有人在被窩中做的生活。自然瞞得人狠。不意人知道得更切。況人一有了私情。那無心言笑之間。或一舉一動。自己雖要瞞人。不由得就要露出馬脚來。況凡事怕的是冷眼留心。不留心時。任你怎樣不覺。一有了心。無處不是破紋(綻)。牛氏自從勾搭了這和尚。那一番待他的情西(面)。與以先自然加倍。到了晚間。做那一件事。快活起來。到那忘情之際。未免就露出些聲息。或低聲微笑。或氣喘噓噓。那些丫頭僕婦難道個個睡死覺的。更闌夜靜。豈無一兩個聽得些須。不幾日。丫頭傳與便(僕)婦。僕婦說與丈夫。一家盡知其事。有些知事的家人。知道逄(這)不是家奴管得的。只權不知。況主人是個就(獃)物。說也無益。有些不知事的孟浪惡僕在背後紛紛談論。又有那兩面三刀的婦人要討奶奶歡喜。又把這些話吿訴了牛氏。牛氏老羞變怒。叫和尚且去時。暗與了他頭二十兩銀子。釣(夠)他過一兩個月。等事冷一冷再來叫他。牛氏正在得味之時。把個心肝般的老和尚去了。一腔怒氣就借這些丫頭們的皮肉來發〖阝疌〗(洩)。茶裡不尋飯裡尋。屬鐵匠的。一味只是打。把些丫頭們打得望影魂飛。臉上的血痕。身上的靑紫。新陳相接。漸漸尋到這些多嘴的男人們老婆身上來。譬如叫一聲。答應得略慢些。說他見主母年紀。故意渺視不理。就是一頓。略答應得響亮爽快些。說他叫着使性子。也是一頓。或者叫做甚麼。微微遲慢。說他大膽不服呼使。又是一頓。要走快了些。說他目中無主。使着氣昻昻的。便是一頓。若明話答得低了。便說你不理我麼。這樣低聲嫩氣。也一頓。答應得高了。便說我又不聾。你欺負我。唬嚇我麼。又是一頓。這樣尋事。把這幾個僕婦竟是人生有打須當打。一打何曾到九泉的打法。輕則嘴巴數枚。重則皮鞭幾十。一個個打得鼻塌嘴歪。皮開肉綻。當面雖敢怒而不敢言。背地哭啼啼吿訴厥夫。亦人情之常。這些不知死活的奴才。未免喞喞噥噥的抱怨道。不過是爲擠去了和尚。所以拿衆人出氣。說幾句無知的話。也是有的。又有人把這話傳入牛氏耳中。牛氏心生一計。到他父親家中說道。人娶媳婦原是要孝敬公婆。今我們離得窵(遙)遠。還不曾見公婆的面。雖公婆心疼兒女沒得說。我們自己也過不得意。我要往任上看看公婆去。況只得四五日路。我來替爹爹說一聲。我早晚就去。那牛質是極要奉承老親翁的。況女兒說的是正經理性話。遂誇道。這是你做媳婦的孝心。極好的事。但你少年嫩婦。孤行不便。我叫你哥哥同你去。牛氏歸家。收拿行李。帶了幾個老僕婦。却把前日有閒話的八九個家人叫隨了去。衆人可敢不遵。牛氏遂擇日起身。牛耕也帶着六七狗(個)家人。一行男女二十餘人。渡江到浦口。五日就到了鳳陽。先差人去稟知。馬士英同蹇氏聽說媳婦同哥哥來了。忙差人去接進衙門。牛氏拜見了公婆。說了許多思念公婆要來孝養的話。那馬士英夫婦乍見媳婦生得人物果然齊整。說話又賢德。見有這樣個好媳婦。心中那裡歡喜得盡。牛耕也拜見了。唱戲擺酒。一來算接風。二來算會親。熱鬧了幾日。不消說得。過後無事。家常說閒話。馬士英問及家中長短。牛氏就借這個因頭說道。兒子老實一點。閒事不管。媳婦少不得要當家照料。就是帶來的這幾個家奴欺我年幼。不服調度。公婆的人我雖不好打他。罵他們幾句是有的。因爲媳婦閒着悶得慌。有一個八九十歲的老和尚會說因果。媳婦叫了來說兩三日是有的。養媽媽都同在跟前。這些惡奴就造了多少閒言雜語汚衊我。在媳婦値甚麼。使外人聽得。豈不辱了公婆的臉面。我故此帶了他們來。求公婆戒斥他個下次。馬士英正疼這媳婦如心頭之肉。聽了他這話。可有不怒的。次日到大堂上。也不說長短。將跟來的八九個家人。只說他目無幼主母。不分靑紅皀白。每人二十大板。打得死去活來。督撫衙門那牛筋纏的龍鬚板十下就可以送命。皀隸因係打府中內司。徇了多少情面。還打得兩腿肉都飛去。血濺滿身。這些家人只好死捱。當堂可敢說出幼主母私通和尚的話。牛氏見打得如此。把胸中氣恨全消。心裡暗喜不虛來此一場。住了有十數日。馬士英並不知他賢郞同媳婦還未圓房。心疼兒媳年幼。不忍叫他久離。要打發他回。因向牛氏道。我兒。你來一場。算盡了你的孝心了。但家中無人。你回去罷。牛氏見處治了這一番。料道下人再不敢多嘴。他正想回去同和尚大開旗鼓。痛樂一回。但不好說得。聽說叫他回去。心眼裡兒都是快活。故意說了些不捨公公婆婆的話。假裝出許多戀戀的樣子。掉了幾點淚。把那馬士英蹇氏疼愛得了不得。與了許多東西。因看媳婦面上。又厚贈了牛耕。治席送行。差了家丁並門下人十數個送他回去。臨行。又吩咐\endnotemark[3]前次打的衆人道。你們此後須小心。若如前放肆。我知道了。定拿來處死。衆家人忍氣吞聲跟了回來。牛氏到家。牛耕也回去了。過了兩日。恐公婆記掛。打發送來的人回任上去。他又到父母家來走走。留住了兩三日。辭了回家。恰好路上正遇見那和尚在街上敲鉢說唱。牛氏滿心歡喜。叫了個跟轎的小廝約那和尚到家來。牛氏到家。剛進入房中。不一時。那小廝回說和尚來了。牛氏叫他進來說唱了一會。給他飯吃了。將起更。仍叫他到樓上去睡。約將二鼓。牛氏見人都睡靜了。竟自己摸到樓上去。就敎那他家的祖先堂做了行樂之場。兩個人都是久渴了的。這和尚得了牛氏的十多兩銀子。這兩個月壯雞肥肉將養得身子更有力量。牛氏也經開闢多次。可禁大敵的了。西廂上的兩句竟是他二人此時的光景。道是。

\begin{quotation}

一個恣情的不休。一個啞聲兒廝耨。

\end{quotation}

兩個幹了歇。歇了又幹。或這個上。或那個下。足足的忙了一夜。五鼓漏殘。牛氏纔下樓要寢。他心暢神怡。直睡到將午纔醒。牛氏滿心以爲丫頭僕婦都是他打怕了的。不敢多嘴。幾個可惡閒話的家人。前在鳳陽每人領了那頓肥打。料道也再無閒話。同這和尚竟公然大弄起來。日間一時高興。就到樓上取樂一番。晚來或叫和尚到他床上同宿。或他上樓去睡。肆無忌憚起來。這些僕婦又都去吿訴丈夫。牛氏只說威可以服衆。孰不知怨毒之於人大矣。疲犬猶能反噬。\endnotemark[4]何況於人。再無不想報復者。又何況於無知之下人乎。這幾個家人聚在一處道。他明明的養漢。前日到了老主任上。送了我們那一下。幾乎喪命。他今日又同和尚勾搭。我們何不捉住了奸情。看他還說甚麼。且出這口惡氣。有幾個有年紀的知些道理。怕事。說道。罷。前日這頓打。認自己的晦氣罷。古語說。兒不捉母奸。我們下人是捉不得主母奸情的。弄得不好。就着了自己身上。那便了不得。留着命吃碗糙米飯罷。內中一個年小的。叫做吳知。就是大管家吳義的兒子。性極剛拗。他素常恃着是總管之子。在衆家羣人中他定要出尖逞能。他便挺身說道。我拚死也做一下。我想來。把小主請了同去。就算是夫捉妻奸了。怕甚麼。又有三四個同聲相應道。吳大哥這主意好。就是這樣的行。那幾個又勸道。使不得。你看那小主可說得一句話出來的麼。就算拿住了奸。小主是不能殺的。你我下人自己背着個別罪替他殺去麼。旣不殺。私休不得。就要到官。一來小主說不出。二來官官相護。那時反弄到自己身上。勸你省些事吧。那吳知氣忿忿的道。你們這樣老婆一般的漢子。一點膽量也沒有。幹不起大事。我正要弄他到官。叫那淫婦出些醜。纔出得我的氣呢。那三四個道。吳大哥說得是。俗語說。秀才謀反。三年不成。不要木匠多。把房子都蓋歪了。我們拿定主意。就是這樣行。那幾個道。我們是老婆。看你漢子們做去吧。但恐弄得不好。求像我們這老婆還不能呢。無(吳)知道。呸。蹋死放屁蟲。可惜白給你們一張人皮披着。遂不聽那幾個人勸。他五個人齊了心。知會了自己老婆。又關會了丫頭們。這些婦女的心腸只想要報仇。那裡知道利害。還欣欣得意。這一夜。牛氏正約了和尚在他床上高興了半夜。都乏睏極着了。婢婦們留心看明。悄悄把門都開了。通知了他衆人。吳知同那四個家人跑到書房中。那馬台正睡得着呼呼的。被他們搖醒了。知道對他說是沒用的。只替他穿了衣裳。擡着他。一擁到上房來。見牛氏開(同)和尚正摟抱而睡。一個上去。先搶了兩條褲子。一個將和尚打了兩拳。精光的拉下床來綁了。牛氏到了此時也沒法了。蹲在床上。拿被蓋着。衆人道。奶奶。你是推不掉的。捉奸已拿住了雙。還說甚麼。請下來。到衙門裡去。又一個道。難道叫他光着身子去麼。只不與他褲子。衣服要穿的。要了一個丫頭的靑布衫藍布裙。立逼叫他穿上。這牛氏到底年小。心也嚇昏了。又羞愧難當。住(任)人調度。外邊天已黎明。衆人纔要擁着走。只見養氏跌跌撞撞跑了來。攔住道。你們這些斫千刀的做的好事。他一個小男婦女。你們叫他那裡去。吳知道。你是個有年紀的奶媽。小主子不知道甚麼。你不防範他。叫他做出這樣的事來。你還敢來護他。只恐怕老主子知道了。你還有半張桌兒呢。往那裡去。同到衙門裡憑官府發放罷。養氏也無言可答。料道攔阻不住。把頭上的包頭取下。替牛氏把頭罩了。衆人簇擁着到了前廳。叫了乘轎子來。養氏還拉着牛氏不肯放。被吳知上去把他一陣搖搡開了。叫牛氏坐了轎。去掉了簾子。恐他在轎內尋死。好看着他。又一乘家中的轎擡了馬台。這呆子憑人舞弄。他突(究)竟也不知是做苦事。其餘的家人見事弄到這個地步。私按不下來。怕有後禍。着幾個跟着主人。幾個飛跑利(到)牛家報與牛質。牛質大驚。即刻就到牛尚書處說了。關係大家臉面。聞知到中城察院衙門。這御史姓壽名可托。是牛尚書的門生。差一個的當家人。忙到衙門去說。要他婉護這件事。那家人忙到衙門。聞知官府家中有事未來。跑到他家私宅稟見了。說了備細。那壽御史叫了班頭來。吩咐道。你到衙門裡。那牛氏叫他回去。馬公子也不必囗候。只將馬台五個家人收鋪。和尚與他一條褲子穿了。另行看守。到明日早堂審問。班頭領了鈞語。到了衙門。吩咐叫牛氏馬台回去。將五個家人按名字鎖靠了。叫班上人領去看守。把和尚放了綁。也鎖了。與了他條褲子穿上。另帶了去\footnote{一案奸情輕輕了過。這察院眞正可托。}。此時這幾個家奴見局勢不好。面面相覷。纔悔往事做壞。已是遲了\footnote{因一個無知惡少。送了四個孟浪家奴。}。那牛氏他不曾回家去。牛質不知察院將事體如何回。打發了兒子帶着十數個家人遠遠的打聽信。聽得說叫牛氏回去。牛耕接了他家中去了。再說宦蕚同鄔合在茶館中坐了一會。他家人來說道。老爺請回罷。人都散了。宦蕚問是甚麼緣故。那家人道。小的也不知詳細。纔在那裡見一個班頭傳察院老爺吩咐。只把和尚同家人拿起。那馬公子同婦人都叫回去了。宦蕚道。白要我等了半日。一場掃興。同鄔合別了。親自到花鋪廊內買了幾粒揭被香。又買了數丸金鎗不倒紫金丹回來。到晚間。先用燒酒將春藥服下。然後夫妻二人同上床睡下。宦蕚枕在侯氏臂上。咂乳摸陰。摳摳挖挖。假做頑戲。暗暗將兩粒揭被香已悄悄塞入戶中去了。不多時。只見侯氏腰肢不住亂扭。向宦蕚道。我今日這個裡頭作怪得很。怎麼又熱又癢。好不難過。宦蕚笑道。像是你心裡想做那事。發起癢來了。侯氏道。放你的屁。就是想做甚事。也不犯着這樣癢法。就像蛆拱的一般。又火燒火辣。熱烘烘的。說不出來那種那(難)過得很。宦蕚假裝要睡。總不理他。任他說。也不答應。侯氏此時實在有些過不得。忍不住了。見他不做聲。伸手向他腰間一摸。那宦蕚此時藥性亦到。這陽物堅如鐵杵。不住亂跳。其熱如一塊火炭。侯氏摸着。情不能禁。側轉身來就他。牝戶剛對着玉莖。不期他假做翻身。面朝外去。侯氏急了。推他道。你醒來。我有話說。宦蕚故意道。我纔睡着。叫我做甚麼。有話明日說罷。侯氏道。你不要吃了烏龜肉裝忘八憨兒。我今日實在難過得很。不知甚麼緣故。你的那東西又挺硬在那裡。來救他一救。宦蕚道。我要睡。你又叫我起來。先說明白了再來。我若弄得正高興。你要叫我住。可不難爲壞了我。你旣要我弄。除非憑我弄得興敗纔歇。不要到高興的時候又不肯了。侯氏笑道。算命的先生吃螃蟹。你瞎〖扌扉〗的是甚麼。你的本事可是說的。淸水下雜麪。你吃我也見。往常只有你見敗的回數多。我還有怕你的麼。昨日軟得鼻涕似的。求嬌花來替你咂。纔隔了夜就忘了麼。宦蕚此時也忍不得了。起來道。你今日看本事還錢。你這會兒說嘴。硬邦邦的。過會不許嘴軟。侯氏道。空說嘴中甚麼用。做出來纔見得呢。你旣說嘴。再要軟了求嬌花咂。我可也不依。一面笑着。忙仰臥了。宦蕚將他兩腿夾在肋下。把龜頭在也(他)牝戶邊左挽右晃。總不入去。急得侯氏將屁股就上來。他又往後退退。侯氏纔把屁股落下去。他又將龜頭往下聳聳。急得侯氏亂扭。罵道。狠心的忘八。你要我死麼。宦蕚笑嘻嘻總不理他。忽然將陽物用力往下一揷。盡沒至根。頂了兩頂。侯氏覺得內中滾熱脹滿。有趣不過。急將屁股拱起來迎。宦蕚又拔了出來。如此數次。他陰中淫水一陣陣像小解一般冒將出來。只是閉着眼哼。那宦蕚見奈何得他夠了。遂把他兩腿放下分開。身子伏下。兩手拄個結實。然後用力沒稜露腦抽送起來。弄得侯氏心花俱開。顫成一塊。丢了又丢。先還用兩手扳住他的腰。後來兩臂酥軟。也扳不住了。直挺挺睡着。就是弄死人的一般。宦蕚只是亂搗。侯氏半晌回過一口氣來。叫道。好哥哥。你饒了我罷。我來不得了。渾身像癱化了的。再弄弄。骨頭全要散了。宦蕚也不答他。仗着藥力。重新用起狠來。出必至腦。入必盡根。又有千餘。侯氏又丢了兩次。實在動不得了。陰中也有些疼痛。嬌聲哀吿道。你難道當眞要弄死我麼。你歇歇罷。宦蕚道。這個話先說過的。我正發興。你就要住。你說你不怕。怎這會子說不應先的話了。你先說我隔夜的話就忘記。你怎麼纔說的話就忘了。你顧了你。叫我如何過得。侯氏實實支撐不住。便道。你興不足。叫嬌花來弄弄罷。宦蕚道。我叫他來弄。你又好發惱。侯氏道。是叫他來替。我如何又惱。宦蕚巴不得這一聲。聽了滿心歡喜。便叫嬌花。這嬌花昨夜正將得味的時候。被主母一驚而散。這一件美物未經飽足。就如小孩子嘗着了芝蔴糖。\endnotemark[5]又香又甜。焉得不想吃。此時聽見他兩人動作。悄悄走來。躱在床後邊竊聽。聽得那些聲息有兩三種。一層有趣似一層。起先聽得主母是急的哼。那是嘴裡的聲息。次後是弄得快活的哼。那聲息是從鼻孔中出來的。再後是弄得不死不活。微微喉中有些聲息。被下面弄得那響。如人在泥淖中行。滑撻滑撻的不住。又把那喉中之聲蓋住了。聽不甚明。把那丫頭急得臉上火冒一陣。陰中水流一陣。喉管中發煙。不住的嚥唾。要去又捨不得。要聽又過不得。正在難過。忽聽得主母說叫他來替。如窮花子恰(拾)了錠金子。也沒有這樣快活。先那兩條腿總酸麻得動不得。此時聽了這話。忽然健硬起來。兩三步忙忙走到床前。宦蕚將他一把抱上床來。正要替他脫掉褲。伸下手去。原來竟是光着嫩股。倒摸了一手騷水。連他兩條腿都是濕漉漉的。忙替他把衫子脫了。兩個蒸餅般的嫩乳。緊緊貼在胸前。指頂大一個小乳頭。一身細皮淨肉。令人好生可愛。抱着親了兩個嘴。將他放倒。因昨晚唐突了他。今日不敢冒失。輕輕的送將進去。一來兩件都是濕透了的東西。二來又是昨夜小和尚掛搭過的了。故不覺煩難。也就盡根而入。那嬌花也不覺得苦楚。不過有些脹悶。知道後來還有樂境。如吃橄欖一般。先不嘗有酸澀。後來如何得有甘香。也就勇於承受。宦蕚見他不似前番畏縮。也就施展鎗法。大戰起來。後又演那百步穿楊的箭法。下下皆中紅心。那丫頭是見過主母樣子的。不知不覺把兩條白森森嫩藕般小腿蹺在兩邊。嫩臀攧攧扭扭。口鼻中也哼哼喞喞。宦蕚見這個樣子。魂消而骨醉。此時藥性已過。不覺大洩。緊緊抱住。睡了片刻。拽將出來。拭抹乾淨。那嬌花嘗得這美味果然有趣。這樣沒骨頭的一個棍子魚。比山禽海味。異果奇肴。都爽口些。心滿意足。連衣裳也不穿。欣欣然抱在懷中。自去睡了。宦蕚將侯氏一看。此時却是眞正睡着了。動也不動。還赤露着身子。牝戶大張在那裡。宦蕚低頭看看。聞得內中一陣香氣撲鼻。知是先那揭被香的藥味。拉過被來替他蓋上。復聞他的嘴。也有香氣噴出。輕輕親了兩親。然後臥下。他也乏倦了一夜。直睡到東方日出。然後纔醒。侯氏也醒了。問宦蕚道。夜裡我睡着了。你同丫頭弄到多昝纔歇。宦蕚道。這是你的美情。我方敢如此。你旣睡着了。我怎肯瞞你取樂。不過弄丢了就罷。那丫頭也懼你王法。不多一會。他就去了。我就替你蓋了被同睡。雖興還不足。恐怕勞碌壞了你。不敢動作。只輕輕親了兩個嘴。只得忍住睡了。侯氏喜得眉花眼笑。親了他兩個嘴。說道。你這樣敬我愛我疼我。還有甚麼說的。你若時常像這樣不躱懶。我便將丫頭與你服事也是肯的。宦蕚聽了。如天上掉下個寶貝來一般。喜得沒入脚處。忙道。奶奶。你這樣好恩到我。我今後一日一日自然加倍慇勤。敢懶惰麼。抱着侯氏又要弄弄。侯氏道。日頭出得大高。我還酸軟懶動。你留些精神夜裡使罷。宦蕚此時也並非高興。因聽得把嬌花與他。不過是謝恩之意。見侯氏璧謝。他也就虛邀了。侯氏伸手摸他的陽物。已經綿軟。笑道。你夜裡那樣強頭強腦。好不利害。怎這會也瘋癱了麼。大笑了一會。兩人纔起來洗臉梳頭。那嬌花精神抖擻。笑容滿面。在傍服事。甚是慇勤。侯氏叫過他來。吩咐道。我看你倒還膽小。又知規矩。今後我擡舉你。你就貼身服事你主子。但不許瞞我做事。倘偷饞抹嘴。我知道了。就了不得。你不容同別的丫頭到那屋裡去了。你就安個鋪在我床後頭睡。那丫頭笑嘻嘻忙叩了七八個頭。答道。蒙奶奶恩典。這樣待我。我怎敢欺心。侯氏又將自己的衣衫裙褲之類。查了幾件賞他。又與了他幾件首飾。嬌花又叩頭謝了。心中暗喜。自不必說。那宦蕚在傍看着。喜歡得笑得眼都沒縫。暗想道。好妙計。好春方。把一位妒奶奶不但治得服服貼貼。又還得了一個美婢。眞是快樂。此後但是他夫妻幹事。就叫嬌花在傍。或是推送。或是扛腿。做了一個幫手。或替他兩人揩抹。也不似先前畏縮。也知打混揷趣。取侯氏的笑顏。但是侯氏飽足之後。餘瀝也常波及於他。雖不能適口充腸。也強似當日饞眼嚥唾。雖是把個宦蕚喜得說不出的那一個快活。却也弄得他兩邊接應不暇。他每日在侯氏跟前假做殷懃說笑。拿出那感恩報德的樣子來。好不麻肉。忽然一日。家人傳進來說。二舅老爺奉差往雲南去。如今從水路回來。已到上新河。差人來說信。宦蕚忙起身騎馬去接。侯氏吩咐備酒伺候。不多時。一同來家。進到上房。兄妹相會。禮畢坐下。說了一會家常。〔侯〕捷帶了許多土儀來相送。

\begin{quotation}

草殼檳榔。普兒(洱)茶。雞蓯菜。永棋。牙筆筒。象尾牙簽。水西皮韉。皮臉盆。皮碗。皮盤。

\end{quotation}

宦蕚作謝收了。須臾。擺上酒肴。他夫妻陪坐閒敍。你道這侯捷他有甚事往雲南去。如何又從水路來到南京。他便是侯太常的次子。侯敏之弟。侯氏之兄。他在京做官。歷陞苑馬寺正卿。他管馬久了。深知馬之好歹。這時魏忠賢正立內操。因嫌大馬不伶範。他素知滇黔蜀中三省所產之馬。登山渡坂如履平地。欲勅地方官送來。恐其按數送來塞責。不能如意。特差侯捷往三處揀選採買。馳驛而往。侯捷奉了這個美差。他自河南由潼關走陝西到四川去。他雖係魏璫所遣。却算是奉旨的欽差。沿途大小官員送程儀送吃食。好生熱鬧。不能詳述。一日。到了漢中府武功縣。那知縣姓沐名仁。是侯太常的切己門生。乃浗(沐)國公的族中子弟。出境遠迎。不但是接欽差。且要奉承老世兄。接着了。一同到縣。就留在衙門中住。以見親厚之意。敍了些寒溫。擺上酒來。沐知縣道。弟所轄斗大一城。處在山僻。連梨園子弟都是沒有的。老世臺駕臨敝邑。着實簡褻得罪。有一個搽粉虞侯。弟欲叫他來祗應。老世臺尊意若何。侯捷笑道。古人說。

\begin{quotation}

蜜戚戚二三知己。嬌滴滴一個紅裙。

明晃晃兩枝銀燭。響噹噹一個骰盆。

\end{quotation}

這都是極妙的事。有何不可。沐知縣遂吩咐家人叫傳了來。原來他先已叫到署中伺候着。聽得叫。就嬝嬝娜娜走了進來。向侯捷叩頭。侯捷見他生得頗有可觀。有四句贊他道。

\begin{quotation}

粉面紅光襯。朱唇絳色勻。

蛾眉橫月小。蟬鬢疊雲新。

\end{quotation}

侯捷見了甚喜。問他話時。是一口北京語音。嬌聲嫩語。愈覺可愛。你道一個官衙中如何叫進妓女來。明季天下皆有官妓。厥夫名曰樂戶。名載册籍。子孫相承。代代世襲。再脫不掉的。俗所謂上銅板册的烏龜是也。一年交納錢糧。謂之金花銀兩。送到大內庫中。爲后妃胭粉之費。這是永樂皇帝創下的一個奇政。貽害後世。各官皆准叫去承應。惟不許公然留宿。大約暗暗的私諧鴛侶也沒處查帳。那女子在傍鶯聲嚦嚦。唱連像兒邊關調侑酒。飮到掌燈酒闌之後。侯捷同那妓者喁喁笑語。大有留連之意。沐知縣笑道。這妮子頗少。有丯韻。老世臺若不嫌他鄙穢。留下他相伴罷。也抵得陶穀郵亭一夜眠。臺意如何。侯捷笑道。這是老世臺官署中。如何使得。沐知縣道。老世臺果有此興。這倒不妨。那侯捷也是個酷好此道的。沿途因欽差尊重。不好去嫖。今見他如此說。正投所好。便道。旣承雅愛。敢不從命。酒也吿止。沐知縣同他到了書房內。床榻早已鋪設停當。又坐談了片刻。笑向侯捷道。歡娛夜短。一刻千金。弟不奉陪了。吿了安置進去。侯捷上床。那妓者服事他寬衣睡下。然後纔上床。也脫盡了睡下。侯捷撫摸他身上。皮膚甚細。嫩乳酥胸。頗可動人。勃然興發。便如此云云。事竣之後。共枕而臥。侯捷問他道。我聽你是北京聲口。如何到了這裡來。那女子先還不敢答應。問之再三。他流淚說道他父親姓劉。係北京人。是太學生。因爲表兄所誘。私奔逃出。不幸表兄病歿。被樂戶騙來到此。詳細吿訴。涕泗滂流。侯捷問道。你今意思如何。他道。但求得出火坑。爲良人之婦。死亦無憾矣。侯捷道。你意果決。我同你一宿姻緣。我當救你。那女子要下床叩謝。侯捷摟住道。不肖(消)了。他在枕傍叩了數十。侯捷拉他睡下。那女子因感他恩私。逞嬌獻媚。那侯捷興又復動。又雲雨了一番。然後就寢。且說這女子的父親名字叫做劉文韜。與一個汪時珍。皆係北京順天府宛平縣人。俱住在阜成門外。他二人生同齒。居同里。幼同學。長同游邑庠。交甚相善。及汪生男。劉生女。又同日。里人親友持湯餅交賀兩家。謂他二人道。此天授佳兒佳婦也。他二人亦心願。遂締姻好。未幾。汪時珍夫婦染疫病篤。以幼子並家財囑於劉文韜。道。我本客籍。無期功強近之親可以屬目。我與兄丱角相交至於今日。且又係肺腑之親。藐茲遺孤。惟藉字之。俟其成立相配之後。再以家產付之。吾目瞑矣。諒兄義人。決不負我所托。遂卒。殯葬之後。汪時珍產業盡歸於劉文韜。他由是持籌握算。數年遂成巨富。乃納粟入雍。以女改許貴兒。汪氏子年至十五。尚不使就學。蓬頭垢面。露肘決踵。\endnotemark[6]與家童爲伍。甚至操畚鍤\endnotemark[7]以就飮食。劉文韜終歲不使一見。天啓五年。詔舉監生科。劉文韜希圖僥倖。乃就僧舍肄業。僧舍去其家不半里。一夕。鄰家郭氏子暴卒。召僧誦經入殮。師徒盡往。囑文韜守舍。他讀書漏深。神思困倦。憑几假寐。陡聞飄風\endnotemark[8]颼颼。自遠而來。漸至庵前。倏焉排闥直抵中庭。燈昏月暗。簷馬悲鳴。牆篁慘戞。竦然驚醒。遂心蕩神沮。\endnotemark[9]乃起而就榻下帷。箕踞枕簟。以候僧歸。少焉。聞門外有呼其名者。細聽。則故人汪時珍音也。不覺毛髮悚豎。股戰肌栗。歛手屛氣。不敢出息。頃之。則拒門入室。據榻前竹椅。細數道。汝七八歲家貧。就學我家。我解衣推食。未嘗或吝。\endnotemark[10]及長。赴郡邑試。凡百資斧。吾與汝共。迨補諸生。汝巾衫脩脯不能措。吾爲汝辦。汝年三十未娶。吾助汝婚。後各生子女。又結姻婭。歷年來汝不治生產而足衣食。伊誰之力耶。汝嘗指天地。誓日月。呼鬼神。而吿予曰。吾頭敢斷。心敢剖。肝膈敢刳。君恩不敢忘也。言猶在耳。抔土未乾。將女改\endnotemark[11]許。奴隸吾子。吞沒吾財。負恩忘義。狗彘不若。天地鬼神森羅\endnotemark[12]昭布。尚思取科第。倖爵祿。榮一鄕。耀一時耶。吾在夜臺十餘年。隱恨在心。未嘗刻忘。今日特來與汝訣\endnotemark[13]生。死在今夕矣。文韜大恐。乃振衣從牖中躍出。汪踉蹌尾後。至殿上。愴惶迷惑。又黑暗如漆。不得其門。只聞得脚步在後。追捕甚急。乃大呼乞命。遶殿而走。汪復追逐不休。至四鼓。文韜力竭氣盡。僵仆於地。汪倚\endnotemark[14]殿楹。恨罵不絕。僧在喪家誦經畢。將舉屍入殮。則失其所在。遍覓不得。合家\endnotemark[15]驚訝。僧辭神歸。見文韜伏地呻吟。又一人倚柱挺立。舉火燭之。則郭氏所失之屍也。大驚。使其徒報喪家。並呼文韜之妻。少頃皆至。郭氏輿其屍歸。劉妻以薑汁灌文韜。五鼓方甦。問其故。文韜具言始末。聞者無不吐舌。遂輿回。惟張目發狂。數日而卒。無子。妻亦相繼病死。劉文韜之叔主其家。以其產盡還汪子。其女初改許貴婿。貴婿又殤。猶在閨中待字。他有一個表兄時常到他家來。日近親。那女子陡然高興。等不得媒妁了。瞞着父親。竟兩人暗暗成其夫婦。後來二人情厚了。又怕父母得知。將所有之首飾捲而懷之。相約而逃。一直就逃到保定府地方。那表兄得了個夾陰傷寒。此時他囊中已罄。無力醫藥。數日死了。恰値有一個陝西客人也在店中。見這女子生得頗佳。情願替他葬夫。開發店賬。要娶他爲妻。這女子正無所歸。只得從他而去。這正是。

\begin{quotation}

明知不是伴。事急且相隨。

\end{quotation}

誰知到了家中。竟是一個此道。歇後語說的。半夜回家不點燈。烏歸。他身落火坑。少不得倚門獻笑。送舊迎新。做了煙花的道路。今遇侯捷問他。他實呈始末。侯捷動了一點憐憫心腸。次早對沐知縣說了。即刻差人拿了烏龜來。問了他個拐良爲娼。重責三十板。枷號兩個月。進來問侯捷道。此女還是老世臺帶了去。還是弟養在這裡以俟駕旋。侯捷笑道。弟那裡要他。祈老世臺擇一良善無妻者配之。陰功莫大了。沐知縣知他衙門中有一個少年小吏無妻。叫了來。吩咐庫中取了十兩銀子給他。爲花燭之費。他夫妻二人叩謝了。侯捷同知縣歡喜而去。劉文韜貪利負義。爲汪時珍\endnotemark[16]活奪其魂。世之負心人寧無畏耶。女落煙花。產業乃歸汪子。愛的便宜處却在那裡。此一段可作負心人頂門一針。不可視爲泛言。閒話休題。且說侯捷要辭行。沐知縣苦苦款留不住。地方寒苦。不能厚贐。遠送當三杯之意。直送至百里之外而別。侯捷朝登紫陌。夜宿紅塵。不日來到棧道。見了許多崇山峻嶺。峭壁懸崖。蒼松怪木。異草奇花。眼界中倒也覺得新奇。不日到了成都。這府治在萬山之中一塊平陽。沃野千里。眞古所謂天府之國也。進城見了蜀王。會同巡撫。傳諭各府官員採買馬匹。住了兩月有餘。挑選上好川馬一千匹。交與巡撫。遣指揮一員。領百餘兵丁。先送往京中東廠交割。他辭了蜀王。同衆官由水路下夔門。過江陵縣。至常德府。常德由水路至鎭遠者。於西門覓舟。大者曰辰船。可容二十餘人。舟至辰沅而止。\endnotemark[17]小者曰〖舟秋〗船。容三四人。可泝五谿直達㵲水。逆流牽挽。層累而上。計程僅一千二百里。然灘多石險。一月方能達。且辰沅一路不設遞運。故乘傳之使盡皆從陸。侯捷自常德起旱。到桃源縣。西行二十里即進山。從此以往。高高下下。一路皆山。五里至白馬渡。溯流沿山而行。左瞰空江。右挾岩壑。渡江登嶺。折下平田。五里過桃川鋪渡小溪。折而南過仙徑亭。乃入桃源山。山之陽有洞曰桃源洞。又名秦人洞。攀登而上至洞口。石壁峭立。縱廣丈餘。洞外有瀑布千尺。掛絕壁而下。雖大旱不絕瀉潭中。行里許。伏地不復見。又北行三里。與桃溪合流出大江。潭在洞門外。深不可測。辰邑山溪最爲奇勝。自船溪至邑四十里。一望岩石林立。態色之妙。彷彿太湖靈壁。輿馬都從石〖阝少日小〗中行。或高如峭壁。或砌如欄堦。或如馬馳虎踞。或如祥鸞奇鬼。或如樓閣。或如煙雲。種種駭異。居民皆藏石塢中。短行週遭。時見煙升。從風捲散。酷似輞川圖北垞意景。至如辰溪。城市四圍。俱石骨奇支。謂之五城山。楚威王使將軍莊蹻定黔中。至此因山築城是也。城南一帶。則有石屋巉空。臨江數仞。與疾流奔響。互爲吞吐。尤出人耳目之外。鎭遠府河勢紆曲。水由黃平州萬山中來。峰巒縱拔。上出重霄。無城郭。依山爲衛。隔河有衛城。設立指揮使。實以守兵。誠控蠻之良策也。水路上阻諸葛洞之塞。止可到此。故舟車軿\endnotemark[18]輳。貨物聚集。下十五里即兩路口。乃通黎平之道也。黎郡北連楚壤。南接西粤。中有九股黑苗聚落於古州八萬。方二千餘里。泉甘土肥。出五金礦。民物豐阜。俗以十一月爲歲首。其地家畜\endnotemark[19]肥牯。織五色布。每元旦殺牛焚布以祀天。自古不服中國。出鎭遠西門即油榨關。雖不甚險。實鑿開一線之道也。過關。平路十里。至相見坡。三重迭起。高皆千仞。計程有三十里。登首坡則尾見。立中坡前後顧則首尾見。陟尾坡則首見。行旅者此以手招。彼以口答。響應若咫尺。而不知三十里之遙也。望城坡。登其巓可望偏橋衛城。故名。南里許即諸葛洞。相傳武侯征蠻鑿運糧者。然非洞也。乃兩山陡\endnotemark[20]立。中夾一溪。後爲大水衝兩崖巨石梗塞中流。舟楫難行。萬曆中設偏沅巡撫衙門。以壓鎭諸苗。半年駐沅。半年駐偏橋關。爲楚黔重鎭。偏橋下七八里有白雲洞。深十里許。中有蛟龍獅象石床石凳等景。頗可觀遊。倒馬坡之巓曰武勝關。土名上馬營。五里穀子鋪。十里西陽鋪。十里至黃絲鋪。設腰站。此地俗近苗夷。女德不戒。而黃絲鋪爲尤。斯夜郞之桑間濮上也。彼處謠云。淸平豆腐楊老酒。黃絲的姐兒家家有。興隆衛。即古牂牁郡。\endnotemark[21]歷三十里渡崇安江。江之上流接新\endnotemark[22]添衛諸山之水。合平越葛鏡橋麻哈江下。入古州八萬。經生苗地。二千餘里達黔陽。合\endnotemark[23]大溪趨洞庭以入大江。楊老驛。去淸平縣三十里。有竹王祠。三月間香火極盛。漢時夜郞女澣於遯水。忽有巨竹三節上流浮下。中有兒啼聲。剖得一男。\endnotemark[24]育之。及長。有材武。自立爲夜郞侯。以竹爲姓。能以威德撫諸蠻。武帝封爲夜郞王。疑其欲反。復殺之。羣\endnotemark[25]蠻思之不置。請立。後乃封其三子皆爲侯。旣卒。羣蠻立祠祀之。黃絲驛亦有廟。香火亦盛。葛鏡橋。跨麻哈江而造。兩山壁立千仞。相束一江。水黑\endnotemark[26]如膠。有風不波。霧罩山昏。鮮見天日。昔皆懸絙以渡。沈溺者衆。嘉靖間里人葛鏡建巨橋。旋圯。再建\endnotemark[27]復傾。乃齋戒沐浴。率妻子刑牲以誓曰。橋之不成。有如此江。遂破產經營。即成。至今來往者賴之。平越郡城內有張邋遢修道故蹟。邋遢名三丯。閩人。洪武間以軍籍戍郡。蓬頭赤足丐於市。故呼爲邋遢。府南五里。隔溪絕壁有三丯遺照。戴華陽巾。側身攜杖西行。儼然圖畫。傍有神留宇宙四大字。下有夜雨\endnotemark[28]滴金橋。雖晴夜亦雨灑數點。又有晚霞斜照。不計晴雨。皆有斜暉。新添衛十里有憑虛洞。洞深十里。有瀑泉聲如雷吼。俗呼母珠洞。下五里甕城橋。長百餘步。其\endnotemark[29]水入蜀。龍里衛南關外有留人洞。幽靜可愛。客至每留戀不忍去。然淺迫只可容數人。龍洞。去貴州省城五里。淺窄不可遊。省城之水皆流入廣陸河下烏江者也。自省以西。山川迥異。皆各各自生。不相聯絡。無復依迴瞻顧之狀。田皆石底。上惟尺土。五日不雨。則苗枯槁。世所謂雨師好黔。言黔中無五日不雨也。良由彼蒼愛人之至。惟恐禾荒民飢。故常以雨滋之。貴州省城。明初設立貴州宣慰司。至隆慶中改爲貴陽府。環城九里七分。自南至北街道甚闢。市肆咸聚。從來皆謂滇南重地。而取道於黔中一線。設黔省。總爲通滇計。然二百餘年。黔常多事。不及滇雲之盛\endnotemark[30]也。天啓二年二月初七日。水西安酋\endnotemark[31]寇省城。至十月初一日圍始解。議於舊城外聯築一城。以護城外居民。周九里三分。於天啓四年工竣。黔省苗蠻種類甚多。有。

\begin{quotation}

花苗。東苗。西苗。牯羊苗。靑苗。白苗。谷藺苗。紫姜苗。平伐苗。夭苗。九股黑\endnotemark[32]苗。紅苗。生苗。羅漢苗。陽洞苗。黑羅羅。白羅羅。八番苗。打牙犵狫。剪頭犵狫。木狫。狆家苗。土人苗。狪人。〖犭羊〗獷苗。楊保苗。蠻人苗。狗耳龍家苗。馬鐙龍家苗。僰人。宋家。蔡家。

\end{quotation}

共三十餘種。風俗各異。惟宋家蔡家馬鐙龍家。乃戰國時楚伐宋蔡龍

三國。俘其民放之南徼。流而爲苗者。知中原禮義。衣服祭祀。婚嫁喪葬。

揖讓進退。一稟於周。而花苗東苗\endnotemark[33]西苗牯羊苗四種淳樸外。其餘者

皆不可以禮法敎。惟土司官威刑始得以制之。諸苗中狆家最惡而險。

通漢語。知漢書。到處皆有其種。蓋其商賈於諸苗之中。如徽人紹人之

於中原\endnotemark[34]也。然秉性匪良。專造藥弩。種蠱毒。搆結生苗。劫\endnotemark[35]掠百姓。爲

害最烈。捕則竄入深箐。\endnotemark[36]無從追緝。生苗者。不服土官黔\endnotemark[37]束。無頭目。

雄而強者即爲長。或聚至數百人。即僭號稱帝。儼然冕旒黃衣而端拱

於荒山之巓。以受數百之朝賀而呼萬歲。官兵至。則遁而之他山。逢人

即殺。見物即劫去。有司官或統人四面截殺則盡之矣。他日又有羣聚

者。仍然焚掠。而狆家多通諸苗之言。多識僰人之文。復誘而入內地。劫

殺商賈。爲害不可勝言。苗俗每歲孟春月。男女各麗服相率跳月。男吹蘆笙於前以爲導。女振鐸於後以爲應。連袂把臂。盤施宛轉。各有行列。終日不亂。暮則挈所私歸。謔浪笑歌。比曉乃散。聘貲\endnotemark[38]視女妍媸而定多寡。必生子然後歸於夫家。惟紅苗爲甚。每至立春日。擇男女之麗者。扮各故事以迎於市爲樂。男子之麗者。即古之潘安宋朝有不及焉。女子之麗者。漢之飛燕唐之太眞亦無能出其上矣。此種女子。欲購之者。牛馬當以千計而始首肯。男子皆不樂爲龍陽君。有犯之者。輒自殺。惟此一事。乃中國之所不及也。每冬月。苗女子採刺梨入市貨人。得浙江楚豫客買之。苗女喜曰利市。謂得佳客交易也。本省人買則倍其價。江南人或物色之。則舉筐以贈。曰愛莫離。愛莫離者。\endnotemark[39]華言與你有宿緣也。或有調戲之者。則大怒曰。落勿渾。落勿渾者。華言沒廉恥也。山峒中諸苗男女見有鮮衣怒馬僕從呼擁而至者。舉家皆出而膜拜。有不知者。輒大聲呼曰。快出來睨漢郞。睨者。視也。漢郞者。\endnotemark[40]漢官也。或下馬過其家乞水火。必舉家男女跪而奉之。其愛慕中國如此。威淸衛有兩尖峰平地突起。俱高千仞。上各有廟。每仲春。遊者絡繹。平壩所有珍珠泉。又名貪泉。平時無水。焚楮帛。則泉\endnotemark[41]湧如沸。高噴數仞。天臺山有神女廟。女冠所居。翠樹菁葱。頗可遊覽。有泉一勺。即十萬人飮之不竭。安順州。在元爲習安州。城圍九里。闤闠宮市皆宏敞壯麗。人家以白石爲牆壁。石片爲瓦。估人雲集遠勝貴陽。昔嘗欲立省會於此。因秤土輕重不及貴陽。故捨此從彼。附郭有普定衛。明初設普定軍民指揮使司以襟帶三州。其權甚重。故黔民但言普定而不知有安順。威之所懾久矣。安順乃黔西孔道也。出西關四十五里有龍井。每年立秋日。井內發聲如鼓。聞數十里。名龍打鼓。主來歲雨多。至安莊衛。此路山川又一變矣。山亂如麻。俱高萬\endnotemark[42]仞。山巓突起。其峰如槊如笋者。指不勝屈。安莊衛西南行五里有觀音洞。中安大士。洞中又有洞。石乳溜滴成柱。大數圍。擊之。一爲鐘聲。一爲鼓聲。從右\endnotemark[43]直上又有雙明洞。傍又有一洞。極小而黑。境界之奇莫能盡述。十五里至白水鋪。行未里許。見瀑布如簾。倒瀉石壁。羣峰直上。高入雲表。五里白虹橋。橋上瀑布更飄。三四折入溪。疾渡橋下西去。過橋逐溪行。忽聞有轟雷聲聲不息。輿人云。此望水亭泉聲也。又五里。上望水亭。\endnotemark[44]龍湫直下。白練千條。噴珠捲雪。注入百尺綠潭。雖天朗氣淸。而激水噴濺。如行大霧中。數里以前所聞雷鳴者蓋此。隔岸有雪練晴川四字。里人云。潭內有水犀。風月淸皎時往往出現。其龍湫掛處。懸崖數仞。怒濤翻湧。內有水簾洞。深穾\endnotemark[45]不可窮。覇陵橋即關索嶺。水從西北而趨。自萬山中來。亦合盤江。至粤西以入海。關索嶺爲黔山峻險第一。路如之字。盤折而上。山半有關帝祠。即龍泉寺。有馬跑\endnotemark[46]泉。甘碧可飮。相傳關帝少子索用鎗刺出者。廟在高臺之上。臺下有竹奇絕。名曰綿竹。俗曰旛竿竹。圍大如松。菁葱可愛。廟門外有啞泉。昔孔明南征。軍士誤飮此水皆\endnotemark[47]啞。後人封之。有碣曰亘古啞泉。西巓即順忠\endnotemark[48]王索祠。鐵鎗一株。重百餘斤。以鎭山門。俗稱小關王廟。廟貌甚偉。苗部俱畏威德。入廟者無不羅拜。下嶺即關山驛。自此以西。俱高峰揷天。煙雲無陰晴。瀰漫山谷。象鼻嶺。兩峯壁立。相夾一嶺。如象鼻然。濶三丈。長百餘丈。登其西畔高峯。視黔東諸山如培塿矣。頂站即永寧州。地穹窿無極。去天不遠。山頂在雲霧中。濃陰成雨。終古不晴。秋月重裘。奇寒徹骨。此地水即西流。山勢崎嶇險惡。站前後數十里。驛騎倒斃甚多。道旁僵臥。臭穢逼人。城北二里許有觀音洞。深十里。秉燭快遊。亦勝事也。內有石床一。光潤如大理文石。坐臥則錚錚有聲。反側間如絲竹亂\endnotemark[49]耳。鐫題\endnotemark[50]於石曰遊仙榻。三十里外新鋪西有潭。深百丈。潭旁石壁高千尋。如千層餅狀。亦奇觀也。\endnotemark[51]鐵索橋。駕盤江而造。江源出自烏撒苗境深山中。冬日水涸。始見其泉。凡七十七處。俱\endnotemark[52]臨於一谿。遶曲靖道畢節而注安南縣。\endnotemark[53]合粤西烏梅江而下海。入滇所必由也。兩峰夾峙。一水中絕。斷岸千尺。飛流如駛。蓋天設以界黔滇\endnotemark[54]者也。往往舟濟多覆溺患。天啓二年。監司朱家民始冶鐵爲絙者三十七。粗有數圍。長數十丈。將兩崖巨石鑿孔以貫之。覆以木板。相類棧道。然絙長力弱。人行其上。升降不已。身隨搖撼。不克自持。車騎必下。前者陟岸。後者始登。若接武而行。益增其險。上坡不過里許。然陡峻難行。不亞上關嶺也。安南衛有峯揷\endnotemark[55]天。上有元帝廟。南關亦有峯相類。上有玉皇閣。俱可登眺。閣下即南峯寺也。語云。冷頂站。熱盤江。行至安南討火。向八十里之間。寒熱三變。天地之間至此獨異。自南關上坡至觀將軍飮馬泉。歷五雲坡。過仙人洞。徑老鴉關。皆斜盤百曲。但石道寬平可行耳。道傍石刻朱書鳥道千重四字。頗壯麗。度萬人橋至江西坡。山嶺差平。然曲折紆迴而上。深澗大壑。心目茫然矣。新興所出南門。上坡至觀\endnotemark[56]音洞。過九峰寺。遍山皆羅漢松。黔山俱童。自此始有林木。谷中多白雲。陰晴皆然。度〖石反〗橋至鸚哥嘴。嘴嶺甚險。有鸚鵡寺。自此以上俱山上\endnotemark[57]生山。大山之水俱注澗溪。小山之水衆峰環遶。無趨洩\endnotemark[58]之道。俱由地中行。或流入洞。當春夏霪雨。山巓汎濫如湖。秋冬水涸。又成陸地。白雲\endnotemark[59]坡甚峻。兩山壁立萬仞。中夾一澗。橫流淙淙。俯而視之。心目蒼茫。新興所當黔滇之交。高山萬重。俱出雲表。關嶺雖峻。\endnotemark[60]亦無出其右也。碧雲洞在郭外數里。石屛當門。遊者撫摩。光潤如玉。\endnotemark[61]幽泉旁流。聲如擊筑。內有石磐。扣之錚錚。入洞甚黑。行百餘步。豁然開朗。一線天也。石罅漏日。洞見一切。黃匏大如斗。瞿曇大士羅漢各一。或倚屛獨立。或傍榻跏趺。或踞崖仰視。鬚眉宛然。絕壁數仞。有龍上昇。鱗甲欲動。爪牙若舞。再進則巨浪排空。驚濤湧地。一溪橫流。燃炬以照。旁有一徑甚窄。側身可入。盤旋數轉。丹竈藥爐在焉。轉彎一浮屠矗天。玲瓏絕巧。再行里許。有石田千頃。\endnotemark[62]石閣五楹。石榻石墩具焉。出洞。則在峯頂俯視萬山。竟同丘垤。雲安坡俗呼雲南坡。高萬仞。極其險峻。至嶺西道濶僅數尺。如一線相連。止可一騎獨行。稍一失足。則人馬俱墜。如轉圓石於仞之山矣。仰視諸峯。皆逼霄漢。諸蠻多聚族而居山半耕鑿。其坡險仄\endnotemark[63]迢遞。將及巓。名龍擺尾者。險絕難行。凡六十丈。又最上爲江滄口始陟頂。此處斷崖成徑。峻滑不可支足。過一小庵。又西上爲避陰坡。凡此三險。總曰雲安坡。又三十五里至大坡。十里娥嫏坡。此二坡亦高而長。又十里至亦資孔。亦資孔者。夷語也。有驛在焉。其地名有革納\endnotemark[64]撒麻蛾螂魯尼多羅矣納者。察皆苗中鄕談。其鋪家之婦當壚招\endnotemark[65]客。其爲桑間濮上亦猶黃絲鋪也。又四十里上坡。乃入滇境。左右有兩坊。一曰滇南勝境。一曰彩徹雲衢。平夷所則雲南境中矣。自楚至鎭遠。則黔省已在最高處。又從黔省至滇南。所過萬山皆拾級而上。間有下坡。然較之上坡。\endnotemark[66]十不及二三。及至此望貴州。如在釜底。向之所歷諸峯參天蔽日者。皆俯而視之。則滇之高不待言矣。過平夷所。南渡兩重石橋。道傍有淸溪洞。深十餘里。諸景與碧雲相類。大抵洞者皆洪水趨洩之門路也。其中景勝。凡洞俱有。皆大同小異。出淸溪後戶即紫泉洞。亦幽深可愛。遊者不倦。過揚威哨。皆如中原坦道。兩山繁林木矣。又多鸚鵡諸禽。鳴聲上下。頗傾客耳。山多鷓鴣。行不得也哥哥六字絕分明。不似他鳥言須以意會。望之如家雞然。交水西北百十里。往烏撒必由之道。交水兩水相交。平疇萬頃。民物豐厚。恍如江南風景。去曲靖府三十里。馬隆州有義象冢。天啓二年。水西安氏叛。撫軍調陶土司禦之。陶有一象。日將暮。伏山澗中。鼻吸泥水數斛。突出咆哮跳躍。鼻\endnotemark[67]噴泥水。直抵賊壘。寇皆驚駭。復以鼻捲一賊。擲空墜死。乘機逐北。遂獲大捷。及曉收師。象中毒弩而斃。土人德之。葬於南山。春秋祭掃不絕。木密關即木密所也。有小關索嶺。上有武侯及索祠。祠前銅馬一。乃唐時物也。古柏參天。俱大數抱。道傍有碑云。武侯平蠻會盟於此。按史丞相亮盟南人於木密。即此也。易隆驛去城十里。有溫泉可澡。大鼎山有海潮寺。寺頗淸幽。多竹木。面海子。濶數十里。週百餘里。隔岸即嵩明州。去寺半里。道旁有毒泉。碣云。此係毒水。飮者傷生。楊林所屬嵩明州。出東關五十五里。即楊升庵題詩處也。板橋驛出西關三十五里。歷鷓鴣哨。度石梁。而至歸化寺。去滇城只五里矣。登金馬山俯瞰城中。煙火萬家。樓閣參差。雙目頓爽。沐國公同巡撫率領文武衆官迎接至此。簇擁進城。送侯捷到公館住下。宣了採買馬匹之旨。巡撫行下各府。立限送驗。送下程請接風。俱不用細說。侯捷閒暇遊覽滇城諸景。會城內有三山。五華其一也。上有武侯祠。螺峯在城東北隅。倚山建圓通寺。頗多亭榭。名人題句甚多。俱刻岩石。松楸頗盛。四時綠陰交覆。白雲瀰漫。差足遊覽。夏桂洲有五言律一首鐫崖石上。其辭曰。

\begin{quotation}

古寺翠崖陰。危亭絕頂臨。

鶴巢松有夢。雲出岫無心。

仄徑攀蘿上。叢臺刻竹吟。

南蠻秋日暝。哀響合猿音。

\end{quotation}

後書。正德十三年秋七月五日。廣信夏言題。崖畔有一洞甚深。洞門外有一潭。洞中一石上有股印。俗傳云係紅孩洞。石上乃紅孩所坐之跡也。城\endnotemark[68]南四十里即太華山。高峻凌虛。下臨昆池。城西三十里即碧雞山。相傳漢時有鳳儀此。所以王褒持節來祀也。城北蛇山。直出雲表。如列屛翰。金馬碧雞坊在南關\endnotemark[69]外。東曰金馬。西曰碧雞。乃百貨滙聚。人煙輳集之所也。富庶有江浙風。金馬坊之東數里。有大白塔。下有四門。訛傳孔明斬孟獲頭藏於內。此不見經傳之言也。然至今猓玀不敢自門內行走。云過則頭痛。亦一異事也。東郭有金牛寺。寺外八角亭中有銅牛一。重將萬斤。以鎭水怪。蓋此地緣溪。每春夏霪雨。東北萬山之水奔流如駛。往往衝圯民居。\endnotemark[70]故範牛以鎭。而水\endnotemark[71]患稍減矣。銅瓦殿在會城東十餘里金馬山西北麓。乃眞武殿。倣武當殿。三楹盡範銅爲之。而飾以黃金。春月遊人畢集。昆明池方數百里。跨昆陽安寧晉寧三州郡。水如倒流。故曰滇水無洩處。或曰由西北流入金沙江以趨蜀。侯捷聞安寧州溫泉有楊升庵題曰天下第一湯。傳云此水甲於諸泉。稱三絕。第一無硫黃氣。二則身有垢。不假澣濯。入水俱浮。三有疥癬者。一澡即痊。往浴之。果如其言。夜觀北斗。訝其甚低。考北京北極出地四十五度。江南北極出地三十二度。雲南北極出地二十四度。則北斗之低也宜矣。地高則風勁。故曰貴州無日不雨。雲南無日不風。風多揚沙拔木。然風每從西南來。未解其故。他在滇中收足馬匹。也差人先送進京。然後收拾起程。有司官皆各有厚贈。他先路過貴州時。已經宣過上諭採辦馬匹。及他回到貴州。馬已齊集省城。他挑選了一番。足了數。也差官押送起身。他又收了許多贐儀。到了鎭遠。他一來下水圖快。二來要賞玩水路的景致。遂坐了〖舟秋〗船到辰州。又換辰船到常德。一路見了些險惡灘洞。而餓鬼洞灘水尤大險惡。浪與舟相觸。滾滾直入艎中。多方掩拒。衣被鮮不淋漓。惡灘更惡之甚者。灘長里許。浪大而石險。舟行稍不戒。輙破碎淪溺。其大王灘二王灘三王灘亦險。而大王灘尤甚。在灘上視前船埋巨浪中。只露桅杪。及下灘回顧後船。如在山巓。雖舟迅如矢可喜。然亦可怖。由平溪行。江右一帶石質如疊雪。每石不下幾千層。方如書帙。高高下下。狀若充棟。沿江不一而足。俗名其地曰千卷書。辰溪縣左岸西有巨室。外貌雄渾而虛其中。名曰鐘鼓洞。洞中有藏書室。相傳穆天子藏書處。楠木洞稍前絕壁之上。石縫中有船。長可八尺許。俗稱仙人所留沈香船也。常德倒水岩仙蛻石。石皆壁立。水濱逶迤高廣。上鑿石竇者十。下臨絕壑。內一竇中藏木槥五。舊傳爲沈香棺。土人云。水漲時。健兒引絙而上。棺朽。遺蛻尚\endnotemark[72]存。舟人戲以竿撩之。雷輙怒擊。亦未知何代所留。善卷山。堯時善卷讓位。避居此山。今孤峯絕頂有善卷先生古壇。枉渚有善卷先生釣灣。其村亦曰善卷村。山容聳秀。曲渚依流。令人有出塵之想。沅江至此如一砥柱。過此則百里平疇。直趨洞庭矣。洞庭湖白泥窖長十餘里。湖水淺不及尺。舟行須水尺五。不得已。盪舟膠泥中。螺蚌碎石與艎板相軋聲。剌剌不休。適以風猛甚。瞬息而過。舟人以爲此乃神窖。非風不行。數里之地。水涸時。人力推挽。行一二日者有之。篙頭皆綴橫木。形\endnotemark[73]如卜字。其銳者一入膠泥不能復出。過此。楠木窖洞庭夾。未至夾數里。四天陰霾。舟行黑風濁浪中。舟子驚相耳語。剪牲焚楮。色甚匆遽。初不解其故。少頃。見神木直逼舟傍。不及一丈而返。遙望課船。遇之船破。賴賈筏得救。舟子色稍定。乃曰。此楠木神。每遇暴風晝晦。輙出游湖中。神首色沈綠如螺髻。往來於神木窖之前後左右。終古如斯。故稱神木。岳州城門左側有鐵牛一。蹲踞西望而張其口。若有吞湖之意。想亦五行尅制之理。與滇省銅牛制水之義同。門外砂磧中置鐵鈕五。其一較小。不知起於何代。亦竟不知何用。新堤爲魚米積聚之地。沿江廬舍綿亘十五里。有小江通沙湖。上下洪湖及沔陽仙桃荆州安陸諸處。商賈雲集。井陌成行。有豐亨之象。漢口南數里。則漢陽府治。東渡江即武昌省城。十里之內置郡者二。蓋上當滇黔秦\endnotemark[74]蜀之衝。下控左右兩江之要。故特於此嚴鎖鑰焉。商城。古高陽氏封子庭堅於此。漢成帝綏和元年。封殷後孔佶爲紹嘉侯。故曰商丘。楚相孫叔敖埋蛇之地也。田家鎭有吳甘興霸廟。地有神鴉。往來江上。帆檣過此。不拘餅餌粒食。撇空飼之。羣鴉飛舞接食。百無一墜。食畢。間有集舟檣之杪送出廟境。俗謂將軍遣使送客。其聲啞啞類慈烏。上下三十里皆有之。亦一奇也。二十里過富池。百一十里到九江。過涇江口\endnotemark[75]鎭。俗云鱒魚嘴。土人言此地每歲有豬婆龍爲害。天寒水涸。輒崩岸壞屋廬。今舊岸已在大江心。泊舟者油物煎熬。龍即出舟。人切戒之。夏月則不忌也。自此以往。經安慶蕪湖采石抵南京上新河。沿途無可紀錄。直怒帆張風長江順流直下而已。宦蕚同侯捷飮酒之間說道。常聽得人說萬里雲南。我當是離天邊不遠。不想二哥竟有此一遊。可將所見所聞詳細向我說一番。我記在心裡。一則長些見識。二則後來會着人說雲南的古蹟。我也好說說天話。侯捷從北京起身。歷河南陝西到四川。自川至湖廣。走貴州上雲南。把六省所見所聞的景致說與他聽。宦蕚聽得比每常叫人念鼓兒詞還覺有味。所以日日不放\footnote{愷(獃)公子之習氣每每及此。不爲作者刻薄。}。飮酒畢。大家到晚安歇。次日。戲筵款待。約了賈文物童自大相陪。次日。侯捷要行。宦蕚侯氏要留他多住幾日。侯捷道。奉命限期只許一年。今已將滿。不敢躭延。他夫妻見說有日限。也不便強留。賈文物童自大來拜。賈文物覿面專請。他也力辭\footnote{庸俗之輩專好密(覓)此等交。}。侯捷忙去一答拜。就要動身。宦蕚吩咐家人廚役往浦口去備麼(宴)餞行。他親自送過了江。雇了頭口。宦蕚陪他住了一宿。次早。回京復命去了。侯捷的大管家私下孝敬了姑老爺兩個緬鈴。一個。有黃豆大。是用手攥着的。一個有榛子大。有鼻如鈕。是婦人爐中用的\footnote{此管家竟識竅。不愧爲大管家矣。下文方得姑爺厚賞。}。宦蕚大喜。賞了他二百兩銀。當日晚間便同侯氏試驗。叫他手攥着一個。陰戶內送進一個。侯氏遍體酥麻。樂得哼聲不絕。次早。用絲綿包好。如寶貝一般收貯候用。要知後事。須看下文。

姑妄言卷十一終



\endnotetext[1]{「悍妻」原作「妒妻」,「妒心」原作「悍心」,據書前目錄改。}

\endnotetext[2]{「巴不得」原作「得不的」,據文義改。}

\endnotetext[3]{「吩咐」原作「咐分」,據文義改。}

\endnotetext[4]{「疲犬」原作「〖疒彼〗大」,「反」原作「及」,據文義改。}

\endnotetext[5]{「芝蔴糖」原作「脂麻餹」,據文義改。}

\endnotetext[6]{「踵」原作「〖目重〗」,據陳鼎《留溪外傳》卷十七《厲鬼傳》改。}

\endnotetext[7]{「鍤」原作「揷」,據陳鼎《留溪外傳》卷十七《厲鬼傳》改。}

\endnotetext[8]{「風」字原無,據陳鼎《留溪外傳》卷十七《厲鬼傳》加。}

\endnotetext[9]{「沮」原作「伹」,據陳鼎《留溪外傳》卷十七《厲鬼傳》改。}

\endnotetext[10]{「吝」原作「各」,據陳鼎《留溪外傳》卷十七《厲鬼傳》改。}

\endnotetext[11]{「改」字原無,據陳鼎《留溪外傳》卷十七《厲鬼傳》加。}

\endnotetext[12]{「羅」原作「罷」,據陳鼎《留溪外傳》卷十七《厲鬼傳》改。}

\endnotetext[13]{「訣」原作「決」,據陳鼎《留溪外傳》卷十七《厲鬼傳》改。}

\endnotetext[14]{「倚」原作「依」,據陳鼎《留溪外傳》卷十七《厲鬼傳》改。}

\endnotetext[15]{「家」原作「衆」,據陳鼎《留溪外傳》卷十七《厲鬼傳》改。}

\endnotetext[16]{「珍」原作「畛」,據上文及陳鼎《留溪外傳》卷十七《厲鬼傳》改。}

\endnotetext[17]{「止」原作「上」,據許纘曾《滇行紀程續抄》改。}

\endnotetext[18]{「軿」原作「駢」,據陳鼎《黔遊記》改。}

\endnotetext[19]{「畜」原作「蓄」,據陳鼎《黔遊記》改。}

\endnotetext[20]{「陡」原作「陟」,據陳鼎《黔遊記》改。}

\endnotetext[21]{「郡」原作「即」,據陳鼎《黔遊記》改。}

\endnotetext[22]{「新」下原衍一「新」字,據陳鼎《黔遊記》刪。}

\endnotetext[23]{「合」原作「令」,據陳鼎《黔遊記》改。}

\endnotetext[24]{「剖得一男」原作「部得男」三字,據陳鼎《黔遊記》改補。}

\endnotetext[25]{「羣」原作「郡」,據陳鼎《黔遊記》改。}

\endnotetext[26]{「黑」原作「里」,據陳鼎《黔遊記》改。}

\endnotetext[27]{「建」下原衍一「再」字,據陳鼎《黔遊記》刪。}

\endnotetext[28]{「雨」字原無,據陳鼎《黔遊記》加。}

\endnotetext[29]{「其」下原衍「餘步其」三字,據陳鼎《黔遊記》刪。}

\endnotetext[30]{「盛」原作「勝」,據許纘曾《滇行紀程續抄》改。}

\endnotetext[31]{「酋」原作「酉」,據許纘曾《滇行紀程續抄》改。}

\endnotetext[32]{「黑」原作「里」,據陳鼎《黔遊記》改。}

\endnotetext[33]{「苗」字原無,據陳鼎《黔遊記》加。}

\endnotetext[34]{「中原」原作「原中」,據陳鼎《黔遊記》改。}

\endnotetext[35]{「劫」原作「却」,據陳鼎《黔遊記》改,下文或同。}

\endnotetext[36]{「箐」原作「籌」,據陳鼎《黔遊記》改。}

\endnotetext[37]{「黔」原作「箱」,據陳鼎《黔遊記》改。}

\endnotetext[38]{「貲」原作「資」,據陳鼎《黔遊記》改。}

\endnotetext[39]{「曰愛莫離,愛莫離者」原作「曰愛離莫者」五字,據陳鼎《黔遊記》加改。}

\endnotetext[40]{「漢郞者」三字原無,據陳鼎《黔遊記》加。}

\endnotetext[41]{「泉」原作「帛則帛」三字,據陳鼎《黔遊記》刪改。}

\endnotetext[42]{「萬」字原無,據陳鼎《黔遊記》加。}

\endnotetext[43]{「右」字原無,據許纘曾《滇行紀程》加。}

\endnotetext[44]{「聲也」、「又五里,上望水亭」九字原無,據許纘曾《滇行紀程》加。}

\endnotetext[45]{「穾」原作「突」,據許纘曾《滇行紀程》改。}

\endnotetext[46]{「跑」原作「跪」,據陳鼎《黔遊記》改。}

\endnotetext[47]{「皆」字原無,據許纘曾《滇行紀程》加。}

\endnotetext[48]{「順忠」原作「忠順」,據陳鼎《黔遊記》、許纘曾《滇行紀程》改。}

\endnotetext[49]{「亂」原作「乳」,據陳鼎《黔遊記》改。}

\endnotetext[50]{「題」字原無,據陳鼎《黔遊記》加。}

\endnotetext[51]{以下有錯簡。下文自「鐵索橋,駕盤而造」至「自此始有林木,谷中多白」爲一葉(首行書眉註明「此在前二篇」),原置於二葉之後,今據文義移至「雲,陰晴皆然。︙︙」該葉(首行書眉註明「在後二篇」)之前。}

\endnotetext[52]{「俱」原作「供」,據陳鼎《黔遊記》改。}

\endnotetext[53]{「節」原作「飾」,「縣」原作「嗣」,據陳鼎《黔遊記》改。}

\endnotetext[54]{「滇」原作「漁」,據陳鼎《黔遊記》改。}

\endnotetext[55]{「揷」原作「抽」,據陳鼎《黔遊記》改。}

\endnotetext[56]{「觀」原作「覩」,據陳鼎《黔遊記》改。}

\endnotetext[57]{「上」字原無,據陳鼎《黔遊記》加。}

\endnotetext[58]{「洩」原作「曳」,據陳鼎《黔遊記》改。}

\endnotetext[59]{「雲」下原衍一「墟」字,據陳鼎《黔遊記》刪。}

\endnotetext[60]{「峻」字原無,據陳鼎《黔遊記》加。}

\endnotetext[61]{「玉」字原無,據陳鼎《黔遊記》加。}

\endnotetext[62]{「頃」原作「傾」,據陳鼎《黔遊記》改。}

\endnotetext[63]{「仄」原作「反」,據許纘曾《滇行紀程續抄》改。}

\endnotetext[64]{「納」原作「約」,據陳鼎《黔遊記》改。}

\endnotetext[65]{「招」原作「抬」,據許纘曾《滇行紀程續抄》改。}

\endnotetext[66]{「坡」原作「坂」,據許纘曾《滇行紀程續抄》改。}

\endnotetext[67]{「鼻」字原無,據陳鼎《滇遊記》加。}

\endnotetext[68]{「城」下原衍一「西」字,據陳鼎《滇遊記》刪。}

\endnotetext[69]{「關」原作「門」,據陳鼎《滇遊記》改。}

\endnotetext[70]{「民居」原作「居民」,據陳鼎《滇遊記》改。}

\endnotetext[71]{「水」原作「少」,據陳鼎《滇遊記》改。}

\endnotetext[72]{「尚」原作「高」,據許纘曾《東還紀程》改。}

\endnotetext[73]{「形」字原無,據許纘曾《東還紀程》加。}

\endnotetext[74]{「秦」原作「泰」,據許纘曾《東還紀程續抄》改。}

\endnotetext[75]{以下原書空一頁一行又三字,然非缺頁,參許纘曾《滇行紀程》可知。}

\setcounter{footnote}{0}

\theendnotes

\part*{姑妄言第十二卷}
\addcontentsline{toc}{part}{姑妄言第十二卷}
\markboth{姑妄言第十二卷}{姑妄言第十二卷}

鈍翁\endnotemark[1]曰。錢貴約鍾生之來。一則久別所必至之情。二則錢貴〇〇〇夢兆。提出鍾生見色不迷之公案以勸警世人。又〇宦蕚縱惡張本以完題面。

〇〇〇〇一段。寫淫婦之巧言飾非。愚父之溺愛聽讒。察〇〇〇托狥私。和尚之奸狡脫罪。一一寫出。至於結果衆〇〇夫婦者。不肯使此輩得志。且令馬士英家醜外揚也。董\endnotemark[2]布德的是國手。今日之名醫皆本他傳授而來。律中庸〇〇〇〇〇〇再犯不着。

〇〇〇〇傳法於馬台。想頭奇絕。不愧爲撫養之乳母。善〇〇〇〇也。師死。爲弟子者心喪三年。乳母死。緦麻三月。〇〇〇於此師當衰。經於此乳母當期。庶可以報敎訓之〇〇。

〇〇自譽佳作。在今日無不皆然。宦蕚想入詩社。亦紛〇〇是無足異也。

頂燈長跪。正假斯文之伎倆。販賣漢或倒不如是。

要刻畫多銀之淫之醜。故寫宦蕚到游家來同楊爲英作〇〇求淫不遂。一番大罵。至游夏流被夾。方更見其醜態〇〇。

〇〇〇出氣一段。不過隨筆成趣。

〇〇〇〇之有賢妻也。他人之癰可吮。而閫內夫人之痔〇〇〇〇耶。罪當云何。熊氏以游夏流一言而怒其罪。豈〇〇〇〇〇〇〇〇〇游夏流能以巧言解甘壽之禍於〇〇〇〇〇〇〇〇〇〇愈顯多銀之惡過於熊氏也。

〇〇見書中云。徼外之女國有四。有一國以犬爲夫者。大〇〇〇從彼處效法來。

〇〇〇〇焦氏水氏之淫。楊大兩夫妻眞是對手。水氏之〇〇〇〇其量不下昌氏。可惜二人不曾一較。水氏半日〇〇〇〇夫。卜通數年僅獲一焦氏。得便宜處失便宜。善〇〇〇〇往如是。鑒卜通之事。愛便宜人亦可爲哉。

〇〇之妻至於淫狗。已不勘(堪)言矣。且更至於淫驢。游於下流者當愼之。多銀可謂不圖。爲樂之至於如此其極也。可謂樂極生悲。人生十分太過之快心事不可多做。亦當作如是觀。或有謂多銀一段。事之必無。未免墮口業罪。余笑曰。子看書不照應前後。反責他人耶。多銀是何人托生。狗與驢又是何人轉世。將前面一想。此一罵猶不足盡他三人之罪也。

宦賈童在錢家肆惡一段。此銷繳三人惡字餘波。今而後不復蹈前非矣。

鍾生錢貴有此一抑。此後盡揚眉吐氣時矣。禍兮福所倚。正以喩人失意處不足介意也。

\chapter*{姑妄言卷之十二\\
第十二回 鍾情百種鍾情 宦蕚一番宦惡\\
附 甘壽表弟兄受閫罪無輕重 水氏親母女淫人畜有死生\endnotemark[3]}
\addcontentsline{toc}{chapter}{第十二回 鍾情百種鍾情 宦蕚一番宦惡}
\markboth{第十二回 鍾情百種鍾情 宦蕚一番宦惡}{第十二回 鍾情百種鍾情 宦蕚一番宦惡}

話說錢貴那日自遇宦蕚衆人之後。心中甚是自悔。暗思道。瓊枝玉樹。安可置於糞土之中。況人生百歲。孰無一死。豈可畏刀避劍。與狂且爲伍以自汚乎。今後任是勢豪紈褲。雖鼎烹斧銼。萬不可再辱。又想起鍾生儒雅彬彬。風流瀟灑。更嘆人才難得。因想起昨日場期已過。鍾生不知可得意否。欲待約他來一會。因作了一首詩寄他道。

\begin{quotation}

愁心悲夜月。病體怯秋風。

爲憶多情種。思來入夢中。

\end{quotation}

寫下了要寄去。又無人可托。悶坐懨懨。竟幾乎有個眞病之勢。次日。悶臥在床。忽代目來說道。那梅相公在外面來看姑娘呢。錢貴正想人寄信。聽見他來。忙扶了代目迎出來一拜。讓了梅坐(生)坐下。梅生說道。久未得來相訪。今偶步過此。特來奉看。錢娘爲何淸減了些。錢貴道。自從暮春別後。懨纏一病至今。故此瘦損。梅生道。鍾兄一向可曾來麼。錢貴道。鍾相公原說要在家中用功。故此不曾到這裡來。但作(昨)日場期已過。相公定然在場中相會的。梅生道。我還是七月內同他相會。近日因寒家有些要緊的事。連場期都躭誤了。這一向未曾得會。如今正要去望他。錢貴道。相公若去。妾有一事相煩。敢求勿却。梅生道。錢娘有事。但說何妨。錢貴道。相公若遇鍾郞。懇將妾意轉達。妾數月來望眼將穿。此衷時刻如有所失。況鍾相公場期已過。斯時已無事矣。請到寒家來一晤。以解思念之苦。還有一小柬。欲求寄去。不知相公肯金諾否。梅生道。我當有甚大事。這便道傳書。有何難處。因笑着道。我今若見了鍾兄。只用對他說兩句舊詩。道錢娘

\begin{quotation}

自從別後減容光。半是思郞半恨郞。

\end{quotation}

他聽見了這話。或者今日就來。雖是中秋後一日。也還是月圓之夜。來

與錢娘做一個人月雙圓也不可知。若不然。或他有事。明早必到。錢娘

但請放心。我此刻就行。錢貴叫代目將昨日封了的那一首詩取出。送

與梅生。梅生遂起身辭去。錢貴見他已帶了信去。知鍾生多情必來。歡

喜非常。在房中炷了一爐好香。叫代目把床上換了一副新衾枕。預備

了些酒肴等候。又淨了一淨下體。是不用說。看看至晚。正在思念之際。

耳中忽聽得說。鍾相公來了。【這一句寫得入神。妙。錢貴此時一心惟以鍾生爲念。目不能視。惟側耳靜聽。忽聞鍾相公來了五字。如轟雷灌耳。心喜非常。並不曾聽得

是誰之聲音也。】錢貴喜動顏色。因無人在傍。自己忙摸出房門來。那鍾生早已走

近前扶住。道。賢卿一向好麼。錢貴聽得果是鍾生。如同天降。二人攜了

手進房坐下。代目忙點上一枝大燭。隨捧過茶來吃了。郝氏聽得說那窮酸又來了。不由得氣起。張了張。見他雖穿得不爲富麗。比前番體面了許多。恐這一次或者有嫖資。也假做歡笑。進來陪坐了一會兒方去\footnote{一者寫虔婆勢利。二者暗寫鍾生前得錢貴之贈也。}。鍾生見郝氏去了。攜着錢貴的手。道。自別賢卿之後。幾至廢寢忘餐。感卿深情。形於夢寐。因讀書無暇。故不曾得來相探。昨出場之後。本待就來。因連日困倦。在家稍憩\footnote{這幾句萬不可少。不然鍾生之於錢貴。萬種深情。豈待約而始來耶。}。今早本擬要來。因有朋友過訪。不得脫身。午間會着梅兄。說賢卿芳容憔悴。又見佳章。知望我甚切。今特來看你。錢貴道。自君別後。妾想念之苦欲言。非片時可罄。容當細訴。但君昨鏖戰文場。可得意否。鍾生道。我昨在場中十分努力。雖自覺頗有可觀。但恐才疏命薄。不知可能博朱衣暗點否。錢貴道。郞君高才。雖未必掄元奪解。定獲高魁。妾前已得嘉夢。高發無疑。況多情若此。上蒼寧不垂念。鍾生撫他之背。笑道。賢卿有何夢徵。大約是企望我徼倖心切。故形之於夢耳。錢貴道。不然。妾自與君定盟之後。煩名手繪了一幅慈航大士小像供養。每日晨昏虔誠焚香頂禮。通郞君之名。懇求默佑。妾也不學那愚夫去持齋念佛。每日但將小靑的那四句詩。

\begin{quotation}

稽首慈航大士前。莫生西土莫生天。

願垂一滴楊枝露。灑做人間並蒂蓮。

\end{quotation}

妾將他當了經典一般念誦。那一夜。似夢非夢。聞得半空中人語喧鬧。忽聽得大聲呼君之名道。第四十八名鍾情。兩次見色不迷。拔置高魁。妾夢中歡喜醒來。忙到大士香案前叩拜。案上每夜點燈的。忽然一個燈花炸得奇響。爆到我的臉上。代目又說燈光忽吐兩焰。明亮異於常日。此豈非郞君高發的先兆。郞君必定還有甚麼陰功。鍾生聽了他這個夢。想着月餘前郗氏李氏的事。此言不爲無據。又懼又喜。懼的是神靈咫尺。昧心即是害己。欺人即是欺天。前日若有一毫苟且。眞是一失足成千古恨了。喜的是倘若應了他的夢。不但自己耀祖榮宗。且可以娶他報恩酬德。心雖如此想。却不肯說出郗氏李氏的話來。便道。我一介寒儒。何處來的陰功。至於說見色不迷。我生平從不敢淫人妻女。說話間。代目捧上酒肴來。擺列停當。錢貴要了一個酒杯。滿貯香醪。高高持在手內。奉與鍾生。鍾生笑着忙起身接下。道。我二人舊知心。何勞賢卿錯愛如此。錢貴笑道。預賀新貴人。敢不致敬。鍾生亦斟上一杯。道。我若是新貴人。卿就是新貴人之妻了。亦當奉賀一杯。遞在他手中。錢貴接了。二人喜笑着一同飮過。代目又從新斟上來。二人訴一番相思苦楚。講一會恩愛深情。說說笑笑。飮得甚是有興。錢貴道。妾向日自別君後。以此身有托。曾作絕句一首。以誌欣喜。但俚語不堪。諒情郞決不笑我。鍾生道。賢卿佳作。自然精工。你我知心。爲何忽然作此謙語。錢貴道。妾非謙辭。於郞君之前屢屢不惜獻醜。恐汚君之目耳。因叫代目將向日的那詩取出。遞與鍾生。鍾生接過看了。道。卿之佳作。雖班姬道韞不能過此。但內中企望我甚切。不知我可有福能副卿之望否。錢貴道。中之一字。郞君不必過慮。但只是一件。郞君一金榜題名。妾就望洞房花燭了。鍾生道。這是我自己身上的大事。何須卿囑。二人又飮了數杯。錢貴又備述別後矢身槪不會客。雖遭母親凌逼。誓死不從。後因宦蕚來訪。將他母親苦勸的話。並他不得已的意思。說了一遍。又道。妾誠負君。望君垂諒。鍾生道。卿之心跡。或(我)豈不知。但爲我如此。使我感愧交集。所說不得已陪侍宦蕚。但此人是本地有名作惡的呆公子。我雖未覿面。聞人之笑罵久矣。卿昨屈身侍彼。還是知機的妙事。若不然。這呆公子一時發起呆性來。就有不測之事了。錢貴將他三人粗俗假文。把行令譏誚他的事。也細說一番。又將編了打趣他們的那首詞也拿與鍾生看了。二人大笑。又吃了幾杯。叫代目把杯盤收拾了去。方攜手上床。解衣就寢。這是半年久別。兩次相親。更加恩愛。千般旖旎。百種綢繆。自不必說。次日起來。錢貴對鍾生道。君今已無事了。可多住數日。俟放榜之期再回家聽喜。何如。鍾生應允。遂住下了。再說那個牛氏。在察院門口光着屁股擡到他父母家中\footnote{大約自古以來。出嫁之女從未有光着屁股回家歸寧父母者。牛氏定算破天荒頭一個。}。他生母計氏見女兒這樣個裝束。含着淚。一把攙住。到自己房中。忙拿衣服與他換。見他下身光着。咬牙切齒。咒罵那些家奴。忙取出一條新褲與他穿了。見脚上還穿着睡鞋。又拿高底鞋褶衣。都叫他穿上。梳洗了出來\footnote{細甚。脚穿睡鞋。未曾梳洗。是半夜被拿獲者。}。到上房見了牛質。苟(牛)氏放聲大哭。反埋怨父親。說把他嫁了恁樣個女婿。呆得人事不知。只會穿衣吃飯。家中事務一絲不能照管。公婆不在家。我少不得當家料理。這些奴才不服拘管。我前日到公婆任上。公婆問我。我細細說了。每人打了一頓。是哥哥親眼見的。他們心中懷恨。我昨日因悶得慌。叫了個老和尚來宣卷。夜晚了。就叫他到祖先樓上去睡。他衆人男婦串通。今早有五更天氣。他們到樓上把和尚拿了下來。我還當是強盜來打劫。嚇得癱在床上。只見他們如狼似虎凶神般。生生的在被窩裡把我拉出來。做起這一番事汚衊我。爹你想一想。一個八九十歲的老僧。一風都吹得跌倒。還做得甚麼壞事。若是年少些的。我也不肯留了。就算着女婿不知道甚麼。我若做一點沒廉恥的壞事。養媽媽是他的一個奶媽。他也依得麼。公婆不在跟前。爹再不替我做主。我也沒臉面到他家去了\footnote{善偷漢的婦人。再無有不善說者。此無足異。古云。婦人無才便是德。伶牙俐齒。善於巧說。無詭譎之才。焉能及此。那一種愚而且鹵。訥訥不能出一語之婦。雖欲偷漢。亦無此才。世人有能幹善說之妻。不可不深防之。}。苟氏此時雖四十八九。兒子牛耕也老大的了。他還時常同胡旦比比肚子。做那摩臍過氣之法也。恐人談論。便接口道。我的兒。你不要急。不要說你年小小的。沒有這樣的事。他們寃賴你。就做着主子不長進。幹了這樣醜事。奴才也是管不得的。這些奴才們這樣放肆。你爹要不替你做主。外人不笑話你。把你爹就不當人了。那牛質先聽見女兒出了醜。心中也甚是忿恨。還罵了計氏一場。說他做娘的脚跟不正。纔養得女兒不長進。計氏此時在傍說道。老爺方纔罵我。因在氣頭上。我不敢說。老爺看看這樣花枝般知文達禮的女兒是不長進的\footnote{別的文或未必知。詩經上鄭衛兩國風大約記得很熟。}。若果然女兒幹了壞事\footnote{你令愛竟果然不曾幹了壞事耶。}。自家打掉了牙。只好嚥下肚去。今日被奴才們陷害。若不替他正過這名聲來。不但可憐女兒一輩子擡不起頭來。見不得人。就是老爺人上做人的人。也難見親友。牛貿(質)聽了女兒這些巧語。又被苟氏一激。計氏又拿話敲打着。大怒道。這起奴才。不但辱了我家。連他主子的臉面也沒了。我兒。你不用哭。也不必惱。我替你報仇。處死這些奴才。方才出得氣。我還寫信與親家去。說知男婦通謀這些詳細。牛質就到他族兄牛尚書家。把前後的話說了。尚書也十分動怒。差長班請了壽察院來。當面細言其故。並托他從重處治。做官的人聽得家奴誣害家主。可有不惱恨者。況是老師的話。自然是眞的了。次日到堂上提出衆人。他昨日見老師所囑。過後細想。還恐有情弊。心中也還未十分釋然。今見了這和尚老到這個樣子。他是裝慣了老的。走着還像要跌倒一般。豈能做風流的勾當\footnote{這場官司打得奇。和尚本是被吿。不意反做了干證。}。況那十六七歲的少婦。可肯愛這樣個老朽。明明是假捏奸情了。又見這幾個家奴。精壯壯的小夥子。硬頭硬腦。越發無疑是同謀害主。遂拍案大怒道。你這些奴才。奸謀狡詐。陷辱主母。萬死莫贖。家家都像你們這樣的惡奴。也不敢用下人了。吳知抗聲道。小的們跟隨小主親自捉奸。如何是陷害主母。衆人都纔要分辯。察院愈怒。喝叫掌嘴。傍邊答應一聲。皮靴底就是幾下。嘴中鮮血直冒。吩咐夾起來。衆役喝了一聲。便都夾起。衆人但一張口。就打嘴巴。這幾個家人只得咬牙死捱。又叫敲了五十槓子。每人四十大板。和尚入人內室。本該薄責捌拾。免刑攆了出去。察院回到私宅。差人去復了老師。牛尚書大喜道謝。便即刻着家人去報知牛質。他一家夫妻母子心中大樂。纔把怒氣出了。這一起在大官府家做大叔的人。仗着主子的勢騙銀錢受用是他本等。何嘗受過這夾而打之。嘴其巴焉的全副重刑。況是前在老主任上蒙恩賞了那大板來的尚未痊癒。這叫做雪上加霜。兩人已斃杖下。那三個擡了回去。捱了幾日。也就完賬。却說馬士英得了親家的書子。着實不好意思。連夜差人回來復信。與親家陪了許多不是。叫接媳婦回家。又叫來人把這幾個家奴拿去任上重處。後聽得都死了。又有信來。叫把這幾個僕婦盡皆賣去。牛質見親家如此週到。把女兒送了回去。牛氏將幾個丫鬟僕婦從頭至足拆洗一番。細細的敲打。以洩前番之恨。然後叫媒人來領出。吩咐都要賣與娼家。身價不惜與他平分。這媒人的心是秤鈎一樣。還安個倒鬚的。可有一個略有天良。這不過是做主子氣頭上的話。他只圖分得銀子多。竟遵命奉行。把這些婦女全全送入煙花之內。香姑只想把他們下了地獄。心中稍舒宿恨。圖一時快樂。就不想到這些婦人到了這個場中。還怕你拿主母的勢打罵他不成。他這一腔忿氣。遇一個孤老。就把主母的妙處稱誦一遍。醜話只有增無減。把這位馬士英之令媳。牛質之乃愛。呆子之令正。乳名香姑的美名。幾幾傳遍天下。所以人知之甚詳。香姑雖把人送下地獄。自己這個聲名也就下了地獄了。古人兩句道得好。他說。

\begin{quotation}

責己備以周。責人寬以約。

\end{quotation}

誠至言也。那香姑雖出了氣。却也再不敢招攬那和尚了。又買了幾個老實丫頭使用。又過了有兩三個月。牛氏忽然吞酸嘔吐。害起病來。茶飯懶吃。伏枕不起。誰知腹中有了和尚的法嗣。害的是人病。他從不曾經過這種症候。也只當是身子不快。這馬台的乳媼養氏。先牛氏的這些事。他豈不知。但和尚是他招惹了來的。日間與牛氏解悶。誰知他竟拿了夜間解悶。事由己起。說不出口。二來馬台是他乳養大的。他要爭體面。怎肯說牛氏偷漢。說不得。不說不得。故只推不知。雖如此說。心中恐老主責備。未免憂慮。見事體已完。心纔放下。今見牛氏有病。養氏也不疑他是害喜。只同老和尚略偷了偷。那裡肚中就有了小和尚。叫人去請了常走動的一個醫生。姓董名布德\footnote{他原是取陽春布德澤之意。}。就借他的名字起了個混名。叫做懂不得。來看香姑。到了內室。那牛氏自帳子裡伸出手來。這懂不得將三個指頭也不知放在那裡。按了一按。便道。知道了。請問這是位奶奶還是位姑娘。要是位奶奶。像是有喜。要是位姑娘。就決乎不是。養氏道。是位奶奶。他道。老奶奶。這位奶奶可是喜不是。養氏道。先生看脈。如何反問我。他道。老奶奶。不是這話。我又不是他肚裡的蛔蟲。怎知他肚裡有喜沒有。脈上雖像是有喜的。然而拿不穩。你們看着肚子大不大就知道了。養氏道。看不出來。他道。這就像不是了。大約不知可是感冒風寒。又不知是停滯飮食。再不然或者就是經水不調。養氏道。他有些發惡心嘔吐。懶吃東西。經水有兩三個月不行了。懂不得道。何如。我就是活神仙。一切脈就知道是停滯飮食。這惡心嘔吐就是胸中有宿食了。這經水不行。或者是有喜。還不可知。這値甚麼。不過十日。包管就略好些。養氏道。先生該用甚麼藥。開個方子。他道。這停滯飮食。吃湯藥尅伐了脾胃傷人。況又恐怕是喜。煎藥傷了胎氣了。當不成府上這樣門第一位正經奶奶的貴恙。可是兒戲混下得藥的。我家有異人傳授祖留的仙方做就的萬應至寶靈丹。百病俱治。慢慢的磨去疾患。把這積滯一淸理了。自然就好。但這個藥工本大得很。我是半積陰功半養身。照本只要五錢紋銀一丸。先取十丸來吃了。看好了便罷。要是還未必就全好。再服十丸。自然見些功效。好了却是要謝的。先小人而後君子。我先說下。說完。起身去了。他這樣人家可稀罕五兩銀子。就封去取了十丸來。他這個何嘗是藥。他因一毫醫理不懂得。倒還有良心\footnote{近日行醫有良心者是誰。}。不敢混下藥。怕吃壞了人\footnote{有這一點菩提心。就該發財。較庸醫費人者。何啻天淵。}。他拿粳米炒煳磨麪沙糖爲丸。有核桃大\footnote{此藥不但可以治病。還可以備荒。荒年無糧。大約服二三丸。豈不捱過一日耶。}。每服一丸。白滾湯調下。他也並非定價五星。總是看人家起發。三錢五錢一錢五分不等。遇了小戶人家。十個錢他也賣。藥本用不得二厘一丸。這還有一本七八利呢。病重的人家見吃不效。少不得另去請人。病輕的捱好了。他却要居功受謝。他但只時運好甚。請他的都是不該死的病。被他這鍋巴丹也治好了許多人。有一個大老卒得暴病。請了他去。一丸鍋巴丹竟救好了那病。也不知因那一經而起。忽然舉發。忽又自好。那大老以爲是他的靈丹治好。送了他一個匾額。是一匕回春四個大字。有那不識字的。念做一七回春。說死了一七的人他還救活了。因此闔城轟傳他是馳名的國手。久之。他將名醫二字也竟居之不疑。這牛氏吃了他十數丸。他原是害娃娃。過了那幾日就妙了。又得了馬台的神針替他一治。竟果然百病消除。却被這懂不得索取了一分謝禮。但這馬台呆到極處。他會用甚麼神針。有個緣故。那養氏見香姑患病。吃那丸藥不甚見效。疑他少年幼婦。想是腰中那小和尚。那知他肚裡害的是那小和尚。但馬台呆到這個分地。再敎不會。急出一個主意來。對他丈夫單佑道。這呆子這樣呆。怎麼處。他這呆頭呆腦。我們也不必怕他。竟面前做了敎他。或者學會了。也不可知。不然躭誤着人家女兒。怎麼是了局。那單佑道。也罷。就是這樣行。他夫妻二人脫光了。叫馬台在傍邊坐着看。一面敎他。單佑把陰戶指與他看了。把自己陽物也與他看了。然後叫他看着。送了進去。抽抽動動的。問了他幾遍可會了。又對他道。娶了那女人與你。就是叫你這樣的。他也知道點頭答應。他夫妻就大抽大弄。做了一回與他看。單佑還不放心。扯開他褲子看看。他那一根陽物竟硬而且大。又再三囑咐他。捏着他的陽物。指着養氏的陰戶。道。你今日晚上同那女人。把你這個送在他那裡頭去。他看見養氏的陰戶大張着。許多黑毛。他指着道。我怕這鬍子嘴會咬我的雞雞。他夫妻忍不住大笑起來。單佑把他的牙摸着道。那是沒有這個的。不會咬。恐他害怕。拉他低着頭。將手〖扌扉〗開陰與他看。道。你看見了。可是沒有牙的。養氏笑着向他道。你不要怕。那個人連鬍子都沒有。還是張光嘴呢。單佑恐他還不懂得。二人又做了一齣與他看\footnote{司富之敎宦蕚也易。養氏之敎馬台也難。一則宦蕚不過愚蠢。尚呆不至十分。司富又以己身親敎之。故會之甚易也。馬台則呆之至矣。且養氏又是乳母。不便以己身設敎。想出兩口做事使彼效之。此亦萬不得已而思其次也。幸而一敎竟會。足見此一事。世間即呆至於馬台者。猶敎而能之。可見色之一字。人人不能免也。}。到晚來。養氏把他帶了上來。此時牛氏已經睡下。那養氏把丫頭都叫出去。關上門。附着牛氏的耳低聲道。我敎會了他了。你兩個成親罷。那牛氏還假裝害羞不肯。養氏道。這是你一生的大事。還要你敎着他些。你倒還是這樣的。動手替牛氏把衣褲脫光了。牛氏正在飢渴之時。只含羞微笑。閉着眼。任他所爲。叫他仰睡了。纔替馬台脫了衣服。扶他上床。養氏又拿過燭來照着。指與他看。道。可是沒有鬍子又沒牙。你不要怕。看他那厥物也竟跳了起來。養氏扶他上了肚子。捏着陽物。替他對了門戶。安上了笋。笑着道。我們先敎你的那麼動。你也動罷。我去了。馬台道。媽媽。你來帶我睡。養氏道。叫他帶你罷。我在那邊睡罷。明日來帶你。笑着帶上門出去了。牛氏見他人雖呆。倒有根成文的陽具。比和尚的還肥胖長大些。心中甚喜。見他伏着不動。便扶着他胯骨。道。你動動。搊着他一上一下的抽。他也就知道了些。弄了好一會。牛氏竟丢了一次。滿心暗喜。只見他又動了幾下。身子伏了下來。叫道。阿洗喲。阿洗喲。牛氏當是他要屙屎。忙道。你要屙屎。下地去屙。他道。不細喲。阿快活洗喲。牛氏聽他說要屙快活屎。恐他發呆屙在床上。忙大聲叫道。媽媽。不好了。快些來。那養氏還不曾睡。正等着聽聽風聲。忽聽得牛氏叫他說不好了。又不知有甚扠(差)事。慌忙跑了過來。見馬台還睡在他肚子上。不肯下來。問其所以。牛氏把屙屎的話向他說了。那養氏笑得打跌。道。你悟錯了。他是個咬舌。說話不明白。他想是弄洩了。大約是快活得很。從沒有經過。他說是我死囉。我死囉。你叫他去屙屎。他急了。所以說不是囉。我快活死囉。那裡是要在床上屙快活屎。罷了。夠了。這是你的造化。他竟通人性了。那牛氏纔懂得是這緣故。也不由得大笑。那養氏笑着同馬台戲道。下來。我帶你去睡罷。看他掐你的雞雞。馬台道。我要他帶我睡。不要你帶囉。我的雞雞。他那沒有鬍子沒有牙的那個裡頭裝着呢。不怕他掐囉。說着。又見他動動抽抽的起來。養氏方放心去睡了\footnote{世間那得有此等好乳母。}。馬台竟足足弄了一夜。他何嘗有通宵的本事。這呆人乍嘗得這件美味。他總不肯下肚子來。洩了伏下來睡一會。有些硬了。牛氏叫他動。他就動個不歇。叫他住。他就住了不動。所以就弄了一夜。牛氏生得嬌怯。雖馱着他覺得吃力。但因有利於己。也只得勉強承受了\footnote{牛氏竟是看過國語的。秦宣后曰。昔吾侍先王寢。先王以一足置吾腹。吾不能受。因無利於我也。先王或以全體置吾腹。吾竟能容之。因有利於我也。牛氏大約亦此意。}。到了天明。他還不肯下來。牛氏推他。他便摟得緊緊的。死命壓住。牛氏被他壓得氣都出不得來。急了。又叫養氏。養氏也正起來了。忙走過來。牛氏道。他不肯起去。死命的壓着我。氣都要壓背了。怎麼處。養氏道。這容易。在(待)我哄他。原來這呆子酷好吃糖食。養氏是哄慣了他的。走到床前。說道。起來。我給糖吃。馬台聽得他說給糖吃。忙探起身子來。被養氏趁勢一把拉下肚子。道。我替你穿了衣服。拿糖你吃。替他穿完了。果然拿了些糖食與他吃纔罷。牛氏方纔得身起來。從此以後。他一刻也不肯離牛氏。連牛氏到床後去上淨桶。他也跟了去。蹲在傍邊。間或日間一時高興。也不管丫頭僕婦在面前。就拉着牛氏要弄。牛氏一來強不過他。二來也不是甚麼苦事。叫人出去帶上門。也就憑他弄上一場。一日。飯後無事。牛氏叫丫頭拿過一個枕頭來。側身歪在春凳上。馬台見他的嘴直豎。以爲是陰戶。看上興來。扯開褲子。陽物硬邦邦的。上前抱住香姑的頭。便往嘴裡塞。丫頭們看見。都笑着跑到門外張他。香姑忍不住好笑。忙把嘴摀住。他還在臉上混搗。香姑一把攥住他的陽物。說道。這不是的。把褲子扯下。拉他的手摸着陰戶。道。這纔是呢。他看了看。方放了頭。上身弄了一齣。後來慣了。這婢婦們但見主公去拉主母褲子。就帶了門出去。每一傍晚。他就拉着牛氏上床。定要在肚子上過夜。動上悉憑香姑調度。好生像意。他疼這個呆子。眞像至寶一般。心中想道。要是嫁了個伶俐丈夫。未必這樣由得自己指揮。反埋怨養娘。若早敎會了他這種絕技。當日何必去尋那老和尚。話休繁敍。他夫妻成親之後。過了七個來月。牛氏竟生了一個兒子。他娘家送厚禮。送衣服被褥。搖籃熏籠。各色粥米。是不必說。他家中一面差人到公婆任上去報喜。一面叫媒人雇兩個奶娘。叫畫匠畫蛋。婦女們染紅綠果子。三朝送親友。一家鬧鬧吵吵。只可憐這呆子。守着牛氏傍邊。坐着呆看。不但不知這兒子是那裡來的。且並不知兒子是個甚麼東西。人給他果子蛋吃。他便接着。不給他。他也並不要。香姑所生的這娃娃。惟他自己同養氏心下明白。也不是兒馬的駒。也不是乳牛的犢。是那禿驢傳下的一個小驢種。當時有四句打油嘲他。道。

\begin{quotation}

這是誰人產下兒。如何弄得馬家支。

或因分得呆人氣。但問娘親便得知\footnote{馬氏現報。天斬其嗣。}。

\end{quotation}

香姑分娩的第三日。苟氏計氏來看外孫洗三。香姑恐馬台呆頭呆腦。一時混拉着要弄起來。豈不是笑語。叫養氏拿糖哄了他出去。自香姑生產的那一夜。他就拉着要弄。如何行得。香姑向養氏說。費了許多力。纔哄了外邊去睡。一天亮就要進來。日裡要弄。便拿些糖哄他。過了有二十來日。死也哄他不住。夜間不肯出去。定要同香姑睡。養氏也沒奈何。只得任他上床。之後定要同香姑弄。香姑身上也潔淨了。也便由他高興。且說那家人到鳳陽報喜。馬士英蹇氏歡喜非常。以爲得了長孫。在衙門中還慶賀了幾日。帶了許多金銀紬緞與媳婦做衣綳等物。並屬下人員送的麒麟項圈手鐲鈴鐺之類有許多。重賞這報喜家人而回。後來雖有人知道這娃娃來路不明。只好背地笑罵。誰敢當面說他。這孩子到了七歲。延師替他起個學名。這先生也知他底裡。便道。昔日唐時四才人中。有一個盧照鄰更爲傑出。此子異日當加乎其上。因此就叫馬加盧。馬士英甚喜。孰不知他暗藏深意。馬傍加個盧字。謂係禿驢之種也。當年晉朝以牛易馬。今日他家又以驢易馬。何馬家之不幸也若此。這正是。

\begin{quotation}

張公吃酒李公癲。盛六生兒鄭九憐。

駑馬獲駒驢下種。奸臣斬嗣報由天。

\end{quotation}

這不在話下。且說宦蕚與賈童鄔三人。自從錢貴家散後。過了數日。又相聚在一處。宦蕚對賈文物道。錢貴那妮子果然竟有些才學。行得好狠令。若不是我們肚子裡有些貨\footnote{肚子裡惟有耕田種圃的貨。}。幾乎被他難倒了。鄔合道。他先還有些自恃。虧後來賈老爺的促才。大老爺的奇書。纔壓服了他呢。童自大道。他們只說他文才好。我却只愛他標致。我每常看見我家奶奶的相貌福態。心裡不由得害怕。昨日見了他那嬌模嬌樣。魂都沒了。若不是想甚麼遭瘟的詩。雖然不好摸他下身。捏一捏他的奶頭。聞一聞他身上的香氣也是好的。白白的可惜了。下回再去看。就是二位哥惱些也罷。我是定要摸摸的。也不枉我捱了我家奶奶那一掌。說得衆人都笑了。宦蕚道。我看他手中拿的那把金扇。寫着好些字樣。是有好幾首詩。必定詩好他纔拿着。後來忙忙叫那丫頭替他收了進去。想是那個情人送他的。纔這樣寶貝也似的。賈文物道。詩三百。一言以蔽之。贊美人之作。一闋足矣。何必屢屢。定非才人而可知之矣。鄔合道。賈老爺說的是。量他曾見過些甚麼好詩。老爺也照韻和他一首。壓他一壓。賈文物忙說道。我君子人歟。況詩文雅道。豈肯屑贈瞎妓乎。確乎其不可贈者。錢貴也。宦蕚道。雖說是不屑與他。但作詩爭名。也是才人的妙事。賢弟快作起來。童自大道。大哥是公子大官府。我是個財主小老爺。不作詩還罷了。二哥你是個進士才子。又是個半大不小的老爺\footnote{眞是奇稱。}。若不作詩嚇嚇他\footnote{作詩可以嚇得人。奇語。}。他還要笑你呢。賈文物又推阻道。昨日因坐而飮。心不在焉。不知是何韻脚也。鄔合道。衆位老爺行令時。晚生備細看了一遍。是一韻五首。雖記不得。詩韻倒還記得。宦蕚道。你快說出來。鄔合取過一枝筆。案上尋出一張紙來\footnote{宦蕚書案上竟尋出一張紙來。奇甚。不知要了做何用。}。將多羅歌波四韻寫出。遞與賈文物。賈文物見了。無辭可推。只得又道。俟少頃飮而高興之時承命可也。鄔合道。原該如此。當日李太白斗酒詩百篇。也要吃了酒纔作得詩出來呢。宦蕚因叫家人看酒。不一時。擺列上來。大家同飮。只有賈文物許了作詩。雖推說酒後。恐一時作不出來不好意思。因此愁眉苦臉。食不下嚥。半會又想道。他們三個肚裡也都有限。我不過謅得八句就罷了。想到此處。方把愁心放下一半。又想道。罷是罷了。只恐與了錢貴。人看見是我作的。豈不貽笑於人。把聲名都壞了。正拿着酒杯出神。宦蕚問道。賢弟今日有甚麼心事。這樣悶悶不樂。連吃酒都沒興頭。他詭對道。適長兄命弟作詩以贈錢貴。因係長兄之命。却之却之爲不恭。故弗敢却也。若贈與他。又恐聖人之徒無贈瞽妓之詩者。倘人知之。此汚辱之名。雖孝子慈孫。百世不能洗也。長兄或家中看之可矣。萬不可出之大門之外。宦蕚道。賢弟旣這樣說。你只管作起來。只說是個名公作的。不落你的款何妨。因叫家人另擡過一張桌子來。取了一副筆硯擺下。賈文物沒奈何。想了半日。纔拿起筆來寫。寫了又改。改了又寫。換了十數張紙。方纔騰(謄)淸。鄔合道。賈老爺這樣用心。必是精工得很了。錢貴何幸而得此。那賈文物寫完了。遞與宦蕚。道。請敎。宦蕚道。我自幼在經文上用功狠了。於詩詞一道。不曾十分留心講究。恐怕念得不鏗鏘。倒把你的詩都念壞了。還是賢弟自己念我們聽罷。賈文物因哼哼喞喞的念道。

\begin{quotation}

面似錢姑少。睛同瞽妓多。

\end{quotation}

宦蕚道。好詩。兩句話只十個字。包含着一個標致老婆。就把他說盡了。鄔合道。他那幾首詩也沒有從頭對起的。老爺竟似排律呢。他又念道。

\begin{quotation}

早穿京裡絹。午換浙中羅。

\end{quotation}

鄔合道。這兩個地名對得好是不消說了。說他早起穿屯絹。午間換杭羅。正是如今初秋的天氣。應景之極。宦蕚道。你肚子裡也竟通呢。二弟這樣好詩。虧你也就解說得出。賈文物道。愚弟若非公車北上過。尚不能想起京裡絹三個新奇字眼。又念道。

\begin{quotation}

唱曲聲如泣。

\end{quotation}

因說道。哥哥賢弟不知。這句詩乃古文也。弟敏而好學。信而好古之所記蘇文中。有如泣如訴之語。我特引而贊之者也。他又念道。

\begin{quotation}

交歡哼似歌。

\end{quotation}

鄔合拍着桌子贊道。好摹擬。眞正入神。賈文物又念。

\begin{quotation}

一番雲雨後。淫液漾淸波。

\end{quotation}

宦蕚道。好詩。把他的行徑都說絕了。只怕錢貴聽了此詩。還要拜賢弟做詩師呢。鄔合道。晚生聽了賈老爺的佳作。竟游夏不能贊一辭。老爺結尾這一句五個字。都用水傍。從來罕見。眞是千秋絕唱。童自大道。二哥。我聽得人說。詩從放屁來。方纔也沒有聽見你放屁。怎麼詩就出來了。這想是才學高的緣故。賈文物見衆人贊他。喜得心窩裡都亂癢。嘻嘻的笑癱在椅子上。道。我非生而知之者。好古敏以求之者也。宦蕚叫人到上房取了一把磨骨白扇來與賈文物寫。鄔合道。不要說賈老爺的詩高似他萬倍。只大老爺這把扇子。就比他的想頭好多了。那金扇俗極。這白面何等雅致。那賈文物在扇上寫完。恐怕詩稿留下被別人看見。遂將來收入袖中。把扇子遞與宦蕚。宦蕚道。賢弟再念起來我們聽聽。每人吃三杯賀賀詩。童自大道。我連一個字也不懂。吃他做甚麼。鄔合道。賈老爺這樣才人的高作。也是輕易難得聽的。老爺也要請用三杯。賈文物聽的誇得。興頭越發哼出腔來。又念了一遍。衆人飮了三杯。宦蕚又叫斟了三杯與賈文物。道。這是掛紅的酒。也要吃的。賈文物燥脾之甚。也就一氣連乾。宦蕚道。扇子是有了。只是錢貴有病。去也沒趣。遲些日子他是(果)然好了。我們再去訪他不遲。叫個小廝把扇子收好了。宦蕚又道。我看如今的人。肚子裡一竅不通。拿着古人的詩看還不懂得。動不動也要作詩結詩社。我們難道肚子裡的才學就不如他們不成。我們四個人在這裡。何不各道本色。也學聯他一首。試試大家的學問。賈文物道。妙哉。不學詩無以言。理當而學詩。哥就請起句。宦蕚道。却要合自己。不合罰一大碗。我就先說。

\begin{quotation}

父做高官子享福\footnote{樂甚。}。

\end{quotation}

鄔合道。詩要有眞味。大老爺的佳作眞妙絕千古了。宦蕚大喜。笑道。二弟快聯。賈文物道。

\begin{quotation}

身爲進士妻嚴肅。

\end{quotation}

鄔合道。賈老爺對得却巧得好。眞是名公才子。賈文物道。三弟來。童自大道。叫鄔哥且續着。讓我想想。鄔合道。晚生怎敢僭老爺。宦蕚道。作詩何妨。你肚子裡要有。只管就說。鄔合道。旣然如此。晚生便斗膽了。我也實道其事。

\begin{quotation}

一生只善做幫閒。

\end{quotation}

宦蕚笑道。不錯不錯。接得好。賈文物道。此可謂辭達而已矣。鄔合道。晚生是狗尾續貂。怎敢當二位老爺大贊。只見童自大大笑道。我也有詩

了。便道。

\begin{quotation}

我見了奶奶就要哭。

\end{quotation}

宦蕚笑道。三弟的多了一個字了。你好好的哭些甚麼。童自大道。我不像二位哥哥假裝好漢。我是老實人。有話就實說。我怕得很。怎麼不哭。多一個字那罷了。雖不成詩。押韻而已。宦蕚向賈文物道。賢弟寫出來。改日等他們詩社刻詩。我費幾席酒。請請他們那些假名公。\endnotemark[4]把我們四個人的名字刻上。也好四海馳名。鄔合道。三位老爺的是詩。要刻只刻這三句。晚生的那一句是屁。入不上的。宦蕚道。甚麼相干。你看近日的假詩伯。雖作的是詩。不過都是放屁而已。賈文物道。屁也者詩也。詩也者屁也。二而一。一而二者也。童自大笑道。我的這一句比你們的略高些。宦蕚笑道。想是會哭的緣故麼。童自大道。這七個字的屁。人放得多的很。成了宿屁了。我的雖是屁。這八個字還是個新鮮屁。豈不高些。衆人大笑了一會。宦蕚道。我前日到個親戚家去。見一起假斯文在那裡作詩。題目是甚麼朝日。我不懂得。問他們日頭怎個朝法。他們說朝字音招。朝者。早也。是早起纔出的日頭。我們何不也大家作一道頑頑。就從我先起。他想了一會。喜笑道。我我的詩竟有了。朗吟道。

\begin{quotation}

日頭出來紅〖氵韲〗〖氵韲〗\footnote{着實難聽。}。好似胭脂染簸箕\footnote{胡說。}。

東邊一日出一個\footnote{有理。}。西邊不知幾大堆\footnote{好悟性。}。

\end{quotation}

鄔合道。大老爺眞奇才異想。大約自古來的詩翁。也未必能及了。宦蕚笑道。實在這幾句也難爲我想。二弟也來一首。賈文物道。古云。一之爲甚。豈可再乎。弟先已有過一詩。可以不必再矣。三弟請。童自大道。我只得兩句。可行得麼。鄔合道。古人滿城風雨近重陽。只得一句。老爺何況有了兩句。童自大笑道。列位請聽。我的詩來了。

\begin{quotation}

今日早起天未亮。我便起來浪了浪。

\end{quotation}

宦蕚笑道。這兩句話是怎麼說。童自大笑道。我解給哥聽。天未亮。可不是朝。浪了浪。難道還不是日。又笑了一回。又飮了幾杯。方纔散去。宦蕚欣欣得意。纔要回上房。多嗣說道。剛纔上去取扇子。奶奶問要了給誰。小的回說不知道。不知誰多嘴。說是送錢貴。奶奶盤問了好一回。小的強說不知道。老爺須留神答應。宦蕚聽了。失驚道。造化造化。倒是沒有說出錢貴是瞎姑呢。要一時失口。如何了得。走進房來。侯氏問道。你方纔要扇子做甚事。宦蕚道。老鄔要把扇子送人拜壽。來求我。故要了與他。侯氏道。我聽見你們在前邊吃酒。叫那姓賈的作甚麼詩。寫扇子送甚麼錢貴。你若瞞着我做甚不肖的事。我打聽着了。你却休怪。宦蕚發急道。我幾時敢瞞你做了甚事。就是老鄔要送姓錢的。說白扇不好送人祝壽。煩老賈寫了一首詩。何嘗有別的緣故。況承你的好情。又與了我丫頭。家裡的生活還做不完。還想外邊些甚麼。侯氏聽了。信以爲實。方不做聲。宦蕚暗暗歡喜。二人上床。又幹他的正經事去了。再說那賈文物到了家中。進入房來。富氏還不曾睡。賈文物摘巾寬腹。不想寃家路窄。在袖中抖出那張詩稿來。賈文物就要去搶。已被丫頭拾起。富氏遂叫。拿來我看。丫頭忙忙遞上。富氏接過。原來富氏幼時也讀過幾句書。略識得幾個字。賈文物見他常時看說唱本兒。此時若賈文物不動聲色。任他怎麼斑駁。還好支吾得過。不想他賊人膽虛。恐怕他看出是贈瞎姑的。一見富氏接在手中。他急得搓手頓足。道。自作孽不可活。此之謂也。噫。天之將喪斯文也。吾死矣夫。吾死矣夫。富氏見他着急。疑心頓起。見上面寫着錢姑妓多等字。雖不甚懂。覺得有些古怪。遂搥胸大怒道。你寫這情詩是送那個養漢的娼根做表記的。實實說來。免我拷打。那賈文物魂都嚇走了。膽也驚碎了。癡呆呆不敢做聲。富氏越想越怒。問之再三。他只兩目直視。並無一語。富氏怒甚。罵道。你若不做虧心事。問着爲甚麼不答應。賈文物半會掙了一句出來。道。亡之命矣夫。予何言哉。富氏道。我也沒力氣問你甚麼言哉。我也不懂得。明日拿去問人了。看是做甚麼的。再與你算帳。你且過來跪下。賈文物雙膝跪倒。富氏將他頭髮打開。挽了一個扁髻。叫丫頭將燈臺取來。放在他頭上頂着。吩咐道。你旣爲風流快活。也請你來受些苦惱。好好頂着。若潑了油。熄了燈。你休想要活命。那賈文物面如死灰。直蹶蹶跪着。總無一言。富氏吩咐了一番。自去上床而臥。賈文物整整跪了一夜。渾身骨碎筋酥。雙膝腫大如碗。動也不敢動一動。又不敢哼卿。恐驚醒了床上天尊。又一場大禍。眼淚汪汪。齜牙咧嘴。直到天明。每常那些文縐縐的腔調。一絲皆無。日色東升。富氏起來梳洗。賈文物哀吿道。王赫斯怒。沒齒而無怨言。予豈好辯哉。但屈而不伸。寃哉苦也。富氏見他那樣子狼狽不堪。叫丫頭將燈臺拿下。仍叫跪着。道。我將那詩煩人看了來再講。遂叫僕婦拿了詩稿到外邊。叫個家人送與干不嬌(驕)。看是做甚麼的詩。此時賈文物心中如十五個弔桶打水。七上八下的。還不知來說些甚話。撲撲的亂跳。未定吉凶。不多時。來回話道。干大爺見了這字。大笑了一陣。他說〔從〕來那裡有這樣不通的詩。大約是鼓兒詞上的胡話。不知是做甚麼用。或者是抄了當笑話看的\footnote{把賈進士尊作一貶至此。}。富氏聽了。反過意不去。白白難爲了他一夜。遂問賈文物道。這個果然是鼓兒詞上的麼。不許欺瞞我。賈文物連聲道。夫人不言。言必有中。吾誰欺。欺天乎。富氏道。旣然如此。你昨日晚上爲甚麼不分辯。旣不是送人的情詩。饒你起去罷。那賈文物半晌方纔爬得起來。自己撫摩着膝蓋。道。有痛乎。非夫人之跪痛而誰爲。柔(揉)了一會。然後一瘸一跛走到前邊書房中來。暗喜道。虧殺干兄這救命天尊。不然如何了得。又暗想暗笑道。我聽得人說。從古來作詩的數李杜了。他二人一生坎坷。皆不得其死。我一生纔學作了一首詩。便受了這一場苦難。若再要作他。眞要像西遊記上的唐三藏。九九八十一難都要受了。從此永斷要緊。暗笑了一回。因一夜無眠。精神困怠。又渾身疼痛。吃了兩杯熱酒活活血脈。倒臥榻上。叫了個待詔來。遍身按摩睡了。不題。且說宦蕚當年與游混公師生數載。游混公不但不曾打他一下。罵他一句。連大氣也不敢呵他一口。美其名曰先生。實在只算得個雄乳婆老篾片而已。宦蕚過後甚是感念他。雖不能時常親厚。也還間或來往。因此與游夏流也有數面之識。前次游混公死了。宦蕚到他家去弔紙。見一個騷眉騷眼的少年。頗撩人愛。出來問起。家人有知道的。說他叫做楊爲英。是個卯字號的朋友。宦蕚大有垂涎之意。想叫他到家中來吃酒頑笑。恐母大蟲一時知道了。惹下這天字號第一的奇禍。如何解釋。心中雖然常常想着。因家中雜事繁冗。也就擱過一邊。前因見了錢貴。動了虛火。雖來家同侯氏大戰過幾場。又得了一個美婢。也就該知足知止了\footnote{古人雖有知足不辱。知止不殆二語。然而能行之者爲誰。又何況於宦蕚。}。俗語有兩句道。

\begin{quotation}

野花偏有艷。村酒醉人多。

\end{quotation}

又說妻不如妾。妾不如婢。婢不如妓。妓不如偷。偷得着不如偷不着。他雖有了一妻一婢。常把錢貴兩個字放在心頭\footnote{寫此數語者。點出前生宿願耳。}。因他有病。要尋個別人且取樂一番。妓女中又無一個可取。忽想到這楊爲英身上。道。這個大耳朶貓。生得頗有動人之處。況我從未嘗着這大腸頭的滋味。何不破一破戒。奈無處可爲行樂之場。又想道。何不我出了東道。竟到游世兄家去。那小官又同他相熟。可以一呼即至。我這一團高興。量他也不好推却。叫人去請了游夏流來。宦蕚將前意說了。他怎好說我家老婆利害。不敢奉命。一來出不得口。二來巴不得要奉承這樣有財勢的大老。倘親厚了。後來那裡沒有個緩急相求處。遂滿口應允。宦蕚喜甚。叫家人稱了四兩銀子來遞與他。約定了次日到他家去。這游夏流別了出來。尋着楊爲英。對他說知這宦公子羨慕他的話。他做小官人。見有這樣貴公子要來賞鑒他。他猶恐賤股有汚尊卵。自然惟命是從。可還有個不願的。欣欣然有自得之色。游夏流到家。對多銀謊說道。有一位宦公子。當日是我父親的學生。前次來弔過紙。我娶你時。他也有分金賀過。別了許多時。他明日同一個姓楊的。也是個財主的兒子。要到我家來坐坐。體貼我。怕我沒錢。與了我四兩銀子來預備些酒菜。不知你依不依。我沒敢允他。特來請你的示下。又把銀子拿與他看。卜氏聽了。這一買東西有一個肥嘴抹抹。且又看看這公子同財主是個甚麼模樣。便說道。人這樣好意。拿銀子送上門來。還有不做的麼。游夏流如得了溫旨一般。好生樂意。次日淸早。買了許多東西回來。知道宦家跟隨的人多。擡了兩大罎好酒。先選上分送了些與卜氏吃了。外邊將午。楊爲英先來。宦蕚隨後也到。三人坐下。不多時。游夏流掇上肴來。他家房屋窄狹。不好叫廚子備酒席。只買些現成熟物。自己整理。無非是燒鵝板鴨。薰蹄熟雞之類。並各樣菓子。堆了一桌。然後送上酒。吃將起來。那卜氏有了幾分酒意。久違了陽物。有些火上來了。不住到窗下來張看。見宦公子肥頭大臉。一身華服。七八個管家侍立服事。那一個雖穿得稍次。却少年淸俊可喜。他竟把兩個都看上了。一個愛他壯健。一個愛他淸秀。想了一想。二者不可得兼。只取他陽道偉岸的就罷了。想定了主意。須如此而行。忙到房中。濃濃的把那麻臉上。厚厚的膩了一層粉。黑臉上襯着鉛粉。顯出個萃靑的面孔。把一張大嘴拿胭脂擦得像婦人行經的血屄一般。蘸些象鼻草泡的黏水。把金絲般黃髮刷得光亮亮的。到後院內摘了幾朶大蜀葵花戴在頭上。儼然一個鬼怪。與鄔合當日裝扮的那龍家小子。正好做一對怪夫妻。他自己走到鏡臺前照了照。把頭扭了兩扭。自喜自愛。道。我今日這番打扮。雖眞人見了也要破戒\footnote{眞人見了未必破色戒。恐疑是鬼魅。以法劍斬之。破了殺戒耳。}。羅漢見了也要還俗了。何況這凡夫俗子。又換了一件大紅灑線纏枝金梗白梅衫穿上。下邊繫了一條豆綠繡串枝蓮的縐紬裙。碗大高底一雙大紅花鞋。不住的窗下來往。他家的房子是一龍兩間。前一間抽一條走道隔做兩截。前半臨街做個客位。後半截做個退步坐位。開個北窗。因緊對臥室。是不開的。此時他們就在這退坐內吃酒。家人們在前邊坐歇燙酒。一個天井後面又是一間。也抽一個走道。也隔做兩截。半截做臥室。後半做廚房\footnote{他家房子此時方詳寫。與游混公弄兒子時對看。一絲不錯。前半臨街客位。游混公弄兒子處也。他們飮酒退位。楊爲英潛身處也。後面臥房。游混公睡處也。此處方補明。}。也有個北窗。後面又是個小院。他們要小解。不好往街上去。就往後院子裡來。先是楊爲英到後面去小解。卜氏忙到廚房北窗內張看。見他的那件東西比游夏流也大得有限。遂不在意。單留心要看宦蕚。少刻。見宦蕚後邊去。他忙到窗內一張。因同那小官頑了這一會。有些高興。那厥物脹得硬幫幫的。比那兩個搖鼓兒的還壯觀些。不由得渾身打了一個噤。從頭頂心上一點麻起。直酥到脚底板上。那陰門一張一閉。淫水一滮滮往外直冒。一條褲子濕得不用說。把纔穿的花膝褲。船樣大的紅鞋。那水順着腿淌下去。都弄濕了。酥得他半晌方挪移得動。那一腔慾火由不得勃騰騰攻將上來。又喝了幾碗冷酒。不住來張。此時他們都有酒了。只見宦公子把那姓楊的抱着。坐在懷中說頑說笑。一遞一口吃酒。他心下就明白了幾分。因看上了宦公子那個巨物。把怒氣勉強按住。正張着。只見宦蕚起身。他知是又要溺尿。此時慾火如焚。顧不得了。閃在廚房內。宦蕚正扯開褲子。剛溺完了。猛然見一個人跑來。一把攥着陽物。一隻手摟過脖子。就親一個嘴。宦蕚嚇了一跳。擡頭一看。見是這樣個怪物。黑影裡顯着個死白的臉。不知是人是鬼。嚇得掙脫了。拽着褲子往外飛跑。有一個黃鶯兒說這多銀道。

\begin{quotation}

張得俏乖乖。滿胸中好事諧。擬嘗此物粗而大。心驚意呆。情闌興衰。敎人空把伊相愛。好羞哉。一腔忿怒。四馬出將來。

\end{quotation}

那卜氏一團騷興。被他這一掃。由不得怒上心來。跑到窗外。拍着窗臺子。大罵道。好大膽。要死的忘八。你哄我請甚麼公子財主的。原來弄幾個兔羔子小廝在這裡頑。我家淸門淨戶。是開巢窩的麼。趁早夾着㞠子與我走。走遲了。我拿馬刷來。把你們兔羔子打個晦氣。叫你這臭忘八沒處死。那游夏流嚇得一交癱在地下滿地扭。宦蕚可是容得人罵的。奈是女流家。不好動粗。站起把桌子一掀。碗盞器皿打得粉碎。大怒而去。上馬回家。那楊爲英見勢頭凶惡。嚇得屁響連聲\footnote{未必然。小官多放的是鬆屁。如何得響。}。如兔子見了黃鷹一般。蹺着尾巴\footnote{此更不然。他的尾巴是旋安旋去者。如何得常在股後。}。如飛的跑去了。正纔出門。被卜氏拿了一瓢水趕到門口。潑了一身。罵道。退送這些瘟鬼。關了門\footnote{寫卜氏趕楊爲英潑水幾句。特爲關了門三字而用也。不然。卜氏一腔怒氣。何暇尚有閒心出來關門。游夏流頃刻就去受罪。安得還出來關門。若不一提。豈不漏空。作者婉轉想出卜氏一趕。又不露跡。良匠苦心。看者須知。}。惡狠狠走進來。見那游夏流還在地下爬。掙不起來。卜氏上前。一把擰着耳朶。似牽羊也似的拖到屋裡。叫他跪下。此時卜氏也有八九分酒意。左思右想。這一口氣不得出。一腔火不得洩。叫他脫光了。自己也脫得上下沒一絲。撅了兩節蘆柴來。將他那小陽物夾起。兩頭用膝褲帶紮緊了。疼得那游夏流叫苦連天。眼淚直流。跪着哀求。卜氏罵道。你這忘八。生了這樣不中用一個東西。家裡的事一點做不得。還同人弄兔子來頑。反哄我甚麼公子財主的。樂得你夠了。且受受罪着。夾了多時。游夏流疼得只是叩響頭。要死要活。他處治了一番。怒氣稍息。大氣未消去絲毫。放了他。叫他上床來舔。那游夏流可敢違拗。一隻手搓採着陽物。愁眉苦臉。眼淚汪汪。只得爬上來舔。舔了多時。不得像意。他久不見這東西。倒還挨了過去。今日不但見了。而且是見所未見絕大的異物。那裡還過得。那心裡由不得火燎般發燥。游夏流一面舔着。他一面長吁短氣。舔夠多時。他又恨起。道。人家生個東西都還像個樣兒。怎你這忘八就生出恁個現世的物件來。氣恨塡胸。一翻身坐起。意思還要加他些刑罰。忽然見他的陽物被夾腫了。竟粗了好些。心中一喜。叫他來弄。游夏流正疼得要死。不敢不依。忍着疼。齜着牙弄了進去。因腫疼得麻木了。倒輕易不得洩出。又被夾得火燒火辣的滾熱。雖不能暢他的淫心。也覺得比每常粗大長久了些。弄了一會。卜氏因酒多了。又微微有些樂處。竟呼呼睡去。游夏流疼得忍不得了。悄悄拔將出來。下床燈下一看。原來皮夾破了。又被淫水一螫。所以疼痛難忍。一夜睡不着。到天明起來一看。竟腫得像個魚泡兒一般。忙拽上褲子。恐卜氏見了。又叫去行樂。如何還禁得。忙走到前屋內。那地下的雞鵝鴨肉之類。已被他家那條大黑狗享用得乾乾淨淨\footnote{此處伏下狗。}。將昨日的破碎傢伙收拾了。煮了飯。還有剩的餘肴。等卜氏起來。打發他吃完。自己收到廚下。也吃了些。到街上尋㞠子外科醫㞠子去了\footnote{這種醫生不知招牌如何寫。}。且說昨晚宦蕚正頑得高興。被卜氏一罵。大怒歸家。到上房來。侯氏尚還未睡。見他一臉怒色。問道。你今日到誰家去來。爲甚麼氣得這個樣子。他沒得答應。謊說道。游世兄今日來請我吃酒。他那不知事的老婆在裡邊大罵起來。我一時怒起。把桌子掀了。一直來家。所以氣還未消。侯氏道。這事據我想來。必定是這個男人素常在他跟前懶惰。又壞得很。得罪了這女人。他要是慇勤小心。那老婆就不替他留些臉面麼。宦蕚知侯氏是打草驚蛇的話。無言上答。二人脫衣上床睡覺。宦蕚睡着。想道。那楊家小子倒是有趣。騷模騷樣。好生動火。我摸了摸他屁股。細皮嫩肉。軟濃濃的。他那屁眼也還緊緊就就的可愛。要不是這潑婦一鬧。此時豈不正在賞鑒妙臀。又悔道。原是我錯。他跑來攥我的此物。無非愛上我的一番美情。管他醜俊。何不弄他一下。此時豈不是一箭雙雕。原是我太認眞了些。羞掃了他。怪不得他罵。又想道。也怪不得我。你慢慢的出來講就好了。冒冒失失跑來捏住。嚇我一跳。自然顧不得要跑。我想他必定是個騷淫極了的婦人。要同他弄弄。自然另有一種妙處。錯過了這機會。可惜可惜\footnote{此非寫宦蕚想必有其事。然寫其有此者。今日未改過之時。此等醜婦尚悔其不淫。彼異日改過之後。遇屈氏並賣酒妻之美。竟能堅忍之而不動心者以爲異。}。想到此處。不由得遍身火發。陽物堅舉。伸手去摸侯氏。見他仰睡着。一摸摸到陰戶。想道。放着食在嘴跟前不吃。胡想些甚麼。何不把他這個穴道。又當那婦人的前門。又當那小子的後戶。弄他一下。自然又興頭些。就爬上身來。弄將進去。侯氏矇矓將睡着。醒了。道。我剛纔睡着。你又驚醒了我。宦蕚笑道。你旣要睡。我下去罷。就要往外拔。侯氏忙用手扳住他屁股。道。我旣醒了。你又下去做甚麼。宦蕚將他兩腿擱在肩上。心中想着那兩人。分外興豪勇猛。竭力一場野戰。把個侯氏弄得四肢俱軟。癱於枕蓆之上。宦蕚又橫衝直闖了一會。方洩了下來。侯氏半晌方纔甦醒。笑問道。你這天殺的。有這樣好本事。每常怎麼不使出來。你今日爲何有這樣高興。你吿訴我。宦蕚沒得說。笑道。我方纔偶然想起一件事來。那年我纔娶你的時候。還是你家的嬌客。你爹就把我數說了一場。我氣到如今。方纔一時觸動。故此拿你出氣。那侯氏信以爲眞。道。哦。原來是爲這個。睡了有一盞茶時。想適閒的樂境其然快活。對宦蕚道。我還記得一件呢。我媽那一回不也得罪過你麼。你怎麼就不氣一氣。宦蕚聽了。知他還要弄弄的意思。自己也還有些餘興。復笑着爬起。道。是呀。我幾乎忘了。沒得說。也拿你出氣。又弄了一回。兩度之後。宦蕚也有些乏了。就想睡。將要睡着。那侯氏興還未足。又推他道。我又想起一件來。那年在京裡。我大哥哥也得罪過你。我到如今時常想起來。還替你氣哩。你倒不氣。宦蕚興已足了。有些怕動。推辭道。我記不得了。侯氏道。哎呀。幾年的事就記不得。是爲甚麼甚麼的呢。宦蕚也不答應。他見宦蕚不動手。便道。一個男子漢大丈夫。受了人的氣就罷了不成。男兒無性。寸鐵無鋼。做漢子的人一點氣性也沒有。可還成個人。儘着嘓嘓噥噥個不住。宦蕚被他在耳旁絮聒。也睡不着。知他還想弄。便道。不用多說了。也是拿你出氣。勉強掙着。又弄了一陣下來。實在動不得了。剛要睡。這侯氏又推他道。我大嫂子還得罪過你呢。難道就罷了。宦蕚心中暗笑。口中說道。哎喲。我的氣星兒也沒有了。況你嫂子一婦道家的。我也不同他一般見識。你饒我睡睡罷。侯氏知他不肯動了。也纔安心去睡。那嬌花在床後聽得他二人兩三番大幹。急得火氣上騰。喉中不住發癢。忍不得儘着咳〖口敕〗。宦蕚知他是想分惠之意。笑道。我連你舅奶奶的氣都沒本事出了。何況你咳〖口敕〗。勸你安心睡罷。你便咳楂了喉嚨也沒用了。笑了一陣睡了。不題。再說那游夏流到了一個外科鋪中買了些止痛消腫的末藥。將陽物擦了。回來在他一個表兄甘壽門前過。他想道。自從娶了這羅刹進門。終日在家當奴才。時刻也不敢離。久不曾來看哥嫂了。今日逕(經)過。何不進去看看。一直走到天井內。見甘壽蹶着一嘴白鬍子。眼淚汪汪。頭上頂着一塊大搥衣靑石。兩手扶住壁。立直跪在那裡。游夏流走近前來。低問道。想又是哥得罪嫂子了。甘壽嘆了口氣。道。我吃了雷也不敢得罪他。無事尋事了。好端端的拿我這樣凌辱。游夏流道。我見嫂子。替哥求個情去。甘壽急道。你不要替我添禍。等他性子癱些。自然饒我。那熊氏在房內聽見說話。叫他的老女兒道。老姐。你看那老奴才同誰說話呢。敢是噥噥喞喞的咒我麼。那女兒出來一看。道。是游大叔叔來了。同爹說話呢。那熊氏喜道。我正想要請他來呢。來得好。快請進來。游夏流聽得。忙走入房中。作了揖坐下。道。嫂子這些日子好麼。前次在我家簡慢嫂子回來。熊氏道。你沒得說。一個至親骨肉家。你費那些事。我已酒醉飯飽。還要吃甚麼。那一日。我呷了沒有二三十斤酒麼。你嬸子的酒量倒也替我差不多。游夏流道。我哥怎麼又沖撞了嫂子。叫嫂子生氣動怒。熊氏道。你哥那老奴才。但膽子正氣多着呢。我提起來就牙癢。恨不得嘴他的肉。我前日會見你家嬸子。說燒茶煮飯鋪床掃地全是你。連馬桶都是你倒。好不小心勤謹。說他還要打打罵罵的。我就說了他幾句。一個人不要折福。一個丈夫慇懃膽小到這樣個地位也就罷了。還要怎麼樣的呢。游夏流一肚子的苦正沒處吿訴。便接口道。我昨日受了一場大寃屈。不好吿訴嫂子的。你是有年紀的老嫂子。同母親一樣。就說也不妨。昨日晚上又不曾爲甚麼。拿蘆柴做了個小夾棍。幾乎把我下身夾做兩段。皮都腫塌了。我纔偷空出來尋醫生。上了些藥。因在門口過。進來看看哥嫂。嫂子。你說世上可有這樣非刑。那熊氏道。哎呀。就有天大的不是。別處打幾下罷了。這個虧他下這樣狠心\footnote{不責其行而責其夾。不惜身軀而惜此物。妙甚。}。怎麼捨得。看着不心疼麼。你說我今日爲甚麼生氣。我是個老嫂子。也不怕你。我糞門傍邊原有個痔瘡。這幾日忽然發起來。又疼又癢的。受不得。前日請了個醫生來看。他說我酒色過度纔發了的。這麼胡說亂道的信口胡謅。你看看你哥那個賊樣。還色些甚麼。要果然是這上頭發的病。我就受些疼也沒得怨。何嘗有來。要說酒或者倒還有些。論起來也不多。一日到晚。零零碎碎呷幾斤乾燒酒。怎算得過度。我故此就不肯吃他的藥。早起癢到命裡頭去。沒法了。叫那老奴才來替我舔舔。大約必定好些。誰知他的膽子大多着呢。嫌我的髒。不肯替我舔。我惱了。纔叫他跪着的。游夏流不由得要笑。勉強忍住。假說道。我當別的事。要是這樣說。嫂子錯怪了哥哥了。我哥可敢嫌嫂子髒。這痔瘡是臟毒。全是一團火。人的舌頭上也是有火的。舔的那一會兒雖然受用。過後更疼得利害。這是哥疼愛嫂子。怎麼倒惱他。熊氏道。我不信。人都說有人會奉承大老官的。替他舔癰䑛痔。那怎麼過呢。游夏流道。我怎麼敢欺哄嫂子。事情怎麼比得。那是外人。只圖奉承他那一會兒受用。過後管他疼不疼。嫂子跟前。哥如何捨得。熊氏想了一想。道。要據你這樣說情。還可饒恕那老奴才。起來罷。游夏流忙出去。替他掇下了石頭。扶他起來。甘壽把腰搥了幾下。揉了揉膝蓋。一瘸一點的走了進來。熊氏瞪着眼。道。要不是游大叔替你分辯明白。定叫你跪到明日早起。這一回饒過你。下次再要大膽。牙一咬。道。仔細着你的狗命。又道。你嘴上的毛都白了。還不如大叔一個小夥子。你不羞麼。你別人趕不上也還罷了。自己一個親表弟也趕不上。你不如撒抛(泡)尿死了罷。你活着現世。你看他待嬸子何等小心。是怎個孝敬法子。你也請敎他敎導敎導你。還不去預備些酒飯來投師呢。甘壽虧游夏流救了他。感激得了不得。雖心裡要請他。不敢做主。聽得熊氏吩咐。忙去街上。到大葷館中。掇了四碗上好美肴並些果品之類。又是一小罎上好的金華酒。將菜碟擺下。斟了酒。送上熊氏。讓流(游)夏流坐。游夏流道。哥站着。我怎麼好坐得。熊氏道。他應該站着伺候。你只管坐着。游夏流道。哥在嫂子跟前站着罷了。我是兄弟。如何使得。那裡有這個禮。熊氏向甘壽道。看大叔的面上。賞你坐了罷。他掇了一個杌子。打橫自坐。讓游夏流同熊氏對坐了。熊氏讓游夏流吃了幾鍾。游夏流道。我的酒量有限。倘一時醉了。回去怕你嬸子怪。嫂子請自己多用幾杯罷。熊氏看着甘壽。道。你豎起驢耳朶來聽聽。嬸子的家法這樣嚴。大叔是這樣畏敬的法子。也不枉自是個人。說着話。他一連喝了許多酒下去。把心事趕出來了。向甘壽道。你先聽見大叔說了沒有。他這樣個精壯小夥子。昨日嬸子惱了。還把他的下身夾得稀爛。要論起你這有名無實沒用的老奴才來。該拿鐵夾剪夾得碎碎的餵狗纔好。我見你年老了。這樣寬恩待你。你還不知感激小心呢。又吃了一會。游夏流起身吿辭。熊氏道。又不是外人家。多坐會去罷。忙甚麼。游夏流道。外面晌午了。恐你嬸子要吃午飯。我回家去服事。熊氏道。老奴才。你看看大叔在外邊還記罣着嬸子呢。你在家還懶動。樣樣靠着老姐。他明日嫁了。你也不動罷。除非你就死了。向游夏流道。你要爲別的事。我不放你去。你爲嬸子的飯。這是要緊該去的。空空坐了。閒着常來走走。敎導敎導這老奴才。游夏流道。多擾嫂子。我知道。走了出來。甘壽送到大門口。游夏流道。哥。你有年紀的人了。凡事順着嫂子些罷。不要討苦吃了。甘壽愁着眉道。別的罷了。那個痔瘡怎麼個舔法。如何倒怪得我。游夏流嘆了一口氣。道。不好對哥說的。我甚麼不舔。還打打罵罵。扯不得直呢。也不過頭兩次惡心些。慣了也就不覺。哥。古人說得好。

\begin{quotation}

在人矮簷下。怎敢不低頭。

\end{quotation}

說不得那舔不得的話了。說罷。別了回去。甘壽見他去了。要進去。怕熊氏又尋事打罵。到街上去躱得一會是一會。信着脚步走到大街。見一個廣貨鋪內擺着幾根角先生賣。他心有所觸。想道。把這東西買一個去送奶奶。或者寬恕我些。也不可知。身邊還有先買酒菜剩的幾錢銀子。遂買了一個。藏在袖中。欣欣的回來。一進房。熊氏罵道。你送游大叔出去。就躱死不進來了。我當你永遠不回來了呢。你一般的還要來見我。這老奴奴。樣樣惹我生氣。甘壽等他罵完了。向袖中取出那角先生。雙手捧着。道。我到街上去。偶然看見這件東西。買了來孝敬奶奶。熊氏一手接過。看了看。喜笑道。這東西做得倒有趣呢。正在說着。不防他那女兒老姐一把搶過去。道。媽媽。把恁個魚泡兒給我頑罷。甘壽忙奪過來。道。我的小姑太太。這是我的救命主。你要跌破了。就活殺我了。熊氏要問甘壽用法。見女兒在跟前礙眼。忙拿了十多個錢給他。道。你到門口等着去。看過路有賣的。買一個頑罷。那老姐拿着錢出去了。熊氏笑問道。這東西好是好。怎個用法。甘壽道。奶奶自己用也得。拴在我身上用也得。奶奶請到床上去。我做給奶奶看。熊氏便忙忙上床。甘壽怕女兒來。拴了門。也上去。將兩根襪帶解下。拴在那角先生根。替熊氏脫了褲子。叫他仰臥。又替他紮在脚後跟上。彎着腿。塞入戶中。手扳着脚尖。來回進出。熊氏笑道。好是好。我費力得很。甘壽道。奶奶怕費力。讓我來。替他解下。繫在自己腰間。同他舂搗起來。熊氏久未做此事了。今日忽然得此。那裡肯就住。兩人足足弄了半日。老姐來叫門方歇。因有了此物。熊氏此後待甘壽大不相同。心疼他了不得。甘壽暗暗念佛。又買了三四個來送他備用。還拿了一個供在祖先龕內。朔望燒香叩拜。謝他之力。免了多少打罵。再說游夏流別了甘壽。路上暗笑道。怕老婆竟騙得出肥嘴來吃。可見不可不怕。卜氏那日吃了飯。房中獨坐。又想起昨日的事來。暗道。世上想偷婦人的漢子還恐怕偷不到手。我倒好意去就他。以爲十拿九穩\footnote{偷婦人之漢子固有。偷婦(妖)怪之漢子決無也。如何怪得他人。}。得嘗他那妙物。誰知這樣個大漢子。却是個蠢貨。一點竅也不知道。古人說。大漢不呆眞是寶。一絲不錯。當面錯過了這樣好東西。眞是可惜。心裡想着。慾火又動。昨日還有剩的冷酒。喝了幾碗。走到臨街窗子內。往外張人解悶。忽見兩條狗搭鏈在一處。他家那條大黑狗急得在旁邊亂跳。張着嘴。伸着舌頭喘。抱住那母狗頭。伸出個通紅的㞠子來混聳。他不由得陰中的那水的的達達往外滴。想道。狗的這東西也有這樣大。雖與人的不相像。大的自然也弄得。遂開了門喚那狗。狗見主母喚他。跳下走進門來。卜氏看他那㞠子還不曾收進去。用手捏了捏。道。這比我家那忘八的強大了。我同他試一試。看他比人弄得何如。遂關了門。喚着那狗。忙走到屋裡。脫了褲子。仰臥在一條凳子上。屁股放在凳頭外邊。兩足揸開。狗通人性。見了這個樣子。他也是急了。拿鼻子把陰戶聞了聞。竟伸舌頭舔上幾下。即跳上身來。兩隻前抓(爪)抱住卜氏的腰。那㞠子向陰門亂戳。卜氏忙伸手去捏住。導入牝中。那畜生也知道往上爬爬。儘着往裡聳了幾下。把根子上那大疙瘩都送了進去。狗性屬火。陽物大熱。世間人及畜類陽物皆筋肉而已。惟狗腎內中有骨一條。故此甚是堅久。弄得那卜氏快活難當。他自到游家。不嘗此美味已久。今忽得此。享用奇物。覺比當日那兩個搖鼓兒的同那兩個花子強多。算生平第一次受用了。弄了許久。狗纔洩了跳下來。卜氏覺得渾身鬆活了好些。自得了這件樂處。每日栽派游夏流定要買牛肉四斤煮熟。一日兩次餵他\footnote{從未聞親夫買肉養奸夫者。大約只他一人而已。}。但吃飯。先盛一盆。用湯肉和了餵狗\footnote{卜氏如此敬這狗。也是舉案齊眉的派頭。但所爲略差些。}。飯後無事。就攆了游夏流出去。他定同這狗高興一次。那游夏流只說妻子憎嫌他。故此攆了出來。且落得在外邊躱躱。逢人便高談濶論。數黑道黃。譏議長短\footnote{此輩好譏議人長短者。宜乎有此等妻子。}。那裡知他令政在家中幹這樣的風流樂事。起先卜氏日裡同狗郞君取樂。夜裡還叫游丈夫舔。旣經了狗的此道。他後來覺游夏流之舌不及那狗腎百分之一。舔得一毫趣味也無。晚間將游夏流攆到前邊客位內去睡。他把那狗喚到床上同臥。因同那狗行樂之時。被他那爪子上的指甲抓得皮肉生疼。想了一個妙策。做了四個布套。將他四個爪子套住\footnote{諺云。醜人偏作怪。黑饝饝一兜菜。卜氏之謂。他醜是醜。想頭頗異。}。他馬爬在枕上。黑股高蹶。那狗也就如跳母狗一般。爬上背來弄聳。那雄狗跳母狗。被他將腎鎖住。故不能施展其技。卜氏鎖他不住。任其肆行抽洩(拽)。每次定有一二更纔住。眞是爽心快意。暗悔不早想到配了此狗。却空空守着那沒用忘八的舌頭。那游夏流見卜氏久不用他舔。以爲他大發慈悲。寬恕他了。暗自欣慶。那知這些妙處\footnote{甘壽在祖先龕上供角先生。游夏流亦當繪此狗供於祖先之傍。}。這狗幾次之後也就慣了。一到天晚。就跳上床去等。間或日間也到床上去睡。游夏流見了要攆他。卜氏道。一個看家有用的狗。比你這沒用的忘八強幾十輩(倍)呢。憑他睡睡罷。你攆他怎麼\footnote{自來但聞鵲巢鳩奪。不意下流人之臥楊(榻)竟爲狗所據。}。游夏流或偶然將那狗踢一脚打一下。便被卜氏罵個三日兩夜還不肯住。那狗或一時興動。向卜氏身上混爬混跳。他便攆開游夏流。就脫了褲子。或仰臥凳上。或爬在床沿。任他高興。他疼那個狗勝似游夏流十分\footnote{下流人原不及狗。這却不但他一個。}。約有半年餘。這卜氏竟懷起孕來。柳斗般一個大肚子腆着。游夏流滿心暗意。還只道是他這樣個匪物也可以下種。倘生出個兒女來。不但可以說嘴。或妻子看兒女分上。又寬待他些。也不可知。忽一日。卜氏肚痛。要生產了。游夏流忙去請丈母來收生。到了他家。水氏不在家中。忙同舅子卜之仕尋到他一個乾姐夫楊大家中。纔尋着了。這楊大的妻子是水氏認的乾女兒。只見水氏吃得臉紅紅的。游夏流說他女兒要分娩了。特來請他。水氏沒奈何。纔同來了。你道水氏在楊大家做甚麼事。幾時認的這門親。這楊大是個轎夫。有三十多歲。結結實實一條壯漢。他名字叫做楊大。那個陽物也就放樣的大。他窮苦人。掙了半世纔娶了個老婆。只得十七歲。倒生得小巧。成親之夜。他恁大年紀纔得了個老婆。好似餓虎撲食一般。那管他的死活。一下把這女子的胯骨弄崩了。幾乎喪命。扶養了半年纔好了。那胯骨再合不攏。走路彎着腰還是有些瘸瘸的。後來但是弄一次。這女子哭哭啼啼。禁受不住。楊大總不得暢意。年餘後。這女子懷了孕。有六七個月了。那楊大一日酒醉。拿出擡轎的力氣來。把胎弄傷了。死在腹中。請了水氏來收。整整弄了半日半夜。纔取了下來。水氏夜深回不得家。又因辛苦了。多用了幾杯。就在楊大家睡下。楊大也有半酣。想道。這婆子也纔四十多歲。生得這等肥胖。必定是我的對子。他一個走千家的婆娘。也未必是甚麼貞節的。且弄他一弄。若弄出事來。不過到官挨一頓板子。半夜裡夢見做財主。且快活一會是一會。上前叫了幾聲奶奶。不見答應。又推了兩推。也不見動。大着膽。竟公然將他褲子輕輕褪下。偷爬上身。弄將起來。水氏夢中驚醒。誰知下嘴被他塞滿堵住了。連上嘴都不做聲。楊大見他心悅誠服。遂鼓勇大幹。那水氏不多時上嘴哼哼的震耳。下嘴響亮得驚人。鼓搗了半夜。兩人弄了個爽心快意而後止。水氏被他這一下弄得魂落在他家了。沒三日不到他家來。外面弄得了銀錢就到他家沽飮。二人飮得酒興濃時就幹一度。楊大的妻子不但不吃醋。反感激水氏了不得。再三諄囑。千萬來勤些。他好脫這肉棍之災\footnote{世間婦人皆如此女。則淫風可止矣。}。竟認水氏做乾娘。水氏因疼乾女兒。並沒(波)及乾女婿。常來替女兒當災。這日正在他家小飮。卜之仕知道他娘常在乾姐夫家。同游夏流一尋。便尋着了。水氏滿心三杯落肚之後。等乾女婿拿陽物來孝敬他。不意親女婿來說女兒要生長。不得不去。到了女婿家。女兒肚疼緊了一兩陣。要生了。水氏忙接時。頭一個竟是一個小狗兒。水氏吃了一驚。游夏流替卜氏摟着腰。看見也嚇了一跳\footnote{養得好兒子。這却說不得嘴了。}。卜氏自己心中明白。毫不介意。又陸陸續續下了四五個。也有狗頭人身子的。也有有毛。也有沒毛的。也有產下是死的。也還有叫的。游夏流只疑是得了甚麼戾氣。以致有此妖孽。那裡疑是狗的令郞\footnote{多銀所生者。才眞是狗弟子孩兒。}。恐人聞知。忙忙拿去埋了。他娘知道女兒騷而多淫。也猜了個幾分。却說不出口。但暗暗懊悔。可惜白費了好些錢。送了那些搖籃衣服被褥之類與這樣狗外孫。不若留着買酒肉養楊女婿\footnote{狗外孫。羊女婿。恰對。}。水氏忙忙把女兒收拾完。又到楊家去收拾。女婿到底同他高興了一度。完了願心。方纔歸家。且說那卜之仕聽見妹子生了幾個小狗。他走了來看妹子。坐下笑問道。我聽見你養了幾個狗外甥。可是眞麼。我來看看。怎麼一個也不見。卜氏道。你少要說儍話了。你聽見誰說來。卜之仕笑道。我聽見媽說的。我從沒有聽見人下狗。我故此來看看是個甚麼樣兒。多銀不好答他。拿話支吾道。媽這些時在家做甚麼呢。他道。媽常不在家。十日倒有七八日在姐夫楊家。多銀道。是那裡這麼個楊姐夫。卜之仕道。是媽新認了這麼個肏屄的女婿。多銀道。你又來胡說了。他道。我怎麼胡麼(說)。是我親眼見的。我見他常常在楊姐夫家過夜。又常不住的往他家去。一去就是半日。定然吃得臉紅紅的纔回來。我也有些疑心。那一日我去看看。他家的門掩着。我就推開走了進去。到了院子裡。房門關着。聽見媽哎喲哎喲的叫。我在外間窗子洞裡一張。那乾姐姐在春凳上睡覺。再往那間房裡一張。原來楊姐夫同媽脫得精光。在床上肚子壓肚子弄呢。我看了一會。只見媽仰巴叉的睡着。先還聽見哼。後來動也不動。嘴裡連聲氣兒也沒有了。我說這一下可肏死了。打算要進去叫他一聲。想起當年爲他同爹弄。我笑了一笑。被他打了一頓。我若叫他去。設或沒有死。又好討他打罵。我想就是肏死了。叫做老和尚背着二斗米。是他自討的。與我屁相干。我就回來家。不想他沒有死。第二日又回來了。我想女人家的這東西這樣喜歡人弄。像你在家裡的時候。同那兩個花子兩個搖鼓兒的好弄。我問你。你們的那東西同男人弄得很有趣麼。多銀笑道。你是那裡這些瞎話。信着嘴混講混說。他道。我倒是混講。我看見不知多少回數。我不管閒事罷了。況我的個㞠子倒大得好看。像一根粗皮條。稀軟的不會得硬起來。又弄不得。要我的㞠子會硬。你的屄當日我不會肏。肯給外人肏。多銀見他不住亂說。便道。你家去罷。恐怕媽家裡尋你。他道。我來時。媽往楊姐夫家去了。那裡就捨得回來。有半日肏搗呢。又笑道。我正有一句話要吿訴你。再記不得。方纔提起。我想了起來。你嫁過後。他們四個還常常來在門口搖鼓兒討飯。我先也不理他。那一日。我把他們一把拉住。嚇他道。你們幾個生生把我妹子肏死了。正要到衙門裡吿你們去呢。你們還敢來。衆位街坊來幫我拿一拿。嚇得他們掙脫了好跑。此後再也不敢來了。說畢。哈哈大笑。多銀見他只管混說。怕游夏流回來聽見。數了二十文錢遞與他。道。要買些燒餅你吃。你妹夫不在家。你自己去買了吃罷。你快去。我要睡睡呢。纔把他支了出來。一日端陽佳節。水氏替一家婦人收了生。擾了那人家的雄黃酒。也有半酣。又得了三星謝儀。他買了一隻燒鴨。打了幾斤好酒。又買了些棕(粽)子。到楊大家來。一則過節。二則消酒興。走到內中。悄無人聲。原來楊大的老婆有病。他娘家接了去了。楊大同夥計們吃了一飽燒酒。醉了回來。在家屋裡春凳上睡覺。水氏上前將他搖醒。楊大見了這些東西。笑嘻嘻道。大節下。我沒有得請你老人家的。反倒又費你的事。也罷也罷。我擾了你的鴨子。停會吃上興來。我請你吃雞罷。水氏也笑了。楊大忙去切了鴨子盛上。拿個盤子來裝了粽子。又拿了鍾筷來。二人就並坐在春凳上。一遞一口的飮酒。水氏道。纔到屋裡去。姑娘怎不見他。楊大道。這幾日總不見你來。前日我熬不得了。又同他弄了一弄。把舊病又發了。這幾日。小肚子連腰痛得要死。昨日他娘接他過節去了。水氏道。你也是個冒失鬼。旣知道他有這病根。輕巧些是呢。楊大笑道。何嘗重來。他各人不濟。我還是提心吊膽弄的呢。要是你老人家。還嫌我輕得很呢。那水氏笑着擰了他一下。楊大讓他吃酒。水氏道。我方纔在那人家。他讓我吃了好幾杯雄黃酒。到此時頭還轟轟的。且略消停一會再吃。楊大道。我方纔同夥計們吃公東。多了兩杯。也還不大醒。且趁酒興弄一會子。等醒了再吃。外邊將有午時了。我們肏個毒屄着。就替水氏脫衣裳。他原是爲此而來。豈有不樂從之理。兩人脫光了。水氏就睡在春凳上。楊大站在地下。扛起腿來就弄。且說南京的轎夫論碼頭。一個碼頭上有十二名轎夫。一條街上一個碼頭。單做這一條街上的生意。他們在縣中冊上有名當差。他這十二名算有名正身。縣冊無名。在碼頭上做生意者。謂之散班。月月幫貼些須與他。正身應當官差。南京城中共有八百個碼頭。這是歷來舊例。他們這個碼頭上。因大節下。衆人聚在一處商議。十二個人每人出幾分銀子。買了些酒肉粽子打平火。楊大也在那裡。他丈母昨日來接女兒。楊大打了幾斤酒來相待。他丈母不曾吃。便同女兒回去。\endnotemark[5]楊大昨夜又擡人去吃戲酒。在那家吃了些搶盤。回來夜深。也不曾吃便睡了。淸早起來。聞得酒香。大熱天。涼涼的酒。幾氣喝了。已自有了半酣。又同衆人去吃。一連幾大杯。就醉了回來。那十一個還在那裡吃。內中一個道。楊大哥的酒量每常還好。今日吃不多就醉了。一個道。他來的時候就醺了。一連喝了七八鍾燒酒。又沒吃個菜。他不醉等請麼。又一個道。都不是。這一向我見卜家那胖老婆常到他家來走動。又常帶了酒肴來。一進去就是半日。大約兩個人有個勾搭帳。不然非親非戚。他來做甚麼。今日想是等他拿些東西來過節。兩個弄弄。慶賞端陽。楊大哥享用肉蓮蓬。那胖老婆吞那獨孔藕的意思。一個道。你是這樣猜。據我看起來。兩個學生打架。爲筆。那婆娘替人家收生。或看娃娃病。爭一個錢來就像眼珠子一般。他肯做這賠錢養漢的事。一個道。這倒不相干。楊大哥的本錢也還像個樣兒。他若愛上了。敢是就捨得。正說着。只見一個名張三的出去溺尿。笑着進來。道。那胖老婆拎着一隻燒鴨。一串粽子。一瓶酒。到他家去了。先那一個道。何如。我嘴上有流(硫)磺。一猜就灼。我就猜楊大哥是等他來過節。可被我說着了。他兩個必定又要高興。我們何不大家去躧狗尾。拿住了。訛上他一家。也弄他一下子過過節。一個道。使不得。若單是楊大哥同那老婆幹事。我們去還可以。他有那少年婦女在家。如何去得。又一個道。我昨日見他丈母來接了女兒去了。一個道。他家旣沒人。這還可以行得。又一個道。他難道是呆子。大白日裡兩個人要幹事。可有個不關着門的。我們隔牆爬進去不成。鬧到了官。屄肏不成。屁股上挨一頓好板子。內中一個叫李四的道。看大家的造化。㞠子可行運不行運。我去探探來。若關着門就罷。要是開着門。我悄悄進去看看。他們要是吃酒。我就回來。要是幹事。我來叫你們同去。一個道。他要看見你呢。李四道。要看見我。就說我來看你酒醒些沒有。約他來吃酒。衆人都有了酒意。高興起來。道。有理。你快些去。李四走到楊大門口。輕輕一推。那門隨手而開。並不曾拴死。是水氏來時。不知楊大在家不在家。後來兩人高興時。不但情興如火。況都還醉醺醺的。那裡還想起來呢。李四捏手捏脚走了進來。只聽得一片響。又聽那婦人叫道。哎喲。好大雞巴。我的哥好弄。我要快活死了。你再狠些。又聽得楊大笑道。我這樣狠。你還嫌輕呢。要是你那女兒。好把命都送了。李四輕輕一張。見楊大扛着水氏的腿。在春凳上大幹。那老婆閉着眼。屁股亂攧亂簸。嘴裡混哼混叫。李四忙忙走出跑來。笑道。弄呢弄呢。快些去。一個道。不要冒失。這進去。着一個先搶褲子衣服。李四哥的力氣好。楊大哥素常醋你三分。你便上去抱住他。張三哥手脚伶便。你便抱住那老婆。我們只說同他頑。要抽個頭兒的意思。十分要鬧起來。現有褲子衣服。他二人又是精光着。叫破地方到官也不怕他。衆人道。有理有理。遂一轟齊到他門口。悄悄進去。把門拴好。他兩人正弄到興頭上。也不防有人來。況那水氏淫聲浪語叫得震耳。那裡還聽得人脚步響。衆人推開房門。一擁進去。一個先搶了衣服抱在懷中。李四上前打背後一把將楊大抱住。那水氏正閉着眼哼。睜開一看。吃了一驚。剛纔要掙起來。那張三也從後面連奶胖一把抱住。兩隻手就捏着他兩個乳頭。水氏掙不脫。只拿一隻手掩着那水〖氵韲〗〖氵韲〗的陰門。楊大見衆人。只說來拿奸。面容失色。要想動手。不但人多了。料敵不住。只李四抱住了他。要掙脫了尚難。只得軟說。道。好弟兄。我們一個同行夥計這麼幾年了。就是別人算計我。你衆弟兄該衛護我纔是。你們倒反拿起我來。衆人笑道。楊大哥。你錯會了主意了。我們一個好弟兄。也犯得上來拿你麼。卜奶奶這件好東西是有名的\footnote{這才眞是謬獎。不知在何處馳名。}。你也受用夠了。今日大節下。我們托哥福都沾些餘光。想來嘗嘗新的意思。你捨得捨不得。我們明日湊個大東。一來謝哥。二來培(陪)不是。哥。你怎麼說。楊大見他們不是來捉奸。纔放了心。笑道。你們這些斫腦瓜子的。有話好講。冒冒失失跑了來。嚇了我一跳。老四。你放了我商量。李四笑道。我放了你。你好變卦。楊大道。呆瘟。卜奶奶精光的你們拿着。還變甚麼。一個道。你放了楊大哥。他不是那樣人。我們好弟兄一場。就給我們大家享用享用何妨。這叫做火焼紙馬鋪。落得做人情的。李四也就放了楊大。楊大向水氏道。如今他衆兄弟們是這個主意了。你怎麼說。水氏雖然是個淫浪婦人。此時被一個驀生男子精光抱住。十多個橫眉豎眼的小夥子都望着他。也自羞愧難當的。聽這楊大問他話。低了頭不嘖聲。又一個道。卜奶奶怎麼好說得。哥若肯了。他還有不肯的麼。楊大道。我有甚麼不肯。因向抱水氏的張三說道。你放了手。等我同卜奶奶商議。張三道。這一放。就想要金蟬脫殼了。那一個道。你放了他。褲子衣服我現拿着。他難道光着屁股跑出去麼。張三也就放了。楊大向水氏耳邊悄語道。這事不得開交。不給他們弄一下子。人多勢衆。弄出事來。就大丢醜了。說不得。你給他們了了心願罷。水氏到了此時。也無可奈何了。也悄悄的道。這麼些人大睜着眼睛看着。怎麼好做得。楊大道。這容易。向衆人道。列位弟兄聽我句話。卜奶奶見衆位在這裡。大約也辭不得了。但列位都請到天井裡站站。一位一位的輪着進來。不然都在這裡。不但他婦道家不好意思。就是列位心裡也過不去。衆人道。這使得。我們出去。一個道。我們論年紀的次序罷\footnote{古人兵戈中存禮讓。而此等事尚序齒。到底古風較今不同。一笑。}。省得你爭我讓。我們都是序過的。指着一個年長的道。哥。你就請先上。衆人說着。就出去了。楊大向那拿衣褲的要了穿上。也出去了。這一個上前將水氏抱住。先親了幾個嘴。纔自己脫衣服。這一起轎夫。大熱天都是披着一件小蔴布衫。光着腿。下穿一條蔴布褲。把衫子一撂。褲子扯下。就是精光。他先見水氏赤着身子。一身緊就就雪白的肥肉。兩個滾圓的大奶頭。下身雖用手掩着。覺得鼓蓬蓬的饅頭一般。一大些毛。好不動火。那陽物已硬久了。將水氏放到凳上。挺着就戳。水氏還故意用手掩着。那人笑着在他耳邊道。你合一句俗語。叫做脫了褲子摀着屄生者。拿開手罷。水氏也笑了笑拿開。他是纔同楊大幹事的。陰戶還水淋淋的。一下攮個到根。抽弄起來。弄去一個。一個〈的〉接着上。內中大的小的。粗硬的細的。長的短的。久的快的。種種不一。已輪了六個來的。水氏覺得也有弄得有趣的。也有淡而無味的。到了第七個。陽物旣大而本事又高。輕易不洩。水氏被他弄得着實受用。覺比楊大強多。因外邊人衆了。不敢聲喚。只拿屁股亂顚。樂極的時候。反把他摟過來送嘴遞舌。悄向耳邊道。哥。你好手段。你姓甚麼。他道。我叫做張三。水氏摟住道。好三哥。你再狠些快些。張三承他格外垂靑。也竭力以事。多時事畢。又換一個來。那水氏一看。就是先抱着楊大的那人。弄將進去。水氏覺他的陽物比張三的又還粗長雄壯些。一上手就有千餘抽。水氏被他弄得丢而又丢。心愛至極。問他姓名。他說叫做李四。他弄的工夫比張三更久。方纔完事。水氏經他二人這大弄了半日。興也足了。陰中也就算飽足了。到第九個上。水氏也就覺得有些吃力。低聲的說道。你歇一歇。讓我略停一停。那個道。我們等了這一會。熬得慌了。旣承你的情。就一個人情做到底。略忍忍兒罷。也快了。只得我們三個了。越抽送得利害。水氏只得忍受。十一個人自晌午弄起。直到日色將落。方纔完事。衆人齊笑着說道。多謝奶奶了。又向楊大道。多擾哥。改日酬情。哈哈大笑。出門而去。楊大關了門進來。看水氏時。見他坐在凳頭上。兩隻脚躧在地下。腿大揸着。皺着眉。手揉着小肚子。那陽精滴滴達達往外滴。那陰毛上沾得黏達達的。活像個鬍子嘴喝了奶子茶一般。這水氏四十多歲的婦人。又生得胖壯健大。雖經這些人蠻弄了半日。竟還不覺得狼狽。楊大問他道。你覺怎麼的。水氏道。小肚子脹得很。腰眼裡有些酸疼。楊大道。你還仰睡着。等我來替你揉。扶他睡下。楊大脫下一隻鞋來。按在他小肚子上一陣揉。那陽精一陣陣汨汨流出。楊大一面揉着。一面笑說道。古人說的話一點也不錯。他說。世上人吃到老穿到老學不了。千眞萬眞。我那一日偶然聽人說閒話。他說這些流賊搶了婦人去。有三五個賊人共一個的。或十多個賊人共一個的。再婦人少了。拿了去傳營。四五十賊共一個。弄得那婦人的肚子像柳斗一般了。拿鞋底烤熱了揉。揉盡了又弄。我聽在心裡。誰知今日你竟用得着。水氏笑罵道。都是你這斫刀的引的頭。叫我吃這一場虧。還說這鬼話呢。楊大道。你不怨自己進來不關上門。倒埋怨我。水氏道。我當你不在家呢。後來就忘記了。楊大道。事已過去了。也不必抱怨了。我看你的這東西還是照舊的一個窟窿。又沒有裂開。又沒有塌皮。並不曾弄壞了甚麼。下次小心些就是了。兩人笑了一陣。楊大又用手替他抹了一會。見陰中沒得流了。尋了塊破布來。遞與水氏。揩淨了起來。看地下時。淌了一大灘。楊大撮了些灰來掩了。水氏走進房中。到床上坐下。楊大點上燈。來廚下把酒略溫了一溫。把鴨子粽子掇進來放在桌子上。掇了靠床放下。光脫了同水氏相摟着吃酒。楊大道。你餓了。吃個粽子。水氏道。我不覺得餓。肚子還有些脹。我不吃。呷幾鍾酒罷。楊大道。我從早晨起來就喝了一飽酒。被他們來鬧鬧吵吵的。到這咎晚。連飯糝兒還沒有嘗着呢。我可要偏你了。一面剝粽子。就把鴨子吃着。一面笑說道。說話都有個讖語。我先說擾了你的鴨子。停一會請你吃雞罷。果然就吃了這麼些。水氏把他擰了兩把。頑笑了一會。楊大把鴨子粽子吃了一飽。二人收拾完了同睡。楊大摸着他的陰戶。道。好結實東西。就是生牛皮做的。被這些人搗了半日。也搗鋊了些。你這個就不曾壞了一點兒。水氏笑着將他打了一掌。楊大道。頑是頑。笑是笑。今日大節下。你的興大約是足了。我先雖弄了一會。並不曾盡興。你再給我足一足興是。水氏道。還興呢。再興興。我好送命了。楊大道。不妨事。我試試看。若弄不得就罷了。水氏拗他不過。只得讓他上身。纔抽了幾下。水氏推住。道。來不得。裡頭深處有些疼呢。你睡睡。到天亮看好些再弄罷。楊大也就下來。大家睡了。直到天明醒來。楊大問他道。你這會子好了。水氏道。肚子雖不脹。兩胯骨倒有些疼起來。楊大道。這是那些孽障們蠻撞的。我再弄弄看。二人又弄起來。水氏道。弄不得。裡頭雖不覺怎麼了。這蓋子骨疼得很。楊大下去一看。見腫得像個大饅頭一般。楊大道。你又弄不得。我又脹得慌。這怎麼處。水氏見他着急。爬起來一把攥着陽物。用口一陣裹咂。咂得楊大骨軟筋酥。冒了出來。水氏都嚥了\footnote{一部書中淫婦不少。而未有水氏之可笑者。昨日下嘴去吸了十一人之精而猶未足興。今日上嘴又吸楊大之精。大約他是狐狸一轉。不然何好吸人精髓至此。}。楊大感他不盡。抱住親了二三十個嘴。二人起來梳洗了。水氏道。我此時要家去。下身疼。走不得。你叫乘轎子來送我。去到家打發他錢。楊大笑道。他們還好要錢的。那就太沒天理了。又道。你旣下身疼。住兩日。等好些再去罷。忙甚麼。水氏道。我家中沒人。只大孩子在家。我昨日只說到你這裡來。還回來大節下同孩子過節。誰知被他們囉唣來。就躭誤住了。我今日要回去看看。楊大道。也等我買些菜來。你吃了飯去。水氏道。等不得。我不吃了。楊大留他不住。就到碼頭上去。衆轎夫向他笑道。昨日多擾哥。楊大笑道。你這起天殺的。也放輕巧些。拿出擡轎的本事來弄。被你們把他都弄癱了。他要回去。走不動。你們擡乘轎送送他去。還好要錢麼。內中那張三李四兩個。昨日承水氏親愛。就跳起身來。笑道。那裡有要錢的道理。我兩個送了他去。二人擡了轎子到楊大門口落下。水氏出來上轎。二人笑着道。昨日多謝奶奶。水氏笑着瞅了一眼。上了轎。二人擡上。直送到了他家。水氏下了轎。說道。你等着。我取錢來給你。二人笑道。我們不要錢。多謝奶奶罷。水氏在手指上攄下兩個銀戒指來。每人贈他一個。他二人不肯受。水氏笑道。這不是給你做轎錢。是送你們做記念的。好好的收着。他二人方笑吟吟作謝收了。水氏道。這個不要給你楊大哥看見要緊。二人答應道。我們知道。擡着轎去了。水氏過了四五日纔覺得全好。又遲了兩日。那兩隻脚不由得又走到楊大家來。楊大看見他。一把摟住。先親了個嘴。就伸手去扯開褲子。摸他的陰戶。道。你全好了麼。水氏道。好了。也疼了好幾日。姑娘還沒有回來麼。楊大道。沒有呢。我前日去看看他。病還沒有好。來家沒人服事。他就好時也不禁大弄。倒不如等他在娘家住着去罷。只要你老人家來勤些就是了。稀罕做甚麼。又道。我前日沒有得盡興。你旣好了。我們今日盡一盡着。水氏道。且慢些。等我去關了門來。遂出去把門拴好。猶恐不牢。還尋了根棍子頂住。他雖是傷弓之鳥。也可謂賊去關門。進來同到床上。掀天揭地。大幹了一番。事畢睡下。水氏有心問道。前日衆人中。那兩個姓張姓李的好精壯小夥子。比你還利害。楊大笑道。他兩個都纔二十多歲。有名的勝。叫驢張三。鐵棒槌李四。我們向日幾個人到水西門彎子裡去打釘。賭本事贏東道。數他兩個是魁首。那軟弱些的婊子都有些怕他。大約那日把你弄傷的就是這兩個天殺的了。水氏心中方知他二人的混名。牢記在心。水氏被那些轎夫夥淫之後。這種人可還有甚涵養。顧甚羞恥。以爲是一件趣事。每每互做笑談。傳得這條街上人人皆知。這街上住的有一個富豪子弟。他祖父也都歷仕過。他覆(複)姓單于。單名一個學字。年纔二十以外。生得柔媚。宛如女子。做人倒也不甚凶惡。但生性貪淫。且酷好戲謔。他戲謔起來。所作所爲都是人想不到的謔法。南京上元燈市中。極其熱鬧。自初八日起。到十八止。賣燈者固多。而看燈者更〔多。〕兩邊樓上。或是王孫公子。或是大家宅眷。都租了看燈。窗上懸了珠簾。簷外掛上各種異燈。飮酒作樂。有那一種中等人家內眷。又愛看燈並熱鬧。要租燈樓。又無此力量。只得雇了轎擡着看燈。那市上燈光如晝。眞是人山人海。內中就有那無賴子弟。便伸手到轎櫃中。把婦人的鞋脫了下來。那婦人要叫喊。又怕羞。那時就是叫。他縮手閃開。無千帶萬的人。知道是誰。燈市中婦人失鞋者。十有五六人。人知有此。而年年有人去。是這一種風俗。他本家的男子也不能禁。單于學最恨這一種脫鞋的惡少。他想了一個妙謔。到了燈節。他自己扮做女裝。做一隻假脚。微露轎簾之外。却用釘子釘住。繡裙掩覆。他盤膝坐在轎上。手中拿着二尺餘長的粗錐子。磨得極尖極利。他眼睜睜看着那鞋。叫轎夫擡着在市上來回走。但有人伸手來捏脚。便是那一錐。那人被戳。又不敢做聲。只好忍疼避去。如此數日。那轎櫃的板上。血竟汚滿。結得大厚。你道他這戲謔有趣不有趣。利害不利害。更有妙者。每逢端陽。秦淮河燈船龍舟不計其數。那兩岸河房內。人俱租盡。不消說得。而在河裡坐船遊頑者也便如蟻。那來遊的婦女小戶人家。如何來得起。自然都是大家閨秀。船上雖然四面垂着簾子。日光射照。通通大亮。雖有如無。也有一種輕薄子弟。雖不敢以船傍船來賞鑒。把他船在這船的左右前後。總追隨着遊蕩。單于學也做婦人裝束。坐在船上。吩咐船家專在熱鬧處遊。引得這些少年把船都不離他。到了上午。他叫把船泊住在文廟前月牙池內。衆少年的船也都遠遠的圍遶着。他忽然叫捲起簾子。把褲脚裸至膝上。伸出兩隻大脚。在河內濯足。那些少年見了。無不含愧好笑。忙忙都開船避去\footnote{單于學之戲謔。一是處無賴惡少。一是辱輕薄少年。較鐵化之尖刻強多矣。}。他腰中有一根驚人之具。長約七寸有餘。又粗又壯。他恃了這根孽具。且又有一個動人的好臉。就專在婦女上做工夫。家有一妻三妾二婢。竟猶不足。尚在外邊尋獵野食。他每常無事站在大門口閒望。見水氏常常在街上來往。年紀雖非少艾。却丯韻頗佳。眉目中大有騷致。他就高興起來。道。這老婆。看他這個樣子。睡情定有可取。古人說。半老佳人可共。何不想法兒弄他一下。自有別趣。尚恐他面目雖騷。或者心中貞靜。倘一時下手不從。豈不弄出事來。近日風聞得他同楊大私通。被衆轎夫囮上朋淫。笑道。這樣的婦人。弄之尚有何患。遂想了一個主意。每日在門口等他。一日。見水氏遠遠走來。忙叫一個小廝。附耳吩咐了幾句。他便跑了進去。水氏要往楊大家去。定在他門口過。只見一個小廝叫道。卜奶奶。你來得正好。我纔要請你去呢。水氏便走到跟前。問道。你家姓甚麼。請我做甚麼事。那小廝道。我家爺姓單于。低聲道。爺跟前的一位姨娘要生產了。養不下來。叫我去請你。快些進去罷。水氏便同他進來。那小廝引他到了書房門口。水氏道。你家姨娘怎不在上房在書房裡。那小廝道。我家奶奶利害得很。爺偷娶在這裡的。我不好進去。你老人家請進去罷。水氏信以爲實。便掀開簾子走了進去。見一個少婦坐在床沿上。兩手摀着肚子。道。快些來。我已生下個孩子的頭來了。只身子不下來。你快救我一救。水氏道。哎呀。你這麼個大人家。怎也不叫個人來摟摟腰。自己一個人在這裡生長。一面說着。忙脫了上蓋。解去裙子。把兩隻袖子捲了捲。伸手到胯中一摸。不見孩子的頭。却摸着一個小和尚的光腦袋。挺硬的豎着。水氏攥在手中。覺比張三李四的還長大些。是生平未見的奇物。笑得了不得。攥住說道。原來是耍我。我把這孩子拉出來纔罷。攥着往外扯。單于學見他毫無羞怒之意。知他是愛上了。便兩手將他拉了上床\footnote{拉上床。妙。單于學身小。水氏胖大。抱他不動。故用拉也。}。解帶脫褲。他並不拒。任憑脫下。單于學便伏上身。一連幾搗。全全入去。水氏覺得內中之樂無窮。眞嘗所未嘗之樂。下下戳在一個癢癢酥酥的去處。大有妙境。聳身上迎。正在高興。只見單于學抽了不到十數下。早已吿竣。水氏一把扳住他屁股。道。哎呀。歇不得。我的祖宗。你這一歇。我就要死了。快些掙着再動動。單于學見他騷到極處。也還要勉強抽抽。不意那物件已像蚰蜒蟲一般。毫無硬氣。把個水氏急得亂叫。單于學原也有百十合的本事。因自己斲喪過度。遂至如此。被水氏推下身來。一面穿着褲子。說道。這樣的武藝還想調弄婦女。保着家裡的不吃野食就夠了。穿上了下床。又穿了衣裙。就往外走。單于學滿臉含愧\footnote{單于學此辱。自取之也。亦可爲不自量者戒。}。說道。你吃了酒飯再去。水氏道。嗤。我稀罕你家的吃呢。嘓嘓噥噥。揚長出去。被他引得心中火起。到楊大家痛痛的樂了一番纔罷。且說多銀自生了那些小狗\footnote{多銀所生才眞是犬子。}。他想道。不過是下些狗了。又不得傷命。是落得快活的。每日買肉飽餵那狗。十數日後。他身上乾淨了。也等不得滿月。見那狗不住在他身上左跳右跳。便興動起來。同他交合。此後也不拘日夜。或是狗一時高興向他跳。或是他一時興動。兩個便相親一番。不必多敍。一日。他對門開了一個麪鋪。買了兩頭翻肥的叫驢。輪流推磨。他是臨街的房子。沒有後院。無處拴驢。日裡借游家的後院拴歇。天晚牽回家中去餵着。每日許送兩枚燒餅。卜氏聽見有燒餅就依了。過了幾日。卜氏偶然見那驢子。有時伸出那㞠子來。開了花。一個大喇叭頭子硬幫幫的。打得肚子山響。他高興道。這件東西倒長大的有趣呢。比狗的強了許多。我何不試他一試\footnote{金甁梅王婆說風情。內有潘驢鄧小閒五件。今多銀略去其四。只取了一驢字。}。難道也會下小驢子不成麼。興不可遏。見那個大驢子的㞠子太大。那一個小驢子的略細短些。他道。先向這小的試試。若不妨事。再試大的。那一日將午。鋪中將大驢牽去。換了小驢來暫歇。不多時。那驢子便將㞠子伸了出來。多銀忙拿了春凳來放下。褪掉了褲子。一手撁着驢子。自己仰着在凳上。將驢子身上拉。那驢子蠢夯。沒有狗通人性。見人睡着。不但不敢上身。竟反往後退。如此數次。多銀急了一身冷汗。坐起看時。那驢子連㞠子倒縮了進去。伸手去捏捏他。反混跳起來。多銀沒法了。火都急了上來。又見那狗往身上混爬混跳。他牽着驢子看着。又復睡下。憑那狗幹一度。一連幾日。那驢子也看熟了些。大畜生也有些靈性。他雖不敢便跳。也就把那鼻子到陰門上聞聞。仰着頭。齜着嘴一會\footnote{有一笑談。一人云。狗屄甜。驢屄酸。人屄有病。又一人問云。何以見得。彼云。狗屄若不甜。公狗每每好舔者何故。驢屄若不酸。驢但一聞便齜着嘴鼻何故。人屄若無病。何以一弄便哼。多銀之物。驢也齜嘴。豈酸耶。}。又來聞聞。卜氏見他敎得有些要會了。越發拿狗來做個樣子與他看。忽一日。卜氏纔在春凳上睡倒。牽驢到跟前。他聞了聞。嘴齜了齜。㞠子挺硬的突出。就往身上跳。卜氏大喜。忙捏住了。送入牝中。那驢子也往裡聳了聳。卜氏覺得陰中塞了一個滿足。渾身都覺得受用了。怕他太長。一隻手攥住了小半截。那驢子聳了幾下。不得盡根。用力一送。那春凳一歪。卜氏幾乎跌了下去。忙放手。把春凳一搬。不意那驢子覺那㞠子上沒了阻攔的東西。狠命往裡一下。直攮到心窩之內。卜氏早已吿斃。那驢子那裡知道人死了。他還痛快弄了幾下纔拔出來\footnote{試看首段。一想多銀爲誰。驢子爲誰。狗爲誰。則不足爲異矣。}。午後。游夏流來家。要打發卜氏吃午飯。到門口敲了幾下。不見來開。疑是睡着了。不敢驚他。等到日西分。恐誤了煮晚飯。又要獲罪。在心中着急。只得輕輕將門撬開。到房中一看。不見有人。到了後院。見多銀光着下身。仰睡在春凳上。兩條腿拖着。那條大狗爬在身上抱着亂聳。見了人來還不肯放。游夏流還當是卜氏偶然醉了。乘涼睡覺。狗來放肆。到跟前。忙把狗打開。陰中鮮血淋漓。又見滿地全是鮮血。吃了一驚。看卜氏時。已經死了。心中大疑。不解其故。忽見那驢子散着在地下啃草。㞠子外邊血滴滴的。方知道是被他弄殺。又見那狗的這一番作爲。方悟到前次所生皆此犬的令嗣。並非甚麼戾氣。只得將死屍抱了進去。展盡血汚。將腿扯直了。替他穿上褲子。去下塊門板來。停好了。忙跑去丈母家中報信。恰巧水氏纔從楊大家回來。面上還帶着些酒意。游夏流將他令愛的死狀細細奉吿。那水氏聽見女兒這個樣風流死法。遍紅了臉皮。說不出來。卜之仕咧着嘴。不住嘻嘻的笑。水氏同游夏流到了他家。進去見了女屍。哭了一場回去。游夏流買棺盛殮。這一回他不遵文公家禮了。竟延僧道念經禮懺。也不用庶人葬禮。整放了三七二十一天。又雇人下鄕報信與卜通。原來卜通在土山一個姓易的財主家處館。私通着一個學生的母親焦氏。是個寡婦。打得火熱。戀着他。有半年多不曾回來。那裡知他夫人也厚上了個乾女婿。可見男人以爲在外邊偷老婆是件極便宜極快活的事。孰不知自己的老婆在家中也會偷漢子。更受用更快樂呢。世人想到這個上頭。像這樣風流的事不做他也好。卜通聽見女兒死了。不得不來。到了女婿家。也哭了幾聲。水氏同游夏流都只說多銀病過。不曾吿訴他那些妙處。一來他心中記罣着焦氏。二來夜間水氏交合時。費盡筋力。毫不見他有樂意。那水氏自經了楊大同衆轎夫之後。色量大開。大非向日之比。卜通又不好問得。他覺全無趣味。等不得女兒下葬。推館曠不得。忙忙又下鄕去了。水氏正嫌他來家礙眼。正要他早去。見卜通去了。也不管死女兒家中念經。且去同乾女婿飮酒作樂。游夏流將卜氏出殯埋葬。不必細說。事體完了。游夏流見那狗滿床混跳。終日噑叫。似有追念卜氏之意\footnote{狗尚有情。人於夫婦之間。待之等於奴隷。視之同於陌路。或無故而休逐。狗亦勿若。}。不勝恨怒。一頓棒打死了。見甚肥壯。煮而食之以洩恨\footnote{此非食狗。乃食奸夫耳。何不更寢其皮。這狗吃了游夏流許多牛肉。今游夏流吃他。只算得還席。但此狗游夏流之恩人也。殺之未免太過。豈非以怨報德耶。自多銀得了此狗。游夏流的舌頭省了多少辛苦。豈非狗之德。}。那頭驢子。多銀死後數日。又不曾病。無故死了。游夏流聞得。心中暗喜\footnote{他雖喜。麪鋪却晦氣。論起來。此驢游夏流當感之。不當懷恨。多銀若非他這一弄而死。閫罪受到何日了。}。他家這些事。外人怎得知道。游夏流與水氏說信時。卜之仕在傍聽得。他以爲是一件奇聞。四處吿訴人。我妹子下了一窩小狗兒。又被驢子肏死了。所以傳揚得四處皆知\footnote{游夏流之下流大名。屢屢彰著。卜之仕不知事之美名。此一回內方大顯。}。一日。游夏流自思道。我因不孝父母。自甘游於下流\footnote{游夏流尚能自知。更有自己下流而竟不知者。游夏流不若也。}。娶了妻子。受了多少凌辱。他這個死法。雖是他淫多惡甚的報應。但我這醜名難掩。我這個樣子。今生也未必能生子了。再娶一個。又是如此。豈不是自討苦吃。他老子所遺的數百金。自娶卜氏費了些。又這兩年毫無進益。卜氏每日要肉要酒。不敢不供。又是這一番殯葬。已幾幾將盡。他發了個狠。將房子什物全賣了。拽着些銀子。做了道士。往陝西終南山出家修行去了\footnote{以便異日好歸姚澤民帳下。}。游混公做了一生的壞人。混了一輩子徒弟。落得兒子出家絕嗣。媳婦被驢弄殺。眞可嘆息。卜通做了一世不通的先生。生女如此。亦足寒心。私淫學生之寡母。其妻亦淫擡轎之假婿。報應絲毫不爽。幸得兒子卜之仕。只呆儍不知事而已。還不曾有大醜大惡處。又不幸中之幸也。然而血祀已斬矣。人生世上。天理良心四字可忽乎哉。按下不提。再說宦蕚自會過錢貴。時常想慕。但同賈文物童自大相會聚飮時。無一次不講他怎樣風流。如何標致。一日。中秋下旬。宦蕚約他三人在家中賞桂花吃酒。那桂花有十數大盆。皆有鍾口粗。絕精磁盆栽着。有紅白黃三種。開得甚是芬芳馥郁。有一首古詞贊他道。

\begin{quotation}

花則一名。種分三色。嫩紅妖白嬌黃。正淸秋佳景。雨霽風涼。庭前四處飄蘭麝。瀟灑處。旖旎非常。自然風韻。開時不惹蝶亂蜂忙。攜酒獨揖蟾光。問花神何屬。離兌中央。引騷人乘興。廣賦詩章。幾多才子爭攀折。嫦娥道。三種淸香。狀元紅是。黃爲榜眼。白探花郞。

\begin{flushright}右調金菊對芙蓉\end{flushright}

\end{quotation}

飮酒之間。宦蕚又說起道。久不見錢貴。大約像是好了。我們此時花已看足。何不乘着酒興。到他家去一樂。童自大道。哥這想頭眞好。我這兩日也正想他呢。快些去。先對二位哥說過。我這一回是定要摸摸他的。二位哥不要吃醋。大家大笑。宦蕚叫家人將前次寫的扇子拿着。一羣惡少遂一轟到錢家。來到得門首。門却緊閉。家人們就上前敲門。敲了幾下。只見郝氏把門開了。鄔合道。三位老爺又來訪你令愛了。郝氏道。小女病尚未好。得罪衆位老爺。不能陪侍。宦蕚對衆人道。不要理他。我們只管進去。郝氏攔門站住。道。實在有病。老爺就進去。也不能奉陪。宦蕚怒道。胡說。推開門。郝氏見衆人逕往裡走。那郝氏不敢十分阻攔。人也多。阻攔不住。宦蕚走到錢貴房門口。早見他同一個俊俏書生並坐。互相談笑。不覺大怒。道。這廝好好在家。如何哄我說有病。放着我們這樣的大老不留。倒陪着酸丁頑耍。我煩了名公寫了詩扇來贈他。他反不識擡舉。這等可惡。惡向膽生。因吩咐衆家人。替我打。這些惡僕跟主人作惡慣了的。況此時見主人惱了叫打。遂將門窗槅扇。桌椅擺設。無不打到。把個郝氏嚇得躱的無影無踪。代目也不知躱在那裡。財香藏身在蘆柴堆下伏着。宦蕚又叫家人採那瞎奴出來。衆人正待上前。倒虧鄔合攔住勸道。大老爺請息怒。大約這是虔婆的不是。與錢貴無干。萬不可因這些小事氣了老爺玉體。正勸着。那鍾生見打得太不像樣。又見他叫採錢貴出去。着了急。顧不得捋虎鬚了。上前說道\footnote{見此數語。方見鍾生非是不知時務之人。輕捋虎鬚乃護錢貴。爲情所使。故奮不顧身耳。}。煙花之地。貧富皆可來往。即回有病。亦無甚大過犯。先生何必如此太甚。宦蕚聽了。越發大怒起來。道。你這小畜生是何等樣人。敢稱我先生。也敢在我老爺面前來講話。童自大仗着宦蕚的惡勢。大讓(嚷)着道。反了反了。就是我。誰敢不叫我一聲老爺。況我大哥。你叫他做先生。你也睜開眼睛看看他是誰。世上有這樣的體面先生。好可惱。可惡。打呀打呀。揎拳擄袖的混叫。鍾生道。我黌門中人。稱人先生足矣。況我們雖是布衣之士。上可以對王公。如何就說不得話。且官府也不過是秀才做的。何得便如此小覷人。賈文物搖擺着道。他二老者。江南之大老也。你不要把自己之靑衿太妄自尊大了。你就中舉焉。不但我是你的前輩。諒你一貧窮之舉人。而何能及我巨富之進士者乎。吾語汝弗如也。由此觀之。汝則一妄人也已矣。宦蕚道。你就算是秀才。我且打了。看你那裡去吿。不要說你那學道敎官。你就三學約上千把秀才。往午門叩閽。到東廠吿狀。我也不怕你。誰不知道如今當朝的魏上公是我同二弟的祖爺。童自大道。哥。那裡有力氣同他講。這樣打得不快活。拴起來帶他家去。吊在馬棚底下打個稀爛。纔出得我這口氣。不然我不惱。怎就不叫我們老爺有這樣天翻地覆的事。我肚子都氣脹了。宦蕚被他一挑唆。竟聽了他。就叫三四個家人將鍾生拿住。把錢貴拴起。鄔合又苦勸道。晚生乞個恩。他這少年人不知事。認不得衆位老爺。錢貴又是個瞽目人。可憐見的。求大老爺開恩罷。正勸不住。只見宦實京中差來的一個家人。遠行裝束。跑得氣喘吁吁的。走到宦蕚面前。叩了個頭。禀道。京中有緊急信到。隨將一書遞上。原來是宦實的一封釘封家書。甚是嚴密。宦蕚忙拆開一看。內中說道。天啓已崩。崇禎今上御極。魏上公事已大壞。發往鳳陽看陵。在途中阜城縣地方已經自縊。磔屍問罪。目今訪拿他黨羽。從重議處。我向日拜他門下。未曾助惡。幸得隱密。故尚還未露。爾在家要十分收歛。恐爲人摘發。身家性命難保。萬要小心。諄囑諄囑。不盡之言。來人口述。宦蕚看到此處。一團惡興化爲冰雪。不覺面色如土。賈文物接過一看。他也是有心病的。嚇得屁滾尿流。大家擠擠眼。一轟出門而去。那些家人見主人如此。也不知是甚緣故。也把鍾生錢貴撇下。趕了去了。這正是。

\begin{quotation}

從前做過事。沒興一齊來。

\end{quotation}

這宦實父子後來如何。鍾生與錢貴幾時纔成配偶。要知衆〇來事。仔細接看後文。

姑妄言第十二卷終



\endnotetext[1]{「鈍翁」二字原缺,據體例補。本卷首三葉因遭燒損,故缺字頗多。}

\endnotetext[2]{「董」字原缺,據正文補。}

\endnotetext[3]{「輕重」原作「重輕」,「死生」原作「先死」,據書前目錄改。}

\endnotetext[4]{「名公」原作「名工」,據下文改。}

\endnotetext[5]{「回去」原作「女兒」,據文義改。}

\setcounter{footnote}{0}

\theendnotes

\part*{姑妄言第十三卷}
\addcontentsline{toc}{part}{姑妄言第十三卷}
\markboth{姑妄言第十三卷}{姑妄言第十三卷}

鈍翁曰。世上呆人固有。再無呆至於不知好淫者。如童自大可謂呆矣。向見仙桃一笑而癡。今見錢貴之美而移情於美郞。彼豈呆於平日而忽乖於一時。平日之呆爲鐵氏威之所鎭。非眞呆也。今一旦興之所至。忘其威而呆亦減。故有此舉。豈多年之美郞。至今日方知其美耶。試看鐵氏威嚴稍霽。他便知說鬼話倣春宮。以解鐵氏之怒。後試肥後庭。買角先生諸事。豈純乎其呆者耶。如馬台之呆。可謂出類拔萃之呆矣。猶敎而能。何況於他。然而童自大說話行事。終帶三分呆氣者。此所以爲童自大也。

余前云葵心蓮瓣即鐵氏下體之形。或有以我爲強解者。試看他今被童自大水旱齊行。而葵心蓮瓣皆屬於彼矣。豈謬言哉。

嬌嬌阮最以淫死。郟氏阮優以殺死。愛奴花氏以國法死。馬氏苟雄相攜而逃。阮家門風興旺至此。大鋮諛逆璫。屠忠義。縱逃得性命。又不若崔呈秀五虎五彪。雖受一刀。還死得乾淨。在彼時伊得漏網。正人君子曷嘗不以爲恨。彼自己又曷嘗不以爲祖宗之護庇。孰不知留得他在。正是神鬼之報施。上蒼之厭惡人也。

阮大鋮之所爲。余深有所不解焉。大鋮之於逆璫。可謂諛之至矣。而所獲之爵位。不能及魏廣微之堂堂宰相。崔呈秀之吏兵尚書也。所獲之金多。不能及崔呈秀玉田之家藏。京邸之暗寄之廣也。所操之權柄。不能及田爾耕許顯純之能生殺也。而助(視)彼所爲。毒惡殆過於諸人。徒貽罵名於後世。是何心哉。

末後龐周利遇馬氏一段。而報應有四焉。阮大鋮之如夫人落爲萬人之妻。其報應者一。苟雄奸主母。又拐小主母而逃。爲亂箭攢死。其報應者二。馬氏背夫主拐逃。落而爲娼。較一死尤甚。其報應者三。強盜殺人即罹法網。其報應者四。或曰。龐周利旣識馬氏乃主人之妾而淫之。何以無報。全(余)曰。馬氏旣已爲娼。龐周利之罪似可稍減。然終有後報。特緩其期年。

\chapter*{姑妄言卷之十三\\
第十三回 鐵氏女水陸二路齊行 童自大粗醜兩鬟並納\\
附 阮宦淫兒婦首郟次花 苟奴奸主母先毛後馬}
\addcontentsline{toc}{chapter}{第十三回 鐵氏女水陸二路齊行 童自大粗醜兩鬟並納}
\markboth{第十三回 鐵氏女水陸二路齊行 童自大粗醜兩鬟並納}{第十三回 鐵氏女水陸二路齊行 童自大粗醜兩鬟並納}

話說這童自大雖然是個財主。在家中終日守着一位其軀如牛。其性如虎佳人。左右所使只剩兩枚粗醜之婢。當日雖見過仙桃標致。只呆臉看了一看。幾乎闖了潑天的大禍。後來見了家中略是人形之婦人。怕惹是非。就遠遠趨避。從來未曾親近過一個俊婦。他在家中慣了。就是在街上遇見人家婦女。也忙忙低頭走過。前次在錢家同錢貴共桌飮酒。看了他那一段風流的嬌態。美貌的花容。十分動火。這次滿意要來綽趣。不意敗興而返。雖同宦蕚衆人跑了出來。半路各散回家。他還〖目夢〗〖目夢〗戇〔戇。〕不知是甚緣故。問着人又不答應。到了家中。且不歸房。走向書房中睡了一會。天色已晚。掌上燈來。心中胡思亂想。慾火按納不住。他向日買的那蘇州小廝。名喚美郞。生得聰俊識字。放在書房中收拜帖管帳目雜事。眞是一個上好的龍陽。因自來懼內。不敢妄想及此。二者從不曾嘗過這種滋味\footnote{大約其味臭而苦。然而又有以爲香美者。則脾胃不同耳。}。故此不曾作興。侄(値)他今日一團高興。無處發洩。意欲領略領略他這妙股。雖然怕奶奶知道。想道。他今日知道我不在家中。未必就來探聽。遂放了膽。剛纔要叫。恰好這小廝斟了一杯茶送\footnote{眞湊趣。}。童自大此時色眼迷離。燈下看他那嫩臉紅紅白白。越覺可愛。情興勃然。也顧不得內政嚴肅了。忙就拉過來。摟着脖子。親了個嘴。那小廝眞是行家。更會湊趣。就吐過舌頭。童自大從未經此趣。覺酥麻。叫小廝關上了門。童自大將他橫按在榻上。兩足立地。美郞忙自己褪下褲子。高蹶白股。更嫩得愛人。他自己也無暇脫衣。只將一條鸞帶束了腰。把衣襟找拽起來。忙忙取出肉具。向糞門上直頂入去。原來這小廝竟是輕車熟路。全無艱難之態。童自大初嘗此味。緊箍箍。熱烘烘。美不可言。下下皆有着路。比每常享用他令政那肥而且大。寬而且深之妙牝。大不相同。抽抽扯扯。正在入神之際。不想這一日。那鐵氏飯後無事。偶然到他書房中來走走散悶。各處翻看。忽然在書架抽屜內翻着了一本春宮。就是宦蕚前次看見。記了幾樣勢子。回家同侯氏試驗的那本册頁了\footnote{前後照應。一絲不漏。}。鐵氏見上面花花綠綠。畫得好看有趣。想道。這天殺的。有這樣好東西。他藏在外邊自己看。不與我見。我且拿了去。叫他逐樣做來。方纔饒恕他。遂抽身回房。又細細看了一遍。皆他平生不曾做過的樣式。想道。他旣然看過這款式。怎從不學做一個。眞正有幾分呆氣。看畫的男人那陽物個個粗而且大。又想道。這畫的怎樣大。人身上的却又那樣小。或是男人中竟有這樣大的。心中猜疑不定。數了一數。二十四幅。看他慾火升騰。口乾面赤。牝中如千百條蛆蟲亂拱。麻癢難當。淫水直淋。將一條紅褲一個襠都濕透。本要等童自大來家。且依樣做一齣。散散火氣。看看等到日暮。還不見來。只得叫丫頭拿酒來吃。本借杯中物。一解心內火。誰知這件黃湯。只能助火。不能散火。飮了半晌。將及起更。還不見他的影兒。把牙咬得格支支的響。恨道。這天殺的。遭瘟的。不知又往那裡肏攮黃湯去了。他只圖自己貪口腹。就不顧我的死活。怎麼這時候還不見回來。因叫葵心丫頭。到外邊打聽打聽。看你老爺回來沒有。葵心去了來。說道。老爺下午就回來了。在書房裡關着門。不知做甚麼呢。鐵氏聽了。心疑道。他每日回家就到上房來。今日在外邊做些甚事。也不點燈。叫兩個丫頭扶着\footnote{扶着妙。不但是胖了難走。且又是黑影中也。}。走將出來。看見書房窗子上燈光明亮。遂走到窗下。將舌尖輕輕舔開一個小洞。向內一張。只見童自大兩手扳着美郞的胯骨。正幹得好。抽抽扯扯。垂首看那出入之勢。那小廝又善於逢迎。做出許多嬌態。口中哼個不住。將屁股朝上亂聳。鐵氏見了。那還忍得住。眞是須彌山紅了半邊。一腔忿怒直從脚底板衝將上來。高聲大罵道。沒廉恥的奴才。幹得好事。這一聲。如半空中一個霹靂。把一個童自大驚得一交跌在地。動擔(彈)不得。那小廝聽是奶奶的聲音。也嚇得魂飛魄散。不及穿褲。光着屁股走將出來。慌忙開門要跑。不想鐵氏也正來打門。撞個滿懷。將鐵氏跌了個仰巴叉。那小廝也一交撲在鐵氏身上\footnote{這小廝竟是跌在綿花包上了。}。爬起來好跑。那鐵氏幸得肉厚身肥。不曾跌重。又虧兩個丫頭扶住。所以不致跌傷。只把肥屁股蹾了一下。陰門震得有些微痛。那小廝見撞跌了主母。也顧不得性命。如飛的不知躱到何處去了。那鐵氏口中只是喊罵。兩個丫頭將他扶了回去。這童自大起初已是嚇得發昏。又聽見小廝撞倒了奶奶。越發着急。渾身亂抖。想道。這場禍事不小。萬萬不能饒恕了。把腰間鸞帶解下來。待要自縊。又捨不得這萬貫家資。想了一會。道。罷罷罷。拚着身子與他打罵。大約也還未必就得傷命。醜媳婦免不得要見公婆。料道也躱不脫。上去憑他處治罷。戰戰兢兢。一步一怕。走到了房中。見鐵氏睡在床上叫疼喊痛。忙跪在床面前。用手替他撫摩。道。奶奶。我該死了。一時錯吃了這口屎\footnote{這句話用在此處。恰當至極。}。打罵由你。不要氣壞了身子。原來這鐵氏半日來慾火如焚。吃了幾杯酒。淫興愈熾。滿心等童自大回來瀉瀉火氣。聽見他在書房中。疑他是醉了躱懶。故此出來。有個就敎之意。不想又看見了這一幅活春宮。這一腔火如何嚥得下去。雖然跌了一交。他渾身是肉。如何得痛。故意裝做着重的模樣。好等他來賠罪。自然盡力。可洩這點慾火。有話明日再講。主意打算已定。有一隻黃鶯兒。描寫鐵氏的心事。道。忽念子孫樁\footnote{此三字新鮮。不意此物又得一雅號。}。動淫心往就嘗。誰知他反偷奴臟。情慌意忙。魂迷興狂。蠻抽緊送騷模樣。惱人腸。襄王別戀。枉  自下高唐。鐵氏見童自大跪在面前。他也不動聲色。只假做怒容。道。你原來瞞了我。同小廝幹得這樣事。你自己就說。該個甚麼罪。童自大忙分辯道。奶奶。你看錯了。我何嘗敢做甚麼壞事。我今日在宦哥家吃了幾塊冷西瓜。又呻(呷)了幾杯冷酒。回來小肚子疼得很。疼得我沒奈何了。叫小廝拿熱屁股替我熨熨肚子的。放着奶奶這樣香噴噴的好東西我不受用。反肯去鑽那臭屁股門子。當眞就到這個田地不成\footnote{呆人說巧話。愈覺其呆。妙甚。趣甚。}。鐵氏明知道他們是幹那事。也不認他的犯頭。故意冷笑道。你還說不呆。旣然肚子疼。難道小廝的熱屁股熨得。我的熱小肚子倒熨不得的麼。那童自大以爲這一番進來。不知如何打罵。誰知反是一片好話。眞是夢想不到。得不的一聲便跳起。脫個精光。忙爬上床來。一面說道。我只道奶奶睡着了。不敢驚動。要知道奶奶還沒睡。我早來求奶奶替我熨了。說着。鑽入被中。摸見鐵氏脫得上下一絲皆無。仰臥着。眞像濃濃一床大厚褥子鋪着一般。軟得好不動火。童自大便伏上身。一挺盡根。抽送起來。鐵氏假意道。我不見你的罪就夠了。你還想來做這事。童自大道。我的娘。我這樣分辯。你還疑我麼。我要哄你。把㞠子就爛掉了。說着。將他兩條桶粗的肥腿。儘生平之力扛將起來。架在肩上。要圖得他的歡心免禍。咬着牙。瞪着眼。掙命似的竭力幹訖一度。童自大渾身汗流如水。力盡筋疲。下身來睡在枕上。張着嘴只是喘氣\footnote{夫妻幹此事曰交歡。若皆似童自大如此。則苦之至矣。}。那鐵氏遍身的火熱了這半日。得此一番狂弄。雖不能大暢其欲。覺得比先也洩去許多。便酥酥睡去。一覺醒來。已是半夜。餘興未已。看那童自大時。齁齁鼾睡。忙搖醒了他。問道。你此時的肚子疼好了些沒有。可還要熨熨。我這會的小肚子倒滾熱的呢。童自大每常在他跟前。稍有失意。非打即罵。今日這場大難。輕輕恕過。反討了溫存言語。眞是感恩無地。死亦弗辭。明知他還要弄弄的意思。詭答道。隱隱的有些呢。你若不嫌絮煩。再替我熨熨更好。又掙起來。沒力扛起他的腿。只將肥臀墊得高高的。把吃奶的力氣都使將出來。又做了一番。方纔睡到天明。二人起來。鐵氏本來滿腔醋氣。一則見他夜來慇懃。將功贖罪。二來還要他竭力報効。做那春宮中解數\footnote{這一本春宮。竟是童自大護身符。}。恐一時發將出來。不好收轉。只得捺住。但將美郞說他懶惰無用。交與媒人轉賣。童自大雖然心疼不捨。但自己免了大禍。已是徼天之幸了。可還敢護庇小廝。只得由他〈他〉賣去。這日。宦蕚來請他去說話。也沒有吃酒。老早回來。路上遇見擡着兩口棺材出殯。街上人指指搠搠。紛紛的笑罵。又聽不明白。不知是甚緣故。叫個家人童淸去打聽明白了來回話。童自大先到了家。不多一會。童淸也來了。說。方纔那棺材是阮大鋮老爺家的。小人去訪問。一個人說死的一個是他的愛妾。就是當年白佔去白家的那女旦。一個是他的大兒子。都說是爲奸情死的。却問不出詳細。你道這死的是誰。原來是阮最。因患時疫。病倒在床。延醫吃藥。服了幾劑。毫無效驗。他便不肯再吃。終日昏臥。有時醒來。郟氏同丫頭沒一個在面前照看。天氣旣熱。又是那心裡發燒。一時口渴起來。要杯茶吃。也沒人遞。害得一絲兩氣。叫得聲又不高。叫上幾十聲。總沒一個人答應。等得郟氏同丫頭過來。他怒道。我害着病。你們就〈就〉不着一個守着我。連要杯茶也沒有。都躱在那屋裡作甚麼。難道怕瘟病就過了你們麼。郟氏也不答不睬。次日仍復如是。阮最心中動疑。却也猜料不出。過了幾日。覺得身上略好些。隱隱聽得西屋有人聲嘻笑。又聽不明白。他掙了起來。走不得。拿過一根窗戶栓拄着。慢慢的挪出房來。見西屋門關着。悄悄到窗外。往裡一張。只見郟氏仰臥在一張醉翁椅上。愛奴赤着身子大弄。丫頭在後面推搡。阮最氣得昏了過去。一交跌倒在地。他三人正做到處。忽聽得窗外一聲響。愛奴忙拔出。走到窗前。向外一張。原來是主人公睡在地下。嚇得心驚膽戰。向郟氏說了。郟氏也心中着忙。連忙穿了衣服出來。將阮最扶起。擡到屋裡床上。撅救了一會。纔醒轉來。怒說道。你們做得好事。等我好了起來替你算帳。郟氏也放下臉來。道。算甚麼帳。我不過是個死。還要拉兩個伴兒呢。我偷小子該死。那奸庶母的。同偷兒子淫婦。難道又饒得過麼。大家將就啞打些罷。不要自搬磚自磕脚。那時纔悔遲了呢。阮最聽了這話。聲也不嘖。只嘆了兩口氣\footnote{郟氏私愛奴。若阮最不知。還報應得不爽快。使他親視而不敢言。才是眞報應。}。從此病又反重。郟氏索性竟不過來。日夜都在西屋裡。只叫丫頭在這邊照看他。阮最也無法奈何他。惟有暗恨而已。又睡了月餘。纔下得床來。他秉氣原弱。又病了這兩場。害得懨懨一息。此時八月中旬。餘暑末消。他睡得昏頭昏腦。只得掙將起來。扶拐而行。過了兩日。覺得頭目略淸爽了些。身子還飄飄的。偶然心中想念嬌嬌。一來久疏濶了。二來郟氏的事。這一口暗氣在心。無人可說。要想去吿訴他。慢慢的一步一步走到他那裡看看。這一去。正是。並非去看舊相知。却是來尋催命鬼。那嬌嬌每常阮大鋮父子三人日夜供他一人之樂。猶未愜意。這一次阮大鋮往京裡去了。許久未回。阮優又隨去了。阮最又病倒。他這一個奇騷極淫的陰戶。空閒了許多日子。十數年來從沒有的缺典。眞捱一夜似三秋。度五更如兩憂。這一日心有所思。其實難忍。竟有些要死的樣子。走到房門口來。癡癡的望。望了一會。不見人影。心中猶如火熾一般。十分難耐。正將一隻手縮在衣內。將門前後戶不住摳挖。口中咨嗟嘆息。忽然見阮最走了來。如從天上降下一位救命王來了\footnote{不是救命王。却是送命王。}。忙伸出手來。兩手捧住\footnote{捧字奇。不知如何捧法。}。同到房內。叫賽紅在外邊看着。忙拴上房門。把阮最摟在懷中\footnote{男女鍾情。兩相恩愛。皆男抱女於懷。此反是嬌嬌摟住阮最。乃淫之極。非情之深也。此等處皆要留心看出。方見作者之妙。}。坐在床沿上。說了無限相思的話。一會兒含含他的腮。咬咬他的頸。又吐舌到他口中。互相吮咂。一會臉兒廝偎。口兒相接。忍不得了。便伸手到他褲襠中去捏捏陽物。眞是但不知那些兒纔爲好\footnote{把一個無恥的騷浪淫婦。寫入到骨髓。}。做盡嬌模嬌樣\footnote{不負名叫嬌嬌。}。騷態百出。意思要替他起起病。自己也要醫醫病\footnote{四百四病中。倒不知這一種騷病如何醫。}。阮最鑒貌辨色。見他騷得可憐。那一種淫浪之態。又令人可愛。自己也因病久。虛火甚熾。陽物也就鐵硬。二人脫光上床。幹了一次。阮最伏在他肚皮上喘息了一會。將郟氏愛奴的事吿訴了他。嬌嬌道。事已至此。你也不消氣惱。你又身子不好。只做不知不見就罷了。你想。我同你這樣厚。你爹不知道也就罷了。我同你爹不過是個名色。一心一意倒同你是夫妻一般。你再要不然。竟把你娘子撇開。任他去罷。有我和你相守着。可不好麼\footnote{雖是边(勸)阮最。却全是利己的心腸。妙甚。}。阮最道。你說的是。我此後把這淫婦當死了的罷了\footnote{孰不知你竟先淫婦死了。足見世間事焉能預料。}。兩人雖說着話。陽物未曾拔出。嬌嬌興致正濃。那顧他的死活。又見他陽物還硬着。用兩足勾住他兩條腿。兩手扳住他屁股不放。自己陰戶不住疊着往上就。阮最病弱了的人。先那一下。業已頭腦轟轟的響。眼睛內金蒼蠅亂飛。但陽物虛火把住了。還十分脹硬。又見他這個騷極了的樣子。心裡過不去。只得又掙着命同弄。阮最喘噓噓。雖費盡了力氣。嬌嬌只覺他的勁小。將他兩股用力往下搬。自己的屁股不住往上迎。口裡連聲叫道。好親親。好心肝。你下狠些。又弄了好一大會。嬌嬌方滿心快暢。正在得意之時。覺得阮最的陽物在陰中跳個不住。知他是又洩了。只見他身子平伏了下來。垂着頭。閉着眼。動也不動。嬌嬌急看他時。已脫陽死了\footnote{病人雖然醫好。把個醫生倒死了。眞可笑。昔日曾有四句道。隱婆生子收生處。醫士醫人死病家。更有一般堪笑處。捕官被盜叫爺爺。不意應在他兩人。}。吃這一驚非小。忙把他推下身來。摸摸口中。一絲氣也沒有。此時他的陽物雖軟。渾身倒都硬了起來\footnote{趣語。要知渾身硬不如此物硬。}。自知陰中他洩的陽精淌了大灘。嚇得沒法了。左思右想。無計可施。只得穿上衣服。滴了幾點淚。拿了條汗巾。拴在欄杆上。將頭套入\footnote{可是先說的。我和你相守着。可不好麼。此時却遂了心了。}。有四句打油說道。

\begin{quotation}

淫亂還須有肺肝。緣何苦苦只偷奸。

今看懸索悲啼際。應悔多貪一晌歡。

\end{quotation}

過了許久。賽紅在外觀風\footnote{今後此差免矣。}。等了半日。總不見動靜。疑是他二人弄乏了睡着。恐有人來撞見。走來推門。要叫他們。門是揷着的。推不開。叫了幾聲。也不見答應。走向窗洞中一張。見嬌嬌吊在床欄杆上。慌得跑了出來。喊聲救人。那阮大鋮的正妻毛氏聽見。問他叫甚麼。丫頭道。我姨娘上吊呢。毛氏暗暗歡喜。你道何故。這毛氏少時生有幾分俏麗。在家做女兒時就毛手毛脚的。不待父母之命。煤妁之言。竟自己暗嫁了他的表兄韓繼壽。他父母也有些知覺。恐醜聲敗露。意思也就要將他二人配合。不想韓繼壽得個怔忡病死了。沒有把破女兒留在家一世的。後來恰遇阮家來求親。就嫁了與阮大鋮。成親之夕。阮大鋮知他這件鮮品是被人嘗過新的了。要退他回去。毛氏再三跪着哀求。乞存臉面。只求占這一個正室的虛名。要娶妾置婢。悉聽尊意。不敢稍忤。阮大鋮因岳家也是科甲世族\footnote{此等人家偏多生此等子女。不知何故。請他父母將胸中一摸便知。}。送回去彼此無光。又圖他賠的粧奩豐富。也便留下。先也是把他虛設着的。總不沾身。後來尋了幾個妾婢。顏色皆不如他。想起他做女兒便會自己嫁人。定有一種風騷可取。又從新同他親熱起來。不意他騷淫得十分有趣。枕蓆之間。那一種極淫浪的妓女。也沒有他這一段騷致。阮大鋮素有騷淫之性。今遇騷淫之人。棄其貞而取其騷。頗自相得。十數年來。只他生了阮最阮優兩個。別的婢妾皆無所出。後因得了嬌嬌。不但美過於他。且年又少艾。騷淫更勝。此時毛氏也四十多歲。騷淫雖勝當日。無奈面孔減了許多丯韻。就把他打在贅字號聽提去了。毛氏雖不敢明明吃醋。這隱恨在心十有餘年。今聽得他自己上吊。巴不得死了。眞是。

\begin{quotation}

拔去眼前釘。挑却肉中刺。

\end{quotation}

猶恐去快了又救活轉來\footnote{誅心之言。}。故意慢條斯理。遲了一會。纔叫人下邊去叫僕婦們上來\footnote{毛氏這是決西江之水救涸轍之鮒。一個派頭。}。進房去解救\footnote{寫盡妒婦心腸。許多工夫。只算得去解放。却非是解救。}。衆婦女到了那裡。見門栓着。打開窗戶進去。見床上精赤條條還有一個。仔細看時。原來是大相公。忙去報知奶奶。毛氏正在那裡私心竊喜。想嬌嬌這一死了。阮大鋮必定還來同他尋舊好。用手摸着陰戶。笑道。你熬淡了多年。將來又要開葷。有肉吃了呢。忽聽見兒子也死了。一面哭着\footnote{可謂先笑後唬(號)咷。}。如飛的走來一看。見兒子精光着死在床上。褥子上許多遺精。就知他是把兒子弄死了。然後急上吊。撫屍痛哭。郟氏知道了。也乾嚎着跑了來\footnote{有聲無淚曰嚎。寫淫婦心腸。是個淫婦妙筆。}。此時一家婦女都到\footnote{此一句揷入。妙甚。後賽紅說他二人妙事。故知之者衆也。}。大家動手替阮最穿了衣服。嬌嬌已解下來。了(久)矣斷氣身亡。郟氏假哭着丈夫。還伸手到嬌嬌褲襠中。把陰門擰幾下。以忬向來之恨\footnote{趣甚。向來二字妙極。郟氏非恨其此時弄死丈夫。恨其向來占去丈夫此物耳。}。毛氏把賽紅細細拷問。賽紅把嬌嬌同他弟兄兩個怎樣通奸。起先是母女吃醋。後是兄弟爭鋒。怎樣和好了。一個弄前一個弄後。又怎樣背着。把歷來他三人所作的妙技。都詳細說了出來。衆婦女聽得無不掩耳唾笑。毛氏纔知〈知〉他二人是久交。今日做了同生同死的厚友。又聽見連小兒子也有奸情。恐阮大鋮回來。倒難爲阮優。再三囑咐衆人隱瞞。連這兩個都說是瘟病死的。這丫頭留着到底恐有洩露。忙忙叫人領去賣了。放了數日。阮大鋮在京。値魏璫事壞。父子抱頭鼠竄。星夜逃回。又見愛妾長子雙亡。嚇了一跳。因在有事之秋。自己身家性命還不知如何。也顧不得查問他二人如何死的。只大哭了一場。也不開喪出弔。救(就)叫人擡出去埋葬了。倒是阮優哭得傷心。也不敢明哭嬌嬌。借哥哥的屍靈。哭心上的人。聽見花氏吿訴他說嬌嬌是吊死的。越發傷心。一日兩三場哭。飮日(食)不思。眼也哭腫了。喉也哭啞了。別人看他。好個愛長兄的悌弟。那知他是個想庶母的孝兒。毛氏雖叫人瞞。家中僕婦人嘴衆多。三人口闊一尺。如何瞞得住。早已哄傳里巷。剛剛只瞞得阮大鋮一人不知。所以出棺這一日。街上人指撅(搠)笑罵。就是這個緣故。童自大叫家人去問。人如何好詳細相吿。只說奸情而已。童自大聽了。也不在意。恐鐵氏昨晚之事未能盡釋。忙忙走到上房。鐵氏道。你今日往那裡去的。來得這樣甚早。童自大討好道。宦哥打發人來請說話。我往他家去的。因心裡掛着奶奶。酒飯都沒有吃。就趕忙回來了。鐵氏因想起昨日的春宮圖。取出來向他道。這東西是那裡的呢。童自大一見。嚇得〔面〕容失色。答應不出。掙了一會。道。這是大舅姆娘家火大哥的家譜。我借來看的\footnote{天地間有此等家譜。令人可笑。}。鐵氏笑着道。不要胡說了。他家二十四代都是做這事的麼\footnote{鐵氏呆矣。因爲做這事。才得流二十四代。若沒這事。一傳而絕矣。}。況他家也是敎門。你看這男女的那上頭都畫着有毛。如何瞞得我\footnote{眞是老見家。}。我不怪你。只問你那裡得來的這樣好東西。不拿來我看。放在外邊做甚麼。童自大見他毫無怒色。倣(放)了膽。順他口氣答道。是我在一個鋪子裡看見畫的有些趣。借了來。要送與你看。恐怕你惱。故不敢拿進來。鐵氏將一張三四寸濶的蟠桃口咧到耳根傍。笑道。你眞是個呆子。這樣稀奇的好東西。我看了爲甚麼發惱。但恐那鋪子裡來要。怎麼處。他不知可肯賣。要賣。買了他的也罷了。童自大道。我是扯謊哄你。怕你嗔說拿銀子買這東西。我是買了來的。你若愛。只管長遠留着。鐵氏喜道。這却好。我想你怎麼越發呆了。拿銀子買這樣好寶貝。我怎肯嗔你。不強似當日買〈麼〉監生麼。你想想。這東西有多少用。你買了那一張監生的紙來。放了這幾年。可有一點用處麼。他坐在涼床上。叫童自大坐在他懷中\footnote{叫童自大坐在他懷中。妙。他身子胖大。若坐在童自大懷中。不但童自大禁不得。且如一堵照壁遮住。看不見矣。此等細處。非心細如髮。如何看出。}。將春宮放在桌上。二人細細同看。指指點點。說其中妙處。那鐵氏看得勃然興動。放細了喉嚨。做嬌聲問道\footnote{東施後身。}。你得了這書\footnote{此而謂之曰書。是個蠢婆娘說話。}。也曾同人做過這個樣子麼。童自大道。我除你之外。婦女們連看還不敢看他一眼。就滿心要試。叫我同誰去做。鐵氏將他脖子咬了一下\footnote{騷極。}。笑道。難道定要同別人試。我不是婦人。就做不得的不成\footnote{鐵氏是讀過毛遂傳。}。童自大此時坐在他腿上。如靠了一個大厚椅。背墊了一個錦(綿)軟坐褥。已經興發。又見他乜斜着雙眼。溫溫柔柔。每常見那凶暴之氣。一點俱無\footnote{不意一本春宮。不但能使鐵氏變化氣質。而且能陶養他性情。}。從不曾經此光景。遂道。怕你不肯。我巴不得呢。趁此時就試試罷。就替他寬衣解帶。鐵氏並不推阻。由他脫下。童自大也自脫了。拉下床虎丘蓆。鋪在地板上。兩人坐下。童自大把那春宮本頭一張翻開。問鐵氏道。就照這一張做罷。他點頭依允。再一看時。是一個順水推舟之勢。婦人仰臥。兩足大蹺。男子竭力前聳。童自大扶着鐵氏睡倒。他竟一見便悟。就蹺起腿來。牝戶大張。紅鈎赤露。他回子家女人。陰毛是常常要拔淨了的。他牝戶上並無一毛。光滑滑。鼓蓬蓬。如發酵催粧的大饅頭一般。有幾句俗語贊他這個物件。道。

\begin{quotation}

一雙豎眼。竟與世人相似。只有眼而無珠。一張直嘴。却與衆人不同。但有嘴而無舌。紫〈紫〉威威一個心子。像沒牙口含着一個葡萄。紅通通兩片肥皮。似痘風眼生了兩塊努肉。揸開時。如饞人張口等佳饌。合攏後。像餓漢閉嘴吞冷氣。人人知道是件利害東西。個個都當稀奇寶貝。

\end{quotation}

鐵氏他面貌雖醜。這件肥牝戶却令人十分可愛。童自大見了。麈柄突然而興。心中愛極了。拿手攥了攥那個肥物。一手還攥不過來。然後對了陰門。一揷無餘。是因纔看春宮時已濕透了。且童具小而鐵孔大。故此順溜。童自大抽起來。不多一會。那鐵氏腿粗肉重。不能常蹺。要放在他的肩上。童自大肩膀昨晚被他兩足壓了半夜。幾乎骨拆。此時如何還禁得起。又不敢違拗。只得假說道。旣要學樣子。須要依他。纔做得有趣。畫上兩條腿是蹺着的。你若放在肩上。就不像了。做來也沒興頭。鐵氏道。我的腿蹺得酸疼。怎麼處。童自大想了想。道。你旣然蹺不得。叫丫頭來替你扶着罷。鐵氏不肯。道。靑眉白眼。叫兩個丫頭看着。是個甚麼樣子。童自大着急道。不然不做這個樣子。樣子再換一個別的罷。鐵氏正在興頭上。又懶得起來。急得沒法了。只得道。也罷。你叫了丫頭來罷。童自大便叫葵心蓮瓣。誰知那丫頭相貌雖醜。淫心一般。見主人主母白日交鋒。正躱在窗外偷看。聽見叫他。走到跟前。童自大叫他二人坐在兩傍。每人將鐵氏一隻腿扛在脖子上。然後大張旗鼓。直攮紅心。兩個丫頭見主人公同主母的兩件東西合而爲一。但每常的水手篙子是向水中穿。主人公是將篙子在船腹中穿。像是把舟穿漏了。推得那舟中之水。順着舵眼不住長流。連篙攢都揷不住。幾乎滑了出來。儘力推了一會。水手力也乏了。篙子頭也使軟了。方纔歇手\footnote{就以順手推舟四字寫此一段淫事。趣甚。}。兩個丫頭看得他那牝中流出的水。比奶奶穿出來的還多。每人屁股底下。不但衣褲。連蓆子也濕了一大塊。見船穿到了岸了。放下了腿。忙忙走出。每人喝了一大碗涼水。那臉上的紅。心內熱。還不曾澆了下去。童自大與鐵氏也不穿褲。只披了上衣。吃了戰飯。飮了幾杯助興的酒。到床上又演第二齣去了。他二人上床。脫了衣。鐵氏怕他躱懶。向他道。我們不必挨次去做。隨手揭出一張就照着樣兒。定要做得入神。我做得不像。罰我一兩銀子做東道請你。你做得不用力。罰銀一兩請我。童自大道。我可敢不依你。只是你做得不像又不肯罰。我敢把你怎麼的。鐵氏道。說過的話。我若如此失信。你後來還肯聽服我麼。童自大道。旣這樣說。你就自己去揭。省得我揭了出來。又說是我揀的。叫你疑惑。鐵氏笑道。你這話說得也有理。就伸手揭開一張。看時是個倒澆蠟燭的勢子。男子仰臥。婦人騎在上邊動作。童自大道。奶奶。這張來不得。我那裡馱得動你。鐵氏道。你頭一張就躱滑。後來還想我依你麼。童自大聽了這話。怕他動怒。只得仰睡着。鐵氏也跨了上去。就套上坐下。兩手拄定。蹾了幾蹾。他身子沈重。不由得氣喘噓噓。便伏了下來。壓在身上。童自大忙叫道。奶奶。來不得。看壓斷我的腸子。你再壓壓。我就一塊豆腐乾兒了。鐵氏笑道。原說要做得像。就壓扁了。也顧你不得。童自大忙忙將兩手用力托住了他的胸脯。鐵氏又蹾了幾蹾。自己也甚覺費力。方纔下來。童自大喘了好一會。纔說得出話來。道。夠了我的了。吃了這一個大苦。我看你揭出別的來。依我不依。鐵氏也歇了有一盞茶時。喘息已定。又揭了一張。二人齊看。是一個婦人伏在枕上。屁股蹶着。男子在背上。厥物頂入肛門。婦人在下。一手揉着花心。是一幅後庭花的故事。鐵氏看見。方要另揭。童自大按住。道。你方纔自己說得牙淸口白。不許撒賴。如何換得。鐵氏道。這一張原不算的。童自大道。旣是不算的。起先何不早說。你又是看過的。這會兒揭了出來。如何換得。童自大生平來昨晚纔初嘗美郞的這種妙趣。忽被驚散。未得快暢。今日巧巧的揭着這一張。正要盡一盡昨晚未盡之興。那裡肯依他換。又見鐵氏和顏悅色。咧着一張大嘴只是笑。他便撒嬌撒癡。倒在他懷中滾。道\footnote{一個滾字。寫得呆人活跳。}。你自己的令。如何賴得。不拘怎樣。給我嘗嘗纔罷。你方纔幾乎壓死了我。你怎不換。纏了許多時候。鐵氏也因自己說的話悔不得。沒奈何。問他道。這件事從沒有做過。不知疼不疼。你昨晚與美郞弄事。必定知道。童自大道。我熨肚子的。何嘗弄屁股來。你只是寃賴我。鐵氏道。你少要說鬼話。我看得很明白。你同他弄的。如今人也賣了。我又不惱。你說與我好做商量。童自大聽得此話。量出眞情。遂答道。我起先原是熨肚子來。後來那東西不知不覺就自己鑽了進去。連我也不知道。鐵氏道。不要胡說。我見他蹶着屁股。往上一迎一送的。嘴裡哼喞喞。難道這裡頭也快活不成。童自大道。必定是快活有趣。若是疼。他怎麼裝出那個模樣來。鐵氏道。你一起手弄時。他可曾說疼呢。童自大道。我唾沫也不曾用一些。只輕輕一聳。就進去了。他也沒有說疼。並不見他做聲。鐵氏道。要是這樣說。這事也還做得。你多多的用些唾沫搽搽㞠子。再放些在屁股眼內。須要慢慢的。不許冒失。童自大聽說。喜歡得一骨碌爬起。忙道。我知道。不勞你吩咐。包管你一些也不疼。我難道就呆到這樣地位。連屁股都不會弄麼。鐵氏也想試試這件妙事。就學畫圖。伏在枕頭上。高聳着肥臀。童自大把龜頭搽了許多的唾沫。又將他糞門上也抹了些。然後捏住陽物。對準肛門。往裡一頂。突的一聲。就將進去了一個頭子。又兩三抵。已全身皆入。童自大滿心歡喜。說道。你怕我不在行呢。你摸摸。這不全弄進去了。你道鐵氏是個未經弄過的後庭。如何這等容易。因他股大溝深。腸肥油厚。不知不覺便弄了進去。也只算得一半。那一半被臀肉隔住。所以不覺得艱難。童自大雖然弄了進去。尚恐他疼。還不敢十分動作。鐵氏先也覺膽怯。只當不知如何痛苦。以爲這個去處原是天生與人出糞的孔竅。井非納腎的東西。那知如今的小夥子們拿他做了納腎的正門。反做了出糞的餘洞。鐵氏見弄了入去。並不覺其痛楚。只微微有些脹意。用手一摸。已進了大半。想着美郞那種光景。必然還有妙處。向童自大道。你動動看。童自大便抽抽扯扯弄了一會。扯出許多丫油。甚是滑溜。鐵氏覺得裡面酸酸的。有些佳境。回顧童自大道。你再快些重些。童自大知他已安。遂兩手扳住胯骨。用力抽扯。口中哼哼的道。好肥東西。我吃了一輩子肥肝板腸。也沒有這樣的好滋味。一陣亂搗。搗得那鐵氏酸癢難當。哼個不住。把肥股一拱一拱的往上迎送。童自大見他已得樂趣。自首至尾。加力扯拽了數百。那丫油滴了一褥子。鐵氏哼成了一塊。後庭中爽利不消說。牝戶中也一陣麻癢起來。陰精溢出。覺比每常交媾還更有趣。不由的伸了手去揉着花心。不期然而然。做得與畫中十分相似。童自大情興如火。怡然感之。一洩自注。扳開肥股。盡抵至根。樂不可言。鐵氏亦舉股承受其精。盤桓了半日半夜\footnote{好精神。}。皆身體困倦。拭抹乾淨。共枕而臥。一覺醒來。童自大初嘗珍味。覺得異常肥美。意思還想要領敎領敎。摸着他的後庭。說道。奶奶。我這一回越發在行了。你給我弄弄。鐵氏道。這不過是偶然做做。若只管走起旱路來。把我這條河道壅塞了不成。童自大道。奶奶。我有句話。你不要惱。鐵氏道。我不惱。你有話只管說。童自大道。不瞞你說。你身子胖大。底下的那件寶貝雖是肥得出奇。只是又深又厚。又寬又大。我的這件東西有限。弄進去。摸不着一個邊岸。就像小孩子走到一個大城門裡站着。那裡見個影兒\footnote{蠢才譬得過大太小。幸爾鐵氏不怒。若謂如和尚站在關中則可矣。}。就是你容易也不得爽利。倒是這後門裡緊揪揪。弄得你也好。我也好。兩好並一好。可不好麼。鐵氏聽了。想他這話倒也眞。故意道。你說雖然有理。若只弄後邊。前頭就棄了。叫他長遠把齋不成。童自大〈道〉想了一會。笑道。我有個妙法兒。包你都不脫空。鐵氏道。是甚麼妙法。童自大道。你此時且同我弄了着。我到晚上來同你試法。鐵氏道。你哄我要弄罷了。那裡有甚麼妙法兒。難道你又生出個㞠了來不成。童自大道。我可敢哄你。若不如意。也罰我一兩東道。說着。就扳過鐵氏的屁股來。鐵氏此時也正有些餘興未息。就將屁股拱在他懷中。那後門內還有餘瀝。童自大也不用唾。就勢一頂而入。兩人又翻騰了一場方罷。次日。童自大起來。想道。我看奶奶的那件東西實在有些怕人子。靠着我這個匪物。想圖他歡喜。是再沒用的。我常看見那角先生。得一個大大的來送他取樂。纔可以換得他的後庭。但不知在那裡賣。吃罷早飯。走了出來。問那家人童祿道。你可知道賣角先生的鋪子在那裡。童祿道。郭先生的鋪子倒知道。他敎着二三十個學生。就在這大街口上。我家的當鋪隔壁\footnote{應前童自大說先生敎學生詩處。細。}。倒沒有聽見他賣不賣。家裡又沒有小相公。老爺要買他敎學麼。就是敎學。雇他也罷了。又買他做甚麼。童自大笑道。蠢才。我問你的是那牛角做的角先生。好好的。問那郭先生做甚麼。童祿道。哦。那個麼。在承恩寺斜對過魆黑的那一條廊底下有幾十家賣他。老爺到那裡要幾擔也有。老爺要買得多。小的跟了去挑。也饒他幾個來頑頑。童自大聽了。又好氣又好笑。罵道。蠢才。屄養的\footnote{主人亦未見其乖。}。那東西要幾擔做甚麼。想留着傳代麼。他袖了個銀包。也不帶人。自己步到廊下。走入時。香氣竄腦。到一家鋪內。見擺列着無數。童自大揀了一個比他陽物粗長些的。那開鋪的道。尊駕買了作何用。童自大不好說買了送他夫人。扯謊道。要同人頑戲做酒杯。要知這件東西是件冷貨。做他的多。買他的少。不過是發賣與過路客人。見他說買了吃酒。巴不得總成他多買幾個。說道。要嫖婊子頑耍。一個就罷了。旣是要做罰酒杯了。大大小小多買幾個纔有趣。將一個頂大的拿過來。道。這個原做了是吃酒頑耍的。婦人中那裡用得〔這〕樣大物。又取過一個至小的。道。這留給量窄的人吃。童自大想道。據我看起來。這個大的或者竟用得呢。若買了這個二號的去。要不中用。豈不白走一回。索性都買了去罷。問道。你這三件要幾個錢\footnote{眞是財主口角。錢這樣貴重。}。那人聽他問這話。心中忖道。原來是個大利巴\footnote{江南土話。謂人不在行曰利巴。}。我且烹他一烹。便道。買這樣東西是論不得價的。只在尊意。若遇了出手的大老官。甚麼十兩五兩。萬不然照本錢二兩銀子是一分少不得的了。童自大從不曾買過。不知價値。又不好爭講。他平素極吝。此時竟慷慨起來。說道。銀子便依你二兩。有甚麼好春方。送我些做搭頭。那人這三個先生値不過三五錢銀子。因見他是外行。故拿大價哄他。誰知他一口就依了。滿心暗喜。說道。旣承照顧。只是難爲了小鋪些。就取過一根白綾帶子。有五六寸長。中一段裝着藥。說道。行房時將這帶子束在根下。比每常分外堅久粗硬。一根可用五七次。尊駕若試驗果好。下次還求照顧。拿一張綿紙。同那角先生包在一處。童自大打開銀包。稱了二兩足紋給他\footnote{竟不是送魏如豹那一種銀子了。}。拿了回來收着。晚間聽用。那鐵氏素常與童自大交媾。也覺得他的物件放在內中如太倉一粟\footnote{較小孩子在(站)在城門洞裡更不堪。}。沒有甚趣。只因慾心火攻來。沒奈何。叫他殺火。間或也乏。這是他情急了。雖不能暢其慾心。到底有個男子在肚子上爬爬動動。興之所至。也就乏了。這個只弄得他自己乏。井非是童自大本事弄丢了的。昨晚嘗着這後庭中滋味。悔道。早知這穴道中有這樣樂處。何不棄前而取後。況且後邊得了樂趣。前面也有許多妙景。攻其一而兩得其樂。何樂不爲。又聽見童自大說兩不脫空的話。猜測不出。料他又未必是說謊。滿心巴到天晚等他來如何試法。只不見黑。急得如熱鏊子上螞蟻相似。走投沒路。等到日落。忙忙同童自大吃了晚飯。又飮了幾杯助興的酒。然後上床脫衣。童自大將白綾帶子束在陽物根下。把三個先生放在枕邊。鐵氏道。你說兩不脫空。是怎麼樣的。要是說謊。罰出銀子來與我。童自大笑嘻嘻。將那個頭號角先生拿出來。在眼中一晃。道。你看看這件寶貝。就藏在背後。鐵氏只見眼前一亮。不曾看明。笑道。是甚麼寶貝。怎與我看看又藏起來。童自大遞與他。道。是這麼一根降魔杵。我請了這個先生到你肥館來坐坐。如何。鐵氏認不得是甚麼東西。只見光亮亮的。有一個西江月贊他的形狀。

\begin{quotation}

腹肉(內)空空無物。頭間禿禿無巾。遍身華美亮錚錚。腰較富翁還硬。一個光頭釋子。假名冒做先生。端詳注目看分明。可喜粗長且勁。

\end{quotation}

鐵氏接過來一看。原來是一個八寸餘長。鍾口粗細的陽物。上面還有些浪裡梅花。他心中又喜又怕。笑成一堆。道。這樣棒槌大的東西。只怕放不進去。童自大道。還有一個副先生。一個學長呢。先拿了試試看。又將那兩個取過來遞與鐵氏。鐵氏看時。一個有五寸來長。一圍稍大。一個長只三寸。也不堪(甚)粗。問道。這樣好東西。你那裡得了這幾個。童自大道。是我特買來送你的。做謝禮的。補報你昨日屁股的情。鐵氏笑道。你竟比當日在行了好些。這樣好東西就會自己去買了。像這等好物件。就多破費些銀錢也不枉。自拿着那個小的。道。這個太小。只好送黃花女兒。我這裡頭只好在傍邊做楔子。正經處用他不着。這個大的又太大些。不是兒戲的。這二號的比你的粗大些。且拿他試試着。童自大坐在傍邊。把他腿抱起一隻。將那第二號的物件往陰門裡一塞。略重了些。竟像個老鼠見了洞。一鑽就不見了。竟全身塞了進去。那鐵氏尚自不覺。問道。你說試。怎又不放進去。童自大笑道。你摸摸看。全身鑽進去了。鐵氏伸手來摸。果然都在內中。笑着說道。這樣看起來。那個大的恐怕也還用得。你也試他一試。童自大伸了指頭在他牝中。把那沒用的副先生拉了出來。把那頂號的拿將過來。鐵氏道。這個大的利害。比不得先那一個。你須慢慢的來。童自大也不敢冒失。將那大光腦袋在牝戶門口㨪了幾㨪。有些濕了。方往裡一送。喞的一下。進有二寸。鐵氏每常與童自大弄時。弄了半月(日)。還不知進去不曾。此時被這件粗物。覺得陰門撐得有些脹意。囑道。有些意思。你慢慢的送。童自大拿着巨物一進一出。不多幾送。也就一絲不剩。童自大見了。暗暗的吐舌。道。這樣個大物件。還輕輕巧巧送了入去。可憐我這個匪物。每常不知分量。還想討他個歡喜。豈不是癡。此時鐵氏這一件寬兮綽兮的肥物。可也被那先生塞了個毫無罅〖阝少日小〗。鐵氏甚覺有趣。一面笑着。一面用手指着牝戶。道。這先生雖然魁偉壯大。渾身又華麗光鮮\footnote{這先生在今日必定大行。}。只是死板得很。一些活動氣兒也沒有。怎麼樣處。童自大道。等我同你把後面的筍安上了再講。鐵氏正要看他如何作用。聽說。急忙爬起來。要蹶着屁股與他弄。不想一翻身。突的一聲。那先生見東家略動動身。他就逃出館來\footnote{這怪不得先生。東家先說他死板來。}。鐵氏道。這怎麼處。就了你。這個又掉了出來。就了他。你又弄不得。如何纔得兩不脫空。童自大〔道。〕不是這個弄法。你還仰睡着。須憑我擺佈。方纔如意。鐵氏忙應道。任你怎麼樣。我都依你。童自大叫他仰臥了頭。睡得平平的。把枕頭取下來。將屁股墊得老高。恐他腰下空着。拿來被替他塞好。鐵氏道。要這麼高做甚麼。童自大道。奶奶。你身子胖。屁股溝子深得很。我的東西只弄得一半進去。你我都不自在。這樣一墊高了。把兩條腿揸得開開的。屁股溝子淺了好些。便不礙事。方弄得盡根。鐵氏也就依他。將兩腿大揸着。道。我的腿長蹺不得。怎麼處。童自大道。還是叫丫頭來扛着罷。鐵氏點了點頭。童自大叫兩個丫頭來。說道。你都脫光了上來。穿着衣服礙事。那丫頭不敢不脫。脫了上身衣服。纔要上床。童自大叫連褲子都脫去了。他醜自醜。到底是女孩家。有些子作難。鐵氏望了一眼。道。唔。你不理麼。兩個丫頭嚇得打了個冷戰。慌忙脫下。紅着臉微笑。一隻手遮着牝戶。精光着上得床來。童自大叫他還像昨日將奶奶的腿每人挓(抱)了一條。大大的分開。因墊得高了。那肥股竟是仰着朝上。溝都平了。毫無阻礙。童自大滿心歡喜。將脚帶兩條接了一條。把那角先生根下拴了個結實。繫在腰間。笑對鐵氏道。你昨日笑話我還長得出一個㞠子來。這不又長出一個來了。鐵氏見他上下兩個硬邦邦的東西。喜歡的笑得眼睛只剩一條細縫\footnote{是個胖人的臉。}。童自大方要動手。見兩個丫頭光着身子。雖然面目不佳。也還白白淨淨的皮肉。小小的奶兒。圓圓的肚兒。還有那一條細細的縫兒。也甚動人。那童自大看上呆興來。忽然哈哈的呆笑起來。道。你兩個沾沾奶奶的福。也不要脫空。一手拿起那個五寸來長的角先生。把葵心一下按倒。將他的腿〖扌扉〗開一隻。吐上一口唾。搽在他陰門上。狠狠往裡一塞。竟自塞了個頭子進去。塞得那丫頭哎喲連聲。又被他使蠻。兩三下塞個盡根。那丫頭雖有二十多歲。因家主婆利害。不曾吃過墅(野)食。被他這樣幾下。塞得痛苦難禁。幸得年紀大了。雖然受得住。還疼得兩淚汪汪\footnote{這眞是惡取笑。}。童自大笑着拉他起來。道。憑他在裡頭。不許掉出。你穩穩的坐住。將他夾緊。要是掉了出來。我叫奶奶打你五十鞭。那丫頭雖則怯疼。料比奶奶打的還好捱些。也就依他坐住。猶恐掉了出來。動也不敢動一動。童自大又拿起那個小的。對着那蓮瓣道。也來試驗試驗。那丫頭不肯。童自大發威來。道。小騷奴。好意給你嘗嘗新。你倒做出這樣個浪兒來。那丫頭只得將腿蹺起。他對眞也是一塞。一來這丫頭也十七八歲了。二來那先生渺乎小爾。並不覺其煩難。便塞了入去。也叫他照樣坐緊\footnote{角先生。婦人或有用之者。若處女以之破身。大約自此二婢始。}。再看鐵氏時。牝戶大張。將有一掌。那兩邊的肥肉因騷極了。就像劃開鼻子馬一般。吸呼吸呼的亂動\footnote{妙想奇譬。}。他將腰中那先生送入鐵氏牝中。有四句口號道。

\begin{quotation}

非緣設帳請先生。只爲夫人物可驚。

今日相延肥館內。西賓便可喚卿卿\footnote{先生大得便宜。}。

\end{quotation}

他自己的厥物頂進後庭之內。童自大笑向鐵氏道。看這個樣子。我想起一副對子來。鐵氏發急道。忙忙的且說甚麼對子。難道當眞是先生坐館麼。童自大道。我聽見人念後門口的對子。道是。前門增百福。後戶納千祥。我改幾個字。今日就合着你了。這是前門撐巨物。後戶揷纖陽。可好不好。說着大笑。抽動起來時。那鐵氏等了許久。又見他同兩個丫頭做作這一會了。正騷興大發。見他兩件物事一齊進內。只覺其樂。欣欣得意。弄夠多時。那陰中之水。肛內之油。兩處齊流。將白綾帶的藥性泡發。那陽物脹得分外粗大。其熱如火。鐵氏前門中塞得脹滿。已美不可言。後門又滾熱的這件硬物出出進進。樂得他聲喚都叫不出來。童自大見他這妙景。又得藥性助着。也分外用力。乒乒乓乓。弄得那響聲如數十條鰍行泥淖中相似。鐵氏口中只噯呀噯呀響。別無他語。兩個丫頭起初也覺得裡面塞緊。又疼又脹。悶得慌。甚不好過。到此時見了這番光景。那小肚子裡竟熱烘烘。一陣一陣的流出水來。却並不知疼。只覺其癢。也就不知不覺起起坐坐。扭扭晃晃。那先生在裡邊雖不能十分活動。也覺得在內中挨皮擦內。竟甚是有趣。他二人亂扭亂蹾。那鐵氏的腿是他兩人抱着。他的身子動。那主母的腿自然是要動的了。他二人把屁股往上一擡。那鐵氏的身子往上一迎。他二人向下一坐。主母之臀也往下一落。他二人扭。主母的身子也扭。他二人晃晃。主母的身子也晃晃。那鐵氏已經樂極。又得這兩個幫襯着\footnote{他兩個非幫閒。乃是幫忙。}。眞是說不出來的妙處。他二人原是幫襯自〈自〉己的。不想無意中倒幫襯了主母。做了一對大功臣。有一調黃鶯兒道他幾人的樂處。

\begin{quotation}

前後一齊攻。腿高擡。興致濃。肥軀竭力相迎送。童陽力舂。鐵陰快鬆。牝津股液如泉湧。喜融融。丫頭起坐。樂亦在其中。

\end{quotation}

這一番舉動眞是驚人。自點燈上床。直到二鼓。方纔歇手。童自大與鐵氏之樂自不必言。這兩個丫頭雖不曾嘗金莖玉露。如自幼吃胎齋的人。忽然嘗着了些葷味。也覺可口。他二人將牝中之物也不繳還主人。竟自取了出來。拿在手中。抱着衣褲跳下床。笑嘻嘻的走去。兩人輪着效法主人同主母的法則去了。鐵氏因那小東西也不要他。故不尋問。一宿晚景休題。次日。童自大不在家中。鐵氏飯後獨坐自思。人說見識見識。不見不識。果然不錯。我只說男女趕(幹)事。不過是爬在肚子上這樣弄了。誰知昨日見了這本畫兒。纔知有這些樣數。學做了一兩樣。果然有趣。我又當是天下人的物大小都差不多。每常我也疑心我的物這等寬大。他的這樣細小。昨日見了這個奇物。雖說是假的。必定也有這樣大東西。人纔照樣做出來。況且弄了進去一般恰好。可見是不曾見識的緣故。床頭間將那角先生取出。坐在春凳上細看了一番。又撫摸了一會。又量量。又箍箍。越看越愛。不忍釋手。又在抽屜內將那春宮取出來看。看一幅便閉着眼睛摹擬那神情光景。看了一會。困倦上來。叫丫頭拿過枕頭來枕着。就在春凳上睡着了。這兩個丫頭昨夜覺得也有些趣味。正要想去試試。恐主母叫。今見他睡着。二人輕輕將那春宮悄悄拿過來。看了幾頁。動起興來。這葵心就伸手到小丫頭褲襠內一摸。見水〖氵韲〗〖氵韲〗的。就拿指頭替他摳。那蓮瓣也伸手過來替他挖。又看了兩幅。都摳挖得有些不自在起來。把册頁仍舊放在主母面前。他二人拉着手往後邊去了。鐵氏睡了一會。偶然失手。把那先生掉在地下。猛然驚醒。他素常起身。因胖狠了。好生的費力。此時一個翻身。比瘦怯人還伶便。一骨碌爬起。忙向地下拾起來。連啐了幾口。道。怎麼就害了瞌睡癆。把他就掉了下去。若跌壞了。怎處。忙細端相。毫無傷損。纔放了心。還恐怕他跌得疼一般。又揉摸了一會\footnote{形容得甚趣。}。拿了一條湖縐汗巾包好。拿出一個錦糊的扇子匣來裝了。放在枕傍。以便不時取用。一時口渴要茶吃。叫了幾聲丫頭。不見答應。只說他們去偷睡。遂起身到彼邊來。聽見屋裡哼哼喞喞聲喚。驚道。難道是他回了。在這裡偷丫頭麼。悄悄一張。原來兩個丫頭學主人主母的樣子呢。葵心仰臥着。兩腿揸得開開的。蓮瓣坐在傍邊。抱着他一條腿。一隻手拿着那中等先生。在那裡一進一出的搗。是葵心口裡哼。那鐵氏忍不住笑道。小淫婦們也會這樣作怪\footnote{只許大淫婦作怪耶。}。那蓮瓣聽了主母聲音。連慌把那個角先生往葵心的花心裡一揷。起身跳下床來。忘記了他那蓮花瓣中也有個小先生在裡頭。喞的一聲。像燈節放賽月明似的。冒了老遠。那葵心也一翻身。纔要爬起。他那葵花心內的先生。也是喞的一聲冒了出來。他二人嘻嘻的笑。連鐵氏也笑得東倒西歪。回房中來。心中有些興動。況昨日那些光景。也是兩個丫頭見過的。何必怕他。見他兩個在跟前。叫他關上了門。上床脫光。叫丫頭也脫了上床。還像昨日。一個人抱了一隻腿。各伸出一隻左右手。拿着大小兩個角先生。前門用大的。後戶用小。弄將起來。用手拿着更覺有趣。比童自大拴在腰中弄法更好。要深就深。要淺就淺。要高就高。要下就下。慿自家心中所愛。只許(須)一言。丫頭自然奉命。把他二人的手腕幾乎累折。那鐵氏也幾乎樂殺。興盡而止。自此以後。把這兩個丫頭倒像活寶一般疼愛。興之所至。就叫他二人來殺火。把童自大倒似有如無。他弄也罷。不弄也罷。不似當日拘管。把那前番非打即罵的樣子全盡蠲除。那童自大見他放鬆了。也竟公然躱了身子。偷空同兩個丫頭弄聳。那丫頭的模樣雖醜。較主母還妖嬈些。且這兩件妙物緊而且嫩。童自大得意是不消說的。就是兩個丫頭也甚戀家主這根皮裹純筋的傢伙。比那光骨頭的先生有趣些\footnote{嗟乎。腹內空空之先生。不及一純筋之陽物乎。雖罵得刻毒。却罵得甚當。}。況且那個二號的。主母又收去爲後庭之用。只剩個小物。太覺不堪。所以遇便就與主人公沾在一處。鐵氏就是看見。只做不知。一來念童自大薦賢自代之功。二來時常要這兩個丫頭兩手維持之力。因此愛心一萌。威不復作矣。他這一家從此倒和氣許多。鐵氏的咆哮竟化爲烏有。此皆童自大請先生之力。正是。

\begin{quotation}

欲消妒婦淫和悍。須請先生大又長。

\end{quotation}

再說阮最的妻子郟氏。也是個頭號騷淫之物。阮最在日。因不曾領敎過他的妙處。反嫌他死相。不會風流。別戀着嬌嬌。撇得他冷淸了。他便風流起來。也就偷上那愛奴小廝。只好暗地風流。却不敢放肆。偶然遇便偷弄一下。一來做得隱秀。二來這小子是阮最心愛的。故不疑他。所以不曾露出馬脚。十數年來。這小子已長成一條大漢。專一酗酒肆惡。阮最念向日之情。每每護庇他。自阮最病中害得七死八活。郟氏膽子就放了些。時常在西屋裡同他做那一件樂事。後雖被阮最看見。他不久又死了。郟氏更無忌憚。一個月中竟有十數夜暗暗叫他進房陪宿。雖然愛奴的陽具不甚雄壯。却身強有力。頗得郟氏歡心。這郟氏因向年丈夫說他不活。他後來看見嬌嬌那些態度了。也學得扭頭捏頸。抿嘴咬唇。未語先笑。渾身顫巍巍動個不住。就像年下賣的鬧攘攘一般。走動兩邊搖晃。好似一個美人燈。一風都吹得倒的勢子。風騷得異常。也不像個寡婦。每日描眉畫眼。嘻嘻哈哈。那種浪態。令人看得好不肉麻\footnote{古云。楚王愛高髻。宮中高一尺。此可謂。阮最愛風流。郟氏騷得極。}。竟連阮大鋮一個老漢而兼公公之人都看上火來。想算計他的那一點風流孔竅。雖不好驟然下手。但見了面由不得就做起光來。那郟氏是個伶俐婦人。也就看破了幾分。這阮優也久矣看上了嫂子。當日因哥哥在。不敢放膽。今哥哥已歿。他就想學起陳平來。\endnotemark[1]見嫂子纔三十多歲。妖妖嬈嬈。活狐狸精相似。好不風騷。魂魄都被他攝去。間或打牙撩嘴調戲他兩句。他也似推似就。如送如迎\footnote{想起阮最調戲嬌嬌之日。想到今日阮優調戲郟氏否。}。要想法弄他一弄。但不定他的心腸。恐怕不從。一時喊叫起來。怎麼處。近日風言風語。聽得說他老子在郟氏屋裡。大白日關着門。不知做甚麼要緊的事。好一會纔出來。他心中暗想道。嫂子旣然肯偷公公。不是甚麼貞節的了。況我豈不比老子少壯些\footnote{可謂跨竈之子。}。他可有不愛我之理\footnote{此理不知出自何典。}。遂日日在郟氏房中走撞。坐着說閒話。偶沒人。就說句把風流話兒勾引他。那郟氏也不惱也不答。只抿着嘴笑笑。或斜瞅一眼。一日。阮優笑着向郟氏道。我昨日聽見人唱一個劈破玉兒。很有趣。我唱給嫂子聽聽。遂唱道。

\begin{quotation}

小寡婦上新墳。身穿着重孝。拿着香。提着紙。直哭到荒郊。見新墳。忙下拜。把我親夫來叫。實指望與你同偕老。誰知你半路裡把奴抛。我捱不得這冷冷淸淸也。夫君呵我要去偷小叔了。

\end{quotation}

郟氏瞅了一眼。笑笑不做聲\footnote{笑者不可測也。}。阮優笑道。當日這種道理我就不明白。譬如這嫂子。總是別人家的女兒。旣嫁得哥哥。就嫁得兄弟。何必分甚麼叔嫂。何必竟像男人一般。娶了姐妹兩個。一個做妻。一個做妾。這女人嫁丈夫。倘那家有弟兄兩個。何不把哥哥做了妻。兄弟做了妾。那些兒不好。嫂子你說我想的可是。郟氏笑道。嚼舌根的。你的嬸子明日就是這樣。阮優笑道。我要兄弟。早叫他嫁了\footnote{不用急。雖沒有兄弟。却有愛奴。}。那裡像嫂子這樣古板。郟氏也不答他。只是笑。阮優道。昨日見人新編的小寡婦鬧五更的銀紐絲兒。作得果好。我唱給嫂子解悶。唱道。

\begin{quotation}

一更裡思夫。過黃也麼昏。思量年少俊卿卿。好傷心。緣何撇我赴幽冥。奴身獨自苦。帶影共三人。想親夫。眞個心腸硬。空房孤守。誤我靑春。痛斷肝腸。淚珠也傾。我的夫哪我恨卿卿。又把卿卿恨。

二更裡思夫。月上也麼堦。當初指望永和諧。淚盈腮。撇奴獨自好難捱。羅衾空半幅。繡枕半邊歪。淚珠兒濕透了香羅帶。翻來覆去好傷懷。痛的夭亡。我命也乖。我的夫哪我帶孤辰。命把孤辰帶。

三更裡思夫。月正也麼明。猛然夢裡遇親親。放悲聲。懷中摟抱訴衷情。離愁腸萬結。未語淚先傾。正綢繆。忽被鐘聲震。醒來仍自擁孤衾。桌上的殘燈。乍暗也明。我的夫哪我傷情。眞個傷情悶。

四更裡思夫。月轉也麼西。翻身側耳聽啼雞。好孤悽。羅幃寒氣逼香肌。他人鸞鳳合。我獨子規啼。悶殺了奴。受這孤單罪。思量轉痛轉傷悲。就是那螻蟻。也效于飛。我的夫哪我爲誰來。却把誰來爲。

五更裡思夫。天色也麼明。無眠整夜斷人魂。恨夫君。爲伊苦守也無因。貞節雖有。難輪到我身。倒不如轉嫁圖歡慶。那時節。攜手赴鴛衾。被底的風流。樂殺也人。我的夫哪恨憑君。憑君恨。

淸江引

五更已罷天將曉。日上三竿了。對鏡理容妝。嘆我靑春小。細尋思。還去做新人好\footnote{阮最調嬌嬌也以戲文。阮優調郟氏也以小曲。雖是前後遙遙一對。內隱報應之理如影隨形也。}。

\end{quotation}

郟氏聽他唱得旣好。又打動了心事。長嘆了一口氣。復笑道。我知道〔這〕個曲子就是你這個爛心的編的。笑着惡毒毒瞅了一眼。阮優見有幾分光景了。就思量要做實在事了。心中想道。我那一日溜到他屋裡躱着。等他睡熟了。然後下手。倘偷上了。或者他嘗着了滋味。不致變臉。古人說。色膽如天。要不放大了膽子去做。等到那一日纔得成就。主意拿定。時時刻刻在郟氏房中來撞幾會。一晚。天黑了。他到郟氏房中來。不見人影。他心生一計。閃入床後一個僻靜處蹲着。等他回來下手。原來郟氏被大鋮請了去。到他一個妾房中。做些不三不四的雅事。那個妾只圖主公歡喜。那管他公公媳婦該弄不該弄。還在外邊聽梆聲。替他觀風。你道這件事是如何成就了的。那阮大鋮素心只貪淫。自嬌嬌死後。無可行樂之人。毛氏雖騷淫可取。但吾(五)旬外的老嫗。有年紀了。陰戶如掉了牙的癟嘴一般。兩片寬皮。鬈毛佈滿。不但不可用。而且不可觀。只可以作老伴。不可以共歡樂了。雖有幾個妾。只平平然。又都不甚出色。一時高興。忽然注意到郟氏身上。想道。他〔少〕年寡居。未必不思快樂。看他的姿色。可與嬌嬌相匹。論他的身段舉動。比嬌嬌還騷浪些。可以計擒之。娛我老景。想了一番。他有一個妾是揚州人。原是個瘦馬買來的。他就姓馬。行六。叫做馬六姐。阮大鋮同他戲耍。常叫他做馬泊六。他比衆妾乖巧些。又識字。又會些彈唱。當初嬌嬌在日。阮大鋮就獨鍾愛他些。嬌嬌死後。自然數他是第一個愛寵了。一日。阮大鋮到他房中。坐在椅子上。摟他在懷內膝蓋上坐着。笑對他道。我有一件事。你若替我謀成了。倘不洩露。我同嬌嬌當日一般待你。倘或你奶奶要死了。我就立你爲正\footnote{馬氏當云。等到那一日。虛情不敢領。}。你可肯替我做麼。馬氏笑道。老爺這話就奇了。我的骨頭肉都是老爺的\footnote{骨頭肉雖是老爺的。恐那片皮要屬苟雄了。}。叫我死。敢不死麼。要叫我做甚麼。我還敢不盡心。我也沒福想做奶奶。只要老爺另眼一眼就夠了。阮大鋮摟過他脖子。親了個嘴。他就連忙送過舌頭尖來。阮大鋮咂了一下。道。且說正經話。着馬氏縮了回去\footnote{掃眉(興)之甚。}。阮大鋮笑着附耳朶說道。大媳婦着實風騷。我心中十分愛他。你想個法兒弄到你房中來。我同他了了心願。你心下如何。馬氏笑道。老爺不說到這裡。我也不敢說。老爺若勾搭上了大娘子。也只算得個眼前報應。阮大鋮驚道。這是怎麼說。馬氏道。當初嬌姨在日。同大相公厚了多年。只老爺一位不知道。後來兩個人還是同死的。奶奶叫瞞着老爺。所以沒人敢說。阮大鋮細問緣故。他把嬌嬌如何將阮最弄死。急了上吊的話。細說一番。阮大鋮聽了。說道。旣如此說。越發放他不過了。馬氏想了一會。道。明日如此如此。老爺打點精神做新郞就是了\footnote{應前嬌嬌向阮優道。他會哄漢多着呢。至此。果前言不謬。}。阮大鋮大喜。被他說上興來。同他弄了一度。以當起媒。到書房中養精蓄銳。以待明日大舉。却說次日早飯後。馬氏到郟氏房中來。見郟氏靠着桌子悶坐。手托香腮。心中不知想甚麼呢。原來郟氏性雖淫濫。當日有阮最在。不敢放肆。偷上了愛奴。也就心滿意足了。近見丈夫已死。沒了管頭。便放大了膽。又見愛奴不似當日小心慇懃。甚不適意。時常見阮優到屋裡來撞。不但愛他精壯。想起嬌嬌阮最死後。那丫頭說嬌嬌怎樣愛他陽大力強。又怎樣在行會弄。滿心要勾搭他。又回想。我先下手。未免爲他所輕。須等他來動手方可。却只見他嘻皮笑臉。言語勾搭。並不見他動作。是甚緣故。又想。他那個樣子。決放不過我。不過稍遲日子。但只是就同他偷上了。到底怕人口聲。吹入公婆耳內。不是兒戲的事。我見公公這些時見了我。眉目中那光景。像有些古怪。看他雖說不出口。也像是愛上我臍下的這件東西。他雖老不濟事。要是同他勾上了。連婆婆也不敢多管。這一家還怕誰來。那時可以任我行事。但我做媳婦的。怎好去調戲公公。怎樣纔得諧這一件美事。心中左思右想。正想得火發。忽看見馬氏走來。連忙站起讓坐。馬氏道。大奶奶做甚麼呢。郟氏道。不曾做甚麼。閒着無事。在這裡打盹。馬氏笑道。我看大奶奶今日紅光滿面。像有甚麼喜事一般。郟氏嘆了口氣。微笑道。一個寡婦。有甚麼喜事到我。馬氏也笑道。天地間的事。那裡定得。焉知今日你就沒喜事。郟氏笑道。我看你喜氣洋洋的。倒像昨夜得了甚麼喜事一般。馬氏笑道。我守着老爺。這是常事。那裡算得喜。像你寡婦着遇了這巧宗。纔算得喜呢。兩人笑了一會。馬氏道。大奶奶旣然悶倦。到我那裡走走。說說閒話。我還有一件〈事〉好東西給你看看消悶去。郟氏笑道。怕老爺到你房裡來。有甚麼要緊的喜事。我在那裡。恐躭誤了你的。你好抱怨我。馬氏笑道。一個親公公媳婦。我就有甚麼事。也不消避得。你就在傍邊看看解悶也好。要看上興來。你也就做一齣。又不是外人。兩人又笑了一番。那馬氏立起。拉着郟氏的手。對那丫頭道。你好好的看家。我同大奶奶走走來。二人攜手出門。同到馬氏房中坐下。閒話了一會。漸漸說到那村淫房闈之事。又笑道。大奶奶。虧你這樣少女嫩婦的熬得。要叫我。就要急死了。郟氏笑道。少沒廉恥罷。說着不害牙硶。你方纔說有甚麼好東西給我看看散悶呢。拿出來我看。馬氏笑道。有有。遂將阮大鋮所蓄的春宮手卷册頁拿出來他細細賞鑒。內中一幅一個老兒同一個少婦幹事。馬氏笑道。這兩個像是公公同媳婦爬灰的樣子。你看這個老兒畫得活像老爺。這個婦人活像你。這個畫畫的人也奇。怎把你兩個的行樂圖先就畫出來了。郟氏笑着將他擰了一把。笑了一會。兩人又看了多時。馬氏一幅幅指點着說內中的妙處。要引動他的春心。看得那郟氏面上火攻上來。紅一陣。白一陣。不住嘻嘻的笑。看完了。馬氏叫丫頭拿上果碟來擺下。並鮮甜久窨下的酒斟了一杯。送給郟氏。道。你看了那寶貝。火上來了。吃一杯澆澆心火。馬氏因受了阮大鋮之托。叫人去尋了這陳封缸酒來讓郟氏。那郟氏不會吃酒。推醉不飮。馬氏道。這酒像蜜水一般。是不醉人的。大奶奶不信。你嘗一口看。郟氏嘗了嘗。果然甚甜。被馬氏苦勸。吃了有三四鍾。又坐了一會。覺得頭目發暈。四肢癱軟起來。說道。不好。我醉了。頭眼發迷。身子獨軟了。我回去罷。站起身要走。却晃晃蕩蕩。把持不住。馬氏忙扶住他。笑道。你那裡是醉。這是少年人寡居久了。這些時沒人殺火。方纔又看了那件有趣的寶貝。不覺慾火上攻。除非得個趣人兒洩洩火就好了。郟氏也笑道。我聽見說老爺叫你做馬泊六。就替我去尋一個來。馬氏笑着道。這在我。要謝媒的呢。郟氏笑着要走。馬氏道。你旣然頭暈。且在我床上睡睡着。郟氏道。恐怕老爺來呢。馬氏道。他今日出門去了。大約還未必回來。你只管放心睡。扶他到床上。說道。你穿着衣服睡不安穩。把上蓋寬了罷。那郟氏矇矇矓矓。任他將上衣脫了。只穿一衫一褲。替他將被蓋上。不多一會。見他已經睡沈。忙叫丫頭到書房裡去。快請了老爺來。丫頭去了。他將被揭開。輕輕把郟氏褲帶解了。褶褲帶也解下。將褲子褪了下來。打一看時。好一個豐滿的物件。稀稀幾根毛。用指頭探探。又堅又暖。那郟氏如死人一般。總不知覺。馬氏笑着仍把被蓋上。且說那阮大鋮知道馬氏將媳婦誆到房中去了。在書房專等好消息。急得滿地亂轉。他服了一丸如意丹。此時藥性又發。陽物脹得好不難過。正拿手捏攥。忽見丫頭來請。笑吟吟忙走上來。那馬氏一把拉着他的手。掀開被。指着郟氏的陰戶。道。這樣個好寶貝。總成你受用。看你怎樣謝我。阮大鋮歡喜如狂。摟過馬氏的脖子。親了嘴。道。你且看着門去。我自然厚報你這馬泊六。馬氏笑道。你只管放心。但要你拿出老手段來弄。我替你觀風去。千萬不要到門就沒謝帖。出了醜。我就不管了。說着。笑了出去。帶上了門。阮大鋮忙上床。脫光了。爬上身。輕輕分開兩腿。送了入去。覺得比嬌嬌的緊暖許多。心中更樂。興致愈豪。仗着藥力。捨着老命盡力舂搗起來。那郟氏雖不會吃酒。但吃得不多。偶然一時發迷。睡一會已覺好些。被他這一陣狂弄。心中覺得十分爽快。睜開眼來。見是公公在腹上高興。雖合了他先想的心事。固然暗喜。但良心難昧。媳婦的肚皮上爲公公高據。未免滿面含羞\footnote{未免兩字不得活。這羞還在有無之間。}。反把眼閉上。粉頸略略扭着\footnote{此浪也。非羞也。}。阮大鋮見這個嬌態。更覺魂消。心愛得要死。伏下身子來親了個嘴。附着耳道。我的乖兒。你害甚麼羞。一來我憐你靑春孤另。二來阮最那奴才當日撇了你同嬌嬌相厚。我近來纔知道。我同你也厚起來。正好替你出氣報仇\footnote{報仇二字奇絕。丈夫偷庶母。拿公公的陽物來報仇。眞是奇事。}。一面說。一面又深抽淺送起來。那郟氏心中想道。事已如此。還羞甚麼。把他的心勾住了。纔好長久行事。心旣邪了。便由不得將兩手勾住了他的腰。兩足也漸漸舉起。阮大鋮見這光景。連命都顧不得了。弄夠多時。動不得了。方纔歇手。二人並枕而臥。阮大鋮摟着他。道。我此後一得空。常叫馬氏來請你。你就來。郟氏道。恐怕人知道了。不好意思的。阮大鋮笑道。笑罵由他笑罵。樂事且同幹之\footnote{笑罵由他笑罵六個字。死後便可做他的墓誌。}。做得隱密。也不妨事。郟氏要起來。道。我去罷。怕有人來撞見。阮大鋮猶依依不捨。還抱着親了幾個嘴。要他伸過舌頭來。郟氏微笑不肯。大鋮嘴對嘴道。親親兒。弄都弄了。這怕甚麼。郟氏徉羞帶笑。將舌尖吐出些須。阮大鋮咂了幾下。把手拍着他脊心。道。我的兒。我這幾根老骨頭要送在你身上。又把雙乳咂了咂。纔放了他起來。二人穿衣下床。阮大鋮來開門。那馬氏笑嘻嘻向郟氏道恭喜。我替你尋了殺火的乖兒。你拿甚麼謝我。那郟氏紅着臉\footnote{紅着臉。寫得入神。雖極淫浪之人。纔同公公如此。見人豈無羞色。}。笑着瞅了一眼。道。壞人\footnote{妙極。多說不得。不說不得。只此二字足矣。}。便往外走。馬氏叫丫頭送他去了\footnote{細。}。過了幾日。阮大鋮又叫馬氏約了他來高興一番。如此多次。人總不知。郟氏把他十數年未曾發洩出來的技倆。全全施展。較之嬌嬌。騷淫雖不相上下。而柔媚過之。毛氏則卑卑不足數矣。郟氏這是。

\begin{quotation}

酒逢知己飮。詩向會人吟。

\end{quotation}

阮大鋮疼這媳婦眞不啻活寶。好頭面衣服。瞞着毛氏。無樣不給。每日吩咐廚上。收拾上好飮食供給。又怕人動疑。向毛氏道。媳婦靑年守寡。替我家爭氣。理該分外待他。那郟氏見公公疼愛溫存。比阮最當日勝過十分。也自輸心貼意。一日。又在馬氏房中作樂。阮大鋮道。在這裡固然好。未免馬氏在外面礙眼。我還罷了。你到底心裡不得暢快。又不〈不〉敢脫光了。恐一時有人來穿不及。我想要到你屋裡去。纔得放心快活。只因你那丫頭在跟前。瞞不得他。恐他口嘴不好。倘或傳開了。雖然不怕甚麼。到底沒趣。想不出個妙法兒來。怎麼處。你可有甚麼好主意。郟氏道。我也是這樣想。除非把丫頭你也弄上了。纔得安穩。阮大鋮把他摟得緊緊的。道。我也想過這個法子。恐怕你多心。不好說得。旣然如此。你明日打發他來。我自有法。這樣這樣的行。郟氏應允。到了次日。阮大鋮在馬氏房中睡午覺。馬氏知他們的計。避到毛氏上邊去。丫頭也帶了同往。那郟氏在房中看那日色。知到了相約的時候。叫丫頭道。你往馬姨娘房中。有我昨日要的花樣兒。去取了來。那丫頭去了。到了馬氏堂屋裡。叫了一聲姨娘。不見答應。伸頭往屋裡一張。阮大鋮故意問。是誰。丫頭道。是誰。阮大鋮道。你來。一個人也不在跟前。你把我的夜壺拿了來。那丫頭到窗外拿了夜壺到床前。阮大鋮不曾穿褲。將陽物拿着。向他道。套上。我溺尿。那丫頭又不敢走。要送來。〈來〉又有些羞愧\footnote{羞惡之心。人皆有之。這丫頭還知有些羞愧。何阮大鋮之無恥至此也。}。阮大鋮笑道。怕甚麼。還不拿過來呢。那丫頭只得將壺嘴替他套上陽物。把臉扭着\footnote{四字入神活跳。}。阮大鋮溺完了。道。就放在床底下罷。那丫頭纔彎腰放下。阮大鋮見他蹶着屁股。伸手去抄後一掏。那丫頭忙立起身來。被他雙手抱到床上。就扯褲子。那丫頭見是老主如此。可敢違拗。況他被阮最愛奴弄過多次。知道此事有妙處。任憑褪下。阮大鋮還當他是個處子。用了些津唾抹了龜頭。往裡一頂。竟熟滑無比。一攮到根。阮大鋮笑問他道。你這丫頭好大膽。我當你還是個女孩兒。原來是個破罐子。同誰偷弄來。實吿訴我。我不怪你。那丫頭只是笑。不做聲。阮大鋮再三追問。他不得已。不敢說出愛奴。只道是〔當〕日大相公破身的。那阮大鋮也以爲實然。遂不再問。只苟且了事而已。那丫頭道。我去罷。恐怕奶奶問。阮大鋮道。不妨。我還有話問你。你大奶奶這樣少年守寡。他也想人弄麼。那丫頭道。噯喲。這也是你公公口裡說的話\footnote{此語出自他人之口。不足爲異。出於無知此〇〇〇小婢。則阮大鋮眞禽獸不若矣。}。阮大鋮笑道。呆丫頭。婦人家那個是不想弄的。說頑話何妨。丫頭道。他就想弄。他也不肯吿訴我。我如何知道。阮大鋮道。你只看他間或日間坐着長歔短嘆。夜裡翻來覆去睡不穩。那就是春心動了。丫頭道。這倒有些。阮大鋮道。我倒愛他得很。你幾時拿說話兒勾他。他要同我弄上了。我重重的賞你。丫頭道。你老人家不害羞。一個媳婦也想弄他。阮大鋮親了他個嘴。道。呆奴。人說肥水不落外人田。我的媳婦我不弄。設或他騷將起來。同外人混弄。如何管得他。你只留心。我明日先賞你幾件好衣服簪棒。後來還配你一個好漢子。丫頭道。一時他惱了打起來。你顧不得我。阮大鋮道。不妨事。你只管上心去做。床頭間摸了一錠銀子與他。道。這賞你買果子吃。那丫頭喜孜孜接了。道。多謝老爺賞。身邊無處收放。就拴在褲帶頭上。阮大鋮笑道。你若做成了。還有重賞呢。那丫頭穿上褲子。笑嘻嘻去了。回到房中。郟氏忽然怒道。你爲甚去了這半日。丫頭道。姨娘不在屋裡。我等了這一會。還不見來。怕奶奶望我。纔來回話。郟氏道。你還瞞我。你頭髮都亂鬔鬔的。同誰頑去來。你可實說。我不打你。那丫頭死說沒有。那郟氏是心照的。就把他衣裳一掀。那丫頭不曾防備。被他掀開。見他褲帶頭拴着一錠銀子。故意驚怒道。了不得。你原來做賊去來。是那裡偷來的。快快實說。不然活活打死。那丫頭白瞪着兩眼。無言可答。郟氏取了一根窗子栓。狠狠要打\footnote{此栓不知可是阮最拄了去看郟氏私愛奴者。今日又爲郟氏嚇丫頭之用。欲私公公。阮最泉下若有知。亦悔當日烝淫庶母一着錯否。}。那丫頭急了。方說。是我纔上去。老爺賞我的。郟氏道。我不信。老爺爲甚麼賞你。逼之再三。方說。老爺拉我睡覺。纔賞我的。郟氏道。還同你說些甚麼。丫頭道。沒有說別的。郟氏道。我倒不打你。你還不實說。那丫頭也有些乖巧。見郟氏雖說要打。却不甚怒。這丫頭當日被小主人弄過了無數。偶有小過。尚不免捶楚。只有威而無恩。愛奴更弄得多。要一根糖吃還不肯。今蒙老主一幸之愛。就與銀子。又許衣服簪棒。感恩不盡。想起老主相托的話。暗道。我顧不得。竟實說了。看他怎樣。遂道。老爺問我。奶奶可想人弄。我答應不知道。就把阮大鋮的話細細說上。郟氏道。我就不信老爺有這話。定是你謅說的。你去請了老爺來對。若眞就罷。若是說謊。我了不得。那丫頭道。我去請老爺。奶奶只管對。忙忙又走上來。阮大鋮同丫頭弄了一度。乏了。正然睡着。那丫頭見沒人。掀開帳子。推醒了。道。你害我奶奶要打我呢。叫我來請老爺去對話。千萬不要害我打。阮大鋮滿心歡喜。穿褲着衣。悄悄的同那丫頭到郟氏房中來。郟氏迎着讓了坐下。他笑着道\footnote{這一個笑字。謂譎計已遂。心中暗喜。不覺形於外也。眞寫得好。}。方纔這丫頭說了許多的話。果是老爺叫他說的麼。要是說謊。我要打他。阮大鋮道。與他不相干。是我說的。不要難爲他。望着門。向那丫頭把嘴一努。那丫頭也懂局。徉徜出去。把門帶上。阮大鋮摟着郟氏。親了個嘴。道。你好妙計\footnote{不勞過獎。此計郟氏行之熟矣。}。兩人相攜同到床上。脫得精光。放了心痛樂。相摟相抱。睡到日暮方散。次日。阮大鋮果然悄悄賞了那丫頭幾件紬絹衣服。並數根簪棒。囑道。人若問你。只說奶奶賞你的。那丫頭歡喜得了不得。阮大鋮又摟住。問他道。我弄的比你大相公當日如何。那丫頭笑嘻嘻不答。阮大鋮再三問他。他道。老爺這東西雖同相公差不多。却沒有他的硬實。阮大〔鋮〕聽了這話。恐郟氏嫌其罷軟。各處尋覓好春方。欲供他之淫樂。那丫頭得了衣飾。拿與郟氏看。郟氏叫他收起留着穿。從此後也分外待他親厚。那丫頭感恩不盡。巴不得他二人時常大弄。以做報恩的一件事。或無人處見了阮大鋮。便道。我奶奶在屋裡閒着呢。老爺何不頑頑去。或見了郟氏閒坐。便道。奶奶悶得慌。我去請老爺來罷。如此者多次。那阮大鋮雖到了郟氏房中。恐太走勤了。被人看破。也還常約了郟氏到馬氏房中作樂。却叫丫頭隨着。以免人疑。一日。中伏天氣。郟氏午間洗了個澡。上床去睡。丫頭也接着水洗了。正在堂屋坐着舂盹\footnote{坐着舂盹。寫丫頭眞是個丫頭。}。這日。大鋮正得了些好春方。要來同郟氏試驗。悄悄的進來。見那丫頭打呼。把他鼻子一捏。他驚醒來。見是老主人。忙站起身。笑道。我倒是沒有罵呢。阮大鋮摟過來親個嘴。道。小油嘴。低聲道。你奶奶呢。丫頭道。纔洗了澡睡覺呢。阮大鋮纔轉身。那丫頭道。老爺請回來。我有話對老爺說。阮大鋮笑着回過來。道。你說甚麼。丫頭嘻嘻的道。不說甚麼。阮大鋮道。小奴才也哄我。我知道你是急了。要弄弄的意思。也罷了。我救救你。遂同他在椅子上略略見意。要留精神去對付郟氏\footnote{寫這丫頭一段何故。見人持身不可不正。阮大鋮若無禽獸之行。淫兒婦至及此婢。此婢焉敢戲弄老主。又見小人女子近之則不遜之意。}。走到房中。揭開紗帳。見郟氏上下一絲也無。面朝裡臥。如一個玉人。懷中抱着個竹夫人。一條腿跨在上邊。睡得正濃。不覺淫心驟起。把衫褲脫了。低頭向股下細看。陰戶之妙不可形容。微張一〖阝少日小〗。略吐花心。那肛門通紅的皺摺密簇。想道。這件美物。我雖閱歷甚多。但美人之物。却未曾嘗。大約又自不同。向日嬌嬌我多次要弄。他定然不肯\footnote{嬌嬌之淫濫可謂至極。無以復加矣。其後庭肯與阮最阮優而不肯與阮大鋮弄者。亦猶李夫人臨終不肯見漢武帝。留個有餘不盡之意耳。}。今趁他睡着。這機會不可錯過。且試他一試。吐出許多唾液。將郟氏糞門輕輕潤了。又向裡挖挖。緊緊的有趣。又將自己陽物搽得透濕。然後摸着關竅。往裡一頂。竟進了一個龜頭。那郟氏一驚醒來。回頭見是他\footnote{回頭。妙。是在後弄也。見是他。更妙。或疑是愛奴。}。說道。這是甚麼頑法。弄得我生疼的。還不拿出來呢\footnote{阮大鋮當云。當日阮最那奴才把嬌嬌的屁股不知弄過多少。我今日替嬌嬌報仇。}。阮大鋮緊緊抱住。道。我的親親。我活老了。從不曾弄過美人的這件妙物。我方纔細看。你比別人的更妙。你容多弄一下。我就死也甘心了。說着。又往裡送了送。那郟氏也不覺十分艱難。想要買公公的歡心。且他也是個淫物。也圖嘗嘗這味比前面何如。倒把屁股往外就了就。笑道。捨你這老花子弄罷。阮大鋮如獲至寶。雙手扳着。狠命弄了一番。精洩之後還不肯拔出來。趁那滑滑的勢兒。又緊抽一陣。郟氏也覺大有妙處。極力迎送。將屁股往他懷中亂拱。多時方歇。拽出那話。郟氏在褥子底下掏出塊陳媽媽來。同拭淨了。對面摟着睡下\footnote{虧他不怕熱。才洗了澡。又是一身汗。}。阮大鋮道。親親。你原來有這麼個好寶貝。比前面的更妙。連親了幾個嘴。道。這是我老運亨通。享用你這兩件妙物。郟氏笑道。你這老沒廉恥的。一個媳婦的前後門都被你鑽起來\footnote{你這小沒廉恥的。一個媳婦的前後門都給公公鑽起來。}。還說甚珍珠寶貝的。阮大鋮笑道。我同你還是甚麼公公媳婦。是前世的寃家。今生相遇一處。只好除死方休\footnote{孰不知是同令郞死。}。阮大鋮說上興來。又道。先在背後弄得不得力。不大受用。我捨老命同你弄個快活的。那郟氏也更樂從。阮大鋮叫他仰臥。將股墊高。兩足大分。叫他用手扳住。合上肚皮。對準後門。就着先洩的餘津。兩送到根。極力抽提。響聲不絕。郟氏覺得比先次更加快活。叫道。你狠狠快快的。哎喲。我過不得了。將股亂疊。阮大鋮也竭力大弄了一場。纔興足而歇。自此以後。那郟氏是個淫蕩之物。覺得後面也各得其妙。但與阮大鋮交合。定叫他留一半工夫在後路頑耍。阮大鋮也正投所好。竭力以博他的歡喜。古語說。若要人不知。除非己莫爲。來往多次。也就有人知道。但阮大鋮係一家之主。誰敢多管。微有風聲吹到阮優耳內。故此也就想下手。他這日郟氏因去孝敬公公。故此房中無人。阮優在床後等了好一會。郟氏同丫頭月光下回來了。此時房中月色映得大亮。也不點燈。時已夜靜。就脫衣而寢。阮優聽得他在床上翻翻覆覆了一會。不見動靜。微有鼾聲。知是睡熟。他輕輕走出來。到了床前。脫光了上床來。掀開帳子。一見月光映得明明白白。郟氏臉向床裡睡着\footnote{同一月也。阮優今日偷他時在此月下。異日愛奴動手行凶時也是此月下。今日月下何其太樂。異日月下何其太苦。}。慢慢揭開被一摸。一個光屁股朝外。阮優輕輕伸手去摸他的妙物。稀稀幾根毛。竟是合了相書的。道是依稀見肉始爲奇。陰中尚有些餘精流出。就知是纔同他令尊領敎了來的。淫興大發。陽物直豎。側倒身子。捏着正對了牝戶。趁着那濕意往裡一頂。不知不覺送了進去。郟氏同公公大幹了一回。身子乏倦了。睡着全然不覺。及至驚醒時。已被他送到盡根。阮優見他醒了。恐他掙動掉出。忙把右手從肩下伸過去\footnote{右手妙極。是在床外臥者。此等閒話。亦不稍錯。}。摟着脖子。左手將他胯骨扳緊。用力抽搗。郟氏爽快不過。把屁股也便亂就。阮優見他如此。知他得了樂趣。料無別話。纔放心大弄。那郟氏起先還疑是公公。但纔高興過了。五旬外的人那裡又有這樣興致。且上邊人多。他如何下得來。又疑是愛奴。覺得這個陽物比他兩人都粗大些。幹法也甚是在行。被他抽得氣都回不過來。那裡還說得話出。口中只問得。你你你你是是。那個誰字再吐不出。心中也猜了個八分是小叔。直等弄完了。方要問時。聽得說道。我的親親心肝。我想你久了。今日纔得遂了心願。郟氏聽了聲音。果然是他。忙翻過身來。笑嘻嘻擰了一下。道。我就疑惑是你這賊短命\footnote{寫出喜之至。却又騷浪之至。}。你多昝進來的。門關着。怎麼得開了進來。阮優道。我先來裡屋裡時一個人也沒有。我在床背後躱着來。郟氏笑道。那知你這樣個小夥子原來會作賊。阮優也笑着連親了兩個嘴。道。我是個偷花賊。爬起來。叫他睡平了。上(手)揷入摟着。親嘴咂舌。頑笑了一會。阮優笑道。我久要想弄弄你的。心想怕你心腸不定。譬如老早要下手。你可肯麼。郟氏也笑道。自己叔嫂。又不是外人。怕些甚麼\footnote{叔嫂便不妨如此。眞淫婦語。}。你哥哥在日。我就愛上了你。你若早要。我也依你。你不動手。難道我好先拉你的。你自己躭誤了怨誰。阮優摟着道。我的親親。就從今日起。也還不遲。你我都正靑年。後來的日子多着呢。正是。人心雖是如此。天理但恐未然。他兩個痛痛的弄了半夜。以償數年想思之債。自此夜間常來同他相伴。情同伉儷。阮大鋮只日間來。同他做白晝生活\footnote{當日阮最同嬌嬌做白晝生活。夜間阮大鋮還得同臥。今郟氏同阮大鋮做白晝生活。夜間阮優同臥。阮最竟不得一相傍矣。豈不便輸一籌。壞人其鑑之。}。夜間不得下來。郟氏所以放膽同阮優通宵行樂。一夜。阮優同郟氏事畢之後。說道。實不瞞你。婦人的東西我也見過許多。外邊的娼妓不算。如當日寶妹子雖然生得好。但他的年紀小。一點風情不知道。你嬸子也不爲醜。我雖心愛他。不知因甚緣故。但同他弄的時候。一毫毫高興也沒有。當日嬌姨雖好。一來年紀太大。二來他的此道也太寬得沒影。我同哥哥兩個人的一齊進去剛好。怎如親親你模樣旣標致。這東西又生得緊緊暖暖。實在有趣。眞是個妙物。大約婦人中像你這等緊的也就少了。那郟氏近來已把那後庭弄慣了。次次要前後俱來。方得暢快。同阮優弄了多次。想他的陽物比公公的又粗大些。弄在後庭中自然更有一番妙境。雖然想弄。怎好自己舉薦。今借他這話。便隨機應道。你說我這個緊麼。還有緊的呢。阮優道。我不信還有妙似他的。況且別人的緊不緊你怎得知道。這是你過謙的話。郟氏笑着道。不是別人。就是我身上還有個緊的。因拉他的指頭向糞門一塞。道。這不更緊些。阮優道。這件美物。我只弄過嬌嬌的。果然有趣。好嫂子。你只當積陰隲。賞我嘗嘗。就將他扶起。那郟氏並不推辭。就爬伏着。如道士伏章一般。屁股高蹶。阮優將陽物先塞陰中。先借他所洩之精。將後庭與厥物都潤濕了。然後一頂而入。大弄了一場。那郟氏淫聲艷語。股扭身搖。較淫娼浪妓猶勝。阮優喜愛至極。狂了多時方歇。你道這郟氏他也是個宦家閨秀。比不得嬌嬌出身微賤。怎麼就淫賤無恥到這樣地位。凡事有個來歷。必須敍明始末。方知道內中的緣故。他的祖父在嘉靖時係嚴嵩的門下。阿諛他父子。深得其歡心。官直做到戶部侍郞。嚴嵩事壞。世蕃伏法之後。他見倒了泰山。方纔吿老歸家。却也弄了許多宦囊。郟氏的父親叫做郟鉦。是廕生出身。他做刑部員外時。因父親老病。便吿了終養回家。他母親早故。他父親跟前有一個少年美妾。姓姬。纔得二十多歲。十分寵愛。常對郟鉦說。我今年老多病。全得這女子早晚扶持。着實慇懃。我若死後。可擇一個好人家將他嫁去。屢屢囑咐。到了臨終時。忽然變了舌頭。又向郟鉦道。此女隨我將及十年。我心甚是不捨。我死後可留着替我守靈。切不可遣嫁。原來郟鉦素常愛這姬氏。背了父親的眼。常同他調情勾引。兩下都有私意。却不敢大膽宣淫。郟鉦聽了老子臨終的話。心中暗喜。竟棄了常時的治命。從了臨危的亂命。將姬氏留下。他父親柩尚在家。衆人都在棺材左右伴靈。他二人眉來眼去。一日。偷得有空。兩個到他父親房中榻上。便成了苟合的事\footnote{繼述先志。眞孝子。}。姬氏伴了這老兒多年。有夫名而無夫實。經的是麪筋般陽物。今嘗着郟鉦這骨頭似的硬具。始知人道之樂。其喜可知。他父親死後不上一年。這姬氏便生了一女。就是郟氏了。郟鉦雖瞞了衆人。假說是他妻子所生。外人也就有些知道。但係閨房秘密。各人家務。誰人管他閒事。去聲揚敗露他。後來滿服起補。他拜在魏璫門下。仗魏璫之力。驟陞顯職。官至大理少卿。雖不曾如阮大鋮諸人依附作惡。免不得也是個閹門鷹犬。他與阮大鋮都是同類。故當年結了親家。圖彼此扶持。後來魏璫伏誅。他罪在三等。革職而已。這姬氏名雖是他亡父之寵。暗地竟做了他的小星。你想一個做官的人。受朝廷恩典。不能爲皇家出力。父子皆在權相逆璫門下阿諛以圖富貴。就該萬死了。且烝淫父妾。又在哀絰\endnotemark[2]之中生女。天道好還。此女焉得有不淫賤辱及在家門姓氏者耶。不必多需(敍)。且說郟氏當日偷那愛奴。因那阮最冷淡他。是無可奈何。將小廝來解饞。後來守了寡。小廝是故交了。自然撇他不得。不想這小廝漸漸膽大。以爲說主人已死。主母除我之外。尚還有何人敢爲彼之小夫。便不是當日小心。每同郟氏睡時。就拿出那小丈夫的樣子來。凡事要憑他的心性。郟氏心中甚怒。却說不出口。久欲撇他。無奈除他之外。再無〈無〉他人應急。只得強留備用。今遇了阮優。不但是小親小叔。且陽物與幹法俱勝他幾分。情愛甚篤。況又有公公時常來點綴。如何還稀罕那小廝。況恐或有洩露。豈不爲公公小叔所輕賤。怎肯棄了這兩個甜桃。倒去尋他那一枝苦李。遂將他撇在腦後。有多半年總不叫他進來陪睡。即白日相見亦不理他。反做出主母身分。有凜然不可犯之色。面上一點笑容俱無。那小廝猜測不出。暗想道。偷了十多年漢子的婦人。從新又守起貞節來。決無此理。同我恩愛了這些年。何一旦薄情至此。今日晚間我硬走了去。看他怎樣待我。到了掌燈後。他悄悄走到郟氏門口。輕輕將門一推。原來不曾拴。是開着等阮優的。他便挨身而入。走了進去。郟氏已經睡下。聽得脚步響。只道是阮優來了。笑道。短命的。你今日來的早。小廝只當是說他。也笑嘻嘻的道。我怕奶奶自己一個孤悽。故此來早些作伴。郟氏聽得是他的聲音。忙將帳子掀開。見他正脫衣服。怒說道。你來做甚麼。那小廝不看勢頭。還笑道。我來服事奶奶。還有誰呢。郟氏恐阮優來撞見。忙裹着被坐起。怒道。我當日一時失錯。同你做那不正經的事。如今悔已無極。你快快出去。再遲一會。我便〖口么〗喝起來。你就了不成。愛奴見他發怒。恐怕他當眞一時喊叫起來怎處。慌忙抱着衣服。含恨抱愧而去。過了數日。小廝偶然張見郟氏往上房去了。他忙忙走到房中。見那丫頭正脫了褲子坐在床上捉虱子。他看見了。跑上前抱着。親了個嘴。伸手摸了摸牝戶。就將他按倒。那丫頭是熟主顧。也不推辭。便兩足高蹺。小廝取出肉具。弄了一陣。兩人恐郟氏回來。忙忙完事。穿了衣服。小廝摟住他。問道。我同奶奶相好了這些年。也不知弄過幾千百遍。你是知道的。爲甚麼近來待我這樣情薄。當日有相公在。他倒偷我。今日相公歿了。他反從新要做節婦。定沒這樣的事。內中定有緣故。你定然知道。可吿訴我。這丫頭與他是久契的了。因念老主再幸之恩。厚賜之德。見郟氏旣私公公又偷小叔。他心中也忿恨不平。常想道。老爺這樣疼愛他。他還瞞着做這樣沒廉恥的事。幾次要吿訴老主。因見老主與郟氏相愛至極。不敢開口。且阮優只同郟氏作樂。不但毫無恩波相及。連靑目也不能夠。含恨怨已久。今見愛奴問他。他不說出老主。但道。你還坐在鼓裡呢。奶奶同二相公相好了這幾個月。七八連底子都好搗通了。你還問甚麼綿布絲布呢。那惡奴聽了這話。含恨道。他放着自己有老婆。又去占嫂子。反把我的好事打脫了。其情可恨\footnote{自己偷主母便罷了。小主偷嫂子便氣不忿。眞是惡奴心腸。然而又有說焉。昔余目擊一事。一兒子毆打老父。其孫在傍大怒道。沒天理的。這樣個老父親。你也忍心打他。便揮拳將父痛打。彼怒乃父之毆父。他便不想己所毆者亦父也。此正是人心天理處。}。尋思半晌。怒從心起。道。罷\footnote{這一個罷字。已見其切骨之恨。今之奸花氏。異日之弑主。皆從此字出。}。我幾時去偷上他的老婆。纔出得這口怨氣。他每夜留心看着。那一晚正在暗處張看。只見阮優開了房門出來。往郟氏房中去。那郟氏的門是虛掩着等他的。阮優推開進去。又掩上。他等了一會。悄悄到阮優房中來。微有月亮。到床前。脫了衣服爬上來。那阮優的妻子花氏。見丈夫常撇了他去偷嫂子。正一肚子忿氣。睡不着。忽見有人上床來。只當是丈夫不去了。問道。你同那淫婦肏搗去。怎又回來了。那小廝見他認錯。滿心暗喜。不敢出聲。只將他的腿〖扌扉〗開。要上身去弄。花氏還推推搡搡的不肯。道。我不稀罕你。你同那心愛的人弄去。那淫婦等得不知怎樣大急大發呢。看急壞了他。那小廝挺着個硬東西向腿縫中亂戳。花氏被他戳得癢癢酸酸的。也興動了。略放鬆了些。已被他攮了進去。弄了一下。花氏覺得與丈夫不同。渾身細細一摸。全然不是。大驚大詫。道。你是誰。那小廝弄也弄了。料道不怕他反悔。便道。我是愛奴。花氏驚道。你好大膽。怎敢半夜三更走來奸我。他道。有個緣故。大奶奶從大相公在日。同我相厚了十幾年。今日被二相公占了去。把我撇開。我見奶奶年小小的。相公弄(丢)了你。倒同別人去作樂。我怪氣得慌\footnote{他這氣奇得很。}。特來替奶奶作伴。相公旣偷得嫂子。奶奶就偷不得我麼\footnote{不想這惡奴竟會講因果。}。不但你出出氣。我也出了這口氣。花氏已被他弄了。說不出來。心中也恨丈夫丢了他去偷嫂子。有了這小子也可相伴寂寞。便不做聲。愛奴要得他的歡心。爲長久之計。又同他儘力盤桓。弄了一度還捨不得下來。一面抽抽扯扯的說道。蒙奶奶恩典不棄。可容小的常來服事麼。花氏道。那淫婦偷了我的漢子。倒望了我做嘴做臉的。我也氣他不過。你相公如今一心只撲着他。待我比當日淡了許多。我便同你好了也不爲過。你每夜悄悄打聽。但是他過去你便進來。笑道。你要留神。不要給那沒良心的撞見纔好呢。愛奴道。我知道。自然留心。見天色將明。還緊抽了一陣。纔起身穿衣出去。頂頭遇見阮優也從郟氏處回來。撞了個滿懷。阮優大疑。問道。你大淸早起來做甚麼。他無言可答。〈驚〉慌慌忙忙走出。阮優也疑了幾分。忙進房中。到床前就去掀被。花氏不曾隄防。被他掀開。就伸手將他陰戶一摸。花氏忙用手摀時。已被他摸着。花氏還夾着塊紬帕在襠中。黏〖氵韲〗〖氵韲〗濕漉漉的。弄了一手。是方纔弄了一陣未曾流淨之故。阮優大怒。將他光屁股上打了幾掌。罵道。沒廉恥的淫婦。你背着我同這小廝。我我同你了不得。花氏老羞變怒。也大哭大嚷道。捉奸拿雙。你拿住了麼。你同你嫂子偷弄得不値了。倒反賴我養漢。我同你到公公婆婆面前去講。那阮優欲待聲張。因自己現偷着嫂子。怕花氏在父母跟前說出。咬牙切齒。恨了幾聲。只得忍住\footnote{阮最見郟氏偷愛奴。因自己私嬌嬌不敢做聲。阮優見花氏偷愛奴。因自己私偷郟氏不敢做聲。前後遙遙一對。却無一語相同。}。次日尋了那小廝一件風流罪過。幾乎打死。吊在一間空屋內。思量要取他的命。阮大鋮夫妻知道。反責兒子酷虐。吩咐饒放了。此時阮優若將緣故向父母說明。暗暗處死了。倒也無後患。無奈賊人膽虛。自己也有毛病。只得叫人解放。饒恕了他。此後再不與花氏同床。連日間也不同他說話。只在郟氏房中說笑。花氏也是好此道的。又正在靑年。見丈夫總不理他。因有這一番曖昧的事。沒奈何。說不出口。只好暗恨在心。那阮優夜夜到郟氏房中去睡。不覺過了月餘。那愛奴小廝強盜一般的人。棒瘡已好。他是死裡逃生。心中恨怒至極。暗道。你偷嫂子就行得。我偷你的老婆就行不得。罷了。我送你的命。長遠受用你的老婆。出出我這口暗氣。又當替那大相公報仇。他去買了一把殺牛的牛耳尖刀。磨得風快。藏在身邊回來。晚間又來等候。那阮優不但不知他棒瘡已好。就是知道。那裡疑他敢來動手行凶。並不隄防。興興頭頭走入郟氏房中去了。愛奴看眞。到一更天氣。見門不曾上栓。輕輕推開。躡足去了。進去到房門口聽聽。聽得郟氏道。這些時你夜夜過來。想是嬸子惱我。他見了我。氣恨恨的那個樣子。好不難看。阮優道。你理那淫婦做甚麼。我還不曾吿訴你。我那夜在你這裡。誰知愛奴那奴才同他偷上了。我撞了個滿懷。我因爲同你有這件事。不好說得。有個把月不曾與他同床了。所以纔把愛奴尋事處了個半死。我本要治死他的。老爹奶奶不知就裡。又叫放了他。我又不好說出他們的事。恐怕他們也說出你我來。只得認着罷了。郟氏觸動心事。便道。愛奴的膽子大多着呢。你也要留心防着他。阮優道。那奴才再要膽大。我也顧不得老爹說了。定能治死了他。那愛奴聽得怒氣直騰。就想要下手。恐他們驚覺喊叫。只得耐着性兒等。又聽得阮優笑着說道。你方纔說愛奴的膽子大。我聽得人說他同你還有私帳。是舊情人呢。可是眞麼。你不消瞞我。郟氏頓了一頓。方說道。還是你哥哥在日。我那一日在房裡洗澡\footnote{淫婦再無不善巧言者。他頓了一頓。話便隨口而出。這兩句是眞。}。乏假(倦)了\footnote{假。}。也沒有穿衣裳\footnote{也眞。}。就上床睡着\footnote{假。}。誰知那奴才走進來看見。就把我奸了\footnote{假半。}。及至我醒時。聲張已是無及\footnote{假。}。後來要吿訴你哥。又礙口識羞。不好說得\footnote{更假。你哥哥張着倒是眞了。}。只得忍耐\footnote{假。你何嘗忍耐。阮最張見不敢說。倒是眞忍耐。}。那奴才得慣了濟。但是你哥不在家便來纏我。我已被他奸過了。推辭不得\footnote{此數語半眞半假。}。常同他弄弄是有的\footnote{此句眞。一篇話眞假相半。欲加之罪。何患無辭。淫婦善說。}。親親。你是我的心肝一般。你問我。我故此實話吿訴你\footnote{只算得半虛半實。}。你不要笑我。我如今有了你。還肯稀罕他麼。不瞞你說。有一個月前頭。他又要來想同我睡。被我要〖口么〗喝。攆了他出去了。阮優道。這奴才眞膽大。等我慢慢治他。又笑道。我還聽得說老爹也同你有些話說呢。郟氏笑道。他是公公。我是媳婦。大壓小。他要同我睡。我如何拗得過。也是沒奈何。勉強依從。怎像你可我的心這般恩愛。就是你哥在日。我同他夫妻一場。還沒有〈有〉這樣親厚呢。阮優笑道。看不出你這件東西。倒嘗過好幾個美味。二人笑了一回。阮優又道。你這後路。他們可曾做過麼。郟氏道。啐。怪短命的。你把我看得太不値錢了。這是我愛你得很。纔憑你翻來覆去的受用。你倒疑我同他們這樣。阮優道。我同你背後走得多次了。今日弄個新樣兒。郟氏道。怎麼樣弄呢。阮優道。等我仰睡着。你跨上我身來。臉向脚頭。背套在屁眼內。你兩隻手拄在褥子上。我用手搊着你的屁股。一起一落。看那出進的勢子。你低着了頭也看得見。可不妙麼。郟氏也就依他。兩人嘻嘻哈哈。便不見說話。只聽得吁吁喘氣。愛奴聽得明明白白。想道。這淫婦原來如此淫賤。我殺他也不爲過。又聽了多時。方沒聲息。過了一會。三鼓將完。聽得有了鼾聲。悄悄走到床前。月光映着窗子。甚是明亮。掀開帳子一看。二人弄乏了。正摟抱睡熟。那愛奴看得眞切。風快的刀在脖子上一刀一個。早已了帳。這是古人的六個字。一毫不謬。他道是。

\begin{quotation}

賭近盜。遙(淫)近殺。

\end{quotation}

豈不確然。那小廝正走出房門。那個丫頭恰恰起來小解。看見了他。滿心歡喜。只當他以肉鎗來敍舊。那知他是以鐵刀來弑主。還笑吟吟的低聲道。你來了麼。二相公同奶奶在床上睡覺呢。你到我床上去罷。愛奴心下尋思。旣殺了主人。明日豈不被他說破。陡起凶心。道。也顧你不得。劈胸一刀搠倒。怕他不死。連戳了兩三下。將刀撇在屍傍。帶上門出來。走到花氏房中。脫衣爬上床來。花氏月光下看見是他。心中甚喜。也正想他來弄弄。問道。你好了麼。他答道。我好了。今日纔報了仇。我們此後可放心做事了。花氏問他緣故。他道。且弄了再對你說。花氏速忙睡好。愛奴雖上了肚子。那陽物再不得硬起。花氏見他不揷進去。伸手一摸。縮得軟叮噹的。問他。這是怎的了。這小廝素常雖然凶惡。却不曾殺過人。今一連殺了三個。且又兩個是主子。雖沒人知道。心中却害怕。那陽物如何得硬。花氏又問他。他方把殺了三人的事吿訴了。花氏嚇了一身冷汗。道。這如何了得。愛奴道。事已到了這田地。說不得了。一露風聲。你我都是死數。你不要怕。我此後每夜來陪你。你也不須着急。花氏聽了。心中亂跳。也毫無興頭。便道。你且出去。着人見了。不是當頑的。那小廝也怕人知。就下床穿衣出去了。次日。到了日色大高。燒洗臉水的僕婦見郟氏房中丫頭不來取水。只當是睡癡了。送了水來。推開門。見丫頭血漓漓的殺倒在地。吃了一驚。進門叫了兩聲大奶奶。不見答應。掀開帳子。只見大奶奶同二相公雙雙殺死。嚇得一步一跌的喊着。報與阮大鋮夫婦。嚇得忙來一看。見他叔嫂二人殺在一床被中。雖然知是奸情。却想不到被何人所殺。爲何連丫頭都殺了。刀也撇下。心下不明。叫了二媳婦來問。花氏雖然明白。恐事出自己奸情。可敢實說\footnote{此猶可願(原)也。}。況且還要留着小廝長遠作伴\footnote{此則一剮不枉。}。只得假做慟哭。說道。他同我不同床久了。每夜說到書房裡去睡\footnote{淫婦個個善說謊。}。我正疑惑不知甚麼緣故。原來他過來做這樣事。我並不知道。也不知他被甚麼人殺了。阮大鋮怕醜聲傳出。忙買棺材裝殮。衆婦女替他二人穿衣服時。阮大鋮瞥見郟氏雪白身屍。不禁失聲慟哭了一場\footnote{阮最借哥哥屍靈哭庶母。阮大鋮借得兒子屍靈哭媳婦。也是一對。}。棺殮畢了。兩處停放\footnote{嬌嬌同阮最死是兩口棺材。郟氏同阮優死又是兩口棺材。他家的喪事好熱鬧。}。方差人到親家處報喪。此時郟鉦的妻子已故。便是姬氏當家。也有五十餘歲了。郟鉦同他暗地綢繆。雖夜間在被中拿他做個老妾。日裡少不得還要把他當庶母。一家皆是尊稱之曰老奶奶。聽見女兒死了。放聲大哭。忙同郟鉦到了阮家看時。已經裝入棺內釘上。姬氏郟鉦大怒。說道。爲何不等我們來見見屍身。竟自入材。定是女兒死得不明。快快啓棺。待我驗看。阮大鋮含着淚。將他叔嫂通奸。不知被何人所殺。連丫頭都殺了。詳細奉吿。因勁(頸)斷血汚。放着恐親友來看見不雅。故忙忙裝殮了。姬氏郟鉦聽得他乃愛是如此吿終。羞得愧赧無地。只哭了幾聲。便連忙回去。到家。深自悔恨。悄向姬氏道。我家幾代仕宦。今此女如此死法。親友問知。門楣盡辱。何以見人。這是我該死。你是父親愛妾。我竟烝淫了你。奸生此女。理應如是。姬氏道。你父親當日叫你將我嫁人。你爲何把我留下。又是你引誘奸我。不是我先偷你。就是女兒。你若把他嫁個好人家。如何有這等的事。你難道還不知阮家的壞麼。他家當日求親。我何嘗沒有阻攔過你。你說他是科甲門第。決定要給他家。你怨得誰。這是一個女兒報應了兩家。郟鉦無言可答。惟有嘆氣。自怨自艾而已。他雖自悔。然已無及\footnote{雖然無及。能知自悔。尚良心未曾死盡。其如阮大鋮竟不知自悔何。}。人生在世。素行豈可不十分檢點也。再說阮大鋮將阮優郟氏放了二十餘日。擡出埋葬。丫頭也埋在郟氏墳後。不題。這愛奴果然夜夜偷進來同花氏同臥。連花氏的一個丫頭他也弄上了手。堵住了他的嘴。且按下一邊。再說那個阮優郟氏被殺之後。阮大鋮疼兒的心只有一二。那疼媳婦的心倒有八九。提起時時墮淚。毛氏衆人只說他想兒子。自己忍着心疼。多方勸解。惟有馬氏知他心事。一日。又見他咨嗟悲慟。勸道。死者不可復生。老爺想念他也無益了。一來老爺有了年紀。二來大奶奶也是沒良心的。老爺這樣疼他。他還背了偷二相公。二相公也算自作自受。老爺也不必悲切了。如今還有一個頂窩兒的。老爺何不取樂一番。解了心事罷。阮大鋮道。大媳婦當日是我一時高興。你說阮最同嬌嬌通奸。我拿他來出氣。今日二媳婦無故。怎好又弄上他。又嘆道。佳人難再得。大媳婦雖然不長進。偷小叔。我倒也不怪他。我做公公的偷得媳婦。他做嫂子的也就偷得小叔了\footnote{心有偏愛。不拘如何。都可待諒。情之實然。}。只可恨阮優這奴才。放着少年標致媳婦不去受用。反去偷嫂子。你說自做自受。一絲不借(錯)。我那裡還想他。馬氏道。我說二相公不是偷大奶奶一個的話。阮大鋮〔道。〕還有誰呢。馬氏道。大相公死時。奶奶拷問那嬌姨的丫頭。他說的硶死了。說嬌姨嫌老爺年老不濟了。大相公軟弱。二相公生得又強壯。下身的東西又粗大。但是老爺不在家。兩個人就關着門大弄。比夫妻還恩愛幾分。後來大相公也知道了。弟兄吃醋。幾乎成仇。嬌姨勸他兄弟不要相爭。替他們和事。三人滾做一床。怎麼一個弄前。一個弄後。又怎樣背着弄。眞沒有耳朶聽。那一日好些人在嬌姨房裡都聽見說的。奶奶怕老爺知道。難爲二相公。吩咐瞞着不許傳說與老爺。這樣論起來。就把二奶奶弄弄也不爲過。阮大鋮道。阮優奴才罷了。嬌嬌這樣淫賤。可惜他死了。要不死。我碎割了他。馬氏道。還不止嬌姨呢。連寶姑娘未嫁時就同二相公就勾搭上了。後來纔偷上嬌姨。母女兩個吃醋爭鋒。多少醜聲。誰不知道。阮大鋮道。我也隱隱聽見寶兒在勞家不長進。我還不信。疑是人寃誣他。原來在家時就這樣壞。有這樣娘。就生這樣女兒。可恨死遲了\footnote{阮大鋮一家妾女子媳所做所爲。若不知猶可言也。旣詳知之。而毫無自反自恨之心。眞奇異。令人不解。}。這樣說起來。二媳婦不可不弄他一下。出我之忿。慢慢的想方兒。因向馬氏道。我看你比他們都好。還疼愛我。有話還肯對我說。我自然分外疼你。不要學嬌嬌那淫婦嫌我老。馬氏道。哎呀。老爺怎拿一個比一個。我模樣雖不如嬌嬌。我的心腸與他不同。我見老爺同我幹事。我又不敢阻老爺的興。我生怕老爺有年紀的人費了力。我暗暗心疼得了不得呢。阮大鋮被他甜言蜜語哄得滿心歡喜。摟他在懷中。說道。你旣這樣疼我。我難道不偏疼你麼。遂伸手去扯開褲子摸他的陰戶。那馬氏也伸手去捏他的陽物。彼此撫摩了一會。那馬氏有些興動。見他陽物不舉。蹲下身去。將陽物放在口中舔咂。阮大鋮不禁情興如火。同他到床上。放下帳子。脫了衣服。阮大鋮道。嬌嬌這淫婦。我要同他弄弄屁股。他千難萬難。誰知他倒給阮最阮優兩個奴才弄。我一生酷好這件事。你可肯給我弄弄麼。馬氏道。老爺。不要說弄我的屁股。就是要弄我的嘴。我還有個不依的麼。我每常也想送老爺。恐怕老爺嫌髒。不敢開口的。若不嫌棄。憑你怎樣弄法。阮大鋮歡喜得無限。摟着他。親了幾個嘴。他就扶伏在床上。屁股高高蹶。阮大鋮笑嘻嘻用了些津唾。款款頂入。馬氏道。你只管憑着高興。狠狠的頑。不要說怕我疼。阻了你的興。就弄出臟頭來。我也不怨你。阮大鋮愈加歡喜。用力抽提。正大弄着。一來也是姻緣湊巧。二來他阮家門風合當敗壞。這日花氏偶然有句話要向馬氏說。走上來。見房門又不曾關。放着帳子。疑是馬氏睡覺。再想不到他們打白仗。那阮大鋮同馬氏正弄得高興。也不曾聽得脚步響。那花氏要揭開帳子。心中想道。我冒冒失失把下身掐他一下。嚇他一嚇頑頑。遂伸手就去一捏。不想剛剛伸到阮大鋮的陽物上。捏着水淋淋的。連忙放手。揭開帳子一看。原來公公同他弄屁眼呢。捏的是公公的此道。羞得徹身通紅。慚愧難當。回身就走。阮大鋮先被他冒冒失失一捏。倒也吃了一驚。不知是誰。見帳子掀開。原來是他。心中正在想算計他。不想有這個奇緣。忙抽出。跳下床來。一把抱住。推在床上。道。我兒。自己翁媳怕甚麼。就去扯他褲子。那花氏羞愧滿面。自己失手錯了。又不敢叫。只攥着褲腰。東扭西扭的亂掙。那馬氏笑向他道。二奶奶。不要呆了。靑春年少。落得受用。你不看當日大奶奶在那時同老爺相好。老爺何等疼他。吃好的。穿好的。你二相公又不在了。你不靠老爺靠誰。且落得享福。有老爺做主。還怕人說甚麼不成。我勸你是好話。快不要戇。就相幫着去撥他的手。那花氏一個水性少婦。也有些動心。又聽馬氏勸他的話。也希圖公公疼愛。料想也掙不脫。把手略鬆了些。已被阮大鋮脫下了褲子。伏上身弄了進去。花氏只閉着眼。一語不發。阮大鋮同他弄完了。摟着問他話。他總不答。馬氏笑道。你好呆。害甚麼羞。我也是婦人。同你一樣。怕甚麼。花氏也不做聲。掙了起來。穿上褲子。羞羞慚慚的去了。那阮大鋮歡喜無限。自幸得此奇遇。你道這馬氏爲甚麼兩次三番攛撥阮大鋮奸兩個媳婦。他當日總成阮大鋮偷上郟氏。原圖阮大鋮歡喜。額外加惠於他。是利人利己的心腸。不意阮大鋮有了郟氏。一心貪在他身上。馬氏穿的戴的。阮大鋮雖然加原(厚)。但那一件要緊的事越發稀了。人說飽暖思淫慾。他不愁穿不愁吃。不想這一道還想什麼。他每每悔之無及。恰好他也得了個奇遇。故此又攛撥阮大鋮奸了花氏。他好另做兩圖。你道他是個甚麼奇遇。那阮大鋮的正妻毛氏只有正室之名而無伉儷之實。又年老了。阮大鋮整年不到他房中一次。他天性自幼好淫。老來這癟牝中竟不得稍嘗雞味。越覺難過。但說不出口。眞是啞巴吃黃連。苦在心裡。却也無時無刻不想此處。阮大鋮有一個心愛的家奴。名字叫做苟雄。係北京大名府人氏。三十來歲一條大漢。身材膂力都好。又會些武藝。阮大鋮\endnotemark[3]當日在北京時。見苟雄時常在街上使拳棒化錢財。愛上了他。收在身邊做個親隨。他也自己行事不好。恐人暗害。特特擡舉苟雄做個護身的心腹。帶到了南京。時常叫他上邊來取東取西。毛氏便看上了他這漢仗。又知他有大力。一心思想要他褲襠縣中黑松林裡。似眼非眼。似嘴非嘴的這件癟物犒賞他\footnote{主母以此物犒賞家奴。眞是異典。}。却不得其便。一日。毛氏偶然到嬌嬌住的那房中走走。到了院子裡。見花臺上一塊太湖石掉了下來。叫丫頭道。你去叫了苟雄來。不多時。苟雄來到。毛氏道。那塊太湖石掉了下來。你擱了上去。苟雄走到跟前看了看。約有百餘斤。毛氏也走了來看。苟雄把上衣脫了。只穿短衫。雙手抱起那石頭來往上放。他因使力。胸脯腆着。下身未免就往前挺起。毛氏有心。見他褲襠中一團凸起。好生動火。心生一計。向丫頭道。我一時肚疼起來。你去生個炭火。拿陳六安茶泡一壺來我吃。丫頭去了。苟雄放好石頭。也穿衣要走。毛氏道。你且來着。他走到房中一條春凳上睡倒。道。丫頭不在這裡。我肚子疼得很。你替我揉揉。苟雄意思不敢。毛氏道。我還養不下你來麼。家人同兒女一般。怕甚麼。苟雄只得伸手去替他揉。纔揉了幾下。他道。這沒用。我有這個病根\footnote{乃騷根。非病根也。}。每常痛起來。老爺拿光肚子替我一熨就好了。你也來替我熨熨\footnote{他果有此病。醫生大得便宜。}。苟雄笑着不敢上前。毛氏急了。把褲子脫下睡倒仰着。道。快些快些。我要疼死了\footnote{倒怕是要癢死了。}。那苟雄見他如此。知他是要如此如此之意。若不如此。恐他反怒。況他一個壯年無妻小夥。見毛氏之物雖毛多而癟。到底是個婦人之具。陽物也就大舉。也不管甚麼名分尊卑。扯開了褲子。扛起腿來。就攮了進去。儘力大弄了一陣。毛氏久違此物。連丢二次。怕丫頭送茶來。叫他歇了出去。毛氏見苟雄不但力大身強。且那一根厥物也出類拔萃。生平嘗所未嘗之美物。豈但強似當年之表兄。還覺大勝今日之夫主。喜出望外。時有厚贈。但是阮大鋮不在家。就悄悄叫了他來。到嬌嬌那房中去行樂。丫頭也都知此事。因受了主母厚賞。故不層(曾)洩漏。也相厚了許久。不防馬氏一日到毛氏房中來有話說。不見毛氏。問丫頭們。都不做聲。馬氏道。這丫頭們怎都啞了。問你奶奶在那裡。怎不答應。那丫頭沒得說。答道。奶奶往嬌姨房裡去了。馬氏動疑道。往那空屋裡去做甚麼。也就到那屋裡來。推門進去。見苟雄扛着毛氏兩隻腿。在一張椅子上大弄呢。毛氏大驚。推開苟雄。也顧不得羞恥。精屁股跳起來。拉住馬氏。跪下道。好姨娘。你看〔我〕素常待你不薄。你千萬不要對老爺說。後來你不論要甚麼。我都與給你。就要我的肉吃。我也情願\footnote{但恐肉騷臭。吃不得。}。那馬氏連忙拉起毛氏。道。好奶奶。你待\endnotemark[4]我恩典還少麼。我肯壞你的好事。你只管放心。我要洩露了你的事。不逢好死。我去。你只管放心取樂。就假意要走。毛氏又拉住。道。好姨娘。你雖這樣可憐我。我到底不放心。須得你也動(同)他弄弄。我纔信得過。附耳在上。道。他的本事比老爺強幾十倍呢。弄得快活到心眼兒裡頭去。你試試看。馬氏道。這如何行得。我不說就是了。那毛氏又跪下去。道。好姨娘。你不依。是不肯可憐我了。我跪着。看你可過得意去。馬氏見他這樣下氣。又見那苟雄也精光着跪在旁邊。只是叩頭。腰問那話又粗又長。紫威威。沈甸甸。好不怕人。心愛得了不得。忍不住笑吟吟的道。奶奶。你請起來。再做商量。毛氏見他口軟。站起。向苟雄道。你還不謝姨娘呢。那苟雄磕了個頭。爬起。大膽上前。一把抱住。放在一條春凳上。就去脫褲。馬氏口中道。我不消。你留着精神服事奶奶罷。說着。已被他褪下。弄了進去。苟雄盡力弄了有一個時辰。馬氏丢了數次。他攧簸哼喞。淫聲浪語。連毛氏都看得肉麻起來。弄完了。馬氏覺得與阮大鋮大不相同。方知這竅中竟有如此妙境。大家穿衣回去。此後毛氏揀上好衣錦常常送與馬氏。誰知那苟雄他雖蒙奶奶擡愛。不過只圖他的賞賜。見毛氏一個老婆子。臉上許多皺摺。頭毛也花白了。脫光了時。兩個乳如兩個曬乾了的蝙蝠茄。一個陰戶塌了下去。蓋子上一塊大骨頭。且自小肚之下兩腿凹中一片黑毛。如落腮鬍子一般\footnote{這才眞是毛氏。}。不但一點趣沒有。又甚是難看。有一個駐雲飛道他那陰戶的好笑。怎見得。

\begin{quotation}

口似荷包。皺摺攢圍縫一條。皮閉羊腸道。毛護風流竅。噫兩足大分蹺。愈增醜笑。好似那掉齒老翁。張口無聲叫。他尚自假做風騷股戰搖。

\end{quotation}

請想這個樣子。同他還有些甚麼樂趣。不意遇了馬氏。又年少。又風騷。歡喜無限。馬氏三十多歲。乍遇了這件寶貝。一心撲着他。兩人十分恩愛。常常偷空就幹。倒把毛氏撇開。二人恐毛氏吃醋。商議想要逃走。有一調西江月\endnotemark[5]說他二人道。

\begin{quotation}

夫主防身健僕。東君閨內韶客。私歡裁就兩情濃。眞是雄雞雌鳳。認道良緣輻輳。那知主僕私通。此身已陷淤泥中。還道這人情重。

\end{quotation}

馬氏將所有細軟都陸續轉了與他。做同逃之計。不想阮大鋮同(因)郟氏死了。無處去尋樂地。時常在他房中。馬氏甚是礙眼。故此勸他奸了花氏。使他二人情熱。他好得便抽身。所以力成其事。花氏那日同公公弄了一下之後。那愛奴雖夜夜進來伴他同宿。花氏也不好向他說得。那阮大鋮隔三五日到花氏房中。支開丫頭就弄一下。花氏也被他弄過。推辭不得。只得依從。雖然多次。阮大鋮心裡固然愛他年小標致。但交合之時。他從無歡顏相對。古古板板。像無可奈何樣子。故阮大鋮不甚眞歡喜。你道何故。花氏一則嫌他年老不濟事。是無可奈何從順的。況且又有愛奴這樣個精壯寵奴。所以他與阮大鋮乾(幹)事。不過如應差而已。一日。阮大鋮往親戚家吃戲酒。五鼓方歸。小廝打着燈籠到上房。逕到馬氏房中來。黑魆魆的。以爲都睡熟了。自己接過燈籠。命小廝出去。他進到房內。見房門大開。到房中掀開帳子一看。不見有人。叫了兩聲。也不見答應。心中甚是疑惑。走到那邊。見丫頭酒氣沖人。呼呼大睡。搖醒了。問道。你姨娘呢。丫頭揉了揉眼晴。答道。在床上睡覺呢\footnote{情景逼眞。是順口語。}。阮大鋮道。在那裡。何嘗在床上。丫頭還〖目夢〗〖目夢〗\endnotemark[6]戇戇的道。想是到奶奶上邊去了罷\footnote{妙。是日間的語慣了。不覺說出。的是個蠢小丫頭。}。阮大鋮大怒。夾臉兩個嘴巴。道。半夜三更到上頭做甚麼去。你還胡說。那丫頭被這兩下纔打得醒過來。道。昨晚點燈時。姨娘強着賞了我兩碗酒吃。我醉了來睡覺。不知姨娘在那裡。阮大鋮復又到馬氏房中。見桌上放着隻蠟臺。點灼了。開了箱櫃一看。都是空空如也。毫無所有。知他是拐帶逃走。叫那丫頭來。問道。他旣逃走。你可有不知道的。你實說。他同誰有奸。跟誰去了。那丫頭道。我不知甚麼叫做奸\footnote{妙極。是無知小丫頭語。}。他往那裡去。又不曾吿訴我。我那裡知道\footnote{更妙。如聽得一個小丫頭說蠢話。}。阮大鋮越怒。上前打了幾拳。踢了幾脚。那丫頭大喊大哭。疼得滿地打滾。道。腿在他身上。他走了。我如何曉得。我要知道。我也去了\footnote{妙妙。愈答愈奇。我也去了。不知他去作何事。}。阮大鋮更怒。揪過頭髮。又踢打了一頓。道。你快說。不然我打死你。丫頭怪叫道。殺了我。我也不知道。與我甚麼相干。我每常只見苟雄常來屋裡。姨娘就把我倒扣在那邊。我間或看見他腰裡塞些東西出去。別的我不知道。此時毛氏同衆妾聽見吵鬧。都起身走來。毛氏聽見這些說話。暗暗吃驚叫苦。生怕阮大鋮處治苟雄。阮大鋮叫上夜僕婦下去叫那一個管事的家人龐周利來。吩咐道\footnote{毛氏何幸。苟雄始去而傍州例之。家人即現。樂哉。}。看苟雄在那裡。叫了來。龐周利去了一會。來回道。苟雄反鎖着門。小的擰開看時。房中一空。大約逃走了。阮大鋮知是他拐去了。心中痛恨。要報官緝拿。又怕馬氏說出他偷媳婦的話來。只得暗恨忍住。惟獨毛氏更咬牙切齒。恨這馬氏把他一個活心肝生生的摘了去。再說愛奴一夜同花氏睡着講閒話。忽然想起郟氏的事。向他道。你道大奶奶這淫婦該殺不該殺。我動手的那一夜。聽得他向二相公說老爺那老禽獸同他也是厚間。這沒廉恥的淫婦。公公媳婦也做這樣的事。就是騷極了。寧可偷別人也不肯偷公公。花氏聽了。暗想道。倒是老爺奸我的話不曾吿訴他。若他知道。把我也看得不値錢了。這夜兩人高興了一番。正然睡熟。花氏夢中忽然一驚跳起。愛奴也驚醒。忙一把抱住。道。你怎麼了。花氏定了半晌。方說道。我夢見姆姆房中那丫頭。一身鮮血。來向我索命。罵我說不是我私通了你。如何得害了二相公同姆姆。因你殺了他兩人。故此纔又殺了他。你的一死不消說。連我也放不過。我再三求吿他。他決不肯放。向我身上一撲。一驚醒來。魂都幾乎嚇掉了。愛奴聽說。心中也有幾分害怕。只得勉強安慰他道。這是心上夢。理他做甚麼。口雖如此說。心下未免懷着鬼胎。那花氏日間間或陪公公。夜裡每宿伴愛奴。過了數月。竟懷了孕。也不知是那一個的種。漸漸豐肚。那花氏要把公公奸他的話說與愛奴。或商量出個法子來。竟往阮大鋮身上一推。諒阮大鋮自然替他想法。花氏因前愛奴說郟氏的話。他硬口怕羞。不肯說出。但向愛奴道。這怎麼處。若露了出來。就不好了。那愛奴問他要了幾錢銀子。尋了些打胎藥來。吃了數劑。毫無效驗。愛奴道。如今沒法了。只有逃走一着。他一個官宦人家媳婦跟家人走出。決不好報官訪拿。苟雄同馬六姨不是樣子麼。我同你到他鄕外府。做一對夫妻過日子去罷。連丫頭也帶了去。萬不得已賣了他。做盤船(纏)也好。花氏一來無可奈何。二來他心中實愛愛奴。憎嫌公公老了。便依從他。問那丫頭。丫頭恐主母走了。追問他起來。可有不知情的。也情願同去。遂將細軟打了兩個大包。愛奴背了一個。丫頭背了一個。花氏包了頭。穿了丫頭的布衣裙。三人悄悄開門而去。次早。管門的人來開大門。見重門洞開。吃了一驚。走了進來。層層門都開着。見花氏的房門也大開。叫了兩聲。不見人影。入內一看。見滿地舊衣服。東西撂得亂三攪四。主俾(婢)二人都不見了。忙上去回了阮大鋮。阮大鋮又吃一驚。命查。家人說愛奴也走了。阮大鋮雖知是他拐了去。但家奴拐去兒婦。說不出來。只暗暗通知了親家。這花氏的父親花知縣也是個在閒鄕宦。聽得乃愛演了紅拂記。可還說得出一句話來。當年司馬懿假瞎。他也只好假聾罷了。可笑這阮大鋮奉承魏璫。做了多少惡事。富貴二字不曾圖得一件。積作得一個正妻。兩個兒子。兩個媳婦。兩個美妾。一個愛女。都報應做出這等好事。他不但不知警省改過。心腸愈醜愈辣。後來便見。且說那愛奴同花氏並丫頭偷出了大門。天尚未明。覺得眼前一個黑影攔攔擋擋。及走到了跟前。却又不見\footnote{顯報則說明易曉。此等是隱隱忽忽報應。看者須知。}。愛奴心中甚是疑影。每常是走熟了的路。此時昏頭昏腦。總看不淸街道。直至東方大亮。眼前黑影不見了\footnote{向花氏夢中索命是他。花氏腹中之物也是他。此時黑影也是他。此時作書者暗含報應。不肯說得活現。恐人訊說鬼話也。}。纔走出了水西門。要雇船往上江去。因見來往的人絡繹如織。恐遇着熟識。心下未免驚慌。面上的顏色便有些變異。不想正遇着幾個捕快出城拿賊。見他三人旣無行李。只背着兩個大包。慌慌張張。見人都有驚懼之色。又見花氏雖布衣淡妝。面孔非病(貧)家婦女。知是逃走的人。上前一陣盤問。那愛奴是心虛的。面容失色。嘴中話都說不淸白。那花氏同丫頭臉如白紙。渾身抖戰。捕快將他三人帶到一個僻靜小廟中。把愛奴拷問起來。他忍受不得。方說是阮大鋮的家人。拐的一個是幼主母。一個是丫頭。他衆人又問花氏。花氏今雖做了淫奔的婦人。當日也是宦家的閨秀。何嘗見過這些惡事。他先見拷問愛奴的那些非刑。魂都沒了。恐怕拿他也拷問起來。二來冥冥中也有個神鬼。那郟氏阮優雖有可死之道。而愛奴非殺他之人。況愛奴花氏罪更浮於他二人之上。豈有逃脫之理。花氏遂將如何通奸起。如何遇了阮優。如何將他責打。如何殺了他丈夫嫂子丫頭三個人。又如何通奸有孕。纔逃了出來\footnote{阮大鋮造化。到底虧他害羞。不曾說出也。}。鬼使神差。細細說出。捕快遂帶到縣中。詳細稟知。知縣先問花氏。花氏又細說了一遍。然後問愛奴。也不曾用夾棍。也就一一招成。二人畫了供。知縣將愛奴打了三十收禁。花氏因有孕免責。也下了女監。丫頭交與官媒保出。申報了上司。上了本。愛奴因奸殺害家主。問了凌㓾。花氏雖非同謀。知丈夫被殺不首。反與愛奴通奸私逃。與同謀殺夫罪等。也問了剮。阮優郟氏叔嫂通奸。律絞。已死勿論。丫頭免議。並贓物給還原主。愛奴到了監中。衆禁子一來因他無錢打點\footnote{這是第一件。}。二來恨他凶惡。日鑽夜〖扌匣〗。受了無限苦楚\footnote{此因無銀打點耳。若有錢。彼奉承不暇。何恨之有。}。花氏又帶上了兩個禁子\footnote{此極寫禁子之惡。}。每日每夜上下口都有得受用。等他樣(養)過了娃娃。纔帶他二人到了市上。上了木驢。受用了一剮。臨刑的前一夜。愛奴花氏同夢見郟氏的那丫頭。笑容滿面。向他撫掌道。你們也有今日。二人醒了。自知死期一到。欲悔從前。已是無及。再說那知縣差人去叫阮家來領丫頭並贓物。阮大鋮回書都不要了。任憑發落。知縣命將丫頭官賣。贓物入庫。那也就是他囊中之物了。且說花氏的這一件事。也是眼前報應的一重公案\footnote{這一部書講淫褻的事。千言萬語總不過歸到報應兩個字。看花知縣這一重公案。〈沒〉似乎贅筆。可以不用。然是一個要緊報應。亦可警掌刑名之輩。勿謂其爲蛇足也。}。他父親花知縣。名叫花翩。倒也是一榜出身。做官雖不甚貪酷。却任性多疑。凡事偏拗\footnote{爲官者任性已大誤。再多疑偏拗。焉有不枉殺民命者。}。他問公事。若任性起來。憑着幕賓朋友百般勸戒。他再不肯聽。人知道他是這樣個倔強性子。也就沒人肯苦口勸他了。因此上地方上的百姓也吃了他許多的虧苦。含了無限的怨恨。且把他的事略敍一兩件。便知他的爲人了。他縣治中有個百姓叫做司新。\endnotemark[7]家雖貧寒。却識字知書。心地奸狡\footnote{嗟乎。讀書識字。原圖效法聖賢。若讀書但能奸效(狡)。讀之奚益。}。他有一座祖墳。與一個土財主名錢泰的山地相鄰。他欺心想謀這錢泰的地擴充他家的墳山。因使了個奸心。弄了幾塊大磚。寫了基址界限。倒寫了數十年前的月日。用刀鐫刻了。暗暗埋在錢泰的地上。也過了十多年。錢泰的妻子死了。就請地師在這塊地上點了穴。要來安葬。司新爭執說是他家的墳山。不容下葬。兩家爭競起來。司新便到縣中去吿。說土豪恃富霸占窮民墳地。錢泰倒運。剛剛撞在花知縣手裡。花知縣一接了狀子。便疑心錢泰是財主欺壓貧窮。霸占是實。隨拘了錢泰來問。錢泰稟稱。這是小的幾輩傳流的山地。山鄰皆知。非強占。況還有當年買地的文約爲據。上面寫着與司家的墳地爲界。花知縣命取了原契。並衆山鄰來問。次日。又審衆山鄰。異口同聲都說。小的們素常聽得說是錢家的是實。花知縣問司新道。衆人都說是錢泰家的地。文書上地界又寫得明白。你如何吿他霸占。司新稟道。老爺天恩。他倚富欺貧。想白占小的的地。小的可敢賴他。文書上雖寫着與小的家的墳地爲界。但那一片全是兩家的地。並不曾寫着畝數長尺。如何做得準。這些山鄰都是他買出來的硬證。總求老爺上裁。這花知縣先有個疑團在胸。牢不可破。聽了這些話。越疑錢泰霸占。却無可爲憑。躊躇了一會。忽問司新道。你說的固是。但你執定說是你的。可有甚麼憑據麼。司新說。小的父親在日。曾向小的說。墳山後來恐有人吞占。山地界址都有磚字埋在地下。雖向小的說了埋的地方。却不曾眼見。年深日久。不知可還有沒有了。花知縣道。這就是憑據了。縱然年久。必定還有形蹤。隨差衙役押他衆人同去眼看刨挖。果然在疆界上挖出幾塊磚來。錢泰所點之穴却在司家磚界之內。差役回衙呈上。花知縣見了那磚非一日之物。字跡尚還可辨。心中大怒。以爲錢泰霸占是眞。重責二十板。衆山鄰各責十板。將地判還司新。你道這節事可是他疑心的偏處。這還是小事。還有一件人命大案。被他任了性。將一婦人受了極刑。更是寃枉。那時有一個百姓。姓于名魯。是個孤丁。他不但生性愚鹵。且形狀鄙猥。百無一能。以賣菜爲生。他父母在日。替他娶了個妻子汪氏。這汪氏雖是窮家之女。却生得一貌如花。竟有七八分姿色。他嫁了于魯。甚是賢慧。並不憎嫌丈夫。他家租了一間臨街的房子住着。後邊又沒院子。這婦人潑水倒漿。少不得往街上去倒。他少年嫩婦未免懷慚。在門內往外一潑。便撤身進去。不想活當有事。一日正去潑水。一個人在門口走過。潑了那人一身。汪氏情知理虧。一個臉緋紅。忙陪笑道。一時失錯。大爺不要見怪。那人是個標致少年。穿了一身華服。他姓宋名奇生。生性浮浪。家中有數千金之產。纔二十多歲。因娶了個奇醜妻子。兩不相睦。時常在外三瓦兩舍嫖妓宿娼。淘碌空虛。現在弱病在身。還不知檢。猶自貪歡\footnote{有此數句。伏後交合即死之故。詳細。}。這日在此走過。不想汪氏潑了一身髒水。正要發作。猛回頭。見是這樣個妙人。遍體酥麻。見他有自愧之色。忙陪笑臉。低聲道。失錯何妨。若不嫌棄。不妨再請潑些。不住望着嘻嘻的笑。汪氏見他話雖輕薄。却是自己的不是。又見他俊淸和善。也徹(微)笑了笑\footnote{這一笑笑得不好了。古云。怕閒漢。任有烈性女子。禁不得有閒漢勾挑。無有不壞了事者。即此五件事中小閒二字一理也。是婦女但此一動心。則不可復制矣。}。縮身進去。那宋奇生還不住回頭望着去了。誰知這一潑。把個宋奇生的魂竟潑在了他家。一日不住的五七遍在他家門口走。總不見這婦人的影兒。倒看見一個時常在他家賣花翠的老婆子。這婆子姓密。因他有一張好利嘴。衆人借他的姓起了一個混名。叫做老蜜嘴。就在這婦人的緊隔壁住。宋奇生滿心暗喜。到家忙叫家人叫了老蜜嘴來。到書房讓他坐下。袖中摸出一封銀子送他。道。我有一件要緊的事托你去做。若替我做成了。謝你紋銀二十兩。這是五兩。先送你發個利市。那老蜜嘴歡喜得了不得。滿臉是笑。說道。大爺有甚事。只管吩咐。我若力量做得來。再沒有個不盡心的。宋奇生便將隔壁那婦人如何潑了他一身髒水。如何望着他笑\footnote{一笑之禍。}。要求他做個馬泊六之意。成全此美事。這老蜜嘴與汪氏隔牆。來往甚密。汪氏常有事煩他。他從不推辭。汪氏感他的情。認他做個乾娘。兩人甚是和美。無一日不見面。今聽得宋奇生這話。心中暗道。這婦氏(人)同我住了這幾年。從不曾見他走甚邪路。又是乾女兒。這話如何開口。便推辭道。這人是我緊鄰。夫妻和睦。從沒有聽見他有甚麼壞事。這個我不敢許。宋奇生見他推托。忙道。你的蜜嘴是有名的。你若肯盡心。一片甜言自然說得動他。若是嫌少。事成了我再加十兩謝你。老蜜嘴一年賣花所賺的錢不過只夠養家。何嘗見過這些銀子。聽見許他三十兩。利慾薰心。遂轉了念頭。便道。這銀子大爺且收下。我去探探他的口氣。看事成了再來領賞。宋奇生大喜道。你若不收。便是推辭了。只管拿去。我專聽好音。那婆子也就笑納。回到家中。就到汪氏家來。汪氏連忙讓坐。說了一會閒話。婆子忽然笑說道。我看天公甚不公平。你這樣個標致聰明的人。甚麼上樣的丈夫配不得。却嫁了這樣個女婿。傍人也替你叫寃屈。我娘兒們說話。你不必掩藏。你心裡可想相與個趣人兒麼。汪氏道。一來是我前生造下來的命苦。二來我父母雖窮。也是淸白人家。若做些外事。醜名一揚。不但一身名節喪盡。連父母的臉面都沒有了。婆子笑道。聽你這話。是個顧羞恥的好婦人了。怎麼有個標致後生說你有情意到他。想念你了不得。托我來探你的口氣。汪氏紅了臉。含羞怒道。這是那裡的話。是個甚麼人。婆子笑道。你不要發急。事情必有個緣故。一個少年的財主姓宋。是我的一個大主顧。他向我說那一日在你門口過。你故意潑了他一身水\footnote{故意二字妙。}。還笑着對他說話。他想得你夢魂顚倒。故托我來探你的話。據我想起來。你兩個正是郞才女貌。若果然相愛。我替你引進。汪氏聽說。知是前日那人了。答道。我那一日失錯。潑了他一身水。並非有心。因爲得罪了人。只得腆着羞臉陪罪是有的。何嘗有甚私情私意。媽媽不要聽他枉口拔舌。不要理他。那婆子見說不進去。只得到宋奇生家。將婦人的話詳細回覆。原銀繳還。宋奇生不肯接。再四央求道。你只管收下。再看機緣。全仗你的力量。我決不敢忘你的恩。那婆子也就收了。應諾而回。且說那汪氏自聽了婆子一番說話。少年水性。未免動情。暗想道。這人倒也是個多情的。我潑了他一身水。不但不惱。倒反愛起我來。但說我是有心勾引却是寃枉。看他年少標致。若嫁了這樣個丈夫。也不枉爲人一世。心作此想。未免就有個相感之意。不想這宋奇生因不見老蜜嘴回信。眠思夢想。廢寢忘餐。他素常身子怯弱。就病倒在榻。他因夫妻不睦。便在書房中養病。一日。叫了老蜜嘴到家。說道。這婦人是我前生的寃家。我這條命眼見是他送了。床頭取出一封銀子。道。這是二十五兩。送你老人家。煩你去向他一說。他若肯救我的命。便是我的大恩人了。我竭力照看他一家。若斷然不肯。是前世無緣。只得憑命罷了。但願你儘力去說。成不成銀子都送你。我後來還有重謝。老婆子得了這一大包銀子。歡喜無限。就別了回家。又到汪氏家來。便將宋奇生如何因想念他成病。看看待死。托他來求救。他把宋奇生的話詳細達上。又再三慫慾(恿)道。我們這樣人家。料道貞節牌坊輪不到。若相與了這樣個多情多義的人。且落個後半世快樂。你不要癡了。這婦人素常心不動倒也罷了。前次聽婆子說宋奇生想念他的話。也感動了些。今又聽說因他病重。又聽說照看他一家的話。便動了個知己之感。雖然不曾許出口來。但紅了臉。又不做聲。只嘆了兩口氣。婆子見這光景。知他心軟。便抽身出來。到宋奇生處將前話說了。道。我看他雖不做聲。已有肯意。你明日可掙挫到他家。苦苦哀求。包你一箭上垜。便是一時變臉。我來解救。宋奇生聽了。一心歡喜。病竟好了多半。次日。打扮光鮮。到老蜜嘴家打了照應。看看街上無人。竟走入婦人家來。汪氏正坐在窗下做針指。忽見宋奇生推門進來。便道。你這人非親非戚。到我家來做甚麼。宋奇生忙把門關上。到跟前雙膝跪下。低聲吿道。向日蒙你垂愛\footnote{此句妙。便把有心潑水賴在他身上。}。我爲你一病到今。性命幾乎不保。我料想也活不成了\footnote{語中之讖。}。今日特來見你一面。死也甘心。你肯與不肯。憑在你的慈悲罷。就一把摟住了他。汪氏見他這光景。又可憐。又動了個愛字。也不怒。只紅着臉。低聲道。這如何行得。看我丈夫回來。快些出去。宋奇生見事無變局。就站起。將他抱到後半間床上。便替婦人脫褲。汪氏雖用手擋拒。却不做聲。被宋奇生纏繞多時。也就情動。手略稍鬆。便被他脫下。宋奇生也忙將鞋襪褲子脫去。也無暇脫上衣。就上身交媾起來。汪氏含羞閉目。任其所爲多時。只見他身子伏下。便不見動。汪氏以爲是他洩了。也便由他。好一會。壓得受不得了。低聲道。你下來罷。也不見應。只得將他推下身來。定睛一看。原來宋奇生已送其生了\footnote{雖與阮最一樣死法。却毫不相同。}。汪氏心膽皆裂。忙穿上褲子。沒了主意。他每常認得娘家。如飛的走回去了。這老蜜嘴見宋奇生到汪氏家去多時。不見動靜。心下暗喜。打點明日往他家索謝。且關門坐着聽信。那于魯到下午賣完了菜回來。進門歇下擔子。不見汪氏。走到後面。見睡在床上。到跟前要叫他時。却是個男子。光着下身。心中大駭。再一看時。竟是個死屍。不知何故。忙往外跑。要叫鄰舍。不想驚慌了。被門檻一絆。一交栽倒在門外。不知跌了那處要害。哼也不哼。早已氣斷。過路的人看見。聚攏來看。還以爲是他跌背了氣。扶起他來。方知氣絕身亡。他的鄰舍也來了。進屋叫他妻子要問時。見床上還死着一個。大家都不知是甚緣故。此時老蜜嘴也來。見了心中暗驚。他是緊鄰。少不得同四鄰到縣中去報。花知縣究問他妻子下落。衆鄰說汪氏別無親戚。只有父母家。定然是走了回去。花知縣差四衙帶忤作〈作〉去驗屍。又差人同一個認得汪氏娘家的去拿汪氏。去了一會。都來回話。怍(忤)作回報。奸夫一名。不知姓名。下體赤露。死在床上。親夫于魯跌死在門外。二人渾身細驗。並無傷痕。差役繳籤。汪氏拿到。花知縣叫將帶上來。一見。便怒道。這樣個年小婦人。怎敢大膽謀死奸夫。嚇死親夫。你這一剮是萬萬免不得的了。這奸夫叫甚名字。如何通奸起。可細細的供上來。汪氏哀哀啼哭。便將如何潑水起。以至老蜜嘴說合成奸止。備細說了。又道。奸夫自死是實。並非謀害。親夫跌死係小婦人回去之後。更不知情。花知縣命拶了一拶。敲了五十。口供如前。命放了。叫過老蜜嘴上去問。老蜜嘴也照實供了。與汪氏所說無二。但兩人之死實不知道。花知縣定汪氏的罪案。說道。你向之潑焉之無意\footnote{眞是以莫須有三字定人罪案。}。後來雖是他和奸。然致奸夫喪命者。實首於你勾引之罪也。親夫之死。你即不知。緣因奸夫之死。方致親夫之死。與同謀殺何異。你這惡婦。一剮以償二夫之名(命)也不爲枉。汪氏苦苦哭求。花知縣任性執拗住了。那裡肯聽。又擬宋奇生已死勿論。着本家親人領屍回去。密氏兩家勾引。以致連喪兩命。若加一辟。但二人之死。彼實不知。欲擬杖流。又係婦人。拶一拶。敲一百。責三十板。以正兩姓勾挑之罪。贓銀三十兩追出入官。花知縣定了汪氏的罪。幕賓與刑房書吏再三說罪太問重。未免傷德。他那裡肯聽。只得照他的主意申了上去。那汪氏收入女監。心中癡望。猶以爲上司或批駁。尚有生路。不意上臺竟准行。上本奏過了。奉旨依議。到剮的這一日。汪氏方知。不勝憤恨。道。我之一死固該。但不至於剮。今日陷我至此者。花知縣害我也。呼天自誓道。死後無知則已。若有知。我來世與他爲女。再拚一剮。必定辱壞他的門風。報這一點怨恨。汪氏死後有年餘。花知縣一夜正睡着。夢見汪氏笑吟吟走進房內。向他道。我生前蒙老爺的恩德。今日來相報了。花知縣猛然驚醒。正値他夫人肚痛。生下一女。他心中也甚疑影。過後見那孩子形容宛似汪氏。雖也心中鬱鬱。久久也就罷了。花知縣到底因性拗上。被上司題參。革職回籍。他這女兒過後長大了。十分標致。又聰明伶俐。反疼愛得了不得。阮大鋮聞知他的女兒美甚。央人求親。遂將這女兒嫁了阮優。做了這一番醜事。花知縣方想起昔年汪氏之夢。說來相報的話。不勝愧恨。深悔當日做官斷事任性多疑之錯。憤恨成疾。但閉上眼。便見女兒血淋淋在面前。又是那傷心。也不久身故。可見做官的人不可偏執己見。須要詳細察問。方無差謬。後來有好講因果的人說。這花氏是汪氏托生來報恨的了。這愛奴定是宋奇生轉來。他前世坑了汪氏一剮。今世成就奸情。以完前生宿願。陪了一剮。以償汪氏之死。若果如此言。孰謂冥冥中無鬼神耶\footnote{或曰。汪氏托生花氏。拚一剮以報恨。恐無是理。余曰。不然。怨憤至極。視一死如鴻毛耳。如昔之荆軻聶政爲他人雪恨報仇。尚不惜挾面碎身。何況切己之恨。且係鬼神之事。置之勿論可耳。}。閒話休題。且說阮大鋮在家中時常打聽北京的事體。見逆璫一案漸漸冷下。心中雖放了些。到底有心病的人。未能全釋。毛氏的兄弟毛羽健現做御史。阮大鋮打發大管家龐周利往北京去寄信與他。托他將逆案內中詳細寄一信來。庶幾放心。那龐周利去了有兩個來月。回來了。呈上舅老爺的回書。阮大鋮見了槪不株連之旨。心纔落下。那龐周利稟道。小的路上看見馬六姨來。阮大鋮忙問道。你在那裡看見的。原來龐周利回來之時。到了山東紅花鋪地方。素常知那裡嫖子甚多。偶然嫖性大發。問店家道。你這裡有上樣的好嫖子麼。店家道。近日新來了一個婊子姓馬。叫做馬賽蘭。說是南京有個馬湘蘭。是馳名的妓女。雖文墨大通。却生得不甚標致。這馬賽蘭也識一筆好字。模樣果然生得好。纔三十來年紀。不知他今日有人接沒有。爺要嫖。我叫店小二去看。龐周利道。這好得很。你快叫他去看。沒有客就接了他來罷。店小二去不多時。同了來了。一進門。兩人相見。都覺些面熟。却想不起來。那龐周利聽見他說話是揚州聲音。甚是動疑。遂陡然想起主人的小奶奶馬六姨。却不好問得。你道他兩個是一家的人。又相離不久。爲何就不相識。但馬氏那時是阮大鋮的愛妾。下人何因常見。不過偶然一覩而已。在龐周利還有幾分認得他。在馬氏做小主母時。家下人甚多。那裡個個認得。只依稀似見過而已\footnote{解釋得好。省得冬烘先生許多辯駁。}。兩人吃了酒飯。上床雲雨之後。龐周利道。你可認得我麼。馬氏道。正是呢。我一見面時。就像在那裡會過。一時再想不起來。龐周利笑道。你可是南京阮老爺的小奶奶麼。馬氏吃驚。不敢答應。龐周利道。你不消瞞我。我就是阮老爺的家人龐周利。見過你多次。你難道忘了麼。你跟苟雄逃走了。如何落在這裡。苟雄往那裡去了。馬氏聽說着了脚跟。料瞞不住。二來今日到了這個場中。見了他。竟如見了親人一般。哭將起來。說道。我當日一時念錯。跟苟雄逃了出來。他原是北京大名府人。要帶我還鄕。不想路上遇了響馬。他只該讓他搶去東西。還逃得性命。他仗着有些力量。就動起手來。被三四個強盜一陣亂箭攢死了。把我搶了去。每日輪流淫宿。過了兩個月。被官拿獲殺了。說我是強盜妻子。發了官賣。我再三辯說我是良人妻子。丈夫被害。我是搶了去的。官府那裡肯信。我又不敢說是老爺的小。逃出來的。只得憑他。誰知道賣到水裡。走了這條路。當日好好的在家罷了。若不是奶奶這老淫婦害我。我怎麼到這個田地。龐周利道。你自己做的事。怎麼怨奶奶。難道是奶奶叫你逃的麼。馬氏道。你不知道裡面的詳細。若不因他。我如何得走。遂將毛氏如何私偷苟雄。如何被他撞見。如何毛氏求吿也纔偷了他。後來情厚了。纔同逃出來。事豈不因他而起。叫我如何不恨。龐周利方知內中細故。心中暗喜\footnote{喜得惡甚。所以名龐周利也。}。兩人又風流了一度。次早起來。龐周利就給他嫖資之外。又私贈了他三兩銀子。馬氏灑淚而別。龐周利來家。當件新聞報與主人。見阮大鋮問他。可敢說曾嫖過。只說到了紅花鋪。偶然看見問起來。是如此如此。但把後文毛氏的話截去。阮大鋮聽了。又愧又恨。咬牙罵道。那奴才死得好。這淫婦也現報得好。他只知暢快別人。就不曾想想自己更現報得好也。要知鍾生錢貴二人事體如何。下文便知詳細。

姑妄言十三卷終



\endnotetext[1]{此句原有眉批「侮前人」三字。}

\endnotetext[2]{「哀絰」原作「衰經」,據文義改。}

\endnotetext[3]{「鋮」原作「鍼」,據文義改;下文或同,不贅。}

\endnotetext[4]{「你待」原作「待你」,據文義改。}

\endnotetext[5]{「西江月」原作「江西月」,據文義改。}

\endnotetext[6]{「〖目夢〗〖目夢〗」原作「夢夢」,據上文改。}

\endnotetext[7]{「司新」原作「新司」,據下文改。}

\setcounter{footnote}{0}

\theendnotes

\part*{姑妄言第十四卷}
\addcontentsline{toc}{part}{姑妄言第十四卷}
\markboth{姑妄言第十四卷}{姑妄言第十四卷}

鈍翁曰。鍾生之娶錢貴。大登科之後小登科。完他一對多種(情)種子而已。

鍾趨之讓居。熟竈內添柴。乃人情之常。當思身歷其境。亦是此等否。不可使笑鍾趨也。

易老兒占盡便宜。刻苦一生。一分家資屬於猴子之子。而易氏祖宗不血食矣。易于仁借種家奴。他年產業又將付與勤壽。己身亦斬其祀矣。父以刻。子以淫。易老兒之罪可言也。彼不知易于仁非其子也。易于仁自知之。自欺之罪浮於乃父。後來所以不得其死。且連禽獸假子仍無。此輩戒之哉。

易于仁與妻妾之淫法。已爲奇矣。而奇姐同僕婢之淫。愈出愈奇。其父其女不負其名。眞是異乎於人之奇淫。寫奇姐奇淫。內夾寫一貞姑之貞。貞者更顯其貞。淫者愈覺其淫。是兩襯法。

卜通遇焦氏。彼時未嘗不以爲樂。但恨彼死後無知。未必知水氏之嫁乾女婿。卜之仕呼姐夫爲爹爹耳。

這一回書。鍾生錢貴好合之後。自易老兒娶容氏起。至奇姐死止。全是淫汚之語。到鍾生納代目爲小星。眼目爲之一淸。不意結尾出林報國拿邪道一段。令人氣爽神豪。是用唐明皇羯鼓解穢之法。

\chapter*{姑妄言卷之十四\\
第十四回 多情郞金馬玉堂 矢貞妓洞房花燭\\
附 易于仁父子獸而人人而獸 牛希冉夫妻男作女女作男}
\addcontentsline{toc}{chapter}{第十四回 多情郞金馬玉堂 矢貞妓洞房花燭}
\markboth{第十四回 多情郞金馬玉堂 矢貞妓洞房花燭}{第十四回 多情郞金馬玉堂 矢貞妓洞房花燭}

話說那日鍾生見宦蕚三人正在作惡。忽一陣跑去。不知何故。遂將錢貴扶進房中。錢貴倒在鍾生懷內。柔聲痛哭道。以妾之故。致君受辱。此心如割。恨不欲生。哀哀不止。鍾生將他摟住。寬慰道。彼之怒我。因我挺撞之故。與卿何涉。卿之受辱。實因我在此相累。我甚不安。卿何反言。此一夥狂且舉動如惡犬齚人。不必介意。但我(他)們忽然撇去。不知有何事故。我雖寒儒。諒不懼彼。恐他不能忘情於你。還要受他之累。我今且去細訪。看他們做何行止。再來爲卿設計避之。且自將息。甚勿過慮。錢貴見他說得有理。也便不留。遂道。郞君一有風信。幸即來吿我。鍾生道。卿之事。即我之事。何用叮囑。錢貴又將歷年之私蓄取出。付與鍾生。道。此非我久居之地。此數百金。君可持去。速爲我作從良之計。萬不可緩。鍾生也就接着。道。此雖你之事。乃我之責。何敢尚緩。我中與不中。自有以報命。你但放心。說罷。收在身邊。辭了去了。那郝氏見勢頭不好。避入鄰家。丫鬟嚇得東藏西躱。直到晚打聽得人散。都纔回來。財香也自柴堆下鑽出\footnote{不漏。}。郝氏一進門。見家中打得七零八落。又是那心疼。又是那怨恨。因走入房中。將錢貴埋怨了半夜。錢貴見事因他起。也只得吞聲領受。郝氏同丫鬟收拾破碎傢伙。不必細說。且說那鍾生到家。將錢貴所付之物收好了。見日色已暮。不能出門訪信。小廝拿飯來吃了。且自宿歇。到了夜間。忽聽得門外一陣人聲。打門甚急。鍾生驚訝道。莫非是宦家來尋我麼。那小廝也驚醒了。當是鍾生睡着。叫道。相公。外面有人打門呢。鍾生道。不要理他。正躊躇。那一起人已打進門來。燈籠火把。照耀如同白晝。鍾生想。一間斗室。料難躱脫不能。忙忙穿衣起身。仗膽看時。原來是一起報錄的。衆人見了鍾生。問道。相公可是諱鍾情麼。鍾生道。正是。衆人道。恭喜相公高中。遂將紅報單貼起。鍾生舉目看時。高高中在第六名亞魁。喜不自勝。一來喜的是一介寒儒。平地步於靑雲之上。二來喜的是今得成名。不負錢貴一番苦心。可以娶他報德\footnote{念念不忘。的是多情種子。}。衆人知他家寒。只請他寫了一張賞單而去。連那個雇的小廝也喜歡得爬起來滿地亂跳。道。我相公中了。我相公中了。少間。就有人來拉他去赴鹿鳴宴。至午後。方頭巾。靑圓領。披紅簪花。鼓樂迎歸。到了家中。只見有許多伯伯叔叔。哥哥弟弟。都是十餘年不見面的。擠了一屋子。還有無數從來不曾會過的親戚也來賀喜。因他只得一門小屋。褊窄之甚。連天井內都坐滿了。這些桌椅板凳都是街坊人家情願送來借與他用的\footnote{情願二字妙甚。見得非我去借。乃他情願借來與我耳。把勢利炎涼眞寫得活現。}。梅生雖不曾入場。他有許多親友去考。又一心記念鍾生。不知他中與不中\footnote{世間那得有此等朋友。}。半夜就去看榜。見鍾生名列高魁。心中大喜。早來了替他支應事務。連那陶老也說遠親不如近鄰。走來幫忙。那小廝笑笑跳跳。忙忙的搬東搬西亂跑\footnote{寫到小廝如此忙亂。才見熱鬧之甚。}。鍾生進門。先拜了天地祖宗。然後與衆人作禮。衆人也有送衣服的。送銀子的。送尺頭的。送酒席的。還有送家人來服侍的。鍾生一槪推辭不受。只有叔父舅母所賜不敢過却。只得收了。熱鬧至極。不一時。擺上酒來。斟鍾道喜。大家揖遜一番。坐下同飮。那些族中長輩對鍾生道。我們祖墳上有許多地師看過。說風水甚好。子孫定然要發科甲\footnote{閱此偶憶一笑談。一人新得一馬兵。請親戚同到祖墳祭祖。彼在墓頂左右顧盼。向衆道。這風水也見不得甚麼好。怎就出了我這樣個殺星。}。你又肯讀書。久知道你自然總有今日的與祖宗爭光。果然不錯。親戚們說道。久聞新貴人才貌雙全。自然要高發。但恨小親們都不曾會過。貴人明歲還要連捷呢。我們叨在親末。亦皆有光\footnote{十餘年不見面之叔伯弟兄。從不曾會過之親戚。決無是理。作此語者。特爲炎涼二字加倍出色。}。大家贊不絕出(口)。鍾生一味謙遜。毫無驕矜之色。鍾生當日一介寒儒。雖親叔如陌路。今一旦中了。不知何處來的許多親友趨承\footnote{貧居鬧市無人問。富在深山有遠親。前人已言之矣。}。有幾句感嘆世情。道。

\begin{quotation}

人生何境是神仙。服食求師總枉然。

寒士得官如得道。貧儒登第即登天。

玉堂金馬眞蓬島。御酒宮花實妙丹。

漫道山中多甲子。貴來一日勝千年。

\end{quotation}

梅生向鍾生道。弟今早看榜。見眞先生的令婿不驕\endnotemark[1]干兄也發了。鍾生道。與弟是同房中的。忽然道。可惜可惜。梅生道。干兄中了。兄爲何道可惜。是甚緣故。原來梅生知道干生是鍾趨的棄婿。見鍾趨在座。故意問鍾生以譏他之意。鍾生不好答得。支吾道。弟別有所謂而言。非謂干兄也。只見鍾趨臉紅項赤。內中私故。他三人心照而已。鍾生向梅生道。令表弟多兄昔日同窗。今日又是同年了。梅生道。家母舅積德一生。不能博一第。今日舍表弟徼倖。也足慰他老景了。弟淸早到家母舅處一賀。因兄府上無人。就來相幫照看。鍾生道。足見長兄以骨肉視我。感何如之。彼此閒談。飮至抵暮。方都散去。次早起來。就有個長班來投。鍾生此時正用得着。就將他留下。跟了出門。天啓七年丁卯科南京正主考陳其慶。副主考張士範。稟見過了。又去謝房師。拜同年。回拜衆親友。又上墳祭祖。整整忙了多日。城中那鄕宦財主。見他旣靑年又高中。知他未娶。許多人家倒央人來說要與他做親。他都回已聘過了。一槪謝絕。土山有個財主。姓易名于仁。托了許多親友來說他女兒生得甚美。要贅鍾生爲婿。鍾生苦苦相辭。他家不捨。再三再四央人說合。鍾生見人煩瑣得多了。序齒錄上竟刻上了錢氏。纔止住了衆人。那個雇的小廝。他父親情願將兒子投靠充當家丁。鍾生見這小廝倒還老實。且又伶俐可使。與了他幾兩身價。改名鍾用。留下使喚。這一間斗室不成規模。又托人轉尋房子。又過數日。事情稍暇。着鍾用請了梅生來。坐下。先謝他前日來相幫的情。然後說道。弟有一要事懇煩吾兄一往。務在必成方妙。梅生道。兄請見敎。若可效力。敢不從命。鍾生道。弟春間蒙兄厚愛。攜弟同訪錢姑。兄曾云恐小弟一去。還在他知心之列。不意此語竟成先兆。錢姑見我之後。十分親愛。諄諄以終身相托。弟感其情切。即與之定盟。今敢煩兄做一月下老。到彼對他母親一言。弟欲娶彼女爲室。若要多少身價。悉聽他意。望吾兄千萬玉成其事。小弟容圖後報。梅生聽罷。想了一回。道。吾兄命弟做此些微之事。敢不效奔走之勞。以弟愚見。或行不得麼。兄還當三思而行。鍾生道。請敎何故。梅生道。以吾兄新貴。且又正在靑年。何患無富貴門楣閨閣嬌娃爲配。若娶此煙花瞽女。寧不懼爲他人所恥笑乎。鍾生長嘆了一聲。道。吾兄不知此女與弟萬種深情。豈可相負。彼初會弟時。不鄙我寒賤。即托終身。臨別又贈我數十金爲燈火之費。弟仗此無薪水之憂。始得潛心苦讀。方有今日。且彼矢身自守。雖受伊母之凌虐不辭。人旣有深情於我。背之不祥。古云。海可枯。石可爛。惟情不可移。況士爲知己者死。吾兄請想。弟自幼孤貧。骨肉親友視同陌路。他一遇我即親愛若此。一瞽目婦人勝有眼男兒萬倍\footnote{罵盡世情。}。亦可謂稱弟之知己矣。負心人豈我輩爲耶。至於恥笑。聽之他人。於我何與。況昨日序齒錄上弟業已刻上錢氏是嫡配了。梅生道。原來有這些緣故。弟却不知。弟此時即去。一有佳音。定然回報。起身作別。鍾生送他出門。纔待轉身。他的嫡親叔父鍾趨到門。這鍾趨自與哥哥析居之後。他一腔精神命脈。全在這一個利字上用功。晝夜盤算。屢年來家資也就積得富厚。向日鍾生孤處做貧士時。他全不瞅睬。但因他是尊行。每年新正生辰到門兩次。他家若先有富貴親友在座。恐鍾生襤褸。玷辱了他。還不容進去。三年五載不但不見叔叔家中一盞淸茶。竟連叔嬸的慈顏。同二位堂兄的金面。想見一見。也是難事。鍾趨今見姪兒中了。前次來過。今日又來。鍾生連忙迎接進內。讓他高坐。鍾趨道。賢姪諸事都畢了麼。鍾生應道。都完了。鍾趨道。你今中了。非比往昔\footnote{這四個字。令人痛哭流涕。前也是骨肉。今也是骨肉。不過稍有貴賤之分耳。何便謂之非比往昔。}。我看前日那些親友到此。都沒處起坐。我家房子頗大。向日原住不了。本要分些與你。因你是個貧士。孤身一人。不拘何處。可以安身。如今已是個新貴。尚住在此。不成規模。我今將一宅分爲二院。讓一半與你。已收拾潔淨。可搬了去同住。也與我做叔叔的爭光。鍾生道。姪兒自幼父母見背。蒙叔父撫育成人。今日托庇徼倖。尚未曾孝養叔父。稍報培植之恩。怎敢蒙叔父費心。鍾趨不知姪兒是好話。只疑是向來太情薄了。姪兒拿話來敲打\footnote{或者有些也不可知。}。紅着臉。用話掩飾道。我同你父親是同胞兄弟。非遠族。自家至親骨肉\footnote{貧賤時再沒人肯說這句話。}。怎說這樣客話。當日你做貧士時\footnote{如何算得姪兒。}。我雖是分家各戶。也曾想招攬你家去\footnote{違心之談。}。又想使你受些飢寒困苦。纔肯發憤上進。這是我激勵你的一個美意\footnote{無情之人尚有可恕。惟極無情而專會說假好看假親熱之語爲可恨焉。得利刃斷其舌始爲快。}。今日你高中了\footnote{這才是說骨肉呢。}。自己親叔叔家不住。難道另尋房子不成。豈不怕人笑話\footnote{賊人膽虛。別人那得工夫來笑你。}。鍾生見叔叔如此說了。一者不敢違長者之命。二者也不好十分推却。見得叔叔當日無情的樣子。也就道。蒙叔父下愛。敢不遵命。俟擇吉日就搬過去。說畢。那鍾趨去了。原來鍾趨一者是趨奉姪兒新中。二來見他的棄婿干不驕也中了。鍾趨抱怨兒子。說他二人當日不該攛掇把妹子另嫁。做了這沒良心的事。鍾吾仁鍾吾義又抱怨父親。當初不該希圖豪貴。起這不端之意。恐干生有舊恨在心。怕算計他。故此要鍾生搬來同住。就是干生有甚舉動。看同年的叔父。或可包容。要他做個護身符意思。故當日鍾趨要悔盟之時。鍾生力要諫阻。到叔父家去過數次。不得見面。他看這個樣子。雖見了面。人微言輕。忠言定是要逆耳的。只得罷了。前次梅生說及干生中了。鍾生見叔父棄却此佳婿。由不得口中吐出可惜二字。又問。但這話可是稠人廣衆之中梅生說得的。只得拿別話推過了。惟有鍾趨明白。所以那時面紅耳赤。那干生倒也是天空海濶之腹的人。毫不介意。鍾趨以小人之心度君子。不得不爲之防。他這些族間同親戚們聽得鍾趨送了鍾生一所宅子。大家都來湊熱鬧\footnote{眞令人有時來誰不來之嘆。}。送床帳。送桌椅。送擺設。送骨董。把一所新房塡得富麗之極。鍾生擇日遷移。衆人送席送戲來作賀。又熱鬧了一番。鍾生的舊房因眞敎官在任上。知干生是他令婿。將房子付他收管。干生也送還典價。鍾生進了新房。又買了個丫頭配了鍾用。又投了兩三房家人。尋了兩個上樣的丫鬟。預備服事錢貴。這番規模。不是前番那寒士氣象了。你道鍾生這銀子是那裡的。就是錢貴付他的了。他想。鍾生要中了。自不必說。設或不中。恐鍾生無顏。即欲爲他贖身又無力。故將歷年私蓄數百金盡付了與他。就不怕又蹉跎了。這就是錢貴一片深心。鍾生今已中了。要娶他。少不得把家中收拾個待缺鴛鴦社。以俟新人。且說那錢貴自鍾生去後。心中也甚憂疑。次早不見動靜。疑宦蕚或能忘情。稍放下了些。飯後正在房中兀坐。忽聽得街上〖口么〗喝賣題名錄。忙叫代目去買了一張進來。命他一看。念到第六名上就是鍾情。錢貴見他中了。眞喜歡非常。忙盥手焚香。拜謝了天地。在大士像前也叩拜了。此時那宦蕚的事被這喜一沖。竟撂在東海傲來國去了。叫代目請了娘到房中。將他與鍾生如何定盟。許中後娶他的話。細說一遍。又道。他今已高發。定來娶我。母親尊意如何。郝氏聽了。半晌道。哦。怪道你向來不肯接客。原來就是爲他。我正疑你旣不留人。爲何又留他住許多日子。我看他人品果然生得好。但不知心地如何。今已高中。兒呀。你不要太認眞了。從古來負心的人可是一個。他當日是個寒士。見你與他綢繆。便發下千般海誓。萬種山盟。今日做了貴人。怕沒有富貴人家扳親。他還肯來想着你。錢貴道。鍾郞決不負我。倘有人來作伐。萬望母親依允。郝氏道。你如今旣不接客。留你何益。我們這樣人家得個舉人女婿。還有何說。且看他來與不來。再做道理。不覺過了十數日。郝氏到錢貴房中道。我兒。我做娘的話何如。他若有心於你。爲何這些日子還不見一些音耗。多管是成畫餅了。錢貴道。鍾郞心跡。兒知之甚深。定非負心人。倘彼背盟另娶。兒披剃入空門。長齋繡佛。自誓一死。不復再嫁矣\footnote{有他母女這兩番議論。愈顯鍾生多情。錢貴多識。}。正說着。聽得外面有人叫道。錢媽媽在家麼。郝氏忙走出一看。原來是梅生。讓進客屋中坐下。說道。相公許久不光顧了。今日何幸降臨。梅生道。我前中秋次日在此的。未曾得會媽媽。今日特來替媽媽道喜。郝氏道。光(老)身素履平平。並沒有甚麼喜事。怎敢勞相公大駕。梅生道。我來給令愛作伐。送一個新貴女婿與媽媽。豈非大喜。郝氏道。請問相公說的是那一家。梅生道。就是我敝友鍾兄。他托我來致意媽媽。他說春間在府上時。承令愛不棄。曾與定盟。約過中後方娶。果然天從人願。竟僥倖了。因連日有事。未得遣媒。至今方遐。特特懇我來奉懇。但要多少聘金。聽憑媽媽尊意。郝氏聽了暗喜。說道。鍾相公今是貴人。但恐小女無福。不敢仰攀。況小女係老身親生。安有要身價之理。梅生見他說不好要財禮不敢仰攀的話。疑他推托。說道。媽媽不要錯過這門親事。說起我這鍾兄。眞情種也。昨日許多富貴豪門愛他的年靑品秀。欲得之爲婿。他因與令愛有約。皆苦苦一槪辭絕。於序齒錄上已將令愛刻上做嫡配了。他一片心思注於令愛。今誠懇托我來求。望媽媽慨諾。成其好事。媽媽不必過謙。況成就之後。媽媽就是岳母了。也得個下半世快樂。豈不甚妙。郝氏道。相公見諭。老身安敢不依。但憑鍾相公尊意。擇吉迎娶便了。梅生聽了。道。旣承金諾。我去回復了鍾兄。俟定下吉期。再來通信。起身作別。郝氏道。還有一說。鍾相公處聘金。老身一絲不要。但小女去時。老身也沒有甚麼妝奩。煩相公轉達。梅生道。不要聘金就是媽媽盛情了。豈有爭賠嫁之理。說了。辭去。那郝氏笑盈盈走進房中。對錢貴道。兒呀。恭喜你了。你好慧心巨識。鍾相公果煩梅相公來替你作伐。再四求我。我已依允。兒呀。你這一嫁去。將來就是夫人命婦了。他母女二人滿心歡喜。自不必說。先梅生與郝氏說話時。錢貴都聽見了。聽得說多少名門巨族要把女兒嫁他。他都辭却了。序齒錄上已刻上了錢氏。錢貴更感他的深情。又喜自己有知人的見識。錢貴許了鍾生。連那代目聽見了。也私喜得了不得。這是何故。他原是好人家兒女。被老子不長進賭輸了准與鐵化。後跟了陪嫁到童家。一笑之過。打發出來。不幸被媒人同惡僕將他送入火坑。喜得數年來因錢貴疼愛。他雖十八歲。尚還保住了女身\footnote{提此一句。爲鍾生小星作地。不然。鍾生妻妾無一個處子耶。}。在這門戶人家。將來作何結局。今聽得錢貴嫁與鍾生。他定然隨去。也巴個出頭的日子。心中滿擬錢貴離不得他。或開恩以小星處之。得爲這美郞君之妾也。不枉當初會時那一番舉薦。他自有這種私心。豈不歡喜。再說那梅生回復了鍾生。擇了好日期納采下聘。隨就娶了來家。他一個新舉人娶親。自然熱鬧。彩轎花燈。藍傘火把。一路上樂聲鼎沸。燈燭輝煌。到了家中。三元百子轟雷震耳。花燭前引。紅氈匝地。扶入洞房。交杯合巹。然後上床。這正是。

\begin{quotation}

畫堂前依然兩個新人。牙床上各出一般舊物。

\end{quotation}

他夫妻二人情義相投。如魚似水。因是貧賤中結下來的。更加親愛。到了次日。賀客塡門。酒筵鬧熱。不消說得。彼時有人笑他說。他一個少年舉人。要甚沒(麼)好人家女兒怕沒有。却要娶一個瞎妓。也有的道。他雖然發跡。不忍負心。到底是讀書人不同。街市上紛紛議論。再說當日土山住的有一個土豪易于仁。他這個姓城中甚少。惟獨土山十戶中倒有四五家姓此。這土山也有數千人家。好一個富庶地方。易于仁當日他父親遺留約有千餘金之產。他雖一字不識。一竅不通。却囗田貪刻。善逐十一之利。如靑黃不接之時。窮家小戶沒得吃了。借他一石穀。九升斗平平量出。到秋收徵還。足大斗推尖量入。一石五斗。名曰加五。已將對合。他豈肯白借與人。有房子田地的。就指房地寫文書做當。沒有房地的。連妻子兒女都當與他。或借銀子。定是五分行息。九五等子稱出。還是九三銀。還時足紋足等。人若不來還。他也不催。窮人家見債主不緊。樂得且捱。不想數年後。被他本利滾算。房地人口都屬了他。眞是個爲富不仁。殺窮人做富漢的惡物。二十年來被他掙了一分大大的產業。雖算不得巨富的大財主。但在這村中。就要算他第一把交椅了。左近一帶田地。十分中有六七分是他的了。所以他家的佃戶也甚多。這易于仁不但在銀錢上刻薄。在那婦女身上更貪好得異常。講起他的這個淫字來。眞出人意外之想。他這種性情。必定生身有個緣故。待我將他的出處細述。便知分曉。易于仁的父親易老兒。他承受祖遺產業。不過數百金。家無多的人。只他夫妻兩口。並一房僕婦使用。生之衆。食之寡。漸漸積趲起來。後來又放些賬目。頗自飽暖過日。却有六旬。尚無子女。後來妻子亡故。鰥居了有半年多。村中有一個姓容的。借過他十兩本銀。歷年欠下利息。算來共有數十金。日漸窮乏。無可償還。那易老兒常常來索債。這容老兒有個女兒二十歲了。曾招過一個女婿。死了也將一年。一日。他夫妻父女在一處商議。容老兒道。我想了一策。你們看可行得。易家這宗賬萬萬不能還他。他肯容我白用的。設或吿起官來。實是我們理短。那時如何是好。我想來女兒年紀尚小。少不得還要嫁人。易老兒也是個孤身。竟煩原中去說。把女兒嫁他准賬。他料還不起。大約也肯。他雖然年紀老了。若還女兒命好。生得下一男半女。這分家私豈不是他娘兒們一生受用。你說可行得。那婆子道。你這主意倒好。但不知女兒心裡何如。容老兒就問女兒道。大姐。妳的意思怎麼樣。那女子自幼隨着父母過窮苦日子。雖嫁過丈夫。也不過是力田度日。飢寒二字自不能免。素常也知道易家寬裕。有何不願。俗語說。八十歲的媽媽嫁人。不圖生長只圖吃。遂答道。這憑爹媽做主。怎麼問我。那容老兒知女兒是肯的口氣。滿心歡喜。忽聽得門外叫道。容老爹在家沒有。容老兒知是那保人的聲音。正中下懷。忙迎出來。道。在家。那保人姓終名仁。放下臉來。道。一家放賬。一家用錢。我不過當日吃得一杯水酒。彼此爲好來。你如今沒得還他。易老爹成日到我家來聒噪。我耳朶都吵聾了。你摸摸良心。過得去過不去。容老兒一臉的笑。道。怪不得老爹生氣。我正要來尋老爹說這話呢。我如今有個主意同你老人家商量。成得成不得再講。遂拉着他的手。笑道。家下不便。到隔壁酒鋪中坐坐講罷。原來這終仁酷好此物。各處與人說事。無非覓鍾酒兒潤喉。聽見約他酒鋪裡坐。惱容變做笑面。道。怎好相擾的。容老兒道。這甚要緊。若事成了。有大大的兩罎吃呢。遂同到酒鋪中來。要了半斤燒酒。一碟炒豆。一碟腐干。一連讓了他三杯。那終仁道。你方纔說有甚主意。你說了我看。容老兒道。我當初借易老爹只十兩銀子。這些年來利上滾利。纔聚上許多。如今我家日食都艱難。瞞不得你老人家。那得還有錢還債。我只有一策。我家大姐是你見過的。也不爲醜。女婿又死了。他今年纔二十來歲。水也似的。後生料道也守不得。今易老爹的奶奶也沒有了。我的意思把我家大姐嫁他。憑他做妻也罷。做妾也罷。准了這賬。除了這法。不要說私要。就是到官。我也不過是條老命。況官府也不追比私賬。但你老人家是原中。拖累你跪官跪府。我過意不去。全仗你老人家美言一句兒。倘或成了。彼此有益。就做着他不肯。我們儘到他是理。又可以擋他些日子\footnote{極寫窮人之苦。眞可謂無聊之極思。}。你老人家怎麼說。那終仁道。我去說了看。大約着十金本錢得個老婆也肯。還少甚麼。你我都是莊農人家。他不過比我們多有幾個錢。又不是鄕宦。甚麼叫做妾。竟說嫁他就完了。容老兒道。這更好了。事成了。少不得請你老人家幾醉。兩個把半斤酒飮完。那終仁道。我此時就去。你在家等着。看他怎麼說。我就來回信。站起來道。且不道擾着。倘這媒做成了。吃喜酒再一齊道謝罷。容老兒道。這好得很了。但願事成。自然奉請。二人大笑。一齊出門。一別而去。那終仁到易家來。遠遠見易老兒站在門首。心中暗喜道。這事有幾分興頭。遂上前道。我往容家去了來了。有一件事來和老爹商量。易老兒讓進客位內坐下。道。他怎說。終仁道。他家實在貧得可憐。飯還沒得吃呢。方纔他說就吿到官也不過是條老命。他只有個女兒。你老人家也見過的。他如今情願嫁與老爹准了這賬罷。叫我來說。老爹的意思是怎樣。看官聽說。大凡人生在世。色慾之心入土方休。這易老兒他當日三四十歲時。守着那婆子。只以銀錢爲急務。生子一事倒還不十分着急。後來五十多歲。手頭厚了些。未免就憂子嗣。雖有此心。因那婆子情性有些古怪。不敢妄想。今鰥居了半年。要想娶個妻子。一來作伴。二來圖他生子。十分醜的又難爲情。略像樣些的恐又費錢。兒子固要緊。銀錢更要緊。況且又怕人嫌他老了。少年婦人又未必肯嫁他。他原圖生子。若娶個老的來做甚事。今聽見這話。況容家女兒是時常看見。人物又好又伶俐。年又少。無限歡喜。答道。我家正少個當家的人。我也久有此心要求他。怕年紀不對。不敢開口。旣承他美意。是極好的了。就煩你做個媒。別的不敢許。喜酒是有得吃的。煩你去問問他要怎麼行。幾時可娶。問明白了來。我預備酒候你來起媒。那個終仁聽見備酒候他。如飛而去。不來(多)時便來。道。恭喜老爹。準備做新郞罷。一眼看見桌上四個菜碟。還有幾塊醃鴨蛋。一大壺的酒。歡喜非常。易老兒笑道。且坐下吃一杯再說。他哈哈笑着坐下。易老兒篩了一杯遞過他。他接過來一嘗。是家中窨的封缸。大喜道。好東西。一口汲乾。道。好酒。老爹旣費事。我再吃幾鍾再說。連飮過數杯。夾塊醃蛋壓了壓。說道。容老爹說他家是一絲嫁妝是沒有的。不敢講。行下憑老爹。日子也儘在老爹。隨早隨晚。揀了日子。只管娶他。不過是個空人。易老兒道。我們南京鄕風用禮金。原是與他買嫁妝的。執盤錢是與女家買零碎雜用。他旣沒得賠。我家的箱櫃床桌都有。禮金執盤不必用了。他家旣艱難。女兒嫁我一場。原文書還他不用說。我不但不要他一絲東西。我還封幾兩折果餅的銀子。送他買柴米用罷。你道如何。那終仁道。這是老爹的情。他更感激了。復哈哈笑道。人說骨頭面上的筋。老婆面上的親。你老人家奶奶還沒進門。就疼起丈人來了。易老兒也笑道。禮是不下了。再煩你問他。若不怕忌諱。我死鬼的衣服首飾還有些。將就用罷。再者。我一個老頭子娶老婆。他家一個後婚嫁人。也不必揚名打鼓的。揀個好日子。擡了來罷。我家中備個酒水。豈不兩家省事。你吃了酒。煩你再走走來。終仁道。我吃了這一壺就去不得了。我去了來吃罷。易老兒道。更妙了。我殺個雞請你。他說道。老爹太費事了。去不多時。又回來道。他聽見老爹送他折果餅的銀子。感激得了不得。滿口說任憑老爹之便。他是不忌諱的。易老兒也甚歡喜省費。少刻。煮了一隻小筍雞。五個白煮蛋。同他飮完酒。又拿飮(飯)來吃了。終仁起身作謝。易老兒\endnotemark[2]道。等我揀了日子。再來請你說信。終仁去了。易老兒次日煩了個敎書先生。看了一個好日子。打點下頭面衣服之類。又封了六兩銀子。把原契查出來。家中煩人來預備了幾桌酒席。請了終仁來小飮了。一面煩他帶着衆人送了去。次晚娶了來家。吃酒成親。不必細說。那易老兒許多年守着個老婆子。今日忽然得了這樣個妙人兒。一來怕他嫌老。二來想他生子。因他自幼不曾斲喪過。年雖六十。倒還精壯。三兩日之內。定然竭力舞弄一番。那容氏當日過的是裙布荆釵。黃虀淡飯的日子。還要燒火做飯。洗衣縫補。雖然招了個丈夫。日間做工累得七死八活。夜間枕蓆之上還有甚高興。倒下頭直到天亮。間或十日半月動作動作。也不過應應卯。點綴而已。至於其中樂處。並未曾嘗得。今日到了易家。雖不能錦衣玉食。頭上竟戴了鍍金銀首飾。身上穿了松江細布。竟還有\endnotemark[3]幾件上蓋紬衣疊在箱內。飮食雖不能日日雞鴨。因易老兒圖他歡喜。三五日中定有些魚肉到口。這是他當日成年不得嘗的罕物。而且有個家人使用。終日惟有飽食高坐。到了夜間。在家時床上鋪一條草薦。上面一條燈草蓆。蓋的是粗布被。如今是大厚的褥子。墊着紬面布裡的被。又溫又軟。好不受用。那老兒又常常竭力要種種子。容氏方知天地間。日裡有這樣安富尊榮。夜間床龦(幃)中夫妻有此種樂處。不但不嫌他老。把他竟當老寶貝一般。十分恩愛。那易老兒先猶恐他憎嫌頭上嘴上的這幾根銀絲兒。今見他着實相親。那愛他疼他也足足有二十分。易老兒一夜笑向他道。我初娶你時。怕我年紀大了。你見我這幾根白鬍子。同你這樣個嫩面挨着。不知怎樣憎嫌呢。誰知道你倒疼起我來。容氏雙手摸着他的臉。道。我看見黑鬍的人多。見了你這花白的。覺得分外有趣。叫我怎麼不疼愛呢。易老兒倍加歡喜。愈增恩愛。但這老兒娶他來時。以爲一進門下了種就有收成的。故常常去盡力鑽硏。誰知到半年後。竟毫無影響。他有年紀的人。幾個月種也將枯了。累得力盡筋疲。便興致索然。精神倦怠起來。不能如初了。但這樣一個嫩婦在一床同臥。又不忍久疏了他。十日之中。免不得還強掙着應應故事。後漸漸覺有些支撐不來。只得一上床就假鼾睡。容氏毫不驚他。以爲他眞是睡着。反替他塞塞被。自己倒離遠些。易老兒甚不過意。他心中一來是愛容氏。二來感激他這相親之情。夜間雖不能用力。日裡只得買好東西給他吃。或容氏要買甚麼。他無不奉命。雖暗裡心疼。無奈本事不濟。只得拿勤勞折之。那容氏處在樂境。未免靜極思動。見丈夫相待甚好。只得如守活寡一般。心中也覺難過。況當日嫁他家。穿吃猶次。原圖生個兒女。以爲終身之計。今見老兒連種都不能下了。如何還望收成。未免又暗自着急。終日悶悶。一日。那家人媳婦進來。笑嘻嘻的道。門口賣的好一個大猴子。差不多打到我的肩膀。又會翻筋斗。又不咬人。乖巧老實得好頑。容氏倒也是無心。想道。我閒着一點事也沒有。買了來頑耍解悶也好。問道。老爹呢。那媳婦子道。老爹也在門口看呢。容氏道。你去請了來。去不多時。易老兒進來。容氏撒嬌撒癡的道。我成日家坐着。悶得慌。聽見有個賣猴子的。會打筋斗頑耍。要是賤。你買來拴着給我解悶罷。那老兒要奉承他。連忙允諾。忍着心痛。顧不得貴賤。買了牽進來。容氏一看見他。有三尺多高一個大猿。問道。他不咬人麼。易老兒道。很老實。不怕的。容氏笑吟吟走進前來。〔道。〕打個筋斗。那猴子就翻一個。他喜歡得了不得。又道。再打一個。那猴子果又打一個。容氏忙取些飯來與他吃。易老兒就把他拴在堂屋門槅子上。過了幾日。但是容氏在他面前過。或餵他食。他就把裙子一掀。伸頭向胯襠一張。若同易老兒在跟前。他就不敢。容氏先也不覺。後來幾次如此。忽然想道。這畜生眞有些古怪。我走去站着。看他怎樣。剛走到跟前。他又來一掀一張。容氏站着不動。他見容氏站住。他就坐在地下。兩腿大揸。拿手弄他那通紅的㞠子挺硬着。有大指粗細。四寸來長。兩手對着一陣捋。\endnotemark[4]冒出些精來。又起來掀開裙子張看。容氏恍然悟道。我也曾聽見說猴子通人性。可以同人弄的。這畜生想是看上我了。他一個少年婦人。易老兒久矣吿免。一月之中。見他經盡之後。圖徼倖於萬一。種一次子。何能解饞。正在無可奈何。今忽見此。一時間淫心大熾。想道。他這東西也還不十分渺小。長處同老兒差不多。不過略細些。要弄也盡可弄得。我試試看他怎麼樣的。遂把院子門拴上。這日。易老兒有人請去說話。他那家人除掃地送飯之外。再不上來的。容氏又走到猴子跟前。他又來一張。容氏蹲下。伸手去摸他的厥物。那畜生果靈。一交睡倒。將腿大揸。硬邦邦一個㞠子憑他捏弄。容氏也替他捋了幾下。此時慾火如焚。站將起來。把衣服擄起。褪下褲子。露出那件妙物。那猴子一見。就不是他了。攛起來一把抱住。把容氏倒嚇了一跳。只見他抱緊。一個㞠子向小肚子上混戳。容氏向他道。你放了我。帶你屋裡去。那猴子也不知他懂不懂得。容氏伸手去解那皮條。他竟像知些人事的。放了手即跳下來。容氏一手提了褲腰。一手牽着他到床前。拴在欄杆上。上床脫光仰臥着。那猴子跳上床。也竟知爬上肚子來弄。但他兩條後腿是站着。婦人臥着低。兩下就不着。容氏急得心裡難過。猛省道。是了。凡是畜生都從背後來\footnote{余聞之象奴云。象之交合自對面來。與人同。不知果否。}。必定他也是如此\footnote{好悟省(性)。}。將他推下。翻起身。馬爬着。果然那猴子爬上脊背。戳了幾下。一下弄了進去。也知往裡送送。送到了根。不住抽將起來。雖然不能大樂。叫做飢不擇食。覺得比老頭子陽物還堅硬些。容氏淫情大動。竟乏了一度。滿心暢快。那猴子也洩了下來。容氏睡倒。想道。無意間買了他來。竟有這些妙處。不多時。只見那猴子又來推他。像個要他起來之意。容氏覺是如此。又起來爬着。他又上身弄了一次。容氏又睡下。不一盞茶時。他又推他。容氏想道。他旣通人性。就是對面也可。背後弄的到底不妙。遂把枕頭墊在股下。牝戶大高的腆着。那猴子也就爬上來。容氏用手將來㞠子。送出牝門。果然更妙。那猴子弄了一會下來。容氏索性睡着不動。猴性最淫。若雌雄拴在一處。一日要幹數十次。他不多時又爬上來容氏肚子去弄。如此者數次。容氏恐易老兒回家。起來穿衣。那猴子還抱住不放。容氏笑對他道。你放我起去。怕老爹回來。改日再同你弄。你聽我說。你若有靈情。要有人在跟前。切不可混拉我。那猴子也似有知。就放了他。容氏穿完。衣(依)舊牽他拴在堂屋內。開了院門。從此後。容氏或同易老兒在堂屋內。或有人在跟前。那猴子或在地下爬。或是坐着。總不近前。但是沒人。他見了就抱住不放。有求歡之意。容氏歡喜不盡。私自拿錢買果子饝饝與他吃\footnote{也算賠錢養漢。}。但是易老兒不在家。就同他弄上幾次。如此者又有半年。那容氏竟有三四個月經水不行。想道。不要是這畜生弄了胎在肚子裡罷。也還疑未必是。恐是經閉。又過些時。肚子一日一日大起來。裡頭梭梭的動。纔知果是懷了孕。心中倒捏了一把汗。不知生下個甚麼怪物來。易老兒知道容氏得了胎。心中大喜。到滿足之日。做衣裳。請收生婆。又接了他的母親容媽媽來照看。好生快樂。只容氏懷着鬼胎。不知是禍是福。到臨產之時。很快當。竟生下一個兒子。與人一樣。只是小些面目。尖臉縮腮。究如猴形。但只沒毛。容氏暗喜是不消說了。易老兒六十多年紀纔得了這個寶貝。那裡還管他醜俊。送喜蛋喜果。吃喜酒喜麪。熱鬧得了不得。容氏一個月不出房門。那猴子不見他。時常在外吼吼的。容氏恐他餓了。叫人常拿飯與他吃。他也不吃。餓極了纔吃些。容氏知是他想念。因那僕婦時刻在屋裡服事。不便出去。心中好不難過。滿月這日。衆親戚攢分子請易老兒去吃喜酒。那僕婦也下去了。容氏記掛着猴子。走出來看他。那猴子一見。抱得緊緊的。那種親熱了不得。但只說不出話來。容氏這一個月也疏濶他了。牽他到房中上床。猴子一眼見床裡睡着那小孩。他到跟前撫摩。有無限疼愛之意。容氏見了。暗暗點頭嘆息。他摸了一會。然後纔同容氏弄。那裡還肯住。足弄了七八次纔歇。此後容氏愛他眞如小夫一般。且說這孩子易長易大。到了五歲時。易老兒買了個十一歲的江北丫頭背他頑耍\footnote{丫頭何必曰江北。蓋江北粗夯丫頭其價甚廉之故。}。夜間容氏帶他睡。這孩子父母旣疼他。他却也是個頑皮。一日到晚憨跳不住。他心疼那猴子了不得。問父母要錢。無樣不買與他吃。那猴子也有奇處。一見了他就翻筋斗。同他親熱頑耍\footnote{古人戲彩娛親。他却翻筋斗娛子。世有娛子者。皆猴類也。}。容氏覺他是天性所感。暗暗失笑。這孩子到了九歲上。就刁鑽古怪起來。見帶他的那丫頭已十五歲了。無人處或抱着他亂聳。或挖他股後。他雖九歲。身材矮小。像別人家五六歲的孩子。那丫頭見小。只說憨頑。也不理他。一日。這丫頭正帶着他。一時尿急。說道。你頑一會。我就來。忙往裡走。他悄悄隨後跟去。見那丫頭走到床後邊去。他躱在床頭張看。見丫頭擄起後面衣襟。褪下褲子。一手揭開淨桶蓋。坐下溺尿。他一見了那老屁股。那個小㞠子也就硬起來。那丫頭溺完了。去拿淨桶蓋。失手掉在地下。彎腰去拾。胯下那件東西正與他覿面相親。他好不迅速。一手扯開褲子。捏着陽物。一攛到跟前。雙手抱着屁股亂聳。但他矮小。夠不着妙竅。那丫頭先吃了一驚。一回頭。見是他。說道。你這是甚麼頑法。快放手。不然我叫老爹奶奶呢。他道。就是叫老爹奶奶我也不怕。憑你怎麼的。給我弄弄纔罷。一面說着一面聳。那丫頭十五歲了。也有所知。況常見那猴子不住向他弄那㞠子。也有些動心。因不知其味。所以不去貪求。今被他腿上戳得癢酥酥的。便道。你放手。我同你說好話。他道。我放了。你會跑了去的呢。丫頭道。我不跑。你想這個地方怎麼弄得。不怕奶奶看見麼。我同你到倉房裡頭那間空屋裡去。他道。你會哄我的。你先與我摸摸親個嘴着。那丫頭道。憑你罷了。他纔放鬆了。伸手去摸摸那縫兒。更覺興發。拉着那丫頭。叫他彎下腰來。親了個嘴\footnote{彎下腰來親嘴。妙。高矮夠不着之故。}。纔放了手。丫頭笑道。豆兒大的人。也會幹這些營生。見他的陽物雖是一個尖頭。竟有那猴子的長。還略粗些。暗想道。恁個小人兒。倒有恁個大東西。我常見街上熱天。小孩子們光着身子。十二三歲的還沒有他的大呢。丫頭塞上褲子。蓋上淨桶蓋\footnote{細。}。同他拉着手。悄悄到倉房內。就在地板上做了雲雨之場\footnote{昔有云。行雲行雨在何方。土坑。那裡有春風夜月銷金帳。今當改一字云。行雲行雨在何方。板上。}。兩人弄了一會。彼此不知其味。嘗新而已。此後他兩個不拘何處。見無人就弄。那丫頭怕褪褲子費事。把褲襠縫拆開了些。好不便宜。左右無人。擄起衣服就幹。又過了一年。他已十歲。送到學堂中去念書。先生見他相貌異乎於人。起個學名叫易于仁。又道。易于爲仁是極好的。豈不暗合二義。這易于仁見了書本就打瞌睡。一日書也背不得一句。倣也不知寫的是甚麼。倣影在半邊。他畫的在半邊。連字形都認不出來。念了幾個月。一個字也認不得。先生也打過多次。總是如此。只得由他。却又頑劣無比。先生一不在學堂。不是〔同〕這個學生打。就是同那個學生罵。把別人的書都扯破。筆也塗爛。放學吃飯。再無一次不同學生吵鬧。先生見打不過來。恐怕鬧散了學館。對易老兒說知。易老兒心疼兒子。叫了回來。他瞞了父母偷些錢出去。到山僻處等着。遇有扒柴的村婦。不論老少好醜。送幾十文餞。要求野合。這些婆娘可知甚麼羞恥的。況見他一個小孩子。要試他可會。樂從的也甚多。就有不肯的。歸家吿訴丈夫。也只說他小孩子頑耍。未必是眞會此事。到十四歲上。老兒又想。孩子要替他娶個大幾歲的媳婦。遂娶了本村山後袁家的女兒。這袁老兒幼年時是個貝戎出身。獲了利。做起人家。雖然改了舊業。還是個橫行村坊。損人利己的惡物\footnote{入此一句。妙。不無其父如此。安得生此辱門之女。}。知易家富厚。故此結了親。這女子雖算不得標致。也還生得白〇〇〇嫋嫋娜娜的。易于仁從未遇此。以爲是天仙降世了。他〇〇〇個陽物竟有六寸來長。把這女子一夜弄了七八次。喜〇〇〇子十八歲了。身子還結壯。起初二三次他還受了。後來〇〇〇弄。袁氏不依。他就混咬混抓。又不好叫喊。只得依他。一夜未曾合眼。下身腫痛異常。次早掙起來時。對鏡梳洗。看見自己的臉形都脫了。一個臉萃靑。眼都睜不開。飯時他母親來開臉。見了大嚇一跳。不知是怎的。來問他又不肯說。少刻。沒人在跟前。悄悄又問女兒。他含淚不言。被娘逼之再三。方把緣故說知。那娘癡了一回。想道。看〈人〉不出恁點點人兒這麼利害。我先還疑他是個小孩子。未必知道做這事。誰知他有這樣本事。我們做了半世婦人。還不曾經着這樣好東西呢\footnote{大有垂涎乃婿之意。}。因笑着安撫女兒。道。我兒。這是你的造化。反哭甚麼。那女兒急了。道。好造化。再一夜我可死了。娘道。還有嫁一輩子丈夫不能夠這樣的呢\footnote{聽此婦語。大約其夫甚是不濟。}。女兒道。我不信。像刀割的一般難受呢\footnote{昔有一笑談。一女出嫁。伊母親自送親。留下戲酒過夜。婿女成親。女不能禁他。喊殺人了。其母忙到房中。道。姑爺。今日個大好日子。怎拿刀弄伏(斧)殺人。是怎麼說。其婿跳下床。拿陽物與看。道。岳母請看。有這麼把殺人刀麼。}。他娘道。我兒。我做娘的有哄你的麼。今日夜裡就好些\footnote{要知山下路。須問顧來時。}。袁氏聽說。料娘未必哄他。纔放了心。到了夜間。果大得其趣。雖還有些微疼。因樂多而苦少。便不覺了。過後袁氏反不肯放鬆。一夜少了三五次。他不肯歇。他夫妻眞可稱爲佳配。又過了兩年。那易老兒年將八十。老病死了。少不得開喪出殯。容氏從丈夫過慣了省儉日子。皆不過從省而已。又有兩年。易于仁已十八歲。只像個十二三歲的孩子。雖然矮小。却生得廝趁。頭臉手脚身材。無不小巧\footnote{無不小巧。惟獨陽物大。世果有其人。}。倒也不覺醜看。比那種粗肥而短的人強了許多。但他性情比父親還刻薄。不過只知有己。而再不知有人\footnote{世上而今半是君。}。家私倒比他老子在時還厚些。容氏已四十外的人。自娶媳婦之後。淫興也就闌了。那猴子也老得動不得了。一日。那猴子死了。容氏暗暗墮淚。對兒子道。這猴子在我家二十年了。他當日同你頑耍。好不乖巧。今日死了。你可買口小棺材裝上。埋在你父親墳後罷。那易于仁也不覺慘然。道。我的意思正是如此。遂買棺材將那猴子埋於易老兒墳後。容氏到五十歲那一年。得病不起。忽夢見那猴子來說道。我同你恩情一場。兒子是我的骨血。我同你緣法來了。可同我去。再做夫婦。明日日中我來接你。你不要當是夢。容氏驚醒。道。你如何又會說話了。猴子道。我如今如何還比得生前。當日雖不能說話。你說的話我都懂。你可記得年先的事麼。你今大限已終。你可對兒子說知我是他生身之父。使他知道是我的骨血。也不枉我與你十多年的恩情。臨去。又囑道。千萬記着。撒手撇去。容氏哭醒來。原來是一場大夢。漸覺沈重。想着夢中的話。要說又難啓齒。不說又辜負了他。事在兩難。只是掉淚。次早。兒子媳婦來問病。道。今日可好些麼。容氏道。我今日日中就去了。還好甚麼。易于仁驚道。奶奶怎知道。容氏道。我做夢來。是你爹爹說的。易于仁道。夢如何信得。但見他漸漸有些危勢。將到午時。見他不住往外望。只是隨(墮)淚。易于仁心疑。問道。你老爺(人)家望甚麼。容氏道。你爹爹來接我了。易于仁見他有些不好的光景。也就流淚。說道。奶奶。你有甚麼放不下的話。替我說說。容氏道。你已成人娶媳。又會成家立業。我還有甚麼放不下的。易于仁道。旣然如此。你老人家爲甚麼只是傷心。那容氏悲咽了一會。却哭不出淚來。掙着說道。你爹昨夜叫我有句話對你說。我不好開口的。他此時又催我說。易于仁道。旣爹有話。奶奶你說何妨。容氏把眼望望媳婦同下人。易于仁會意。都叫出去。關上門。復來問道。奶奶。有話你說罷。容氏纔要開口又忍住。易于仁也傷心起來。慟哭道。一個人也不在跟前。我又是你養的。有甚麼話說不得。那容氏一把拉着他道。我。你不是你這個爹爹生的。易于仁驚道。我是誰生的。容氏道。你就是死的那猴子的兒子。你不見你像他麼。我說的你爹爹就是他。他再三囑咐叫對你說。啓出他的棺材。同我葬在一處。我昨夢見他。此時來接我了。說完。聽得喉中嘓兒一聲。漸漸沒氣。易于仁大哭。開門叫進遠(袁)氏衆人來。他衣衾棺槨早已預備停當。裝殮了。喪事較易老兒死熱鬧許多。不消說得。他叫匠人打了一口上好棺材。將那猴子的棺材啓出來。就裝在這口材內。做了個外槨的意思。也油漆了。他到送殯這一日。將他父親的墳刨開。叫挖了一個大坑。先下了容氏的棺。又叫將那猴子的棺材同容氏一並放着。易老兒的材倒還離得遠些。然後掩土。親友驚疑問他。他謊說道。當日先父遺年(言)。說養了他二三十年。叫埋在墳中相伴。衆人也不好細問。過後。袁氏偶然想起這事。問他道。當日老爹臨危。我也在跟前的。並不曾聽見吩咐把猴子同葬的話。易于仁自以爲猴子生人是件異事。遂將詳細吿訴了袁氏。那袁氏一日回娘家。因同嫂子娣妹們說閒話。大家講新聞。這個道。某家抱了一個雞。三隻脚。那個道。某家下了一個小豬。還是一隻人手呢。他嫂子道。都是瞎話。我不信有這樣奇事。袁氏不覺失口道。這有甚麼奇。你姑夫還是猴子的兒子呢。衆人不爲奇言。追問其故。袁氏已經說出。悔之不及。被逼不過。只得實吿。囑道。千萬不要傳揚。這些婦人得了這件新聞。說與丈夫。丈夫又傳與別人。人還有不信的。想到他拿猴子同他母親合葬。纔以爲眞。故此人皆知他是個猴兒肏的\footnote{看官知之乎。說了易于仁千言萬語。總歸重在這一句上。謂世間人有了幾個錢。小時便枉炸跳躍。大來則貪淫好色。損人利己。全異於正人君子之所爲。皆是猴兒肏的也。}。他雖聞知風聲。也恬不爲怪。他這妻子袁氏。只能在被窩中做生活。至於女工針指。當家立計。前(全)然不會。除了行房之外\footnote{他事一件事。}。但能食粟而已。易于仁家中的婢妾有二十餘人。他要高興的時候。不是一個一個的去斡旋。製了一張大榻。叫這些婦人光了仰臥在上。他睡在衆人身上滾。他身材小巧。又甚伶便。雖在人身上滾。這婦人們也不覺吃力\footnote{大約世間婦人雖身上馱着大漢。亦未必覺得吃力。}。不拘誰人。滾到跟前。只陽物恰巧對了陰門時。就抽上一陣。重復又滾。那婦人造化高的。竟三四次滾着。那滾不着的甚多。他也不論甚麼白日黑夜。也不管院外房中。興動則來。興盡則止。他這後房內買了許多春宮的畫。貼得滿牆都是。又買了許多角先生來。他要交媾時。袁氏看爲首衆婢妾都脫光了。着一半婦人將假陽物根子上用帶子縫緊。繫在腰間。那一半婦人並排仰臥着。指着壁上春宮。要做那個勢子。他先同袁氏做起。叫衆人都同他一樣。要緊齊緊。要慢同慢。參差不一者。罰酒一碗。弄過換那一半。同這一半又弄。或叫袁氏同衆婦人脚後跟上紮一角先生。一齊臥倒。將那物送入牝中。自己用手扳着脚。他却擂鼓。叫衆人隨他的鼓聲徐疾一出一入。到那鼓擂得如雨點時。衆婦人手慌脚亂。一齊亂搗。他却看了大笑。後園搭了個鞦韆架。用一塊濶厚板。上安兩個靠背。他坐在上面。叫婦人跨在他身上套入。兩邊\endnotemark[5]着有力人往來推送。一起一落。自然有出進之妙。他兩人只用手攥住絨繩。毫不費力。甚是得趣。又打了許多醉翁椅。叫衆婦仰臥。將脚擱在兩邊。肚上牝戶大張。他在十步之外。手執着陽具。對着一個。如飛跑來。一下剛中紅心。便大抽一陣。若戳不着。又如此弄第二個。或備\endnotemark[6]一匹小川馬。他騎在上面。也叫婦人跨上套入。叫人牽着馬。在園中四圍攧着走。出出進進。甚有妙趣。又將紬子縫做圓毬。以綿塞上。如胡桃大。叫衆人屁股高蹶。他立數步。用小軟彈弓彈之。正中紅心者。便弄一度。又叫衆婦仰臥。將角先生送入牝中。以手堵住。一齊放手。用力一努。以〈以〉冒出遠者爲勝。大約只(自)四月半間天緩(暖)起。至九月重陽後將涼止。這幾個月婦女們都不穿褲。只來一條羅漢裙。他自己也是如此。到冬來。婦女皆做小綿襖。緊緊箍在身上。褲子皆做開褲。以便高興便不用脫。他一日之內。竟有行七八次。他自己說。寧可三日不食。不能一日離婦人。他婢妾雖多。總不生兒女。弄過了兩年。忽然想起他是猴子生的。又買了幾個大猴。拴在後園。叫這些婢妾先同猴子弄。他在傍邊看。看上興來。也就弄上一陣。他於此道中。千奇百怪。無不想出法兒來弄。夫旣有奇者。其妻則更有甚焉。那表(袁)氏更淫得可笑。一日到晚仰睡着。選兩個壯實丫頭。一個姓馬。一個姓水。將頭號角先生拴在腰中。輪流替他抽弄。到吃飯吃酒的時候。還將角先生套入牝中。拿那帶子前後緊(繫)在褲帶上。他坐在椅子上。那屁股不住起落。使他在內中活動。睡覺之時。亦用此法。着丫頭用手一推一推。不住的動。若睡着了。仍放在裡面。陰中空了一刻也過不得。他見易于仁同婦人在馬上弄。他悟一個法來。叫人備了馬。他將角先生套入牝中。騎在馬上攧着走。甚覺有些妙境。他夫妻二人的淫法。眞是寰中之(第)一。宇內無雙。他家的後園內。周圍有些樹木。上面的那些禽鳥。時常見他行淫。物有靈性。但是見他同這些婦人淫媾時。也都爲之交合。他指着對這些婦人道。你看羽毛尚知行樂。豈可人而不如鳥乎。他有這許多婢妾。猶不愜意。家中使用的那些大脚婆娘。雖奇形異常。不但都要領敎領敎他們的緊鬆深淺。連這些佃戶的妻子。形如鬼魅者。也要哄了來家。試驗試驗他們的乾淺(濕)瘦肥。這些村中愚婦。知道甚麼叫做羞恥。貪他些小惠。無不樂從\footnote{以上一段。雖是寫易于仁縱淫。却是宣明他的罪案。}。因他這樣貪淫。就引動了一個淫妖。他這山後有一個老狐。善能變化。從來沒有聽見迷惑婦女的事\footnote{下此一句者。見妖由人興之意耳。}。因偶然到他家來。見他這樣淫穢。遂動了淫心。他有一個美妾鄒氏。也不是甚麼天姿國色的美法。不過在他家這羣妾中算個翹楚。這妖就看上了他。那一夜。鄒氏正睡。似夢非夢。見一個美少年據在他的腹上。一根極偉岸陽道放入他牝中。伸伸縮縮。弄得異常受用。却是再掙不醒來。弄了有半夜。鄒氏丢了數次。直到覺時。那人不在身上了。方纔醒轉。睜眼四處看看。並不見人。摸了摸牝中。淫液汎濫。褥子濕了好大一塊。還疑是做了一個遊仙好夢。此後同易于仁睡便不覺。但是獨寢就是如是。鄒氏也就想到了這上頭。他就是個極好淫的婦人。不但不懼。心猶暗喜。低低祝道。我夢中與你相遇多次了。若果然有緣。何不我醒着之時。使我得一實在樂處。也不枉這一場奇遇。化(他)臨睡時又禱吿了數遍。方脫衣上床。剛臥下。只得見一個美少年坐在床沿上。笑嘻嘻雙手捧着他的臉。親了一個嘴。道。承你不棄。我來相伴你了。那鄒氏毫無畏法(怯)。歡喜非常。攜他的手上得床來。那少年脫衣進被。同他交媾起來。與那夢中無異。此時是醒覺着更快樂。怎見得。你看他兩個。

\begin{quotation}

酥胸緊貼。粉面相偎。玉腕輕輕摟抱。金蓮款款交加。雄赳赳如渴馬飮泉\footnote{喩其狠}。急攘攘似飢鳶啄食\footnote{喩其疾。}。情濃處喘氣吁吁。興酣時嬌聲怯怯。翻來覆去。效鴛鴦狎戲蓮漪。上倒下顚。學鸞凰盤旋雲際。溫存繾綣實消魂。旖旎風流眞欲死。

\end{quotation}

或去或來。約有一載。他來去無踪。竟無一人知道。一夜。這少年同他交合了一次。向他道。我明日別你去了。鄒氏大驚道。我們相厚得好好的。你要往那裡去。那少年道。實之(不)瞞你。我是千年仙狐。已成起(氣)候。從不敢犯此淫戒。前因在院中。你們白晝宣淫。我也就動了淫意。後來算了一算。我與你該有一載的宿緣。今期限已滿。豈敢久戀。以遭天譴。你腹中一月前已得了孕。本當是該男胎。但你夫主貪淫無度。又多淫人妻女。命該絕嗣。\endnotemark[7]\footnote{大書特書。}你懷的孕雖是我子。但我在暗。他在明。少不得要算他的。故此轉做女胎。却又有陽物而非陽物。總陽陰不能生育的人。傳說二形子就是這樣的了。上半月爲男。下半月爲女。你受孕那一夜。次早又感了你夫主的淫氣。這女子異日必定奇淫。即以淫死。這也是爲父母貪淫之故。鄒氏見他要去。戀戀難捨。滿眼垂淚。他又勸解一番。又幹了一次。作別時已五鼓。那少年穿衣下床。鄒氏灑淚。要起來送他。他道。你不必動了。保重罷。脫然撇去。鄒氏一驚。却是醒着。又像做夢。嗚嗚的哭了一會。天明起來。兩三日都沒一點精神。果然此後再不見來。光陰似箭。不覺十月滿足。鄒氏生了一個女兒。那小陰上有段肉蓋住陰門。却與男孩子毫不相似。鄒氏想那仙狐的話。一絲不謬。說這女兒後來奇淫。就起他個乳名叫做奇姐。這奇姐到了十四五歲。生得妖麗非常。他下身那一段肉。長得有一虎還粗。長有六寸。間或硬起來時。只有圓滾滾一段沒頭沒腦的物件。到了下半月。便不能硬。稀軟的蓋住陰門。人不認得。都說他是個門簾屄。惟有鄒氏聽得仙狐說過。知其所以。却不肯說出。易于仁見這女兒生得姿容秀美。要選一個好女婿相配。那日偶到城中。正遇着迎舉人。他見了鍾生正在妙齡。心愛至極。打聽得他尚無妻室。越發歡喜。他心中暗想。若做成了這門親。不但女兒得其所天。且有了這新貴女婿。更覺體面。在村中更可橫行。再三托人來向鍾生說他女兒如何標致。纔十五歲。若肯做他家女婿。願以三千金爲暗送之資。鍾生因有錢貴之約。〈若〉苦苦辭了。易于仁一場掃興。他向日無子息。暗暗常想道。我這樣一分家私。沒有兒子。後來都與了女兒不成。何不想一借種之法。寧生雜種。不可絕種。他有兩個寵婢。又是袁氏的心腹。一個姓馬。因他要蛋(密)生兒子。故名馬蜜兒。一個姓水。要想他生好兒子。故名水良兒。因將借種的話同他二人適(商)議了。遂將二人配了兩條精壯夯漢。一個名苗秀。一個名谷實。配了不到半年。就都有了孕。馬蜜兒水良兒對主人說知。易于仁叫了苗秀谷實到跟前。說道。這兩個丫頭當日已懷了孕。我知道把他配了你們。他肚中旣是我的骨血。如何與你家奴做得兒女。把這兩個丫頭還叫上來。後來〈後〉再另配你們妻子。那家奴可敢與主人相爭。只好俯首聽命。也摟着快活了幾個月。並不曾吃甚麼虧。這兩個丫頭到月分足時。竟生了兩個兒子。易于仁以爲天從人願。歡喜非常。以爲有了後代。這兩個兒子都到了十歲。愚鹵至極。蠢夯異常。他家是個財主。少不得要沽個讀書的名。又無到人家去附搭的理。只得請了個先生。你道是誰。就是卜通了。卜通如何到他〔家〕做先生的。他先次考了個四等。恬不知恥。但是衙門中有人打官事。他無一次不到。不論事情曲直。他賴在堂上纏繞。知縣十分惱恨。後値宗師下車。知縣約同敎官。將他的劣行細細禀明。考後宗師看他的文字又甚不通。放了他個六等。到發落之時。宗師道。你這不通的生員。本不該辱我的刑杖。但你所行的事。又不得不加一番重責。喝叫皀隸重打了三十板。革退逐出。他這一回進不得衙門。再要開館。這不通的大名在外。也無人肯來就學。沒奈何。托了個親戚。要在鄕間覓一館地。那人與〈他〉易于仁有些瓜葛。曾托過他要請個先生敎兒子。第一件要有名。第二件要價賤。那人就薦了卜通。易于仁聽見他兩次考過十等的秀才。定然是大才子了。便請了他來家設帳。卜通進館之後。替他兩個兒子起了兩個學名〔大〕的叫易勤。小的叫易壽。易于仁圖省供給。在大門口騰了三間房子做學館。房錢算了兩個兒子的脩金\footnote{此不過做笑談耳。近日此事甚多。}。許外人來附搭。這却虧他的體面。左右前後人家。招攬了有七八個大大小小的學生。先生吃飯輪流着學生家每人供給一日。房東不在其內。卜通敎了五六年。這易勤易壽連對課還課不來。一日。八月初旬。卜通偶見雁過。叫易勤來。出了個對與他對。道。

\begin{quotation}

一羣征雁往南飛。

\end{quotation}

那易勤算計道。蒸對燒。雁對鵝。飛對走。南對北。忽然喜笑道。有了。我對上來了。我對個。

\begin{quotation}

兩隻燒鵝朝北走\footnote{眞算聰明。}。

\end{quotation}

可好不好。那卜通見他對得這樣不通可笑。也無不說的。叫易壽道。他(你)也對一個。那易壽想了一會。道。我對個兩隻燒鵝朝東走。那卜通只得笑笑。贊了一句。道。大公子好悟性。二公子好記性。又對易壽道。那個對雖然是你的記性好。算不得你對的。我出一個五個字的你對罷。

因道。

\begin{quotation}

美女櫻桃口。

\end{quotation}

易壽道。美女拿甚麼對呢。卜通道。美女是人。〈他〉也拿人對就是了。他道。就拿先生對罷。卜通〔道。〕只要底下續得順。也可以對得。他又道。口對甚麼呢。卜通道。口是身體。是上身就可對了。櫻桃是果子。也拿果子對就是。他又想了一會。〔道。〕先生先生。我對個

\begin{quotation}

先生橄欖頭。

\end{quotation}

只見那易勤拍手打掌。大笑道。烏龜纔是個橄欖頭。先生。他罵你是烏龜呢。那易壽紅了臉。道。我對對要你多嘴。我就肏你的親媽。那易勤道。罵我的媽。我就肏你的祖奶奶。那卜通勸易壽道。他是你的哥哥。你怎麼開口就罵他。易壽道。他是個忘八羔子。我〔那〕有這麼哥。易勤道。你罵我忘八羔子。你還不是娼婦粉頭養的麼。我聽見說你媽還給谷實的奔兒奔兒的響呢。易壽道。你罵(媽)還沒有給苗秀肏麼。把屁還肏出來呢。兩人相罵急了。就揪着廝打起來。卜通不敢打他。呼喝着他又不理。只得橫着身子在裡頭勸。那易壽見打不成。急得罵道。把那勸道鬧娘送給叫驢肏。卜通只做不聽見。勸了多時。他兩個性子消了。纔各上位下坐。卜通心中暗慮道。他兩個餘恨未息。到家中要打起來。東家豈不怪我。正在躊躇。只見他兩個人嘻嘻哈哈笑成一堆。頑成一塊。纔放了心。歇了一會。又叫過一個姓高的學生。名叫高文學。說道。你素常還對得好。因指着院中雞冠花。道。草花惟有雞冠最發在後。秋來獨他茂盛。你就對個。

\begin{quotation}

院內雞冠花後發。

\end{quotation}

那高文學應聲道。

\begin{quotation}

牆頭狗尾草先生。

\end{quotation}

卜通道。好好。這館中將來定然是你出衆。上位去罷。那高學生纔坐下。只見那易壽念道。牆頭狗尾巴先生。頭尾巴先生。混念個不住。一日。易于仁到大門外場上看人打稻。偶到學房中走走。卜通忙讓了坐下。便道。這兩個學生聰明異常。對得出奇的好對。將來府上定出兩位科甲。這是我包得定的。易于仁道。我是一個字也不懂得。先生這樣誇獎。我看他未必有這才學。卜通道。若不信。何不叫他來當面考考。便道。易勤〈道〉你過來。我出對你對。想了一想。道。也罷。令尊老爹來看打稻。即景爲題。說道。

\begin{quotation}

爹來看打稻。

\end{quotation}

你對。易勤想了一會。對道。

\begin{quotation}

媽去學肏屄。

\end{quotation}

卜通極贊道。好好。好想頭。眞算聰明。易于仁道。他對的是甚麼胡話。先生怎麼還誇他。卜通道。話雖不成話。文理却有深意。爹看打稻。乃是積穀防飢。他對媽會肏屄。纔可以養兒代老。豈不深妙。易于仁聽了。也甚歡喜。又叫易壽過來。他知道這易壽更蠢夯得出奇。生對的萬萬不能對上來。因想昨日高學生對的那對。他念了數十遍。或者還記得。遂道。院內雞冠花後發。那易壽白着眼望着他。卜通知他忘了。用手指着牆頭道。這就對得。那易壽忽然想起。對道。牆頭狗頭先生。卜通嘖嘖贊道。對得倒好。再下些就是了。他道。狗要先生。卜通道。再下來。就不知這蠢才再想不起尾草二字。況他昨日就念錯了狗尾巴先生。他今日連尾巴都忘了。聽見叫他再下些。便道。我想起來了。

\begin{quotation}

牆頭狗雞巴先生。

\end{quotation}

卜通說得說。只得道。對得工致。好得很。只得(可)惜略差了些兒。那易于仁見先生誇他兒子。他也不知是那裡帳。逢人說。我家有個奇童。十六七歲竟會作對。那學館隔壁有一家也姓易。是易于仁的族姪。他男人沒了。只一個寡婦。他有個兒子。也隨着卜通念書。這寡婦姓焦。有三十多歲。蠟渣潢(黃)一個刮骨臉。人都稱呼他爲進(焦)面鬼大娘。俗語說的。這樣婦人。那件東西只好撒把黑豆叫豬拱。這卜通叫做情人眼裡出西施。不知如何看上了他。就勾搭上了。如糖似蜜。如膠似漆一般戀住。成半年總不歸家。那知水氏也正在同楊大如魚似水。也巴不得他不回。卜通眞是外面拾得八兩。家裡失却半斤。你道卜通同這焦面鬼大娘如何就偷上了。這婦人性極貪淫。他丈夫當日也是個做莊稼的結實漢子。自從娶了焦氏。他日間辛苦下力。夜間焦氏又不肯免他的差徭。他丈夫或一夜懶動。要隨(睡)一覺將息將息。他不是假說頭疼。便是肚疼。哼哼喞喞。吵得徹夜無眠。只等弄過一次之後。他纔肯安然去睡。不上兩年。一條壯漢被他弄得骨化形銷。奄然長逝。這婦人守了幾年的寡。思想要嫁人。人都知他有些利害。那下苦的窮漢不敢娶他。怕當不過差事來。有些有錢的閒人又嫌他生得醜。他雖想要走走邪路。因一個大寡(刮)骨黃菜葉臉。招牌不濟。所以沒有主顧。況且村莊中人都還在老實一邊。沒有浮浪子弟。倒保全了他的名節。但他那心中。日裡茶思飯〔想。〕夜閒夢倒魂顚。何曾一刻放下這件奇物。他有個八歲的兒子。也送在卜通處讀書。這卜通的三間學館。兩明一暗。兩間學生讀書。一間做臥室。與焦氏的房僅隔一板。那焦氏聽得卜通就在隔壁。恨不得將板打開。兩家合而爲一。每聽得卜通在房中或說話或咳嗽。他便嬌聲嬌氣這樣那樣的鬼話。後來忽見板上有一個松節。他拿刀子剜掉了。有鍾子口大一個洞。就時常蹲下身子來張。不想這卜通又是個沒行止的人。聽得這婦人妖聲浪嗓。又知道是個寡婦。也就留了一番心。見了這個窟窿。知是婦人所爲。定然是有心相愛。暗暗歡喜。也〔不〕住的往那邊張看。無巧不成話。一日。卜通到房中來。關上門。脫了小衣捉蝨子。偶然嗽了一聲。這婦人聽得。就蹲下來張。一眼看見他好個像樣的陽物。硬邦邦豎在那裡。那婦人久不見此物。今忽乍見。眼中火星亂爆。喉嚨中的火就攻了上來。喉管一癢。忍不住一陣咳。卜通聽得。知是婦人張他。忙跑來一看。兩個人的眼睛正正相對。卜通笑着悄聲道。不知奶奶在這裡。看我赤身露體的。奶奶不要笑話。那婦人也沒話搭應。只笑笑站了起去。將晚。學生散了。卜通到房中來。聽得隔壁水響。也去蹲下一張。原來是那婦人蹲在一個脚盆中洗下身。看他洗完。蹶着屁股揩。卜通見他光撻撻。牝淨無毛。不覺陽物就跳將起來。故意也咳了一聲。那婦人聽得。忙來一看。笑道。好先生。偷看女人的屁股。沒廉恥。卜通笑道。我並不曾看見甚麼。要得見這稀奇物就造化了。那婦人笑道。你要看。索性給你細看看。他就回過身子去。把屁股靠着板壁。彎着腰。拿陰門對着那洞。道。請看。卜通一見魂消。站起來。將陽物伸入洞中。用力一頂。不曾頂着陰門。却頂那婦人的股上。用力大了。把那婦人頂得往前一失(交)幾乎栽倒。他忙用手向地下住住。卜通見頂不着。縮回來。又蹲下。見那婦人也蹲下。笑道。冒失鬼。幾乎跌了我一交搶(撞)了臉。卜通道。奶奶。旣承你不棄。可拿張杌子。你爬在上面。就穩實了。那婦人果掇張杌子爬在(住)。又將屁股對那洞蹶着。卜通將陽物伸了過去。那婦人將陰戶左就右就。一下就着。弄了進去。被板子隔住。又是臀尖礙着。尚不曾弄進半截。抽了幾下。不得痛快。卜通拔出。蹲下。見他還蹶着呢。伸手指戳他。道。你蹲下來。我同你商議。那婦人也蹲下來。卜通道。這樣弄得不受用。我看後牆不高。我這邊又沒人。你夜間上牆。我接你過來。好好的快樂一番。焦氏道。你一個男子漢倒過不來。叫我一個婦人家爬高上低的去就你。你倒會自在。卜通道。不是這話。你身邊有孩子。怕不方便。焦氏道。不相干。他睡夢不知顚倒的娃娃。怕甚麼。他一放倒頭。就是一夜到天亮。你經心聽着。若是孩子摟(睡)着了。我喚貓你就過來\footnote{卜通那得如貓。何不竟喚狗。}。北窗子我不上栓。你推進來就是了。兩下約定。將近一鼓。卜通側耳聽着。見那婦人謎謎的喚。卜通忙開了後門。見那牆雖人高些可以偷(踰)得。但是土牆恐爬得有跡。拿出一張桌子靠牆放着。又放上一張椅子爬上去。往那邊一望。見有一張梯凳。知是婦人放着接他的。心中大喜。輕輕攛上牆頭。踏梯而下。將北窗一推。果然沒拴。推開鑽了進去。摸到床上。脫了衣裳。掀開被摸那婦人時。已精光仰臥。耑候光臨。先不暇開言。兩個就〔弄〕起來。一個怨女。一個曠夫。一度不止。兩次不休。一連弄了三下。卜通也離家久了。覺這婦人比水氏還淫浪些。也甚是動興。盡力\endnotemark[8]盤桓。都乏倦了。然後收兵罷戰。相摟相抱。敍了些彼此渴慕的話。睡了一覺。醒來已經五鼓。那婦人將卜通一把抱緊混扭。送嘴遞舌。那種騷態。眞眞是異常。卜通心愛得了不得。知他餘興未已。也就爬上身。纔抽得幾下。卜通心愛的摸着他的蓋子。道。你這件寶貝。裡邊雖然好得很。又緊又乾。但這個像刀山一般。先弄着還不覺。此時我這塊骨頭損得生疼。用不得力。怎麼處。婦人道。你把身子吸起些就好了。卜通依他。又抽了幾下。道。不想遠空身子。越發不好用力。你上我身上來試試。那婦人到他身上。果然兩無妨礙。就做成了例。定了這陰陽倒置的款式。天色將明。卜通復踰牆而回。仍將桌椅搬進。他欣欣自得。以爲奇遇。忽然想起鑽〖阝少日小〗相窺。踰牆相從從(這)兩句。他不住贊道。孟夫子不但是亞聖。又是眞仙了。怎就知二千年後有我。就先把這兩句說定了。聖人說。百世可知矣。也一絲不錯。又拿過孟子來翻。翻見踰東家牆而摟其處子。說道。這一句略差些。我是踰西家牆而摟其寡婦。聖賢也還有說不着處。到晚。又過去做那地天交泰的事。每夕如此。不必細說。日間偶然高興。還在那板洞中交媾幾下。雖不能大暢。兩人聊爲適興而已。此後卜通不但不要他的學錢。把別人家得來的束脩都贈他爲衣食之費。卜通愛他騷淫善戰。故此不捨歸家。況且見水氏已四十外的人了。兒子又老大。料道決無他事。所以更自放心。那裡知他同楊大相公與得更契厚。一日。焦氏生辰。卜通先也送了三百文錢與他。道。沒有甚麼與你慶壽。你拿這錢。煩人打些酒。買斤肉。收拾兩碗菜。我同你夜間敍敍。況向來都是一來就睡。總不曾坐一會兒。那婦接過。次日預備停妥。到晚上他兒子放了下學。看見了肉。定問娘要吃。焦氏給了他些。鄕下人容易不得見些奇物。那小孩子未免就多吃了些飯。焦氏要等卜通過來暖壽。也不顧兒子飽脹。忙忙攆他睡下。聽他睡着。然後喚貓。卜通〔越〕垣而來。二人明燈對飮。先是一遞一口的吃。後來你含了哺我。我含了哺你。說說笑笑。又彼此脫了小衣。互相摸弄二物調笑。無所不至。卜通道。我們向來前(全)是黑地摸索。今日點着燈做一個快活的。那婦人也興動了。忙把傢伙收拾。開燈剔亮了。一同上床。卜通臥到。叫他上來。焦氏道。我還虧你做先生。連禮都不知道。每常罷了。今日是你替我祝壽。你是主人。還叫我上去費力。卜通爬起。笑道。有理有理。祝者。築也。築之一事。應該是我在上。遂客反居卑而主居高。兩人弄將起來。一個是祝壽。一個是領情。祝者祝之不已。領者領之不休。不肯便住。不想那孩子吃多了些就睡了。忽然肚子脹疼得醒來。纔要叫娘。一睜眼。見先生精光着壓在他娘肚皮上亂搗。他娘不住的哼。嚇得不敢做聲。忙閉上眼。不覺又睡去。二人狂了半夜纔睡下。天明。卜通過去。那孩子醒了。向娘道。我夜裡看見先生來。他娘道。你在那裡看見的。他道。我肚子疼醒了。要叫你。看見點着燈。先生精光着壓着你肚子上亂動。我不敢叫。又睡着了。那婦人不好意思。假說道。胡說。那是你做夢。半夜三更。先生到這裡來做甚麼\footnote{來說(祝)壽的。}。那孩子道。我何嘗是做夢。明明看見先生在你〈裡〉身上一動一動的。你的屁股攧着。還哼呢。焦氏把他打了兩下。他叫哭起來。到了館中。卜通問他道。你必定在家中又淘氣來。我聽得纔打你呢。那孩子道。我何嘗淘氣。我纔對我媽說我昨夜看見先生在媽身上睡着動。他打我呢。卜通紅了臉。喝道。放屁。不許胡說。喝了過去。這些學生聽了這話。背地拿果子饝饝與這孩子吃。哄着問他。一個八九歲的娃娃知道甚麼。把他所見他令堂的這行樂圖細述。這話外邊也就傳開了。地方上沒有生事的人。也無人管他閒事。晚上卜通過去。二人說起。笑話了一會。此後再不敢點燈。只是一味黑幹。過了二年。這孩子漸漸大了。有些知覺。夜間常醒。他二人正在興濃。一聽得這孩子有些展轉聲息。只得要住。常常阻興。深爲不便。兩人商議將板子撬開一塊。僅可側身而去。安個活筍。日裡安好。夜間除下。焦氏過來就敎。始得點着燈。放心大膽的做。也混了四五年。易勤易壽也成了大漢。仍一字不識。易于仁也不叫他念書了。卜通只得辭了歸家。水氏查問他數年束脩下落。卜通無言可對。夫妻大鬧了幾場。水氏還借名在外做生意。不住還同楊大往來。卜通無所事事。靠着老婆吃飯。耳中也風聞得水氏有些走邪路。又不敢查問他的來去。一日私下問卜之仕道。我不在家這幾年。你媽常同誰來往。卜之仕道。自從爹爹下鄕。我媽認了個楊姐夫。常到他家去同他睡覺。卜通暗暗氣惱。又一心思想焦氏。到半年就懨懨病故了。楊大的妻子七病八疼。半年前也死了。楊大此時年已四十。水氏亦將望五。只過了卜通百日。竟帶着卜之仕做了拖油瓶嫁了楊大。女婿忽驚(變)爲丈夫。岳母變妻子。更可笑者。那卜之仕叫了多年姐夫。忽然爹爹起來。豈非卜通誤人子弟。奸淫孀婦之報乎。水氏嫁楊大之日。有人知他是三嫁了。就將一首古歌唱着送他。道。

\begin{quotation}

辭靈羹飯化金錢。哭出先天與後天。

今日洞房花燭夜。三天門下會神仙。

\end{quotation}

又有人知他相交甚多。又作了四句贈他。道。

\begin{quotation}

鵲橋偷渡曾多火。百輛于歸事已三。

何羨三天能覆載。天天天外有諸天。

\end{quotation}

且說那焦面鬼大娘同卜通相厚幾年。又常得他資助。一旦分開了連理枝。拆散了鴛鴦伴。好生難過。欲守不但無倚靠養活。且臍下這件作怪的東西。不得些肉吃便不能安靜。欲衒色自嫁。奈這一副妝金的妙容。久無售主。欲偷或者還有那一種餓眼見瓜皮。不擇精粗的人來賜顧。兒子又大了礙眼。成日家行住坐臥一處。又沒處驅逐。每到難過的時候。便放聲痛哭一場。易于仁常常聽見。想道。這婦人同卜先生私偷\footnote{應前也就傳開了句的話。}。近日先生去了。他過(故)此這樣傷心。他大約也是個極淫的婦人。我何〔不〕收他回來。以備行樂之用\footnote{孰知是他年送命之由。}。遂叫人去對他說憐他母子無依要收養他的話。那焦氏素聞易于人(仁)連佃戶的妻子〈的妻子〉都不肯放過。此去不但有得吃穿。料道也還必定收用。遂千恩萬謝。謹尊(遵)來命。易于仁收他母子到家。叫他兒子相伴易勤易壽。焦氏雖面目可憎。易于仁是不擇美惡的。纔到了房中。就同他幹了一度。那焦氏別了卜通多日。一腔念此時盡發洩來。口哼股疊。足顫手扳。衆婢妾都在傍賞鑒。看得好不肉麻。無不含笑。無不水流。易于仁正投所好。甚是歡喜。又覺得陰戶乾而且緊。乃家中諸婦所不及者。更自心愛\footnote{此婦形容如此。若再無此一妙牝。卜通何所戀於前。易于仁何所愛於後。此句斷不可少。妙矣。}。但易于仁婢妾多。恩波不能常及。他也分得了一個角先生。借此以爲消遣。見後園中那幾個大猴會同人交媾。他但見人不在面前。褪了褲子。蹶着屁股。送這個弄一陣。又送與那個弄一陣。颇不寂寞。雖不能暢心。強似以前常常空曠。那一年二月盡間。春景融和。百花大放。易于仁帶了他的妻妾子女到牛首去踏靑。不想牛質的兒子牛耕也往牛首來遊賞。忽然見了這奇姐。魂不附體。只見他。

\begin{quotation}

臉際芙蓉掩映。眉間楊柳停勻。若敎夢裡去雲(尋)。管取襄王錯認。殊麗全由帶韻。多情正在含顰。司空見慣也錯(銷)魂。何況少年光棍。

\end{quotation}

牛耕心中十分相愛。目不轉睛的看着他。誰知道這奇姐心愛他更勝。俗說。槽頭罵馬看母子。這牛耕係苟氏\endnotemark[9]所生。苟氏已是個淫美之婦了。況且又是胡旦之種。那胡旦又是個淫美的男子。二美相合。有了這樣的好模子。印下來的兒子自然是標致的了。奇姐在家中。不足(過)見些粗蠢童僕。何嘗見過這樣男子。不要說這個主人。連跟隨的八九個披髮俊童。都生得秀美可愛。他二人四目相覷。兩情眷戀。竟有個分開不得的樣勢。兩處都要歸家。少不得分頭走路。兩人頻頻回應。戀戀不捨。牛耕叫家人打聽是甚麼人家的女子。家人去了一會。來說是土山易財主的家眷。那個年小是他女兒。牛耕回到了家。他父母只這個獨種。疼得如龍卵子相以(似)。在他身上百依百隨。牛耕撒嬌撒癡。問苟氏說。我今日遇見了土山易家的女兒。又年小。又標致。我要他做媳婦。若不要娶與我。我就去做和尚。再不娶老婆了。苟氏聽了這話。嚇得了不得。忙對牛質\endnotemark[10]聽(說)了。牛質見兒子心愛。況且也是財主人家。正是門當戶對。就依了他。煩人去說親。那易于仁聞他是尚書之弟。而且又財其主也。前日在牛首也看見過牛耕。人物齊整。眞是點着燈還尋不出這樣門第同這等佳婿來。可還有推辭的事。只假說幾個不敢高攀。欣然允諾了。牛質怕兒子想壞了。趕忙就行茶過禮。四月盡就娶了過來。次早拜堂。牛質見果然好個婦媳。眞是一對美貌夫妻。心中大喜。原來這牛耕小時。父母鍾愛太甚。凡事任他性兒。因吃傷了飮食。又寒暑不均。成了個休息痢。又怕與他藥吃。苦了兒子。日久把臟頭努出數寸來。脾胃弱極。收不上去。通紅的一段翻跳着。好不硶看。纔着了急。忙替他依(醫)治。過了半年有餘。雖然好了。因日久受了風毒。成了個臟頭風。先還不覺。後來大了又作喪了。作喪就發起來。一時間肛門內外發癢。眞癢得要死。沒發(法)了。他弄個木槌兒戳戳。雖然受用。但木頭死硬。肛門雖是殺癢。裡面戳得甚疼。因叫了個龍陽小子來。叫他把陽物弄硬了。甚是渺乎小爾。也只得叫他來試試。他脫了褲子。伏在枕上。屁股高蹶。叫那小子弄他。那小子先還不敢。因主人再三開諭了。也就挺然而入。這小子的陽物雖微而堅久。弄得牛耕其樂無比。自從得了這個妙趣。把家中的幹(乾)淨精壯小子送了八九個來服侍。紅梅的兒子雖精壯而愚蠢。故不在選內\footnote{虧不入選。後來才留得在(住)。}。牛耕把這幾個小子。與他們穿得好不光鮮。每夜輪換着兩個弄他的後庭。纔睡得着。一夜也少不得。他間或也弄小子們。但他弄人的少。人弄他的多。傍人只說他是好此道。却不知他是要人弄他的此道。且還有一說。古書上說。昔岳忠武部下有一軍士。其妻懷孕數月。此人因犯軍法斬首。其妻後來生了一子。長大時身如大漢。頭臉只有小孩子大。有格物的人說。人皆秉父母之情(精)脈氣血而生。此子在母腹。他父被刑。父子之氣相感應。故此頭就小了。即如岷山西崩洛鐘東應一個道理。氣感尚還如此。何況這牛耕是胡旦所造。胡旦的後庭也不知經歷了多少此道。這牛耕雖不生臟頭風。也自然是好人弄他的。是四月二十八日娶親。這個月是小盡。初一是三朝\footnote{說得如此詳細。竟似實有其事者。}。請吃會親酒。他丈人家的這些親戚多敬了新姑爺幾杯。有些醉了。晚間上床睡覺。他前一連兩夜。因愛奇姐過甚。弄了八九次。乏困了\footnote{不及令岳多矣。}。故不覺得。這第三夜不但弄不得了。且又沈醉。睡不多時。他的糞門是夜夜離不得人弄的。過了兩宿。此時又癢起來。他已醉了。見有人同他睡着。當是每常小子們陪侍他。想要弄弄奇姐。把個屁股儘着向奇姐跟前拱去。奇姐不知其故。忙向後退縮讓他。他又蹶着就了過來。不見動手。口中模模糊糊的道。我屁眼裡癢得很。你怎麼不弄。倒躱開了\footnote{他二人成親這一段。與魏卯兒同邊氏的事。前後一對。而兩人確是兩樣。}。奇姐牝中昨夜怎(乍)得了些甜頭。止(正)想其中的妙境。這初一是陽氣發生之始。他淫情一動。那一段肉也便大硬起來。聽得牛耕說要弄的話。雖不懂內中的緣故。想道。他旣說要弄。我何不試他一試。前日牛耕弄他時曾用唾。他也學擦了些。摸着他糞門。一頂而入。只見牛耕把屁股亂拱。他也用力連頂。直弄到根。一陣狠搗。覺得弄他的屁股比牛耕弄自己的陰戶還有趣味。那牛耕每常叫這些小子弄他。但以僕弄主。未免跼跼踧踧。只不過殺癢而已。今遇了奇姐的這段奇肉。又粗又長。而且又硬。大肆衝突。弄得他有無窮的受用。忽然醒來。見是自己的親洞房。却又有人弄他。心中大疑。回頭一看。竟是新娘子大弄新女婿。他忙用手摸摸他屁眼中。乃是新娘子的陰門上的那一塊肉門簾。叫他拔了出來。問他原由。奇姐妨(方)吿訴他是胎中帶〈來〉下來的一段肉。上半月能硬。下半便軟。牛耕大異。忙下床。剔明了燈。拿過來照着。細看了一會。道。我前日不好問你的。我先還疑是你的病。後來我同你弄。礙着他。又不見你說疼。我當是拖的一心子。原來是這樣個奇物。拿陽物同他比比。奇姐這肉比他還魁偉許多。心中喜不容言。不但是娶了一個美婦。且又得了一個美夫。從新上床了。他倒仰臥了。把屁股墊高。叫奇姐上他身來。拿那肉送入後庭。他自己用兩手扳着腿直豎。整弄了半夜。弄得牛耕哼成一塊\footnote{一塊二字奇。大約謂哼聲總不斷之意。}。屁眼中丫油抽得一片聲響。丫頭們聽見。還只說姑爺弄得姑娘這等受用。那知反是姑娘弄姑爺。他兩口子這個恩愛眞是少有。互爲夫婦。果是一對奇夫妻。夜間或牛耕先弄奇姐。或奇姐先弄牛耕。他二人。

\begin{quotation}

夫妻不須拘次序。誰人興動即先來。

\end{quotation}

到了十六的夜間。奇姐的却不能硬了。牛耕吿知他有這個病根。時離不得人弄的。上半夜陪他睡。下半夜到書房去睡。這叫做蘿蔔纓子滿天飛。尋頭子去了。故(過)了數日。奇姐偶然一夜睡不着。心中想道。這兩件事各有妙境。人弄我固妙。我弄人更妙。但我雖可男女並行。到底是女人。要尋幾個男人來弄。自然難出於口。等我硬的時候。拿個丫頭試試。要與弄屁眼一樣有趣。我買些好女子來。也可取樂。叫丈夫擔着虛名。人只說我賢慧。買來服侍丈夫。我却又得了實惠。豈不大妙。想定了主意。到了發硬之時。叫了個丫頭試試。覺得與糞門又是一種滋味。各得其妙。他就破囊買妾。他是易于仁的愛女。又攀了這一門好親家。又要圖體面。與了女兒壓箱的銀子三千兩。奇姐叫媒人外面尋了八個好上樣女子來。都與他們製了上好衣服首飾。一個個打扮得嬌嬌滴滴。親自帶了上去與公婆叩頭。說道。公婆只生得丈夫一個。故此替丈夫多尋幾個小。圖多生得些兒女。將來可昌大門戶。那牛質苟氏都是心疼兒子的。見媳婦這樣賢德。誇之不置。那知內中深微底裡。半月之內。奇姐把這八個女子都開闢了。方知這件東西俗名又(曰)屄。文其名曰陰曰牝。其形外扁而內圓。門小而中大。其形微有不同。其內中滋味則大異矣。即如總是一個豬肉。或煮炒熩煠燒煎。其味自別也。奇姐嘗過新了。然後叫牛耕去刷鍋。牛耕見了這些妖妖嬈嬈的小女子。穿得花紅柳綠。粉面油頭。愛得了不得。儘力盤桓。在\endnotemark[11]奇姐身上倒不應付。他心中十分感激奇姐。又心中很愛他。自己應接不暇。不能供他之欲。過意不去。把那八九個小子都贈了奇姐爲小夫。奇姐也就欣然笑納。復諭衆小子不必畏縮。當各贈其技。論優劣行賞。這些小子們聽此恩諭。方各展其能。他一個個都細細領其物之形質。雖大同小異。然而內中之味亦自有別。此後上半個月奇姐爲正。牛耕副之。輪番弄這八個女子。或奇姐選領兩個子(小)子弄牛耕。下半月牛耕爲正。衆小子爲副。倒班來弄奇姐。這些小子們同這八個女子叫做上有好者下必有甚焉者矣。不拘早晚日夜。偷得有空。就大家混弄一場。把他這幾間臥房竟可牓其名曰淫窟。大家混弄了二三年。這八個女子中竟生得有六七個兒女。雖不知誰氏之種。自然都算在牛耕名下。牛質苟氏喜得異常。見這許多孫男孫女。每每贊奇姐的賢德。即牛耕亦以爲螽斯之慶。每見這些娃娃抱在面前。便誦奇姐的好處。他以雜種而生雜種。原不是甚麼異事。當日人謂李昊世修降表李家。此可稱祖傳雜種牛宅。一日。香姑回來看父母。〈回來〉留他住了兩日。他同奇姐年紀旣相當。花容又堪匹。素常兩人着實親密。那日香姑在奇姐房中坐着。說了一會閒話。笑向奇姐道。這兩三年了。我從沒有同嫂子夜間講話。我今晚同嫂子睡罷。你可離得哥哥麼。奇姐每常聽得他陪嫁的丫頭說馬台之呆。講他的些笑話。久矣想同小姑娘做些勾當。今聽他說這話。眞是送上門的買賣。心中暗喜。〈道〉忙笑答道。我嫁了你哥哥。是做他的妻子。有甚麼便宜處。今日若姑娘肯來同我睡。我又是你的丈夫了。這是極妙的事。我還稀罕你哥做甚麼。香姑笑道。你要是個男人。我就嫁你。你討我的便宜。我就來同你睡。看你夜裡怎麼打發我。奇姐笑道。包你有個絕妙的方法。打發你個快心暢意。不然我就算你的老婆。可好麼。兩人笑了一會。到了晚間。香姑果然與他同睡。牛耕帶着小子們到書房中去睡。奇姐香姑他兩個都是騷淫極了的少婦。都脫得精光。共枕同衾的睡着。淫辭艷語無般百樣不說出來。嘻嘻哈哈在(不)住的笑。兩人都說上興來。奇姐將香姑一把摟緊了。道。我的心肝。我愛殺你了。連親了幾個嘴。香姑也抱着他。笑道。你旣要做我的丈夫。快些打發我。奇姐笑道。在我。就跨到他身上。香姑也當他是頑戲。不想果有個東西在胯中戳了幾下。戳了進去。抽將起來。香姑急用手摸時。竟是嫂子的傢伙。此時淫心如醉。也不暇問。兩人用力盤桓多時。纔各睡下。香姑捏着〈的〉那肉。問他緣故。奇姐詳細相吿。兩人這一夜的恩愛。眞到一百二十分的地位。此(明)日起來。彼此相看。不住的笑。香姑經了奇姐的此物。覺得大小雖與馬台的差不多。但馬台是個蠢然一物。只知在肚皮上弄混而已。連趣話也不知說一句。親嘴這件事是極易的了。他尚還不懂。每常他耍弄香姑。還有受用處。故不阻他。却一點情處(趣)也沒有。今日同奇姐兩人枕頭上〔笑〕談謔浪。有多少親愛。那奇姐又千奇百怪的弄法都同都做出來。兩個如獅子滾繡球一般。豈不有趣。此後望前之內。香姑定要回來一二次。同奇姐作樂。不必繁敍。且待我再把牛耕奇姐夫妻二人的妙處略舉數件。也可一新耳目。那奇姐一日向牛耕道。每常大常(家)混弄。有何趣味。我想了一會(個)妙法。做個大家歡樂如何。牛耕道。怎麼叫做大家歡喜。奇姐道。你只聽着。做出便見。遂叫衆小子同丫頭都到跟前。說道。我們今日大家拈鬮。鬮上照男女人數寫兩個一字。兩個二三四等字。搓成圍(團)。放在兩處。男的在一處拈。女兒的在一處拈。拈着了號數對的就做一對。大家一齊弄起。若那個男的不濟。先丢了動不得。罰他跪着。等衆人弄完了。纔許他起來。女的若不等男人興足。要說夠了。受不得了。也要〈還〉罰跪。你道有趣麼。牛耕道。好好。就是這樣來。奇姐遂解衣。道。都脫光了着。大家都是混弄熟了的。男女毫無羞愧。答應一聲。解帶脫裩。鬆扣卸衣。笑嘻嘻都脫得精光。數年來。這幾個小子皆長大了些。那陽物粗長細短也都改頭換回(面)。大非昔比。牛耕做了鬮兒與衆人拈。內中有一個小子姓王。混名叫王彥章。他的陽物雖不甚粗。約有七寸來長。一個大長的光頭子堅硬如鐵。本事〈只〉可以熬一兩個時辰。因王彥章當年人稱爲王鐵鎗。奇姐因他的陽物尖細堅長。故贈了他這個美號。奇姐每常又喜他弄得長久。又有些的怕他太久。幾個女子都懼他幾分。他每常同奇姐弄。不過是奇姐自己飽足了就叫他歇。那小子可敢不依。他再不得遂竟(意)。今見奇姐這話。暗禱道。怎得奶奶拈着同我一對。就是造化了。此時衆小子見了這些女子的妙物肥瘦高低不等。毛光多少。各〔各〕陽物如旗竿般豎起來。像和尚撒酒瘋似的亂跳。奇姐見王彥章的分外挺長。如筆管槍相似。指道(着)笑道。不知誰造化低。拈着他呢。向牛耕道。你同他們拈。我同丫頭們拈。各人拈了一個。打開看時。奇姐是個三字。那王彥章恰好也是個三字。他歡喜欲狂。也顧不得了。上前一把抱住。道。我服事奶奶去。抱到床上。撳起腿來就弄。只聽得一個丫頭叫做蔣迎兒。說道。我造化低。偏偏的對着金三兒。你道爲何有這綽號。這個小子叫做金三。他那東西着實不濟。又小又快。弄不上三五下就完了動不得。當日金三兒轅門拜倒。因此拿了做他的綽號。金三道。你不要發急。等我掙命也多弄一會。盡你的興就是了。衆人聽說。笑〈道〉着各尋對子。也有在椅子上扛着腿弄的。也有在春凳上將腿夾在肋下幹的。下(也)有地板上鋪着蓆子對面弄的。也有爬在杌子上打背後弄的。正都纔動作。只見那蔣迎兒道。你當眞掙命麼。動不得。下去跪着。我不圖快活罷了。還把我當褥子墊着睡麼。往不(他)盡着推。金三死緊的抱住。道。我等歇歇。或者還動得。你何苦這麼性急。迎兒聽他這樣說。也還想他或者再動幾下。就不推。耳中聽得衆丫頭這個哼喞。那個呼叫。由不得心中火發。見他儘着不動。急道。你到底是弄不弄。那金三沒奈何。把身子探起些。掙着還想抽抽。誰〈不〉知陽物如鼻涕般掉了出來。他連忙拿兩個指頭捏着往裡塡。倒折了回來。那裡進得去。迎兒叫道。奶奶你看。金三不遵奶奶的令。軟得掉了出來。拿指頭捏報(着)都塞不進去。還不肯下來呢。奇姐笑着叫兩個小子將他擰着耳朶拉下來。跪在地下。迎兒坐起。一面揩着牝戶。說道。受瘟罪的。有名無實。生出這樣現世的東西來。我叫做槽(糟)鼻子不吃酒。虛躭其名。一點樂處也沒有。倒把胯襠弄得黏〖氵韲〗〖氵韲〗的。看見別人正弄得高興。他由不得氣來。再看金三的陽物。越發縮得如肚臍一般。他又是氣。又是那好笑。罵道。掙命鬼。看看你這個賊樣子。方纔還想等硬些再弄呢。再縮進去些。好像個老婆子。儘着嘓噥個不住。大家弄了多時。內中有三個洩了的動不得。那幾個丫頭一齊叫道。奶奶。他們都動不得了。該怎麼樣。奇姐正被王彥章弄得上氣不接下氣。閉着眼哼呢。聽得說。睜開眼睛一看。見牛耕在內中。不好罰跪的。便顫着聲兒說道。這這也還罷罷了。免免免罰罷。那牛耕同幾個小子聽得這話。都纔拔了出來。那金三道。我動不得就發(罰)跪。他們就饒了。奶奶這樣偏心。迎兒向他啐了一口。道。他們像你這樣不長進來。弄了這麼一會。還要怎麼的。你要有這本事。我就替你念佛。難道一日弄到晚纔算得麼。那金三瞅了他一眼。又低頭看看自己的陽物。笑着嘆了一口氣。再過了一會。大家都歇了手。這王彥章拿出了本事來。一陣緊似一陣。把奇姐弄得骨軟筋酥。是他自己發的令。〈要〉說受不得要罰跪。只得咬着牙死捱。不想他越弄越精神起來。奇姐實在有些擋不住了。遂摟過他脖子來。悄向他道。你把我也弄夠了。我禁不得了。你歇了罷。他也悄聲說道。我從不曾在奶奶身上丢過。賞我這一遭罷。奇姐道。我實受不得。你弄壞了我呢。那迎兒先同金三弄得不像意。你同他去弄。要洩的時候〈沒〉就再上我的身上來弄。遂你的心就是了。且讓我略歇歇。那小子見他說得苦楚。又不敢得罪他。只得依允。遂跳下床來。只見那迎兒拉着這個問道。你快活了幾下子。又問那個道。你受用了多大一會。衆丫頭見他着急。越發要急他。這個說如何快活。那個說怎樣受用。他正在急得恨不得掉淚的樣子。咬牙切齒的咒那金三。王彥章笑着上前一把抱住。道。你不要罵了。我替你消消氣罷。把他抱到奇姐床上。他連忙把腿蹺開。王彥章一挺而入。一陣亂搗。迎兒叫道。好親哥。好東西。不枉是個男子漢。弄得眞好。像那樣膿包。空與他個男人做。獎這個一句。貶那個一句。衆人看着。不住的笑。後來弄得他屁股亂攧。兩條腿如害瘧疾一般亂顫。口中連聲叫道。好哥哥。好漢子。你肏死了我罷。我知道你快活死了。我打屄心子裡受用到心窩裡去了。噯吓。我的親爹。你好弄。他無樣的言語不混叫出來。又有許久。他道。罷了我了。便閉着眼不做聲。王彥章見他那樣子。也甚是有興。蠻舂混搗了一陣。竟得精來。叫道。奶奶快來。奇姐先被他弄得軟癱熱化。叫他歇了。此時看見迎兒的這個浪態。興又大發。正要叫他來弄。聽得他叫。忙忙迎(仰)臥。也將兩足直豎。王彥章就勢妨(放)在肩上。自根至頂。抽了數十下。方一洩如注。兩人歇了。那迎兒纔醒轉來。贊道。好本事。這纔叫個雞巴。眞好漢。奇姐笑道。你先把金三也罵夠了。此時也不用你誇他。你下去罷。叫金三道。都完了。你也起去罷。那金三看了王彥章這一番狂弄。又見迎兒這一種騷浪。他的陽物又有些硬氣。見迎兒纔下床。他來拉着道。你纔笑話我不得硬。這會子怎又起來了。我再同你弄弄。足足興。迎兒用指頭在他臉上一掃。道。不害羞的。還想受罪呢。雞打〖月戎〗一般。你硬一百回。還不如別人一會呢。我一輩子沒有人弄。也不稀罕你。衆人齊笑。連金三也笑起來。時已將晚。吃畢飯。掌上了燈。奇姐道。拿酒來。論功行賞。王彥章三大杯。次者兩杯。又次者一杯。向金三道。你跪苦了。雖不濟。也賞一杯。大家說說笑笑。吃了一會。奇姐摟着牛耕上床同臥。衆丫頭各尋日間的伴侶。牛耕先弄的那楊嬌兒跟住王彥章。道。奶奶同相公去睡。我應該是你的。迎兒道。我同姐姐伴他罷。嬌兒笑道。你各人有對子。如何同我共一個。迎兒道。他也算得個人。我是不要他的。因低聲道。好姐姐。你看奶奶那樣本事還敵他不過。你留着我。或你乏了。我與你做個替身也好。你只當積陰隲罷。拉住王彥章。道。姐姐就殺我。我也不放他的。嬌兒見他有些着急。笑道。我倒肯容你。怕金兀朮捨〈捨〉不得。金三道。罷罷。咒罵得利害。我不敢惹他。我各自睡罷。衆人又笑了一陣。方纔各寢。一宿淫媾。自不必說。過了幾日。奇姐的那肉發興起來。又叫了衆男女到跟前。道。今日再弄個新樣兒。叫丫頭們將紅氈鋪在地板上。上設錦褥。擡過一條春凳來放着。又叫取一罎酒來。道。這做罰酒。吩咐道。都脫了着。衆人齊脫光。奇姐道。今日先男後女。指着金三道。你不濟。用你不着。你只好等人弄。你就頭一個爬在春凳上。他只得爬着。奇姐又指着一個小子。名李四。混名叫做疙瘩頭。說道。你就弄金三。你道怎麼叫做疙瘩頭。他的陽物只得一虎多粗。有六寸來長。一個龜頭像個大蛋一般。衆人起他混名叫疙瘩頭。那金三道。我造化低。不叫我弄人罷了。還叫我捱這大疙瘩。衆人笑道。這只怨你的㞠子不爭氣。不要怨人。李四道。你不要怕。我多用些唾沫就是了。他搽了。往糞門中一頂。那金三雖是弄熟了的。但這頭子大得利害。他咬着牙。哼的一聲。纔被他弄了進去。出了一口氣。道。夠了。我受得了。那奇姐又指着一個道。你就弄李四那小子。他就揷上。一個個挨\endnotemark[12]次弄上了。只剩牛耕奇姐王彥章三個。奇姐叫牛耕道。你弄孫五。牛耕也弄了進去。又叫王彥章道。你的本事好。服事你相公。王彥章不敢草次。用了許多津唾。慢慢的頂入。奇姐笑道。該我弄你了。兩手〖扌扉〗着屁股。也不用唾。對準往裡狠狠一下。進去半截。王彥章道。奶奶也略用點唾沫是呢。幾乎把我的弄裂了。奇姐笑道。前日你把我也弄夠了。我這算報仇。王彥章道。料道弄不死我。我捱着。奶奶索性弄到根罷。那奇姐往裡幾下。弄沒至根。王彥章道。大家動罷。奇姐道。且不要動着。叫丫頭取了幾塊舊紬帕來。道。你每人拿一塊兜着下身。都過來看着我們弄。等弄完了。看你們淌出來的騷水。論多少罰酒。多的多罰。少的少罰。衆丫頭都笑嘻嘻依着兜上。又叫到面前來看着。說道。動呀。大家一齊抽動起來。先還不覺。後來一片聲響。又是那笑聲盈耳。不多時。早有幾個完事的伏着不動。那不曾洩的還亂抽亂拱。又過一會。只有王彥章同奇姐不曾完。奇姐扳着王彥章的胯骨。王彥章扳着牛耕的胯骨。搗個不歇。奇姐往下一送。王彥章也往下一送。兩人的力。弄得那牛耕快活非常。哼聲不住。多時。奇姐興過。說道。都歇了罷。拔了出來。王彥章雖未足興。不敢不遵。也只得拔出。衆人挨次起來。那疙瘩頭往外一拔。金三兒一個大屁。異常響亮。衆人大笑道。好東西。金三笑道。你們笑甚麼。這叫做放砲收兵。奇姐驗看衆丫頭的帕子。無不(一)個不淌得精濕。每人罰酒一大鍾。歇息了一會。奇姐道。丫頭們看得苦了。都過來仰睡着。衆丫頭正都急得難過。聽說。忙忙睡倒。都將兩腿蹺開等候。奇姐道。不論誰弄誰。每人輪流一百下。只要狠狠的弄。不管他們丢不丢。丢了是他造化。不丢怨命。要弄得輕。罰酒一杯。不許多抽。多的也罰酒。金三叫他在傍數數。數差了也要罰。金三道。我不爲(會)弄罷了。難道數數都不會。他遂坐在紅氈子上。道。你們弄。讓我數。那奇姐就到了一個丫頭身上弄上了。道。你們都弄上了。讓他好數。王彥章就爬在迎兒身上。奇姐一看。道。相公同我並你們九個人。只八個丫頭。少一個。怎麼處。也罷。你們那個不濟的情願吿饒。就免了罷。這些小子都好此道。聽得這話。你看我。我看你。都不做聲。那牛耕先自己弄丢了。又被王彥章弄得他渾身通泰。覺得乏倦。便道。叫他們弄罷。我睏了。且去睡一覺再來。遂到房裡睡去了。奇姐道。你相公旣懶頑。你們各尋對兒弄。衆人都弄上了。奇姐看見。叫金三道。你好生數。遂大家一齊動作。自首至尾的抽將起來。一下重似一下。數到一百。丫頭們也有丢的。也有不曾丢的。只有王彥章弄那迎兒。他那長物直攮到底子。下下皆中要害。只四五十下。迎兒就丢了一次。此時又將要丢。見數已足。忙把兩手摟緊了他〈們〉的腰。把屁股往上亂就。道。好哥哥。好老子。你可憐我。再抽幾下。這一歇。我就要死了。活祖宗。我哀求你。王彥章見他騷得可憐。也十分動興。又狠搗了幾下。只見他鼻孔中哼了幾聲。道。哎喲。好親哥。可夠了我了。摟着王彥章親了幾個嘴。奇姐笑道。丫頭不遵令。王彥章徇私。每人罰一大鍾。二人吃了。奇姐道。不要亂了。挨着換。奇姐爬到迎兒腹上。衆人都挨次換轉。正纔要動。只見金三兒道。哎喲。我多喒倒洩了。淌了一氈子。衆人都笑得打鉄(跌)。那迎兒接口道。我勸你倒不如割掉了。當個老公罷。那東西還要他現世。金三兒道。你笑話我。有人還愛他呢。迎兒笑道。只好石女兒愛他罷了。女人們是用他不着的。奇姐道。動罷。又一齊抽將起來。到了七八十下。迎兒將奇姐的屁股兩手儘力下搬。奇姐笑道。你怎下死力扳着我的屁股。扳得我不疼麼。迎兒道。奶奶。你是我的恩主。只得二十來下了。說不得你忍着些。我扳着你還有些力。大家弄足了數。又輪班專(轉)換。奇姐道。這一回大家弄個快的。遂一齊亂抽。那金三兒數不淸了。舌頭在嘴中亂轉。說不明白。奇姐大笑道。你說會數。如何數不來了。罰了一碗酒。這一陣緊抽。有幾個洩了動不得的。每人罰了一大鍾。又弄多時。奇姐把八個丫頭都弄遍。也興足歇了。問王彥章道。你呢。他答道。我還早呢。奇姐道。不要苦樂不均。那幾個先歇了的丫頭們都沒足數。差多差少。你都去補足了罷。這幾個沒有弄足數的丫頭正在那裡暗惱。聽了這話。一個個笑逐顏開。道。奶奶恩典。眞是公平。這個道。我差四百。那個道。我少五百呢。又一個道。該我先弄。那一個道。是輪着我的。相爭相鬧。奇姐道。都不許吵。叫取了些拳馬兒來。叫他們幾個猜狀元拳。誰先猜着誰就先弄。遂一齊猜。一個贏了的。王彥章也不等別人猜完。拉過來就弄。那丫頭也巴不得弄足了數。一個個挨次補完。那王彥章就洩了。他方弄了個心滿意足了。過了十數日。奇姐這陽消陰盛的時候了。叫了衆人到跟前。指着八個小子說道。你們雖都同我弄過。或今日這個。明日那個的。今日叫你們均沾雨露。你們憑我指名叫着。到我身上來弄。不許爭嚷。不拘工夫多少。只等你們弄丢了爲度。却不許你們洩在我的裡頭。怕小肚子脹。放一個碗在傍邊。臨洩時拔出。冒在碗裡。到臨了看有多少。王彥章道。小的也要求奶奶與我弄丢了呢。奇姐道。你利害。恐我熬不得。也罷。你同相公先弄。等他們輪完了。也就好一會工夫。你要不住的抽。也就有好幾千下了\footnote{恐胡旦之股尚未必能禁止。牛耕也可謂跨竈。}。然後到我跟前。或者差不多了。奇姐遂脫了衣裳。衆人都脫盡。奇姐叫丫頭拿個墊子。雙摺着墊在股下。仰臥着。王彥章也拿個墊子。摺了做(放)在奇姐身傍。與牛耕墊着屁股。扶他仰睡。架起兩腿。將那大長的陽物對面送入他糞門中。不住的輕抽慢扯。看着奇姐作用。奇姐向金三道。你不要說我偏心。你實在算不得。只好在傍邊看着罷。那小子急得幾乎掉淚。跪下不住叩頭。道。奶奶的恩典。我雖不濟。求奶奶一視同仁。賞小的也弄弄。沾沾大恩。說着。只是叩響頭。震得地板通通的響。奇姐見他這個樣子。心中可憐見的。想了想。笑道。也罷。你就在擋頭陣罷。他滿淚(臉)是笑。答應一聲。爬起來就上床。伏在奇姐腹上。說道。蒙奶奶大恩。但小的這一弄進去就要冒的。恐一時拔不及。冒得滿到處。奶奶不要見怪。奇姐見他〔如〕此說。倒反心愛。便道。許你洩在裡頭罷。別人不許。金三笑着向迎兒道。你笑罵我。你看奶奶獨加恩到我呢。一面笑着把陽物送入牝中。竟動有二十來下纔完事。他喜道。造化造化。今日爭起(氣)。我好快活。奇姐笑道。果然你今日算好的。還動了幾動。他一面抽出來一面說。一來是奶奶的恩。二來是奶奶這寶貝好的緣故。指那迎兒道。他不怪自己的下作屄不好。倒罵我不濟。那迎兒一口唾沫吐了他一臉。他指着笑道。你笑話我弄進去就冒了。你還不等人弄進去。怎就冒出這樣一大朶子來。說得衆人都笑了。奇姐指名。一個個叫着上身去弄。也有抽幾十下的。也有一二百抽的。也有三四百抽的。只疙瘩頭抽了有千數纔完。奇姐同別的小子弄時似有如無。只疙瘩頭弄得他纔哼喞了幾聲。屁股略動了動。衆人到臨洩時都拔出。拿碗接着。冒在碗內了。此時王彥章也把牛耕抽了幾千下。那牛耕也興足了。王彥章見衆人上上下下。眼中急得冒火。見都完了。道。奶奶。我來罷。奇姐點了點頭。他忙忙拔出。就到奇姐身上。忙忙揷進。一口氣就有千餘。奇姐通身爽到(利)。把兩隻腿勾住了他下身。兩手摟緊他腰背。又一會奇姐渾身都動。口內嬌聲嚦嚦。聽得人魂消。他丢了。雙手捧着王彥章的臉親個嘴。道。還是你〔行。〕那小子見奶奶獎他。又重鼓威風。沒稜露腦抽了一陣。道。我也要完了。纔要拔。奇姐兩手勾着他。道。你也洩在裡頭罷。那小子又着着實實抽了幾下。方伏着不動。定了一會。方下身來。先牛耕被王彥章弄得渾身酸軟。停了片時。且奇姐弄得那嬌聲騷態。着實愛人。他爬起。拿枕頭靠着。看他們弄。王彥章弄完了。他又覺興動。奇姐纔要起來。他道。且住着。等我來找個零。奇姐就不動。他爬上身來。因看得火動久了。只幾十抽就完了事。那奇姐也十分興足。覺陰中精滿。拿塊紬帕用手搗住。坐起看那碗中。笑道。也有這麼些呢。叫丫頭倒在淨桶內。他也下床坐在淨桶上。挖出許多黏黏涎涎的東西。把牝戶揩淨了\footnote{奇姐一敵九人。較女敬德還多一個。}。到一張醉紅(翁)椅上坐着。笑對衆人道。你們的東西我今日全試明透了。我替你們考個等次。看你們心服不心服。指着王彥章道。你的物件旣長。工夫又久。只可惜細些。若再有李四的疙瘩那樣粗。就眞是個異寶了。雖說。此衆人中少不得算你第一。又向疙瘩頭道。你的陽物也不爲短。工夫也還看得過。若得上下一半(般)粗。王彥章也不能僭你的先。可惜犯了夵字的病。只好算第二了。又叫過鄭二周四來。道。你兩個大小也差不多。都不過三幾百的本事。指着周四道。你弄得比鄭二略在行些。你算第三。他算第四。只見那這金三兒笑道。我不消奶奶批評。我自己會考。我又小又快。又細又軟。倒過來我是頭一個。我算第八。奇姐衆人都笑。奇姐又指着錢五孫七道。你兩個眞是一對。大小長久都是一樣。但錢五又不及你些。孫七第五。錢五第六。只見那李六道。奶奶考的我不服。我的㞠子不比他兩個的大些。就是我的工夫雖趕不上王彥章疙瘩頭兩個。比他四個的都長久些。怎麼倒把我考在第七。奇姐笑道。金三自己還知道短處。你竟不自知。還不如他了。這樣說。還該考在第八纔是。你的東西雖大。却不堅硬。男女幹事全要陽物像鋼鎗一般\footnote{戳通肚子。奈何。}。下下着實。方有趣味。你的弄在裡頭。竟不知覺。間或頂在花心上。倒軟了回來。再不得爽利。不要說你有幾百抽的本事。就有徹夜的工夫。有甚麼妙處。指着金三道。他算第一不濟了。像他方纔抽的那一二十下。我還覺得有個硬東西戳得癢癢酥酥的。你弄了那一會。我裡邊竟不知道。那李六被這一番話說得垂首喪氣。迎兒在傍揷口道。我前日起他個混名。叫做李皮條。他還罵我呢。笑着向李六道。你聽奶奶說的。我起的混名錯不錯。李六道。閉着騷嘴罷。蔣賽貓。奇姐笑問道。你怎麼叫他蔣賽貓。李六道。那貓叫秧。還不等公貓上身。就貓貓的叫。直等弄完了。纔不做聲。他只㞠子挨到問(身)上就叫起來。弄完了他還不住聲。所以我叫他蔣賽貓。奇姐大笑道。這名字不錯。迎兒道。你把嘴夾着罷。李六笑道。你要夾得住。倒沒有那些水淌出來了。衆人都笑了一陣。金三兒向奇姐道。奶奶方纔批評我的那幾句。小的臉上爭了多少光。眞感恩不盡。奇姐對衆丫頭道。你們都是我細賞鑒過的。我也替你們考個次序。那八個丫頭都〈都〉赤條〔條〕笑嘻嘻齊站在面前。奇姐指着一個馮美兒道。你的這陰戶要算絕品了。又暖又乾還在次。弄將進去。陰門像個荷包口兒緊緊收住。還不足爲奇。那裡面軟膿膿裹住陽物。樂不可言。大約千人中還選不出一個來。自然是第一了。因問衆小子道。你們都同他弄過。我說的是不是。衆人齊應道。我們每常同他弄。只覺得快活有趣。也不能說他的妙處。纔聽奶奶的話。一絲不錯。果然出奇。那丫頭得這番通獎。笑着滿面欣欣自得。奇姐指着楊嬌兒道。你雖不及他的陰戶。淺得有趣。下下搗着這花心。你也受用。男人也受用。該在第二。又對迎兒道。你這的風騷在他衆人之上。就是你的陰戶也不在美兒嬌兒兩人之下。可是李六說的淫水太多。一弄進去。抽不幾下。那水一陣陣往外冒。令人的陽物都揷不住。弄一次要拿盆接着。大約也有半盆。那迎兒笑道。奶奶說的怕人子剌剌的。我這是條肉溝。又不是陽溝。那裡就有這些水。金三接口道。你前世是個水淹死的人托生來的。脹了一肚子水。拿肉棍子一道(通)。水就打這洞裡淌出來了。說得大家都笑了。奇姐指着沈艷兒道。這丫頭的生得異樣。你們可覺得。衆小子們道。小的們那裡知道這些奧妙。奇姐笑道。蠢材。可惜屄與你們瞎弄。他的陰門生得甚高。在小肚子下。離糞門有四五寸遠。你們看看別人有像他的麼。衆人笑道。是呀。別人果然沒有。可惜我們都混弄了幾年。奇姐道。他的又光又肥。可惜太鬆。再要緊暖些。也算得第一二。迎兒第三。他只好算第四了。疙瘩頭道。是眞。我弄別人。到門口還要緊緊的。惟獨他。輕輕一送就到根。全不知覺。奇姐又指着個韓媚兒道。你無可取。一個陰門同糞門連在一處。對面再不好弄。所以我每常不是叫你上我身來。就是叫你馬爬着往後弄。都(却)有一件妙處。是婦人中極難得的。問道。你們可知道。衆人道。小的們越發不懂得了。牛耕忽說道。我覺得有一種異樣。但同他弄裡(那)快活的時候。像有些微微的香氣。說不出來的那一種甜絲絲的味兒。在他屄中冒出來。可是麼。奇姐笑道。着。還是你知些竅。這些蠢奴們。別的不知道罷了。難道連鼻子都沒有的。衆人說道。我們也常聞見些香味。只說他用香肥皀搓的香。那裡知道是那裡頭的妙處。只見金三道。我的武藝不濟。也從沒有弄得他快活。並不曾聞過這香。走過來低下頭道。我聞聞看。那丫頭笑嘻嘻一個大嘴巴。金三摀着臉道。我好意贊你。你倒打我這一下。他了(們)混搗倒罷了。我連聞聞都不依。那丫頭笑着又一張手。他忙躱開了。王彥章笑着向奇姐道。他前日一個笑話。我還不曾吿訴奶奶。我同他弄了一會。他的水把糞門都淌濕了。他一時高興。叫我狠狠的弄。我便出出進進。狠狠的亂搗。忽然一下戳到他糞門裡頭去。因用力大了。幾乎攮到了根。他不怪自己的兩個眼子長在一處。倒還罵我。把我擰了幾下好的。奇姐笑了一場。指着一個陳鶯兒一個褚燕兒道。你兩個分不得好歹。都深得沒影。我的也有六寸多長。從不曾挨着底子。鶯兒的又還緊暖些。算第六。燕兒第七。王彥章道。奶奶眞是別寶的回回。不要說奶奶的東西。我的比奶奶不還長個寸把。還摸不着他兩個的底子呢。小的前日說他欲窮到底。除非丈八蛇矛。他還罵我嚼蛆。奇姐指着衛嫣兒道。他的也不爲深鬆。也不爲濕冷。倒好個陰物。只可惜有些臭。那個婦人的不臭。但洗洗就好了。你的便拿一擔香熏了。也是沒用。夏天勤洗晾着些。還不覺。冬天蓋着棉被。越弄越臭。衝人腦子。任你怎麼高興。那一熏。就毫無情趣。這是胎裡帶來的病。也怨不得。只好屈你做第八了。那金三笑道。我有個笑話講與奶奶聽。

\begin{quotation}

一個瞎子娶了個老婆。陰臭得當不得。那瞎子怨恨道。怎生這樣個臭東西。那婦人道。你不要沒福。這是鮝魚香。上等的好物。你倒嫌臭。那瞎子想了想。笑道。不錯。果然鮝魚是這個味兒。瞎人疑心最重。他要出去算命。再三囑咐女人道。你千萬不要到大門口去。日日如此。那婦人依他。只在屋裡坐着。一日。瞎子回來。恰好一個賣鮝魚的擔子歇在門首。他聞得那味。一進門就亂嚷道。我叫你不要出來。如何又到門口來站着。一路〖口么〗喝。問了進來。那婦人正坐在屋裡。問道。你叫些甚麼。瞎子發急道。叫你不要出去。你又出去做甚麼。婦人道。你見鬼來。我坐在這裡。誰去來。瞎子道。你還強嘴。你要不曾出去。怎麼鮝魚味兒都香到街上去了。

\end{quotation}

衆人大笑。金三兒笑向衛嫣兒道。你明日也要香到街上去呢。那嫣兒笑着罵道。斫千刀嚼舌根的。人說只有爛了的棗兒。沒有爛了的嫂兒。我的鮝魚臭。還有人同我弄。強如你那秤鈎兒一樣的東西。還沒人稀罕呢。金三道。誰說。你們這些壞人罷了。奶奶現還心疼我呢。你笑話我是秤鈎兒。我就說個古話你聽。

\begin{quotation}

一個後婚女人要嫁。託那媒婆說。我要像鐵一樣的東西。我纔嫁呢。媒人說成了親事。嫁了去。晚上成親。弄了幾下。那㞠子彎了過來。婦人急了。次日。罵媒人道。我說要像鐵一般的。你倒尋了個秤鈎樣的東西來。那媒人道。你好呆。秤鈎兒難道不是鐵的麼。

\end{quotation}

說得衆人哈哈大笑了一陣。奇姐又叫衆小子道。你們的我都考過了。我的你們也弄過多次。大家也說說我的何如。王彥章道。奶奶的眞是絕頂的了。又淺又熬得久。下下攮着底子。果實有趣。奇姐笑道。人豈不自知。我的也未必很淺。還是你的長。要說熬得久。指着衆丫頭道。他們都不及我。疙瘩頭道。我只覺奶奶的緊得有趣。奇姐搖頭道。也未必。還是你的頭子大。然而也還不很鬆。一個道。奶奶的眞乾得好。奇姐笑道。乾也不能。指迎兒道。還不像他那些水。一個道。奶奶的那裡頭像個火爐。弄在裡面。似拿熱水泡着一般。受用多着呢。奇姐道。很熱也未必。我自己覺得裡頭還不冷。一個道。奶奶是十全的。用手指着陰戶道。你們看。不像沈姐姐一般的高麼。奇姐笑着用手摸着陰門。道。我的雖沒有他的高。也還不十分低。金三道。我說個笑話兒奶奶聽。

\begin{quotation}

一個呆子娶了個老婆。摸着了陰門。驚道。甚麼人研了這麼個大口子去。那女人道。是屄。呆子道。造化。虧是低。要高些。連腸子都研出來了。

\end{quotation}

大家笑了一會。周四道。美人在風流。你們不在行。奶奶的風流還有對兒麼。這就是普天下沒有的。奇姐笑道。風流二字。我不敢多讓。要說普天下沒有。就是謬獎了。又一個道。你們各人說的只是一樣。據我看起來。奶奶的這件寶貝。乾也有。淺也有。緊也有。暖也有。高也有。沒一件不是好的。奇姐笑道。婦人的陰戶有五好五不好。五好呢。是緊暖香乾淺。五不好呢。是寬寒負(臭)濕深。我的雖五好未必俱全。大約五不好也沒我的分。那奇姐見金兀朮獨不做聲。笑向他道。你雖然不濟。不曾十分嘗着滋味。你也還弄過多次。你就不批評一句。他笑道。奶奶的好得很。我也沒得說。奇姐道。好歹不妨說兩句。他走近前。跪下。用手〖扌扉〗着牝戶。聞了聞。道。我只覺得香。奇姐道。這是你假奉承我的。雖不臭。要說香也不能。金三兒道。小的可敢說謊。看見奇姐陰門內如龍眼大一塊肉。碎糟糟似一朶花。心愛極了。伸舌頭舔了幾舔。又拿嘴合在陰戶上含着。咂了幾咂。道。不但香。還甜呢。又伸着舌頭到陰戶中亂舔。奇姐覺甚有趣。把屁股往外探探。身子靠在椅背上仰着。〈他〉竭力舔攪了一會。奇姐心愛得了不得。摟着把(他)親了個嘴。道。你雖然不會弄。倒知趣愛人。此後奇姐分外疼他。倒常同他弄弄。衆小子道。奶奶是菩薩心腸。個個施恩周到。這幾個丫頭中。奇姐獨鍾愛迎兒。因他性情風騷。與己相合。他有一種生成的騷態。並非矯柔(揉)造化。陽物只送了進去。他兩腿似綿花一般。一癢過頭。陰門上腆。渾身如弱柳迎風。口中的淫聲艷語無般不叫出來。到將丢之時。星眼矇矓。雙蛾微蹙。那種騷態。不要說同他弄的人消魂。傍邊看的人更覺筋酥。奇姐要同人弄的時候。先叫一個同迎兒弄。他自己同着那小子在傍看。看得陰中之水不住長流。那小子的陽物脹得靑筋暴湛。看到十分忍不過了。然後同那小子去弄。那陽物分外堅硬。他自己陰中更覺有一種說不出的妙境。所以但要幹事時。定叫迎兒做一員先鋒。那丫頭也乖巧。善能迎合主母之意。奇姐待他也十分加厚。有幾句道這奇姐的異處。

\begin{quotation}

窈窕內。腰間有健男子之碓。嬌媚中。胯下兼數婦人之勇。孽具偏能識竅。嘗得出衆女子之乾濕深鬆。牝中更善面評。辨得明諸狡童之細長粗短。淫婦班中推獨異。妖狐隊裡可稱尊。

\end{quotation}

他夫妻淫穢的事。也不能盡述。只看牛耕這樣生出來的兒女。非雜種而何。有此聲名在外。所以人皆稱爲雜種牛宅。眞可發笑。再說牛質有個妹子。嫁了一個姓文的老學究。他生了個女兒。小名貞姑。自幼父親敎他念書。把古來節烈的事常常講說與他聽。到大了。貞靜賢淑。言笑不苟。人都稱他爲迂夫子姑娘\footnote{與腐頭巾阿姐遙遙一對。}。貞姑嫁的丈夫。姓鮑名復之。是一個少年好秀才。他是鮑信之\endnotemark[13]的堂弟。這貞姑嫁到他家。眞是四德咸備的婦人\footnote{鸞(寫)一貞姑。爲奇姐作一反襯。貞淫不並立。故奇姐死於他手也。}。夫妻相敬如賓。貞姑常到牛質家來。奇姐見他古古板板。無多言無妄笑。他本是個騷淫無匹的人。眼睛中如何看得慣這等迂腐女子。心中嘗想要弄他一弄。破破他的腐氣。但貞姑總不留宿。未得其便。這一日又來走走。奇姐定要留他過夜。說了許多賢德的話。道。我們姑嫂雖會過多次。從來姑娘沒有在這裡過夜。姑娘若不見棄。我們今晚同宿一宵。說說家常。也見至親的親熱。那苟氏疼這媳婦像心肝蒂兒一般。見他要留小姑娘。也再三相勸甥女。貞姑見舅母表嫂這樣好情。只得住下。夜間奇姐叫牛耕往書房去睡。他陪着貞姑說長道短。坐到三更有餘。有心算計無心。那貞姑見表嫂這般親熱。雖然睏極。怎好撇了去睡。只得坐着。奇姐見他睏得很了。然後道。姑娘像是倦了。請安歇罷。一同上床。那貞姑睏了的人。倒下頭便睡着了。奇姐各有心事。他却不睡。等了一會。聽他睡沈。叫了兩聲。又推了幾推。總不見他動。遂揭開被。輕輕將他褲帶解開。把褲子褪下。扶正了他身子。輕起兩股。上得身來。把那一段硬肉慢慢塞了進去。弄將起來。及貞姑驚醒之時。已被他抽洩(拽)數十度矣。貞姑大驚。不知是誰。忙叫道。你是甚麼人。奇姐壓在他身上。附耳道。姑娘。是我。貞姑見是表嫂。就急伸手一摸。竟是腰中之物。忙道。你快下來。奇姐笑嘻嘻的道。你與我姑嫂頑耍。何妨於事。貞姑怒道。你不下來。我就叫喊了。把他推下身來。忙把衣裳穿起。下床坐着。奇姐笑道。姑娘。你又不是女孩。還怕羞麼。我們女人對女人頑。虧他(你)也認眞惱麼。這是極快活的事。你怎做這個樣子。還向他說說笑笑。他一臉怒色。總一言不答。坐到天明。梳洗了。定要回去。牛質同苟氏再三留他吃了飯去都不肯。立逼叫轎子去了。到了家中。怒容滿面。鮑復之道。你在舅舅家來。何故這樣煩惱。再三相問。總不回言。鮑復之不解其故。坐了一會出來。覺得心驚肉顫。坐立不寧。纔要進去。聽得他的妹子大叫道\footnote{他這妹子伏後。}。不好了。哥哥快來。嫂子上吊呢。鮑復之慌忙跑進去。見妹子在窗縫裡張。房門關着。將窗子打開。跳將進去。見貞姑懸樑高掛。忙解救下來。幸而未久甦醒了。放聲大哭。鮑復之問他何故。他道。我不幸爲人所汚。尚何顏生於天地之間。鮑復之叫妹子出去。細細問原由。他方說爲易氏所淫。詳細相吿。鮑復之大笑道。你想差了。婦人家要自己做了醜事。一死應該。若無心被男子暗算。尚非己罪。何況婦人與婦人淫戲。這有何妨。何故尋此短見。我常見書內說。婦人中有此一種可男可女之人。名爲二形子。又叫做二尾子。即此也。你若忍得過去就罷。不然思一報復之計\footnote{提出尊名。}。便可出你之氣了。貞姑聽了丈夫之言。恍然大悟。便道。他雖是婦人。其心不端。他設計誘我。情更可恨。我必要雪了此恨。心纔可釋。鮑復之道。你只須如此如此。便可報復了\footnote{奇姐算計貞姑如此。罪固難辭。鮑復之設計如此如此。未免太惡。}。貞姑大喜。鮑復之到外科醫生處配了些爛肉的藥來。付與貞姑收好了。過了些日子。十一月半後。牛質生辰。貞姑先一日去拜壽。把那爛藥裝在荷包內。緊帶在身邊。到了牛家。奇姐滿臉笑容迎着。道。前日怎樣得罪了姑娘。一刻也不肯緩就回去了。貞站也假做笑臉相對。却不答言。晚間也不用人留。竟欣然住下。苟氏仍叫奇姐伴他同臥。上床之後。奇姐笑着道。姑娘。你太認眞了。我姑嫂頑耍。怎麼也煩着惱。貞姑道。我不惱。那日怪有些害羞。故此回去。奇姐道。你我都是婦人。羞的是甚麼。二人睡了多會。貞姑等他來下手要算計他。總不見他動作。只得睡了。到了天色將明。一覺醒來。心中想道。想是他前次見我惱了。所以他竟不來。如此這恨如何報得。我旣被他淫過。何妨捨身報怨。反伸手去摸奇姐的下身。貞姑那知他到了下半月是硬不起來的。這日已是十七。摸着了軟叮噹的一條粗肉拖在陰門上。此時奇姐也醒了。笑道。姑娘。你想他麼。他却硬不起來了。貞姑縮下身去一看。見與男子的全不相似。一把攥着。放在口中吮咂\footnote{大約貞姑生平來此是頭一次。}。笑對奇姐道。我前次睡着了。他大硬的偷我。這次我明公正氣要他弄弄。他却稀軟的。我恨他得很。我咬一(下)他一截子來罷。奇姐也只當他是頑話。笑說道。你捨得咬就咬。不意被他猛然一口。咬得伶仃將斷。奇姐哎呀一聲。疼得昏暈過去。貞姑忙將帶來的藥取出。替他擦上許多。忙穿衣下床。多時。奇姐醒轉。叫苦連天。苟氏知道了。忙下來看他。問他何處疼痛。他又不好說。只得說下身疼。貞姑忙忙收拾。辭了回去。香姑也在家中。因貞姑同奇姐睡。他在計氏處宿。聽見奇姐忽得重疾。連忙來着。低低細問。奇姐吿其所以。香姑看了看。心疼得要死。又無法替他救治。惟有嘆氣痛恨。抱怨貞姑頑得太毒。牛耕在外邊正陪那來拜壽的人。聽見小子們悄悄吿訴說奇姐不知何處疼痛。十分利害。急得要進來看。又不得空。多時。人客略散。纔忙忙進來。見奇姐臉都疼白了。眼淚長流。連忙問他。奇姐吿知其故。說貞姑頑得這等惡毒。還不知他是安心來報前恨的。牛耕忙揭門(開)被一看。幾幾將斷。血流滿褥。急得只是捶胸。小姑咬了嫂子陰門之物。又吿訴不得人。又發作不出。只得忙叫人去買刀鎗藥來擦上。那知他已經上過爛藥。一日一日漸漸腐爛。臭不可聞。奇姐疼得晝夜昏暈幾次。叫不住聲。因在陰門之上。又不好請太醫看。只說下身破了。拿藥來敷擦。毫無效驗。牛耕差人往丈人家去說信。易于仁袁氏鄒氏都來看視。惟有嘆氣而已。不上一月。把一個花枝般的美人瘦得形像俱脫。一身僅存皮骨。那段肉直爛到根子底下。連陰門都爛得有小碗大一個窟窿。先是尿脬破了。小便不住長流。又過了兩日。腸子都拖了出來。然後氣絕\footnote{奇姐淫穢的事敍了幾半本。只這一段是他的正傳。先那幾千百語。不過是這段的引子。凡看書者。當留心看這幾句。先那些語。不過帶眼看過。若只注目在前那些話。反將此忽略。則大誤矣。}。一家大小無一不哭。牛耕不但喪室。而且亡夫。哭得悲慟不消說。就是這些小子丫頭。想他的陰門。念他的肉具。況素常待他們極其恩厚。個個都哭得傷心。丫頭中惟迎兒有知己之感。更自悲傷。小子中王彥章金三兒也悲哀特甚。而金三念奇姐那一番相待之恩。哭得死而復甦。香姑聽見奇姐的凶信。忙坐轎子一路哭了回來。進門撫屍慟哭。悲切得了不得。像死了丈夫似的。牛質苟氏那裡知其中備細。都說他在生賢德。不但這小姑疼愛嫂子。哭得如此哀慟。連這些下人感恩。悲傷到這個地位。都嘆牛耕沒福。可惜失此賢配。把個苟氏哭得嘔了幾碗血。病了一場。幾乎喪命。媳婦雖然賢孝。婆婆何得傷心至此。內中有個緣故。十月內。一日大雪。牛質同妻妾擁紅爐飮美酒。慶賞豐年佳兆。到晚都醉了。苟氏許久不會胡旦。趁着牛質醉醺醺同衆妾取樂。他便叫紅梅約了胡旦到一間秘室內相晤。二人久濶。弄了一度不已。又還要個連拳。睡到黎明將別。戀戀不捨。又幹了一次。正纔完事。忽聽得牛質說着話走來。胡旦膽都嚇碎。精赤條條跳下床。忙把衣服鞋襪抱在懷中。鑽入床下躱避。苟氏也慌。恐他進來高興試出。忙把紬帕用指頭掏着。也顧不得疼痛。將牝內摳挖。擦得乾乾淨淨。聽了一會。不見動靜。叫紅梅張張。牛質去了。忙叫胡旦出來。穿上衣褲。着紅梅送他出去。你道牛質是受用慣了的人。大雪天氣。這樣早做什麼。他有一個朋友在遼東做買賣回來。送了他兩張出奇的好貂皮。他偶在族兄牛尚書家。牛尚書要買好貂皮做帽套。看了許多。總不像意。他說起有兩張好的相送。及到家要取。却說不眞放在何處。尋出好些皮子來都不是。這夜因大雪。想起許了尊兄之物。不送了去。不但說失信。還恐疑是捨不得。左思右想。忽然想起收在這秘室的外間櫥內。故此大淸晨急忙起來拿出。差家人送去。這胡旦已是四十外的人。又作喪得虛飄飄一個空殼兒。這一嚇。又一凍。成了個急陰。第二日就遊地府去了。苟氏同他相與了二十多年。兒子長了若許大。孫子都見了。雖然牛質不知。他自己心中明白。如何不心疼。況這一死。明明他是因他這三弄一凍一唬之故。怎不傷心。又不敢哭。噎在胸頭。只好夜間在枕上暗暗飮泣而已。今見媳婦又死了。又是一心疼。兩事併一。那得不到悲痛嘔血的地位。那紅梅也是四十多歲了。牛質仍收回聽用\footnote{忙中夾寫此一句。非無謂之言。做後日收伊子地步。}。再說那鄒氏當年得孕之初。老狐云此女後當以淫死。果應其言。此狐亦神矣哉。易于仁穢淫。鄒氏妖淫。生此不陰不陽之奇淫。而奇姐死法亦奇。萬惡淫爲首一語。可不戒乎。易于仁雖有勤壽二字(子)。而其實宗支已絕。牛質雖有一子數孫。而血祀亦斬。淫之一字。更屬寒心\footnote{此等語乃此書之大旨也。}。按下不提。且說那郝氏久要改嫁竹思寬。因女兒前次同他成了寃家。聞聲即罵。恐不能相安。今見女兒嫁了鍾家。得了好處。他自己屢年來積得私蓄。約過千金。年紀尚未很老。捨不得竹思寬的那根異物。把他倒踏門招了來家。成其夫婦。那竹思寬又帶了個標致小子來。郝氏問他緣故。他道。這孩子是童百萬家賣出來的。老童就是鐵回子的妹夫。郝氏道。哦。我知道。代目原也是他家的。竹思寬道。我聽得人說。鐵回子這妹子着實不賢。大約是見這孩子生得乾淨。怕老童愛。他吃醋打發出來。我看見了。買了他來做個兒子。料道你我今生未必能生育了。郝氏也甚是歡喜。把他當親兒子一般。將他舊名的郞字去掉了。添了姓竹。就叫做竹美。郝氏的那財香丫頭也有十八九歲了。模樣也還看得。就配了竹美。做了一房兒婦。一家四口過活。不在言表。那錢貴自到鍾生家中。因無兩眼。只好高坐。他自思道。人之娶妻。原圖主中饋。我終日閉了雙目。如何料理家務。鍾郞雖是情深。說不出口。我也自過不去。又念代目數載相隨。知心貼意。遂將他收拾了。另備了一間房。要與鍾生做妾。叫他照料家事。那代目可有不願之理。暗地私喜。鍾生起初不肯。後見他意思眞切。兼之代目容貌原通。今長成人。出落得十分俏麗。若無錢貴相形。他也就算得中等佳人了。況且又頗知文墨。鍾生却也就逆來順受。晚間成其好事。那代目還是個處子。交合之際。逡巡畏避。一段嬌羞。自與久歷風波者不同。鍾生得嘗新物。方知個中又有此消魂妙境。輕憐重惜。十分鍾愛。事竣之後。聞(問)及他的家世。代目將他的祖父姓名。並他到錢家來的來歷始末原由細述。又說明他祖母父母的居址地方。求鍾生着人去問一問。鍾生次日着長班去訪。回來說道。問他的街坊鄰舍。都說數年前不知搬到何處去了。鍾生說與代目。落了幾點淚。只得罷了。過了三日。依舊到錢貴房中來宿。此後兩處分寢。他夫婦大小無事之時共坐。談談詩詞。說說家務。好生恩愛快樂。有幾句贊他三人。道。

\begin{quotation}

男同子建。女類夷光。評品丯婆(姿)。似兩朶瓊花倚着一株玉樹。形容態度。如一輪皎日分開兩片輕雲。把男子推班出色。到處成彈。將婦人接羽移宮。皆能合調。允矣無雙樂事。誠然對半神仙。

\end{quotation}

一日。錢貴偶問道。郞君那日說要訪宦蕚撇下跑去的緣故。郞君次日即有捷音。料不曾去訪。他也不見動靜。近來可知道些影響麼。鍾生道。我前日見邸報來。今上即位。知魏忠賢罪惡滔天。發往鳳陽守陵。後又被人參効(劾)。他覺事體不妙。於途中自縊。奉旨查他黨羽。一體拿問。前日二叔的親家勞御史。也是他的一黨。已經伏誅。勞家姊丈同大姐姐都發往陝西充軍去了。這宦蕚的父親原係他之門下。雖然漏網。恐事露連累。定然戒諭兒子。叫他謹守。他想〔是〕聽見〈這〉此信。故慌張跑去。那日他正在作惡之時。那一個寄書的來人。似遠行的形狀。大約即此。近日聽得說他收歛了許多。閉門在家不出。錢貴道。這廝惡貫滿盈。明歲郞君北上。倘高捷後。當發彼奸惡。彈其陰私。豈可容此匪人欺凌良善。鍾生道。賢妻謬矣。我若向日與彼無〖阝少日小〗。他正在熱鬧場中。我或徼倖一官。倒可上爲朝廷。下爲黎庶。彈他的罪惡。今日我與他有此一番芥蒂。且他目下又在有事之秋。君子不乘人之危。我若與彼爲難。雖公亦私了。人豈不以我爲挾仇報復之小人。與宦蕚又何異哉。此等無知之徒。只當付之於度外而已。況天理昭彰。惡人自有報應。只爭遲早耳。我何足介意。錢貴聽了。肅然道。妾乃女流。無識見淺。今聆君之言。不勝起敬。君有大量。必有厚福。妾一片恨彼之心。今亦冰釋矣。鍾生此後仍舊在家苦讀。以備明歲會場鏖戰。正是。

\begin{quotation}

不是一番寒徹骨。怎得梅花撲鼻香。

\end{quotation}

一日。鍾生正在書房看書。涉獵那些程文近作。聞得說梅生來訪。忙迎入共坐。鍾生道。連日未晤。兄今日到何處去來。梅生道。外面有一件可笑的事。兄曾聞否。鍾生道。弟在家兀坐。並不知道。兄幸見敎。梅生道。數月前不知何處來了一個邪道。據他口說。是江西廣信府龍虎山來的。姓張。是張天師的遠派子孫。也無從查考。他來到這裡。便串通了些走寺撞廟。持齋念佛的老道婆。他在油坊巷租了三間大樓。樓上供了無限的神像牌位。妖言惑衆。說他善替婦人們求子治病。禳災順星。但行好事救人。並不計利。只要婦女們潔淨虔誠去燒香祈禱。自然獲福。這些道婆替他四處倡揚。勾引這些無知婦人到那裡去。先去的還是小戶人家婦女。後來竟連官宦人家的夫人奶奶都去走動。或是丈夫。或是女伴。或是家人婦女隨去。都在樓下。只這一個本身祈福的婦人同兩個道婆上去。他說無故的人要到樓上。沖犯了神聖。不但無福。且要降禍。一上樓。就將一塊門板放下蓋上。人在下面。只聽得樓上搖得手鈴響。或急或慢。並不聽見念些甚麼。約有兩三個時辰。方纔開門下來。這些婦人也有去過一次再不去的。也有一個月去上四五次的。布施的錢米不計其數。也有人不信。疑是奸情。但去的婦人甚多。難道就沒有一個貞烈的。都任他淫汚不成。況且大官宦家夫人奶奶都有去的。又有這道婆同在樓上。猜不出眞僞。誰人肯多管這閒事。前承吾兄盛情。替小弟作月下老。娶了弟婦。家表兄知道了。自天長縣來與弟道喜。不想被他拿獲了奸情。把這妖道送官處死。道婆也校(杖)責了。殊快人心。鍾生道。令表兄尊姓。今在何處。是怎樣捉獲的。幸爲詳示。梅生道。家表兄姓林名忠。字報國。係天長縣人。乃先姑父之子。先姑父諱友梅。是個不求聞達。懷才抱德的隱士。當日同先父自幼莫逆。常笑謂先父道。我這個賤名。原取和靖先生妻梅子鶴之意。倘以令妹俯結絲蘿。豈不合了賤名。先父當日也極敬愛他。遂成了這親眷。先姑父這樣一個文墨之士。不想生得這家表兄堂堂英雄之表。虎面虬髯。濃眉的大目。眞使人望而畏之。他胸中韜鈐。那是他祖父所傳。不足異。而兩臂有千斤之力。武勇絕倫。眞爲奇特。他今年三十歲了。也不肯謀仕。只在鄕黨中做些濟困扶危的義舉。他有兩位結義的朋友。一個姓尚名智。一個姓慕名義。一個是家表嫂的令兄國守\footnote{借梅生口中閒話一提。後來出現。便似熟識。妙。}。都是英豪。那年先姑父去世。弟去弔喪。與他三位會過。那豪爽氣槪。自與世俗之鄙夫不同。與他共談。如飮醇醪。坐春風中。鄙吝頓消。前日家表兄到了這裡。在舍間小飮。聽得一個敝友說這妖道一事。他鬚髮皆豎。目光如炬。大怒。說必要去拿他的奸弊。弟也只說他是怒激之言。誰知他昨日果然到了那裡。直入樓下。正有幾頂轎子在門外樓下。還坐着幾個僕婦管家。家表兄問他們誰家的宅眷。家人說是阮圓海的令夫人。因他長子亡故。哭兒。得了個心疼的疾患。醫藥無效。故此來求他療治。家表兄聽了。竟往樓梯直上。衆家人要阻擋時。兄想。他那樣個膂力如虎的人。可是攔得住的。兩下一分。衆人都跌跌倒倒。被他上去。推了推門。是上面蓋下閂着的。被他輕輕一下。閂斷門開。走了上去。這個妖道正在淫那個阮夫人\footnote{毛氏樂哉。未免自恨不是醒着。}。把手鈴拴在褲帶上。放在股後。一抽一動的。所以那鈴不住的響。兩個道婆在一邊坐着。大約是看着難過得很。閉着眼。咬着牙。哼哼的念佛\footnote{咬着牙念佛。趣。}。被家表兄上前一拳。把那妖道打倒拿住。看那阮夫人時。昏迷不醒。家表兄問他緣故。他不肯實說。被家表兄將他十指叉起。用力一捏。比拶子還利害。骨頭都捏癟了。他忍受不得。方說一到樓上。他有一種迷人的咒語。念了便不知人事。任意奸淫\footnote{毛氏似多此一咒。醒時未必不樂從。}。事完了。用水噴面纔得醒轉。方悟到這些婦人旣被汚了。是自己尋出來的事。回去向丈夫說不出口。只好忍在心頭。有些貞性的吃了這遭啞苦。不肯再去了。那無恥淫賤之婦。所以源源而來。家表兄叫了阮家僕婦上樓。把他主母噴醒。那阮夫人也自覺慚愧。忙穿了衣褲。又叫他跟來的男人叫了地方總甲多人。將兩個道婆也拿了。同到縣衙去。阮家的人也去了。家表兄到了縣裡。把這些詳細備呈。縣公想的也是。他說這一申報了上臺。題請這妖道一剮是不用說。這些通謀的道婆約有數十。誅之不可勝誅。且這個名聲一張開了。叫這些去過的婦人何處生活。況內中還有大人家內眷。關係非小。丈夫要存臉面。自然要逼死婦人\footnote{阮大鋮便不然。}。恐傷得人多。未免有損陰隲。且上司知道。他是地方官。失於稽察。也有老大不是。這倒也是良心話。他將這妖道責了四十板收監。吩咐豪子夜間取了氣絕。兩個道婆也不深究。每人一拶十五板逐出。着實獎譽了家表兄幾句出來。昨日下午就有人知道家表兄這一番識見義舉。要來拜望他。他是不沽名的人。今早就回江北回(去)了。弟纔送他去來。順路到此。一來望兄。二來奉吿這件異事。這些愚婦人專信邪魔外道。自取其辱。也不爲過。但他家丈夫是做何事的。如匹夫匹婦。愚闇無知。尚不足責。至於詩禮門楣。簪纓世族。即如阮圓海先生。也是科甲門第。任着婦女胡行。豈不可笑。更見世風婾薄。千奇百怪的事無所不有。鍾生點頭嘆道。縣父母這一慮固是。但便宜了這個妖人。這也是他投鼠忌器之意。倒也罷了。所可惜者。令表兄這樣一位當世的英雄。弟竟不得一謀面。眞是當面錯過。梅生道。兄旣要會家表兄。此後他若有事到城來。弟同來一晤。說罷。起身別去。再說毛氏在妖道處出了這一番醜。到家諄諄囑咐衆男婦不要傳出。俗語說。瓶口紮得住。人口如何紮得住。不幾日。傳得合家皆知。阮大鋮也微有所聞。因他正同郟氏打得火熱。自己不正。如何還管妻子。不但不敢說。且毛氏也是他備而不用之物。裝聾作啞罷了。要看後事如何。下回便知分曉。

姑妄言第十四卷終



\endnotetext[1]{「不驕」原作「不嬌」,據第二回及下文改。}

\endnotetext[2]{「兒」字原無,據文義加;下文或同,不贅。}

\endnotetext[3]{「有」原作「布」,據文義改;下文或同,不贅。}

\endnotetext[4]{「捋」原作「將」,據文義改。下同。}

\endnotetext[5]{「兩邊」原作「邊兩」,據文義改。}

\endnotetext[6]{「備」原作「鞴」,據文義改。下同。}

\endnotetext[7]{此句原書右有夾批「必定」二字。}

\endnotetext[8]{「盡力」原作「力盡」,據文義改。}

\endnotetext[9]{「苟氏」原作「狗氏」,據第七回改;下文或同,不贅。}

\endnotetext[10]{「牛質」原作「牛贊」,據上文改;下文或同,不贅。}

\endnotetext[11]{以下有錯簡。自「奇姐身上倒不應付」至「奇姐笑道在我」爲二葉,原在後;自「就跨到他身上」至「說道我們今」爲一葉,原在前,書眉註明「此頁在後方二頁」。今據文義改正。}

\endnotetext[12]{「挨」原作「換」,據文義改;下文或同,不贅。}

\endnotetext[13]{「鮑信之」原作「鮑姓之」,據第七回改。}

\setcounter{footnote}{0}

\theendnotes

\part*{姑妄言第十五卷}
\addcontentsline{toc}{part}{姑妄言第十五卷}
\markboth{姑妄言第十五卷}{姑妄言第十五卷}

鈍翁曰。放下屠刀。立地便可成佛。人能改過遷善。孰不可爲聖賢。況宦蕚之惡。不過一片呆公子氣習未除。心性暴戾。賈文物不過欺世盜名。童自大不過鄙吝刻嗇。雖皆爲造物所忌。然其罪未至於殺人淫人。天良尚未泯滅。一朝悔悟。便能出人頭地。非異事也。所可異者。鄔合以篾爲生者也。自他三人改過後。而鄔合諛亦減於往昔。爲可異。然亦無足異也。如裴矩爲隋家之侒(佞)臣。而後爲唐室之良臣。顧其主爲何如耳。

富氏蓄怒一段。寫得層層次次。自一二分而積至十分。眞是生花之舌。令人絕倒。

寫賈文物之病。因要引出鮑信之含香。引出鮑信之含香。又好引出道士。引出道士。賈文物方得受藥以服富氏。服了富氏。然後將金銀珠玉一齊合攏來。不然。賈文物怕到何時是了。這四婢年俱二十以外。終留爲老婢乎。抑遣而去之乎。且不因此。含香不能使出。含香不出。後來何以親密。委委曲曲。算到賈文物一病。眞入神妙之想。

峨嵋山人去得乾淨。此處寫他者。爲傳藥與賈文物耳。藥已傳了。倘又遇着。剌剌不休。便成贅文。

道士雲遊天下。找(早)結第一回內。彼云要往四處雲遊。不如此寫。要說他這些年坐在何處修行。再講他如何靜養。如何學道。便是呆筆。

寫裘氏同衆妾叫僕婦們說白話。長舌婦講笑話。見得一夥淫婦人相聚。無聊之極。思牛親哥之創造。二婢之搶奪。裘氏取來入己。又轉贈菊姐醫病。總是寫諸婦之淫濫不堪。皆不過好此而已。

寫裘氏和尚之死。道士遁跡他往。總是要結衆人。不然。將紙筆只管拖長了。

姚予民之遣嫁衆婢妾。不是單說他的好處。也是隨手收拾衆人。不然。作何結局。

道士重訪到聽黑姑子。雖有物是人非之感。總是要始終照應。一筆不肯漏處。

賈文物歸家時。隨筆帶出富新一段。後來再說他的事。見得先曾有此人。不是臨時強扭來湊合。

寫和尚道士宣淫於姚宅。雖說僧道之壞。却是傍筆。巧極。力寫衆婦人不堪處。正是寫姚澤民父子不堪處。更是寫姚廣孝之不堪處也。

第一卷開首所出三人。到此回內。到聽已死。道士一去不復再見。只一黑姑子矣。

\chapter*{姑妄言卷之十五\\
第十五回 惡少改非 仙方療妒\\
附 萬緣和尚仗雄陽力竭取救兵 峨嵋道人逞異術興足多淫女}
\addcontentsline{toc}{chapter}{第十五回 惡少改非 仙方療妒}
\markboth{第十五回 惡少改非 仙方療妒}{第十五回 惡少改非 仙方療妒}

話說宦蕚自那日在錢貴家正然作惡。得了他父親的密信。一驚。跑了出門。在途中就同他衆人作別。獨自歸家。忙叫家人把大門關上\footnote{活是呆公子。若有禍臨。關上大門便躱得過乎。}。心中惶懼之極。茶也不吃。飯也不吃。在家中走來走去。因想道。我向來只說魏上公是長遠在的。我故倚勢橫行。到處指名唬嚇。說魏上公是我的家祖。誰知有今日這番事。但人惱我的多。倘一時有人混說是他的孫子。這却怎處。想到此處。坐臥不安。侯氏見他如此。疑他有甚麼外遇\footnote{這是婦人疑丈夫的第一件事。}。再三詰問。他悄悄將始末吿知\footnote{何必悄悄。豈怕丫頭輩聞之乎。昔有一鄕人。在田中回家。其妻曰。你的鋤頭呢。答曰。我忘在地裡了。妻曰。你悄聲的。恐人聽見拿去。快去取來。其人去了一會。空手而回。妻曰。鋤頭呢。彼悄悄的道。不見了。宦蕚同此。}。侯氏也吃了二(一)驚。吩咐家人不許在外面胡走生事\footnote{這還有理。}。到次日。忽見那多嗣來說道。小的纔在門首看見迎新舉人。昨日錢家那小秀才也在內中\footnote{宦蕚中囗豈止多嗣一人。前次侯氏問扇子乃是多嗣。此處見迎新舉人又是多嗣。此何故。因起初說錢貴之名是他。故此但是錢貴之事便是他。以見是他多事說起。方多出這番爭鋒的事耳。}。宦蕚聽了。又吃了一驚。道。昨日在家好好的吃酒賞花罷了。又訪甚麼錢貴。爭鋒打鬧。弄出這番事來。他這一中了舉。若懷恨在心。他是同鄕同里的人。我家的事都是知道的。若對了他座師房師同年混說起我的根底來。如之奈何。越想越急。因叫家人悄悄的將賈童鄔三人請了來商議。不多時。都到了。坐下。童自大道。昨日一團高興去訪他。不留我們這樣有錢的老爺\footnote{是極。有錢已設(該)敬。況是有錢的老爺。更該敬。此語譏刺不小。}。倒留那個窮酸\footnote{人一窮便覺〈便〉得酸。可嘆。要知窮酸勝富臭也。}。正打得興頭。我纔燥脾。哥爲何跑了回來。宦蕚道。還說打呢。如今打出事來了。你們可知道昨日那小學生竟中了。我家人看見今日在門口迎過去。因向賈文物道。三弟沒有昨日那把柄還罷了。你我都是八千女鬼的那把刀。他一時記恨。混說起來。怎處。賈文物道。君子不爲已甚。兄昨亦過甚矣。我兩人有終身之憂。尚何言乎。即三弟亦不能辭其責也。昨日浸潤之譖。膚受之愬。皆三弟爲之。彼豈不在心乎。且三弟足之蹈之。手之舞之而罵焉。我看他其人之品淸矣。必小有才。倘明歲會場中言必有中。後生亦可畏也。童自大聽了。幾乎掉下眼淚來。說道。我雖是個財主老爺。終日縮頭在家\footnote{財主老爺會縮頭。罵得惡而趣。}。守着幾個錢。連樹葉兒掉下來還怕打破了頭。從不敢得罪人的。昨日仗大哥的威勢。故罵他幾句。學樣兒\footnote{可見世人皆有一點天良。都是學壞了。}。誰知就弄出事來。原來人是欺不得的\footnote{此語悔心之明。}。我想來。我比不得二位哥的勢。要我去替他叩頭賠個禮。或者他也罷了。不然。他後來果有造化。做起官來。懷恨在心。茄子揀軟的掐。我這個家私就有些保不住了\footnote{世上肯顧惜身家者。便是好人。}。賈文物道。三弟之言。不太卑乎。當從容議之可耳。宦蕚道。我倒想了一個道理。叫老鄔去訪一訪他。姓甚名何。在何處居住。我們且聽着。他若有話講。我們再做道理對付。他若總不計較。也還是個好人。雖然窮些。我們相與他。也還不錯。再煩人去對他說。我們向日不認得。得罪了他。如今要給他賠禮。同他做朋友。他自然也肯。童自大道。哥好算計。他若是這樣好人。我還要送他一分短八厘的一分厚禮\footnote{的是江南口頭語。}。賈文物道。善哉言乎。但使乎使乎之任。孰能當之。宦蕚道。昨日老鄔在那裡勸鬧的。改日若去。除非是他。鄔合道。這事晚生當效勞。他大家因有心事。也無有興頭吃。各自散去。次日。鄔合來對宦蕚道。晚生去訪了來了。此人姓鍾名情。中在第六名上。他房師座師見他靑年飽學。甚是得意。他家在鳳凰臺住。宦蕚道。看他不出。年小小的。倒中得高呢。你可再去暗暗打聽那話。鄔合去了。過了些日子。又來說道。晚生日日打聽。並無話說。倒打聽了一件新聞。這鍾舉人他叔叔送了他一處大房子。已搬了過去。竟將錢貴娶去做妻子了。宦蕚聽了。又驚又喜。喜的是不見他有甚話說。庶可放心。驚的是說他一個新舉人。如何娶個瞎妓爲妻。更恐錢貴懷恨。挑唆鍾生同他爲難。說道。這些話你打聽得實確麼。鄔合道。晚生有一個相識。新投在他家當長班。都討的他口裡實話。宦蕚這纔信了。又過了幾日。總無動靜。宦蕚約了賈童鄔來。說道。那人毫無話說。我們前日之議該行了。賈童亦無異辭。因對鄔合道。我備一分厚禮。煩你明日去對他說。要把我們的意思說得妙方好。鄔合道。晚生雖愚鈍。決不敢辱三位老爺之命。宦蕚連日來見事情稍冷。心中又放下了些。就留他們小飮了一回方散。宦蕚到了房中打點禮物。侯氏道。你拿禮送誰。宦蕚不敢說爲爭鋒打鬧賠禮的話。只答道。我有個姓鍾的朋友。新中了舉人。打點賀禮送他。侯氏道。我從不曾聽見你有個姓鍾的朋友到我們家來。宦蕚道。這人曾在賈家會過。纔得二十來歲。生得標致非常。滿肚才學。只關門在家讀書。容易不肯出門。所以不曾到過我家。侯氏道。是怎麼樣個人。就生得這等標致。幾時他來。等我張他一張。又道。這樣男子。不知誰家有福的女兒嫁他。宦蕚失口道。就是前次所說要接來唱與你聽的那個瞎姑。他娶了去了。侯氏驚問道。這瞎姑姓甚麼。怎有這樣造化。他一個新舉人。又怎肯娶他。你必定知道。宦蕚不留神。將要說出錢貴。猛想起前番扇子的話\footnote{照應得到。}。忙改口道。倒不知他的姓。只聽得他與鍾舉人是相知的。所以一中了就娶了他去。侯氏暗想道。這鍾舉人如此美貌。又這樣多情。我一個千金小姐。反不如這瞎姑命好。若嫁了這樣丈夫。也不枉爲人一世。長吁了一口氣。道。這鍾舉人眞是好人。他與這瞎姑不過是露水夫妻。就這樣的恩情不捨。我同你夫妻多年。你全是假意待我。宦蕚道。我是千眞萬眞。可敢攙一毫假。侯氏道。你若有恩愛眞心到我。如何時常躱懶\footnote{不躱懶者便是眞心恩愛。婦人之見大率類此。}。自從我好意把丫頭與你。我見你凡做事時。倒留一半心在他身上。宦蕚見他說到此處。針着了心病。忙答道。我那裡有這個心。這是你猜疑的。你要我不躱懶。凡事肯依我麼。侯氏道。我便依你。看你怎樣不懶。宦蕚見左右沒人。忙掩上房門。笑嘻嘻上前抱住。親了個嘴。就替他脫褲。侯氏先聽說鍾生標致多情。往他身上想。動火已久。任他脫去。也不上床。就在椅子上架起兩條腿來。做了一齣懶漢推車\footnote{這一齣就借懶字生發。}。他二人從不曾白晝交鋒。這是初次。覺得比被窩中十分親切。騷興大發。一場狠弄。那侯氏陰中如狗舔糨糊一般聲響。極力抽提。方纔興過。牝中淫水流得地板上濕了好大一塊。拭抹穿衣。不在言表。却說鍾生在家讀書。還是做秀才光景。總不出門。一日。忽見鍾用來說道。外面有個姓鄔的來拜相公。將名帖遞上。鍾生看時。上寫着晚生鄔合拜。鍾生想道。我相識中並沒個姓鄔的。他來拜我何事。因道。你回他罷。鍾用道。小的回他的。說家主閉戶讀書。槪不會客。他說定要求一面會。還有要緊話說。我纔來稟。鍾生道。旣如此。請他進來。那鍾用去了。鍾生也就迎了出來。只見鄔合已走進門內。後面兩個人掇着兩個大篾絲緞盒。鍾生拱讓進廳。鄔合曲腰足恭。其態甚謙。他一到廳上。便深深一揖。道。晚生驚動老先生。得罪得罪。鍾生讓他坐下。說道。小弟寤寐平生。未曾相識。何敢承鄔兄過謙乃爾。鄔合打一恭。道。晚生那日同宦公子在老夫人府上曾識荆的。鍾生細把他一看。方記起那日在錢家。在中間勸鬧是他。因向他舉手道。向日承兄解紛。小弟與拙荆不致十分狼狽。深感深感。但今日承兄賜顧。有何見敎。鄔合又深深一恭。道。不敢。晚生向來在宦府走動。不意那一日宦公子開罪於老先生。同他在那裡的二位。一位是賈進士先生諱文物的。一位是童援納先生諱自大的。皆因不識老先生。故爾冒犯。後來知道了。甚是不安。今他三位要來荆請。不敢造次唐突。特命晚生先來奉聞。兼備了些微薄禮。稍致一芹之敬。望老先生莞納。遂在一個家人手中取禮單來遞過。鍾生也不來接。說道。尊帖請收回。那日之事。小弟之過居多。與他三位何涉。小弟全不介意。承他不苛刻追求。就荷愛多矣。何敢當荆請二字。小弟與他諸公雖住一城。所謂風馬牛不相及。怎敢當此隆禮。至於說要來賜顧。一來小弟要閉戶讀書。從來不會一客。二來小弟雖然徼倖。還是一個貧士。怎敢與他諸公交往。煩鄔兄婉復。鄔合道。宦公子三位因慕老先生大名。故要敬來奉拜。老先生何拒絕太甚。鍾生道。鄔兄言重。弟何人斯。安敢拒絕於人。特不敢當耳。就來賜顧。小弟也不敢會。倒是客日小弟無事。先去奉拜則可。望鄔兄轉致他諸公。說厚情心領。鄔合見他苦苦推辭。只得別了回去。鍾生送他出門之後。回到內中。笑對錢貴道。適纔宦公子托了一個姓鄔的會我。就是當日在你家勸鬧的那個人。說向來不知得罪。今要來賠禮。又送我一分厚禮。我苦苦辭却去了。可謂前倨而後恭矣。錢貴道。此等小人。君不可拒絕太甚。恐狂奴舊態復萌。又生枝葉。鍾生道。他旣知如此修飾。大約非昔日咆哮舉動矣。錢貴道。他也是恐居(君)不能去懷。故來結交耳。鍾生道。此雖容或有之。也是他一番美意。不可滅他美情。說罷。往前邊去了。且說鄔合回到宦家。他三人正在等回信。一見他來。便問道。所說何如了。鄔合道。晚生將三位老爺的意思細述了一道。他再三遜謝。說向日是他得罪了衆老爺的。與衆位何干。決不敢當此厚禮。也萬不敢當衆位老爺去拜。他要讀書。就去也不敢會。倒是他閒了先來奉拜則可。不敢勞先施。宦蕚道。他的樣子像還不能忘情麼。鄔合道。據晚生看起來。他眞個絕頂的好人。謙和至極。說的話都是眞心眞意。連待晚生的那一種禮貌也謙虛得了不得。一毫狂妄的氣兒也沒有。宦蕚沈吟了一會。對衆人道。世上有如此好人。人辱了他。他還說是他得罪了人。我每常凌辱了人。還說是人觸犯了我。這樣比並起來。豈不自愧。我想時勢也有盡了的日子。何不做個好人。只管作惡何益。況如今魏上公已完。泰山已倒。我家的勢漸漸差了些。況且人生可有長生不老的。我家老父百年之後。這些豪勢豈不冰消瓦解。我只顧目前作惡。倘後來遇了我這樣有錢有勢。比我還惡的惡人。得罪了他。就未必肯像鍾舉人這樣包容了。那時起(豈)不弄出天大的是非。我從今後決不做宦惡了。因吩咐衆家人道。你們自今以後再不許生事。都要改過遷善。若再以當日倚我的宦勢與外人作惡。我就要在家與你們作惡了。可闔家傳諭。衆家人領命應諾。童自大接着說道。哥這想頭主意是極。我想我家有百十萬銀子了。見人送我一個錢。我就喜歡出屁來。恨不得連人的手都接着。我要用一個錢。比抽一條筋還疼。就像殺我的命一般。如今老鍾一個窮舉人。見送這樣厚禮。是落得收的。要叫我。就像冷手抓着熱饅頭。死也不放了。他還不肯受。可見銀子錢也有該要也有不該要的。況且人不能活一百歲。一死了。一文也拿不去。仍舊撂下。我何苦這樣刻薄臭吝。被人指指戳戳。臭呀臭的笑罵。且是天道最忌滿盈。我的財也算多了。再不學好。倘被那紅鬍子姓火的老爹請我去搖起會來。豈不弄個乾乾淨淨。我如今也看破些罷。此後也不銅臭了。至今我的老爺是個紙老虎。原是個假的。只好嚇小孩子同鄕下人。二位哥使勢還有一說。我怎麼仗別人的勢。狐假虎威。鑽在人腰裡硬起來\footnote{世上鑽在人腰裡硬的人甚多。}。幫扶作惡。倘撞着吃生米的。與我做起對來。只怕這家私性命就有些不穩。我從今後也不自大了。只隨高逐低。縮頭藏頭。安分守己。在家受用罷\footnote{保身秘訣。千古來多少聰明乖巧人不能及此。不意被這臭呆悟透。}。賈文物也嘆了一口氣。道。我想我不過是仗着孔方兄之厚。借着富泰山之力。夤緣了一個舉人進士。就以爲遍江南獨我尊。便不曾回想天下之舉人進士。車載斗量。而且眞才實料的亦自不少。不知有多少科甲大老先生都謙謙自遜。我假文的是甚麼。從今再不假文欺物了。如鍾舉人一個眞才子。尚在家閉戶讀書。我一個假進士狂到那裡。今後也去學做些正經事吧。因對宦童二位說道。我們彼此大家做些好事。聖人云。旣往不究。又云。過則勿憚改。當痛悔前非。留個好名。有何不妙。況我三人皆無子嗣。積些善行。倘然得個兒子嗣續。不斬祖宗。保得血食。也可免不孝之罪。何苦胡做非爲。與人唾罵。與自己有何益處。空爲人做千秋笑話。宦蕚童自大道。此言甚是有理。三人遂焚香設誓。自今悔過自新。若再蹈前非。人神共殛。此後三人竟大變起來。宦蕚一絲也不倚宦作惡了。童自大也不刻薄銅臭了。賈文物也不假借一毫之文以欺人物了。合城賢愚見他三個絕頂的壞人忽然自己都改變了。皆轟傳以爲異事。人雖有恨他們的。見他如此改過。前憾也都釋然。故他三人得無後患。單說賈文物別了回家。深悔往非。坐在轎中不住嘆息。到了家。進房中來。見富氏同他的一個族間姪兒正在好好的說話。一見了賈文物。忽然就把臉放了下來。你道富氏的姪兒到他家來何事。他姓富名新。他父親雖是個飽學老儒。却是一個學霸。各樣便宜的事他無不會占。奈時運淹蹇。被這一領靑衿困了他一生。到老還是個精窮的措大\footnote{此正是學霸的報應。見得壞人終無結果。}。他係富戶部遠房姪兒。這富新纔十三歲。生得面容嬌媚。宛如一個美女。性極聰慧。得他父親的家傳。讀了滿腹時(詩)文。不幸昨日他父親病故。家無一文。他母親是個沒脚蟹。無門可吿。眞是苦惱。古語兩句道得好。叫做。

\begin{quotation}

上山擒虎易。開口吿人難。

\end{quotation}

他見丈夫的屍骸暴露。無棺可殮。千思百想。想起富氏來。他們雖係一家。向因貧富不敵。不大上門\footnote{令人傷心。此類富宦皆范文正公之罪人也。}。今沒奈何了。只得叫富新到姑娘家報喪吿助。富氏性雖潑悍。只待賈文物同家人嚴厲。他在外人倒還有點慈心。聽說哥哥沒了。沒有棺材。甚覺不忍。忙取了三十兩銀子付與富新。道\footnote{是個大家手段。不愧姓富。然而若是個富男子。或倒捨不得。}。你回去對母親說。將你父親的大事趕着料理要緊。隨後我再送些柴米來與你\footnote{此眞是雪中之炭。今日尚有此等人否。}。那富新千恩萬謝去了。賈文物坐着。尚嘆聲不已。富氏喪着臉問道。你往那裡撞屍遊魂去了一會。回來望着我嘆氣。做甚麼事。想是見我給姪兒銀子。花了家私麼。賈文物忙道。我豈敢爲此。因我當日年幼無知。倚仗着財勢。凡是可欺凌刻薄之事。無不踴躍爲之。後來同宦童結盟。大家又同惡相濟。況自從一第以來。假充文墨。欺世盜名。近日又欺辱了個姓鍾的寒士。誰知他竟一舉成名。我們要去賠禮。他再三謙遜說不敢當。況魏公今已伏法。泰山已化做冰山。或有不虞。身家性命所係。我三人今日設誓。痛改前非。嘆息之故。爲悔當日之無知耳。富氏聽了丈夫這番話。要是賢德婦人。自當慫慂獎譽一番才是。他反放下臉來。道。魏太監剮了。你這無用的忘八拿去殺了也不虧你。你這種沒用的東西。不若早死早超生。要你活在世上現世。你做這個賊樣。望着我短嘆長吁。要來魘樣我麼。賈文物一篇好話。本意也圖富氏誇他兩句。不想討出這種好贊語來。雖不敢怒。未免也有些艴然之色。便答道。因你下問。我纔敢上呈。並無一字衝撞。何須動怒乃爾。富氏大怒道。好大膽。我跟前也許你回嘴麼。你把屄臉彈子放下來。我難道怕你不成。跳起身來。伸手要來拿他。嚇得賈文物往外就跑。恐怕衣服長絆倒了被他拿住。兩手拽起前衿來摟着。如飛而去。你道這富氏與賈文物夫妻也十〈分〉多年了。越發性子潑悍到這個地位。連好話都容不得一句。是何緣故。他當了(日)在家做女兒時。因尊性猖獗。合郡馳名。人皆不肯求此溫柔佳配。等到二十多歲。雖不知男子的味道如何。情竇已開久了。那一種願爲有家的心腸。時刻在念。況他自幼無母。他父親跟前這些妾婢們。肆無忌憚。說頑說笑。村言淫語。何所不出於口。皆以爲姑娘年小。尚無知識。可以不必防他。孰不知他年紀雖小。耳朶是有的。且人在幼年時聽的話。就是終身也不能忘記。及至年紀大了些。想起那些話來。他們說得這樣津津有味。裙帶之下個中定有佳境。不想只管磋跎住了。倒合了古詞二句。道。

\begin{quotation}

欄杆十二。哥(倚)遍又還重倚。二十八宿。手中輪數不到。星張翼軫。

\end{quotation}

他心中雖然暗急。沒有個在家的閨女好向父親說我年紀大了。摽梅期過。想要女婿之理。只好隱之而已。他暗地又自思自解道。假如十四五歲嫁了人去。不過也是十四五歲的男子。一個乳臭小兒。吃飯尚不知飢飽的時候。料也無濟於事。我今已若許的靑春。定然佳婿的芳年不過彷彿上下。那二十外的小後生。正是人強馬壯之秋。只要多用些工夫。也可補前之不逮。不意嫁到賈家來。一見了賈文物。還是個小孩子。自己若再大得幾歲。竟可以做他的阿母。與前在家的算計。一絲也不合。你叫他着急不着急。不由得那一腔怒氣發動了一二分。只得權且按住。晚夕成親。那賈文物雖只十三歲。他曾領敎過此道。也還知親親熱熱。爬爬弄弄。竟像個子母懷中抱着個耍娃娃在那裡戲弄。幸得他生性好此。每夜定要動作一番纔罷。富氏雖然年大。還是一朶鮮花。未曾經過風雨。並不知如何是個丢。怎麼叫做樂。只似乎有個〖礻聖〗(蟶)乾大的東西。在牝中動動扯扯。微微也有些癢癢酥酥的。覺得比在家做女兒成年空閒着他到底差強。過了些時。就不能像起初慇懃了。但這賈文物他是個老來子。未免生得單弱。又且是十三歲的孩童。就鬼弄這些把戲。他也只盡自己之興而已。並不知此道中婦人也有妙境。他一個血氣未定的人。把這品鹹蚌肉吃傷了些。未免臉黃瘦了\footnote{見此四字。想起一笑話。一龍陽娶妻。日漸肌瘦。一人贈之詩曰。個個人兒忒殺矬。看看臉上肉無多。算來家公眞難做。不如依舊做家婆。}。咳咳〖口敕〗〖口敕〗。懨懨無力的樣子。不但他心有餘而力不足。他的母親見他這個形狀。疼兒心重。又見媳婦忒大了。先媒人瞞着。只說大四五歲。後來方知大了兩個五歲還有零。恐怕把兒子當起家常茶飯來。日日不離口。如何了得。心中急了。只得背地勸兒子。這件異品只可當果子。偶然吃些。不可當飯吃的。過飽了定要傷人。諄諄囑咐。那知賈文物也正在要吿免催徵的時候。恰又遇有母命。焉敢不遵。就一曝十寒起來。那富氏未免又增了二三分的怒氣。雖然含怒胸中。怎好說夜來不勤謹的打鬧一番。戒他的下次。只得含忍。待時而動。後來見他調戲丫頭這番舉動。怒有四五分的地位。暗想。必須拿住他眞贓實犯。纔好施威。洩洩怒氣。故吩咐丫頭們設計誘他。不想賈文物還像個夢井落在他的圈套中。捱了那兩次肥打。雖然鬱怒覺得稍舒。却被婆婆絮聒了兩番。終是未曾洩得。後來又聽說他與婆婆的丫頭。不但是新偷。竟還是敍舊。一枝嫩笋反被丫頭先奪去頭籌。那六七分的怒氣。火騰騰的攻將上來。那裡還忍耐得住。所以那日一見了含香。就如焠燈上的硫黃。見火就灼起來。故此有那一番大鬧。尋死覓活。次日聽得老子來。只道來替他出氣。誰知反是來敎訓他的。一個肚子幾乎蠱脹起來。後來喜得賈文物領過這兩次辣麪。知道這女諸葛的智謀利害。已經過二擒二打。若到了七擒上。就未必肯如那慈悲的軍師。還肯七縱蠻王的性命。富氏有六七分的恨怒。賈文物也就有六七分的膽怯。拱手服降。俯伏在地。夫人天威。男人不復再敢矣。倒也太太平平過了兩年。賈文物雖然生得身材瘦怯。也長成大人的規模。不似先小孩子的行徑了。他身子旣長大。那厥物自然也就大些。比得上沒疙瘩的海參。較那蟶乾又壯觀了許多。他又歷練了些。每於床幃之中。也就比先在行。富氏方知這件海味果然美口。只是賈文物連身子都被他降服了。何況那腰中之物。到了交合之際。不由得轅門拜倒。十度盤桓倒有六七次掃興。富氏雖然心恨。自己破開一步想。雖不能適口充腸。又強如當日食而不知其味的時候。那怒氣雖不曾添上一分。他舊日蓄在胸中的也不曾消釋半點。富氏正想再激勵他一番。或者有奮勇之時。不想被那不知疼癢的父親。把個纔知竅的女婿又叫往京中去了。好不難過。及聞他中了進士。以爲他這一回來家。離了半年有餘。不但於此道中或者長了些學問。他今日得了功名。身子旣然發達。或連身邊的那件物事也發達些。亦未可知。終日在家潔具粗(淨)牝。恭候早光的等候。誰想公公沒了。丈夫回來開喪出殯。家事紛紜。又接着婆婆病故。又忙亂了多日。此時賈文物方自己當起家來。百事俱要自己操心。雖也常與富氏點綴點綴。不過應卯而已。也無心情只管去鞠躬盡瘁。富氏此時又添有一二分的怒氣。與前那六七分合併在一處。足足的竟有八九分的局面。後來父親亡逝。又忙過了些日子。纔完了喪事。後兩家合爲一家。家業越大。身子越忙。況且中了進士的人。勢利中又多有一番應酬。他名字叫做賈文物。如今又學起假斯文來。一舉一動無不文文縐縐。後來演習慣了。雖到夫妻交合之時。那富氏急得要死要活的時節。他也還是這等彬彬儒雅。不由他不怒目切齒。富氏此時三十多歲的壯婦。正是慾火蒸炎的時候。俗語說。婦人三十四五。站着陰門吸風。蹲着牝戶吸土。可是看得這般舉動的。把怒氣整整積到十分。別的怒氣\endnotemark[1]向人訴說訴說。也可消去些須。這一種氣。雖父母兄弟之前。亦難出之於口。況左右不過是些婢婦。向誰說得。只好自己鬱在胸中。因其人而蓄者。即以其人而洩之。所以一見了面。輕則罵而重則打。從無好氣。就是他獨自坐着。丫頭們見他面上。即如當日褒姒一般。從不曾見他一點笑容。那賈文物雖怕到十分。却不敢避他。日間推故躱在外邊。每晚必定同床伴宿。自己也知這假斯文不好。惹他憎惡。但習以成病。欲改不能。如今雖不敢望其垂愛動憐。可還敢離開了。添他的怒氣。天地間的事。譬如疼愛那個人。雖有天大的不是。不拘怎樣。都待諒得過。如惱怒那個人。雖百般都是。還要在那是中尋出不是來纔罷。俗語說得好。在雞蛋中還要尋出骨頭來。就是此謂。今日賈文物一番好話。他不但四馬了。而且還要才丁。賈文物到了這個性命干係的時候。假斯文不得了。只得認眞的一跑。跑到書房中。着了一嚇。又忍了一口氣在胸中。倒在一條椿凳上。不覺沈沈睡去。此時深秋天氣。金風颯颯。寒氣侵肌。一覺醒來。已經日暮。覺得頭痛眼花。胸腹悶脹。身熱如火。口內呻吟。不能動履。衆家人見主人有病。問着不答。忙擡到床上臥下。蓋上了被。如飛去稟知富氏。富氏餘怒未息。罵道。那裡就得死。你們見神見鬼。輕狂的是甚麼。憑他睡在那裡。不必來向我說。家人不敢多言。諾諾而出。富氏毫不在心。夜間衆家人守着。見主人沈沈昏睡。十分着急。到次日。大家商議。主母旣不管閒事。我們請個醫生來看看方好。內中一個老家人道。使不得。老爺病勢來得甚重。奶奶不做主。我們知道請誰好。醫好了呢。是造化。倘有一差二誤。干係誰人擔得。衆人俱道。有理。正在躊躇。忽門上賈閽進來。道。鮑信之來看老爺。叫我進來說聲。衆人聽得他來。甚喜。道。來得好。他認識的人多。同他商量商量再處。你快去請他進來。你道鮑信之爲何認得賈文物。到他家來。他娶的妻子就是賈文物自幼相知的那個含香。他原有百金本錢。就在富戶部左近住。門口開個錢鋪。爲人又老實又和氣。富家使錢都往他鋪中兌換。這些家人都相認識。日久熟了。値富戶部命家人尋個好人家。一文不要。打發這丫頭。衆人知他無妻。舉薦了他。遂將含香嫁了與他爲室。他見一文不費。不但得了個好老婆。又還蒙富戶部賠了那女子許多器皿衣飾之類。感恩不盡。料道富戶部可稀罕他的酬報。因係衆家人的總成。他也甚是知情。衆人但到他家中來。非茶即酒。相待得十分契厚。衆人見他如此親熱。竟認做親戚往來。及至富戶部故後。這些家人都歸到賈家來。衆人念他情長。舉薦到門下。做個換錢的主顧。賈文物也知道含香在他家。念其婦而及其夫。甚照顧他。見他本錢短少。應付不來。借與他五百銀子。只要一分利息。借這點恩私。以報含香當日的情義。這也是賈文物的一點好處。他添了這些本錢。又搭上賣米。鋪子大了。就興旺起來。大有所獲。夫妻感他不盡。時常尋些好東西來孝敬。這日因打門口過。聽得賈文物有病。要進來問候。衆人忙接了他進來。就把要請醫生的話同他商議。他道。我且看了老爺着。走到床前。帖(恰)好賈文物醒〈來〉轉來。他忙上前問道。老爺尊體是怎麼樣。門下特來請安。賈文物讓他坐下。道。我昨日在宦家吃了些飮食回來。在椿凳上睡了一覺。着了涼了。身子沈得很。甚不好過。鮑信之道。還得延醫用服藥。發表發表纔好。賈文物道。我不過是感冒了。又沒甚大病。吃那藥做甚麼。況目前的醫生。可有一個好的。好人醫死的多。病人醫好的少\footnote{我以爲目今如是。不意當年已是如此。有一笑話。一醫生搬家。辭衆街鄰時。各送藥一服作別敬。衆人云。我們沒病。要藥做甚事。醫云。你吃了我的藥。自然就會害病。}。倒不如捱兩日。自然就好了。鮑信之道。老爺千金之軀。可是輕易得的捱的。懨纏日久。怎麼了得。本地的醫生。門下也不敢舉薦。近日洞神宮。四川來了個老道。自稱峨嵋山人。在那裡賣藥。不論疑難雜症。多年宿疾。一服就癒。貧不計利。治好了許多人。合城都是知道的。請了他來看看罷。賈文物道。那些走方賣檔。都是騙人的太歲。他知道甚麼。請他何益。鮑信之道。也一例論不得。這個道人。門下眼見他治好了許多人。請他來看看。診了脈。若說透病源。便服他的藥。若說不着。只丢得幾錢銀子。是有限的。只當是請了來說評話。替老爺解悶。賈文物見他說得有理。依了。就托他去請。他道。這老道古怪着呢。他不甚肯到人家去。他自己說。要有緣的呢。不請也去。無緣的呢。請也不去。果然有那大官府財主慕名去請他兩次三番。他決不肯去。有那貧窮的人不敢請他。說了病來求藥。他忽自己要去。人也不知他是甚麼緣故。老爺旣請他。須發個名帖。打發一位管家爺們。門下同了去請。賈文物叫了個家人。拿帖子同他去了。不多時。請了來了。鮑信之陪了進來。那老道向賈文物舉手道。居士。貧道不爲禮了。賈文物見他仙風道骨。鶴髮童顏。一部長髯如銀絲相似。長有尺餘。好一個仙姿道貌。

\begin{quotation}

布衣草履。昻藏無流俗之風。道貌長軀。磊落似神仙之品。蕭蕭幾莖華髮。望見藹然可親。落落一部蒼髯。行來肅然起敬。只知是今日施藥神醫。那識乃當年採陰道士。

\end{quotation}

賈文物忙道。賤軀有恙。不能奉迎。得罪了。讓他坐下。鮑信之陪着。茶罷。到床前來診了脈。完了復坐下。便道。尊恙乃飮食後感冒風寒。叫做內傷外感。可是麼。賈文物疑是鮑信之路上吿訴他的。也不答應。他又道。這回內傷。非止飮食。因着了驚嚇。又着了一口暗氣。如今是氣裹了食。在內中作禍。所以沈重。賈文物見他說着了病根。如同目睹。連連在枕上點頭道。不差不差。老道笑着道。貧道也略知風鑑。我觀尊相面上隱隱有些驚懼之容。又帶些忿怒之色。胸中有說不出的一種隱恨藏蓄久了。古云。冰厚三尺。非一朝一夕之寒。所以今日這一鬥着。就病得沈重了。賈文物這十多年的心事。無門可訴。鬱在胸中久了。今被他一語道破。便道。眞神仙眞神仙。遂問道。尊師看弟子的賤恙還不妨麼。老道道。這個浮病有何慮得。一服就管痊癒。居士心中之恙。古人說得好。心病還須心藥醫。等居士尊體健了。貧道再來商議救治。解開藥囊。取出一丸藥來。如龍眼大小。道\footnote{不知可是鍋巴丹。}。用薑湯調服。出微汗。不可太過。再行過一二次。明日即痊癒矣。起身作辭。賈文物道。恕不送了。那老道把手一舉。飄然而去。賈文物隨叫家人封一兩藥資趕了送去。鮑信之送了老道出門。復翻身進來。問道。這老道看得何如。賈文物道。眞是神醫。多謝你的盛情。薦了他來。鮑信之也謙謝了兩句。辭別而去。這賈文物多年的心病被他看透。覺得身子竟好了些。忙用薑湯服了藥。出了些微汗。午後又行了兩次。病勢已退。只是身子軟些。叫煮了些冬舂米粥。用小菜吃了一碗。睡了一夜。次日平復如舊。心中大喜。見那富氏毫不瞅睬。也不問一聲。如同陌路。心中恨道。人之無良。一至於此。十數載夫妻。毫無一點情意。想道。昨日老道許來替我治心病。看他定是個異人。倘有妙法。把妻子這個凶惡治好了。豈不是萬幸。但要求人。不可托大。須要盡一個禮。今日再養息一日。明日再講。還在書房中宿了。次早起來。吩咐家人備一桌豐盛蔬齋。寫了一個拜帖。一個請帖。親自坐轎去拜這道人。到了他寓處。他尚在屋內靜養。還不曾賣藥。他做定的例子。早飯後賣起。午飯後即收。他要做早晚工夫。賈文物問明了住處。也不用人傳說。就走了進去。那老道正趺坐着。見了。也就立起相迎。賈文物深深一揖到地。起來。親手遞上拜帖。道。昨承尊師下降。又蒙賜仙丹。使賤軀平復。特來拜謝。那老道道。昨日旣承厚儀。今日又勞光顧。深感了。相遜坐下。賈文物又親自送過請帖。道。寒舍備一餐蔬齋。要奉屈仙駕。不敢定日。或今日。或明日。聽憑尊便。老道道。貧道要說無事。每日賣藥濟人也是一件事。要說有事。我一個出家人。如閒雲野鶴。何日不可以高飛。可是羈絆得住的\footnote{近日僧道比在家人更有覊絆。成了檻猿囚鶴矣。}。只是怎麼好奉擾。賈文物又深深一恭。道。一餐便飯。猶恐褻尊。何足云擾。不過弟子欲親道範。以聆淸誨之意耳。倘蒙不棄。受愛多矣。那老道見他這樣殷殷誠懇。便立起道。居士請先回。貧道即刻便到。賈文物吩咐家人。快叫一乘轎子來。我同尊師同去。老道止住道。貧道兩隻忙履將歷遍四海。這幾步路又坐起轎來。賈文物道。弟子奉屈尊師。安敢自己乘輿尊師步履之理。老道再三不肯。只得道了罪。辭了出來。老道送到寓所門口。賈文物讓他進去。又一揖。道。耑候了。上轎回來。到廳院中。方纔下轎。賈閽跟進來。道。老道士來了。賈文物吃一驚。道。這老道果有些奇異。轎子走得如飛。家人們跑着還跟不上。他如何走得這等快。定然有些妙處。分外恭敬。忙忙的走出迎接。到書房坐下。老道舉手道。適纔有勞。賈文物道。豈敢。屈駕不敢耳。吃了茶。齋飯預備現成。就安了桌子。讓了坐。篩了一杯酒。執在手中。問道。尊師可用酒。老道道。也飮一杯。賈文物遂雙手將酒遞過。然後坐下相陪。蔬菜一碗碗送將上來。酒過數巡。老道〔道。〕不用了。送飯吃罷。撤開。又送過茶來。老道吃着茶。問道。承居士一番敬愛。無以相報。可將心中病根說來。商酌治之。以答盛情。賈文物見許多家人在傍。不便說得\footnote{此果自愧耶。或恐傳知富氏耶。}。那老道哈哈大笑。道。居士不過因閫政太嚴之故耳。此乃人之常情。何須隱諱。賈文物被他一句說得毛骨悚然。吩咐家人都迴避了。衆人出去之後。他出位深深一揖。道。尊師旣洞鑒弟子肺腑。可有療妒奇方。使弟子癒此心病。沒齒不忘大德。老道道。居士試道其詳。賈文物遂將他夫妻十餘年並無美言悅色。相見非打即罵。如同仇敵一般。更性情凶暴。家中奴婢稍有失意。凌虐不堪。弟子每每見之。不禁目慘心裂。開心見誠。細細相吿。復一揖。道。今日幸遇恩師。何以敎我。老道道。居士休怪。令政已犯七出了。何不棄之。賈文物道。賤荆雖不賢。乃先嚴慈所聘娶。且當日先岳愛我如子。況遺我許多厚產。故不忍休棄耳。老道笑道。居士非不忍。特不敢耳。賈文物聽了。紅了臉。答應不出。老道又道。居士可知婦人中這種悍妒的緣故麼。賈文物道。自然是天性使然。老道道。非也。人生自幼至老。其性不改。方謂之天性。居士請想。人家女子在閨中悍妒的可有麼。間有一兩個性凶粗暴者。乃父母失於敎訓之故耳。此孟夫子所謂。性相近也。習相遠也。豈天性使然耶。賈文物聽到這裡。將座兒挪近。促膝坐着。道。求尊師明以敎我。老道道。婦人未有悍而不妒。妒而未有不淫者。若果能遂他的淫心。那悍妒之氣自然就漸漸消磨下去。居士試想。任你萬分悍妒的婦人。他到了那枕蓆上心滿意足的時候。可還有絲毫悍妒之氣否。皆因不能飽其淫欲。使忿怒之氣積而成悍。陰性多疑。以爲男子之心移愛於他人。故在他身上情薄。此心一起。悍而又至於妒。婦人犯了淫妒二字。棄之爲上。旣不能棄。萬不得已而思其次。古云。治水當淸其源。只有把他的淫情遂了。他那悍妒就不知其然而然自化爲烏有矣。賈文物聽了。沈吟了半晌。道。尊師金諭。一絲不錯。但弟子不敢瞞尊師說。賤軀微弱。賤具亦甚鄙猥。力不及此。奈何。老道道。此非我出家人所知也。賈文物不覺跪下。道。尊師所見若神。若不救拔弟子。將來此軀就不知作何光景了。竟有個墮淚的樣子。老道扶起他來。道。承居士一番厚愛。此雖非我世外人所當管。但救居士的災難。化妒婦的凶心。也是慈悲一案。不得不如此了。然當愼之。他悍妒之氣一消就罷了。不可過用。倘有傷性命。不但貧道有大罪過。居士亦損陰德。說着。就取過藥囊。拿出個葫蘆。倒出兩粒大丸藥來。又將一個葫蘆倒出有綠豆大的七八丸來。包好。附耳傳了許多的妙訣。又道。但遵而行之。自當有驗。萬不可過。至囑至囑。賈文物滿心歡喜。接將過來。深深揖謝。道。蒙尊師大恩。弟子思自救耳。豈敢縱惡傷人。老道提了藥囊要走。賈文物再三留住。道。屈尊師在此下榻一宵。老道執意不肯。賈文物見留不住。叫家人進來。吩咐到當鋪中取銀一百兩來。爲恩師一茶之敬。老道笑道。我要那東西何用。貧道賣藥之餘。盡行週濟貧乏。我何需此物。又要走。賈文物道。恩師雖如此說。但弟子蒙恩。白骨再肉。若不得稍盡寸心。如何過得去。老道也不回答。將手一舉。道。請了。大笑着大踏步走出。賈文物忙隨着趕到大門外。見他已去遠了。這老道正合了古語四句。

\begin{quotation}

坐如鐘。立如松。臥如弓。走如風。

\end{quotation}

賈文物想道。這恩師定是個異人。他雖然不受財物。我明日備一套衣服。親自去拜謝纔是。仍回到書房中。到臥下時。要了一壺暖燒酒。將那兩大丸藥取一丸用酒細嚼嚥下。放下帳子。取出長不過四寸。粗不過一圍的匪具來。將那丸藥用燒酒調開。把陽物週身搽到。又飮了幾杯。然後睡下。睡不多時。藥力發作起來。覺得陽物熱脹得好不難過。虧得先因心中歡喜。將一壺燒酒盡情飮在腹中。有了幾分醉意。脹了一會。就睡着了。一覺直到天明。也不覺熱脹了。用手一摸。嚇了一跳。忙起來低頭一看。大非昨夜之比。竟長將七寸。粗踰雞子。紫威威一個茄子相似。心中比當日中舉中進士還加倍快活\footnote{舉人進士乃身外之榮。此物粗大。不但是身內之榮。且可免許多凌虐。其快活豈止加倍而已哉。}。贊道。恩師眞神丹〔仙〕也。忙起身洗沐了。叫家人拿了幾疋尺頭數對好布。親自坐轎去謝老道。以爲他或者不收紬緞。求他收幾疋布。心中纔過得去。不想到他寓處。門已鎖着。問別的道士時。說他昨日回來。今早又往別處雲遊去了。賈文物悵然而返。轎中自思。這尊師果然是個異人。或是上蒼憐我改變心腸。降下這位眞神仙來救我的苦難。也不可知。他的藥這一椿驗了。別的自然應驗。依他法則去行。萬無不效之理。不一時。到了家中。心內道。此時且不要去招惹他。設或變下臉來。一時難以收拾。豈不誤了晚上的大事。索性等掌燈後再進去。吃了早飯。要養息精神。一覺直睡到下午。又吃了飯。已掌上燈。他走了上去。心中還不住亂跳。走進了房。那富氏也將要睡。好端端坐在那裡。一見了他。顏色頓改。惡狠狠的道。你跑了出去罷了。又進來做甚麼。你拿害病嚇我。你便死了。看可在我心上。我守活寡不如守死寡。還有個名望呢。賈文物總不敢答一言。他罵了幾句。氣忿忿上床去睡了。賈文物等他睡下。然後也脫衣上床。同他共枕而臥。伸手去摸。見他穿着小衣。便去解帶。富氏道。你旣沒這本事就罷了。強掙這個命做甚麼。緊攥住了褲腰。不肯放手。賈文物道。我病中離了你這幾日。心裡想你得很。我今番旣樣樣都改過了。我這一回決不文縐縐的。若不像意。憑你怎樣的打罵。富氏心中也要吃一杯。恐纔興豪。壺已吿罄。或半途而廢。倒心裡難過。所以不肯。非是不好。聽見他說這話。或者他養了這兩日。比前略好了些。倘得一次的樂處。也不可知。不可錯過機會。心裡旣如此想。那手自然就鬆了些。賈文物趁勢脫下。他這一遭。一點的斯文氣也沒有了。還拿出幼年偷丫頭的架勢。一個鷂子翻身。便到了他肚子上。將他兩腿分開。因自己的東西大了。用手捏着。對準了門。下力往裡一頂。進去了一個頭子。富氏哎呀了一聲。道。你拿甚麼東西塞我這麼一下。急用手摸時。竟是他的陽物。還有些疑心。急忙叫他拔出。爬起身來。掀開帳子。燈光照着一看。不是是甚麼。還點頭合腦。對着他一跳一跳。富氏大驚大喜。道。你這是甚麼法兒。弄得這麼大。便一手捏着。尚握不過來。笑得他了不得。賈文物道。我也不知是怎麼緣故。我昨夜睡着了。夢地(底)下覺得發脹。及至醒來。就長得這麼大。可惜醒早了。若再睡一會。長個尺把長。鍾子粗。可不好呢。富氏笑嘻嘻的攥攥又量量。說。你也就得一望二的。這麼大就儘夠了\footnote{此婦竟還知足。}。還要大做甚麼。你的話我就不信。世上只有暴發戶的財主。那有暴發戶的雞巴。賈文物推着他。道。你要看。改日慢慢的細看。此時不要說閒話。誤了正經事。他聽了。忙放了手睡倒。賈文物爬上身。對眞一搗。就進去了好些。富氏道。你好冒失。這還比得往常那一點子麼。慢慢的抽抽着。賈文物那裡理他。一連幾聳到根。富氏覺得內中滾熱。且又塞滿。便不動也甚有趣。賈文物定了一定。大抽大送起來。約有數百。那富氏身不搖而自顫。足無意而高蹺。忽大叫道。不好。你且歇歇着。我要溺尿呢。賈文物知他要丢。越發加力緊扯。只見他道。我要死了。就脚灘(癱)手軟。雙目緊閉。鼻孔中微有哼聲。賈文物也不緊了。只淺抽慢送。培養力氣。却也不歇。過了一會。富氏醒來。問道。我怎麼樣的了。賈文物道。你怎麼樣。如何問我。富氏道。我裡頭急得像要溺尿一般。你不肯歇。忍不住滾熱的流出來。我從頭髮根麻起。直到脚跟底下一酥。就不知道了。賈文物也不答應。又一陣緊似一陣的抽送。不多一會。富氏道。又不好了。又是先那樣子。有一調黃鶯兒說這富氏。道。

\begin{quotation}

雙足自高呈。聳花心任送迎。通身暢快渾忘恨。方纔罷停。須臾又興。仙丹助力能連陣。問卿卿。多年咆哮。此際可嫌憎。

\end{quotation}

此時賈文物也有些乏了。就伏在他身上。停了一會。他又醒過來。道。我怎麼又是一陣熱。身上一陣麻。是怎麼說。賈文物道。每常我丢你是知道的。你這也是丢。富氏道。你每常弄時。幾遭裡面間或有一遭我也麻麻的。有些水流出。不像這等快活。你又說也是丢。賈文物道。雖都是丢。却是兩個道理。當日我的短小。只弄到你這門裡不深。男女交合都有些興頭。弄得工夫長久些。癢癢酥酥的。也就丢了。那出來的是些淸水。如今我這個長大了。直頂到你小肚子裡最深處。叫做牝屋。下下搗着。這一丢是從骨縫裡出來的。是黏糊糊像糨子一般。所以快活得大不相同。富氏歡喜得要不得。道。我的親親。這是誰傳授你的。怎麼這些年不曾聽見你這話。賈文物生平不曾聽見他親親熱熱叫。這麼一聲。不覺渾身也快活的麻了一下。高興起來。又是一場大弄。這富氏連丢三次。也就軟了。叫他道。我的哥哥。你也歇歇罷。不要累壞了你。我可夠了。賈文物纔發市。也覺有些乏倦。便道。依你。歇歇罷。拔了出來。睡下。富氏覺得陰門口一陣熱熱的流了出來。伸手去摸了摸。如稀糨子一般。笑着道。果然你說的不錯。揩拭了。摸見他的陽物還跳呀跳的。笑道。你往常不多一會就像一根皮條。今日也算久了。爲何還是這樣挺硬。必定有做(何)緣故。你吿訴我。賈文物道。我前日有病。鮑信之舉薦了一個四川來的老道來替我醫治。我先還不肯。他再三勸我請了來。不但治好了病。又傳了我這個方兒。你說好不好。富氏道。你好造化。遇着了這樣恩人。不該重謝他麼。賈文物道。你說我造化。難道就不是你的造化。你就不該謝他。富氏道。謝他一千兩我也肯。明日就送了去\footnote{漢文帝云。百金乃中人產。富氏視千金等鴻毛。談何容易。蓋富氏乃一不知稼穡之閨中女子。視千金易而得此巨物難。且又是富宦之嬌女口氣。做平常人說話不出。故妙。}。賈文物道。我要謝他。他一個錢也不要。我親去拜謝時。他已不知那裡去了。富氏道。可惜這麼個恩人。就不得謝謝。難道(怪)鮑信之薦了他來。他又時常送東送西。一事兩勾當。也該謝謝他纔是。賈文物乘他歡喜。對他道。你說鮑信之常送我們東西爲甚麼。他就是含香的漢子。因沾着這些。故此他纔常來。富氏道。旣然是他。爲何〈何〉不接含香來走走。當個親戚往來也好。賈文物笑着道。他怕你打。不敢來。富氏雖說着話。手中不住的捏弄着那話兒。聽他說了這句。笑着將陽物狠狠的攥了一下。道。你還記着舊仇麼。賈文物爬起來又要弄聳。富氏道。我軟癱熱化得動不得了。明日晚上罷。賈文物笑道。誰叫你攥惱了他。替他賠個禮是。富氏捏住。笑道。你這個好怪的東西。每常膿袋似的那個賊樣。今日狗仗人勢起來。就想要我賠禮。賈文物也要養息精神好明晚試法。也就住手。兩人都有些困倦了。嘴對嘴。胸貼胸。手交手。足勾足。睡了一夜。自從成親十多年。這算親熱第一次了。二人一覺直睡到日高三丈。方纔下床梳洗。那富氏精神抖擻。眉開眼笑。把素常那一副惡狠狠的面孔。竟不知往何處去了。丫頭們隨了他多年。並不曾見過他這歡喜樣子。甚是動疑。又不敢問。賈文物雖見他和顏悅色。笑容滿面。大不同往日。恐這一下床。又變起卦來。怎處。且得趣抽身。好圖晚間作用。就往外走。富氏見了。叫道。你回來。賈文物見他叫。倒有些心怯。又不敢不來。走回問道。叫我說甚麼。富氏道。大淸早你往那裡去。賈文物假說道。外頭還有些事。富氏道。料道沒甚要緊的事。這麼大二十多歲的人。還不知愛惜身子。纔好了兩日。大空心就往外跑。外頭風颼颼的。你吃了飯再去不得。賈文物是膽嚇酥了的。有些怕他。故要躱出去。聽見他說了這幾句知疼着熱的話。好生樂意。隨接道。也罷。我吃了飯再去罷。這丫頭們從不曾見姑娘有這恩愛的話到姑爺。今忽見他這樣親愛關切。賈文物雖不怕了。丫頭們倒有些怕起來。此是何故。向日順着姑娘捉弄姑爺。姑爺久知道了的。每常仗着姑娘的勢。諒姑爺沒法奈何。今日若姑娘姑爺和美了。以前的事。姑娘自然不肯認帳。都要推在丫頭們身上。姑爺若追究起來。如何禁得\footnote{姑爺若追究起來。不過送給姑爺一錐。則冰釋矣。}。各人在肚內尋思。却懷着鬼胎。賈文物富氏同吃了早飯。富氏一來想起鮑信之舉薦老道的情。二來聽得含香在他家。想起舊日的事。恐丈夫記恨。要做些情在他身上。以圖丈夫歡喜。況他嫁夫多年。料道決無別事。叫了個家人來。吩咐道。你到鮑信之家。對他娘子說我心裡想他。接他來走走。他要推却不肯來。你是必接了他來。家人應諾而去。賈文物也就出去。到書房睡覺養神去了。那家人奉主人之命到了鮑家。鮑信之正在櫃上穿錢。見了。忙道。請坐。到此有何貴幹。老爺全好了麼。我這兩日忙得很。也沒有去問安。家人道。我們老爺麼。吃了道人的藥。第二日就好了。又請了那老道一席酒。後來又親自去拜。送禮與他。他已經去了。我聽得說送他一百銀子。他一文也不要。眞是個老呆。今日奶奶差我來。叫請你娘子去會會。說想他久了。是必要去走走。鮑信之道。旣奶奶好情來接。敢有個不去的。走進去對含香說了。他倒吃了一驚。想道。當日原是瞞着他的。他如何知道了來接我。恐未必是好意。不去的是我不去。他沒奈我何。到了他家。一時有些口角起來。就不好了\footnote{含香後旣去而先作此想者。見得是個有心機伶倒(俐)女子。非愚蠢猛浪而往。寫得好。}。推道。我今日身子不好。出不得門。鮑信之道。你好好的在這裡。如何會不好起來。況且你是他府上出來的。他好意來說個請子(字)。多少體面。你推辭不去。顯得我們就不識拾舉了。一力撮掇。鮑信之只知他妻子自富家出來。並不知是賈家的人。以前那些事含香又不好說得。沒得推辭。他生的兩個孩子都不乳食。離得的了。也不帶去。只自己打扮了。叫轎子到賈宅來。來便來了。測料不出是甚主意。不多時到了。下轎進去。跟着那家人到了上房。家人說道。鮑家娘子接了來了。富氏一看。好幾年沒見。也出挑的一個大婆娘了。比當日白淨胖大了好些。穿着紬絹衣裙。稀稀戴着幾件首飾。涼線冠子。蜜蠟冠簪。俏生生走進房來。富氏也就站起。他見了。連忙下跪。叩下頭去。富氏忙拉住。道。快些起來。你是客。這是甚麼道理。含香道。奶奶是舊主。應該叩的。富氏再三拉着。道。使不得。拜拜罷。那含香強不過。起來拜了幾拜。富氏也回了他一福。一手拉着。讓他坐下。親親熱熱。說長道短。含香纔放了心。說道。我久想奶奶。不敢來的。今日不是奶奶差管家爺們去叫。我還不敢來呢。富氏道。我起先不知道。只說你不知嫁到那裡去了。昨日聽得你老爺說。纔着人來接你。你是過世老奶奶手裡舊人。就是親戚一樣。時常來走走。可不好麼。含香道。奶奶這樣恩典擡舉。我可有不來的。他又道。奶奶這幾年生過幾位姑娘相公了。富氏道。倒小產過兩三胎。醫生說是怒氣傷了的。總不曾大生一個。你有幾個小孩。含香道。生了兩個小子。大的五歲。第二的兩歲半。我身上又還落明年正月\footnote{敍話處。確乎是婆娘道的寒溫。}。富氏道。好好。是你的造化。那含香道。好幾年不見姐姐們了。我會會他們去。說了。站起身來。富氏笑道。今日早起。替老爺裁了幾件子衣服。分給他們趕忙去做。你不必去。我叫了他們來。遂叫了四個丫頭來。他們都平拜了拜。富氏復讓他坐下。拿上果碟來吃茶。家長裡短說話兒。好不親香。吃了茶。就擺上飯來吃了。此時天氣漸短。日色將已落西。富氏叫丫頭道。看你老爺在外頭做甚麼。去請了來。說鮑家娘子在這裡。請來。我有話說。含香心中也想會會他。因有當日的事。不好說得。聽見去請他。遂道。我還不曾見老爺叩頭呢。只見丫頭來說道。老爺沒往別處去。睡了一日\footnote{此等閒話。後還一照。}。纔醒了吃飯呢。吃了飯就來。賈文物知含香在內。恐富氏多心。不便進來。聽見來請。吃罷飯就進來了。含香一見。忙跳起身來。就要跪下去。賈文物不好拉他。叫丫頭拉住了。問他道。你這幾年好麼。含香眼睛紅紅的。忍住淚。答道\footnote{入神之筆。此所謂笑啼俱不敢。方見做人難。欲哭。不但富氏在傍看着。且兒已有夫。何得還向舊情人灑淚。若竟不哭。幾年的恩愛。百種深情。數年久別。竟恝然如陌路。世上寧有此鐵心人。只如此眼〔睛〕紅紅的。忍住淚幾字。寫得不即不離。妙甚。}。托老爺奶奶的福。將就過窮日子罷了。富氏接過來道。我纔問他。原來他家使的是我們的本錢。賈文物道。鮑信之那年借的伍百銀子。你難道忘了。富氏道。我那裡記得。他是我婆婆眼前的人。你就看顧看顧他兩口子也該。賈文物道。因此我只要他一分利錢。富氏道。噯呀。你好小器。我家怕沒錢使。稀罕一個月要他五兩利錢。因對含香道。你當日出去。我們折(扣)針也沒與你一根。明日叫你家裡來把那文書改成四百兩的。那一百兩算我送與你做本錢\footnote{富氏處處行事大方。斷乎做他人不得。}。含香聽了。道。我怎敢當奶奶老爺這樣厚賞\footnote{稱得妙極。先稱老爺奶奶者。禮也。此係奶奶厚賞。故曰奶奶老爺者。權也。}。富氏道。你要推辭。敢是不稀罕我的。含香眞歡喜出屁來。忙要叩謝。富氏一把拉住。道。多大事。也値一個謝。他又要叩謝。賈文物富氏也拉住了。他辭道。蒙老爺奶奶賞。天晚了。我回去罷。富氏道。你且站住。叫丫頭把方纔那個包袱拿來。丫頭抱過來。富氏打開。道。沒有甚麼與你的。這套衣服與你打粗穿罷。又在頭上拔下一對金花針。替他揷在頭上。含香又謝了。富氏叫了先那家人來。問道。他轎子可在這裡。家人道。在外邊伺候呢。又叫替他把包袱拿了出去\footnote{細。}。賈文物在傍看着。心中暗感激得了不得\footnote{當感激令師。}。再說含香到了家。下了轎。那家人在轎櫃內把包袱取出。遞了與他。含香對那家人道。煩大爺到家謝老爺奶奶。又多謝大爺送我來。那人去了。鮑信之把轎子也打發錢去了。此時他已關了鋪子。隨跟了進來。問道。叫你去做甚麼。含香不好說別的話。只說。奶奶念我當日是去世老爺打發出來的。叫我去看看。遂將給的衣服簪子拿與他瞧。又許明日叫你去換文書。與一百銀子做本錢的話。說了一遍。把個鮑信之喜歡得幾乎打跌。道。這樣好事。你先還不肯去呢。鮑信之滿心只說含香當日是他父親的寵婢。今日想起父親。故看顧他夫妻。再想不到是照看他丈夫的情人。要博丈夫的歡心。再說賈文物夫妻二人共坐。吃了幾杯消夜酒。上床而臥。富氏問賈文物道。今日含香我給他這些東西。你知道爲甚麼。賈文物道。這不過是你的恩典。富氏道。我並不是恩典。我是三爲。一看(者)爲是婆婆的舊人。二則看是你的舊情人。三來是暗謝他男人薦道士的謝儀。賈文物見他一個惡鬼母變了一個善菩薩。心中想。尊師的那種藥可以不必用了。又想道。不好。恐或有變。須遵尊師的法度。遂笑道。我們且做正經事着。故意道。我且摸摸你的這東西。可比每常寬大些沒有。他手中藏了一丸藥。假做摸他的陰戶。摳摳挖挖。已輕輕的送進去了。賈文物却不動手。只對他說些趣話。動他的興頭。不多時。只見他嘴中雖也說話。屁股只是亂扭。賈文物道。你做甚麼只是扭。他笑道。我的這裡頭有些火辣辣的。不好過。賈文物笑道。你就像那饞人一樣。昨日嘗了些好味道。今日看見。就要吃起來。富氏笑道。就把你那東西說得這樣稀奇寶貝一般。我這些年怎麼過了。雖是勉強說着。又見他把腿伸伸。又縮縮屁股。越扭得利害。那手不住的一會兒伸去摸摸。有個十分難忍的光景。賈文物知他內中藥性到了。對他道。我有些饞了。爬上他身子上要幹。他故意夾着兩腿。道。不說你饞。倒說我饞。我偏不。賈文物笑道。算是我饞罷了。強攀他的腿。他也就借意兒把腿放開。賈文物把那話纔對了他牝門。他已將屁股就了上來。陰門口水淋淋的。賈文物笑着道。你旣不饞。如何嘴裡淌出這些口水來了。富氏也笑着道。偏有這些閒話。你要弄就弄。不弄就罷。賈文物見他心裡要。口裡硬。便不動作。只塞進半截。那富氏只管將屁股亂疊上來就他。他總不深入。富氏急了。問道。你怎動也不動一動。賈文物道。到底是你饞我饞。說明白了好弄。富氏此時也硬不來了。便道。就算我饞。怕甚麼。賈文物笑着盡力向內一抵。直頂到花心之上。覺得龜頭撞着。甚是有趣。就認定那個去處。箭箭皆中紅心。起先那一下。只聽得他呀的一聲。後便如豬哼一般。鼻孔內呼兒呼兒的響。再一會。連這個聲都沒有了。惟聞喉中格格略有聲息。就像人臨死掙命的光景。兩腿一蹬一蹬\footnote{此一段與宦蕚降服侯氏時對看。意思微有相似。舉動行事並全不相合掌。}。賈文物雖自幼弄了這些年的此竅。從未見這局面。興致勃然。一頓狠搗。猛然那富氏把他一把抱得緊緊的。道。罷了我了。我可死了。賈文物倒嚇了一跳。看他時。已動不得了。他也就歇歇力。將那話拽出半截。憑他放在戶中。有一盞茶時。只見富氏又往上就呀就的。賈文物知藥力又作。想道。再與他個憇(甜)頭着。又極力衝突一陣。富氏又丢了一次。道。歇歇罷。我乏了。賈文物拔出來。揩抹了睡下。睡不多時。只見富氏下身又是不住的扭。賈文物想道。等他大熬熬。給他個辣手。方可治服。便假裝要睡。過了一會。富氏有些忍不得了。搖他道。我睡不着。你醒醒。大家說說話。怎麼只是要睡。賈文物道。昨日熬了半夜。我睏得慌。讓我睡睡罷。富氏道。昨日我不曾熬夜麼。你今日還睡了半日\footnote{映前丫頭說老爺睡覺語。}。我還是打早間起來。眼睛還不曾合一合呢。偏我就不瞌睡。說着。由不得伸手去捏弄那話。捏弄了一會。賈文物見他手中不住的捏。口中不住的哼。究竟連他自己也不覺得有這種聲息。賈文物聽得暗笑。自己也興動了起來。道。我再弄弄着。那富氏正在熱癢難過。眞要死的時候。却硬捱着不肯叫他。忽聽他說要弄。如得了命一般。忙將身子睡平。兩足高擡。見他纔上身。捏着陽物往牝中亂塞。賈文物心中又好笑。又恨他嘴硬。上手一別氣就有千餘下。富氏又丢了一次。賈文物不歇氣。又是一陣大弄。富氏又丢訖一度。此時身子也軟了。膀子也扳酸了。腿也蹺疼了。便道。我可夠了。你也下來睡罷。賈文物道。我看你每常饞撈撈的。就像要吃多少的一般。怎麼如今略弄弄就說夠了。恨不得求饒。怎這樣不濟。富氏笑着道。虧你文縐縐的呢。連一點道理都不知道。譬如一個人飢着。一頓只與他一個燒餅吃。一日到晚。零碎吃五六個燒餅。名算吃了五六頓。如何得飽。把大酒大肉放在跟前。儘性吃飽了。一日不過吃兩頓。敢自夠了\footnote{富氏此比。深悟得此道中三昧者。}。說得賈文物也笑了。也就下來。不多時。〈一〉那富氏心中實在足了。無奈那陰中又一陣熱癢起來。先還咬牙忍着。過了一會。忍不得了。故意問賈文物道。我看你這個東西。他那樣強頭硬腦的。也有本事一夜弄到天亮麼。賈文物道。又不是鐵的。那裡有這樣本事。我弄了這一會。也就有些怕動的了。這兩句話。一則是知他想弄。故推懶動急他。二者要激出他的話來。好降服他。富氏一面笑着。一面用手指將他臉上一掃。道。不害羞。你這樣的本事。開口就笑話我不濟。看你濟的這光景也有限。他這話也是激賈文物再來弄弄的章思。誰知正落在他彀中。賈文物道。你我也不必爭講。我們打下一個賭賽。我就動不得。拚命也做做。富氏道。怎麼打賭賽。我不怕你。我小時聽見老婆們說。有怕屄的屌。沒有怕屌的屄\footnote{不意今日竟怕屌也。}。任你怎麼來。賈文物道。我們兩個此時弄起。不許歇。我若說乏了。算我輸。你若說夠了。就算你輸。我輸了呢。明日我篩一杯酒。站着雙手送與你。作揖哀吿說。好姐姐。兄弟知道你利害。饒了罷。下次再不敢犯上了。你若輸了。也是這樣。要叫親哥哥。妹子你可敢賭麼。富氏要弄得很了。說硬話道。不怕不怕。看我可求你。你就來。賈文物摟住了他。笑着一翻身爬起來。他把臀墊起。極力搨打。約有數百下。看他那樣子。像又丢了。賈文物息了一息氣。又是一陣。更加勇猛。富氏又過了。覺得有些支撐不住。却不肯輸口。你想一個婦人的陰戶。弄了大半夜。丢了五六次。就是鐵打的陰門也磨鋊了\footnote{娼妓之牝戶。鐵耶。鋼耶。想情更不知鋊之如何。}。水做的陰津也流乾了。何況是皮肉。賈文物看他有些難支架了。笑着激他道。我看你像要敗了。你求吿一聲。我饒了你罷。那富氏是崛強硬慣了的人。不做聲死捱\footnote{竟有恆心。}。賈文物又繫(緊)提慢抽的弄將起來。富氏嘴中的聲氣與先大不相同。先前是快活的哎呀二字。那是帶些喜樂的腔口。此時雖還是這兩個字眼。聲音是帶些痛苦的光景。賈文物見他有些受不得了。趁此好收服他。鼓勇直前。一下重似一下。一抽重似一抽。那富氏忽然一個寒噤。便昏迷過去。賈文物也就不敢動。伏下身子。口對了口。見他只出冷氣。雙眼緊緊閉住。就如要死的一般。幸得都是老道預先說到\footnote{補出先老道附耳傳授的話。}。不然這一驚不小。他度了半會的氣。將有半個時辰。方見他漸漸醒將轉來。也沒聲氣了。低低的道。哥哥。我知道你的厲害了。饒了我的命罷。又有個黃鶯兒道他兩人這番光景。道。

\begin{quotation}

魂斷雨雲鄕。羨兒郞興致狂。高擡玉股淫情蕩。強陽焰張。柔肢軟僵。都傳老道仙方上。喜盈腔。回生妙訣。此法實無雙。

\end{quotation}

賈文物笑着道。你認輸了不。富氏道。是我輸了。賈文物道。你求饒。明日可替我遞酒賠禮麼。富氏微笑着不答。賈文物道。你還嘴硬。我直弄到天亮纔罷。又要抽動。富氏忙陪笑道。我賠禮。我賠禮。賈文物又笑道。是你不濟。是我不濟。富氏連聲道。你是好漢。是我不濟。你歇了罷。賈文物道。你不要慌。等我弄丢了着。富氏慌了。道。哥哥。你可憐我罷。我渾身骨頭都軟了。受不住了。賈文物也不答。放了一口氣。亂抽了一陣。他的陽精也冒了出來。富氏覺得內中有一股熱水似一澆。那熱癢全消。你道他快活不快活。賈文物下得身來。那富氏陰戶也沒力氣揩。身子也沒力氣翻。就是那樣仰〖扌扉〗着。揸着腿。動也不動。又不像死又不像活的樣子。賈文物聽聽外邊已交五鼓。身子也乏了。同他蓋上了被。一覺睡着。直到次日飯時方醒。賈文物先起。富氏又睡了一會。掙着起來。覺得腰酸背折。兩腿軟得站都站不住。暈昏昏的。就像害了許久病的人一樣。賈文物看他的面色如一張金紙。鼻凹烏靑。嘴唇雪白。眼睛也摳下去了。眼皮子餳着睜不開。想道。尊師再三囑咐不可過用。恐傷性命。今晚若再一用。定然要送命了。那富氏要洗臉。兩隻膀子擡不動。將就撂了一把。他頭是丫頭梳慣了的。不用自己費力。梳洗完畢。拿上飯來。他也懶吃。賈文物強讓着。勉強扒了兩口飯。吃不下。只喝了幾口湯。賈文物飯罷。將鮑信之的文書查出。拿着往前邊去了。剛到書房坐下。只見賈閽進來說道。鮑信之在外邊。賈文物道。叫他進來。不一時進來了。見了便道。門下的女人。昨日在府上蒙老爺奶奶賞酒飯。又賞衣服頭面。感恩不盡。賈文物讓他坐下。問道。昨日叫你換張四百兩的文書來。你娘子對你說了不曾。鮑信之道。蒙老爺奶奶的天恩。門下帶了來了。遂在袖中取出。立起雙手遞上。賈文物打開。見利錢空着數目。便道。這一百兩銀子是奶奶與你娘子的。我如今這四百兩銀子。連利錢也不要你的。只後來掙了餞。還我本錢就是了。遂把那舊文書還了他。鮑信之千恩萬謝。道。改日還着門下的女人來叩謝奶奶。辭了回去。到家中與含香說了。好生歡喜。商議道。蒙他這樣大情。你改日買分禮親自叩謝奶奶去。含香道。他家甚麼沒有。稀罕我們的禮物。除非尋得幾樣外路出的好吃食。纔拿得去。鮑信之道。你說的有理。過了兩日。鮑信之街上去尋了四個龍豬。八隻雄鴨。四隻固始鵝。兩個果子狸。又買了一罎金華豆酒。着含香親自送與。富氏謝了。留他酒飯而回。此後也時常來往。鮑信之又得了這一百兩銀子。他家中這幾年也積有二三百金了。他將賈文物的銀子用了兩年。送還了他。此是後話。不題。再說那富氏茶飯都懶得吃。悶昏昏一覺睡到日色沈西。方纔起來。雖覺得精神了些。身子還酸軟怕動。賈文物也出門回來了。進房問富氏道。你吃了些甚麼沒有。富氏道。自從你去。我睡到此時纔醒。一日湯水還沒有嘗着呢。賈文物叫快拿飯來。不多時。擺上同吃。富氏此時覺好些。也餓了。強吃了一碗。撤去。拿上果碟來吃酒。賈文物想起。在紬(袖)中拿出文書。叫丫頭拿過匣子來收了。向富氏道。鮑信之拜謝。改日還叫他女人來替你叩頭。說罷。笑道。你忘了一件事了。富氏道。我忘了甚麼。賈文物道。賠罪的酒。你不要裝憨兒。富氏嘻嘻的只是笑。不做聲。賈文物道。你賠我個禮好呢。你要這回失了信。下回看我聽你不聽你。又笑道。這也憑你。只不要怪我。富氏笑道。丫頭們看着甚麼樣了。瞅了他一眼。賈文物見他說。便叫丫頭們都出去。富氏笑道。只遞酒。不說罷。賈文物道。我不強求\endnotemark[2]你。你不叫。後來再求我歇一歇。看我可依。富氏當眞有些怯他。恐弄個不住禁不得。二則要圖得他\endnotemark[3]的歡心。到了此時。把以前降丈夫的手段一些也記不得了。笑着道。你仗他的勢子降我麼。罷了。我替你賠了禮。你明日再不要落在我手裡。口說着硬話。却拿過一個杯來篩了酒。起身遞與賈文物。他只是嘻嘻的笑。賈文物道。你不說不拜。我也不吃。也不算。他笑着下來。拜了一拜。道。親哥哥。小妹妹再不敢了。你饒了我罷。把個賈文物喜得說不出來。笑着一把抱住。道。親姐姐。你不要再得罪我了。吃罷。也回敬了一杯。說笑了一回。然後上床。脫衣睡下。賈文物暗想道。今夜藥是用不得了。却不可放空了他。還要給他個心服。一時間\endnotemark[4]摸摸捏捏。動興起來。向富氏道。再來嘗嘗新。富氏此時如狗偷熱油吃。又愛又怕。道。我身子還稀軟。頭還迷呼呼的。怎麼樣。賈文物道。不怕的。你沒聽見人說酒投酒麼。說着。跨上身來就弄。幹訖一度。富氏雖覺難支。也還受了。少刻又動起來。富氏覺當不得了。將陽物攥住。道。我心裡顫呵呵的。頭一陣陣發迷。你再弄。我實在要死了。我情願求饒罷。賈文物道。我再弄兩下子也就罷了。富氏道。一下也來不得。不要說兩下。賈文物道。當日我弄不得的時候。你不是打就是罵。如今我要弄弄。你就是這個樣子。這是人說的。只許州官放火。不許百姓點燈。就是你了。富氏見他說起舊話。自己也有些過不去。便道。當日金桂洗澡。你弄了他一下。是我一時不是。打了你兩下子。如今叫他來同你溫溫舊。算我替你賠禮罷。夫妻間不要題舊話。那就情薄了。賈文物道。你這會兒好心說好話。過後懊悔起來。叫丫頭吃虧何苦。況我當日穿着褲子混戳了幾下。門邊兒還不曾俟(挨)着呢。也沒有甚麼情溫得。還是我同你弄弄穩實些。又要抽拽。富氏攥住他陽物不肯放。道。這是我好意。你何用疑心。等我叫他來。賈文物恐有後變。是拿話穩他。便道。雖承你的情。後來不可變臉。二者假如我正弄得高興。你又叫起我來。如何使得。富氏道。這是我自己情願的。再後悔起來。可還成個人。你只管放心。賈文物聽了此話。心中暗喜。富氏叫道。金桂過來。那丫頭這兩日見主人主母忽然和氣到這等地位。猜不出是甚緣故。正在狐疑。晚間見他二人吃酒說笑。攆他們出去。雖出去了。都遠遠在門外黑處偷看。見姑娘出位遞酒。雖不曾聽見說甚麼\footnote{妙。}。那種光景看得好不肉麻。別的丫頭雖都二十多歲。服侍了十多年。連姑爺的這件寶貨。張也不曾張見。倒也罷了。惟有這金桂。當年被他混戳了幾下。雖未曾嘗着。何嘗不想。因怕姑娘。不敢及此。今見姑爺姑娘這些舉動。竟像另托生了一番來的。大非昔比。想偷去張張。看是怎樣。此時正在窗外偷看。燈光照着。看得明明白白。那個光景好不動火。聽得叫他。不知何故。倒忙走到西屋。假推睡着。聽得又叫。方走了來。富氏道。把你的鋪蓋抱了來。在我床面前上夜。那丫頭去捲了抱來。鋪好睡下。富氏推賈文物道。你去罷。賈文物也就下床來。鑽到他被中。要扯他褲子。丫頭聽見主母叫下來。雖知是說明了的。沒有個公然笑納之理。假意道。還不去。我〖口么〗喝奶奶呢。富氏道。是我的主意。你叫甚麼。倒爬起來探出身子。拉過枕頭靠着看他二人舉動。丫頭聽見主母的話。手也不推一推。憑着主人公替脫褲子就脫。叫他撞(揸)開腿就揸。他是久慕的了。那裡還推辭。賈文物知他是女兒。用上許多唾。然後對了門路。丫頭年紀雖大。陰門還是整的。主人公之物又大而且粗。一時不能入去。賈文物興發如狂。也顧不得他了。狠命往裡一送。力太猛了。竟攮進去多半。把個丫頭疼得要死。叫道。噯呀。這兩個字與他主母字同而音各別。他主母是心中快樂。喉中微微有噯呀噯呀的字意。他這是疼得受不得。猛然叫一聲噯呀。二字響亮而無餘韻。賈文物見他受創。輕輕慢慢的抽拽。看他那樣子苦到不可言處。皺着眉。齜着嘴。抽一抽。他把嘴咧一咧。賈文物又憐又愛。抽了一會。略略相安。只略重些。他又愁眉苦臉起來。賈文物不得快暢。便道。罷。讓你歇歇再弄\footnote{此一段與宦蕚弄嬌花一對。其意相似。其行事毫不相似。}。拔出。跳上床來\footnote{跳字妙。見得非假斯文了。}。摟着富氏道。丫頭不濟。還是我們來。這件事自己\endnotemark[5]做着不覺。看着別人做。那心窩內眞要死要活。富氏看了一會。身子雖怕動。心裡却十分難忍。先說過的。又不好叫他。見他上來要弄。正中下懷。就乘勢臥倒。任他衝突了一陣。却也就渾身癱軟。心滿意足。酥酥要睡。說道。你讓我睡睡罷。你再同丫頭弄去。賈文物又下來。金桂悄悄的道。疼得很。明日晚上罷。賈文物摟着親了個嘴。也悄悄向他道。我當日爲你。腰都幾乎打折了。你今日就受些疼。也不爲過。這一回不像先了。丫頭強不過。只得聽他。雖然還有些疼。比先似乎可忍。後來也覺有些趣味。弄了多時。賈文物擡身看看富氏。見他沈沈睡熟。便放心同金桂摟抱着睡。到有四更方醒。又弄聳了一番。金桂也微微得些樂處。方輕輕上床。同富氏共臥。睡到天明。只見富氏昏昏沈沈的哼。忙叫着問他。總不答應。又問了幾聲。富氏方矇矇睜開了眼。道。我身上不好過得很。不要吵我。賈文物自己起來。替他把被蓋好了。梳洗過。走來看他。見他面色灰黃。還昏昏睡着。不敢驚動他。你道富氏爲何這個樣子。他雖性情凶暴。身子却不甚健壯。三十多歲未經過大敵。前夜初等(嘗)甜頭。盤弄了半夜未睡。精神未免消耗了些。次日心花俱開。一日不曾眨一眨眼。次夜被藥力一助。丢了七八次。又是一夜。你說禁得禁不得。昨日雖未睡倒。也就是勉強掙挫着的。這一夜次(雖)只兩次風流。傷了的人又復着傷。自然難受。賈文物倒有些暗暗着急。守着他到午間。略醒了一醒。問他可吃甚麼。搖頭不吃。又還是那昏昏的樣子。富氏頭沈。眼睛怕睜。四肢酸軟動不得。他心裡却是明白。想道。我只說這件事只有樂而無害的。恨命的想他。今日看起來。再要一夜。這命就要斷送了。但恐他不肯放我。我如今把四個丫頭都與他。讓我養息養息要緊。叫他那屋裡去睡。我一時有高興。間或叫他來弄弄。適興而已。貪不得的。心裡想着。就睡着了。一直到晚醒來。賈文物強着他吃了些粥。他吩咐金桂。將西屋床上鋪了被褥。給你老爺過去睡。賈文物驚道。這是爲甚麼。富氏道。這件事怕人子。要送命的。你守着我跟前。未免忍不得。倒是分開了好。賈文物道。這不難爲我了。富氏道。只有便宜你的。如何得難爲。叫了四個丫頭到跟前。吩咐道。每日晚上着兩個來替我上夜。兩個在西屋裡服侍你老爺。五日一換。四個丫頭聽了這話。喜得臉上忍不住要笑。你望我。我看你。忙忙去鋪床。大家商議那兩個做一班。恐先後有爭講。齊抽長草兒去了\footnote{這却聽憑姑爺錐究。亦不懼矣。}。賈文物捧着富氏的臉。嘴對着嘴。道。姐姐。雖然你這麼說。撂得你冷淸淸的在這裡。我心裡過得去麼。富氏道。只要你好心。你這一句話就夠了。你只管去。我但是有高興。就來叫你。難道夫妻間還怕羞麼。賈文物見他是眞情實意的話。也就從命。到了那邊。四件不曾經過陽物的原封妙牝。任他着意鑽硏。不必細說。那富氏過了四五日纔好了起來。果然此後夜間。或有高興。叫了他來解解饞。不過一二次即止。仍不許他常睡在身邊。事完還叫他過去。過了多日。他見賈文物同這四個丫頭打得火熱。雖不捨得加辭色到丈夫身上。意思又想在丫頭們跟前施些威。使他自己迴避。又好獨享。省得眼中冒火。賈文物見他有些舊性復萌之意。只得又將一粒靈丹奉承到他牝戶之中。熱癢難當。由不得他要弄。前次傷過了的。這一次足足病倒十數日。幾乎喪命。比後再不復生妒念。有四句打油說他道。

\begin{quotation}

時嫌錯嫁怨蒼天。不遂淫情怒欲燃。

死去復生方釋妒。惡姻緣變好姻緣。

\end{quotation}

不意這幾粒仙丹。把一個悍妒之婦治得拱手服降。安得這峨嵋山人遊遍天下。捨幾擔靈丹。醫遍世間妒婦也\footnote{幾擔靈丹恐怕不夠。}。

\begin{quotation}

鵒鶊昔未療郗妒。丹藥今能治富淫。

\end{quotation}

且說這峨嵋山人突然從何而來。得非是做書的人強爲捏合。湊成賈文物這段佳話。凡看書者須要有眼力。前後注射。又要有記性。始終照應。方知作書者苦心筆力。這個老道就是向年在南京朝天宮做寓。會着到聽的那人。他祖籍陝西。因慕四川峨嵋之勝。到那裡做了黃冠。拜了個異人爲師。傳授了許多異術。那峨嵋山雖係普賢菩薩的道場。但此山甚是廣大。內中緇流的寺刹固多。羽士的廟觀也不少。不曾到過上邊的。以爲單有佛寺。這道士在山修練了二十來年。辭別本師。要往各處雲遊。因想南京係六朝建都之地。太祖又興王於此。又聽人傳說有許多勝蹟。遂迤〖辶里〗到了南京。在朝天宮住下。會着到聽。在接引庵遇了黑姑子那件肉寶。留連了半年。出來各處遊賞了一番。後到了西湖。又遇了那奇淫的昌氏。心中想道。我前在南京流覽那龍蟠虎踞之勝。以爲是大觀了。不想西湖更有別趣。無怪當日完顏亮有立馬吳山第一峯之句。垂涎此處。我平生所遇之婦人也不少了。不意又遇着這昌氏。可見天下之山川人物。各地不同。我何不將天下各省以及名山大川遍歷一番。以豁心胸。或閨中得遇異材。又可長些識見。拿定了主意。他有煉求(丹)之術。路費不愁。他發了遊興。次日到北新關。雇船到湖州。泛太湖。登洞庭山。得食山上所產的楊梅。眞異品也。時正六月。洞庭紅尚還未熟。又食沙瓜。即西瓜也。其大如斗。剖開。內中無瓤無子。滿貯一瓜淸水。香甜之美。莫可言喩。由嘉興復繞到杭城。正値中秋。登城隍山觀潮。錢塘江每日有潮。一年只八月十八大潮一次。水聲如萬馬奔騰。浪頭高有千仞。是日有弄湖(潮)子弟。合城男婦大小往觀。亦一異景也。次日。出錢塘門過江。自西興抵會稽。走山陰道。眞如身在畫圖中。探禹穴。又到天臺縣。過藍橋。遊天臺山。在各寺中玩賞了數日。到象山。遊雁宕。眞好一個去處。昔日林霽山有一首律詩道。

\begin{quotation}

驛路入芙蓉。秋高見早鴻。

蕩雲飛作雨。海日射成虹。

一水通龍穴。諸峰盡佛宮。

如何靈運屐。不到此山中。

\end{quotation}

王十朋也有一首絕句道。

\begin{quotation}

歸雁行飛集澗阿。不貪江海稻粱多。

峰頭一蕩雖奇小。飮啄偏堪避網羅\footnote{引此二作。有意伏後鍾生隱居於此張本。}。

\end{quotation}

又遊賞一線天珍珠簾等蹟。把浙江各處名勝之地都遊遍了。他不拘歲月。但遇有好山水。便多住些時。自饒州出江西。到吉安。過江看文筆山的文竹。天下做筆管之竹皆產於此山。又到南昌。登滕王閣。遊鐵柱宮。復順流而下。過鄱陽湖。上小姑山。覽彭澤之景。到匡廬。登廬山。上香爐峰。遊白樂天書院。又重溯流而上。到南康府。城中十戶有七八家賣紫石硯。歷十八灘到贛州。過大庾嶺。正遇梅花大放。過嶺到南雄。廣州肇慶都歷了。渡海到瓊州。復回到潮州。謁文公祠。看湘子橋鱷溪。又遇夏月。食鮮荔枝。天下之果以此爲勝。將粤東景致遊遍了。由灘河入閩地到汀洲。至泉州看洛陽橋。已是深秋。見秋海棠高有丈餘。圍及二三尺。上四府人物風俗還有可觀。下四府皆綿蠻鴃舌。悉深山老箐。並無佳致。猿猴孔雀遍滿山谷。無心遊賞。遂折入廣西。也遊玩了些時。見了些異言異服之類。不可名狀。因多瘴氣。遂自建昌入雲南永昌界。至大理。登點蒼山。又看洱海鹽井。方到了滇城。登眺金馬碧雞。泛滇池。遊羅漢山。天下之水皆源細流大。惟昆明之水源大流細。故名之曰滇池。遊過了。從曲靖食木瓜梨。過滇南勝境。入貴州界。也遊了旬月。到了鎭遠府。隔河鎭遠衛屬湖南所轄。他步履了萬餘里。到此上了〖舟秋〗子船。自灘河順流直下。沿途見了沈香船銀壺山許多古蹟。數日即抵常德。特往衡州。登眺南岳。看迴雁峰。又到永州看石鏡。到武當山朝眞武看金殿。賞玩了幾日。到荆州拜關夫子。眞像一部長髯。俗畫鬚五綹者。或壯年時如此耳。到岳州。登君山。謁二妃詞(祠)。瞰洞庭。水光接天。\endnotemark[6]一大觀也。過湖抵武昌黃鶴樓下。泊舟登岸。覽省會之勝。到承天府看興獻帝陵寢。至黃州看赤壁。顧一世之雄也。而今安在哉。不勝慨嘆。楚地全遊了。由三峽之蜀過巫山高唐。觀灩澦堆魚腹浦。看武侯八陣圖。嘆羨遺蹟之奇。看那三峽之水。眞有一瀉千里之勢。兩岸奇異之景。不能盡述。惟兵書峽獨異。遙見山巓有書一册。遇風則篇篇翻轉。風止仍自合。奈在萬仞之上。人不能歷。到了重慶。復從陸路至成都。誠所謂天府之國也。重到峨嵋謁本師。已經羽化。不勝有物是人非之感。又去遊了雞足。回來由雅州過江走棧道\footnote{千古險途。到今削成平坦大道。此功眞利於萬世也。}。出漢中到故鄕來。年豐物阜。不禁色喜。過西安到華陰。上西岳。因戀故園風土。住了年餘。偶然遇着一個少林寺出來雲遊和尚。二人甚是投機。偶談及房幃之事。道士把養龜採戰之術傳了他\footnote{伏後傳童自大。}。兩人同出潼關。到河南遊了中岳。和尚別了回寺。道士取路往濟南。轉北到泰安州。風景又爲之一新。又登東岳。復折入山西。遊天(太)行雁門。到五台看文珠(殊)菩薩殿宇。至渾源州上北岳。回入北京界。到眞定參大佛。又到了昌平。看天壽山諸陵。遊了遊西山諸境。纔到了京城。進彰儀門。到報國寺住下。那報國寺雖名曰寺。做各色買賣的塡塞於內。凡各省來京的官員。或閒遊之人。寺中皆有房租住。這道士也就在內租了一間房子養靜。他租的就是萬緣和尚的房子。萬緣見他一表非俗。飄然有仙氣。十分相敬。常陪他談講。他無所不知。後知他有房中秘術。要想求敎他。故時時親近。常常奉請。兩人頗甚投機。權且按下。且再說那姚澤民自奉旨往廣西省親。那桂氏不但無惜別之意。反私心暗喜他這遠去了。歸期尚不知何日。更好放膽行樂。但是萬緣到佛堂來住。他便備下珍肴美酒。只到定更時候。姚步武或來弄過去後。或是不來。便叫素馨約了他進來。二人並肩疊股。摟抱着頑耍。飮酒說笑。有幾句話寫他二人。道。

\begin{quotation}

渾似目連救母。宛如柳翠逢僧。翡翠衾中。桂氏胯間。劈破一雙菡萏。鴛鴦枕畔。萬緣項上。平分半個葫蘆。桂氏道。你大頭因甚不似小頭光。萬緣道。你豎嘴爲何不如橫嘴緊。萬緣道。你上口櫻桃。下口包含紅芍藥。桂氏道。你毛頭刺粟。光頭色似紫葡萄。萬緣道。你上口含我舌。下口含我陽。被我占盡便宜。桂氏道。小頭流出膿。大頭流出汗。看你吃盡大虧。萬緣道。我在上你在下。搗碎你花心方休。桂氏道。我以逸你以勞。箍出你腦髓纔住。桂氏道。我男兒陽物。那裡及得你這驢㞠。萬緣道。我徒弟粗臀。怎能似得你這妙牝。桂氏道。千般寶玩。怎如驢腎可開心。萬緣道。百味珍饈。難比紅蝦能悅口。

\end{quotation}

二人酒興一濃。便黏做一處。桂氏雖好淫而不耐戰。禁不得他的紫筋矛分(花)心幾攮。那桂氏的軟皮劍即無力支持。就是素馨靑梅綠蕚。見了他便覺心痒難撓。及至上身。不多工夫。就遞了降表。猶如那好飮而量窄的人。見了酒就流涎。吃不上三杯。便酩酊如泥。惟有香兒生得身子壯實。可稱勁敵。同萬緣有幾合潑戰。間或萬緣回寺裡去。桂氏便叫盛旺來補空。總是他這身子一夜也不肯獨宿。陰戶半宵也不許空閒。眞如在極樂世界中過日子。這幾個丫頭托主母的餘福。也幾幾成了散仙一般快樂。只那裘氏同那八妾十婢。與姚澤民朝歡暮樂了幾年。忽然一旦分離。也不像去了個兒子。竟像死了個丈夫。茶慵飯懶。淚眼不乾。大家坐着閒話。但題起他來。就不住墮淚。後來想了個排解之法。把家中的僕婦們叫了上來。講新聞說白話釋悶。說了幾日。這些婆娘所知有限。沒得說了。就叫他們將鄙穢村淫的話只管謅着說。那些婆娘要奉承夫人歡喜。無般的不說出來。却都拙口鈍腮。頭上一句。尾上一句。支支離離。說得總不入耳。說了些時。連這謅話都謅不上來了。內中有一個常氏。是裘氏陪嫁的僕婦。生得薄薄的兩片嘴唇。密縫着一雙色眼。能言善說。口舌便俐。當日姚華胄在家時。常上下傳話便是他。他專會無中生有。得不的一點風兒就是雨兒。但是下邊有甚麼話。他便到上邊添出許多枝枝葉葉。輕事重報。吿訴主母。衆家人都恨他。贈了他個美名。稱爲長舌婦。他圖得主母的歡心。小意慇懃。無所不至。早來晚歸。強拿強做。強說強笑。裘氏也着實愛他。分外擡舉。他的男人隨姚華胄去了幾年。他常在上邊上夜。間或也還回去。他也被姚澤民錄過的。姚澤民知他是裘氏心腹。故千方百計弄上了他。在內中做個線索。一日。裘氏同衆妾閒話了一會。心上憂悶。叫長舌婦來說笑話。他就隨口謅了一個。裘氏道。不好。你不管村的淫的。只要有趣。說了我們聽。長舌婦想了一想。道。我說這個笑話。衆位姨娘聽上興來。不要怨我。說道。

\begin{quotation}

一個小媳婦子站在門口。看見一個叫驢跳那草驢。爬上去左戳右戳。再戳不着門。弄不進去。他心裡急得了不得。見一個小孩子手上架着個麻雀兒走了來。他叫道。小人兒。把麻雀我替你拿着。你把那驢子替他送進去。那孩子也高興。就把麻雀遞給他。他一把攥住。那孩子去把叫驢的㞠子扶着。對了門。那叫驢狠狠的往裡一送。進去了大半截。那小媳婦子把牙一咬。渾身替他一趲勁。不覺把個雀兒攥死了。那驢子聳了幾下下來。那孩子要雀兒。這媳婦子張開手看時。已攥扁了。那孩子哭道。你叫我搊驢子肏屄給你看。你把我的雀兒都攥死了。那小媳婦羞得跑進屋去。過路的人聽見了。傳爲一個笑話。

\end{quotation}

裘氏笑得了不得。說道。就是這樣有趣的。你想着說。衆人都笑了一陣。芍姐笑向菊姐道。你每常可這樣趲勁。菊姐笑道。我倒沒有趲勁。我聽見二爺說他同丹姐姐初弄的時候。你倒急得咬牙來。兩人嘻笑擰掐着頑。裘氏道。你們不要鬧。叫他再說。常氏笑着說道。

\begin{quotation}

一個女兒臨嫁。叫陪嫁的丫頭道。我聽見人說。頭一次弄的要疼。我怕受不得。你夜裡醒睡些。我要疼得很。你來替替我。那丫頭歡喜得了不得。他夜裡留心聽着。到了半夜。忽聽得姑娘哼着叫道。丫頭。他忙走到床面前道。姑娘可是叫我來替麼。那姑娘道。不是。你把梳匣子裡的抿子拿了來。那丫頭谷都着嘴道。半夜三更要抿子做甚麼。那姑娘顫着聲兒道。你拿抿子桿。把姑爺的兩個卵子都替我抿進去罷。

\end{quotation}

把衆人笑得一仰一合。蓮姨笑着道。水仙。你姨娘叫你拿抿子呢。臘姨道。抿子我倒用不着。叫碧梧尋個棒槌來與你罷。大家又笑了一回。裘氏道。你就說這樣有趣招人笑的好。常氏道。春姐眼睛紅紅的害眼。我說個害眼的笑話罷。

\begin{quotation}

一個女人屄裡頭生了個毒。疼得了不得。叫男人去請醫生。男人說。我知道甚麼醫生會醫這東西。叫我那裡去請。女人說。他必定有招牌。你去尋就是了。男人只得去尋。一個眼科他家中那日有事。不曾掛招牌。就橫放在門外的櫃臺上。那男人猛看見招牌上畫的眼睛直豎着。想道。這必定是醫此道的了。遂請他到家。那眼科道。須得看看。纔好用藥。那男人同女人商議。這東西如何好與他看。沒奈何。叫女人爬在床上。蹶着屁股。將帳子掀開一縫請他看。那醫生當是看眼睛。先將一個指頭按按。看可腫熱。不想一下正按在那東西裡頭去。將指頭進去了半截。那醫生縮回手。往外就跑。男子拉住他。道。請你看病。怎麼要跑。那醫〔生〕道。爛成了這麼個大洞。連眼珠子都沒有了。還看甚麼。

\end{quotation}

衆人笑得跌跌滾滾的。雪姐問榴姐道。你的裡頭有眼珠子沒有。榴姐笑道。我倒沒看見你裡頭的眼珠子。那日倒見你的一朶大花心。幾乎被二爺搗碎了。大家笑着。丹姨道。你再說。常氏儘着想。裘氏道。說就說罷了。拿班做勢的。常氏道。哎呀。我又不是個笑話口袋。打開了只管往外抖。也等我想想。忽然笑道。我想起一個好的來了。

\begin{quotation}

一家子的老婆。一個錢也不肯給男人用。那漢子想塊肉吃也不能夠。想了一個計策。總不同老婆幹事。那老婆急了。問他。他說。我不知甚麼緣故。把個陽痿了。前日叫醫生看。他說這不是病。不知得罪了甚麼鬼神。須得三牲香紙還個願就好了。老婆說。這是要緊的事。你怎麼不早說。忙取了些錢。叫買三牲紙馬來。安排停當。說(對)男人道。你上香。我祝贊。那男人纔上香。他在傍邊祝道。一炷香。保佑雞巴硬似鎗。男人道。太硬了。老婆說。我好容易花錢費鈔的。也要這樣纔好呢。

\end{quotation}

裘氏同衆人嘻嘻哈哈笑個不住。丹姨向衆人道。你們可都愛這硬似鎗的。雪姐笑道。姨娘。此時就有個皮條軟的給你救救急。你也情願。還想要鎗呢。榴姐笑着接口道。雪姐姐就說的。丹姨他屋裡放着老爺的一桿手鎗麼。他難道不會用他。稀罕那皮條做甚麼。丹姨道。那我用不着。你兩位若愛。我就奉送。常氏道。衆位不要鬧。我又說了。都纔不做聲。他道。

\begin{quotation}

一個女孩子出嫁。纔十四歲。女婿有二十多歲了。娘怕女兒小。禁不得。囑那陪嫁的丫頭道。你每夜聽聽看姑爺姑娘成親是怎樣的。到了回九。他娘問丫頭道。我叫你聽。是怎樣來。丫頭道。頭一夜聽見姑娘叫疼。這兩夜姑爺又叫疼。他娘驚道。姑爺爲甚麼叫疼。丫頭道。說是姑娘把姑爺的屁股扳破了。故此叫疼。

\end{quotation}

衆人聽了。眼淚都笑了出來。臘姨笑向桂姨道。那日二爺在你房裡出來。向我說屁股疼。原來是你扳的。正說着。那常氏往外要走。裘氏道。你往那裡去。常氏道。我嘴說乾了。吃口茶來。裘氏道。不許去。叫秋月倒鍾酒與他吃了。又叫再斟給他。春花拿了個碗。倒了一碗來。道。夫人。那鍾子不濟事。這碗酒叫他吃罷。裘氏笑着點頭。春花拿過他叫吃。常氏道。春姐。我吃不得急酒。放着。我慢慢的吃。春花道。夫人賞你的。等你慢慢的吃。你好嬌貴的性兒。你纔罵我爛了眼珠子。我且官報私仇着。拿起碗向他嘴裡一灌。他只得一氣吃了。抹着嘴。哎呀哎呀了幾聲。瞅着春花道。君子報仇待三年。小人報仇在眼前。又道。我說個吃不得急酒的笑話罷。

\begin{quotation}

一個寡婦要嫁漢子。要尋個大㞠子的。想道。我聽見人說。男人鼻子大㞠子就大。他一日看見個大糟鼻子的人。愛上了。央人去說要嫁他。那人就娶了他去。因衆人來賀喜。多了兩杯。醉了睡着。這婦人見他不醒。心裡着急。解開他褲子一看。鼻涕般一個小㞠子。那婦人急得沒法。見他鼻子大得有趣。就脫了褲子。跨在他頭上。把陰門〖扌扉〗開。套在他鼻子上一陣揉。揉得那騷水長淌。一陣一陣淌在他嘴裡去。他還當是灌酒。說道。慢些慢些。我吃不得急酒。

\end{quotation}

大家又笑了一陣。菊姐道。今日是桂姐姐的壽日。你有上壽的笑話兒說一個。裘氏道。是呀。我就忘了。丫頭們。快收拾酒。晚上替桂姨上壽。常氏笑道。我倒有個上壽的笑話。說給衆位聽。

\begin{quotation}

一個公公生日。三個媳婦來上壽。大媳婦一手抱着個孫子。一手送酒來敬。公公喜道。好好。賞他一疋紳(紬)子。婆婆問說。這是怎麼個好。公公說。他是個女人。右邊抱着個兒子。女傍着個子字。是個好字。他說公公好。故此賞他。二媳婦頭上戴了個大醬篷。過來敬酒。也叫賞他一疋。婆婆又問。公公說。寶蓋頭底下着個女字。是個安字。他說公公安。故此也該賞。第三個媳婦光着下身。拿個筆帽兒揷在陰戶裡。過來上壽。公公大笑道。賞他兩疋。婆婆怒道。這叫個甚麼樣子。倒還多賞他。公公道。你不知道。一個圈兒裡頭又一個圈兒。是個回字。我時常擾他。故此多賞些。

\end{quotation}

說得衆人都笑了。芍姐道。你這會子怎說得沒力氣了。聲氣放大着些也好聽。嬌聲嫩氣的。要是聾些。還聽不見呢。常氏道。我這樣粗喉嚨大嗓子。還怕聽不明白。要是聾子。就再說高些。也是聽不見的。笑道。我倒提起個聾子的笑話兒來。

\begin{quotation}

一家的公公是個聾子。連打雷也聽不見。一日。見外邊失火。問道。媳婦。是那裡失火。那媳婦把他的屁股溝子一摸。他說。哦。是後載門。可知是那條街。媳婦拉着他的手往胯下一摸。他道。是臭水溝。不知是甚麼人家。媳婦拿手把巴子摳了一摳。送在他鼻上。他聞了一聞。道。原來是賣臭鮝魚那家人。道。他不知有甚麼壞處。就遭天火燒。媳婦伸手捏捏他的㞠子。又捏捏他的兩個卵子。他道。該燒該燒。一桿秤用兩個秤錘。這樣傷天理。還不該燒麼。

\end{quotation}

衆人正笑着。他又往外走。裘氏道。你又往那裡去。他道。我方纔吃多了些。一時屁急了。我去放了來。裘氏疑他躱懶。叫春香拉住他。道。你有屁就在這裡放。他果然放了個大響屁。衆人大笑道。這也抵得個笑話。常氏道。我又想起個放屁的笑話來了。

\begin{quotation}

一船人過渡。內中一個婦人一個和尚。那婦人偶然放了一個臭屁。衆人罵道。是那個沒廉恥的。放這樣臭屁。那婦人羞得臉脖子通紅。那和尚知道是這婦人。忙道。列位休怪。是小僧一時失錯。衆人見他承認。便道。你這和尚好不知趣。瘟臭得熏人。那婦人感激得了不得。到了岸。衆人都去了。這婦人叫住和尚。道。多謝師傅替我遮了羞。沒甚送你的。身上解下個香袋。道。這個謝師傅罷了。這和尚拿了回來。放在枕頭底下。每日早晚拿出來聞聞。叫道。心肝好香。被他徒弟聽見了。道。甚麼東西。每日心肝實(寶)貝的。那日。他師傅出門去了。他到了房中。枕頭底下一翻。是一個香袋。想道。不知是那個情人送他的。我且耍他一耍。拆開。把香料掉了。裝了一塊乾屎撅。仍舊替他放好。晚上他師傅回來。就去拿香袋一聞。道。心肝好香。再聞了一聞。有些臭氣。他笑道。心肝。你又放屁了呢。

\end{quotation}

說完了。就往外跑。衆人笑着叫丫頭們道。快拉他進來。衆丫頭也巴不得要聽。把他推推搡搡的推了進來。他道。我說了這半日。也讓我歇歇氣兒。裘氏道。也罷。你再說一個罷。常氏道。還有個和尚的笑話。也說了罷。

\begin{quotation}

一個和尚同人過渡。見那河沿上一個女人蹲着洗菜。褲子破了。把個屄全露着。那和尚道。女菩薩。你露出命來了。衆人笑道。一個屄。你怎麼叫做命。和尚道。列位在家人看見這東西不値甚麼。我小僧出家人見了就是命一樣。

\end{quotation}

芍姐笑向雪姐道。那和尚見了女人的像命一樣。你要見了那小和尚。大約也就像命了。雪姐笑道。只怕你見了。連命還不要呢。桂姨道。不要爭。此時要有一個小和尚。大約大家都是命一樣的。衆人還要他說。常氏道。有還有些。留着時常解悶。一下說完了。改日還說甚麼。這時候也晚了。夫人同衆位姨娘姐姐也該上壽去了。我也該歇歇了。蓮姨笑道。還早呢。你再說個放屁的笑話我聽。饒了你罷。裘氏道。你要有。就說一個罷。常氏道。我只說這一個的。再不說了。

\begin{quotation}

也是衆人擺渡。內中一個婊子放了個屁。衆人罵起來。一個小夥子挨着那婊子坐着。聽見是他。說道。不要\endnotemark[7]罵。是我放的。過了河。那婊子拉着他到家。說道。多謝你在衆人跟前遮了我這場羞。我沒得報你。同你弄弄罷。那小夥子巴不得。就同他弄起來。誰知這小夥子㞠子又大。本事又強。把個婊子弄得白眉瞪眼。大張着嘴。他嚇了一跳。拔出來。往外飛跑。遇見個熟人。問他道。你爲甚麼這樣慌張。那小夥子道。不好了。不好了。我把個放屁的肏死了。

\end{quotation}

衆人笑道。怪不得你放了那樣個大屁。也想人肏死你呢。衆人道。你再說一個。常氏道。我說過只說這一個的。衆人道。這是蓮姨叫你說的。我們衆人還要你說一個纔罷。衆人都站起來圍住他不放。他沒奈何。笑道。罷了。我就再說一個。

\begin{quotation}

一個人家。男人出門去了。只姑嫂兩個。東西屋子住着。這嫂子同隔壁一個男人偷上了。在板壁上挖了一個洞。約定沒人他敲小手磬。就叫那男人把㞠子打洞裡伸過來。他就着弄。也弄了多次。一日。那小姑子到他屋裡來。兩個人說笑話兒頑。嘻嘻哈哈笑了一會。那小姑子看見桌子上放個手磐。拿過來敲了兩下。隔壁那男人聽見。只當是約他。忙把㞠子伸過來。那小姑子是個女兒。從沒見過。嚇了一跳。問嫂子道。這是個甚麼東西。那嫂沒得答應。只說道。不要怕。他是來聽我說笑話的。

\end{quotation}

把個裘氏笑得了不得。衆人笑着。這個把他一掐。那個把他一擰。道。叫你說個笑話。把我們比做㞠子。他也笑着偷空跑了。裘氏同衆人到百花樓上吃了一會壽酒。長舌婦也在傍服事。衆人道。你會說笑話。必定會唱曲子。你唱個我們聽聽。長舌婦道。這我可不會。丹姨道。這除非夫人吩咐他。我們的面皮小。叫他不理。裘氏笑着道。你姨娘姐姐們旣這樣說。你就胡亂唱一個\endnotemark[8]罷。難道是求你的文麼。長舌婦笑道。我那裡會唱。我只會個倒搬槳兒。恐怕唱得不好聽。丹姨道。何如。我們叫你唱。就說不會。夫人吩咐。就說會。你揀村村的唱。唱得不好。唱一夜也不饒你。長舌婦道。姨娘姐姐要聽村的麼。有有。纔要張口。衆人道。你且吃一鍾。把喉嚨衝開了好唱。叫丫頭們倒了兩鍾給他吃了。他拍着巴掌。唱道。

\begin{quotation}

姐在房裡繡花鞋㖿。繡出幾樁故事來㖿。蔴藍簸籮裡翻針線㖿。一下翻出個大雞巴來㖿。好怪哉㖿。坐在家裡發橫財㖿。

\end{quotation}

唱的衆人都笑了。菊姐問蓮姨道。你可有發這樣橫財。蓮姨笑道。我雖沒有發這橫財。大約個個心眼兒裡都想這橫財呢。大家說笑了一會。又叫斟杯酒給長舌婦吃。他道。酒是一滴我不吃了。寧可再唱一個。饒了我罷。衆人道。也罷。你再唱。就像先那樣的就罷。要唱得不好。唱了還要吃。此時長舌婦的酒已有十分。晃晃蕩蕩的唱道。

\begin{quotation}

姐在房中把頭低㖿。自己看見自己的屄㖿。屄毛好似黃稻草。屄心好似倒冠子雞㖿。倒運的屄水〖氵韲〗〖氵韲〗瘟臭的㖿。幾時纔見那東西㖿。

\end{quotation}

衆人笑了一陣。拉住他。又灌了兩杯。他站不住。一交坐倒。吐了一大灘。大家頑笑了一會兒各散。一日。裘氏正悶坐得無聊之極。眉頭蹙着。嘆了兩聲。到堂屋中散步散悶。聽得春花秋月長舌婦三個嘻嘻哈哈頑成一處。走去一張。見他三個人都在地下滾。秋月按着長舌婦。笑說道。春姐。你在他腰裡搜。春花果然在他腰裡去搜。長舌婦兩手摀着腰。不容他搜。你道他們搜甚麼。長舌婦的男子去了幾年。他這樣個騷淫婦人可能久違此道。他想了個妙法。煩人去買了個牛尿脬來。假說要裝東西。他拿到房中\footnote{試猜買了何用。}。端像(詳)了一會。左量右量。又將下身扠了扠。量定了尺寸。拿剪刀剪開。用倒扣針兒細細縫起。縫完了。拿嘴一吹。有一圍粗細。六寸餘長。亮鋒(錚)錚不硬不軟的一根寶物\footnote{文章比角先生又深一層。}。心中大喜。根下用一根新頭繩紮緊。夜間以爲消遣之具。不用時解開頭繩放了氣。裝在腰間鈔袋內。因心愛之甚。美其名曰牛親哥。這日。他三個在後院中說閒話頑笑。春花問他道。嫂子。自從二爺去了。我們每常偶然興發。急得要死。想尋個趣人兒。又出不去。你常在外邊走動。你這東西肯閒着他麼。想是差不多被人磨出繭來了。長舌婦道。放你的屁。漢子是容易偷得的。倘偷着個像樣的。不枉捨身一場。若偷個不濟的。推又推不去。弄得又沒味。可是人說的。十個姐兒九個肯。只怕男人嘴不穩。這些沒良心的漢子。他偷了女人。以爲得意。那裡還顧人羞恥。四處倡揚。實在受用不曾得。只添了一個醜名兒。秋月道。單是養漢的人會揪(撇)淸。那日我見你同二爺弄着。叫到靑天雲裡去。那個浪樣子。連我看着都肉麻得了不得。砢嘇死了。你是忍得住不偷漢子的。我是說實話。只是夫人不許我們出去。若是容我。我偷個樣兒給你看着。管他大小。強如沒有。不要說怕倡揚得人知道。那怕他九門上掛了榜。還不在我心上呢。長舌婦笑道。沒臉的騷奴。就這樣騷發。實不瞞你。我有一個牛親哥同我做伴兒。不然如何過得。秋月道。這娃(姓)牛的是個甚麼人。我們這樣大門第。他怎麼進得來。長舌婦笑道。牛親哥在我腰裡帶着。那裡是甚麼人。春花道。大約是你說謊。我就不信。果然是甚麼東西。給我們看看纔是眞。常氏笑着向腰間取出來。吹脹了。捏在手中。道。你們看這牛親哥可好。春花見了。劈手就搶。長舌婦忙一下捏扁了。裝入鈔袋內。秋月道。你這樣沒廉恥的。你他(也)受用夠了。就借我們用用何妨。那裡就弄壞了你的。長舌婦道。甚麼話。他就是我漢子一樣。難道我的漢子也肯讓你們麼。春花道。秋姐。不要同他好請。我按倒了你搜。長舌婦纔要爬起來跑。被秋月一下按倒。春花就去搜。長舌婦又不放手。故此笑滾在一處。裘氏見他們這樣頑法。不知是做甚麼。就走到跟前。他三人見夫人來了。纔放了手站起來。裘氏問道。你們三個在這裡做甚麼。滾在一處。春花指着長舌婦道。他腰裡帶着個牛親哥。我們要看。他不肯。故此在這裡奪他的。裘氏不懂。問長舌婦道。牛親哥是個甚麼東西。長舌婦笑道。夫人不要聽他嚼蛆。那裡有甚麼牛親哥。秋月道。你在夫人跟前還敢說謊。他先拿出來。我們都看過了。這會兒又說沒有。裘氏笑着道。你兩個搜出他的來看。春花就一把抱住。秋月就向腰間去搜。長舌婦因夫人吩咐。不敢強。被他在鈔袋內搜了出來。遞與裘氏。裘氏見是尿脬縫的個扁東西。不認得是甚麼。說道。這是做甚麼用的。怎麼叫做牛親哥。春花道。我吹給夫人看。接過來吹脹了。捏着根下硬邦邦的。笑道。這是他的漢子。因是牛尿脬做的。故此叫做牛親哥。裘氏笑得眼睛一縫\footnote{喜極之態。}。伸手取過來。氣一放。又扁了。裘氏也用口一吹。脹了。捏着笑道。拿來入官。遂捏着走回房中。收在褥子底下。過了一會。長舌婦進來。裘氏笑着問他用法。他知夫人要試驗了。說用頭繩將根紮住便不癟。或用手持出進。或是紮在枕頭上騎在上面。自己抽動亦妙\footnote{補他自用時所無。}。裘氏點頭會意。晚間如法作用。正是。

\begin{quotation}

嬌兒一去歸何日。且把牛哥暫解饞。

\end{quotation}

弄了一會。雖覺有趣。全要自己費力。不能遂心。用過幾次。也就覺無味。時時刻刻想那孝順兒子。再說那人(幾)個妾中。惟獨菊姐年小。偏他更是騷浪。姚澤民在家時。也同他弄的次數多。姚澤民去後。別人雖想。還強自排解。惟獨他茶裡飯裡。睡中夢中。無一刻釋懷。眼淚不知流了多少。竟有個淚盡繼血的光景。過了些時。茶飯都減。懨懨成病。眞是。

\begin{quotation}

憔悴了含宿雨梨花貌。瘦損了舞東風楊柳腰。

\end{quotation}

裘氏一日走去看他。問道。菊姐。你是怎麼樣的了。菊姐也不答應。只長吁了一聲。眼淚滿面。裘氏道。你不過是想他二爺。但知那寃家他幾時纔回來。你這樣癡癡的想。豈不送了性命。只好自解自嘆些罷了。說到這裡。由不得也掉下淚來。這是。

\begin{quotation}

愁人莫對愁人說。惹得愁人展轉愁。

\end{quotation}

菊姐愈覺傷悲。說道。夫人。我想還是小事。我夜夜夢見他來同我睡覺。及至醒來。還是孤衾獨自。因此越覺傷心。裘氏道。這是你心想邪了。自己秉正着些方好。我看你這病。大約合了牡丹亭上的一句了。陳最良對春花說。小姐這病是詩經上起的。還用詩經去治。經上說。旣見君子。云胡不瘳。小姐這病。得抽一抽就好了。你這病也得抽一抽纔得好呢。菊姐也破涕成笑。道。尋這個君子就難起。裘氏也笑道。如今世上眞君子原難得。我有個姓牛的假君子。拿來給你抽一抽罷。又笑向桂姨道。還得你替他醫治呢。裘氏回去。叫長舌婦將牛親哥送與菊姐。並授他所用之方。因他病弱。自己不能動。叫桂姨替他作用。他原是心想成病。古人說。心病還須心藥醫。況他的病乃淫也。非情也。得了牛親哥相伴。悶來就拿他消遣。心開了。病也漸癒\footnote{八人中獨寫菊姐如此者。百花皆畏日曝。烈日中無有不薦(蔫)者。惟菊越經日色愈鮮艷。黃者更黃。紅者更紅。故寫他更愛日耳。}。那日到裘氏處來道謝。裘氏笑向長舌婦道。不想你的牛親哥竟會行醫。把菊姐的病竟醫好了。長舌婦道。原有個笑話兒。

\begin{quotation}

一個人的㞠子太軟。到賣春藥鋪子裡去買藥。那賣藥的敎他把藥搽上。說道。你不用到家。他就會\endnotemark[9]硬起來了。那人忙往家走。離家尚遠。㞠子十分硬脹。他一把攥住。贊道。好郞中。好郞中。

\end{quotation}

這牛親哥原都是會行醫的。大家笑了一場散了。這裘氏日間叫人說村淫不堪的笑話。以爲歡樂。大家嘻嘻哈哈的笑着。倒也混過去了。夜間想起那些淫話來。越發一刻也睡不着。每夜無眠。日裡精神倦怠。眉頭緊鎖。短嘆長吁。一日。長舌婦在傍勸道。夫人靑春年少。正好享福。何苦自己熬煎。二爺一年半載自然回來。夫人可耐心些。不要憂戀。壞了身子。那裘氏忍不住墮淚。道。你是我心腹人。你叫我這孤悽如何受得。忽嘆了一口氣。道。倒是你二奶奶好。他丈夫去了。毫不在心。我見他比當日更歡歡喜喜的。我學不來。奈何。長舌婦鼻中冷笑道。二奶奶麼。他有。連忙住口\footnote{寫得情理入神。}。裘氏道。你這老婆有話怎不說完。只說半截。他有甚麼。長舌婦道。這話有干係的。所以不敢亂說。裘氏道。呆老婆。你對我說。怕甚麼。他走近前。低聲道。二奶奶有我們家供養的大師傅同他作伴。他還想二爺做甚麼。裘氏蹬了一蹬。道。眞有這些事麼。長舌婦道。我不眼見。怎敢亂說。我見的多次了。我但是回去得遲些。黑影子裡常瞥見素馨同着大師傅進二奶奶房裡去。裘氏道。他是個大和尚。也幹這樣的事。長舌婦笑道。單是大和尚纔肯幹呢。裘氏想了一想。道。你今晚留心去打聽。須看得實了。快來回我。長舌婦答應。到落日之後。他折(打)聽去了。裘氏叫了八個妾來。笑道。你們可知道一件笑話。衆人道。不知是甚事。裘氏道。方纔常老婆說。二娘子養着我們家供養的大和尚。我還疑心不信。他說得千眞萬確。我叫他打聽去了。若果有這事。我們普同供養着的。爲何只他一個人占了去取樂。我們同去〔叫〕那禿驢來。叫他拿小和尚供養我們。省得獨守孤幃。睡夢不安的。你們心下何如。那些衆人一個個的笑逐顏開的道。夫人的高見可有錯的。這是極美的事。我們敢不跟着做。裘氏大喜。遂把十個丫頭也叫齊了。專等長舌婦的回信。大家吃着酒說笑。到了一更將盡。只見長舌婦笑嘻嘻的來了。裘氏問道。打聽得怎麼樣了。他道。等到這昝晚。纔見素馨\endnotemark[10]同他進去了。關了門。我纔來回話。裘氏站起。道。多點上幾個燈籠。我們大家同去。丫頭們點了五六個燈籠。一齊就走。裘氏叫長舌婦道。你先去敲門。他要問做甚麼。你說我得了急症將危。叫請二奶奶快來。又吩咐道。丫頭們把燈籠拿袖子蓋住。不要露出光亮來。等他一開了門。然後一擁進去。到他房中。就做手脚不急(及)了\footnote{觀裘氏。豈非一個聰明女子。古云。盜亦有道。婦人偷漢亦有一番機智。}。長舌婦應諾。先去敲門。敲了幾下。聽得素馨問道。三更半夜。是誰敲門打戶的。長舌婦道。夫人得了暴病。十分危急。衆姨娘姐姐叫我來請二〈位〉奶奶。大奶奶已先去了。快些開門。素馨到房中向桂氏說了。桂氏向萬緣道。我不得不去。等夫人略好些。我就回來。叫素馨靑梅跟我去。留香兒綠蕚陪你。遂拉過被來。將他連頭上下蓋好。在床裡起來。一面穿着衣服。對素馨〔道。〕你去開門叫他進來。我問他是怎樣的來。素馨走出去。纔把門一開。忽見五六個燈籠一亮。夫人在前。八個妾在後。一羣丫頭圍繞着。驚得魂飛魄散。轉身跑。口中不住的大叫道。奶奶。夫人來了。桂氏聽得。也魂不附體。衣裳還不曾穿完。裘氏同衆人已到房中。燈光照料如同白晝。房裡擠得滿滿的人。桂氏嚇得面色如土。脚也挪不動。話也說不出。睜着兩眼望着裘氏。見他雖是一臉笑容。由不得心中亂跳。裘氏就坐在床上。一眼見床裡圓滾滾。一床被蓋着。上去將被一揭。見一個雪亮的光頭\footnote{不知是大頭是小頭。}。知是那禿驢了。叫衆丫頭道。你們來把這被好好的替我擡了上去。幾個妾忙接過燈籠\footnote{細。}。衆丫頭都心照。上前七手八脚。抱頭的抱頭。抱脚的抱脚。也有幫在中間的。大家擡着。轟的一聲去了。只有蓮姨菊姐拿着兩個燈籠。同裘氏還在房中。那桂氏還癡呵呵的站着。裘氏上前拉住他的手。道。你不要怕。風流事婦女們誰人不做。我肯來拿你的姦麼。只怪你偏我獨享。且拿他去同我們大家做個喜樂會場再還你。桂氏纔放了心。雖然捨不得。也沒奈何了。只得答道。我不敢叫他去服事夫人。夫人若愛他。我敢不讓麼。那裘氏笑着。也忙忙去了。桂氏送到門口回來。素馨道。哎喲。我的膽子都嚇碎了。桂氏道。他怎得知道的。素馨道。有一夜。我同大師傅來。黑影裡影影見一個人。雖辨不出模樣。那身段活像長舌婦。今晚又是他來叫門。定是這淫婦搬的舌。桂氏道。我先怕他來拿姦。嚇了我一跳。要是這樣拿了去。倒也還罷了。只怕這和尚被這些騷貨要弄死了呢。素馨道。那個奶奶倒不用替他躭憂。他一個不抵二爺兩三個麼。二爺還不曾弄壞。何況於他。桂氏道。就算不放。我們再要同他常常歡會。料不能了。說罷偢(愀)然。素馨道。去了一個。還有二個呢。奶奶不要煩惱。香兒。我同你叫盛旺去。香兒同他去了一會。同盛旺進來。素馨向他道。奶奶今日心裡有些不受用。你用些力。同奶奶作樂。盛旺連忙將桂氏抱到床上。替他脫了。自己也脫下。受了素馨的指敎。加力服事了半夜。桂氏方有些喜色。不必多說。衆丫頭將萬緣擡着。如同楊貴妃用大襁褓兜着安祿山洗三的樣子。一直擡到裘氏床中放下。先那萬緣也嚇了個半死。聽見擡到夫人的床上。知道不但無禍。而且有喜的了。不過是要賞鑒小光頭之意。纔定了心。將陽物攥着。暗囑道。徒弟。你須鼓起威風。替我爭氣要緊呢。正想着。聽得一羣婦人嘻嘻哈哈的說笑。少刻。又得那夫人嬌聲嬌氣的吩咐道。趕着收拾酒果在百花樓上去。可鋪一個大鋪。你們都同到那裡去會新人。又聽見衆人道。收拾還有一會。夫人且請先享用享用着。聽得那夫人笑嘻嘻的走到床前。上床來把被掀開。道。不要悶壞了。你出來罷。萬緣見左右並無一人。數枝燭光火亮。照見夫人。必(比)桂氏還嬌美。一把抱住。道。貧僧何福。蒙夫人如此大發慈悲。遂要替他寬衣。裘氏笑道。不脫罷。還要往百花樓上去呢。萬緣只將他褲子脫下。爬上身。捏着陽物。往陰中就頂。裘氏的此竅甚覺緊澀。萬緣頂了兩下。不能入去。他愛如至寶。縮下身子。用舌頭將唾津把陰門亂舔。裘氏拉他。道。你一個唪經念佛的嘴。不當家花花的。怎麼舔這腌臢東西。他笑道。怕甚麼。過後漱漱口就乾淨了。那個佛菩薩不從此中出來。道士吃了狗肉還不念天尊\footnote{道士雖吃狗肉麼。未必吃狗屄。}。何況夫人的這香美潔淨妙物。那裡肯起來。舔得興足了。然後上來。一頂而入。抽了十數抽。方纔盡根。他要顯本事。一上手千餘抽不止。一下重似一下。裘氏被他弄得有無窮之樂。口內的嬌聲令人聽得魂消。他丢了數次。說道。人多呢。你留些精神打發衆人。且起來着\footnote{不意此淫婦人竟有大公無我之心。較只知有己之輩猶勝也。}。萬緣也就歇手。裘氏坐起穿褲。想起和尚的衣褲還在桂氏處不曾拿來\footnote{細極。}。叫丫頭開箱櫃。將姚華胄的衣服鞋襪取出來。叫和尚取(穿)了\footnote{姚華胄此時不知可耳熱眼跳否。}。裘氏也穿好。丫頭執燭前導。他二人攜手同出房來。先他二人高興之時。衆妾都在窗下覷聽。看見和尚這場潑戰。喜得非常。互相稱賀得人。見他兩個出來。一擁着同到百花樓上。一張大花梨圓桌已列着美酒佳肴。十個人團圓坐下。有四句話說這衆婦。說道。

\begin{quotation}

只爲貪淫一念。化成百計千方。

同去陪僧閣上。大暨兀該會場。

\end{quotation}

和尚坐下。舉目細細一看。夫人之外。那八位美人雖然不及夫人之嬌麗。也都有六七分姿色。可與桂氏伯仲。喜得心窩亂癢。又見那樓上的擺設鋪陳。眞是富貴氣象。

\begin{quotation}

紫檀桌上。玻璃瓶揷着珊瑚樹。螺甸盤中。宣德爐焚着龍腦香。象牙床。金鈎控着錦帳。沈香几。玉硯傍着牙籤。寶鼎中。香氣氤氳。珠燈內。焰光璀璨。席間器皿盡是精金。座上佳人皆同美玉。不想這閨閣中窃(窈)窕嬌娃。盡化做繡榻上施屄菩薩。

\end{quotation}

又見樓板上鋪開一個大鋪。知道是要做聯床大會了。正顧盼着。裘氏笑吟吟拿起酒杯。向他道。你費了力了。且吃一杯酬酬勞着\footnote{不勝肉麻之至。}。

\begin{quotation}

這纔是猛和尚片刻恩情。勝似那姚華胄多年恩愛。

\end{quotation}

和尚忙合掌道。阿彌陀佛。貧僧蒙夫人同衆位奶奶垂靑。死亦弗辭。敢說費力。衆人都輪番交敬。這和尚是無量不濟的。飮了一會。裘氏笑道。我是偏過你們了。你姐妹們怎麼個來法。衆人道。憑在夫人吩咐。裘氏道。這要取個公平。纔沒爭講。叫取過骰盆來。他撚起兩個骰子。說道。先用兩個擲。照到誰便是誰起。後四個用一個骰子擲。這就算公道了。除了我數。將骰子擲下去。數到該雪姐。裘氏道。你去。衆人中算他年幼。還有三分羞澀之態。笑嘻嘻的不動。裘氏向萬緣道。你不動手。還等人去替他脫麼。那萬緣得不的一聲。先自脫光。衆人先去裘氏窗下。那是遠觀還不覺。此時覿面近看。好件粗大傢伙。怎見得。

\begin{quotation}

紫漒光鮮。靑筋疊暴。纍舉偉長。昻然跳躍。

\end{quotation}

比姚澤民的粗大許多。各各心中暗喜。萬緣將雪姐抱到鋪上。替他解褲卸裩。見他身材小巧。不敢唐突。輕輕款款。抽不上數百。他已嬌聲吿止。裘氏又擲。數着了丹姨。他是第一個浪騷的。連忙自己解衣。就到鋪上脫光睡倒。萬緣將陽物湊着牝戶。已淫水滂流。只一送。便進去了。萬緣見他是個敵手。用力搗了無數。他丢了幾次。尚然不放。裘氏道。夜很短。你還讓讓別人呢。拿起骰子便擲。該是蓮姨。他等得心中正火冒。走上去。將萬緣在丹姨肚子上生拉了下來。他忙睡倒。兩個就弄。丹姨一面揩着陰戶。道。蓮姐姐。你就這麼性急。不害硶麼。蓮姨笑道。我再不硶。你大約獨占到明日天亮了。弄了一會。裘氏又擲點到菊姐\footnote{此處亦是順時序而來。初雪姐。冬也。次丹姨。春也。又次蓮姨來。夏也。終於菊姐。秋也。與前遙遙一照應。}。過了。一個個點到去弄。直到東方將明。八個人纔完了。萬緣看那裘氏不住在床上翻來覆去的扭。知他興尚未足。又上床同他弄了一陣。日映紗窗。方摟抱而睡。衆人辛苦了一夜。都睡到日午方醒。纔起來梳洗吃飯。裘氏同衆妾留住這和尚。那裡還肯放他出去。萬緣稍有餘空。這十個丫頭同長舌婦都攢着他。求他那一點菩薩甘露。以洗衆人淫焰。萬緣見這些女子都還風騷可愛。也俱點綴了點綴。一日。裘氏同衆妾擁着萬緣嘻笑共飮。裘氏笑向他道。我素常聽得老爺說你是一個大和尚。經典詩詞。件件都會。你把今日的事。不拘詩詞偈語。作一個大家聽聽。萬緣道。我是個淫僧。並不是詩僧。那裡作得來。裘氏道。不過作幾句大家頑笑。我們那一個是通的。怕笑你麼。萬緣笑道。阿彌陀佛。你們列位。打屁股底下一個眼兒。直透頂門。那一個不通。裘氏笑着擰了他一把。道。不要嚼蛆了。快些作罷。萬緣想了一想。道。不要見笑。我謅了八句。實道其事了。

\begin{quotation}

懶去看經怕坐禪。但知樂處即西天。

\end{quotation}

因把裘氏一摟。道。

\begin{quotation}

夫人任我隨心摟。

\end{quotation}

又笑指着衆妾道。

\begin{quotation}

衆美憑予着意牽。

\end{quotation}

又摟過裘氏親了個嘴。笑〔指〕着衆妾道。

\begin{quotation}

悶至相攜花底坐。興來疊股象床眠。

\end{quotation}

復哈哈大笑道。

\begin{quotation}

披毛戴角隨他去。一聽閻羅罪萬千。

\end{quotation}

裘氏笑道。你旣會作詩。再村村的作幾句偈語。要惹得人笑纔罷。不然我們每人罰你一碗。萬緣笑道。你們這些惡人。旣要我臉(腰)間費力。又要我心裡費思。這是何苦也。罷了。道。難不住我。又想了一想。道。你們大衆聽着。

\begin{quotation}

我到這花叢下榻。遇着你這些施屄菩薩。人人皆想興闌。個個都思樂殺。老僧一個怎支持。除非向傲來國花果山水簾洞孫行者處学會了分身妙法。咦陽物變做金箍棒。把你們這些陷空山無底洞全部搗搨(塌)。

\end{quotation}

說罷。衆婦人大笑了一場。一齊把賊禿灌了個酩酊大醉。他乘着酒興。將裘氏按倒。就拉褲子。裘氏也正興動。任他脫去。雙鳧肩飛。弄將起來。裘氏矇矓着惺眼。顫着聲兒說道。我的這件東西。被你那小禿驢橫舂豎搗。這樣作踐。你這大禿驢就不贊美他幾句。安慰安慰他。萬緣笑道。容易容易。一面抽着。一面唱一個駐雲風。(飛。)道。

\begin{quotation}

妙竅尖圓。緊暖香乾軟賽錦。邊似蓮花瓣。心裡雞現冠。嗏萬卉總鮮妍。何如斯艷。出進怡然。樂得你燕語鶯聲顫。說甚麼瑤島蓬萊自在仙。

\end{quotation}

唱完了。一陣亂搗。搗得〈你〉裘氏哼聲雜着笑聲。衆人看他兩個好一番做作也。

\begin{quotation}

牙床兩共寢。羅衾內。摟抱互綢繆。似戲水鴛鴦。穿花蛺蝶。並肩交股。同效鸞儔。對銀燭。酥胸觀嫩乳。玉杵搗紅溝。芳舌吐香。粉腮微暈。細腰款擺。尖指頻勾。玉軀搖還扭。嬌聲戰篤。續逞盡風流。偏喜破唇微笑。惺眼停眸。更消魂妙態。花心輕點。兩臀緊疊。眉鎖如愁。情到不能言處。雲雨同收。

\begin{flushright}右調風流子\footnote{妙甚。句句是說裘氏。却句句是衆人眼中看出。身歷其境者反不知也。}\end{flushright}

\end{quotation}

他衆人也歡樂了二十多日。萬緣也有些應付不來了。想道。婦人雖然可愛。性命也是要緊。我一個人。如何纏得過這二十多個狐狸精來。我如今要辭了去。他們決定不肯。須尋個幫手來方可。因想到那道士身上。道。他每常講得此道中津津有味。這些騷貨。除非得他來。纔可征服他們。況且我承二奶奶相愛之情。久疏了他。心中也過不去。若弄了這老道來伴着他們。我或可脫身。同他敍敍舊情。遂向裘氏同衆人道。我承夫人同衆位相愛。但我一個人。不足以供衆位之慾。我有一個道友。是寰中少有。世上無雙的本事。遂將他如何採戰婦人。如何受用。細述一番。衆人聽得慾火直冒。說道。我們不信天下有這樣奇人。這是你要想脫身。放了你去好躱不來。萬緣道。阿彌陀佛。貧僧出家人。怎敢打誑語。我承衆位的美情。可敢負心。這是我將他答報衆位恩德的好意。怎倒疑心起我來。若放我回寺去。今晚不同他來。明朝必到。裘氏向衆位道。人心是肉〔做〕的。你們想。我們的身子都捨與他受用。難道他就這樣沒情。他旣如此說。未必是假。叫人到桂氏處取了他的僧衣來換了\footnote{處處細心照應。一絲不肯漏過。}。裘氏叮囑道。那道士來不來憑他。你是必要來的。不要沒良心。負了我們。萬緣道。蒙夫人衆位這樣布施看顧。貧僧韋馱菩薩是證明。我貧僧若負了衆位。來世變豬變狗\footnote{來世變豬變狗。不如今生做驢。}。還想得人身麼。裘氏叫長舌婦送他出去。到了窗門外。萬緣道。大嫂。你請回罷。我還看看二奶奶去。長舌婦也就去了。萬緣到桂氏處來。桂氏見要和尚衣服帽去。知他必到。正在望他。一見。如同天上落下來一般。忙起身兩手拉住。道。你去了這些時。我怕淘碌壞了你。把我裡病都想出來了。你同那些妖精快樂。心上可還有我麼。萬緣就親了個嘴。搵着他的香腮。道。你那裡知道我的苦心。眞是身在吳廷心在越。我雖身子同他們頑耍。心裡那一刻放下你來。我恐盼壞了你。故此想尋個幫手來。遂將尋道士的話向他說了。道。若得他來。我就可脫身。常同你取樂了。桂氏摟着他。親親的道。你有這樣好心。不枉我捨身與你。萬緣知他這些時等苦了\footnote{雖不甚甜。還不至於苦。}。同他上床痛幹了一番。穿衣要去。桂氏道。你要約了道士來。先到我這裡。等我看看是甚麼個甚(異)人。萬緣笑道。豈但給你看看。必定先還叫你嘗嘗。我纔同他上去呢。桂氏笑了笑。那萬緣去了。回到寺中。衆徒弟問道。師傅從來不曾去許久。我們又不敢去問。擔心得了不得。因附在耳〈上〉朶上低聲道。把兩位師娘急得每日叫我們去求籤打卦。都說是有陰人纏繞住了\footnote{好靈卦。}。端的是師傅在那裡做甚麼來。萬緣道。我承他家供養多年。無可報答。要註釋一部經\footnote{不知可是嫖經。}。替他祈福。保佑他父子在外平安。家中人口淸吉。纔註起頭。因記星(罣)家裡。回來看看。再要去。容易不得回來。你們好生看家。說罷。到密室裡去。同兩個禿眷作別。只見兩個婦人。頭髮蓬鬆着。因問道。你們怎麼頭也不梳一梳。恁個樣子。二人答道。久不見你回來了。病都急出來了。還有甚麼心腸梳洗。萬緣先拉過一個。扯了褲子就弄。內中黏達達的。勉強弄了一度。再弄那一個。也是如此。萬緣已明內中之故。草草了事而已。你道這是何故。這萬緣大大小小有數十個徒弟。都是那些愚人。聽說他是個有德行的大和尚。眞是現在的活佛。皆妄想着一子成佛。九祖昇天的話。把好好的兒子都送來給他做徒弟。那知他〔是〕淫念極重。水旱齊行的惡物。徒弟中不管年長長(年)幼。或醜或俊。個個不饒。都要嘗嘗他臟頭的滋味。他又好弄蔬屁股。此窟如何分得葷蔬。這是他創的一番新論。若是不用唾沫乾弄便是蔬的。用唾便謂之曰開葷。這徒弟們常常被他蔬弄。內中有一個小徒弟。纔得十二三歲。那日被他蔬弄得十分難禁。大哭着叫道。師父。熬不得了。求你開了葷罷。衆人聽見。互相傳爲笑談。一日。他同衆徒弟在後園中吃酒。有幾分醉意。拿着衆徒弟蔬弄。這個抽\endnotemark[11]幾抽。那個搗幾搗。他酒後興豪。陽物分外雄壯。衆人見他醉了。不敢拗強。都咬牙捱受。正然弄着。萬緣忽然要大解。走到竹林中。蹲了下去。他醉眼模糊。不妨一根竹笋。其利如鎗。剛剛戳着他糞門。進去了數寸。那笋尖戳得生疼。大聲喊叫。衆徒弟含笑接耳低聲道。阿彌陀佛。肏蔬屁股的現報了。他看見大怒。罵道。這起小禿驢。見我被戳。不來扶我。你們笑的是甚麼。衆人見他發怒。上前扶起他來。哎喲不住聲。扶入淨室。這些徒弟都受過他的鎗。又恨他。又怕他。後來又見他拐了兩個婆娘。藏在密室。衆人眼中冒火。但見他往姚府去。便有幾夜不歸。遂去調戲他這兩個婦人。這婦人正恨萬緣常不在家過夜。見衆弟子來仰攀。他兩人也便俯就。他(但)是萬緣出門。他們夜間吃醉了。幾個淫禿兩個淫婦便滾做一床。做個亂點鴛鴦譜。這次見萬緣去了多日。以爲他〈來〉未必就回。大膽打個白仗。恰巧他撞了來家。衆人雖罷戰休兵。那二婦牝中如何一時得淨。萬緣明知是衆徒弟替他代勞。他因有了這些美人。這兩個陋婦也就直(置)之度外。讓衆徒弟們做個替身罷了。萬緣出來。就到那道士房中相會。坐下。說了一會閒話。見無人在傍。遂透一句。道。道兄這些時可曾遇着個好鼎器麼。道士笑道。這事不過是機緣湊巧。不是可以強求得的。良家婦女是不敢去淫汚他\footnote{有此一語。見得道士之罪可恕。}。至於娼妓。他內中蘊了毒。是不敢採取他的。那裡有這樣便宜的物件。萬緣笑道。倒有一處有許多。貧僧要薦了道兄去。道兄可有此興麼。道士道。請道其詳。萬緣遂挪過座兒。同他相近。附在兒(耳)上。將裘氏衆人的事相吿。又道。這羣婦女雖係良家。行同淫妓。奸他也不足爲罪。貧僧素守戒律。一個老實和尚。生生被他騙去強奸了\footnote{若以實情論之。桂氏裘氏却(確)是他二人先奸和尚。}。破了我的戒行。他旣可以奸得貧僧。道兄也就可以奸得他了。道士笑道。師兄被這些婦人強奸的話。貧僧也不敢深信。但請問貴檀越喬梓做人如何。要是盛德之人。這閨門便不可汚穢他的了\footnote{此語乃爲道士出罪者。}。萬緣道。那老檀越年已古稀。弄這些少艾在眼前。也就是作孽了。小檀越那不用講。他把庶母烝淫猶其次。連繼母都偷上了。罪當何如。因他同這些婦人作樂。撇了己妻。那二奶奶纔尋了貧僧去做伴。他父子都往廣西去了。後來被夫人知道。又把貧僧拿了去強奸。道兄請想。這種婦人還不該淫他一淫麼。道士笑道。據師兄這樣說。這等婦女無恥貪淫。淫他也不爲大過。據貧道看來。想是人衆了。師兄孤立無緩(援)。要貧道做個救兵之意。萬緣大笑道。道兄洞鑒肺腑。此時容或有之。倘不吝駕。何不此時就行。道士首肯。萬緣叫了徒弟們來。吩咐道。我約這位道兄同去講解經義。恐一時不得回來。你們將他行囊般(搬)到我屋裡去。衆徒弟應諾。他二人攜手同行到姚家來。管門人見了那道士。因萬緣是主人供養的活佛。只說是同來的眞仙。可敢盤問。到了佛堂。開門進去。時已天暮。萬緣在佛前琉璃內取灼了火點上燈。不住到門口張望。恰好素馨出來探信。他道。那道士來了。你去對奶奶說。等人靜了。你來接我們進去。素馨喜孜孜。忙跑到桂氏跟前。道。大師傅同道士來了。說等人靜。叫我去接他們。桂氏喜得心忙意亂。說道。那裡等得人。且快收拾碟子吃酒。今日大相公身上不好。不過來的\footnote{此句不補亦可。補則更妙。}。你就去請他兩個來。且吃着酒。再預備飯。叫丫頭擦抹桌椅鮮明。他自己忙把陰戶洗了洗。剛收拾完。那和尚同道士已到房中。萬緣向道士道。這一位就是貧僧所說的二奶奶了。極是多情多義的。道士向前一揖。桂氏抿嘴微笑。還了一福。不便開口\footnote{四字極妙。他雖淫濫。到底是良家婦人。愧心未死。若再讓坐寒溫。便是妓女腔調矣。}。倒是和尚替他讓坐。道士在東。和尚在西對坐。桂氏面北打橫。不一時。丫頭掇上菜碟來。斟上酒。桂氏初會生人。自然裝出些羞慚的樣子\footnote{裝子(字)刻毒。}。舉起杯來。微微笑着。看那和尚萬緣拿出野老公身分。讓道士飮過數杯。桂氏三杯落肚。把那羞趕到爪哇國去了。鍚瞪瞪兩隻眼睛\footnote{淫態。}。看你道士好個相貌。雖然長髯白了。雙眸炯炯。一面似幼童。又飮了幾杯。桂氏縛不住心猿。望着道士只是笑。道士見他這騷致撩〔人。〕也微笑相答。和尚知機。見桂氏有些火動了。假道。我且失陪道兄。便一便來。起身走出。將門帶上。那道士知他放路。笑向桂氏道。這位師兄約了貧道來奉陪。奶奶可肯俯就麼。桂氏也不答應。笑着走到床上坐下。道士也就跟到床上。替他脫裩睡下。道士寬了大衣。褪褲取出孽具。弄了進去。桂氏覺得還不如姚澤民的大。心疑道。這個匪物怎和尚那樣誇獎。正在躊躇。不多時。漸漸脹滿。熱而且堅。在內中咬將起來。始信所言不謬。粗長雖然與和尚相等。但他的活泛。樂得並無二辭。連聲贊道。活寶貝。活寶貝。頃刻間。採丢了一放(次)。道士見他淫興正濃。又採了一陣。他又丢了。桂氏摟住不放。還有求閏之意。道士笑道。使不得。我這東西不同他人。與婦人交媾。陰精全吸了的。因你從未經此。故敢行二次。若是長弄一次之後。必須養息六七日纔可。不然定要生病。這儘夠了。你不信。等我拔出來。你看陰中可有流出來的餘瀝麼。那桂氏也算幸遇了。依他放手。那道士拔出陽物。桂氏摸摸陰戶。不像每常那樣黏黏涎涎齷齪。方信其言是實。穿衣下床。桂氏開了門叫丫頭。原來他們四個同和尚正在那屋裡弄。聽得叫。都走了來。和尚看着桂氏嘻嘻的笑。桂氏也望着他笑。向丫頭道。拿水來洗手。快送飯來。丫頭們送上水。二人洗了手。已將肴飯擺下。又用了幾杯酒。同把飯吃了。三人坐了吃了一會茶。道士道。師兄在此。貧道還出去罷。和尚道。道兄就在此下榻罷了。爲何又要出去呢。道士道。貧道在此也沒用。倒是師兄在此奉陪奶奶罷。桂氏知他是弄不得的話。便道。師傅不要出去。屈你在西屋安歇一夜罷。叫這幾個丫頭奉陪。吩咐丫頭將錦衾繡褥拿去鋪上。叫點燈親送道士到那邊屋裡。看他睡了。然後同和尚過來。那素馨四個見桂氏去了。他們一齊脫光。擁到床上。那道士也就笑納。每人採了兩次。見香兒壯實。雖不及那黑姑子的精盛。也要在二等數內。多採了一回。不必煩說。那和尚同桂氏上床。抱着問道。他的本事何如。桂氏道。大小與你一般。只多了會咬咂。咬得裡面。癢到心窩裡去。每常你弄得我丢時。渾身一酥。他弄得丢時。個個骨縫都開。竟像癱化了的。萬緣道。這樣說。他比我強多。你自然愛他。我竟不足取了\footnote{大有醋意。}。桂氏摟着他道。因他(你)是我的知心。我纔實話吿訴你。你怎倒疑我。他說弄過一次。定要歇六七日纔弄得。親親。又不若同你每日弄的強了。是古人說的。他如精金美玉。可有可無之物。你如五穀糧米。可是人家一日缺少得的\footnote{善爲說辭。}。萬緣見他這等相愛。足同他盤桓了半夜。直到桂氏動不得了。纔相抱而臥。次早黎明。萬緣就起來。道。恐遲了。有人走動。趁早晨。我同道兄上去。因向桂氏道。你不要嫩了。過兩日。你也竟上去同他們滾在一處。且尋歡樂。你這裡只好頑耍。日裡恐有人來往。倒不如他上邊淸淨。可以日夜行樂。叫做大樹底下好遮陰。桂氏被他提醒。滿口答應。遂一齊同過去。看道士時也起來了。桂氏叫香兒看上邊開了門沒有。少刻來道。纔開呢。桂氏又叫他送和尚道士到總門口回來。那和尚路熟。攜着道士到裘士(氏)臥房來。已被秋月看見。一臉的笑。忙去報知裘氏。裘氏昨夜見和尚不回。正在疑慮。忽聽得說同道士來了。這一喜。如天上落下個異寶來一般。他此時尚在被窩中。只見和尚道士一同進來。和尚見他還未起。向道士道。這就是夫人。道兄就請托契些罷。拉他到床前。抽身出去。拉着春花秋月同到窗下張看。只見那道士脫了衣服上床。將裘氏的腿推起。弄上了。伏着不動。少刻間。只見裘氏渾身亂扭。口內哼聲不絕。

\begin{quotation}

一個是紅顏少婦。渴想奇人。一個是白髮黃冠。深知異術。搵香腮。喚幾聲妙人兒。恣情採戰。摟楚腰。應幾句親師傅。着意抽添。看不盡繡衾中鳳舞鸞狂。早見那玉人兒魂消骨醉\footnote{此是萬緣眼中看道士裘氏。}。

\end{quotation}

萬緣看上興來。將秋月後邊褲子扯下。做個隔山取火。一面看一面抽。抽了一會。那春花急道。也該輪到我了。你儘着搗麼。秋月回顧和尚道。好師傅。不要理他。再來來着。那萬緣依他。只是弄。春花一把抱着和尚的腰往後扯。秋月也將屁股就了來。萬緣見他騷到極處。着實搗了一陣。拔出來。撤開春花的衣服。他早已將褲子褪了。一個光屁股。陰戶騷水淋漓。萬緣也加勁力搗。那道士將裘氏採了一次。纔細看他的嬌容。掀開被。賞鑒他的嫩體。果然好個十全的婦人。怎見得。

\begin{quotation}

髮如黑漆生光。面似海棠舒媚。兩葉淸眉吐秀。一雙嬌眼含春。十指纖纖。雙鳧窄窄。體似羊脂。遍身無一點瑕玷。陰如包蕊。牝峰有數莖氄毛。說不盡千般妖冶。形不足萬種風流。

\end{quotation}

道士心愛無比。又採了一回。萬緣見那裘氏四肢癱在褥子上。眼睛閉着。口內微有哼聲。他看得興到十分。死命亂搗。春花也努力相迎。兩下都洩了。他三個繫好褲子。又張過了一會。那道士纔下床來。裘氏也起來梳洗了。叫請了衆妾都來相會。道士看這八個美姬。一個個。

\begin{quotation}

眉掃靑山。目凝秋水。朱唇如櫻桃甫綻。粉面似白璧含輝。輕盈眩目。恍若月宮仙子降瑤台。綽約飛魂。依稀洛水神姬來漢水。眞是一陣天香來玉骨。千般嬌媚動芳情。

\end{quotation}

那道士看了。心中又喜又嘆。喜的是一旦得遇這些尤物。可謂生平第一奇逢。嘆的是有美如斯。盡都是桑間濮上。未免可惜。裘氏就將百花樓上做了他僧道二人的禪房丹室。這一日。八個妾都被道士採遍。次日。十個丫頭同常氏都領了他的大敎。這一二十個婦人。一個個喜氣洋洋。把向日不曾遇僧道時的那些悽楚。都不知何處去了。裘氏同衆妾議定了個則例。他帶領春花秋月長舌婦當第一夜。丹姨芍姐夭桃紅杏當第二夜。第三夜是蓮姨榴姐碧梧翠竹。第四夜是桂姨菊姐紅葉雞冠。第五夜是臘姨雪姐水仙天竺。空一個六夜。第七夜又是裘氏起。週而復始。輪着的這一夜。約了道士到各人房裡去行樂。正派定了。衆人無不喜遵。不想桂氏受了和尚的敎。他親上來向裘氏衆妾面前講道。大師傅我讓了夫人衆位。今日這位師傅來。如何不算我。裘氏無言可復。叫將羣芳閣做了他的行館。着桂氏帶着素馨香兒靑梅綠蕚當第六夜。那道士一夜也不空了。他眞合了一個骨牌。名叫做臨老入花叢。有一個西江月說道士同衆婦。道。

\begin{quotation}

異道寰中不少。淫娃宇內多人。借淫說法警人淫。非勸淫人也恁。萬惡淫爲第一。古今報應分明。看官心下要留神。淫念須除乾淨。

\end{quotation}

那道士前夜會桂氏時。匆匆忙忙。次早就同他別了上來。未曾細覩嬌容。此時日間相對。看他好個女子。

\begin{quotation}

雲眸杏臉。螓首蛾眉。儀容嬝娜。舉止風騷。神如秋水之澂淸。氣若幽蘭之芳馥。前夜之嬌媚雖佳。今日之丯婆更麗。行行俱勝絕。但恨少貞操。

\end{quotation}

那萬緣和尚也不是禿驢。竟成了一個蜜蜂。每日除了替道士當夜的婦人不算。其餘衆婦的花心任他選擇。高興就採摘一番。這道士和尚如到了西天蓮花村。思衣得衣。思食得食。又似到了衆香國。要採就採。要弄就弄。眞在佛國中過日子。衆婦人如同活佛眞仙般敬奉他二人。他二人也不想出去。這些婦人別無禱祝。每日滿斗焚香。惟願姚華胄父子永不回家。便是造化。過了些時。家人回來報喪。說華胄在任病故。衆人心下一喜一憂。喜的是他死了。再不得回來取厭。憂的是姚澤民在彼無事。恐回來得快。打斷了風流會場。只得家中開喪披孝。裘氏同着衆婦披蔴戴孝。一味乾嚎。到了內邊。還是穿紅着綠。抹粉塗脂。簇擁着和尚道士。嘻笑之聲盈耳。又過了月餘。姚澤民家信來。說他搬喪回無錫安葬。不久來京復命。衆人這却戴上愁帽兒了。大家就效法李白宴桃李園敍上的兩句。道。

\begin{quotation}

人生若夢。爲歡幾何。古人秉燭夜遊。良有以也。

\end{quotation}

他衆人以夜繼日的行樂。猶恐不足。那和尚道士弄得如行山陰道上。應接不暇。又過了些時。素馨的漢子吳實打前站。先回報說二爺回來了。兩三日內就要到家。因恐怕家中懸望。故此先差了他回來。這吳實來家報信。以爲主母們不知如何歡喜。不知衆人聽了這話。如半空中一個大霹靂。癡了一回。比前次來報姚華胄的喪還苦楚些。也有嘆氣的。也有墮淚的。也有暗暗跌足的。也有背地捶胸的。皆面無人色。料道和尚道士留不得了。痛弄了一日一夜。知道此別。歡不可繼。每人定要道士採了三度纔罷。次日五鼓。送他二人出去。裘氏同衆妾婢皆號咷大慟。整哭了半日。萬緣仍同那道士回寺去了。桂氏依舊搬回故宅。那素馨見他男人來家。咬牙切齒。恨道。多少人跟了去。偏是這烏龜先回來。沒奈何。只得回家相伴。過了兩日。姚澤民到了家。面過聖。命他襲了侯封。他晚間同桂氏共臥。行起事來。覺得大異當日。寬而無當。極力抽提。見他毫無樂態。心中雖疑。難出於口。次夜即上去孝敬繼母。覺裘氏之物亦然。過後再賞鑒那八妾十婢。其味如一。向日之極贊美他者。到今俱似有如無。並無一褒語。他以爲是數千里遠來。鞍馬馳驅。身體羸瘦。或者此物也瘦了之故。不及當日。那裡知道是家中供養的尊師同外來的道士弄得如此。衆婦人即如腥葷吃慣了。再吃那沒油鹽的蔬菜。還有何味。裘氏自和尚道士去後。每日悶悶不樂。姚澤民雖竭力在他胯下承歡\footnote{數千年自有承歡二字以來。未有如此用法。不但奇文。而且奇聞。}。只覺得心中似別有所思。口中不住微微長嘆。漸漸的飮食俱廢。終日昏睡。捱了數月。把一個未及三旬的佳人。化做南柯一夢。堪笑他。

\begin{quotation}

滿擬快樂百〔年。〕  豈意春光三九。

\end{quotation}

姚澤民講不得野丈夫的話。少不得同姚澤(予)民做眞孝子。開喪出殯。因他無出。不送去故鄕。就在本京葬了。那八妾見姚澤民回來。先也深以爲恨。久而久之。知道和尚道士是萬不能來了。只得大家簇擁着他。借他來消遣。姚澤民也竟忘其此輩是他乃尊之妾。公然以夫主自居。視爲自己小星。朝夕尋歡取樂。桂氏倒還頗不寂寞。有萬緣姚步武盛旺輪次相伴。雖不能像姚澤民不在家那樣放膽。每夜更闌人靜。約了進房。黎明帶星而出。也就可以足興了。再說萬緣那日同道士回寺。他熱鬧了半日。忽然一旦分離。難割難捨。一路垂首喪氣的歸來。誰知他的那兩個婦人。見萬緣去了許久。他在衆徒弟中選了兩個年壯陽強的小夥子。將萬緣歷來施主家哄騙來的銀錢。一併蓆捲。相率而去。萬緣剛進門。衆徒弟就悄悄報知。他一心迷在桂氏身上。幷不介意。倒是衆徒弟見去了行樂之人。十分着急。又不敢出去訪問。萬緣自從去了兩個婦人。他在姚家成月不歸。姚澤民去陪衆妾。他便去陪桂氏。後見裘氏死了。他也暗暗傷心。行住坐臥。不禁長嘆。過了幾日。他失張失智。精神頓減。那裘氏先(死)後有半載。萬緣一日同姚澤民在佛堂中。他趺坐在禪椅上咬文嚼字。高談佛法。講那些輪迴因果。善惡報應。忽然如物所中。七竅流血。跌在地下。姚澤民忙叫人扶在榻上。用薑湯灌了多時。方醒轉來。兩目直視。姚澤民問他緣故。他儘着搖頭。模模糊糊的道。說不得。說不得。老爺夫人長枷鐵鎖。帶了許多鬼卒。來拿我到陰曹去對案。再問。他只搖頭道。說不得。說不得。再問。便不答。姚澤民忙叫人駕車送他到了寺中。衆徒弟剛擡到房中。只見他大叫道。不用打。不用打。我該死。我該死。口鼻內鮮血直噴。氣絕而亡\footnote{衆徒弟造化。再沒人弄蔬屁股了。}。家人回來說了信。桂氏知道。暗暗哭了四五日。過了幾個月。心纔放下了。晚間冷靜。只叫盛旺來相伴。又過了幾年。陝西流寇叛亂。祟禎皇帝命姚澤民領兵去征勦。那八妾十婢因沒了夫人爲首。他們可敢去招攬外人。都急得抓耳撓思(腮)。幾乎要死。姚予民素常也有些風聲傳入耳中。知道八妾衆婢同兄弟所爲。怕他們又弄出醜來。況留着他們。也非常法。將這些婦人盡皆遣嫁。無一個不替他合掌念佛。鼓舞歡欣而去。一年後。姚予民得病善終。後來姚澤民降了李自成。領兵殘破了鳳陽祖陵。祟禎大怒。南京刑部將姚華胄剖棺戳屍。逆妻桂氏同姚步武等親丁男子。無倫(論)少長。皆駢(併)斬於市。家產入官。其家下男婦皆分給功臣之家爲奴。念姚予民愚蠢無知。妻女免死。發金齒衛充軍去了。姚予民有嫁了父妾衆婢的這一點好處。自己免了一刀。妻女饒得情(性)命。可見人有些微善行。上蒼決不相負。這是後話。再說那老道自姚家出來之後。深自悔恨。道。他家婦女雖不良。我去淫他。豈非我之罪過。發誓痛改前非。別了萬緣去雲遊。從此茹蔬。施藥濟人。以救往過。一日。遊到南京。住在洞神宮。重到接引庵。看看那黑姑子也四十多歲。成了老尼了。他二人雖係舊交。此時道士已戒了色事。只留了一齋。談談舊情而已。訪問到聽。黑姑子說他久矣物故。那老道不勝感嘆。回到下處。施藥救了多人。四處盡聞其名。値賈文物得病。鮑信之舉薦了他來看。買(賈)文物徼幸遇了他。他見賈文物情意殷殷。故贈了他那靈丹。治了妒婦。救了他的苦難。又恐傳出去。有少年膏粱子弟來胡纏。他又悄悄不知遊到那裡去了\footnote{去得乾淨。}。按下不提。要知鍾生收拾赴京會試。後來事業如何。但看下回便見。

姑妄言第十五卷終



\endnotetext[1]{「怒氣」原作「氣怒」,據文義改。}

\endnotetext[2]{「強求」原作「求強」,據文義改。}

\endnotetext[3]{「得他」原作「他得」,據文義改。}

\endnotetext[4]{「一時間」原作「一間時」,據文義改。}

\endnotetext[5]{「自己」原作「己自」,據文義改。}

\endnotetext[6]{「接天」原作「天接」,據文義改。}

\endnotetext[7]{「不要」原作「要不」,據文義改。}

\endnotetext[8]{「一個」原「個一」,據文義改。}

\endnotetext[9]{「就會」原作「會就」,據文義改。}

\endnotetext[10]{「素馨」原作「素香」,據上文改;下文或同,不贅。}

\endnotetext[11]{「個抽」原作「抽個」,據文義改。}

\setcounter{footnote}{0}

\theendnotes

\part*{姑妄言第十六卷}
\addcontentsline{toc}{part}{姑妄言第十六卷}
\markboth{姑妄言第十六卷}{姑妄言第十六卷}

鈍翁曰。鍾生錢貴夢古城隍一段。雖是爲錢貴賜目之故。却是點第一回題目。

寫鍾生夢中攙着錢貴同行。扶着錢貴由傍邊角門而入。喚錢貴同跪倒俯伏。〔拉〕着錢貴膝行到滴水簷前。不留心看去。不過是泛然說話。細細一看。句句是與瞽妻同走。此等細心。眞令人不能及。

寫鍾生之遇鄂氏。不但結去鍾悛。且做將來收小狗子他母子團圓張本。

鍾生爲官之法。凡歷仕途掌刑名者。當書一通。置於座右。細心潛玩。不但凡罪者受福無量。而自己亦獲福無量。寫鍾生做官好處。不過是誇他人品才能。到請裁太監監軍一疏。余不覺掩卷嘆曰。世人豈無忠義爲心者。只爲大家因循過了。鍾生未上疏之先。並不曾見一人言及。鍾生上疏之後。觸了聖怒。就有二十餘員大臣爲他乞恩。許多同年替他分罪。關爵又上疏力救。程閣老諸人又救。關爵一人唱之。自有和之者。齊之王孫賈。漢之周勃。便是千古來的樣子。但恨沒這一個先出頭的人耳。

程閣老之相業。雖無可傳述者。其居官之廉介。世之所無。余知之甚悉。故表而出之。可爲萬世爲官者之師範。

寫宦實。雖是寫他始末事蹟。却實是寫鍾生。不是這一番苦苦力爭。宦家父子朝夕感恩戴德。報以厚產。後來鍾生回家。兩袖淸風。何以養廉。何處居住。且宦家事中。又帶寫劉太初之淸高情義。並梅生郝氏竹思寬諸人。不致寂寞。連美郞也就便一提。我不知作者之心。何精細至此。閻良創\footnote{音臭。}氏。傅厚之輩。舉目皆是。特詳寫之。以供識者之笑。不但爲此輩之鍼砭。亦是救頽俗之菩提心。

寫代目遇祖母父母。不但使鍾生有東道主人。他一部書內。沒要緊的人不肯漏去一個。何況戴遷有關係者。此猶在次之。因此而得遇郗氏。又是特出這一個女中丈夫。云鬚眉所不及也。且又後來榮公流寓土山。作易于仁結果張本。

鍾氏弟兄同室操戈。推刄同氣。大約世上家庭之內。往往有之。至於知縣刑廳。滿心要錢。滿口說道理話。亦未必不個個皆是也。試聽知縣之勸他弟兄。刑廳之責備都氏。說得何等大方。眞是老子(手)。童自大破各(吝)延賓。雖寫其非昔日之鄙嗇。借此成就五對小夫妻。使衆人打成一夥親眷。

或謂錢貴多年瞽目。一夢便得重明。未免似覺荒唐。余曰。不然。此一部書。都無中生有。極言善惡報應。警醒世人耳。錢貴之目不如此寫。不見報應顯赫。況亦不足爲異。如裴度之種帝王鬚。丁謂之換鬼眼。雞冠秀才之三耳。皆見於正經書內。豈盡荒唐者耶。況瞽目重明者。載之各書。比比有之。

\chapter*{姑妄言卷之十六\\
第十六回 鍾麗生致仕歸 古城隍圓宿夢\\
附 戴家父女無意喜相逢 鍾氏弟兄有心惡傾害}
\addcontentsline{toc}{chapter}{第十六回 鍾麗生致仕歸 古城隍圓宿夢}
\markboth{第十六回 鍾麗生致仕歸 古城隍圓宿夢}{第十六回 鍾麗生致仕歸 古城隍圓宿夢}

話說鍾生在家讀書。光陰荏苒。倏爾殘冬。他夫妻一日擁紅爐。賞瑞雪。飮佳釀。談淸話。錢貴向鍾生道。向日妾家與古城隍廟相鄰。我自與君定盟之後。許下一願。保佑君秋闈得意。早諧連理。若果如所願。親到廟中叩謝。今宿願俱遂。妾意欲明歲新正元旦。要同君去酬還。君意若何。鍾生道。古城隍神係漢朝大將紀信。因代漢高帝誑楚焚死。忠義成神。後封王。立廟於此。極其靈感。旣有此願。應當酬還。到期預備香供。我與你同去。撚指間。臘盡春回。已是新年朔日。那鍾生與錢貴備了豬羊酒果。香花紙燭。淸晨到古城隍廟去還願。到了廟中。焚疏化紙。上香點燭。二人跪在地下。默默禱祝了一會。叩謝已畢。散了福物。然後歸家。夫妻三人擺上酒來同飮。慶賀新年。說說笑笑。歡歡喜喜。天晚共寢。方矇矓之際。忽見一尊金甲神說道。大王陞殿。命召你夫婦二人。鍾生錢貴聽說。不知來歷。慌忙起身。問道。請問尊神。大王今在何處。神道。你但隨我來。鍾生只得攙着錢貴同行\footnote{攙着同行。一。}。約有數百步之外。見一王居。金線硃戶。碧瓦飛甍。高門大㦸。甲士環繞。神道。你且在此。等我稟報。須臾出來。道。大王命你進去。鍾生扶着錢貴\footnote{扶着錢貴。二。}。由傍邊小角門循循而入。到丹墀下。遙望殿上坐着一位王者。傍侍官吏數百。莊嚴貴重之至。慌忙跪下。喚錢貴同跪倒俯伏\footnote{喚錢貴同跪。三。}。只聽得那王者道。着他上來。衆人傳呼。鍾生拉着錢貴\footnote{拉着錢貴。四。}。膝行到滴水詹(簷)前。那王道。早間爾夫婦酬願。鑒爾虔誠。吾神已歆其祀。他夫妻聽了。方知是古城隍。忙頓首道。某夫婦蒙大王恩庇。得遂鄙心。但恨無可上報聖恩耳。王道。爾夫婦雖是今生之緣分。却係前世之往因。爾可能記憶否。鍾生道。某下土愚士。已昧往因。求大王指示。王道。此一種(樁)公案。俟將來期到再爲明剖。今只將你二人往事示知。爾錢貴前生姓白。生得頗有姿容。却愛富嫌貧。爾鍾情前世姓黃。家資富厚。欲求白氏爲婚。白氏倒也心願。因他父母見你生得奇醜異常。不肯依允。故爾二人遂兩地相思而亡。吾神因白氏愛錢。命姓錢家做女\footnote{世上姓錢人家女兒。皆前世愛錢者耶。}。爲他不分好醜。故罰瞽目爲娼\footnote{此等人應當如此罰之。}。爾鍾情前世不過癡愚。却無過犯。憐你枉死。故使你初爲貧士。復查爾頗有善行。後博一第終身。與錢貴死(先)做煙花友。後成結髮緣。了却前生相思之債。鍾情本止一第。因爾多情種子。不負初盟。謙謙自下。度量寬宏。見色不迷。持身以正。吾神資爾後福。還可發甲爲官\footnote{此處着眼。}。但好心常存。切勿改變。那錢氏因爾矢貞不妒。良家也是難得。何況煙花。今賜爾二子。與鍾情共守白頭。但爾後來還有命婦。今再賚爾雙眸。因命左右道。將他眼光還與他安上。只見一個黃巾力士。手中拿着兩個明亮亮如夜明珠一般。走到錢貴跟前。向面上一擲。回身稟道。已還他了。那錢貴只覺眶中一涼。透入心髓。把雙眼一睜。無不備見。他夫妻二人歡喜得只是叩頭。王又道。去罷。他二人爬起。慌忙走出\footnote{自己重明。不復用攙扶矣。一絲不錯。}。倏忽雞鳴。鍾生欠伸而寤。細想前夢。宛然在目。適錢貴亦醒。忽見殘燈將滅。因大喜呼鍾生道。我兩目皆明了。鍾生忙起身一看。見他嬌滴滴一對秋波。不勝歡喜。遂將自己的夢說了一遍。錢貴諤(愕)然道。我與郞君所夢。一字不差。方悟他夫妻二人初遇即兩情相愛。乃係宿緣。遂道。神靈顯赫若此。眞可畏也。我二人當叩謝。就起來梳洗。焚香叩拜了神恩。錢貴與鍾生多半載的恩情。今日方得覩良人的相貌。欣喜非常。

\begin{quotation}

一個多時舊識。今方得覩檀郞的芳顏。一個半載恩情。此刻纔觀嬌妻的俊目。一個耳畔聲音無異。只目少差一個。眼前光景皆新。歡心如湧。他夫妻惟戴城隍的新恩。更篤前生的舊好。

\end{quotation}

他夫妻見是前世結下的姻緣。更加恩愛。鍾生見神說資他後福。越發存好生。做好人。行好事。以答神佑。不覺過了上元。打點行李路費。擇日上京會試。選了正月二十二日長行。衆親友得知。送程儀的一槪璧謝。請餞行的終日不斷\footnote{鍾生致仕回時不過數載。非比丁公化鶴始歸。今日送程儀餞行諸人。那時何不見一個接風者。古今勢利。}。鍾生無暇。只十分推辭不却的。方纔領請。先一日。他妻妾治酒。家宴餞別。到晚來上床。又餞了一番。此乃心至之情。不用細說。次日起程。雖送者多人。鍾生都辭回。惟梅生送到江干。方纔分袂。鍾生鍍(渡)江到浦口。雇了一乘馱轎自坐。兩個家人騎了脚騾。長行進京。一日將午。到了淸江浦地方。忽起大風。掌鞭的道。爺。今日風大。恐過不得河。\endnotemark[1]老爺不如在這裡住\endnotemark[2]下罷。前邊河沿上沒店口。鍾生依允。就揀了一座乾淨客店住下。鍾生在房內坐了一會。見天色尚早。到店門外街上閒步閒步。看那來往的人\endnotemark[3]甚是熱鬧。正看時。忽見一個婦人衣裙〖衤藍〗褸。在河下洗了許多衣服。抱了上來。鍾生看了。好生面熟。一時想不起。他哥哥鍾悛撇他時。他已十一歲了。今雖離了十年。還隱隱有些記得。忽然想起。道。這人好像我嫂嫂鄂氏。如何來在這裡。也只疑模樣相同。又不敢問。見他同着家門口一個婦人講話。是南京聲口。越發動疑。留心看着走入一間破草房內去了。鍾生走進店來。問店主人道。你隔壁這家姓甚麼。我纔聽得那婦人說話。好像我們南京城裡人的聲氣。店主人道。這婦人原是南京來的。他前夫姓鍾。就是小店上業主。他家前歲爲了一場官事。纔把這店賣了與我。鍾生道。你可知這姓鍾的叫甚名字。這婦人姓甚麼。店主道。聽得人說這婦人姓鄂。他前夫賣房文書上的名字是豎心傍。放個俊字半邊。我問人。就是筌字。又有念俊字。我到底不知叫甚麼。鍾生聽了。知是哥嫂無疑。忙問道。如今這姓鍾的往那裡去了。店主道。就是那年爲了官事出來。不久就死了。這〈那〉婦人孤身。又沒個親人。無穿少吃。嫁與隔壁這何尚仁爲妻。纔得一年多光景。鍾生又問道。你可知這姓鍾的是爲了甚麼官事。後來是害甚麼病死的。他有個兒子往那裡去了。這婦人現嫁的是個甚麼人。那店主道。說起來話長。爺請坐着。我慢慢說與爺聽。叫走堂的拿了張椅子放下。鍾生坐着。他道。這個姓鍾的先開店時還好來。這個地方是個大碼頭。來往的人多。倒也興旺了些時。這肏娘的到後來刻薄不過。在客人們身上一個錢算得筋盡力出。因此到他店中來歇的就少了。那一日。有一個做小賣買的老兒。在店中住了一夜。次早開發店帳。少了一個錢。他決定不依。那老兒身邊又沒一文。許到街上賣了東西送來還他。他又不肯。那老兒嘴裡不乾不淨。囔嘟幾句是有的。不隄防被他夾臉一掌。不想有年紀的人。大淸早空心肚裡。被這一掌打昏了。一交跌倒。剛剛撞在一塊石頭上。把腦後磕裂。當時身死。他在這裡住了七八年。只許他占人便宜。他從來一文捨不得。街鄰素常都恨刻薄。到了官。就把他證住了。官府也惱他爲一個錢這樣刻薄。定要問他個抵償。\endnotemark[4]他急了。只得將這房子賣了與我。上下打點。房銀子那裡得夠。這一下把這肏娘的家私抖了個罄盡。纔問了個過失傷命。但追燒埋銀兩給與屍親。官事完了出來。他也就是屬太監的。淨了身了。租了兩間房子住着。不多時。便病死了。他的兒子我們不知道。只知這婦人丈夫死了。沒得依傍。纔嫁了這何家。他男人是天妃閘的閘牌子。家中窮苦得很。這婦人靠着替人漿洗衣服過日子。姓鍾的這拉牢的囚。刻薄了一生。落了這樣個下場頭。也就是現世現報了。鍾生聽了。不覺掉下淚來。店主驚問道。這人莫非替爺上有親麼。鍾生含淚道。這就是我先兄。我幼時只知他離了家鄕。並不知他搬到這裡。店主人聽得是他哥哥。惶愧不安。忙賠罪道。我不知是爺的令兄。言語中多有得罪。爺上寬恩。莫要計較。鍾生道。店主不知。這有何妨。不必介意。我家嫂雖嫁了人。我要去問問先兄骨櫬在那裡。並姪兒的下落。煩主人家同我一去爲感。店主道。小人當得奉陪。忙跳出櫃來。同鍾生走入隔壁何家。在房門外叫道。何大嫂。有位令親鍾爺來會你說話。那鄂氏正在房中搥衣服。聽見。忙開了門。認得是店主。問道。大爺說甚麼。店主指着鍾生。道。這位是上京會試的鍾爺。有句話來問你。那婦人讓進房。鍾生同店主進去。鍾生向婦人作了個揖。婦人忙把破衣袖扯了扯。回拜。道。貴人爺折死我了。爺有甚話吩咐的。鍾生看那房中惟有一張破板床。鋪着個草薦。連坐的板凳都沒有。只得站着說話。你道鍾生離鄂氏時。他纔十一歲的孩子。倒還甚(認)得鄂氏。至於鄂氏。那時已二十多歲的人了。如今倒不認得他。是何緣故。彼時鄂氏已是大人了。雖隔了十年。不過老蒼了些。規模不得改。故此還依稀認得。鍾情那時還是個小孩子。今日長大成人。模樣改變。且如今又是貴人體統。鄂氏也決想不到他有今日這一日。雖聽說是姓鍾。就彷彿有些相似。自慚形穢\footnote{此語令人傷心。}。也不敢混認\footnote{爲窮字放聲一哭。}。鍾生墮淚問道。嫂嫂。你不認得我了麼。我就是鍾情。那鄂氏細看了一看。也就哭起來。道。原來果是二叔。你哥哥當年撇了你來。鍾生止住道。已往的話都不必提。哥哥的事。方纔店主說了。我都知道。我來只問我哥哥的骨殖今葬在那裡。我姪兒小狗子往那裡去了。鄂氏道。小狗子那奴才。自幼不成器。好吃好賭。家中的東西無樣不偷。你哥哥三番五次也打不下他來。後來大了。越發不成人。你哥哥爲官事破了家。棄了房子。後來事完了。還剩有二三十兩銀子。還想做個小生意糊口。不想被那斫千刀的輸急了。夜間偷了去。連他也不見了。你哥哥着了一口重氣。得了病。又沒錢吃藥。懨纏了些日子就死了。連棺材也沒有。街坊上各鋪面化了一口材。那裡還有力量買地埋葬。就燒化了。撂在河邊水葬了。我無依無倚。少穿沒吃。租了間房子住着。又沒房錢與人。死守了半年。沒奈何。纔嫁了姓何的這家。小狗子到如今總沒個信兒。我聽見人說他投了一個做官過路的。當家丁去了。又哭着道。我雖沒廉恥嫁了人。也是萬沒奈何的事。我如今雖是回不去。你見我這麼貧苦。二叔。你如今已是貴人。人說不看僧面看佛面。你就不看我。看你過世的哥。照看我照看。只當積陰隲。我替你念佛罷。鍾生也不答應。含着淚。同店主辭了回來。到店中。忙取了些銀子。煩店主買了些登(祭)禮。香燭包皮紙錢銀錠之類。又煩店主收拾了一桌供。到晚來。在河沿上擺設停當。招魂致祭。焚香化楮。哭了一場。哭得好不傷心。連店主悽慘得也掉了幾點淚。上前扶住。勸道。令兄死者不能復生。爺長途辛苦。保重要緊。再三勸止。鍾生方奠了酒。回店中來。叫將祭品收了。送了些與店主。又送了些與鄂氏。餘者分散與家人騾夫。鍾生晚飯也不曾吃。悲切了一夜。次早起來。拿了四兩銀子。煩店主送與鄂氏。鄂氏親身過來千恩萬謝。鼻涕眼淚的哭了回去。鍾生辭謝了店主。起身渡了河。到王家營住了一宿。次早上了馱轎。家人各騎了騾子。往北直發。到了京中。覓了寓所。到了場期。考試過。放榜時。又中了進士。他的座師姓樂名爲善。係北直隸順德府人。現任禮部侍郞。見他少年老成。十分相愛。殿試之日。殿在二甲。選入庶吉。後考選衙門。在刑部觀政。陞了浙江司員外。鍾生到任之後。差人接了家眷來京。不必煩敍。那鍾生在衙門中。惟以救人除弊爲念。把本司中歷來舊弊。一槪淸除。凡有公事。定然細心審究。恐有寃枉。一文不要。百事從公。他將本司重囚。現在監禁的舊案。悉調細看。稍有涉疑者。即提來覆審。平反者甚多。他親執到堂上面講。堂上道。此皆貴司未任之前所審定者。與貴司何事。鍾生道。司官若不在衙門。不在其位。則不敢謀其政。今旣待罪。本部但恨司官職微。不能將十四司案卷盡勘。使獄中無寃民。稍報聖天子洪恩之萬一。若知之而模稜不言。豈不愧李目知乎。堂上又婉說道。貴司所言固是。若必欲正之。獨不爲同僚地乎。鍾生道。劉誠意仲君劉璟對成祖云。臣當讓者不敢不讓。不當讓者則不敢讓。君臣之際尚且然。更何況於同僚。同僚諸公果決獄如神。司官師之不暇。何敢多喙耶。旣知有枉。則不敢顧同僚之面情。和光同塵。而使無辜至於死地也。堂上拗他不過。只得依他。間或堂上斷事微有差謬處。他再三執理面諍。不肯媚人害人。一日。堂上大怒道。你少年新進何知。視我反不及耶。鍾生道。司官雖幼而不能。蒙皇恩不以爲不肖。謬擢今職。司官旣知之而曲隨老大人。是上負聖恩。下欺老大人矣。且司官所執者。不忍人有寃耳。並非一己之私。老大人請細察。司官若有徇私之情。參革議處。卑司領罪無辭。昔范純仁謂司馬溫公云。公爲宰相。則不許他人言耶。若謂司官以老大人爲不及。則司官豈干(敢)。聖千慮猶恐有一失。司官之力爭。正是敬愛老大人處。堂上道。少年人不可執一己之見。當爲功名惜。鍾生道。司官幼失怙恃。無苦不備嘗。甘於淡薄久矣。今雖徼倖一官。除俸祿之外。司官不敢妄取一文。其寒薄猶如昔年寒士時也。此官有也可。無也可。功名富貴四字。司官並不介意。惟之盡力於朝廷。至於死生禍福。聽之於上蒼而已。堂上道。貴司每個(每)固執。不懼有失出失入之故耳。鍾生道。司官若不能洞悉其事。安敢妄言。若果有無罪而失入。有罪而失出。〈入〉自有朝廷之法在。司官領罪。何敢辭焉。堂上要謫他的〈他〉謬處。細細詳察。件件俱是。又心服他。只得依允。這浙江司係十四司之首。凡各司有事。此司皆同審問。堂上先也有些惱他。故將幾件疑難事發與他審理。他一見便能燭奸。寃者伸之。強者抑之。惡者除之。善者旌之。多年老吏還不能如他這等歷練。堂上見了。反着實敬愛起來。後來見他說堂。都齊(霽)顏相待。這些同僚中。或有些私弊。料道瞞他不過。再三婉懇。他見事體無大關礙者。却不過面皮。只得依允。或欲分惠於他。他一文不受。所以這些同僚中。雖然妒恨他。又都敬懼他。他又時常傳四個司獄司道。說世間人之惡。莫過於禁卒。所以置於娼優隸一流而居於末。古人有深意焉。此輩只圖飽他私囊。不顧犯人死活。遇窮苦罪人。不能飽他所欲。則百般凌虐。該司要常常稽察。着實嚴禁。萬不可貓鼠同眠。任其肆惡。本部若有所聞。恐該司不能辭其責。昔于公治獄。大興駟馬之門。何處無非惡積德。本司也着人緝探。若禁卒仍悛惡不改。本司自當呈堂重究。但諸公恐亦難免疏失之過。勿謂我之不早。又常叫衆禁子。吩咐道。本司雖非提牢官。但我旣在刑部。獄中事我就管得着。本司素知爾等不法。凌虐囚犯。索詐要錢。但他犯的是朝廷的法。殺剮流徙。他自無辭。不曾犯了少你禁子錢的罪。又加一等鎖杻。那是他應受者。爾等若加一非刑而索賄。豈大明律中另有此一款耶。旣往不究。此後須改過。若仍前肆惡。本司查出。爾等勿以性命輕試。本司言出必行。爾等務要小心。衆人知他連堂上都不怕。倒也都懼你(他)。收歛了許多。每月喚提牢主事。他便諄諄懇囑。嚴約禁子。恩待犯人。不但是做提牢的分中當爲。且暗暗積了多少陰隲。衆同僚也都爲他所感。在獄中留一片心思。獄中犯人聞知。無一個不感激他。司中這些書辦衙役。在外索賄。他都細心體察。若些須無礙的錢。他也放鬆一着。並不說破。若稍有關係。初則叱辱。再則重處。無不凜遵他的法度。又嚴諭家人不許向爲事人需索。凡有犯事的人。都暗暗禱吿。求分在他司中爲幸。後來如有犯人經他一審。心悅誠服。再沒有稱寃者。他輕易再不肯動夾棍。向同僚道。人之一身雖有貧富貴賤。無非本於父母。血肉之軀。以此三木囊頭加之。何事不成。而內中爲寃多矣。至於謀反叛逆。江洋大盜。固執不招。又有證據甚明。則不得不用此。若其次之罪。自可以細心揣得。何須借此酷刑。況我輩不幸而爲刑官。若一任性。使犯人受其楚毒。誣扳枉認。致人破家喪命。其害非小。不但惻隱之心四字有愧。且損了許多陰德。我見近日掌刑諸公。竟以夾棍爲兒戲。勿論事之大小。先以夾棍示威。視比杖朴猶輕。是豈有人心者哉。我見感應篇內云唐朝師德婁公。一生盛德謹愼。尚失入人罪。以致減祿損壽。何況我輩。敢不細心體察。衆人皆笑其迂\footnote{鍾生向諸人說天理話。猶如孟夫子向齊梁諸君講王道。人焉得有不謂之迂者。}。他又將呂叔簡先生所作戒刑一篇。參以己意。有關於事時者。細心添減。手錄一道。貼於官廳之內。以勸同僚云。

\begin{quotation}

蓋用刑之心。其發如火。其流如波。急宜受之以止。常存此心。便有學有養以調伏之。不見我貴人賤。不知此德彼怨。即是聖賢器。豈僅仕官楷模哉。願居官者留心悉戒。而傍觀者亦直(宜)戒人。勿自認風霆爲至敎。而相諛怒罵皆文章。則世道人心之厚幸矣。

五不打

老不打。幼不打。病不打。人已打我不打。衣食不繼不打\footnote{飢寒切身。打後無錢將養。必死。}。

五莫輕易打

宗室莫輕打。官莫輕打。生員莫輕打。上司差人莫輕打。婦人莫輕打\footnote{恐有寃枉。婦人羞起。多致輕生。}。

五勿就打

人急勿就打\footnote{適速其死。}。人忿勿就打。人醉勿就打。人隨行遠路勿就打\footnote{不能將息。日逐跋涉辛苦。亦恐致命。}。人跑來喘急勿就打\footnote{六脈奔騰。血逸攻心。未有不死。}。

五且緩打

我怒且緩打\footnote{盛怒之時。尚何所惜。萬不可怒時責人。書云。如得其情。則哀矜而勿喜。〔喜〕且不可。況於怒乎。}。我醉且緩打。

我病且緩打\footnote{病中多有火性。}。我見不眞且緩打\footnote{錯後難更。}。我不能處分且緩打\footnote{遇難處之事。難凡之人。一時粗浮。不應所終。而遽加刑。後難結局。且費區處。}。

三莫又打

已拶莫又打。已夾莫又打\footnote{重刑難受。血脈奔潰。又加刑責。豈有不死。且夾棍不列五刑。小民受此。終成廢疾。難以趁食。切宜念之。即審強盜。因夾成招。此心中放不下。惟多方設法。隔別細審。令其自吐眞情。於心斯安。此等酷刑。終不可用也。}。要枷莫又打\footnote{屈伸不便。瘡潰難調。足以致命。若罪心應責。莫如放枷時責之。}。

三憐不打

盛寒酷暑憐不打。佳辰令節憐不打。今方傷心憐不打\footnote{人値不幸。家中正有傷心事。如遭喪失火等類。又加刑責。鮮不輕生。}。

三應打不打

尊長該打。爲與卑幼訟不打\footnote{大關倫理世敎。}。百姓該打。爲與衙門人訟不打。工役鋪行該打。爲修私衙或買辦自用物不打\footnote{不但縱役爲惡。且大壞聲名也。}。

三禁打

禁重杖打\footnote{輕杖即數多亦不傷生。且我見責之多。怒亦稍息。若重杖。只見數少而人已大傷矣。}。禁從不打\footnote{皀隸索賄不遂。每重打腿彎。致有筋斷而死者。或打在一處。潰爛難治。因而致命。}。禁非刑打\footnote{刑中只有鞭杖二種而已。用皮靴底打嘴〈打〉巴。此何刑也。獨不聞面非受之所之語乎。古之笞刑最輕。因其笞背。恐震及於心。以致傷性(生)。故革之。今祖(刑)皆打背花鞭桿。豈不更重於笞乎。是朝廷恐人傷生。欲輕其刑。而刑官特重之以戕命。於心忍乎。}。

\end{quotation}

鍾生但審事之時。不論大小。無不盡心思維。然後纔審。細細問明了。可完之事。或打。或枷。或放。再不懇(肯)留滯。他道。小人窮苦。淹留一日。多費一日用度。輕犯容易不肯發倉發監。恐受禁卒之害。但命招保聽候。到了重犯有不招成者。他體其情。眞罪。常善言撫諭。道。本司豈必欲置爾於死耶。但爾自作之孽如此。我何敢枉朝廷之法以宥爾。若不實承。受刑之後猶不能免。何苦多受一番苦楚。所以有罪者盡皆自認。雖然認了。他必在內中細求。有一線可生之機。必婉轉出之。若萬不可以。然後慘然下筆\footnote{世間果有此等官耶。吾聞其語矣。未見其人也。}。他不但不妄動刑審事。從不疾言厲色罵人。常向着同僚道。他犯法。自有朝廷之法在。律中無一罵罪也。誰非父母所生。開口便傷人父母。此乃市井小人惡習。我輩旣是衣冠仕夫。豈可若此。但是他審的犯人。出來都道。經鍾生爺一番。我們雖死猶感恩德也。因此人將他的姓分開。放了他的外號。背地〈他〉都稱他爲鍾重金。誇他人品才幹比金子還貴重之意。權且按下。再說那宦實向日拜在魏忠賢門下做個乾兒。他不過是功名念重。恐有差跌。倚他爲靠山之意。不能求福。希圖免禍。只算得個屈體的小人。却不曾如崔呈秀阮大鋮田爾耕那些助紂爲虐的乾兒走狗。倚了沒卵袋老子的勢。要害人利己。無惡不作。後來魏璫事敗。奉旨着多官議罪。衆議定了覆奏。略云。

\begin{quotation}

臣太子太傅尚書等官蘇茂相等題。爲遵旨會議事。奸惡魏忠賢。串通逆婦客氏。逼死裕妃。革奪成妃。戕害縉紳。盜匿珍寶。包藏禍心。謀爲不執(軌)。議得魏忠賢客氏俱依謀反大逆律。皆凌遲處死。其崔呈秀並五虎李夔龍等。五彪田爾耕等。相應比照結交近侍官員律斬。其魏忠賢之子姪。魏良卿魏良棟魏鵬翼等。暨客氏之子侯興國。皆決不待時。其廝養乾兒罪之決者傅應星等。皆絞。其門下用事人楊文昌等。發配煙瘴充軍。云云。

\end{quotation}

奉旨准了。他門下這數百助惡的鷹犬。盡皆拿究問罪。宦實那時也就心膽皆裂。喜得他平素未嘗助人作惡。且他歷仕久了。又是進士出身。他同寅同年在朝者多。雖未得敢護庇他。未免有些情分。故此無人摘發。因而遂得漏網。雖如此說。他那一日不提心吊膽。欲要吿歸。恐前脚一勳(動)。後面爲人所算。他在朝到底爵尊位\endnotemark[5]重。人還畏怯三分。雖是如此算計。也如在針氈上一般。無刻心安。崇禎皇帝惱恨逆璫誣陷東林。幾危社稷。搜尋他黨羽不已。有一個大膽的臣子。他也〔是〕逆璫門下。尚未犯出。想道。與其袖手護罪。不若捨命上一本。或者徼幸得免。倒未可知。他竟上了一本。內中有幾句道。

\begin{quotation}

魏璫秉政。人人自危。陛下當日位處親藩。朝廷介弟。猶上請尊崇忠賢。爲之建祠誦德。以免讒忌。何況外廷小臣。生死關頭。依附以求脫禍者乎。伏乞聖恩垂念。赦其舊辜。責其新效。則羣下幸甚。云云。

\end{quotation}

崇禎見了這本。細想。果然不謬。遂有旨道。

\begin{quotation}

逆璫已伏嚴誅。其親黨幷已獲附逆用事諸人。如唐朝依附朱泚逆臣三等問罪之例施行。其未發覺者。槪不株連。

\end{quotation}

後來將逆案結過了。宦實纔放了心。又過了年餘。他放吿老回家。到了家中。富貴的人到(致)仕榮歸。誰不奉承。他家的熱鬧。自不必說。眞是不來親者強來親的時候。沾親帶故。因親及親。算盤打不淸的親戚也都來拜望送禮。只有他一個妹夫劉太初不到。且連妹子都不來。宦實差人去請了數次。他並無多年(言)。只有四個大字相覆。道是無暇多謝。後來宦實親去看妹子妹夫。覿面致請。他也決不肯至。所有贈遺。又力辭不受。沒奈何。只得聽\endnotemark[6]之。宦實見兒子離了數年。比當日大不相同。更改得竟成了一個好人。又見媳婦也賢慧知事了些。嬌花丫頭又生了一個孫子。雖係度(庶)出。老年人見了個孫兒。也自歡喜。況且又脫了這場大難回來。心中這個快樂却也不小。那司富跟着宦實在京。做了大掌家婆。年歲半百。倒越發白胖了。只像未及四旬樣子。一日。侯氏嬌花都到艾夫人上邊去。宦蕚在房中午睡。他走了進來。一屁股就坐在床沿上。推醒了宦蕚。笑着道。他(你)這沒良心的。我還是你的舊師。今日嫌我老。就不理我了。來家這些日子。你連親熱話也不望我一句。我當日怎麼從小帶你來。宦蕚忙坐走(起)來。摟了親了個嘴。道。我怎肯忘了你。這些日子忙亂。又沒個空地方兒。我那一日不想着你。拉他上床。放下帳子。大白晝不好脫衣。單把他褲子褪下。看他你(的)陰戶越發比當日豐滿得可愛。遂抽弄起來。

\begin{quotation}

司富久旱逢甘雨。宦蕚床中遇故知。

宦蕚一番淸晝樂。司富重享大雷槌。

\end{quotation}

司富覺宦蕚的本事大勝昔年。歡樂無窮而散。宦蕚見他年雖五十。丯韻猶佳。時常點綴一番。不必多說。他一家上下好生歡樂熱鬧。是古語說的。樂極悲生。這是何故。當日宦實在朝時。有一個御史。姓陳名忠。係山東人。曾劾過宦實一本。其略云。

\begin{quotation}

河南道試御史臣陳忠謹奏。而愚臣蒙恩內召時。顧無能謹申忠悃之誠。仰乞聖明。俯察斥逐。以肅紀綱事。古稱尚書乃朝廷喉舌之司。非忠誠素著者。何以輔尊聖明。如工都尚書宦實。一味寡廉喪恥。百端婢膝奴顏。位至司空。官非賤矣。尚爲人之鷹犬。年登六十。齒非幼矣。更做人之乾兒子。以朝廷之官帑。爲獻媚之私恩。以朝廷之大臣。爲權奸之奴隸。蒙聖主之恩。視同陌路。受暇(假)父之庇。敬君(若)親生。損人利己之事。無不踴躍力行。致君澤民之術。盡皆棄擲不顧。不但上負廊廟。抑且有玷班行。宜亟賜罷黜。不可不(片)刻留於朝廷之上者也。云云。

\end{quotation}

那時正是魏監當朝。他正要買人心的時候。見參了他年高位重的兒子。可還容得。況本內雖不曾明說出他來。却全說的是他。焉得不怒。本竟留中不發。過了些時。尋了個事故。將陳忠發鎭撫司。廷杖四十。幾乎打死。革職回籍。即刻逐出京城。這是魏璫一者做個人情與他賢郞。二者魏璫因他的本上暗暗株連着他。出他一口氣忿。宦實雖然知道。却並非同謀害他。但陳忠可有不疑他父子同媒(謀)的理。每每同親友談及。便切齒痛恨。他有個兒子叫做陳盡孝。常把這話說與兒子。這陳忠後竟氣忿而亡。不想陳盡孝這科中了進士。見魏黨盡皆治罪。惟獨宦實得免。他上了一本。略云。

\begin{quotation}

唯忠賢之擅權也。雖五彪五虎從傍而鼓之。實致仕工部尚書宦實與之表裡而奸。同惡相濟者也。附己者提之九天。異己者沈之九淵。桁楊斃良善之軀。削奪銷縉紳之骨。以朝廷之賞罰。供一己之愛憎。凡帑庫之銀錢。實一己之囊橐。東廠自有僕役。何須宦實乾兒。宦實自有祖宗。何必忠賢義父。崔呈秀等十人皆以忠賢之義子而誅之者也。楊文昌等多皆以忠賢之奸黨而竄之者也。宦實旣奸黨而乾兒。乾兒而心腹。以一人而諸罪皆備。尚須臾緩其死耶。更有可切齒者。旣爲朝廷大臣。不思爲朝廷出力。反爲逆黨。助彼行虐。生事害人。臣父即其受害者也。且附逆諸人盡皆伏罪。而宦實首惡。反優游林下。得保首領。朝廷之法何在。乞賜嚴誅。方神(伸)衆怨。云云。

\end{quotation}

這本一上去。崇禎見了大怒。御批道。

\begin{quotation}

朕聞成憲者祖宗之遺制。功令者國家之大經。凡爾臣工。罔敢或踰令。爾宦實而朝廷大臣。充逆黨之鷹犬。背棄廉恥。變亂國法。祖宗成憲何在。國家功令安存。勅下錦衣衛。差官校火速鎖拿來京。交與刑部。好生嚴審。從重議處具奏。欽此。

\end{quotation}

錦衣衙接了旨。刻差了校尉。星夜來南。這正是。

\begin{quotation}

歡處忽悲生。喜後兼愁積。

世事夢中身。人情雲裡月。

\end{quotation}

那宦實在家正歡歡喜喜的快樂。忽聽得緹綺(騎)來拿他。又見了御批的嚴旨。如耳根下一個大霹靂。驚得幾死。費了許多銀子送了他們。雖不曾受凌虐。少不得帶上刑具。方纔起身。知此去必無回理。且家妻子還不知作何結局。落了幾點眼淚。幾個家人隨了去了。這宦家上下男婦大小。擡起房子來哭。比死了人還哭得傷慘。宦蕚本要隨父親進京。一時急渾了。沒了主張。他姑父劉太初得了這信。夫婦忙忙同來。把艾夫人安撫了幾句。向宦蕚道。你空急也無用。可作速同人商議。星夜上京。尋門路救他要緊。再三囑咐而去\footnote{閱此。劉太初非無親情。特不肯鑽熱竈門耳。雖孤介太過。然在今日。世間尚有此等人否。}。這宦蕚聽了姑父之言。如夢方覺。思量個門路救父親。又不知尋誰去好。要約人來商議。又不知請誰去的是。正在着急。那賈文物童自大鄔合聽見這信。都來探望\footnote{看至此。賈童鄔三人猶有古道存焉。何以言之。彼諸人不過酒肉朋友耳。非道義之交也。見宦家有事。尚來探視。若在今日。雖骨肉至親。亦趨而避之矣。}。問起緣故。宦蕚細細說了一遍。並說起要尋門路的話。鄔合道。晚生倒想了一條路。不知可用得。宦蕚忙道。你可說了看。若果然救得我家老父。我自重重謝你。鄔合道。晚生蒙大老爺多年培植之恩。怎敢當一個謝字。此不過盡我犬馬之心耳。還不知可行不可行。晚生兩年聞得朋友們打京中回來。說我們城中有個鍾老爺在刑部做官。十分淸正。敢做敢爲。不但爲同官欽敬。就是堂上也十分喜愛他。言聽計從。後來問起名字。原來就是錢貴之夫。晚生說他是同鄕同里的人。存心厚道。定有些桑梓之情。求他說一策以救太爺。不知可行〈可〉否\footnote{孟嘗養士三千。得於雞鴨(鳴)狗盜。還家。門第豈乏富貴親友。今救父之計。出之於一篾。世人只知貴重衣冠而輕視貧賤相識者可爲之甚。}。宦蕚遲疑道。事雖好。但我們當日得罪過他。一。雖賠過禮。他說了那些好話。我們又不曾會過。二。他雖然同城。並無一絲之情相及。三。他不記舊恨就是萬幸了。他如何還肯爲\footnote{有此數疑。後來鍾生力救宦實。實他夢想所不到者。所以感之〔不〕置。念念不忘也。}。鄔合道。晚生看他是〔盛〕德君子。決乎不念舊惡。大老爺若不放心。晚生還想了一條絕妙的門路。宦蕚道。是甚麼門路。鄔合道。錢貴的母親嫁了竹思寬。如今還在舊宅中住。何不去尋他。與他商議。許他重〈他〉謝。約他同往京中。向他兒女說說枕頭上的情。更是靈驗。大老爺說好麼。宦蕚大喜。道。\endnotemark[7]旣然如此。你就同我去。賈文物童自大齊道。爲老伯的大事。我們同去\footnote{此所謂骨肉不如親戚。親戚不如朋友也。}。遂同到了他家。竹思寬接着。讓入坐下。宦蕚道了來意。郝氏出來相見了。宦蕚就說(將)要他同往京中尋他女婿女兒。要他女兒轉央鍾生的話說了。許他重謝。郝氏道。女婿如今做了官。我又另嫁了人。就是女兒肯了。他或者不依起來。我的面皮小。那時誤了老爺的事。反爲不美。我的福薄。也當不得老爺的謝。宦蕚聽了。急得只是跌腿。道。這怎麼處。奶奶\footnote{宦蕚肯下氣稱一聲奶奶者。爲有所求耳。}。你若替我想出個門路來。我定然厚謝。郝氏聽說。因貪他的謝。遂想了一會。竹美掇出茶來。童自大見了驚問。竹思寬遂說要了他回來做兒子。已配了媳婦。童自大甚喜。想起舊情。沒甚麼與他。將頭上根關髮的金簪拔了該(送)他。那竹美叩謝。眼中也點了兩滴情淚。大家正吃着茶。郝氏說道。有倒有一個人。不知他肯去不肯。宦蕚道。請問是誰。郝氏道。有一個梅相公。他自幼與鍾姑爺同窗同案。兩案(人)素稱莫逆。他若肯去。這事定有幾分可成。宦蕚就問梅生住處。竹思寬知道。就說了居址地方。宦蕚謝了他夫婦。又同他三人尋到了梅家。恰好梅生在家。坐下。宦蕚把前事說了。許他成事以千金爲謝。梅生一來想念鍾生。要會一會。趁此同往。不用自己途費。二來倘或事成。想這千金之報。三來就是事不成。他也無人大過。遂滿口應允。宦蕚無限歡喜。約定後日絕早准行。別了來家。次早。差人送了五十金與梅生爲安家行裝之費。又打點帶往京中使費之物。銀子不好多帶。只攜了三千兩。倒帶了一千兩黃物。收拾齊備。又與〈他〉了鄔合三十兩。約他同往京中相幫走動。到了第三日起身。梅生早來。主僕十餘人同渡過江。雇了包程頭口。星夜趕了去了。再說這宦實是奉了嚴旨欽仲(件)。不敢躭延。一到京中。就送到刑部。也是奉特旨的事。不敢稽緩。遂揀選幾員司官同審。鍾生亦在其內。審的時候訊問口供。宦實又想。自己做了一場大臣。又老年了。況在逆璫門下是千眞萬實的事。旣已犯出。如何辯得脫。與其受一審到(刑)罰。依舊推不淸。不如實供。免受苦楚。就是死。也算〇捱了幾年了。主意拿定。遂供道。犯官當日在逆璫門下。原實有其事。那時犯官已爲朝廷大臣。尚何所求。依之並非求福。欲免禍耳。大人請細察。若犯官當日有同逆璫助惡的事跡。雖肆諸市朝。萬死無怨。堂上道。昨日陳盡學(孝)本內道他父親陳忠向日參你。本竟留中。後尋事將他廷杖革職。這豈非你串同逆璫挾仇報復。只這一款。就是你通同黨惡。死有餘辜了。尚有何辯。宦實道。犯官身爲大臣。爲言官糾劾。尚有何面目上本質辯。不過聽朝廷之恩處分而已。後本竟留中。那時犯官以爲先帝念犯官犬爲(馬)多年。寬恩免究。後來陳忠革職。犯官並不知情。堂上笑道。你今日以爲無人質證。故敢強時(詞)奪理。我雖不殺伯仁。伯仁由我而死。這就是你罪案了。還有何辭。遂將先附逆朝臣二等例。擬他一個絞罪。衆皆無辭。只見鍾生起身。道。大人尊見自是不差。司官却不敢執筆。堂上道。你有何說。鍾生道。宦實依附忠賢。以朝廷之大臣。而屈膝於逆璫之門下。一死何足爲惜。若在當日逆璫事敗之時。同三案一體問罪。那有何說。如今已過了數年。且又奉過以後槪不株連之明旨。況昔日依附逆璫之人。漏網者多。今若重罪宦實。使人人自危。更開此吿訴之門。將來就不得安枕了。請大人上裁。內中一個右堂作色道。貴司念宦實鄕里之情。莫非黨護麼。鍾生道。宦實做官的時節。司官尚是貧士。雖與他同城。從無往來。後司官徼倖一第。也並不曾與宦實識面。司官所爭者。爲朝廷惜法。豈惜一宦實耶。正堂道。何爲惜法。鍾生道。王言如綸。其出如綍。旣已奉過聖旨。豈可因一宦實。而使朝廷之綸音二三其說。將來何以取信於天下。原來這刑部尚書與宦實也是年家。雖有心爲他。怎肯捨己救人。今聽見鍾生說到此處。連連點頭道。言故有理。只恐不能挽回聖怒。鍾生道。大人請想。司官愚見。宦實當日在逆璫門下。奴顏婢膝之事則有之。若謂助彼爲惡則未必。逆璫收敗之初。助惡者數百人。一時盡皆獲罪。若宦實果係黨惡。豈無仇家舉首。直至今日。以陳忠無據之案。擬以一死。未免太過。況逆璫革逐陳御史。又並無宦實之實跡。即欲恣(治)罪。不過依三等逆黨株連者革職而已。以莫須有三字加〈一〉人一死。司官不敢。上堂遲疑不決。吩咐將宦實收監。明日再議。遂大家散了回家。宦實到了監中。因適間堂上要擬絞罪。料辯也無一(益)。魂已飛去。不知何往。忽見這樣二十多歲的一個司官上堂。再三替他分辯。感激不盡。後聽得說是他鄕里。他暗道。我南京鄕親在京爲官者。無不相識。爲何遺漏此人\footnote{此語足見鍾生養身之高。不肯自做呈身御史也。}。不知他姓甚名誰。心內躊躇。他但雖有罪。原是大老。司獄司少不得要來見見。坐下說話時問他。方知叫做鍾情。現任員外。獄官去後。他心中暗想。如何得個門路再去求他相救便好。又無可托之人。正然低着頭閉了眼納悶。忽聽見一個禁子進來說道。大爺來了。忙睜目擡頭一骨(看)。果然是宦蕚。又驚又喜。驚的是他來不知家中有何事故。喜的是他來可通鍾生道門路。忙立起。問道。你來做甚麼。宦蕚見父親受了一番風霜辛苦。又着了這一場驚恐。憔悴不甚(堪)。跪倒在地。痛哭了一場。宦實也落了幾點淚。叫他坐下。問他來的緣故。他近前低聲說。父親起身之後。本要同來。想了無益。在家想商量設法求〔救。〕因官校聽着不好說得。後劉姑父也來說叫尋門路。因把他同衆人商量尋鍾員外的話細說了。今日纔趕到。想要到我二舅子家去住。恐怕不便。尋了下處。安定行李。幷帶來的數目說了。此時來請問父親主意如何。好煩梅生到鍾家去說。宦實聽了。喜不自勝。也將今日審的話吿訴他。堂上定了絞罪。鍾員外執定不肯畫押。我正想無人去求他。你來得正好。不可遲了。今晚就煩梅生去。恐明日定案。宦蕚聽說。也是歡喜非常。即回寓所。托梅生速去。許餽鍾生萬(千)金。梅生聞得宦蕚說鍾生這一番話。也自暗喜。這叫個因風吹火。用力不多。此係鍾生力要救他。比不得是我生生的去央情。這一事完。千金豈非囊中之物。忙忙的尋到鍾生私宅來拜。鍾生方下了衙門。不多時。聽得梅生遠來。心中甚喜。眞是倒屣忙迎接了進來。讓到書房中。敍了些寒溫。說了些彼此久濶思慕的話。鍾生道。兄何得有此高興。三千遠來賜顧。梅生命迴避了衆人。遂道。弟渴想兄久矣。因家寒不能遠來。遂將宦蕚約了同來。求他轉尋門路救他父親的話說了。又說宦蕚纔到監中見他父親。說蒙兄力救。感戴不已。求其始終救拔。原(願)以千金爲報。鍾生笑道。故人何不救我。我做窮秀才時。不肯絲毫苟且。今日徼幸爲朝廷臣子。豈肯受人賄賂。私幕夜之金耶。若宦公之罪應死。雖以百萬爲餽。亦不能免。罪旣不當死。一文又不應受。兄去覆他。他盛情我但心領。我若不做官。他令尊生死我不敢保。若弟在衙門中。他決無死法。梅生見他說得斬釘截鐵。事有成局。私心竊喜。辭了要去。鍾生留他下榻。梅生道。弟去將兄這番盛情意說與他知道。使他父子好放心些。且弟未得就回。盤桓有日。鍾生只得放他去了。回到寓中。自然添些話頭。說虧他盡心進言。並鍾生回覆的言語說了。宦蕚忙報知他父親。父子暗暗歡喜。次日。堂上又議宦實的罪。鍾生執定前議。堂上道。倘聖怒不測。奈何。鍾生奮然道。觸聖怒。大人以司官一人當之。勿貽衆累。堂上連道。好鐵漢。好鐵漢。不依(意)你一靑年人有此膽量。我不如也。旣如此。你具個揭帖來。我好做個憑據啓奏。這是正堂一來要救宦實。二來恐累了自己。若動聖怒。拿他來當災的意思\footnote{這正堂是小人心胸。然肯顧年誼。還是小人中之君子。}。那鍾生欣然具揭帖呈上。道。

\begin{quotation}

宦實雖係逆璫門下。但殺人害人之事毫無實據。且事在赦前。若加以重辟。恐於槪不株連之明旨不合。云云。

\end{quotation}

正堂就據了他的話題上本去。崇禎看了正本上說得有理。旣無實據。又果係赦後的事。批了個該部議處具奏。大家又議了一番。定了個他身爲大臣。依無(靠)權璫。本身削誥命。追出祖父封贈。革除兒子恩廕。復了上去。奉旨依議。監中提出宦實。高宣了聖旨。釋放刑具出來。宦蕚同梅生侯捷鄔合都在衙門前接着。大家那歡喜那裡還了得。侯捷要接到他家去住。宦實因一行有二十餘人。不便攪擾。力辭了。同到寓處。一場天大的禍。虧鍾生得放。保全了身家性命。父子二人那裡感激得盡。次日。父子二人攜了八百兩黃物。二千兩白金。同梅生到鍾生私宅來拜謝。鄔合也跟了去見見。鍾生正在家中。先不欲會。因他是前輩大老。且又是同鄕。不好辭得。只得迎了出來。讓到廳上。宦實一揖。先跪下去。道。老夫這一番上致君怒。以爲必死無疑。不意蒙先生恩力救拔殘喘。老夫有生之年。皆先生之賜也。敬來叩謝。鍾生慌忙扶住。拜倒在地。道。老先生請自重。晚生此一番爲朝廷惜法耳。並非爲靑天而掃浮雲。何敢當老先生屈尊言謝\footnote{有此大德於人。而不肯居功。誠君子人也。較今日稍有小惠及人。而滿面便有驕色。視此爲何如。}。彼此拜過。宦蕚也過來拜謝。並道及向年開罪。多蒙原宥。鍾生還禮。道。向承厚賜。雖不曾拜領。心感久矣\footnote{宦蕚之於鍾生。與向年在錢貴家罵小畜生時何如。憶余向年有一相識楊愛生。彼之姪孫僅十五歲。在楊公祠讀書。即彼家之家廟也。余一日偶同數交(友)他遊。過此暫歇。有一輕薄友。見彼幼而美。以言戲之。彼曰。你同我頑。我吿訴爺爺呢。執(孰)意彼當年進學。次年中鄕榜。連捷進士。入祠(詞)林。整二十個月回家祭祖。巍巍然楊老爺矣。因想萬盤(般)比(皆)下品。唯有讀書高二句。誠然哉。}。鄔合也過來拜見了坐下。茶罷。宦實道。先生活命之恩。無以爲報。具有不腆之儀。聊盡愚父子一點鄙衷。其深厚之恩私。唯有子子孫孫頂祝而已。叫家人擡過兩架大食盒來。宦蕚在袖中取出禮帖遞過。鍾生一看。謹具黃米八百擔。白米二千擔。笑着道。先生何故見賜。宦實道。些微之敬。不足以報滑(涓)涯之萬一。希爲莞納。容圖異日。鍾生艴然道。老先生尊見差了。晚生盡力奉救者。本爲秉公。並無私念。老先生若以此相加。是晚生假公濟私了。使外人聞知。晚生上獲罪於朝廷。並獲罪於堂上了。盛情心領。堅持不受。宦實幾墮下淚來。道。老杇以垂白之年得保首領者。先生之賜也。先生欲爲古道君子。使老朽爲負德小人。鄙心何安。鍾生見他情意十分諄切。說到了這話。倒不好過於推辭。便道。罷。老先生如此見愛。晚生再過却。反獲罪於長者了。請將黃物收回。命取過二千兩銀子來。將一千送與梅生。道。弟念兄之情久矣。無以爲敬。今借此轉敬。聊表當年相愛之雅\footnote{千飯千金。何況自幼莫逆。送得當。}。宦蕚道。梅兄俟回府後。小弟自厚酬。以答驅馳跋涉之勞。何須先生費心。鍾生道。此乃弟贈故人耳。非爲酬勞也。梅生故要遜謝。鍾生道。我與兄異姓骨肉。不必做客套故謙。又將百金送與鄔合。道。聊贈故人。以當一飯\footnote{鍾生平生已知。梅生自幼契合。錢貴初遇即托終身。鄔合一見即知其爲盛德君子。只此三人耳。鄔合能識。鍾生不識鄔合。可見知人之難。鍾生不過以篾視之。故贈之也輕。足見世上取人當於牝牡驪黃之外。不可以所處之地而視之也。}。鄔合推辭幾句。也就拜謝受了。復將三百金付與梅生。道。此物兄到家時轉付家岳母。酬他當日不受聘金之情。復轉身向宦實道。承老先生厚愛光臨。晚生本當異日治一杯魯酒爲敬。恐老先生念尊府懸罣。歸期忽迫。不敢留駕。此六百金爲老先生賢橋(喬)梓途中一飯之需。以當薄敬罷。宦實見他一文不受。過意不去。道。先生尊諭。別的奉命了。這些微之物。老朽還領回。眞要愧死了。鍾生道。不然。盛情晚生算(心)領。此又算晚生轉敬老先生。何須謙得。若老先生不受。晚生連那千餘金也就璧謝了。宦實見他執意如此。知不可強。起身吿辭。謝之再三。臨出門。鍾生對梅生道。本當留兄盤桓數月。但兄攜此重資。他日孤行不便。還是伴宦老先生同回府罷。但故人遠來。恝然而別。難爲情耳。梅生見他想得有理。也就辭了回寓。宦實歸家心切。連夜雇了轎夫頭口。次早一同回南而去。宦實恐家中罣慮。先差兩個家人星夜回家報信。自己坐了一乘大轎。衆人皆騎脚騾。一路無話。十數日趕到了家。他一家歡喜是不消說。男女大小無一個不感念鍾生。宦蕚謝了梅生千金。謝了郝氏二百金。鄔合百金\footnote{尋鍾生之策出於鄔合。今宦蕚謝梅生郝氏重。而謝鄔合輕。焦頭爛額爲上客。曲突移薪受薄賞矣。}。梅生陡發二千金。不用說歡喜感激鍾情之情。就是郝氏也得了五百金。鄔合得了二〔百〕金。你說他們感念不感念。鍾生又做了二年官。見流寇猖獗。朝政日非。他感慨自任。道。國家之事已至於此。竟無一人敢言。可謂士風掃地矣。我一介寒儒。食祿數載。今拚此一官。上言得失。以報聖恩。復嘆道。可惜樂老師吿病歸去。他若在朝。乃皇上得用重臣。必有諷諫。或尚不至此。今日我若不言。再無人敢言矣\footnote{此語愧殺那時臣宰。}。他一日見堂上。說道。太監監軍。天下事壞至於此。老大人爲朝廷大臣。忍坐視不一言耶。堂上道。我豈不知。但事出有(自)聖心。不敢觸皇上之忌耳。鍾生艴然道。老大人不言。司官當言之。司官一介微員。又職非言路。自知言出禍隨。但食君之祿。不敢尸位耳。或能以一死感悟君心。亦可含笑於地下。堂上嘆了幾聲。勸他道。子之忠忱固可嘉。但舉朝王公將相文武大臣皆緘默不言。豈皆無忠心愛朝廷者。皆知言之不但無益。而且有禍。所以皆鈐口耳。君子知機。明哲保身。也不可不知\footnote{尸位素餐之徒。無不借此語以爲口舌。}。你又何苦批逆鱗以賈禍。殺身成仁固是好事。但古人云。願爲良臣。不願爲忠臣。懼殺身以成君過耳。鍾生長太息道。食人之食者。〈惡〉忠人之事。司官但知忠其事而已。以報數年之恩。此微軀不暇惜也。昔日世宗皇帝說海剛峰先生道。大臣不敢言而小臣言之。此司官今日之謂。不然。何得今日便不如昔。豈不畏爲先賢所笑。堂上見勸他執意不回。暗暗贊嘆自愧。鍾生回到家中。連夜修了一本。次日親自送到通政司去。煩他上呈。其大略云。

\begin{quotation}

太祖高皇帝辛苦百戰。混一四海。定鼎以來。列聖相承。迄今將三百載矣。天下生(升)平。萬邦樂業。自我皇上御極之始。勵精圖治。首誅逆璫。次除附惡。朝野仰其天威。臣民蒙其聖庇。自崇禎三年。李自成創逆於陝西。張獻忠流氛於西蜀。迨至今日。川湖一帶數百萬之生靈。盡高(膏)鋒鏑。山陝二西幾千里之城郭。皆做丘墟。以朝廷之金甌。成蕭條之草莽。傷心慘目。尚可言耶。此猶其次也。賊殘鳳陽。震驚陵寢。寇屠各省。戮及宗藩。此正臣子錐心泣血。誓不俱生之時也。而陛下屢屢命將興師。賊勢愈加猖獗而不能撲滅者何故。皆緣內臣監軍所致耳。內臣所向。妄自尊大。有謀勇之將。動則爲其掣肘。無才之技徒。借彼爲之護身人。人皆知此害。無一人敢爲陛下陳之。眞可痛哭淚涕而長太息者也。更有可憂者。宰輔重臣。朝廷之股肱也。明知此害。保爵固位。鉗默不言。此大臣疏陛下也。九卿暨闔朝文武。朝廷之耳目也。借以推諉曰。宰輔猶不言。我曷敢言之。此近臣疏陛下也。外之經略閫師。巡撫總兵。皆朝廷之封疆大臣也。咸曰。勝則歸功於監軍之內臣。敢(敗)則加罪於勦賊之將師。皆袖手傍觀。逡巡畏避。所以賊勢日張。寇氛逾熾。明爲內臣監軍之故。而亦不言。僉曰。朝廷之重臣尚俱爲磨兜監。我輩閫外之臣耳。又何敢言之。此封疆大臣疏陛下也。至於各城武弁。守土文臣。有忠義者。賊至則與城俱亡。無廉恥者。寇臨則率土附順。亦曷嘗不知內臣之害。皆異口同聲曰。我小臣也。雖欲言之。亦不能上達九重。是天下之臣工\endnotemark[8]皆疏陛下也。此猶謂異姓之臣也。諸王公將軍。天潢一派。皇族分源。貴戚之卿也。亦不復一言。此親囗疏陛下也。在今日。陛下可爲孤立。可爲寒心。爲今之際。唯有急撤回內臣。責任統帥。庶幾賊可撲滅。奏功有日。若陛下不奮大乾斷。天下事將來有不可言者。小臣不忍坐視狂瞽。冒死上言。不勝激切待命之至。

\end{quotation}

崇禎見了這本。大怒。御批道。

\begin{quotation}

鍾情何物小臣。敢越職妄言。阻撓大計。本當重處。姑念無知。着交與鎭撫司。好生重打。百(再)發往邊衛充軍。欽此。

\end{quotation}

旨意一下。這些在廷諸臣。誰不知內臣之害。但出自聖心。不敢進諫。今見鍾生這本。內中連着他們。也有惱他的。也有些忠義之心的。憐敬他明目張膽。敢直言上諫。約了二十餘人。親求面駕。乞恩寬恕。他的同年有在翰林的。有在科道的。兩衙門的。在部屬的。都被他這本激起忠義之氣來。糾齊了到午門外俯伏。情願替他分罪。崇禎這日駕御瀛臺。見多官如此。聖怒雖稍息。猶未下寬貸之旨。向首輔周延儒道。小臣無知。他謂臣(朕)不當用內臣監軍。但今日無岳飛其人耳。若有那樣大將。醜賊何足平。周延儒奏道。人臣能盡忠於國家。史即多溢美之辭。岳飛亦後人之溢美耳。如今日鍾情倘受廷杖而斃。後人亦曰惜殺此忠〈忠〉諫之臣耳。若從其言。流寇豈足平耶。槪如此耳\footnote{諷諫得好。不救之救。}。崇禎瞿然道。如先生言。鍾情當何以處之。周延儒奏道。天恩出自聖裁。臣何敢妄議。崇禎復向衆臣道。你諸臣公議。當作何議處。衆臣叩首道。鍾情新進無知。不識忌諱\footnote{語中有刺。}。勒令致仕。以張陛下天下之洪仁。臣等皆戴天恩無盡矣。崇禎方纔允了。傳出旨來。放了綁。聖怒正稍息。忽登聞院呈一個本來。崇禎展開看。道。

\begin{quotation}

翰林院編修臣關爵。誠惶誠恐。冒死上言。臣聞古云。木從繩則直。后(君)從諫則聖。又云。君聖則臣直。今日太監監中。不但文武大小臣工知其不可。即閭閻之下愚夫愚婦。亦皆知其不可也。竟無一人敢爲陛下陳之。臣每每無齒痛心。但恨臣位居下僚。職非言路。雖有忠君愛國之心。不能上達。今刑部員外臣鍾情。敢犯顏直諫。眞可謂鳳鳴朝岡。廷臣皆以爲皇上必採納其言。定膺上賞。不意反上於天怒。廷杖遣戍。鍾生一柔弱書生。受杖必斃。皇上上比唐虞。豈可有殺忠諫之名。萬世後視陛下爲何如主。仰乞天恩。赦其罪而賞其功。作在廷諸臣忠義之氣。若陛下不(必)欲死鍾情。臣願與之同死。得從龍逢比干。同遊於地下。爲榮多矣。臣愚昧無知。冒死擊登聞上奏。無非愛君之心。雖因鐵鉞。亦非顧也。不勝待命之至。

\end{quotation}

崇禎大怒。道。關爵以朕爲紂桀耶。交與錦衣衛。好生打着。問是誰人指使。審明白回語(話)。衆臣又奏道。陛下旣恕鍾情。關爵亦仰天恩赦宥。崇禎仰面作色道。他比朕爲紂桀。從子孫罵祖父母父母。律其罪應死。尚可恕耶。衆臣道。彼何敢。關爵所言。欲求皇上爲堯舜之君。不宜爲桀紂之事耳。焉敢以桀紂比陛下。聖怒尚未息。大學士程國祥免冠叩首。\endnotemark[9]道。老臣犬馬之齒已邁。徒受聖恩。毫無補於朝廷。願納上官誥。以贖關爵之罪。崇禎見衆臣諄諄乞恩。老閣臣又免冠叩求。不得已說道。先生冠。臣(朕)爲諸臣。姑恕之。關爵着革職爲民。回籍當差。衆臣見饒了他性命。已出萬幸。可還敢再奏復他官爵。皆謝恩而退。你道這程閣老他却是爲何這樣苦救關爵。一來是他一片忠忱。二來他與閣(關)爵有些情義。程閣老自幼無父。家極貧寒。祖籍南京。上元縣百姓。他十數歲時。做牛角牛骨簪子賣錢養母。他家住在盧妃巷武學後街兩間小房內。每早挑了擔子到內橋頂上銼磨簪子出賣。日夜辛苦。僅能餬口。一日。上元縣知縣在橋上過。程閣老因低着頭銼磨簪子。不曾站起。那知縣看見。怒道。少年人便如此大膽。藐視官長。當街責五板\footnote{程閣老虧此知縣一激而發。亦有(如)韓信之遇淮陰二少年。}。他氣憤起來。道。做官也不過讀書人起的。我難道就讀不得書。做不得官的麼。遂將擔子並傢伙摔得粉碎。歸家向母親哭訴。要去從師就學。母親道。旣有志上進。是極好的事。我家中辛苦紡績。或可得供柴米。但學錢無可奈何。又想了想。道。也講不得。我再忍飢受餓。每日幾文積下以做束脩。成你讀書之志\footnote{賢哉母也。非此母焉能生此子。}。他次日就到一個學館中去投師。那先生就是關爵的王父。是個年高飽學盛德名儒。學生中多有認得他的。向先生道。他是每常在內橋頂上銼骨頭簪子賣的小程。他也來念甚麼書。關先生見他十五六歲纔來開蒙。問其緣故。他將無父家寒。並做簪受責。發憤讀書的話。哭訴與先生。這關先生大喜。道。古云。有志者事竟成。更有二句道得妙。

\begin{quotation}

朱門生餓莩。白屋出公卿。

\end{quotation}

你旣有這一番奮志。焉知你異日不爲朝廷卿相。因取學名爲國祥。又道。你旣家寒。但願你肯讀。那裡爭你一個人的束脩。我不要你的。他感激先生了不得。果然日夜用功。寒暑無間。不數年。讀了滿腹文章。皇天不負苦心人。後來竟連捷中了。歷仕到了閣下。但他做了一生淸官。古人還有一琴一鶴。他連琴弦也沒一條。鶴毛也沒一根。家中舉動。有貧士所不堪者。屢欲報德(答)師恩。不囗爲情。今見關爵是他的世姪。常常在一處談講。因老師世兄皆故。只有他在。愛他如嫡親子姪一般。他今爲了事。且又是一片忠肝義膽。上爲朝廷。下爲年誼。觸了聖怒。可有不竭力援救。出了朝。就同關爵到了私宅。說道。我素知老賢姪以淸白自持。定宦囊羞澀。也與老夫一般。目今時事日非。我進言不納。旣不能匡君輔政。徒做這伴食中書。也無顏久駐。我辭了官。與賢姪一同回去罷。次日。即上疏吿老。崇禎不准。疏凡七上。纔依了。他收拾了行裝。人口不多。關爵也不多的家眷。雇了兩隻民船。自己坐了一隻。與關爵坐了一隻。一齊回南。關爵他祖上有些田在和州孝義鄕。他父親後來就遷往和州鄕中去住。他同程閣老到了南京。然後辭了回去。這程閣老到了家鄕。連住房都沒有。雖人口不多。當年那二間小房如何住得。他的子姪親友們大家公湊。買了上元縣內橋西武學隔壁珠寶廊對過一所宅子。送他住下。他秋冬穿的是一件紫紅布綿道袍。春夏是一件單的。仍然寒士規模。他也不交接一個朋友。只有一個向年同窗讀書的老友。姓白字秀生。人因他是個老童。都稱他爲白秀。每常講(請)他到家閒談。他二人常在花廳西南角小小一間樓上起坐。三文錢沽四兩燒酒對酌。晚間無油點燈。黑影裡看不見滿淺。酒杯中放指頭大一塊烰炭。斟酒是(至)炭浮起。便知是滿了。間或取出幾個饅頭來相待。上面的白毛將有一寸長。餿臭不可聞。白秀不能下嚥。他自己吃得香甜之極。白秀常向人以做笑談。至於魚肉之數(屬)。是成月不得一見。但可惜這樣一個淸官却無後事(嗣)。古來鄧伯道無兒。寇萊公乏嗣。天道難窺。千古同聲一嘆。再者如今人做了一位知縣知州回來。成千成萬的銀子馱到家。美酒羊羔。冬裘夏葛。嬌妻艷妾。呼奴使婢的受用。何況位至閣老。像這〔樣〕的淸官。眞是國家的祥瑞。千百年僅見其一者\footnote{我朝亦有兩江總督于淸端公諱成龍者。}。向日關先生命名。一毫不謬。反有一種無知小人笑他。道他是個眞呆子。做了這樣大官。還不會享福。可謂惡居下流而訕上矣。且說那關爵。他夫人逯氏。子名關必顯。\endnotemark[10]他做秀才時。西鄰有一家姓閻名良。字煥文。妻子創氏。他祖上原是外國人。他有兩個女兒。長名貴姐。次名富姐。他夫婦二人趨炎附勢。做盡醜態。那樣式眞令人看不得。家中也有三二千金過活。他之西鄰。又有一家姓傅名厚。兒子名喚傅金。是個土財主。有數千金之產。傅厚納了個監生。在鄕中眞算是頭一個大鄕紳了。狂妄得不知多大。竟像天底下沒處放他的樣子。這關爵雖是個秀才。却家道貧寒。每常這閻良傅厚偶然或在途中遇見。連話都不說。猶恐怕窮氣過到他身上一般。遠遠一拱即避開。那年關爵同鍾生一科中了回來。知州親來送匾。城中鄕紳舉監賀客塡門。關爵不得不治酒相待。他自己一人持不來。因閻良是緊鄰。約他來陪客。那閻良是一個村中鄕老。生平不曾會過大賓。今日托關爵的體面。竟同這些衣冠中人揖讓同席起來。覺得骨頭都輕了好些。渾身上下就像有幾千萬蝨子爬的相似。無處不是亂癢。好生快活。他高興起來。也送了一分厚禮賀金。又請酒道喜。就打動了他一個趨附仰攀的念頭。央煩傅厚到關家去說情。願把女兒嫁與他爲媳。把兩個女兒的八字都送了來。兩個中任憑選擇一個。傅厚向關爵說了。關爵道。承他厚情要說做親。他大令愛與小兒同庚。自然就定大的了。那有選擇的理。但弟雖徼幸一第。仍然貧士。不能仰攀。傅厚回了他的話。見關爵口聲願要。但不過說是窮。他又煩傅厚來說。一絲一毫不要。不拘怎麼樣。但聽府上尊便。關爵見兒子也大了。巴不得替他娶媳婦。完了一場大事。見閻家如此趕上門來。可還有不依的。況他家女兒。關奶奶也曾見過。大女兒雖不及妹子標致。却生得莊重敦實。遂將家中所有的首飾衣服之類添補了些。將就行了聘。關爵也煩傅厚去說。歲內要完成了兒女的事。纔往京中去會試。閻良可有個不奉命的。悉聽尊裁。關家擇日迎娶媳婦進門。閻良也賠了有百餘金之物。還有一個丫頭。關爵次年臨起身。也請酒送行。又贈路費二十兩。關爵倒也深感他的盛情。關爵到京。又同鍾生中了進士。選了庶吉士。後來鍾生放了部屬。他陞了編修。差人般(搬)取家眷。那家中的熱鬧還了得。不但那鄕中人。就是那城中沾親帶故的。見州裡出了個翰林。那來趨奉的人眞個其門如市。那閻良有了這親家。就像翰林院是他自己的一般。又快活又臊皮。不知不覺大了許多。見人說話聲氣也響了些。走路肚子腆着。腰也硬了些。逢人沒有個舍親關老爺不開口。創氏奉承親母女兒。一口一個親家太太姑奶奶。強說強笑的容悅。他夫妻二人。恨不得把親母女婿女兒頂在頭上過日子。傅厚因閻良有了這翰林親家。想要因親及親的借光。求他女兒富姐娶與兒子傅金。閻良夫妻見他是財主監生。自然喜允。兩家結了親。傅厚同關家算四門親家了。也來湊熱鬧。送賀禮。送路費。到關奶奶起身之日。閻良送了許多米麪吃食。又送盤纏四十兩。極盡親親之誼。關家母親也十分深感。關爵在翰林淸淡衙門做了幾年冷曹。今日削籍爲民。到了家。還是那寒酸氣象。當日來趨奉的那些親友半個也無\footnote{無怪其然。人之半個如何還得來。}。連閻老親翁只互相一拜。茶也不留一鍾。貴姐去看父母。相別了幾年。一句親熱話也沒有。連飯也不留一頓。倒是閻良心裡還過不去。向創氏道。老關一家回來了。我們或是備席酒請請。或是將就送分下程遮遮臉。不然太覺得炎涼了。不好意思的。撒把土也迷迷後人的眼。不要太做絕了。創氏道。呸。我問你這不好意思有多大小。當日爲他家。不知花了我們多少瞎錢。以爲後來靠親家有好處來。把個女兒也白給了他。這幾年我們連半個底錢也沒有見他的。今日這樣個嘴巴骨子回來。還理他做甚麼\footnote{甚矣。炎涼者尚稍有人心。不似臭(創)氏之絕情絕義也。}。你要請要送。你拿錢去用。我是沒有的。窮神的燒紙退送他。還怕去的不遠。你還要招攬他呢。你敢是拾着倒運的票子了。那閻良素常有幾分懼內。不敢不遵。此後兩親家總不大上門。淡然而已。他夫妻更有可笑之處。當日叫關必顯口口聲聲姑爺。今稱女婿。叫貴姐不但不呼姑奶奶。好則稱曰大姑娘。不然則叫大姐。叫傅金富姐。仍是姑爺姑奶奶。那富姐已嫁了傅家。見姐姐家寒。生怕他們\endnotemark[11]借東借西。見面連話也不多說。那傅厚父子越發不消說得。偶然相遇。一拱即別開。關爵見他們這種光景。惟有腹中暗笑。且權擱起。再說鍾生那日在午門外放了出來。他毫無慍色。到寓。連夜收拾回家。也有人愛他是個豪傑。想要送他。恐有朝廷耳目。不敢相親。鍾生做官一場。並無私富(蓄)。唯有兩袖淸風。踽踽涼涼。帶領妻妾兒子。此時錢貴生了一子已四歲。代目也生了一子兩週多了。雇了轎車。到張家灣來。先差家人鍾用去尋店安歇。並雇船隻。鍾用到了那裡。看見一個衝天大招牌。上寫道。

\begin{quotation}

戴家老行。包寫南京各省官座大小船隻。不誤主顧。

\end{quotation}

他便進去問南京的船。一個四十多歲掌櫃的問道。是那位老爺要往南京去。鍾用道。是刑部鍾老爺。原是南京人。如今要回家去。便問道。你們這裡那裡有好店口。我們老爺奶奶權住兩日。好等雇船。那掌櫃的道。這位老爺可是人稱他鍾重金的麼。鍾用道。正是。那掌櫃的道。鍾老爺旣是我們同鄕。又是素常聞名的好官。何必下店。那店中人雜。家眷住着也不便宜。我舍下房子儘寬大。騰幾間將就住着。過兩日等我效勞。看有回頭的民座。價錢賤些的。雇一隻去。鍾用見說。再三道了多謝。忙回舊路。迎着鍾生說了。鍾生甚喜。就到他家來。剛纔把上房騰開。讓了內眷入去。這掌櫃的同他個七十多歲的老叔叔。陪着鍾生客廳內坐。鍾生深謝借房盛情。那老人道。老爺大名。這幾年來來往往的人傳說。老漢聞知久了。今日幸得到寒舍。眞是蓬蓽生輝。況在同鄕。禮當接待的。鍾生道。老丈來了多少年了。他道。老漢來久了。舍姪纔來不上幾年。正然大家閒話。忽聽見裡面幾個婦人哭聲震耳。鍾生吃了一驚。正要叫人去問。只見一個僕婦走出來。道。奶奶叫請老爺陪這位戴太爺戴大爺進去。鍾生驚疑。忙同那老兒叔姪進去。你道是甚緣故。先錢貴同代目下車時。這家一個老婦人同一個媳婦出來接着。讓到上房坐下。稱錢貴爲大奶奶。代目爲二奶奶。彼此說話。那代目看他婆媳兩個很像他的祖母母親。心中想道。他們在南京。如何到得這裡。大約是形狀相似。那兩個婦人也不住看他。又聽得都是南京語音。忍不住問那中年婦人道。府上貴姓。婦人答道。寒家姓戴。代目心下一驚。道。也姓戴。又問道。奶奶。你貴姓。答道。我賤姓〔那。〕代目忙指着那老婦道。這位老奶奶尊姓可是繆。那老婦聽了。驚道。二奶奶。你怎知我姓繆。代目急站起身。上前兩隻手拉着他婆媳二人。道。有一位名戴遷的。可是一家麼。那老婦道。就是我的兒子。代目一把抱着那老婦。跪倒大哭道。奶奶。你又(不)認得我了麼。就是賣與鐵家。你的孫女兒了。那老婦聽說。又忙把他看了一看。叫了一聲。我的親兒囉。想死我了。本日同你在這裡相會。不是做夢麼。於是一把拉起。抱着他痛哭。那氏也拉着他。兒呀肉呀的哭起來。錢貴起來。忙叫僕女請了鍾生同他叔爺並他父親進來相會。哭了一場。悲喜交集。他叔叔同他兩個兄弟都來相見。那氏又帶他去見了小嬸。祖母蕭氏。蕭氏有病。故不能出來。然後〈來〉大家坐下。戴遷問他道。數年前我到鐵家去贖你。說已賠與童家。及至到童家去問。又說嫁到外路去了。如何得隨了鍾老爺。代目不好細呈錢貴履歷。但說。鐵家姑娘待我甚好。吩咐家人叫把我嫁個好人家去。那家人壞心。瞞了主母。把我又賣到奶奶跟前。蒙奶奶恩典。待我如同女兒一樣。後跟着嫁了過來。叫我跟了老爺。他一家又向鍾生錢貴多多拜謝。有一個淸江引兒說他家此時的光景。道。

\begin{quotation}

嬌兒自與爲奴去。我到京來住。抛離十數年。喜得今團聚。謝蒼天。笑容兒頻堆起。

\end{quotation}

錢貴又叫代目抱他生的兒子與衆人看。那孩子眞是眉淸目秀。齒白唇紅。粉團般好個相貌。他們見了這樣個好齊整外甥。分外歡喜。忙治酒接風。次日又備席。會親慶喜。每日款待得十分豐厚。又替兩個孩子做衣服鞋襪。鍾生見他每日豐盛款待。過意不去。托他雇船要行。他一家那裡肯依。定要留着多住些時。鍾生見他情急(意)殷殷。二來又因代目相離了祖母父母十多年。纔得相會。只得住下。一日無事。偶到河岸邊閒行。看那往來的船隻。只見數隻彩畫簇新的一大座船。泊在河下。吹吹打打。好不熱鬧。鍾生竚立長久。只見船上走下一個戴纏粽帽。穿靑絹直緞的管家來。問鍾生的家人道。這位老爺尊姓貴職。家人道。姓鍾。是刑部員外。那人又問道。老爺貴處是那裡。鍾生聽見問他。便道。我是南京人。你問我做甚麼。那人忙陪笑臉。垂手側立。說道。方纔夫人在窗內看見。叫來問的。鍾生道。你們老爺是誰。貴姓甚麼。是那裡人。夫人爲何問我。那人道。家主姓榮。是湖廣人。前任江西\endnotemark[12]撫院。新任禮部侍郞。夫人是南京人。差了來問。不知是甚緣故。鍾生也不再問。那人上船去了。鍾生滿心疑惑。道。他夫人是南京人。莫不是那個親戚家女兒嫁到湖廣去的。但我小時貧窮。也並不認得甚麼親戚。他如何認得我。猜測不出。方要轉身。只見先那管家抱(跑)了來。道。家主在船上拱候。請老爺上船相會。鍾生見他是現在大老。不便褻衣相見。叫家人去取大服。只見那榮侍郞立在船頭上。說道。途路間不必拘之。請上船來罷。鍾老爺見他在那裡候着。忙往跳板上走了上去。榮侍郞滿面春風迎着道。久慕了。鍾生忙深深一恭。道。不敢。晚生並不曾拜謁過尊顏。老先生何以見愛若此。榮侍郞笑道。我學生雖不曾會過。却有一個當日在南京受過先生大恩的人認得。鍾生道。晚生那時在家尚是一介寒儒。自給不暇。焉得有恩到人。榮侍郞道。先生且請進艙。頃刻便知。相讓到了艙中。禮畢坐下。榮侍郞問了些南京話。並問及何故在此。鍾生將上本觸了聖怒。虧諸公保救。休致回家。細細說了。榮公着實贊嘆不已。只見一個丫鬟掀着內艙門簾。道。夫人出來了。鍾生迴避不及。鞠躬而立。見那夫人有三十年紀。滿頭珠翠。遍體羅綺。丫鬟僕婦簇擁。鍾生低頭不敢仰視。又見兩個丫鬟鋪下床紅氈。一個僕婦說道。夫人拜謝鍾老爺。那夫人站在氈上拜了兩拜。就跪將下去。驚得鍾生忙拜倒。說道。晚生並不知是何緣故。恐夫人錯認了。怎敢勞尊。夫人請自重。那夫人拜畢。讓着鍾生一同起來。請鍾生客位坐了。夫人與榮公並坐在主位。那夫人忽然開口道。恩人。你可記得那年七月大雨之後。水塘中救的那個婦人。就是我。我終日感念深恩。不想在這裡相遇。鍾生方知是當年救的你(那)個郗氏。你道這欷(郗)氏一個窮得要死的婦人。如何到了這步地位。俗話道。人不可貌相。海水不可斗量。況婦人們裙帶上的衣食更定不得。他丈夫充好古那時帶了個小夥子到家。要將他陰物換屁股的。誰是(知)就是游夏流的厚友楊爲英。那充好古偶然在個朋友家看見了他。心愛至極。却手頭沒鈔。楊爲英如何肯白捨屁股與他弄。他情急了。暗地同他商議。將妻子之牝物換他尊臀。做個彼此交易而退之意。這小子乖滑之甚。先要看看婦人生得如何。方肯依允。充好古領他家中來。他見了郗氏果然生得好。十分情願。充好古以爲男人納寵是件歡喜的事。他今日替妻子納個小夫。滿心以爲郗氏必定樂從。他又得嘗新臟。不想郗氏不但不笑納。而且一番大罵。眞罵了個狗血噴頭。他掃興而出。那心中的恨。竟像有不共戴天的忿怒。到外邊向楊爲英商議。把他屁股預先支用了。他將郗氏賣去。得了銀子。同他常做一對旱路夫妻。楊爲英先同游夏流契厚。後來游夏流娶了多銀。日裡在家中燒茶煮飯。夜裡舔得舌根酸疼要死。那裡還得來親厚到他。後來說宦公子愛他。滿心以爲賤股得貴人一番賞鑒。仗着錢大的這個肉眼。一生豐衣足食。是滿擬得的了。曷勝欣喜之至。不想被卜氏那一罵。宦蕚呆公子性的人。一團高興。心中着了一惱。連他都撇去腦後。他雖然在外邊。今日伴張。明日陪李。尋些零碎主顧。不過只可糊口。要想個多錢用用也不能夠。今日見充好古許他先且相好了。等賣了老婆償還他。他是個甚麼値錢的屁股。那糞門中也不知經過幾擔陽物的了\footnote{陽物論擔數。此話新鮮之甚。}。還做甚麼身分不成。就一諾無辭。晚間無處可做洞房。充好古當了一件布衫。買了半斤牛羓。同他沽飮了兩壺燒酒。乘着酒興。到一座空破五道廟中。在香案之上成其好事。那楊爲英怕自己的糞門大鬆得沒道理\footnote{趨(趣)談。}。恐招攬他不住。打脫了這肥主顧。故意做出百種騷淫之態。把個充好古神魂都被他攝去。深恨相會之晚。次日即到媒人家去。說他有個寡婦妹子不肯嫁人。如今要嫁他。只要多得些銀子。情願二分酬謝。或與人做小做婢。在京出〔京〕他都不管。只要速成。又向媒人說。要相會只好暗暗地去。恐他知道要尋死覓活。就是事成了。也只好哄了他擡去。到了人家。就不怕他跳到那裡去了。天地間可還有做媒人的〈沒〉良心。他只圖二八提蘭(籃)。厚得媒錢。那裡管人家婦女死活。那時正有一個過路上任去的榮巡撫。因無子息。要娶幾個美妾。因想南京的婦人生得嬌媚。叫媒人找尋。不論女兒寡婦都可。都要生得秀美。媒人聽得充好古說了。同到他家中來。充好古遠遠躱着。指了門與他。那媒婆\endnotemark[13]假意做進去借茶吃。見這郗氏生得果好。可惜是個窮苦日子磨滅壞了。若有些好的穿戴起來。雖不得一位絕色佳人。也就可稱是美婦了。回了榮巡撫的話。打發了家人同他暗暗地來相看。窮家小戶開了門就是臥室的。一到便見着了。甚是中意。覆了主人。講定價銀二百四十兩。做大官的人聽說人物生得好。那惜幾兩銀子。就兌銀擡人。充好古寫了文書。得了銀子。同媒人八刀了。他叫了頂轎子。就同媒人到了家門口。叫他在外等着。等上了轎。遠遠跟隨。送到榮巡撫船上說明白了。他便同轎子往家去。這正是投水的第二日。他淸早見鍾生回去。不多時。拖泥帶水的又來送他銀子衣服。已感他不盡。況又體貼。怕他餓了。恐一時無人換錢。還留下百文與他買點心且充飢。雖至親骨肉也沒有這樣相愛周到。感激了不得。所以欲將微軀相報。見他正言厲色推辭。又敬他。越感激他。買些點心吃了。將完(換)下泥汚濕衣在塘中洗淨曬乾。正思想煩甚麼人去換錢。忽見充好古引了一頂轎子來。道。你哥哥回來了。我纔到他家看他。他說。不得閒來看你。叫我帶來轎子來接你回去走走。你(那)郗氏正一腔怨恨無人可訴。聽見哥哥回來了來接。可有個不去的。那裡疑到是丈夫賣他。看那件布衫也乾了。穿將起來。就坐上轎子。那轎夫一直擡到旱西門來。他在轎中覺不像每常往哥哥家去的路。問那轎夫。他都是說同了的。也不答應。只是擡着走。不多時。到了右城橋側泊船處住下。那個媒婆趕上。叫他下了轎來。方低低吿訴他說。哥哥把他賣與榮巡撫做小了。那郗氏竟嚇癡了。忽掉下淚來。道。\endnotemark[14]這是那裡話。我哥哥不在家。況我有丈夫的。如何賣得我。媒人對他說了姓名形狀。郗氏道。這是我丈夫。那裡是我的哥哥。媒人道。你丈夫旣狠心賣你。你還戀他甚麼。你跟着那樣丈夫。幾時有個出頭的日子。你這樣美貌靑春。豈不躭誤了。如今榮老爺要做小奶奶。圖生子的。你若有造化。生下一男半女。一生受用不盡。況你丈夫旣賣了你。料道是回不去了。他賣你的時節。說是他的寡婦妹子。若老爺問你。也須這樣答應。你若說是他妻子。一個活人妻。將來就生了兒女。也沒光彩顏面。那郗氏到了這個場中也沒法了。那怨恨丈夫的心直入骨髓。也不下淚了。就同媒人上船來。到艙中叩見榮巡撫夫婦。榮公一見。十分歡喜。就吩咐掌家婆領他去洗沐了。渾身換了紬絹衣服。梳了頭戴上許多珠翠。那郗氏生了二十多歲。從不曾這樣體面過。忽然而得。不但不惱恨了。而且歡喜起來。晚間榮公就同他共宿。那繡帳高懸。錦衾重疊。睡在上面好生受用。比那床拔(板)鋪着一床燈草蓆。眞天淵之隔。每日佳肴美食。那裡吃得了。連鍾生與他的那三兩銀子也竟沒處去用。那榮巡撫見他容貌旣美。又和氣又溫柔。雖尋了三四百(個)女子。都不及他。竟有專房之寵。除了正夫人。就要數他了。他每每念及鍾生。就感之不置。一時恨起丈夫薄情。一個結髮夫妻這樣刻毒。更念鍾生一個陌路。又非貪色。這樣恩情畢至。越感念無比。隨到了江西任上。次年就生了一子。這榮巡撫偌大年紀。官居八座。纔得了這個活寶。眞比斗大的一顆明珠還値錢些。愛其子而及其母。先還是叫姨娘。此時竟稱起奶奶來了。二年後。大夫人病故。過了週年。這樣個大人家。沒有個夫人在內中統屬這些姬妾。可還行得。榮公不但是自來疼愛他。古語說。母以子貴。看兒子的面上。竟冊了正。公然一位三品淑人。他常想。若不是鍾相公救我。此時也不知死到那裡去了。如何得有今日。眞是重生父母。何日得報他的恩德。念念不忘。一日。夫妻閒話。他因說起家中舊事。不好說是丈夫。只說他哥哥怎樣沒良心。把他整日餓着。總不管閒事。因苦極了。去投水。虧得一個姓鍾的書生怎樣救他。如何與他盤纏衣服。不想就是那一日。我哥哥把我賣到這裡來。有了今日這日。何日纔得報他的恩惠。榮公是個顯宦的人。見了鍾生有這樣好處。也着實稱贊。且又是稱愛新夫人的恩人。推屋及烏。也要酬他的情。好圖夫人歡喜。後來報陞了侍郞。路過南京。合城的官員拜望請酒。鬧鬧吵吵。榮侍郞一時那裡還想得到這上頭。郗氏夫人雖然刻刻在心。但不知他那〔時〕在那裡。名字叫甚麼。一個大京城。姓鍾的有無千帶萬哩。那裡去尋找。也只得罷了。心頭却撂不下。這日灣了船。正坐在艙中。隔着紗窗。見岸上一個人是個官兒氣象。站在那裡閒望。却與鍾生一模一樣。他是日夜感念。況向日心中又着實愛他。那相貌是時刻不忘的。隔了這七八年。只略有了些微鬚。看得十分眞切。對榮侍郞說了。差人上去一問。果然是他。纔知道做了官。故請上船來拜謝。郗夫人道。就是恩人送我衣服盤纏的那一日。我就嫁到榮府。恩人所賜的那三兩銀子。我至今留着帶在身邊。見了就感念恩私。因叫乳媼抱了他生的兩個兒子並一個女兒來與鍾生看。道。若非恩人水塘中救我一命。如何看得有此三兒\footnote{唐莊宗之劉后滅倫杖父不認者。因劉山人門戶低微。恐玷及己耳。今郗氏不惜自呈寒賤窮苦時事。感念鍾生不忘。眞是女中丈夫。較劉后之心胸。高出萬萬倍矣。}。鍾生看了。一個有五歲。一個約有三歲。那個女兒才一歲多些。相貌旣福態。都是錦裝玉裹。眞好齊整孩子。心中想着。有丈夫的人。如何嫁到這裡。此話可敢問他。但說道。些須小事。何足掛齒。怎敢當夫人這樣稱呼。郗氏又問道。恩人旣做了官。爲何又在這裡。榮侍郞便將他上本得罪。如今同着家眷回南京的話。向他說了。郗夫人道。旣然尊夫人也在這裡。定要請來會會。正說着。傳稟進來。酒席齊備了。榮公讓鍾生到客船上入席對飮閒話。問及幾時起來。鍾生說。原想雇了船。不過二三日就要行的。因把他的妾別了父母多年。今日在此無心相遇。要留着多住幾日〔的〕話說了。因此船尚未雇得。榮公道。先生不必雇船。這一隻船是巡撫衙門官座。我學生進京之後。我賞他數十兩路費。吩咐送寶眷到貴處。況他也是回去的順路。鍾生甚喜。道。這敢勞先生賞他。晚生自然酬他水脚之資。榮公笑道。這多大事。還要先生解囊。多時席罷。鍾生謝了起身。又傳進謝了夫人。然後回來。錢貴問他認得的緣故。鍾生也不好說他原有丈夫\footnote{肯(眞)盛德謹言君子。夫妻間猶不肯露。}。只說是個窮家婦人。因投水救了他。贈他衣銀之事說了。道。不想今日做了夫人。大家嘆息了一會。又道。這銀子就是你贈我那三十餘金之內的。又將送船與他回去。並明日郗夫人還要請他上船相會也說了。甚是歡喜。都說他知情報德。有這樣不忘舊的好心。宜乎有夫人之福。次日淸晨。果然差了兩個僕婦來請。因聽得榮公說他有妾。並請代目同去。都應允了。鍾生具柬潔(竭)誠去拜。並謝昨日之席。留茶回來。少刻。榮公來回拜。鍾生忙迎進來。讓了。道\endnotemark[15]褻尊勞駕。閒話了片刻。然後回船。將午。又遣僕婦來邀。錢貴同代目雇轎坐了。帶着兩個兒子。兩三個婢婦。到了船上。吃了一日酒。郗夫人相待甚是親熱。兩個兒子每人與他一個金麒麟掛在項上。是在江西屬官們送他公子的。臨回。又送了許多江西土儀。葛布夏布磁器之類。過了兩日。榮公要進京。請鍾生到船上。便說。船家學生賞過他了。先生只管坐了去。不必再又費心。鍾生忙忙道了幾個不安。謝了。隨接家人捧出十封五百兩銀子八表裡。榮公道。這是內人送先生做程儀的。鍾生還要推辭。榮公已叫人送到他寓處去了。又道。學生前日來船中所餘的酒米乾菜果品之類。今全留下。夠先生一路費用。綽綽有餘了\footnote{此書寫各人體段行事。無不誥(酷)肖。即此寫榮公夫人的事。八座行事做他人不得。故妙。}。吩咐家人查交與鍾老爺管家。鍾生謝之再三。叫鍾用去查點了。鍾生又叫稟謝夫人。郗夫人又請了去會。囑了些保重的話。鍾生又謝了回來。錢貴代目又到船上來送郗氏。郗夫人又送了他二人些東西做別敬。次早。榮公起身。鍾生送了數里。榮公苦辭。鍾生只得遵命。又到郗夫人轎前作揖。郗夫人在轎中墮淚\footnote{誠所謂感激泣下也。}。又囑幾句。然後回來船頭來叩首。請問起行日期。過了兩日。也就搬了上船。戴家苦留不住。又設席送行。送了許多吃食。又送百金途費。鍾生決不肯收。戴遷就付與女兒。算送兩個外甥的。鍾生只得領情謝了。擇日長行。代目的祖母叔祖父母叔嬸並兩個兄弟都上船送別。大哭了一場方回。鳴鑼點鼓。開船回故鄕來。不日到了東昌。同年干壹現任東昌府推官。又來拜接。送了一分厚下程。辭謝不依。也拜領了。次日。請他夫婦同代目。鍾生見他情意殷殷。都去赴席。內中眞氏相陪。外邊干生同一個幕賓陪待。還有一個抽豐客。是山西人。鍾生都問了姓氏。上席共飮。換席之後。干生指着那山西客滑稽。將當日在李家坐館的話。細細相吿。無不大笑。你道滑稽因何在此。山西大同府被闖賊殘破。李之富已老故。李太的那些桂子蘭孫皆不知去向。滑稽剛剛逃出一條命來。四處飄流。到了東昌。一日。干生出門。他在路傍看見。認得是當日的先生。問人。名字又同。他方去稟見。訴說家園殘破。無地可歸。特來相投。干生念他向年相待頗好。故留他住下。鍾生夫婦抵暮回船。次日起行。看官聽說。如今的人在骨肉親友之間。見那富厚有勢要的。明知我雖奴顏婢膝去奉承他。他猶未必慊意。這是何故。因那奉承的人多了。他覺得總不過是如此而已。這些善於呵脬的人何嘗不知。到了那個時節。竟身子不由自主。不知不覺把個忘八腦袋鎖到人褲襠裡去。俸(捧)着屁股混舔。還有一種背地說那體面話。眞是天下無兩的豪傑。從來不會奉承人。及至見了有錢的富翁。有勢的大官。他就把脖子縮得如出了〖月戎〗的㞠子一樣。那舌頭分外比別人伸的長些。去舔那把溝子\footnote{此類多極。}。到了貧窮的人。不要說陌路。就是至親骨肉。要想他說句親熱話也不能夠。或是他家有點甚麼事情。不但掉臂不顧。且躱在忘八洞裡。連鈎都鈎不出來\footnote{更多更多。}。鍾生與那郗氏毫無關切。不過是道傍的冷眼熱心。不但救了他的命。送銀送衣送錢。且存心不苟。何嘗想他有今日這一日來報他。今得此厚報也不爲過。但是一件。當日古人說。我看天下無一個不好的人。難道我要反過來說。天下無一個好人不成。四海之大。何嘗無好人。施恩於人反以仇報如中山狼者。十有五六。所以人皆心灰意懶。不肯去做好人了。如郗夫人受鍾生之德。念念不忘。此等人在鬚眉中亦鮮。總而言之。堂堂男人不若這一個閨閣婦人者甚多\footnote{此書大主意。不過說世上無情男子不若有義婦人。蓋有激之言也。}。不必多敍。再說宦實自到家之後。每每提及鍾生。不勝感念。但是夫婦父子祖孫在一處歡樂。便長嘆道。使我一家骨肉得保全者。鍾員外之恩德也。每要想報答他的深恩。又無因而前。今忽聽得他上了監軍這本。休致歸來。又敬他的人品。又感他的恩私。因聽梅生說。他向年原住的是他叔叔的房子。他叔叔也死了。房子被他兩個兒子傾掉了。知鍾生將他(歸)。替他買了一處大住宅。置了些田地佃房。及家中動用器皿什物。無一不備。約値萬金。正是。

\begin{quotation}

世間唯有恩和怨。沒齒難忘刻骨深。

\end{quotation}

宦實着人打聽他的船隻何日可到。此話權且按下。且說那鍾趨掙了一分好家私。如何就被兒子一敗至此。原來鍾趨自逼干生退婚之後。不但爲親友所不齒。不想干生又連捷中了。心中懊悔無及。已暗氣在心。他女兒嫁與勞正。得了個御史親家。心內十分中還有三五分可釋。不意魏璫事敗。坐連逆黨。親家伏法。佳婿愛女又充發陝西去了。親友無不笑罵。遂氣成蠱脹。自鍾生進京會試之後。不半年而亡。他兩個兒子。長名鍾吾仁。娶妻計氏。就是計德淸之妹。這計德淸雖是個生員。乃卜通游混公同類。專一把持衙門。調唆爭訟。無風生浪。以便於中取利的都頭。次名鍾吾義。娶妻都氏。他乃兄是個武生。南京呼爲蹺脚鬼\footnote{江南舊有一笑談。一文一武兩秀才同行。値一鄕下人挑一擔子。誤將二人一撞。一個怒道。你這狗骨頭。如何撞我這一下。那一個罵道。你這忘八肏的。鄕下人忙歇下擔子。賠罪道。小人不知是文武二位相公。失錯該死。二人喜道。你怎麼知道我們是文武相公。鄕人道。這位狗骨頭是文相公。那位忘八肏的是武相公。}。二人皆係鍾生之兄。自鍾趨死後。他二人就分了家。每人連房產雜項也將五千金。鍾趨的住宅鍾吾仁住了。將〔鍾〕生所住的那一半分與鍾吾義。他兄弟各立門戶。你我奪勝爭強。這個穿好的。那個便吃好的。〈那個便吃好的〉這個請親。那個便宴友。這個朝朝除夕。那個便夜夜元宵。兩個也不像過日子的人家。竟如石崇王愷鬥富一般。久之。二人都生起疑忌來。鍾吾仁暗想道。兄弟是父母的小兒子。古語說。天下爺娘疼小兒。再沒有做父母的人不偏愛幼子的。在生時必定多與了他些私囊。不然爲何如此奢費。鍾吾義又疑道。哥哥係長子。我幼時他必定偏得父母的多。不然何得這樣花用。世人只自(知)看別人的非。再不知見自己之短。他兩人行事舉動原是一般無二。因疑心一起。彼此窺潛。無一事不戳眼。又經不得內中兩個婦人。這一個在丈夫跟前。那一個在男人面前。都一陣計都。遂將丈夫的心挑發(撥)。這兩個婦人之兄。又係寡廉喪恥的人。調唆妹夫兄〔弟〕興訟。貪圖口腹。或內中有美(羨)餘。更有那些不顧人生死。只知奉承的親友。扛順風旗在傍慫恿。使他弟兄就同室操戈起來。鍾吾義在縣中遞了一狀。說哥哥恃長。分家不均。多得家產。求恩公斷。干證就是慫恿的那幾個親友。又恐縣中不准。買了一尾大鯉魚。肚中裝了二百四十金。煩人送進。那知縣姓臧名繼仲\footnote{世間能有幾個知縣而贓不及重者。諺云。家家賣酸酒。而犯(我)是高手耳。}。是山東人。他說是臧文仲武仲的子孫。故起此名。他見這是有錢的百姓吿家產。眞是點燈也尋〈出〉不出的美事。何況又受了重賄。即刻發籤拿鍾吾仁。鍾吾仁聽見。慌了。忙買了一個大冬瓜。裝了四百金在內。厚賂原差。就煩他暗暗送入。仍補一狀。說兄弟係父母所愛幼子。偏得甚多。求恩追出斷給。就煩舅子約了十來個素常走衙門的秀才做干證。知縣也准了。次日早堂。帶來審問。先把兩家的干證略問了一問。少不得是各位袒其人。然後叫他親戚上去問。衆人道。分家之時。雖有小人們在跟前。房產地土皆係均分。當日係他兄弟二人情願。至於內中私弊。只他們各人自己。我們外人如何曉得。知縣點了點頭。先叫鍾吾義上去。問他口供。大略與狀上相同。又叫鍾吾仁去問。鍾吾仁也照狀上細訴了。那知縣勃然變色。把驚堂拍了兩下。指鍾吾義怒罵道。你這奴才就是個刁頑百姓。自古道。長兄爲父。就有不公。只該央族中親友去講論。你也不該輕易就興詞動訟的吿他。你就不是(曾)聽見古人推梨讓棗麼。況你衆親友都說眼見均分。可見無私弊的了。你何得誣吿胞兄。罪應批誣吿。平人加一等。且打你幾下。警戒你個不弟。然後再定你誣吿的罪。抽了四根籤撂下來。道。\endnotemark[16]本當重責你這奴才。本縣姑念薄責。那鍾吾義先以爲他送過魚的。定上上風。好不放心大膽。見他說話時。全是爲着哥哥。心中疑道。難道忘記我魚腹中之物了。聽他罵了一陣。忽然撂下籤來要打。衆衙役上前拖翻。他急了。高叫道。老爺天恩。念小人是個大愚民哪。那知縣聽他說了這個愚字。吩咐住了。衆役放他起來。知縣呵呵笑道。你說就是愚民。因指着鍾吾仁向他道。他還是個大呆瓜呢。因道。看你的愚。權記打。且送你去稽候所住幾日。耐耐你的刁性。喝一聲。帶了去。將鍾吾仁等逐出免究。鍾吾義到了所中。禁子衆人知他有鈔。一個作惡。一個作好的。狐假虎威。一陣嚇詐。鍾吾義從不曾見過這樣好去處。心驚膽裂。又費了許多使用。他托起先送魚的那人探聽縣官緣故。方知哥哥送了他四百金一個大瓜。始悟臧知縣前說呆瓜的話有因。又叫家中取出二百六十兩湊前足五百之數。拿了去送進知縣。隨帶人去拿鍾吾仁。這鍾吾仁見兄弟下了所。以爲錢神有靈。正欣欣得意。在家中宴那些干證痛飮。不意又被拿來。私問原差。也不知其故。到了堂中。丹墀中跪下。知縣〔道。〕你兄弟屢屢哭訴。說你欺心。你若果然公平友于之愛。你又何若如此。定是你這奴才倚大壓小。待弟刻薄。你可曾聽見鄧伯道棄子存姪。也不過是爲兄弟。許武不惜自汚。以成弟名。也不過是爲兄弟。你待手足無情。也就是個畜類了。今單把他收禁。他心中自然不忿。你也同他坐坐。洗一洗你的獸心。不由分說。帶了去了。鍾吾仁托人打聽。知兄弟送了五百。他添了三百。鍾吾義知道。也添。每人送夠千金。知縣心滿意足了\footnote{山海衛有一知府。在位時混名劉估家。有在衙門中打官司者。家產罄而後已。這知縣只二千金便心滿意足。較之劉太守。可謂淸廉極矣。如何算得贓極重之致。}。吩咐將前\endnotemark[17]狀上有名的親友並干證都傳了來。次日上堂。帶他兄弟二〔人〕到公堂前。和顏悅色勸道。人生在世。除父母之外。再莫過於兄弟了。手足自相殘害。還好得麼。古人說。難得者兄弟。易得者田地。又道。錢財如糞土。仁義値千金。本縣還記得一首詩道得好。念與你二人聽。

\begin{quotation}

同氣連枝各自榮。些須小事莫傷情。

一回相見一回老。能得幾時爲兄弟。

\end{quotation}

還有幾句更說得好。

\begin{quotation}

兄弟同居忍便安。莫因毫末起爭端。

眼前生子又兄弟。留餘子孫作樣看。

\end{quotation}

你們記着。前日本縣禁你們幾日。不過要你們反悔的意思\footnote{恐不止此。或者還是爲家兄。}。本縣是你們的父母官。可有不疼愛你們的麼。我勸你兄弟美的好。因罵兩家干證道。他親兄弟豈肯如此。都是你們這姓(些)無恥的奴才。見利忘義。挑唆人家兄弟鬩牆。本該重處。姑念無知。寬恕。內中有幾個干證的秀才。臧知縣道。諸生旣在黌門。也該惜些廉恥。怎跟着這些下流奴才胡行。後再如此。定然申詳學憲。你們都是讀書人。可將書上孝弟道義的話勸他弟兄。又向他衆親戚道。你們旣係至親。帶他兄弟去替他們和好罷\footnote{眞好父母官。若無那二千金贓。定當考上上。然而這一篇說話。也値得二千金之數。}。吩咐出去。他二人見官府如此說了。還敢說甚麼。忍氣吞聲回來。他兩人不自己責悔不該吿狀。反彼此深恨爲何用銀子陷害。此後更如寇仇。各又想道。原圖費用幾個斷過家私來過。棄少而取多。不竟(意)一文不得。反費去千餘金。此忿如何消得。一日。鍾吾仁帶了兩個家人。要到他一個朋友家去同謀設法到別衙門吿理。不但要翻透千金的本。還要出這一腔子氣。走到文廟泮宮前。一眼望見兄弟帶着個小子。背立在水邊。原來鍾吾義也是到一個親戚家商議要吿哥哥。留着吃了半日酒。有幾分醉了。辭了回家。走到此處。在(正)站着看水。心有所思。忽看見哥哥遠來。只得倒背了臉。此時已暮。鍾吾仁四顧無人。凶心陡起。輕輕疾走到兄弟背後。用力一推\footnote{可謂我已無人。吾何法乎哉。}。那鍾吾義一則不防哥哥害他。二則有酒的人頭重脚輕。便一個筋斗翻入水中。那小子纔要跑。鍾吾仁叫家人陶沃上〔前〕拿住。小子要叫喊。被陶沃將喉管捏住。已將半死。也抛入水內。那鍾吾義在水裡已淹得昏頭昏腦。忽然冒將出來。鍾吾仁忙拾起一塊半截磚。對準腦門。儘力一下。得復沈下去。看了一會。不見動靜。他也不去尋朋友了。歡喜回家。兩個家人每人賞了十兩銀子。叫他隱密。然後吿訴計氏。夫妻無限快樂。痛飮慶賀\footnote{勿謂世間無此等人。北齊高演之殺弟。有甚於此。}。以爲出了惡氣。那都氏晚間不見丈夫回家。叫人拿燈籠往親戚家去接。說已回去久了。着人四處尋覓不見。着實心疑。天又夜了。只得歇息。次早又叫人去尋。聽得人紛紛傳說泮池內有兩個屍首浮出。那家人忙去一看。一個正是主人。一個正是小子。將屍首拖到岸上。只見主人頭顱粉碎。那小子喉籠(嚨)靑紫。忙去報與都氏。都氏坐轎來看了。痛哭一場。叫家人去報縣。知縣差四衙帶忤作相驗了。塡寫屍格回稟。知縣明知是人謀殺。但不知凶手是誰。只存了案。屍首着屍親掩埋。俟拿獲凶身再行定奪。都氏只得將丈夫周(用)棺材裝殮了擡回。家人小子也用棺材盛了埋於城外。都氏也疑是大伯謀害了丈夫。但未得指實。不敢妄吿。只得廣延僧道念經設醮。超度亡魂。看墳塋埋葬而已。看官聽說。天地間有胞兄殺了親弟。竟躱得過去。那就眞沒天理了。鬼神尚何足畏。他慢慢自然有個報應。那日鍾吾仁在伴(泮)池害鍾吾義之時。跟着的兩個家人。一個名鞏濟。自來是鍾吾仁的心腹。一個名陶沃。那陷(掐)死小子的就是他。他素常性極凶惡。因見家主害了兄弟。雖然得了十兩銀子。焉能滿意。因主人有此把柄在他手中。未免就漸漸放肆。鍾吾仁也忍過了半年。事已冷了。一日。計氏生日。鍾吾仁叫陶沃去買辦菜疏。款待舅子。衆親到抵。他去(至)暮方醉醺醺的回來。此時都散了。鍾吾仁罵道。你這大膽的奴才。等着買東西替你奶奶做生日。怎去到此時纔回來。他瞪目斜視。道。我大膽。殺人的纔大膽呢。鍾吾仁見他道着心病。倒不做聲。他轉身反嘓噥道。一個老婆的生日這樣要緊。害兄弟像殺小雞的一般。不要討我說出來罷\footnote{却是天理話。但不該出於惡奴之口。}。鍾吾仁聽了這話。忍耐不住。趕上去打了他一個嘴巴。他大喊大叫道。我犯了甚麼事。你打我。我料道沒有殺了人。我不怕你。你有本事送我往衙門裡去麼。支手舞脚的挺撞。鍾吾仁忍不住。叫衆家人拿住。結結實實打了他一頓。他懷恨在心。走到隔壁。一五一十將前事細說。都氏留住了他。叫人請了他哥哥來商議。因恐這臧知縣是個贓坯。不敢在他手中去吿。要到別衙門吿理。怕也周(同)縣官一類。況同在一城。恐大伯先弄了手脚。遂議定往巡道處吿。京府巡道即是外省的按察司。此時巡道衙門設在鎭江府。都氏帶着陶沃同哥哥往鎭江府去了。鍾吾仁先見陶沃走了。還以爲他逃去。後來方知他同弟婦去吿狀。纔着了慌。叫鞏濟連夜隨去打聽。次日回來。說道。巡道已經批准。發刑廳荀老爺審理。這鎭江府刑廳。他世代科甲進士出身。眞算得一個簪纓世胄\footnote{好體面。}。姓荀名思。是阮大鋮的門生\footnote{跌到此一句。甚覺不堪。}。鍾吾仁急尋門路去求阮大鋮。定要五千金。講之再三。連房產幷現物共湊三千兩奉上。阮大鋮打聽他家已將罄了。纔肯依。寫了一封懇切的書。差的當心腹家人龐周理。星夜過江去役(投)。設(說)鍾吾仁是他至戚。萬望開設(脫)。荀刑廳接了書。心中暗急。道。這張狀子我原想自己吃此美嘴。不想被老師高才捷足者先得去了。沒奈何。只得欽遵來命。因籌畫再四。大悟。喜道。這邊不着那邊着。都氏豈非一塊肥肉麼。遂算計到\endnotemark[18]他身上。過了一日。差役已將鍾吾仁同鞏濟家人提來。鍾吾仁也補了一張辯寃的訴呈。到審的時候。先叫都氏上去問了問。然後叫這出首的家人去審問。這陶沃遂將如何推落水中。如何用磚打破了頭。如何叫他拿住小子。掐得將死。也撂下水去。那刑廳微微的笑了笑。叫上鍾吾仁去問。鍾吾仁道。老爺天恩。當日小的雖同兄弟吿過家產。那時兄弟先吿小的。小的氣不過纔補吿的。蒙本縣老爺勸諭。吩咐衆親友已和過。現有江寧縣案件可查。小的與他兄弟。何仇就到殺害的地位。這惡奴酗酒肆惡。無所不至。小的責處他是有的。人所共知。他就去挑唆弟婦。弟婦一個女流無知。遂聽讒言。以致動訟。小的若果有虧心的事被他拿着。哄還怕哄他不過來。焉敢責他。求天恩詳察。刑廳連連點頭道。理直言壯。說得是得很。又叫那鞏濟去問。他極力質辯並無此事。刑廳又叫陶沃上去詰問。他報(抱)定前辭。謀害是實。〔刑廳〕拍案大怒道。你家主旣謀害兄弟是眞。你次日如何不出首。直捱至半年之後。因受責罰。方纔說出。你主人說得是。他果然實有此事。他有心病。決不敢打你了。你這奴才。因主人一時之小失。就欲陷他於大辟。你心地也太惡了。就據你說是眞。你主人謀害兄弟時。你係同謀殺害幼主。分首從。你該斬。你掐死那小子。投下水。故殺。律又該斬。今日挾仇誣吿主人死罪。反坐。又該斬\footnote{首(看)刑廳律條甚熟。但不知可記得枉法〔貪〕贓是何罪。}。以你一人。得了三個斬。死有餘辜了。吩咐夾起來。打了二十槓子。又問他。還是前辭。刑廳大怒。又加了三十板。發去收監。又叫都氏上去。罵道。俗語道。家有賢妻。男兒不遭橫禍。當日你丈夫在日吿哥哥。這定是你這不賢之婦在內中挑唆起釁。今日又聽惡奴一面之辭。誤吿大伯。本該重處。且發媒婆家看守。俟本廳察出內中情弊。再行發落。本廳看你在我公堂上還這樣妖妖嬈嬈的。焉知不是你有奸夫。通謀害殺了丈夫\footnote{輕輕入一剔(剮)罪。}。因與大伯有宿恨。故買出惡奴來。嫁禍於他。希圖脫罪。等本廳訪明了。你身上的罪也不輕。傳了媒婆來。吩咐帶去看守。又吩咐鍾吾仁討保在外。聽候發落。鍾吾仁出來。想陶沃執定扳他。恐過後都氏再往別衙門去吿。如何了得。將家中剩得餘物。拼拼湊湊。弄了百餘金。罵(買)囑了司獄禁子。將陶沃掇弄死了。報稱受刑後得病。醫治不痊。自斃於司獄司。出結報廳。刑廳心照。也知有弊。他一〈想〉心中想吃都氏。正礙這家人口硬。恐將來有事。也巴不得他死了。沒有對證。見了報單。命將屍拖出存案。都氏在媒婆家看守。聽官府的話不好。正在憂疑。次日。又聽得陶沃死了。越沒對證。心下十分驚怕。請了哥哥來商議。不求柴開。只求斧脫。如今也不想官事贏。自己免禍顧命要緊。將家資湊了二千金。送入私衙。次日。即提出來。說道。你誤吿大伯死罪。本當反坐。念你女流無知。又係聽惡奴挑唆。惡奴又死了。姑念〔免〕究\footnote{都氏當云。多承盛情。}。等本廳申過上臺。再行釋放。也叫討保聽候。也朦朧一角文書申了上去。云。審皆是虛。都氏誤信奴言。念係女流。免坐罰贖。罪當應坐家奴。因斃病故於獄。已膺天誅。餘人應行釋放。做官的人能有幾個肯細細訪察民情。那巡道見了刑廳申文。批准下來。刑廳傳齊衆人。當堂釋放。衆人出來。各自雇船歸家。鍾吾仁記罣家中。阮家來催出房子。急於要回。獨雇了一隻小滿江紅取快。是日風恬浪靜。江中無限行舟。他這船到了江心。忽然一個大旋風。船底朝天。凶人落水。傍船急來救時。只救起兩個船家。鍾吾仁同鞏濟大約到大海中去了。他謀死了兄弟。那鍾吾義還得屍骸入土。就是那小子也還得個棺材埋葬。他主僕二人。竟葬於魚鼈之腹\footnote{他是水葬。}。害人自害。豈不信然。因鍾吾仁弟兄相害。有一調駐云飛感嘆世人手足。道。

\begin{quotation}

手足天倫。同氣連枝骨肉親。貴賤皆天定。貧富何須論。噫勢理起家庭。較人猶甚。同室操戈。血淚如珠迸。嘆世上兄弟相和有幾人。

\end{quotation}

都氏回家。家中還有千金之產。他少年無出。嫁人去了。這計氏家業罄盡。一絲在(也)無。在哥哥家寄住了幾日。也只得抱瑟琶過別船而去。可笑鍾趨苦積萬金之產。被兩個賢郞這樣輕輕花去。不但性命不保。而且覆宗絕嗣。古人說。錢財上寬一分。與兒孫積一分之福。豈欺我哉\footnote{鄙吝諸公。此眞不入目之言。可壓(厭)至極。}。此雖係鍾氏弟兄分爭之罪。實由鍾趨愛富嫌貧。只知損人利己之報也。古云。遠報兒孫近報身。毫厘不謬。不信。但看此一段事。豈不使人不寒而慄。因他兄弟二人互相謀害的這一件事。有幾句打油感嘆世情。又可以警戒此輩。不可說是熟話不看。

\begin{quotation}

世人何故喪良心。但見黃金不見人。

毒計每緣爭阿堵。奸謀乘〖阝少日小〗亂家庭。

僉壬莫怪胸如蜮。天性還因腹有荆。

休道冥中無報應。驅除險惡化和平。

\end{quotation}

不必煩言。且說宦實家人打聽鍾員外的船到了趕(旱)西門外石城橋下。他父子同接了出來。鍾出(生)忙迎進艙中。相揖坐下。道。老先生尊年先輩。何敢當此厚愛。遠勞尊駕。使晚生何以自安。宦實將父子朝夕感念。並將替他置了房產地土。候他歸來的話〈話〉說了。又道。愚父子特來奉迎到新府耳。鍾生雖感之不已。還要推辭。先是梅生同鄔合接到下關。此時在船上同來。梅生見他推辭再三。勸道。宦老先生這一番殷殷厚意。吾兄再却。未免就覺十分固執了。鍾生此時也無可歸家。又見他這般實愛。也就深謝領了。鍾生賞了船頭十兩銀子就發行李。同着家眷上轎。來到新店(居)。甚是寬敞富麗。家中動用之物。無一不備。宦實又備了戲酒來。一來替他接風。二者溫居。鍾生感之不盡。後來竟成了通家莫逆。鍾生一到家。賈文物童自大都來拜望。賀房接風。大家熱鬧了許多日子。錢貴之母郝氏。宦蕚之妻侯氏。梅生之妻李氏。鄔合之妻嬴氏。都來看錢貴。送席。內邊堂客也吃了數日酒筵。過了些〔時。〕鍾生事體稍暇。差人往和州打聽。關爵已回到家園地。二人鄕會同年。做庶吉士時。志同道合。臭味相投。十分契厚。後來雖分了衙門。常常相晤。今〈相〉見他革職是因救己波累。又素知他貧寒。將榮公夫婦所贈之物取出百金\footnote{提此一句者。見鍾生除此以外。別無他蓄耳。}。雇了一隻小舟。親到和州孝義鄕去相探。關爵見他遠來。不忘友誼。心中甚喜。寒素家風。唯設雞黍村醪相待。鍾生將攜來之物奉承。關爵初不肯受。鍾生道。年兄之淸介。弟豈不知。此物若是弟從貪汚中得來。決不敢汚及年兄。及(旣)係他人贈我。分贈年兄。這有何傷。況古人傾蓋相逢。即有束帛之贈。未聞其辭也。何況我二人同年兄弟耶。此些須不過爲年兄薪水資耳。年兄豈疑弟爲世俗之夫。做報德之敬耶。關爵見他情意殷殷。只得道謝收了。相留盤桓了數日。鍾生因到家未久。辭別了回來。却說童自大自己尋思道。我自從與宦實(蕚)賈二哥結拜之後。這幾年了。擾過他兩家大酒大席不計其數。我雖請過他們幾次\footnote{也就算費事了。幾年請過幾次。也便一年請一回。較之生平從不請客者高出多矣。}。都不過家常茶飯而已。連酒也不曾醉過他們一次。從來沒有設席叫戲熱熱鬧鬧這樣一回。我雖改過了。這幾年但只不在銀錢上刻薄。並不曾大施爲施爲。這個臭名終在。我看鍾員外人都這樣敬他。宦哥白白的送他萬金之產。我就破二三十兩頭請請他做個相與也何妨。況且我同宦哥結拜了。他父親就是老伯。他來家這幾年。我還沒有與他接風\footnote{到家數年。方纔接風。也算新聞。}。何不一舉兩得\footnote{還是一事兩勾當。到底臭味難脫。}。又想道。我的主意雖如此。不知奶奶捨得捨不得。須同他商量了。纔好行事。遂走到鐵氏跟前。把這個意思達上。鐵氏也不像奉承他嘴巴的惡態。他三十多歲了。終日飮酒食肉。一無所事。閒了就拿角先生解悶。眞是心廣體胖。他胖得沒樣。到如今越發胖得動都動不得。兩腮的肉墜了下來。脖子與下頦一般粗。要回頭。連身子俱轉。胸前大乳凸得充高。屁後尊臀宛如巨鼓。雖無那凶暴之氣。只是生性吝嗇。却不能改。他因胖得很。總不能生育。即如母雞太肥了。油蒙了心。不能下蛋的一個理。數年來。不想倒是葵花心中竟結了一個子。蓮花瓣內也產了一個女。他娘母雖醜。倒生了兩個好白胖孩子。鐵氏拿來自己養着。都有五六歲了。這日。他〖扌歪〗在一張大涼床上。正鬥着兩個孩子頑耍。聽見這話。但道。你通共百十萬家私。就想這樣大行爲。你度量你的力量去行。我不管你的閒事。只要每日不少我的酒肉就罷了。只不要說你因請人花費了銀子。在我身上扣除。缺少了我\endnotemark[19]的食用。那就行不得了。童自大道。你但請放心。我的家私還夠你受享幾輩子\footnote{此話也難說。百萬財主便能保終始乎。百(昔)江南一趙百萬。家私百萬猶有餘。後年將七十。漸漸虧折。僅存十餘萬。逢人即哭道。我要餓死了。只得十來萬銀子。這日子怎麼過。彼時余尚年幼。常笑之。後來方悟百十萬家私過慣了。到了只得十數萬自然難過。或者連酒肉都捨不得吃。亦不可知。}。遂歡喜喜的出來。到了宦蕚家中。宦蕚正同鄔合在那裡閒話。讓他坐下。他把要請客的話說了。定要請宦實到家坐坐。還要借他的家人器皿雜項。宦蕚都允了。就走到上房。向父親去說。宦實道。你們一起少年去走走。我老了。辭了他罷。宦蕚笑着道。兒子同他相與了這些年。他從不曾請過一次。他一輩子捨不得費錢。家中也沒設過大席面請人。況他纔說這是特爲老父並鍾兄而設。不如去擾他。鼓舞鼓舞他的興頭。宦實聽了這話。也笑笑依了。宦蕚出來與他說知。他見宦實肯去。滿心歡喜。就托鄔合去請鍾生同賈文物。鄔合道。老爺費這樣大事。還該用個請帖。纔成體統。宦太老爺同大老爺賈老爺諸位算是通家罷了。鍾老爺是新客。怎麼好口請的。童自大道。你當我捨不得幾個帖子麼。實不瞞你。我從〈不〉沒擺過大酒席。不知道這些規矩。二來也沒人會寫。就煩你替我買幾個帖子。央人寫寫。我改日酬你的情\footnote{何不像當日初拜宦蕚時用沒字帖。豈不省事。}。宦蕚道。你不必。叫了個家人來。吩咐道。你去叫了書辦來。叫他拿幾個全帖同筆硯來。童自大喜道。這個省事。更妙。只是又煩費哥。不一時。叫了他家中的一個裴書辦來\footnote{裴賠音相似。不但賠了書辦替他寫。還賠了許多帖子。}。宦蕚向童自大道。你要請誰。寫幾個帖。你對他說。童自大道。並沒別人。就是老伯同二位哥。鍾員外。鄔哥。五個帖就夠了。宦蕚道。我老父同我說過了。不必用。你只寫別的罷。鄔合也道。晚生理當來效勞。怎敢當老爺賜帖。童自大不肯。道。我先不知道這個禮性(數)就罷了。旣然該這麼行。如何不用。定要寫\footnote{這叫做不惠之費。不用錢買的帖子。諺云。火燒紙馬鋪。落得人情。}。宦蕚只得依他。他對裴書辦通(道)。該怎麼樣寫。我不知道。你是寫慣的。煩他(你)寫寫罷。裴書辦道。幾時的日子。他道。明日來不及。後日罷。裴書辦替他寫着。宦蕚道。旣然費了這些事。何不添一席。連梅兄也請請。他旣是鍾兄的好朋友。我們都相熟。可使得。他笑道。有理有理。還是哥想得到。帖子寫完。書辦將小姪愚弟兩個帖遞了與宦蕚。說。這是請我家太老爺大老爺的。別的都遞與鄔合。〔童自大〕道。鄔哥。你的帖子你就自己收了去罷\footnote{妙極。請客自己下請帖。也是從來未聞。}。別的就煩你去請請。務必要來纔好。你知道我家沒多人手。改日謝你罷。鄔合應允。接了過來。他約定了。然後歸家。到了那日。叫了一班好戲。一班吹手。廚役茶房酒按摩。一一齊備。宦蕚又打發了十數個家人來相幫。一應杯筯氈毯之類。皆係宦家送來與他用。他又請了舅子鐵化\endnotemark[20]來做陪客。另在回回館中備了一席\footnote{細。}。午間。衆人陸續來到。鼓樂喧天。簫韶震耳。廳上懸燈掛綵。氍毹匝地。十分齊整。讓坐上席。正中一席宦實。東邊首席。鍾生遜讓。梅生決不肯僭。只得坐了。西邊二席就是梅生。三席宦蕚。四席賈文物。鄔合一席略退後些。捱次坐下。他與鐵化在下面相陪。酒筵果然豐盛精美。唱戲吹打又十分熱鬧。屛門後掛了簾子。獨設一席與鐵氏看戲\footnote{外邊賓主八人。內中鐵氏。可謂連婦人焉九人而已。}。葵心蓮瓣也打扮着。扭扭捏捏跟了來看。那鐵氏雖是回子家女兒。嫁來久了。也就無所不吃。早忘了他的敎門了。那日衆人都體貼他這場盛心。直到天明方散。鐵氏嫁到童家十多年了。不但不曾見過這樣熱鬧。也並不曾吃過這些美品。也動起高興來。童自大回到內室。鐵氏道。大家私。你的爲得人。我也要請客。童自大巴不得要他歡喜。便道。奶奶。你憑着要請誰。我可有不依的麼。同他商議了一番。算計無人可請。只請宦夫人艾氏。宦奶奶侯氏。妾嬌花。鍾奶奶錢氏。妾戴氏。賈奶奶富氏。梅奶奶李氏。鄔娘子嬴氏。並他嫂子火氏。當日請不及。他出來把戲子鼓手廚子各項人都定了。明日還要請堂客。又對宦家人說了。留下他們相幫。叫打發衆人酒飯。他去睡了一會。已飯時起來。叫童祿去請了鄔合來。煩他買〈他〉幾個全帖寫了請啓。煩宦家認得的人分頭去請。明日赴席。次日淸晨。火氏便到。飯後。先是嬴氏到\footnote{連此沒要緊去處亦〔無〕不寫得有理路。火氏至親算主。自應早到。嬴氏乃篾片之妻。大老夫人相召。又當先來。妙甚。}。見了禮坐下。不多一會。富氏也到。接了進來。原來富氏數年來因寡慾多男。他也生了一男一女\footnote{他當日曾小產過數次。謂係怒氣所傷。此頭謂寡慾。到底虧息了悍妒之氣之故。}。都帶了來頑耍。奶娘抱着纔坐下。外面又吹打\footnote{先火氏嬴氏富氏來。不曾說吹打。此處云又吹打。則先亦曾吹打過。也是省筆之法。}。說是鍾奶奶梅奶奶戴姨娘到了。代目他姓戴。人見他生了子。都稱他戴姨。代目見了鐵氏。要行大禮。鐵氏連忙拉住。將他細看。認得就是仙桃。好歡喜\footnote{可見當日鐵氏賣他時。雖是妒。却是愛。不然今日見面豈不忸怩。而反歡喜也。}。分外親熱。讓他坐下了。葵心蓮瓣見了他。也着實親香。少頃。艾夫人領了侯氏嬌花下轎進來。衆婦人都迎接到內。彼此各見了禮。錢貴又謝了艾夫人厚情。並謝侯氏前次賀房的酒席\footnote{細。}。坐着。也聊\endnotemark[21]些閒話。外面吹打着催席。鐵氏同火氏讓着衆位到前廳上席。只見芙蓉帳隱。玳瑁筵開。堂掛珠簾。席排金盞。坐位還點(照)前官客座的坐次。傍邊安了二桌。代目同葵心一張。嬌花同着蓮瓣一張。兩個鳩盤荼陪着一對生菩薩。不一時。點了戲。送上酒來。肴饌湯點。一道道送上。熱鬧到將晚撤席。又都到上房來。衆堂客有更衣者。洗手者。勻臉者。點唇者。這都是奶奶的正務。眞是那。

\begin{quotation}

鏡子照得發昏。馬桶響得不絕。

\end{quotation}

鐵氏拉着代目的手。悄悄問他如何到了鍾家。代目將童佐弼同媒婆將他賣與鐵(錢)家的事相吿。鐵氏恨恨不絕。那時大家坐了說話。好不親熱。宦夫人看見鍾生的兩個兒子。賈文物一男一女。童自大一男一女。梅生一女。他自己媳婦生的一女。嬌花生的一男一女。大小十個孩子在面前。恰好是五男五女。好生歡喜。笑着對衆婦人道。你們尊夫都是好朋友。你們何不結了親。大家更覺親熱。衆婦人道。老太太尊意甚好。聽憑主張。艾夫人笑着道。我就做個主媒。分派定了。你們回去商議。看可行得。因對錢氏李氏道。我聽得說。你二位的尊夫自幼相與又着實親熱。梅奶奶。把你的令愛配與鍾奶奶的大令郞。可好麼。李氏感激鍾生當年替年(他)做媒。得嫁與梅生。巴不得把女兒與他做媳婦。以報前情。假做謙辭。笑吟吟的道。老太太主見甚好。只是家寒扳不起。錢氏道。我家拙夫與尊夫〈人〉莫逆之交。怎麼還說外話。我家去說了。再無不成的。艾夫人又道。我家承鍾老爺的情。再感激不盡。把我媳婦生的這個女兒配了鍾奶奶的小令郞罷。錢氏忙謙道。這可實實的仰扳不起了。艾夫人道。你若嫌棄我家就罷。若不然。這門親我是定要做的。錢氏指着代目道。這做(個)小兒是他生的。所以更不敢仰扳。艾夫人道。妻有大小。子無貴賤。我只算報鍾老爺的情。別的我不計較。錢貴見他這番美意。忙拜謝了。又謝了侯氏。叫代目也都拜謝。代目同嬌花彼此也相拜。艾夫人又道。賈奶奶。你的令愛與我孫兒罷。童奶奶的令愛與你的令郞。我的小孫女與童奶奶的令郞。做了五對小夫妻。豈不妙。我也不強你們。回去商量明白。再拜門請酒。衆人都笑嘻嘻的道。老太太吩咐。再無個不依的。等說明白了。再來叩謝〔老〕太太。艾夫人笑着道。若都是成了。我這個老媒婆是要吃喜酒的呢。衆人齊笑道。少不得請老太太叩謝。內中唯有鐵氏聽見艾夫人把小孫女與他做媳婦。把一張大嘴咧着。一臉的肥肉笑得擠成一處。眼睛只得一縫。歡喜得非常。眞是夢想不到。忙叫人對童自大說了。童自大這個喜還了得\footnote{可見富之〔求〕貴。亦猶貧之美(羨)富也。}。忙進來。就替艾夫人叩謝。又謝了侯氏。鐵氏也俱拜謝了。正在熱鬧。笑語喧闐。聽得又吹打催上席了。出來上了席。大家到三鼓方散。辭了各自歸家。次日。艾夫人把聯親的話對宦實並兒子說知。宦公道。大孫女與鍾家甚好。只是小孫女與童家不稱心。艾夫人道。我也想來。誰量得誰\footnote{達者之見。反出自婦人。}。丫頭生的孫女。配這百萬財主的兒子。也就罷了。宦公點頭無語。宦蕚也自歡喜。這幾位奶奶到家。都對各人丈夫說了。都歡喜願意。擇了一個好日子。煩鄔合做媒。都通了信。同在這一日。互相拜門謝允。過後。又彼此請酒唱戲。男客過了。就請女客。臨了這兩日。纔是童自大請。他夫妻二人心中快樂。這次比前越發熱鬧。只苦了鐵氏這個肥人。每日累得這汗淌不住。別處還可。唯有兩個奶頭底下並那胯襠中。竟像潑了兩桶水的一般。俗語說。人逢喜事精神爽。他也竟不覺得辛苦。把這個葵心笑得那嘴差不多比葵花心略小些。蓮瓣竟把嘴笑得比蓮花瓣還大了。把這一子一女竟疼愛得說不出的那個樣子。再說那童自大想道。我總是破了戒了\footnote{他當日不知幾時受得。趣語。}。我門下這些夥計。都是幾十年了。從來也沒有請過他們一次。我替宦哥賈哥結了親。昨日他們都有大分資來賀喜。何不也請請他們。也是我財東的體面。又來與鐵氏商量。鐵氏這些日子看戲吃酒。好生快活。兩個小夫人又在傍慫恿。滿口應允。便道。你旣請夥計。我也要請衆夥計娘子。童自大可敢不依他。連聲答應。果然次日請衆夥計們吃了一日戲酒。到散時候。這些多年的夥計每常一飯也不曾擾過。何況這樣盛設的酒席。兜脬大揖作上許多。再三道謝。方纔別去。次日。鐵氏請衆夥計娘子並鮑家娘子含香。又聞(熱)鬧了一日。童自大罷(道)。索性拚着破費破費罷。把他的親友。從來連水都摸不着他的。都去請了來。吃了一夜戲酒。也請了鮑姓(信)之來。你道他緣何認得他兩口子去請他。前賈文物請他夫婦時。內外席上有鮑信之含香。他看賈文物面上。故此纔請。又把左右街鄰請了一席。道是兒子定親的喜酒。衆人知道同宦府聯姻。都公分買了羊酒來補賀。鐵氏更加高興。對童自大道。我這些日子雖然吃酒看戲。把我也累夠了。你就不該獨設一席。替我酬酬勞\footnote{吃酒看戲還要酬老(勞)。也是乍見。}。童自大自然是要遵命的。留下戲子各項。到次午。擡過一張涼床。鋪了厚褥。放了幾個大枕頭與他靠背。獨排一桌與他受用。童自大側坐相陪\footnote{竟行的是公主駙馬禮。}。鬧了一夜。不但他親友夥計以爲奇事。這些街坊上的人都道。我們與百萬做了幾十年的鄰居。從沒見他家吃戲酒。竟連二連三的這些日子擺酒唱戲。眞是破天荒的事。他如今當眞竟不臭了。傳得各處都以爲奇聞。鐵氏又特設了兩席。單請錢貴代目到家一敍。同代目好生親熱。同他認了姐妹。代目不敢當。鐵氏道。你的兒子同我的兒子是嫡親挑擔。你還謙甚麼\footnote{此雖親愛之情。然係勢利起見。}。他雖一口一個妹子的叫。代目仍稱他奶奶。過後。兩家時常往來。閒話稍住。過了些時。鍾生一日夜間睡不多時。似夢非夢。獨步到街上來。忽見一個大夫第。如王者之居。心中詫異道。這是甚麼所在。看那門首〇〇許多奇形異常猙獰長大的兵。率皆執着器械。又不敢近前去問。心內驚疑。左右顧盼。忽見牆隅一(之)下。宦蕚賈文物童自大三人在那裡站着。鍾生上前舉手。驚問道。此是何處。三兄何如在此。他三人同道。適間有一位神將傳王旨。召我們到此。我們途中問他王是何人。他說是古城隍神。領我們到此。他進府啓王去了。連我們也不知召來何事。鍾生吃了一驚。端的古城隍召他三人來。如何指示分剖。但看後文。便知分曉。



\endnotetext[1]{「河」字原置「老爺」二字之下,據文義改。}

\endnotetext[2]{「住」原作「存」,據文義改;下文或同,不贅。}

\endnotetext[3]{「的人」原作「人的」,據文義改。}

\endnotetext[4]{「抵償」原作「低賞」,據文義改。}

\endnotetext[5]{「位」原作「爲」,據文義改;下文或同,不贅。}

\endnotetext[6]{「得聽」原作「聽得」,據文義改。}

\endnotetext[7]{「道」字原置「大喜」二字之上,據文義改。}

\endnotetext[8]{「臣工」原作「工臣」,據文義改。}

\endnotetext[9]{「叩首」原作「首叩」,據文義改。}

\endnotetext[10]{「關必顯」原作「關心顯」,據下文改。}

\endnotetext[11]{「他們」原作「們他」,據文義改。}

\endnotetext[12]{「江西」原作「西江」,據下文改。}

\endnotetext[13]{「媒婆」原作「婆媒」,據文義改。}

\endnotetext[14]{「道」字原置「來」字之上,據文義改。}

\endnotetext[15]{「道」字原置「了」字之上,據文義改。}

\endnotetext[16]{「道」字原置「來」字之上,據文義改。}

\endnotetext[17]{「將前」原作「前將」,據文義改。}

\endnotetext[18]{「計到」原作「到計」,據文義改。}

\endnotetext[19]{「了我」原作「我了」,據文義改。}

\endnotetext[20]{「鐵化」原作「鐵花」,據第二回改;下同,不贅。}

\endnotetext[21]{「也聊」原作「他了」,據文義改。}

\setcounter{footnote}{0}

\theendnotes

\part*{姑妄言第十七卷}
\addcontentsline{toc}{part}{姑妄言第十七卷}
\markboth{姑妄言第十七卷}{姑妄言第十七卷}

鈍翁曰。這古城隍示夢一段。一提明衆人來路。照應首回。二明三婦改心之故。不是無因。

常平倉之弊。說盡地方官肺腑。爲上司者能一力淸查。上不負朝廷愛民至意。下使飢荒百姓受福不淺。

擁百萬之富。以萬餘石米濟衆。直九牛一毛耳。在慷慨豪傑爲之。何足爲異。所可異者。出在財主耳。況於又是極鄙吝不堪銅臭之財主。竟慨然爲之。出人意想之外。

寫王恩負心處。正寫小人之奸詐。正人君子往往爲其所欺。及到結局時。何嘗欺了人。自欺耳。爲小人頂門一針。

小(少)林僧傳術一段。是他千算萬計寫來。不如此。鐵氏一生終以角先生爲樂具乎。不如此。童自大何以能多子。更有妙處。峨嵋山人雖已結過。此處又將他一影。

樂公初纔臨任。這一片憂國憂民的心腸。眞有寢食不安之意。此等官那可多得。

楊大之殺水氏。寫盡小人之凶惡無良。彼私人之妻則可。人私彼之妻則不可。水氏一淫婦也。固可殺。以卜通之親夫殺之則可。以楊大奸夫而殺淫婦則不可也。故有水氏索命之報。非報殺淫婦之人。索命於殺淫婦之奸夫耳。這一殺也有妙處。不但結去奸夫淫婦一段公案。且完卜之仕結局。

李幕賓之貪。鄭瞎子之惡。劉大悛之毒。寫盡小人心腸。若非樂公之明察仁慈。童自大亦危矣哉。

吳老兒一生貪鄙。宜乎有杜氏爲之妻。以絕其後。繼而有催(崔)命兒爲之妾。以絕其命。要知非杜氏崔氏之罪。乃此老自取之耳。自作孽不可活。斯人之謂歟。

厥夫多誼。又有厚道之妻。所生子女。自然昌大其後。至於夫名忘恩。其婦又薄。所生之女而爲人妾。不亦宜乎。

\chapter*{姑妄言卷之十七\\
第十七回 童自大捨貴糧救苦賑流民 少林僧傳異術爲歡娛胖婦\\
附 樂府尹念窮黎 楊轎夫殺淫婦}
\addcontentsline{toc}{chapter}{第十七回 童自大捨貴糧救苦賑流民 少林僧傳異術爲歡娛胖婦}
\markboth{第十七回 童自大捨貴糧救苦賑流民 少林僧傳異術爲歡娛胖婦}{第十七回 童自大捨貴糧救苦賑流民 少林僧傳異術爲歡娛胖婦}

話說宦賈童三人向鍾生說古城隍召他們。鍾生暗想道。我蒙尊神恩庇久矣。何不同進去一叩\footnote{此寫鍾生自夢到此。妙。若再說神去召來。便不成話矣。}。正想間。只有(見)一個烏幞頭皀袍角帶的判官出來。傳呼道。奉王旨召爾三人並鍾情一同進去。鍾生吃驚道。王何知我在此\footnote{是個夢境。}。忙隨了那判官進到丹墀。俯伏道。某數年未得瞻仰聖容。今幸到此。特虔誠叩謝。那尊神笑道。你來得好。今該爾諸人夢醒之時。特召爾等來剖示明白。鍾情。爾夫妻前世姻緣。吾神向已示知。彼宦蕚等三人。前世係風流文士。却家道貧窮。也求白氏爲婚。他父母本要於中選擇一婿。白氏因彼家貧寒。誓死不從。皆因此抱恨而歿。後都到我案下。因他三人抱一貧窮之恨。遂至捐生。故使他今生愚醜癡頑。豪華富足。與錢氏買笑逼歡。遂彼前生之願。而錢氏一相遇即厭惡彼等者。亦緣前世之故耳。王又喚道。宦蕚家(賈)文物童自大。爾三人奇(倚)勢橫行。到處作惡。本要奪爾紀算。橫死以報。今因爾等悔心改過。姑從寬釋。爾三人皆因絕嗣。因改過之故。皆得生子。只要爾等執定此心。自能保守家業善終。若再蹈前非。明有王法。幽有鬼神。爾當自省。三人嚇得叩首如搗蒜相似。王又道。取那三獸過來。衆人看時。一猴一虎一狐。匍匐案下\footnote{婦人中。奸詐者無一不猴。悍妒者無一不虎。淫媚者無一不狐。見此不足爲異。}。王問宦蕚等道。爾三人識此麼。三人不知何意。不敢妄答。王笑道。着他現了今形。又一個綠袍虬鬚的判官走上前。吹了一口氣。忽然變做三個婦人。他三人正驚疑間。仔細一看。原來是他各人的妻子。心下大駭。王道。此三婦。前世原來本男身。因前生孽重。墮落畜道。後罪限已滿。始得轉生爲婦人。以爲爾三人之妻室。他雖轉世爲人。獸心未能盡革。故爾悍惡淫妒異常\footnote{世上悍惡淫妒之婦。大約皆係畜類托生者。}。爾等遭其荼毒者。以償前世好色輕生之戒耳。今爾等改過遷善。吾神冥冥之中已抽去了他的妒筋。換了他的惡腸。俱已化成人心\footnote{世間妒婦的妒筋惡腸。安得尊神盡都抽去換却。使這些怕婆好漢受福無量。}。與爾等同偕到老。爾等諸惡莫作。衆善奉行。久久必獲吉慶。去罷。兩邊將吏齊喝一聲。出去。如震霆一般。衆人齊叩首趨出。因他三人改過獲福。這一番事有四句打油道。

\begin{quotation}

人能行善當生福。事若違天必受殃。

此理易明何不省。寧爲良懦莫橫強。

\end{quotation}

鍾生一驚醒來。原來是一場大夢。想了一想。一字不忘。喚醒錢貴。向他細說。方知有這些往因。鍾生又想道。我雖得此奇夢。不知他三人可有夢否。改日會着一問。若果此夢皆同。就眞是奇異了。鍾生得夢之夕。那宦賈童並猴(侯)氏富氏鐵氏六人。所得夢皆同。醒了。各人夫婦細說夢中之語。深爲詫異。這三婦甚慚。深悔向日之醜態\footnote{若非抽筋換腸。決未必知慚。世間惡婦妒悍而不知慚悔者。定是未曾抽筋換腸之故。}。這宦蕚還不深信。恐是他自己偶有所夢。尚在疑心之間。叫人請了賈童二人來。坐下。問道。昨夜我做了一個奇夢。夢見你二位連二位老嫂嫂都在那裡。二位賢弟可有夢見什麼。他二人大驚。各述夢中所見所聞。無不稱奇。遂道。昨夜有鍾兄的。我們一同過去再問問他。又一齊到鍾生家來。鍾生問道。三位兄同來賜過。必有所謂。想是都做了甚麼夢。三人驚道。弟輩正是一樣的夢。昨夜兄也在彼的。曾有所見聞否。鍾生亦備述了一番。因笑道。三位尊嫂的前身眞令人可畏。虧三兄的福量好。竟熬過來了。他三人也笑道。神靈已改了他們的心腸。從此不懼了。笑了一場散去。他大家方知這番會合都是前生的事。雖然已是親戚。更加親密。那三位夫人也越發親熱起來。時常往來。此後連一絲悍妒之氣全無。至於枕蓆上之事。又是婦人常情。不足爲責。宦賈二人各有壯大本錢。久矣將侯富二婦征服。只是鐵氏身子越胖。陰戶越肥越深。童自大之物越用不得了。況且又是那大先生將他做了學館。時常出入。揎得其寬無當。童自大間或試試。弄上了一會。只見那人同二物相合並不知覺。童自大竟棄前而取後。前門竟奉讓了先生。日久壞了。又買了八九個來。憑他取用\footnote{一個館內用八九個先生。何須先生之多也。因此憶起一故事。唐六如祝子(枝)山二公。有人托其薦館者。無不允諾。皆約至新正某日進館。至日。先生十數人皆齊集他家。以候送館。二公入內。頭挽了個髻。坐(作)幼童狀。出來拜師。衆驚問之。二公曰。何處有許多學館。就是我兩人延師。衆人知爲所耍。大笑散去。今他一個館中買了八九個先生來。無一個學生。}。只難爲了兩個丫頭的手腕。一夜。他夫婦同臥。童自大道。我好些時沒有走水路了。再試試看。遂弄了進去。抽了兩下。童自大道。這不中用。還是後門有些邊岸。鐵氏笑道。難道你這麼着着就一點樂處也沒有麼。童自大道。四邊都挨不着。就像個小娃娃坐在大澡盆裡面一般。有甚麼樂處。鐵氏道。人在澡盆裡洗澡。到底人也快活。童自大道。這樣說。我弄着。你必定也快活了。鐵氏道。好像個小耳挖放在大耳朶裡。那有甚快活。童自大笑道。你說人在澡盆裡洗澡快活。難道耳挖掏耳朶耳朶裡不快活麼。兩人大笑。將後庭舞弄了半夜方歇。再說鍾生一日在書房閒坐。翻閱宋史。看到韓侂胄建一花園。竹籬茅舍。宛如村莊氣象。心中甚喜。道。惜無雞犬之聲襯點耳。少頃。聞雞鳴犬吠。遣人視之。乃京兆尹趙師〖睪廾〗伏於籬下作雞狗之聲。侂胄大喜。又有一個諫議大夫程松。他買了一個美人進與侂胄。取名松壽。侂胄道。奈何與大諫同名。程松道。正要使賤名常達尊聽耳。鍾生掩卷嘆道。小人無恥。爲諂媚之事。猶可言也。士大夫旣登廊廟。爲朝廷之臣宰。尚然爲此。廉恥喪盡。是何心哉\footnote{笑罵由他笑罵。好官在我爲之。二語盡之耳。}。正嘆笑間。忽梅生到來。滿面笑容。問道。兄所看何書。鍾生答道。弟偶看宋史。到趙師〖睪廾〗程松之媚侂胄。正在可笑。梅生道。千古來。代不乏人。又不獨二人可笑。今日眼下就有一個可堪噴飯。弟特來爲吾兄言之。以供一噱。鍾生道。請道其詳。梅生道。舍表弟昨日曾來奉拜麼。鍾生道。昨日承他賜顧。弟即往拜矣。梅生道。舍表弟當日之岳翁王翰林。兄也曾會過來。弟所說可笑之事。即此人也。鍾生道。弟當日一見其人。即知爲不端之士。故不敢親近。每訝令母舅老年伯高明君子也。當日爲何與彼結親。雖有此心而不敢言。彼令愛已故。令表弟也另娶了。今日有何笑話。梅生細細說他的這可笑之處。正是。君子不失爲君子。小人枉自做小人。你道是何緣故。鍾生的母舅姓多。單名一個誼字。二十歲就遊了庠。是個慷慨丈夫。心直口快的男子。娶親後氏。可稱聰慧賢淑。生得一女二男。女適陳宅。陳仁美中了進士。選了陝西褒城縣知縣。即周幽王時褒似(姒)所產之地。長子名必達。他二人當日與鍾生同窗。都是廣先生的門人。多必達與鍾生又是鄕榜同年。次子必進在庠。這多誼少年的時候有一個窗友。名字叫做王恩。幼無父母。與兄嫂同居。兄嫂待之如奴隸。鶉衣百結。終日枵腹。以草帶束腰。忍飢以度。他兄嫂只當不曾看見。他那令嫂比蘇季子不爲炊之嫂。漢高祖的戞羹嫂。還利害幾分。那王恩苦在心頭。無門可訴。他雖二十多歲。是一個書呆。只知道捏着個書本。一日蒼蠅之聲不絕。哼哼的念。軒轅彌明古鼎聯句中有兩句。正是他的行樂圖。道是。

\begin{quotation}

常於蚯蚓竅。時作蒼蠅聲。

\end{quotation}

他除此以外。別無一能。拿輕不得。負重更不得。他每每要賭氣出來。不但無置身之地。且無餬口之方。別人窮無立錐之地。他眞窮得連錐也無。當日有一個笑話。正合着他。

\begin{quotation}

一個人無處謀生。專與喪家做陪堂。一日。他家出殯。他無(撫)棺痛哭。道。你的屍靈倒有處去了。我的這屍靈放在那裡。

\end{quotation}

正是這王恩之謂了。一日。他嫂子生辰。他娘家送了些魚肉酒麪之類來給\endnotemark[1]女兒。他烹庖了。留着夫妻同享。但礙着小叔。要給他些吃。心中又捨不得。不給他些。又覺不好意思\footnote{還算面皮薄。要在今日。大多好意思者甚多。}。遂忍不住發話道。當日公婆又不曾留下半點家私。今年二十多歲的後生。不想些營運。只啃哥哥嫂子。臉彈子也不害羞麼。成日牙疼似的捏着個書本子。哼也哼得出飯來吃麼。要等你哼出個舉人進士來。哥嫂也好累死了。虧自己也過得去。嘴裡說着。將瓢兒碗兒摜得一片聲響。王恩一腔忿氣。走到多家來。多誼見他滿面怒容。兩眉如鎖。心中像有萬千爲難的事一般。多誼問道。我看兄像是有甚麼不悅之事麼。王恩長嘆了一聲。忍着淚。不能答。多誼道。我與兄自幼同窗。所謂丱角之交。有事何妨爲我言之。古押衙云。老夫一片有心人也。弟雖非押衙之比。然亦有心人也。或可爲兄助一臂之力。也不可知。王恩不得已。將他兄嫂惡薄的話說了。復墮淚道。今日投身無地。欲住不可。是以悲耳。多誼激出一腔義氣來。道。世情囂薄。手足之誼何至於此。罷。兄旣無處棲身。若不見棄。就在我小齋來住着。但恐家常日食不堪。兄若不責。弟還可以供給。就是幾件冬夏衣服。弟也還力有可爲。兄意若何。王恩道。承兄雅愛。弟銘刻五衷。但歲月甚長。如何敢常在府上叨擾。多誼道。朋友乃五倫之一。近來人情惡薄。將朋友一道幾幾廢盡。弟每每痛恨。我與兄多年友誼。猶如手足了\footnote{不愧名多誼。}。何必還做客套語。不妨今日就來。弟掃榻以候。王恩見他義氣俠腸。感之不置。說道。旣承兄見愛。弟還有幾本殘書取來。遂起身別去。少刻來了。捲了一床破被。綑了一束爛書。背負而來。到多家書房住下。他竟毫不務外。終日對着書本咿喔。多誼喜道。他有這一番苦志。將來必有可成。安心要培植他成人。先替他換了一身衣服。又做了被褥與他。數月之後。多誼向他道。弟癡長吾兄三歲。大小女今已八齡。古云。不孝有三。無後爲大。兄今已二十外了。婚姻一事。亦不可緩。王恩道。弟之此身。當日尚不知飄泊何所。蒙兄收留。已出望外。今在此得衣食豐足。可以讀書。就是萬幸了。何敢復何奢望。想及婚姻一事。托兄福庇。異日若稍有寸進。再做商議罷了。多誼也就不做聲。却暗暗叫人打聽。替他〔尋〕親事。說成了一個老童生薄家的女兒。整二十歲。到了下定之日。纔對王恩說知。王恩感恩不盡。道。兄如此愛弟。雖是兄一片熱腸。但使弟何以克當。中心藏之。何日忘之。願終身效銜結以報耳。多誼笑道。丈夫處在世間。於陌路之人施恩。猶不望報。何況你我朋友之間。些須微情。怎麼講報答的話。兄不但輕弟。亦自輕了。王恩不敢復言。惟心中感愧而已。多誼就將書室收拾。做了他的洞房。到了吉期。娶過門來。一應供給。皆出自多誼。是不用說的了。後氏時常請薄氏到後邊吃茶飯。閒談說笑。如嫡親妯娌一般的。那薄氏心地聰明。齒牙伶俐\footnote{世間聰明伶俐人無有不薄。倒是老實人還有些厚道。}。二人着實相投。那年王恩進了學。多誼甚喜。以爲不枉收留他一場。藍衫酒禮幷送學師之費。皆是多誼拿出。次年多誼生了一子。就是多必達了。王恩之妻薄氏同月產了個女兒。時光迅速。日月如流。不覺就是五個年頭。那日多誼同王恩正坐着閒話。見那兩個孩子從裡邊出來。相攜着頑笑。如親兄妹相似。多誼歡喜得了不得。笑說道。我同兄眞算得異姓骨肉了。我看這兩個孩子也如同兄妹。我同兄何不做個先朋友而後親家。把兩個孩子配成夫婦。兄意若何。王恩受了他的無限恩德。三口在他家穿吃數年。門檻都踢鋊了。毫無閒言。連妻子都是他替娶的。何況要他的女兒做媳婦。可有不肯之理。他每常就想扳這門親。好圖久遠。因自己還靠着他家。自鄙寒賤。不敢啓齒\footnote{有此數語。彼後日之負心。愈覺可恨也。}。今聽見說這話。滿臉是笑。說道。承兄不棄。小女得配令郞。眞得所天了。但弟不敢仰扳耳。多誼見他喜允。進來對後氏說知。後氏道。我也久有此意。如此甚好。王恩就吿訴薄氏。薄氏巴不能夠。連聲慫恿。過了兩日。多誼選了個好日期。備了兩席酒。先送了幾件頭面。兩套小衣服與媳婦。做小定。然後請王恩吃喜酒。請了女婿陳仁美。外甥梅根來相陪。做個媒人的意思\footnote{後來始終成全。陳仁美之大力。所以名成人美也。}。內裡請薄氏。後氏母女二人陪他。一家甚是歡喜。自不用說。過後。他男婦四個親家愈加親熱。多誼同王恩走了幾科。總不得中。到了天啓甲子科。他二人同女婿陳仁美同進場去。不意放榜之日。王恩同陳仁美都中了。多誼反落孫山之外。多誼雖然未中。見女婿中了。還在次。見王恩中了。倒歡喜得比自己中了還勝。他女兒去年嫁到陳家。女婿中的這一日又添了個外孫。眞是喜事重重。次年。王恩上京會試。路費家人皆是多誼預備。托女婿與他同往。一路到京會場。又同中了進士。王恩殿在二甲。選入庶吉士。報到家中。多誼那喜眞快樂不過。也不是喜親家連捷。圖他的榮耀。喜的是王恩一個無歸的人。成就他妻子功名。不負當初一片熱心。次年。王恩給假回來祭祖。仍在多家住着。拜謝多誼夫婦。感恩戴德的話說了無限。口口聲聲念之不置。他此時是榮歸了。從不上門的親戚不知從何而來。一日來來往往拜賀不絕。連他那無情兄嫂。雖然不曾像蘇秦的兄嫂側目而視。蛇行匍匐的樣子。也老着臉重新來親熱。做了許多醜態。一應賀客來往。都是多誼替他應酬。限期將滿。要回京去。多誼勸他帶了家眷同往。此時他女兒十三歲了。生得十分標致。多誼夫婦疼愛他無比。恐王恩路費不敷。又送了些盤纏。多誼後氏同他夫婦同居了十數載。一旦言別。心中戚戚然。戀戀難捨。那王恩薄氏毫無留戀之情。歡然而去\footnote{忘恩薄情已見一斑。}。王恩到了京中。那時正是魏璫秉政。他的頭一個乾兒就是大學士魏廣微。王恩初進。不敢投見魏忠賢。就拜在魏廣微門下走動。那魏廣微有了這樣個賽皇帝的太監老子。自己又做了首相。聲勢無雙。富貴已極。是浣沙(紗)記夫差打圍上說的。富貴已極。不圖歡樂待何時。他就是這個意思了。別無他想。只要尋些美女到家中來取樂。差人四處訪求。王恩聽得這信。打動了他一個富貴的妄念。同薄氏商議道。我如今名雖做官。一個翰林院庶吉士。是人說的寫大字拜帖的窮鬼。巴到那一日纔有陞轉。我想走一個捷徑。這魏中堂他因做了魏上公的乾兒。不過一兩年間。就做到閣下。我官卑賤小。不敢望到魏上公跟前。做他的義子乾孫。如今在魏中堂的門下。若得了他歡心。甚麼一日三遷的事怕不得。他如今發狠。在邊外尋美女。我家女兒雖算不得十分絕色。也還算個十全的容貌。雖纔交十四歲。已長成大人規模。我想獻了與他。不愁他不歡喜。果然中了意。我這官。眼見得騰騰的就起去了。他一面說着。一面挺着胸脯。滿地走到\footnote{好形容。}。那時就是琵琶記上的曲子了。唱道。

\begin{quotation}

身穿着紫羅襴。腰繫着黃金帶。皀朝靴在脚下踹。五花頭踏馬前排。

\end{quotation}

請敎那時豈不體而面乎。你也就是響噹噹的一位夫人了。珠其頭而緞其體。鳳其冠而霞其帔。黃其傘而四其轎。呼其奴而使其婢\footnote{則天朝有個四其御史。他今是個八其翰林。}。搖擺着道。何等威武。又把脚跌了兩跌\footnote{描寫醜態甚趨(趣)。}。但可恨許過了多家。當日受他厚情。擾他多年。又替我娶你。這個恩情忘不過去。二來女兒年幼。魏中堂五十多歲了。怕不相配。恐女兒不願。你的意思怎麼說。薄氏道。人說黑心人纔有馬騎。如今世上不忘恩負義的。能有幾個。古語說。大恩不報。何況於小惠。你當日在他家。我是見的。每日不過是粗茶淡飯。沒有見他弄甚麼三牲五鼎的供養。你娶我的時候。不過是幾根簪棒。套把衣服。所費有限。我在他家多年。那一年不幫他做些針指。他女兒出嫁。我幫着做了多少生活\footnote{沒良心人大都如此。受人大德。一笤帚掃得乾乾淨淨。自己稍有小惠到人。便念念不忘。}。你中舉人進士。雖費了他幾個錢。一來是你的命好。二來是他要做疏財仗義的好漢。也是他自己要博好名。豈單是好心爲你。至於女兒許他家。也不過是一時兒戲的話。又不曾大酒大禮的行下。癡癡的守着這個名做甚麼。等女兒到了魏家。你寫個信帶與多家去。只說女兒死了更隱密。他往那裡去查帳。就算着那知道我女兒與了魏家。他可敢到魏家去哼一哼麼。我們有魏府做了靠山。料道也不怕他\footnote{心腸愈轉愈惡。但人心如此如此。天理未然未然。}。我說的可是否。若說怕魏閣老的年紀大。那甚麼相干。他去做閣老的小。穿吃不了。不強似嫁那秀才家的少年兒子麼。況且我們養他一場。拿他替娘老子出些力。也不爲過。就是他不願。且瞞着他。送到了那樣人家去。還怕他跳到那裡去。且顧了我夫妻眼下着。也顧不得他了。你不要呆。趁早去行。我做父母的且博一場富貴。也不枉生他一場。不然。守着這淸淡衙門。活活的熬死人呢。王恩聽了薄氏這些話。笑逐顏開。不住點頭道。說得妙。說得妙。有智婦人勝似讀書男子。好見識。好見識。次早。到了魏廣微私宅門口伺候。等到將午。餓得腰酸腹痛。在管門的人跟前陪了多少小心笑面。再四相求。纔傳稟了。魏廣微在書房中。傳了進去。見了禮。魏光(廣)微叫他坐下。他做了許多諂媚的樣子。說了無限奉承的話。纔說道。門生蒙師相夫子收錄。天恩無以爲報。門生有個親生幼女。不敢稱爲美麗。也還可寓目。愚夫婦意欲送到老師相府中爲婢妾。不識台意可肯俯納。不敢造次。門生先來上達。魏廣微大喜道。旣是賢契閨秀。我怎麼好立爲小星。王恩深深一恭。道。此不過門生仰報老師相天恩之萬一。若能小女得充下陳。留備驅使。不但小女之萬幸。亦門生愚夫婦之萬幸了。魏廣微道。你有這樣好情。我亦當有厚報。旣承你雅意。今晚就可過來。更妙。王恩道。小女在家所穿戴者。不過荆布。如何送得到府中來。旣蒙老師相不棄。還須俟一二日。製些須衣飾。纔可送上。魏廣微笑道。這有何難。問了他女兒身材高矮。遂吩咐小廝。傳了進去。要了一匣子金珠首飾。數套衣服\footnote{是個宰相家行事。}。一個猩紅氈包裝着。拿了出來。魏廣微命交與王恩家人拿着。王恩辭了回家。忙叫薄氏將女兒香湯沐浴徹底。換了衣服。梳頭洗面。戴上滿頭珠翠。那女兒也不知是那裡帳。問着娘。他只是笑。也不回答。收拾完了。日色將暮。一乘轎子。王恩親自送到魏府。傳稟進去。許多丫鬟僕婦出來。簇擁而入。王恩歸去了。魏廣微見好個女子。年又甚少。十分心愛。當晚就寵幸了。那女子知他自幼許了多家。今日忽然被父母送到這裡來。被這個五旬多的蒼髯老漢同他比翼鶼鶼。鸞顚鳳倒起來。心雖暗恨。說不出口。那王恩以爲女兒這一去。雖不能像董卓之於蔡邕。一日三遷。大約不過一二月之中。定然高轉。不想過了數日。便是冬至。天啓童騃愚昧。自己不去郊天。魏廣微是首相。遣他代祭。他半夜就到天壇祭了回來。又朝賀禮畢。他將望六的人。連日幸王恩的乃愛。享那又小又嫩的美物。斲喪過了些。又辛苦了半夜。一早晨神疲力倦。要到他令尊魏噹(璫)處叩賀。因身子怕動。恐這一去。留賜酒飯。未必就得回來。況且父子之間。自有憐惜兒子的。那裡就肯責善。且回家歇息歇息再去。不意魏忠賢朝賀回府。闔朝大小文武乾兒門下廝養都來叩賀。惟獨長子魏廣微不到。他那裡知道是被新得的小媳婦弄癱了。只疑他目中無父。大怒罵道。這狗弟子孩兒。你是個什麼黃黃子。咱擡舉你做個宰相。也就算咱的大恩了。你今日竟公然連我老子都不認得了。這等可惡。要個個孩兒們都看起這個樣兒來。我這個老子豈不是虛設的了。叫過小兒子錦衣衛田爾耕來。吩咐道。魏廣微這狗攮的弟子孩兒。連咱老子的頭都不來磕。好大膽子。你去把他即刻逐出都門。不許容情遲緩。遷延片刻。快快的去了。來回咱的話。那田爾耕奉了恩父的怒命。那裡還顧得長兄的私情。親帶了許多官旗校尉到他家驅逐。魏廣微吃了些人參湯。正在暫歇。聽了這信。魂飛魄喪。這田爾耕素常諂事魏廣微。奴顏婢膝。要一奉十。放一個屁他也是要欽此欽遵的。二人極其親厚。魏廣微此時懇他稍緩須臾。要去面見魏忠賢哀求。或可挽回。田爾耕不但不准。且放下臉來。道。上公待你的恩典也算極厚了。你今日竟公然藐視他。冬節都不去叩賀。不加罪於你就是萬幸了。趁早走路是你的造化。我怎敢徇你的私情。違了上公的嚴旨。況你目中無父。我又焉得有兄。虧你還讀過幾日書。從井救人的事也有的麼\footnote{寫小人反面無情。面孔口角如見。}。快快的走。不要討我個大沒趣。魏廣微見他這樣子。大非往昔。料道求他也沒用。況且又恐那沒卵袋的假老子。比不得有㞠子的眞老子。還有些天性之恩。或再觸了他的怒。連性命還不能保。只得帶領家小踉蹌出城而去。及至王恩得了這信。連忙趕了去。要看看女兒。他已經去遠了。只得忍淚回來。父女連別也不能一別。生生的離散了。那時人人都去拜魏忠賢做老子。也有一個笑話兒道。

\begin{quotation}

一人拜在他門下做了個乾兒。欣欣自得。有一個朋友戲他道。你拜魏上公做老子。倒也罷了。不怕難爲了令堂些。那人沈吟了一會。道。他是沒有卵袋。家母還不曾吃什麼虧。

\end{quotation}

却說王恩見把魏中堂頃刻逐去。把一座泰山化成一泓秋水。悔恨無及。一級不曾陞。半文不曾見。把個嬌嬌滴滴的女兒白白送去。垂首喪氣。惟有咂嘴咨嗟。頓足嘆恨而已。反被薄氏罵了數日。說他見事不確。如何就行。當日說得這魏閣老怎樣尊貴。如何被一個太監老子就攆去了。帶累了他的女兒。王恩也無言可答。只是哎哎嘆氣。後來寫了封書帶與多誼。內中說女兒不幸於某月日身故。不能得終前盟。並許多謝他的鬼話。多誼見了書。念與後氏聽。夫妻着實悲嘆。他倒不惜失此親家。倒可惜失了個好媳婦。也就放過一邊。此時他女婿陳仁美與王恩同榜進士。等了兩年。補了褒城縣知縣。已同女兒上任去了。到了天啓七年丁卯科。多必達同鍾生那年中式。他已定了個荆貢生的女兒爲媳。榜下成親。兩重喜事臨門。又是一番熱鬧。那年八月內。天啓駕崩。崇禎以皇弟信王嗣位。就是魏璫的賢郞楊維垣攻擊他起。舉朝紛紛參劾。逆璫事敗。附逆諸人盡皆問罪。魏廣微雖係逆璫乾兒。後革職逐去。先親後疏。姑從輕議。比傅應星等減罪一等。家私籍沒入官。闔家男婦發陝西慶陽府充軍。王恩的令愛不消說是跟着去了。王恩係魏廣微姻黨。株連革職回籍。他夫妻一場妙算。富貴不曾到手。先送掉一個女兒。後連功名罣誤。雖是忘恩薄情之報。然而人算不如天算。奈何。奈何。他眞是。

\begin{quotation}

王郞妙計高天下。賠了嬌兒又折官。

\end{quotation}

多誼在家聞了這信。向後氏道。王親家別無子女。他與魏中堂是什麼親家。如何就到連累革職的地位。後氏想一想。道。他前次寄信說他女兒死了。我常看那孩子。不像個短命的。我素常疑心。不曾出口。他做了官。恐嫌我們是秀才門弟。或者是把他女兒與了魏家了。多誼變色道。豈有此理。你婦道家見淺。以小人之心度君子。這樣的事。禽獸之所不爲\footnote{要知這樣的事。禽獸所不爲。偏是衣冠中人肯爲。}。他一個讀書的人。可肯做這無恥壞心的事。多必達在傍邊說道。如今的世情。這樣事也是有的。母親這一想倒也不錯。多誼道。胡說。少年人也跟着這樣亂講。你母親婦人之見罷了。你也曾讀幾行書。這話如何出之於口。次年。多必達上京會試。不第而歸。那王恩夫妻已回來了。還是一個空囊。他做了一場官來家。女兒又送了人去。沒有還來多家住的理。只得拼湊買了幾間房子棲身。家中艱難之甚。多誼雖見他女兒死了。念昔日交情。還時常資助他柴米盤費。王恩見多家近來比當日更覺興旺。女婿又中了舉。娶了妻。一家和美。想起女兒來。嫁了他家豈不好。常同薄氏暗暗悔恨飮泣。見多誼還常常照拂。良心不死。又是那內愧。多誼一日偶然同他閒敍。問他同魏家是甚麼親。竟到株連至此。他無言可答。謂說。當日承魏公垂靑。時常到他府中。他有一個心愛的幼兒。認弟做義父。所以說是親家。因此拖累了。多誼嘆道。君子不可不擇交\footnote{辱翁曰。燈臺不照己。}。兄也是大通明理的人。難道冰山泰山都看不出麼。那時逆璫上無君父。自不能久。這些依草附木者。又豈得長。原不該同他親近。都是自錯。怨不得人。可惜十數載燈窗辛苦。功名猶在次。還落一個汚辱之名。只好自恨罷了。多誼是個眞心的人。就把他的假話信了實。那裡知道魏廣微是他令愛沾皮貼肉的親家。還進內向後氏多必達說知其故。復道。你們向日還疑他是那樣壞人。我就知其決乎不然。那王恩夫婦要靠他家過日子。見了多必達夫婦。一口一個姑爺姑娘。假做親熱。多必達聽他兩口子說他女兒之死千眞萬眞。也就信爲確然。多必達幼年同他女兒親如兄妹。又曾下過定。想念舊情。也時常來往。過了兩年。多誼接女婿來信。已經行取進京。陞了山西太原府推官。舅子若上京會試。務必繞道到任上一會。以慰數年久別。多誼見女婿榮陞。心中甚喜。王恩知道這信。越發自恨。他兩個是同年。那一個聽天由命的。何等榮耀。自己趨炎附勢一場。弄得冰消瓦解。隱恨在心。說不出口。且說那陳仁美行取之時。沿路州縣拜往。餽送下程。好不熱鬧。一日。到了慶陽店中住下。他偶然到店門口看看。只見一個人來尋那店主。道。我們夫人問你的回信怎麼樣了。店主道。今日有位老爺下着。不得去討信。明日纔得去。那人道。你做媒人圖中用錢使。倒要我們兩頭跑。咕咕噥噥的去了。陳仁美問店家是什麼事。店主道。小人當着個官媒。隔壁這魏夫人是魏閣老的奶奶。充發到這裡來的。魏老爺去年死了。家中窮了過不得。有幾個小奶奶要賣給人做妾。托小人去賣。都賣完了。只剩了兩個上好的。價錢大些。昨日有人要。叫小人今日去討信。老爺駕到小店。不得閒去。纔又着人來催。陳仁美道。你可知道這兩個小是那裡人。可果然生得好。他也肯與人相看麼。店主道。小人都見過。生得眞好。一個是北京人。一個是南京人。這個南京的還不到二十歲。生得又強些。說他是好人家的閨女。他父親還是個官兒呢。他旣要賣。可有個不與人相看的。陳仁美道。旣與人相。你把那個南京的帶來我看看。遂走了進去向多氏說。多氏道。你要娶小。要那後婚老婆做什麼。陳仁美笑道。我那裡要他。店主說他生得好得很。不過帶來看看。正說着。店主帶了一個女子進來。多氏一見。便覺眼熟。問他道。\endnotemark[2]你是南京那一府的人。你家姓什麼。他答道。我姓王。就是應天府人。多氏忽然想起他是王恩的女兒。他兄弟所定的媳婦了。這女子在他家長了十二三歲。終日相見。還替他梳頭。敎他做針指。如何不認得。那女子別他時年幼。況在異鄕。一時想不起。倒忘記了多氏。又眞(問)了他一句道。你當日在南京誰家住來。答道。在一個姓多的親戚家住的。多氏聽了這話。越發是他無疑。問道。你如何到魏家的。那女子一腔氣慣(憤)。多年鬱結。遂將他父親是官。他並不知道被他父母送到魏家。以至到此處來的話。詳細說了。落了幾點淚。多氏也不再問。仍叫店主領回。他夫妻商議道。王恩這個沒良心的畜生\footnote{畜生中竟頗多有良心者。恐此類連畜生不如耳。}。受了我家多少恩惠。纔得一步好處。便忘恩負義。獻女豪門。還假說女兒死了。來哄我父親。我們如今把這女子買來。帶了去。等我兄弟到京。竟與他做小。帶他回家。看他父母有何臉面相見。定了主意。叫店主講明價錢買了。次日起身。到了京中。後來陞了太原司李(理)。故此寫信回來。叫兄弟到他任上。也不說破其中緣故。多必達中了甲戌進士。回家繞路到山西看姐夫姐姐。到他任上相會了。飮酒接風。多氏道。我替你尋了個小。等了這三四年你纔來。多必達道。雖是姐夫姐姐疼我。恐怕回去父親嗔怪。陳仁美道。不妨。又不是你自己尋的。是我同令姐的意思。我細細寫信稟知岳父。料道決無話說。但這女子原是魏中堂的小。不是女兒了。因爲生得好。我同令姐在陝西買了帶來的。多必達正在少年。離家日久。見姐夫姐姐這樣美情。又聽說女子生得好。有何推辭。欣然領命。多氏命收拾了間房子床帳。叫那女子洗沐。更了新衣以待。這王氏一買來。以爲是陳仁美要他做如夫人的了。數年總不見他說及。每日好食好衣養膳。不知何故。今日聽說是贈他舅爺。是新科少年進士。心中暗喜。到晚上見多必達進房。好一個齊整少年。越發相愛。多必達見他生得果好。也甚快樂。但是覺像在那裡見過一般。十分面熟。再想不起。二人上床。春風一度之後。多必達盤問他的家世。他將衷腸細吿。方知是王恩的令愛。多必達大詫道。怪得我覺面熟。原來是你。也把自家姓氏前後的事說了。王氏羞愧無地。多必達推枕穿衣而起。叫人請了姐夫姐姐來。說道。這女子原來是王恩的女兒。他姐姐笑道。我當日一見。就認得他。我故此買了來。安心叫你帶回去。叫他父母看看。羞一羞這忘恩的小人。看他有什麼臉面見鄕黨親友。不然我替你買個妾做什麼呢。多必達道。他父母如此無良。我怎肯要這女子。陳仁美道。一來時令姐就問過。是他父母瞞着把他送到魏家。他還不知。及到了那裡。欲回已是不能。這也還怪他不得。你如今爲妻則不可。做妾却不妨。不但羞辱他父母。正可出你之氣。多必達想了想。甚是有理。留做了小星。見彼頗聰敏知事。倒也心喜。住了幾日。辭了回家。到了家中。他拜過天地祖先。又拜過父母。然後叫王氏拜見。並見了荆氏。多誼見兒子中了進士榮歸。心中甚喜。見他娶了妾回來。大有幾分不悅。多必達將姐夫的書呈上。多誼看了。多必達又細說底裡。多誼後氏不勝恨怒。道。有這樣沒良心的人。眞是人質獸行了。那禽獸聽得你回來。淸早就在外邊坐着。不要放了他去。再着衆人去請了他妻子來。當着衆親友。叫他父女相見。看他何以見人。遂差人去請薄氏。薄氏聽說女婿中了。歸到家\footnote{當日眞女婿却弄成假女婿。如今雖似丈人却算不得丈人了。}。叫人來請。他來得也沒有那樣快。到了多家上房。有許多親戚內眷都相見了。他見多誼夫婦怒容滿面。不像每常相會親熱。又不敢問。多誼見薄氏來了。叫人出去請王恩同衆親戚都進來。說道。古人有還魂的事。我常不信。今日竟有一個女子死了數載。忽然又活轉來。昨日我小兒在途中娶了他做妾。帶了回來。特請列位來見一見這異事。因對多必達道。你叫了他女子來。頃刻來了。一進房門。王恩薄氏正在疑心要看看這還魂的女子是怎個模樣。不想是他的令愛。他夫妻羞得要死。掩面要跑。被他女兒一把拉住。連哭帶罵。數說了一番。此時對着許多男親女眷。他兩口子比殺一刀還難過。掙脫跑了回去。夫妻互相埋怨了一場。在城中無顏見人。躱了幾日。將房子賣了。遷往遠鄕而去。後來竟不知下落。眞是。

\begin{quotation}

饒伊掏盡西江水。難洗今朝滿面羞。

\end{quotation}

這一件事傳得人人皆知。無不唾罵王恩爲小人\footnote{王恩固當罵。或有王恩之類亦唾罵之。則不可也。昔有一笑談。衆人共坐。不知誰放一屁。其臭不可聞。衆人指定一人笑罵之。其人大笑。衆問其故。彼曰。我笑那放屁的也在那裡罵我。}。梅生那日也在表弟家。目覩這事。今特來相吿鍾生。鍾生笑道。令表姊丈處得他好。把這些負心的小人。也叫他知些警愧。大笑而別。且說自崇禎七八年來。山東河南連年蝗旱。又屢經流寇。生民塗炭。這些逃出命來的百姓。先還羅雀薰鼠救飢。後來連草根樹皮都吃盡了。弄得易子而食。析骨而炊。那困苦之狀。眞個傷心。雖有幾次恩旨賑濟。但這些地方上的州縣官。把那常平倉的米。久矣乾轉入在他的囊中。倉內顆粒無存。上司通同作弊。都素常知道。奉了旨。不過行下文書。來叫賑濟。州縣官正愁這米沒處開銷。見了這文。好生歡喜。也不過空回上一角文去。已經賑濟了。這叫做虛應故事。百姓躭了虛名。州縣得了實利。餓得七死八活的窮民。何嘗沾了一升半合的恩惠。大小官員大家鬼混而已。誰人肯盡心盡力。爲國爲民。這些百姓雖知朝廷有這樣大恩。他們虛沾其惠。料想到上臺處吿也是沒用。不過如水上打了一棒。人說天高皇帝遠。又誰肯到京中去吿。窮的力不能去。富的又不肯去。就有幾個義憤些的要爲窮民去出頭。又想這個閽也是難叩的。事也便中止。這些百姓站不住了。以爲南京是個大去處。都奔了下來逃命。約〈約〉有數萬多人。三停中沿途餓死了有一停。此時十月天氣。這些窮百姓可還有什麼衣服。不過一衫一褲而已。有一件魚網般破棉襖穿着。就算富足得很了。又凍死了有一停。只有萬餘人口。却也都是尪羸骨立。厭厭待斃形狀。人來得多了。又沒處存身。這一年。値南京也大旱。米價湧貴。每常的米不過七八錢一石。一兩就算貴了。這年因湖廣江西兩省都遭流賊之害。也不甚收。地方官不許米糧出境。江南的米價就長到二兩四五錢一石。本地人自給不暇。那裡還有得捨與別人。這萬餘人在街上哭喊叫化。慘不可言。日裡旣不得吃飽。夜間又無處棲身。就都蹲在各寺廟並人家門口過夜。身上單寒。無日不死許多。地方上多官雖未必無救濟之心。但不肯盡心去畫一救濟之術。都推聾裝啞。竟做不知。却說那童自大一日有事出門。在街上走過。看見這些男婦攜兒挈女。百結鶉衣。鳩形鵠面。都不似人形。又聽得人說他們棲身無地。乞食無門的這些苦楚。他心下愀然悽慘。自己暗想道。我家的富也算到極處了。我連年托天福庇田上大收。各礲房內現堆着許多稻子。我一家也吃不了這許多。我的銀子也夠了。又不犯着去賣。不如做個好事。捨了。救這萬把飢民。也是一場義舉。況我前日夢見我家奶奶竟是一隻大黑狐狸。那一位城隍爺說因我改過。神道保佑。暗化了他的凶心。不然我已死在他手裡了。如今他也竟賢慧起來。可見神道爺說得一點不差。前次我雖擺了那幾日戲酒。破費了些銀子。不過只算得不吝嗇了。還恐有人背地說我臭的。我再要做了這件大事。一來報答了神恩。二來人不但不敢說我臭。還要誇我香呢\footnote{自古及今。能流而博香〈而博香〉名者。能有幾人。不意此老呆有此巨識。}。再者。我聽得人說。人生在世。只要求妻財子祿壽五個字完全就好了。自算道。我的妻也有。妾也有。雖然醜些。人說醜是家中寶。他如今又不打我。又不罵我。又不管我。快快活活的過日子。這就儘夠了。我吃的有。穿的有。用的有。銀子堆着的有。鋪面佃房洲場田地樣樣都有。財字是不用說的了。子字我有了一男一女。我如今人說一個兒子是險子。我若再做些好事。或者龍天保佑。再養兩個。也不可知。不然。只求這兩個長命百歲。聰明伶俐些。人說好的不用多。一個抵十個\footnote{他這一種知足的念頭。便應享大福大壽。較那貪無厭足者。何啻天淵。}。也就罷了。祿字人說官高必險。我雖是個監生。人看銀子的面上。誰不叫我聲老爺。敬我幾分。俗語道。有錢的大十歲。無錢的小一輪。我看那沒錢的窮官。還不如我體面\footnote{窮官豈只於不如財主。唐末司空圖曾爲相國。破後至於無食。一日。途遇一銀工。乃向在他門下者。憐而邀至家。盛設款待。司空圖感而贈之以詩。末句云。悔不當初學冶銀。失時宰相求爲銀工而不得。況於窮官乎。}。這也就罷了\footnote{多少讀書人求進而不知止者。較此老呆之心胸何如。}。這個壽字就保不定。要一死了。人說。三寸氣在千般用。一旦無常萬事休。這個大家私白白的撂下。一文也拿不去\footnote{更達。妙。}。我常聽見人說。一個陰隲十年壽。我若救活了萬數多人的命。一百個人保我過一歲。一萬個人可不保我活一百歲了。這豈不妙\footnote{念頭雖貪。以天理人事論之。亦雅當然。}。想定了主意。欣欣自得。他又算計道。不要冒失。且再算算着。扯大帶小。一個人一日半升米。一萬多人一日要五六十擔米。如今是十月起。到明年四月盡。纔接得上新麥。那時就好了。方可歇得。這七個月。一個月用一千五六百擔。毛毛要一萬一二千擔米。我家不知可有這些。不要弄得有頭沒尾。就沒趣了。因叫了個管事的家人童可用來\footnote{諺云。有了銅。救了窮。這名字甚合拍。}。道。你把各礲房堆的稻子帳查了來我看。算算共有多少。童可用把帳取來一算。道。這幾年南鄕江北各莊上收的稻子吃不着。總沒有動。約有三萬多擔。他聽了一算。三萬多擔做得一萬五六千擔米。心中大喜。道。夠了夠了。又想道。這事不要對奶奶說。倘或他一時捨不得。可不把我這場好心打脫了。如今且瞞着他。過後他不知道就罷。要知道了再說不遲。捨了出去的難道還要得回來麼。自己贊道。我這個想頭眞正妙極。忽又算計道。這萬把人得多大地方纔存得住。在那裡煮飯與他們吃。這倒是件難事。想了半日。總想不出個道路來。他道。一人不如二人智。去請了鍾兄同宦家二位哥來。再約了鄔〔合。〕大家來商量個妙法。叫家人備下酒飯。又叫人去請他衆人。不一時。都來了。大家坐下。看那童自大滿面喜色\footnote{喜色。妙。所謂誠心喜捨。不是屈意沽名。才是大英雄手段。}。笑嘻嘻的。都疑他有什麼喜事。鍾生先問道。兄今日喜氣洋洋。府上有甚喜事麼。他笑道。沒有喜事。倒有一件破財的事。故請衆位來。大家商議。衆人道。有什麼破財的事。但請見敎。他遂把看見這些難民無食。意思要獨力養活他們。因沒這個大地方。想不出主意來。故請衆位來計較。二者我家沒多人。還要借二位哥的管家相幫照看。衆人聽見他有這番好事。都贊揚道。賢弟有這一番盛舉。眞是莫大陰功。我們共勷善事。宦蕚道。賢弟旣捨飯食。我蓋幾百間大蓆篷與他們安身。人人都是沒有衣服的。我再捨萬把件棉襖與他們救寒。賈文物道。我雖不能如長兄賢弟這樣巨富。也還薄有家私。柴是我認。醃小菜鹽醬是我出。鄔兄我供他家柴米盤費。托他在那裡照管。只是沒這地方。倒是難事。鄔合道。晚生愚見。萬不得已。借各寺廟分開賑濟罷。童自大道。我也想來。人太多了。一座寺能容多少。廟中分得七零八落。那裡有這些人手照看。做着日裡吃飯罷了。夜間叫他們何處存身。鍾生見他三人如此仗義。各有所任。思量了一會。便道。弟自棄官歸來。從未足至公門。干謁當道。今三兄旣有此美舉。弟也說不得了。明早到魏國公府內去求。暫借敎場中空地搭棚賑粥。以活衆人。以朝廷之地救朝廷之民也。未必就爲不可。他如今理管京營。不得不先去求他。他若不肯。再往各上臺處去講。雖是弟破了戒。此乃公事。非爲私情。也還無妨。衆人大喜。道。妙極。事不宜遲。明日兄就去。倘說明白了。我們明日就要動手的。童自大吩咐拿酒肴來。衆人有此高興。都心中甚喜。說說笑笑的共飮。正飮之間。童自大道。哎呀。幾乎忘了。叫了童可用來。道。你到各礲房。叫他們連夜做米。陸續送來。不可遲誤。童可用答應去了。却說這新任應天府府尹。姓樂名爲善。係原任北京禮部侍郞。向日與輔臣楊嗣昌不合。吿病回去。崇禎素常知他是個好官。因與宰相參差。只得放了他去。此時楊嗣昌以閣部督師在外。征討流寇。他畏賊如虎。探聽得賊在數百里之外。他便引兵趨避。任賊攻城屠殺。他只袖手傍觀。每日在營中叫軍士們搓繩子。云預備綑賊。衆人無不匿笑。張獻忠攻破了幾座城池。殺害了幾位親王。楊嗣昌畏避。總不敢領兵去救援。又恐陷藩伏法。只得在軍中自盡了。崇禎見楊嗣昌已死。又聞知南京荒歉時。起用了他。以侍郞衛管府尹事。他到任纔數日。見了這些流民。傷心慘目。要想救濟。因人多了。不能遍及。就自己一人捐俸。諒不濟事。到任未幾。又不知這些衆官中誰人可以同爲善事。要勸地方上財主共助。這是強不得人的。必定要樂心行善者纔可勸。他想不出個妙策來。偶然想起。道。我的門生鍾情。他是本京人。必定知道這城中可有好善者。除非請了他來商議。況他那樣敢做敢爲的豪傑。胸襟自別有個主見。但我到任數日。他竟不來見我。這也古怪。或者他不在城中住。也不可知。因叫了一個衙役來。問道。有一個致仕回來的刑部員外姓鍾。你們可知道這人在那裡。衙役道。不知可是上本參論太監。壞官回來的鍾老爺。樂公道。正是他。衙役道。這是闔城聞名的。小的知道。樂公道。你問禮房拿我個侍生門帖去請了他來。說我立等要會。那衙役應諾而出。少頃。同了禮房書辦進來。稟道。這鍾老爺做人孤介得很。他終日閉門在家。從不肯到各衙門當道拜往。人去拜他。他往往推病不出。前任慕老爺也曾去拜過請過。他都推辭有病不會。也竟不來會拜。只差人拿帖來謝罪。說病軀不能出門。慕老爺雖久慕他。始終竟不曾會着。如今老爺差人去請他。大約也是不來的\footnote{有此書辦一稟。方見鍾生之高。閉門靜坐。絕口不言當道事也。故樂公到任數日。彼但知其姓而不問其名。若鑽頭見縫。訪聞新府尹姓名。忙忙求見。則是鑽熱竈門之濫鄕紳行事。大非鍾麗生之本色矣。}。樂公笑道。只管叫衙役去請。你看他來不來。那書辦不敢多言。將帖子付與衙役去了。鍾生正在童家吃酒。忽見家人忙忙拿了個名帖來遞上。道。新任府尹樂老爺差衙役到家中。立請老爺去會。小的領了他來的。鍾生接帖一看。見是樂爲善。又驚又喜。道。原來樂老師補了本處京兆。我竟不知。因對他衆人道。這樂府尹是弟會場座師。爲人極忠直仁慈。他吿病回去久了。昨日雖聞得小价們說新府尹姓樂。況他是侍郞。如何改調府尹。決想不到是他\footnote{有此一句。所以更不知其名也。}。弟因從不問當道的事。所以竟不知他的名字。竟不曾去拜見。他今來請。自然要去。又道。人有善願。天必從之。一絲不謬。適見三兄發了這一段菩提心。今遇着樂老師在此。弟去懇求他。轉說借敎場。他萬無不肯之理。豈不強如我求別人。衆人聽說。也是歡喜。鍾生忙叫人去買了個大紅全束(柬)來\footnote{妙。此物是童家所無者。}。寫了。別了衆人。便坐轎到了府尹衙門。先煩巡捕官將門生帖投進。裡面就差人出來請鍾生進到後堂。樂公見了。一把手拉住。笑道。賢契閉門養高。連我也不來會一會。鍾生挪正了座兒。請他坐了拜見。樂公那裡肯。鍾生只得作了揖。跪下。道。門生叩遲。萬望海涵。樂公扶住。道。賢契快些請起。鍾生道。門生向蒙老師培植之恩。毫無仰報。禮當一叩。再者門生被放歸來。惟閉戶在家。所以老師榮任到此。門生竟不知道。叩遲。又當謝罪。樂公道。賢契高尚。我學生盡知了。苦苦拉住。鍾生只得立起作揖。師生坐了。彼此說了許多想慕的話。樂公道。向年我學生吿病回家之後。後來聞得賢契上諫監軍一本。恨那時我已還鄕。我若在朝。寧捨此一官一身。決不肯使賢契抱屈放歸。鍾生遜謝道。蒙老恩師過愛。門生一片愚忱。恨不能挽回聖心爲愧耳。樂公道。賢契雖失此一官。直聲動朝野。無不慕其忠義。羨其膽勇。爲榮多矣。鍾生又謙遜了幾句。復道。老恩師今日憲臨此地。不但門生得覲慈顏。欣喜若狂。古所謂。一路福星。這些閭閻小民皆得蒙恩庇了。樂公慘然道。我學生不才。本心終老林泉。不意荷蒙聖恩。改授此職。連日來見這些流來難民。竟無一策可救。赧愧之甚。眞令我寢食不安。今日屈賢契到敝署來。一者久別。要想一會。以伸積愫。二來仰仗賢契高明。爲我籌一良策耳。\endnotemark[3]鍾生正要求他要轉借地方。聽了這話。滿心暗喜。答道。老恩師這一鍾(種)愛民盛心。百姓聞知。定當感泣。老恩師不須過慮。門生與舍親輩俱有成議了。遂將童自大捐米。宦蕚搭篷捨衣。賈文物助柴助菜。這三人俱是門生先好友而後親戚。只因無地方可爲。正在商議要將敎場暫借數月。門生正擬破戒到魏國公府中去懇求。尚不知他允與不允。今幸老師駕臨。望祈鼎言。或易於爲力。樂公大喜。道。賢契一時之英傑。貴親友定非凡品。他諸兄這一番爲國爲民的盛舉。眞令我輩汗顏。借敎場這一件事。我力任之。\endnotemark[4]鍾生深深一恭。道。老恩師愛民盛心。門生輩亦感激不盡。但這些窮民都凍餓久了。皆將就木的時候。還要求老恩師以速爲妙。樂公道。賢契輩倒如此熱腸。我學生上蒙聖主之恩。下有地方之責。忝爲民之父母。可還有稽緩之理。本欲留賢契一飯。容日奉請罷。我此刻就去拜魏國公。若說明了。明日就可舉事。鍾生大喜。就起身辭別出來。仍到童家。把上項事說了。衆人道。旣如此。必定就有回信。我們大家坐坐等一等佳音。又洗盞更酌。不多時。門上人進來說道。府尹差了個書房來見鍾老爺。忙叫把酒肴撤開。然後叫那書辦進來。鍾生讓他坐。他再三謙讓不敢。鍾生道。你我都是鄕里。況你又是我老師差來的。敬其主以及其使。坐了好說話。他方把座兒挪在下邊坐了。說道。適纔本官到魏國公處。把衆位老爺的盛舉說了。徐老爺也甚是歡喜。道只管蓋棚賑粥。特遣在下來奉復。還說或有不週。他還約這些勳爵老爺們捐俸幫助。鍾生道。煩兄回去多多致謝老師的鼎力。等我們諸事停妥了。同來叩謝。若再會徐公。承他借地。就是盛情了。一應事務都是他三位力行。捐俸一節。不必費他盛心。那書辦辭了去了。鍾生道。事已明白。不必坐了。大家都去行事。就是明日起手。早行一刻。窮民早沾一刻之福。三位兄行此好事。弟無可爲助。我今晚寫數百張報帖。明日黎明遣小价四處張貼。知會衆人齊到敎場。盡我之窮心而已。他三人道。非兄借地。這一段好事還做不成。論起來。吾兄之功還在我們之上。鍾生道。那是樂老師與徐公之美意。與我何涉。衆人道。非兄鼎言。徐樂二公何以及此。大家散了回去。天地間的事。只要有了錢財。何事不可爲。宦蕚回去對他父親說了。宦公也甚歡喜。他次早一面差人去買布疋棉花。雇了幾百裁縫來做棉襖。一面雇了許多扎彩匠。買了許多毛竹杉篙蘆蓆蔴繩。運到敎場。人衆物齊。眞是不日成之。賈文物的鹽醬小菜也運到。童自大各礲房的米。也有人挑的。也有驢馱的。陸續送到。又連買帶借數百口鍋水缸並桶杓粗碗竹筯之類。無不齊備。就搭起竈來。他三家約來了有三四十人。同鄔合前來照看。這些窮民聞得此信。都扶老攜幼。歡呼踴躍。蜂擁而來。

\begin{quotation}

他一個個形容枯槁。盡鳩形鵠面之人。衣敝履穿。俱鰥寡孤獨之輩。老翁攜帶幼子。喘吁吁難向前行。餓夫挽着病妻。氣奄奄不能趨步。婦女歡而男子喜。弱者後而強者先。言語喧嘩。盡喊大恩人救苦救難救餘生。頌聲盈耳。齊祝衆施主多福多壽多男子。

\end{quotation}

那難民中有些精壯的。就去幫着挑水燒火煮飯。鄔合看着每人散了一個碗。一雙筯。賈文物又買了幾千束草來。鋪在篷內地上。與他們睡覺。不幾日。宦蕚擡了棉襖來。每人散了一件。這些人將凍餓要死的時候。忽然有吃又有穿。而且有住處。這個感恩誦德。更何用說。就是闔城的人。也無有一個不誇念他們的好處。一日。那童自大忽然尋思道。我的行事。可是人說的。茅山的靈官。照遠不照近。外路來的難民固然該賑濟。難道本鄕本土鰥寡孤獨那些窮人。是該餓着他的。在十三門。不論城裡城外。揀了十三座寬濶寺廟。就托本寺廟當家的和尚道士。每日早晚。煮兩挑米的粥。與這些無依無靠的人吃。每一處一月米六十擔。柴六十挑。並小菜之類。都送了去。也煩鍾生寫了許多報子。各處貼了。他衆人這好事。直到次年四月盡。新麥上來。天氣暖了。這些人也有回鄕的。也有去傭工的。大家纔散了。這樂府尹着實敬愛他四人。都親自拜望請酒。時常往來。不必多敍。那童自大又送了鍾生一百擔米。鍾生先不肯受。他再三不依。方領了。又分惠了梅生三十擔。郝氏十擔。童自大把這些窮親戚。十擔五擔不等。都送了些。人人感激。一日。他偶然在門口站着。只見一個乞丐跪倒。哀求施捨。童自大正要問他來歷。忽見鍾生同宦蕚鄔合到來。忙迎着拱手。鍾生一眼看見那花子。嘆說道。這樣一個精壯少年。何事不可爲。爲何走了這條道路。童自大道。正是呢。弟方纔正要問他緣故。因二兄駕到。未及細問。鄔合道。此人晚生知道。他父親叫做卜通。做了半世先生。\endnotemark[5]不但誤人子弟。又且行止不端。此人叫做卜之仕。又癡又俊(儍)。好吃懶做。雖然是他自己不成人。也緣他亡父的遺孽。大家嘆息了幾聲。童自大叫家人取了幾十文錢打發那花子去了。你道卜之仕他隨娘嫁了楊大。如何流落做了乞丐。那水氏自嫁楊大之後。夜夜不肯放空。那楊大雖然是強壯之年。當日母上司偶然降臨。還可竭力奉承。如今成了夫婦。日間辛苦擡轎。夜裡當了差。勞碌催科撫字一齊督併起來。如何支撐得住。先還勉強應酬。後來漸漸不能支應。竟掛冠而逃。只說外邊有事。躱在碼頭上公房中去睡。水氏明知其故。不勝痛恨。也曾變下臉來同他鬧過幾番。說道。我是沒飯吃。嫁到你家來吃飯的麼。還是圖你的什麼好門第。嫁你做什麼來。我整夜孤眠獨宿。守了活寡。何不我當日守了死寡。還有個名節。你也自己摸摸良心。可過得去。楊大明知道他是因此道發怒。但自己是擡轎的人。別無進益。一日不擡。便沒米下鍋。先娶水氏來。是他收生着娃娃。生意盛行。所得之物除吃用之外。還有餘積。故此那時可擡可不擡。靠着老婆吃飯。少不得在被窩中要打勤勞。近來水氏因向日人都稱他卜奶奶。而今知他嫁了個轎夫。都改稱他楊姥姥。或稱楊大嫂。他不服氣。也不出去做生意了。他旣賭氣不出門。只靠楊大擡轎度日。日間費力。夜裡又要費力。如何支持得來。要去勉強應酬。自己性命要緊。況當初水氏色量尚未大開。自己儘力。也還可以供他個飽足。自從經過又粗又長之後。楊大已考在三等。把滿身精力使盡。要想拔置前列。亦不能夠。如何有這些力量去對付他。只得裝呆做癡而已。把唐代宗不癡不聾。不做阿家翁兩句金言。做了護身符。且當日未娶他時。偶然一偷。如同獲了尤物。旣娶了來。終日如此。其味不過如此而已。未娶他時。同他偷偷。以爲他是多情不過的婦人。及到了此時。又以他是個淫濫不堪的賤婦\footnote{說盡人情。}。索性躱在碼頭上不回。水氏雖氣恨脹滿胸膛。却也無法可處。忽然一日。有一個姓竹的來請他收生。水氏自己出去回道。我久不做這事了。你另請別人罷。那人道。我知道奶奶不出門。但我家同奶奶還有些瓜葛。我家女人胎死在肚裡。不知別人的手段好歹。不敢去請。奶奶是久聞名的。故此我母親打發來請。水氏道。我同你家有什麼瓜葛。你姓什麼。那人道。我姓竹。叫做竹美。我母親姓郝。當日原在錢家。我家大姑娘如今嫁在鍾老爺家的。就是我母親親生的女兒。是當日過世卜先生的學生。我母親是後嫁我父親的。水氏猛然想起。當年卜通在日。曾做笑話。常說他有個女學生錢貴。他娘相與了個姓竹的。混名叫做賽敖曹。陽物其大無比。後來沒有婦人禁得他的。惟獨這郝氏受得。大約就是他了。又問一句道。你父親可是混名叫賽敖曹的麼。竹美笑道。這是人混說的頑話。奶奶怎得知道。水氏沈吟了一會。想道。我家這沒良心的忘八。絕情絕義。他旣不顧我。我也另走走道路。這賽敖曹的名雖說得怕人。我自己量着我的也還不小。我去看做緣法。或者得嘗嘗是個什麼味兒。也不可知。遂笑吟吟的道。我本是不出門的。旣說起來是親\footnote{此時還算不得親。今夜同竹思寬如此。過明日才是親呢。}。只得去走走。竹美見他肯去。十分歡喜。他進去把頭抿了抿。又把下身洗了洗\footnote{替人家去收生。先自己洗了下身。趣。}。腰裡拽了塊舊紬帕出來\footnote{眞老在行。}。同竹美走着。問道。請我替誰收生。竹美道。就是我的女人。他懷着胎有七八個月了。這幾日總也不動。肚子只往下墜着疼。我母親遂想起奶奶來。故此着我來請。〔不〕多時。到了竹家。進去同郝氏廝見了。水氏看那財香面如蠟紙。愁眉苦臉。水氏摸了摸。急忙下手舞弄了半日。直到更闌。方纔取下。扶財香上了床。水氏洗了手。穿了衣服。郝氏要照看財香。對竹思寬道。楊奶奶是好親戚。勞動了半日半夜。乏倦了。夜深回不去。你就陪在東屋裡坐坐。快看酒飯。竹思美(寬)在堂屋裡答應着。就請水氏到東屋裡去。到了房中。桌椅擺設停當。竹美點上大燭。竹思寬讓他坐下。竹美就送酒菜來。竹思寬忙斟了一杯。奉與水氏。道。着實有勞奶奶。請用一杯。解解辛苦。水氏接過。彼此對飮。竹思寬道。多蒙奶奶盛情肯來。我一家感激不盡。容後報答罷。水氏道。我久不出門了。因方纔你家大官兒說起鍾府上大姑娘來。他原是我前夫的學生。都是瓜葛。我纔來了。竹思寬道。我也知道奶奶不出門。是我老伴兒說你只說得明白。楊奶奶是極有情的人。再沒有個不來的。果然奶奶肯下降。沒有好東西款待。奶奶用一杯薄酒。水氏吃了幾杯。合了楊妃醉酒曲子上的兩句。道。酒興兒高。色興兒漸漸起。想起他那大物來。人雖覿面。不得相親。領敎領敎。淫心一動。兩隻眼餳蹬蹬的不轉眼看着竹思寬。竹思寬是油裡的泥鰍。滑極了的老慣家。心中就猜了幾分。遂笑說道。奶奶當日在卜府上。卜先生是有名的人。配了奶奶。也還不錯。近來嫁到楊家。未免屈了奶奶些。奶奶這樣個人兒。夫人還做不過麼。楊老大有福。怎麼就得了奶奶。水氏緋紅了臉。含愧笑道。也因孩子小。沒人支當門戶。誤聽人言。嫁到了他家。水氏觸動了心事。恨了一聲。道。誰知是恁個沒良心的人。竹思寬接口道。難道他還不遂心麼。眞是得福不覺。要是個好人。得了奶奶這樣有情有義的人。不知怎樣疼愛呢。說着話。又讓水氏吃了幾杯。水氏這幾杯落肚。有些忍耐不住了。先勾一句道。我當日聽見先夫說。人稱竹大爺的大名叫做賽敖曹。是怎麼說。竹思寬已明白他來相就了。又見他有了幾杯。眉目間騷態畢露。也就大膽笑說道。奶奶不要見笑。我的這根賤具。實在要算個放樣的。故此人起我這個混名。可惜他沒福。空有這樣個出奇的物件。沒有遇過妙人。要得遇着奶奶\endnotemark[6]這樣做(佳)人兒。也不枉生他一場。當日長在楊老大身上。他可不就造化了。那水氏靠着椅子背。摀着嘴嘻嘻的笑。竹思寬再讓酒。水氏不吃了。叫竹美拿飯來。竹思寬道。請你母親來陪楊奶奶吃飯。竹美道。母親辛苦了幾日。剛纔打發媳婦上床。他老人家在火廂裡睡着了。竹思寬陪水氏吃了飯。茶嗽了口。又坐了一會。說道。不堪的床鋪。奶奶請歇息歇息罷。我老伴兒又乏睏睡着了。笑道。我要來奉陪。又恐奶奶不稀罕。水氏也笑道。主人陪客。也是理當。竹思寬道。先道了謝罷。笑嘻嘻帶上門出去。在堂屋中支了個鋪睡。水氏吹了燈上床。來(乘)着酒興。脫了個精光睡下。想道。他方纔的口氣。夜裡定然進來。心中胡思亂想。翻來覆去。一點睏意俱無。聽見竹思寬問道。竹美。你睡了麼。竹美答道。睡了。就不見做聲。又聽得輕輕推得門響。心中喜道。來了來了。我假裝睡着。等他上床。省多少客氣。倒仰面假睡。兩腿大開。只見竹思寬爬上床來。輕輕揭開被。摸着他赤身仰臥。爬上身。摸着了門。拿他那如驢之物。就想要往裡頂。水氏此時文章已做到後股。少不得要收尾。故做驚醒。假意去摀陰門。却是要去摸摸他有多大一個。攥着他的龜頭。一把握不過來。心下也吃一驚。道。果然不謬。問道。你做什麼。竹思寬低聲笑道。我來陪奶奶了。水氏道。我好意來替你家救人。你倒這樣。快些下去。竹思寬笑着將陽物亂頂。道。我倒也罷了。奶奶可憐他那樣急。賞他嘗嘗吧。水氏再要做作。被他戳得心口手三樣都軟了。做作不來。說道。你這樣大東西。是弄得進去的麼。竹思寬道。奶奶你放手。包你不妨。水氏將手一鬆。竹思寬搽上許多唾沫。然後再弄。頂了幾下。龜頭進得去。龜稜不得入。水氏淫心火熾。也顧不得了。用手摸了摸自己。吐些唾。將龜稜四週搽了。兩手把陰門〖扌扉〗得開開的。道。你用力頂一下看。竹思寬狠狠一頂。水氏哎喲了一聲。摸時已入。雖然狠了一下。尚不至十分痛苦。水氏陰中先已水出。此時越多。滑溜無礙。漸漸送入。水氏覺得頂到心口之下深處。甚疼。拿手在腹外按時。像條硬棍在裡邊挺着。再摸他陽物時。只剩二卵在外。心中固喜。但有些痛。說道。我深處痛得很。你拔出些來。看搗通了心口。竹思寬笑道。不妨事。難道穿胸國的人不過日子麼。竹思寬也就拔出寸餘。抽了一會。興發如狂。顧不得他了。一送到根。儘力搗起來。水氏雖然內中甚痛。見他高興。不好攔阻。只得任他衝突。往外一拔。扯得快活。便哼了一聲。向裡一頂。到了疼處。便哎喲一聲。竹思寬覺得緊箍箍。又下〔下〕頂着軟肉。心中甚樂。聽得他陰戶中刮答刮答的響。更覺興豪。弄了多時。外面已時三鼓。方纔完了。水氏雖丢了數次。却也疼了幾千疼。只算得苦樂相伴。不能全美如意。二人歇息了一會。水氏捏弄着他的陽物。說道。可恨太大。再短個寸把略細些。就是個寶貝了。竹思寬笑着將指頭探到他的牝中。道。何不說你的再略深些寬些呢。二人笑了一會。水氏道。你生平除了你家奶奶。可還遇過對子麼。竹思寬道。當日還有一個姓昌的禁得。第三個就要算你了。水氏道。我算不得十分對手。只好算七分罷了。竹思寬笑道。怎麼說你的水屄竟不如昌屄了\footnote{他雖不如昌屄。却還强似火屄。}。水氏笑着擰了他兩下。說道。你出去罷。恐一時睡着了。有人看見。不好意思。竹思寬道。主人陪客。也怕人麼。說着。也就笑着摸出去了。水氏也乏捲(倦)了。睡到日出起來。摸摸陰門。腫雖消了些。內中反抽着小肚子疼。少刻。郝氏過來。道了許多勞動簡慢。稱謝不已。水氏剛梳洗完。就看上飯來。郝氏陪着勸了幾杯酒。吃畢了飯。水氏要回去。郝氏用盤子捧出二兩一封謝資。兩頂縐紗包頭。兩條大花布手巾。一塊草紙。水氏只收了草紙\footnote{這是江南收生婆的規矩。}。餘者再三不收。郝氏只管儘讓。水氏只得又收了二條手巾。郝氏甚不過意。水氏回家。養息了一日。下身纔好些。次早飯後。竹思寬押着一架食盒。送了十二色水禮。一罎酒。親自送來道謝。水氏笑道。一個至親家\footnote{至親二字。此時用得當了。}。又多這個心做什麼。竹思寬見左右沒人。笑道。前日勞動。我家沒甚管待你。倒反擾你的美物。今日送這幾色不堪的禮。將就遮遮羞罷。水氏瞅了他一眼。笑着收了。拿了幾十文錢。打發擡盒人去了。說道。你請坐。我去燒茶來你吃。竹思寬一把摟住。道。不敢勞動。捧過臉來親了個嘴。道。吃點甜唾沫當了茶罷。水氏笑着伸過舌頭。咂了一會。水氏道。我借花獻佛。燙壺酒來請你。竹思寬接(摟)他在懷。就伸手到褲中摸着牝戶。道。上嘴當茶。下嘴當了酒罷。水氏道。還當酒呢。昨日疼了一日。今日纔得好些。這個主人做不得。竹思寬道。前夜是初弄。今日旣好了。便沒事。不要辜負了我的來意。水氏也覺好些。便有些高興。說道。等我關了門來\footnote{辱翁曰。是歷練過的了。}。你到屋裡床上去。他家是兩間小房。外邊一半做客位。一半做廚房。給卜之仕睡。裡間做臥房。房後堆破爛東西\footnote{他的房子。前卜之仕向多銀說張見他娘同楊大之妻。已知是兩間。却未如此之詳。今復一詳寫者。爲楊大殺水氏時點明。楊大進廚房取刀。李四得空先跑。楊大奔進裡間時。張三將水氏一掀。亦奪路跑去也。}。水氏關了門進來。竹思寬已上床脫光。水氏一眼看見他陽物豎在那裡。上前一把攥住。吐舌道。哎呀。好像個小人國的和尚一般。前日夜裡弄着還罷了。怎這樣怕人子難看。虧我這裡頭怎竟容下了。竹思寬拉他上床。也脫光睡倒。用手將他兩腿推得開開的。低頭一看。好個肥物件。牝戶大張。也笑着說道。前日夜裡弄着還不覺。怎這樣大張着個鬍子嘴難看。水氏笑着打了他一下。道。都是你撐的。還說呢。竹思寬對上了。往裡送了兩送。水氏連聲道。疼呢。使不得。使不得。還着些唾沫潤潤。竹思寬道。就是弄女孩子。也只頭一回用些。那裡有只管用的。又往裡頭〔送。〕水氏道。你不用。讓我用。你千萬不要狠深了。留些在外頭。裡面疼得受不得。把唾沫用上許多。又吐了一大朶。摜在陰門內\footnote{這一個摜字。陰門之大可知。}。竹思寬笑着把兩腿揸開坐下。將水氏兩足放在兩傍。把他屁股抱過來挨着肚子。然後纔頂了入去。過(送)進了一個龜頭。往外一拔。瓜答一聲響。又一進。又一出。又響一聲。不住的如此。水氏見他屁股一進。忙將屁股往上一迎。他又拔出。總不深入。水氏急得說道。你這叫做什麼頑法。竹思寬道。又說弄進去疼。水氏道。不過叫你留些在外頭。難道只叫你弄進這一點子去麼。竹思寬也不理他。抽着。且聽那響聲。看那一出一進之勢。龜頭大了。將他陰門塞緊。並無一〖阝少日小〗。往裡一頂。連那兩片長心子都帶了進去。向外一拔。那長心子吐了出來。如兩個蝙蝠翅一般翻覆有趣。水氏淫興大動。忍耐不得了。哀求道。好親親。不要弄急我了。快些弄弄罷。竹思寬道。我也巴不得要弄。怕你疼呢。水氏罵道。刻薄鬼。我知道你是要全弄進去。說不得我忍着些。憑你弄罷。竹思寬道。你旣知道。就好講了。起來一伏上身。此時陰中淫液津津外流。已滑溜至極。幾送到根。竹思寬也興濃了。這一上手。就抽了有千數。把個水氏弄得張嘴瞪目。只聽得鼻孔中哼哼\endnotemark[7]的響。弄了多時。水氏將他的腰一把摟緊。道。罷了我了。我的哥哥。讓我逼逼氣。竹思寬也就慢了些。過了一會。重新哼起來道。哎喲。我被你弄死了。抽得氣出不來還罷了。裡頭像鎗戳的一般難受。你拔出些來。我歇歇着。竹思寬也依他拔出了些。淺淺慢送。抽了一會。興又復起。一攮盡根。大抽起來。水氏道。哎喲。受不得了。你淺着些。竹思寬一陣亂搗。搗得那水氏口裡祖宗親爹亂叫。竹思寬見那樣子。心中樂極。也就洩了。歹(又)往內頂了幾下。水氏哎了幾聲。然後他抽出來。水氏揉着肚子。哼哼道。腸子斷了。腸子斷了。竹思寬一面穿着衣裳。笑道。你當眞還疼麼。我當是你哄我的。水氏笑道。活強盜。哄你呢。再要一會。實在要斷了。竹思寬道。好時候了。怕有人來。我去罷。你不必起來。多謝你。改日再來望你。水氏道。你空去了。回去多謝奶奶。竹思寬笑道。我倒沒有空。此時你那裡頭倒空了。說着笑出去。開了門。反帶上去了。水氏疼得起不來。拉過被來蓋着。哼聲不絕。這晚。楊大恰好回來走走。見了這些食物。問水氏是那裡的。水氏沈着臉。也不答他。他自覺沒趣。到廚下同卜之仕煮肉煎魚蒸饅頭熱酒。收拾停當。拿進來讓水氏吃。水氏也不答應。讓之再三。水氏道。我不吃。你們吃去。楊大同卜之仕拿到外邊來享用。楊大悄悄問卜之仕是什麼人送的。卜之仕道。我沒在家。不知道。前日有個人來請媽收生。昨日纔回來。想是那家送來的謝禮。楊大聽得水氏又出門做生意。又有好日子過了。心中暗喜。那知他是出去尋野食吃。楊大吃了個半酣。思量道。他旣肯出去。這日子不愁過了。趁今日同他溫溫。後來好回來受用。晚間捱着不去。要同水氏睡。水氏要是美(每)常。也就笑納了。此時被竹思寬弄得疼得要死。同他睡。可阻得他不弄。說道。我不要你。你到大房裡去睡。楊大陪着笑臉。要挨上床。水氏推推搡搡。決意不依。楊大以爲嫌恨他。故不肯同臥。也氣狠狠的去了。水氏過了三四日纔好些。一日。暗想道。老竹的那東西眞算得一件奇物。可惜我不濟。不是對手。要像這樣弄一會痛一會。不是取樂。竟是尋苦吃了。已嘗過這個辣味。再也不敢招惹他了。我家這忘八心已死透。他不戀我。我還戀他怎麼。還是去尋那張三李四來。一來他們是同類。就時常往來。街上人看着不扠眼。他都是窮漢。我給他弄了。再破着我的私房恩養着他。他再沒有不盡力報答我的。豈不強似塡壞了這沒良心的忘八。但不知他兩個可有老婆沒有。又想道。他就有老婆。也未必強似我。豈有不愛我的。主意拿定。一日。楊大擡應考的秀才往句容去了。水氏叫卜之仕去碼頭上約了他二人來。支了卜之仕出去。水氏已預備下酒肴。搬出來相待他兩個。他二人見水氏約來共飮。知他是要續前情。說道。向日承奶奶美情。我兩個睡夢中都是感激的。又蒙奶奶賞戒指。我們時刻帶在身邊。見了就感念不盡。楊大哥是有福的人。奶奶就嫁了他。我們雖然知道奶奶嫁到這裡。不敢走來親近。今日蒙奶奶叫了來。這是我兄弟兩個的造化到了。李四道。楊大哥有福不爲(會)享。怎麼奶奶在家。他倒躱了出去睡。要是我得了奶奶。拿棍攆我。一夜也捨不得離的。可憐我弟兄兩個。巴一個醜老婆做伴兒也不能夠。何況像奶奶這樣的容貌\footnote{謬獎。}忍心離開。水氏聽他說尚沒妻子。心中暗喜。張三接口道。你我那裡有這樣的福。想得奶奶這等標致老婆。若是奶奶不忘舊情。容我們時常來親近服事。就是造化了。水氏三杯落肚。淫興方濃。笑說道。我當日原愛你兩個。只因同他相與久了。遂嫁了他。誰知這忘八沒良心。早知。嫁了你兩個。何等不好。如今悔也遲了。他兩人道。也不妨事。此後但是楊大哥不在家。得空就來服事奶奶。也不遲。張三向着李四道。我們不要貪嘴。躭誤了奶奶的正經事。水氏笑道。你兩個在這裡怎麼樣的。李四道。三哥。我們還是論年紀。你大似我。你先服事奶奶。我去關門。李四關了門進來。見他二人脫得精光。就在椅子上幹呢。李四也忙脫了。就看他們弄了一會。張三道。老四。讓你罷。李四正等得冒火。陽物脹得如鐵杵一般。忙上前揷了進去。儘平生之力一陣亂搗。水氏不住叫道。好心肝。好弄。不要輕了。就是這樣的。李四一口氣搗了有數百。水氏口中先還聲喚。後來只鼻孔中哼。下邊淫水漰湃之聲震耳。張三看上興來。說道。該讓讓我了。李四也力乏。拔了出來。張三連忙接着就弄。因見水氏先誇李四。他便腰中趲勁。往內直攮。那管撞腫了陰門。搗通了底子。這水氏快活非常。說道。好哥哥。不枉人自叫做鐵棒槌。二人輪流弄了多時。水氏興也足了。二人也洩了。方纔穿衣別去。他二人時常來看水氏。會無又(不)吃。吃無不弄。也來往了多半年。這兩個精壯漢子弄得水氏雖南面王樂也不過如此。他年雖半百。騷淫比少年更甚。交媾一次。他那淫液眞合了他的姓。陰中騷水長流。如到了閘口一般。一日。水氏正同張三弄着。李四在傍候缺。看了一會。陽物脹硬得受不得。向水氏道。奶奶。蒙你這樣大恩。我們是感激不盡的了。但是一個弄一個等。實在有些忍不得。你請看看我脹得這樣靑筋暴湛。眼子裡涎長淌。急得要死。若蒙奶奶再擡舉。我們一個在前面服事。賞我在後面服事。嘗嘗奶奶的寶貝屁股。眞要我死也肯。要我的心肝煮湯吃。我要打個蹬兒。忘八也不如。水氏正弄得快活。閉着眼哼。聽他說得苦惱。睜眼一看。果然陽物脹得多粗。又憐又愛。向張三道。你下來側楞着弄。讓他後頭來。張三就下來側臥弄上了。李四欣喜非常。吐上一朶唾沫。搽在糞門上。就往裡頂。水氏忙道。你慢慢的來。一句話還沒有說完。被他冒冒失失狠命的一下。已將送到了根。水氏哎喲了幾聲。道。這也比得前頭麼。叫你慢些。還這樣冒失。不怕搗斷我腸子麼。李四笑道。我一時急了。粗鹵了些。奶奶不要見怪。纔慢慢抽了一會。見水氏不嘖\endnotemark[8]聲。知已相安。又奮力衝突。水氏被他前後夾攻。弄得哼成一塊。弄了一會。又二人轉換。弄了多時。方纔興止。水氏自有了這二男妾。竟把楊大似有如無。相待甚是情淡\footnote{寵妾棄妻。原太薄情。}。楊大間或回來走走。水氏面上像刮得下霜來一般。惡言惡語相侵。並無一句好話。楊大賭氣也不歸家。心中也疑他有了外遇。又常見張三李四不在碼頭上。心裡就猜了幾分。暗暗留心打聽。世上事可有瞞得人的。這些原委他都知道了。他一個鹵夫。不想當日自己如何淫人妻子來。今見水氏偷漢。他便怒道。這淫婦當日瞞了漢子偷我。今日又瞞着我偷人。若撞到我手中。叫他白刀子進去。紅刀子出來。定然雙雙殺了。方洩我恨。他便留心伺察。一日。冷眼見張三李四往他家裡去。他便隨後尾了來。他三人正在房中取樂。不防楊大回來。見門關着。輕輕掇下。走了進去。向窗洞內張時\footnote{此窗初次卜之仕張他。二次李四張他。這一次是他自己張。便張出大禍來了。}。三個都精光。張三坐在椅子上。將水氏抱在懷中。陽物自後揷入後庭之內坐住。李四將水氏兩腿夾在肋下。對面抽弄。前一推後一囊(攮)的樂。聽那水氏顫着聲道。好哥哥。我要快活死了。我恨當初瞎了眼。嫁了這懶忘八。要早知他是這樣。我嫁了你兩個。豈不是下半世快樂。那楊大不由得怒氣騰騰。惡向膽生。推門進來。跑到廚下去尋切菜刀。那李四正弄着。猛聽得脚步響。忙拔出。往外一看。見楊大一臉凶氣。顧不得穿衣。往外飛跑。楊大見他跑了。奔進來殺這兩個。張三見他來得勢凶。自己性命要緊。那裡還顧得水氏。將水氏搊着光屁股往楊大身上一掀。楊大劈面一刀斫着。張三就這空裡。將楊大夾臉一掌。一個眼花。他也趁空跑了。楊大拿刀趕出時。二人已不知去向。進來看那水氏。頭顱臉鼻劈做兩半\footnote{這眞是快活死了。}。已死了。楊大正收拾水氏的細軟私囊。要想逃走。不想卜之仕回來。見娘精光的殺在血泊裡。嚇得之仕跑到街上大叫道。不好了。我爹把我媽殺了。衆鄰舍聽見殺人的事。都攢將攏來到他屋內。楊大知走不成。只得將三人衣褲並行凶刀拿着。同到縣中自首。將詳細稟了官。知縣差典史帶忤作相驗了。雖然衣褲有據。不曾殺得奸夫。難以開釋。責十板收了監。擬了一個監候絞。把張三李四拿來。和奸只杖。以二男朋奸一婦。行同獸類。且因奸而斃二命。凡奸加一等。杖八十。徒三年。申詳上司。聽候發落。水氏屍骸發前夫之子領埋。定了案。那楊大在監中。但合眼就見水氏赤滌滌血淋淋的向他索命。如狂如癡。混喊亂叫。不多幾日。申文還未下來。早已嗚呼。過(忤)作拖了(出)牢洞。一個背夫偷漢。一個淫人凶殺。皆不得其死。足見這淫之一字。可不深戒哉。卜之仕將他娘買棺埋葬了。水氏當日私蓄原不多。後來又不出門做生意。每日用度只有去無來。半年多買酒買肉供給兩個奸夫。也就罄了。楊大一個轎夫。有何積蓄。房子是租的。所剩不過幾件衣服傢伙而已。卜之仕百無一能。賣一件吃一件。坐食山崩。不久而流爲乞丐。再說鍾生宦蕚鄔合在童自大家閒談了一會。備酒飯款待了。抵暮方散。別了各自歸家。鍾生到了上房坐下。恰値兩個兒子鍾文鍾武放了學。上來作揖。鍾生偶有所觸。向錢貴道。人家兒子不可不叫他各習一技。讀書不成。急尋別路。庶可將來糊口。若因因循循。豈不誤了後輩。我今日見一個壯年乞丐。說起來。他父親名叫卜通。做了半世先生。不能訓子。一旦至此。豈不可嘆。錢貴聽了。顏色愀然。鍾生道。賢妻此是何故。錢貴嘆道。此人乃我先生之子也。我當日蒙先生訓詁。今彼子流離至此。於心何安。故不覺戚戚然耳。鍾生見他不忘舊師。着人尋了卜之仕來。不友不僕。養活了他數年。一日。病較(絞)腸沙而死。此係卜通夫婦子女的結果。不復再敍。再說童自大那日無事。在大門口站着閒望。只見一個和尚走到跟前。打了一個問訊。道。借問一聲。這裡有一位大善人童財主。可是此處。童自大仔細打一看時。好\endnotemark[9]一個和尚。只見他。

\begin{quotation}

雙眉劍掃。兩眼波橫。腰跨戒刀。足穿芒履。身帔(披)七幅布偏衫。手住九環錫禪杖。雖非圓寂光中客。定是空門異樣僧。

\end{quotation}

他龐眉大目。隆準豐頤。就像個泥塑的羅漢。挑着一個衣包。袖衣僧帽蔴履腿綳。像是個遠路來的行脚僧。童自大道。我就是童財主。却不是甚大善人\footnote{人行大善。而自不以爲善。方是眞善。未有些須之善。而洋洋以善人自居者。則小人哉。何善之有。}。那僧人笑道。救了成萬人之性命。不是大善人是甚麼。童自大道。那也算不得什麼善人。師傅。你尋我有什麼話說。那僧人道。貧僧是河南少林寺來的。敝處連年飢荒。又遭流寇之難。今歲五月間。有千餘流賊想來擄掠敝寺。被我合寺僧行一陣連耞棍。盡行打死。只剩得數十人逃去。餘賊知道了。雖不敢到敝寺來。把左近一帶人口屠戮。糧食作踐一空。我敝寺中僧人甚多。日食皆無。因前聽見有鄕親們回去說道這裡有一位姓童的大善人。捨幾萬〔擔〕米。現救這萬餘人性命。\endnotemark[10]貧僧知是一位大知識大施主。故不遠千里。特來募化。結一個善緣。童自大道。旣然如此。且請進去。此時正是臘月初間。天氣甚寒。讓他到書房內圍爐坐下。問他道。師傅。你要化我些什麼。那僧人道。敝寺有五百餘衆僧行。齋糧不繼。日只一食。要求老施主施一二百擔糧與衆僧度命。過此纔(殘)冬。童自大道。糧倒有。齋僧布施也是好事。只是你怎麼拿了去。那僧人道。施主若肯大發慈悲。貧僧再往別處募化水脚銀兩。雇船運去。到了敝省。那就好處了。童自大道。衆人餓着肚子等飯吃。還要等你東化西化。知道等到那一日纔化到手。那僧人道。貧僧巴不得此刻就回。如何得能夠。童自大道。一客不煩兩主。我旣要做好事。一個人情就做到底。是人說的。頭都磕了。又捨不得一個揖。我如今送你五百擔米。一百兩銀。全美了你罷。省得又到別處去化。你如今拿這銀子雇船裝了去。可好麼。那僧人忙立起問訊謝道。怎敢望施主布施這許多。貧僧來意指望化百十擔就是大緣了。童自大道。人的俗話說。齋僧不飽。不如活埋。你寺里人多。那一點子夠做這什麼。你可有來的伙伴麼。你一個人怎麼照料去。你這個水路打那裡去。那僧人道。雇船從長江入汴河直到汴梁。那到寺便不遠了。再雇車運了去。童自大道。這好。這好。因問道。師傅。你吃了飯沒有。要沒吃。吃了飯去。僧人道。若蒙施主見賜。貧僧就拜領。但蒙厚賜多了。何敢叨擾。童自大道。一餐便飯。何必作謙。因笑道。我素常聽見傳說。你少林寺的師傅都吃葷酒。你可用麼。那僧人也笑道。貧僧葷蔬不拘。也不戒酒。但隨施主之便。童自大吩咐家人拿飯來。他如今不像當日待鄔合的一塊冷豆腐幾片臭鹹魚的那個局面。雖不十分豐盛。也就拿了四碗菜來。牽葷搭蔬。魚肉。瓢兒菜。豆腐之類\footnote{先待鄔合時寫臭醃魚冷豆腐者。笑其吝鄙也。今寫此四品者。謂彼雖不吝。不肯過於奢侈者。正謂惜福之故。非笑其仍臭也。}。又叫取了一大壺酒來。他陪着和尚吃。那僧人也不做謙。酒拿起一口一鍾。不一時。壺酒一罄。四碗菜也都吃了個八分。童自大見他不足興。又叫取了一壺酒來。吩咐家人道。我看這師傅的食量好。這幾碗菜不濟事。你快去街上買兩隻板鴨。一隻金漆鵝。他河南人愛吃麪食。把大饅頭買幾十個來。家人如飛而去。頃刻即來。童自大叫快拿了切去。那僧人笑道。旣蒙施主盛心。就是這樣放着貧僧領罷。童自大道。好好。這樣倒也托契。叫拿盤子裝了。放在和尚跟前。他道了一聲多擾。腰間拔出戒刀。一面割做大臠。酒肉點心一齊大嚼。不多時。如風捲殘雲。吃了個乾乾淨淨。童自大都看癡呆了。暗道。這和尚不知餓了多少日子。就吃上這些東西。只見那和尚吃罷。把刀擦了揷上。揩了手。笑着道。多謝施主。貧僧今日却得了一飽。童自大道。師傅。你不要怪我說。你就吃上這些。不怕穿壞了肚子麼。他笑道。貧僧食量頗雄。這纔算得一個半飽。如何得穿着。童自大吃驚道。這纔算半飽。若要大飽。得多少吃。家人收拾器皿。童自大命他叫了童可用來。道。你到當鋪裡要一百兩銀子來。替掌櫃的朝奉說。有當死了的綿直裰。查一件來送這師傅穿。我看他有些冷。那僧人道。敢蒙施主如此錯愛。小僧無可答報。惟有在諸佛菩薩座前。保佑施主發財發福。多子多孫罷。童自大道。我也不求財了\footnote{他人雖呆。但開口便是知足語。宜有大福。}。我只得一個兒子。再求生得一個。也就罷了。我但要圖多活幾年。就是造化了\footnote{人皆有此奢望。不獨他爲然。}。那僧人道。施主這樣積德。況且又是便家。多娶幾個姬妾。自然子嗣就多了。童自大道。不瞞師傅說。我的力量也有限。就有婦女。也沒本事去打發他\footnote{亦是知足語。}。是人說的笑話。不要爲了一個子。先送掉了八乂子呢\footnote{多少明人未悟。而此呆翁悟之。}。那僧人道。貧僧當日到處雲遊。曾在陝西遇見一個異人。是個羽士。傳了我一種異術。他再三囑咐。不可輕傳匪人。罪過不小。貧僧出家人用不着。我見過多少人。沒一個至誠君子。不敢妄傳。今遇施主這樣盛德。我奉傳了。不但多子。且可延壽。童自大聽了。喜笑道。好師傅。是什麼方法。你可吿訴我。那僧人道。施主可知道從來有採戰種子的兩個法了麼。童自大搖着頭道。我活了三十多歲。從沒有聽見這裡新聞。採戰兩個字。不知是甚麼話。至於兒子。是兩口子誤打誤撞遇巧弄出來的。拿個甚麼種去種。這話荒唐。因大笑道。我倒聽見人說㞠子是人種。難道切碎了塞在婦人那裡頭去種麼\footnote{好悟性。}。和尚道。施主不知。\endnotemark[11]等貧僧一件件分解與你聽。古人這兩個方法是分做兩途的。採戰就不能種子。種子就不能採戰。我的這個法則是可相並行的。所以說是異術。方爲至寶。童自大道。你一樣樣說與我聽聽。怎麼叫做採戰。僧人道。男女交媾。男人的陽精就是身上的腦髓。人的頭顱謂之髓海。臨洩時。精時(由)髓海而下走。夾脊至尾閭至腎而出。所以通身快暢。若作喪得多了。腦枯髓竭。所以人就身弱致病。久而久之。如油乾燈滅。命便喪了。若會了採戰。不但自己的陽精不洩出去。反把婦人的陰精採了。吸在自己的身中來補養髓血。坎離旣濟。那身子自然一日一日的強壯起來。身強髓滿。自然就延壽了。所以叫做採戰。童自大道。這個法兒果然好。我倒聽見人說。和尚偷老婆。不說不歇不洩。想就是會採戰了。那和尚笑道。這是人貶罵僧家的話。那裡個個都會採戰。童自大道。我不知道。得罪得罪。你再說怎麼叫做種子。僧人道。婦人不懷孕。或是子宮冷。或是男子的精冷。我有一種藥方。男女皆服。經行之次一交合。便可得子。男人的精脈壯而暖。就是種了。童自大聽得津津有味。笑吟吟的道。你先說採戰不能種子。種子不能採戰。是怎麼說。我到底不明白這話。僧人道。種子是要自己的陽精洩了出去。採戰是要把陰精吸了過來。當日人有採戰的法。只能採過來。不能吐出去。若是把持不住。忽然一走。不但前功盡棄。還要喪命。所以說採戰不能種子。生子不能採戰。我這個法是要採就採。要種就種。旣可保養身子。却病延年。又可多得子嗣。所以不肯輕授匪人。童自大道。這樣說起來。這個法兒果然奇妙。但你先說這事有大罪過。這是人說你們的。做和尚的人偷老婆。自然有大罪過。像我們在家人是家家幹。個個幹。有甚麼罪。要肏屄有罪。連人種都要絕了。那僧人笑道。罪過不是說男女交媾的話。種子不妨。施主不知這採戰利害多着哩。男人的陽物十分大了。死夯也沒用。十分小了。又不濟事。要酌乎中。方纔伶範。這一採起來。那婦人快活到心窩裡去。吸出來的陰精也是他的腦髓。男人的快活。週身通泰。比洩出時更樂。採戰的婦人。二十歲以外。三十四五歲以內的方可。那老的小的都用不得。小的精血未足。老的精血已衰。多致成疾。大捐陰功。就是中年婦人。瘦怯的還行不得。要胖胖壯壯無病的方可。若採過一次。要好好的將養七日。纔得復原。過了七日。又纔採得。若次數\endnotemark[12]多了。要身子虛弱。成癆病死的。就不死。也再不能生子。因他的精血枯了。我說不敢妄傳匪人者。恐他混逞淫毒。縱意亂弄起來。傷了婦人性命。這豈不是我傳法的大罪過麼。所說罪過。就是這個緣故。但這個法。除非像府上這樣富足。纔行得來。若是窮漢守着一個妻子。可幹得這事麼。須得有十數個婢妾。纔可功(供)得過來。這裡頭還有一個不損陰德的妙法。婦女們二十來歲尋了來。十年之內若生了子。就不用說了。那無子的。到三十歲上。就與他一夫一妻嫁了去。再換少年的。這個更沒罪過。童自大道。師傅說了這半日。我只好聽聽罷了。是做不來的。僧人道。這是爲何。童自大道。一來我的奶奶未必肯容我娶小\footnote{懼內者世不乏人。然而無不自悔。童自大逢人便自陳。人則謂之愚呆。我却取其誠實。}。二來我的這件匪物不堪之甚。你方纔說要酌乎中。我的這東西雖算不得六等考下下。是要算五等考下中的。如何做得。和尚笑道\footnote{此笑笑其自言懼內也。}。若恐夫人不容。這就沒法\footnote{此笑之故。}。若說陽具太小。只算得五等。我自然會叫他大起來。超拔到二等上\footnote{不意此僧能操文宗之歡。}。不然何以叫做術。童自大聽了。歡喜非常。道。旣然有這妙法。師傅傳了我。我重謝你。我若學會了。再慢慢的去求奶奶。師傅。這也要學多少日子。那和尚道。也得三七工夫。纔可完成。童自大道。二十一日也不爲多。只是你怎麼等得。和尚道。若施主果要傳此。貧僧同來了五衆。着師兄們先去。我在此傳授了。以報施主盛情。然後再回。童自大喜極。此時銀子衣服都拿來好一會了。童自大交付與他。便道。我也不留師傅了。你同我這家人去到礲房內。兌五百擔米的票子與你。你隨早隨晚打發他們起了身。到我家來住着罷。那僧人打問訊謝。童自大送他出門。和尚又謝。童自大也叮囑他快來。僧人同着童可用去了。過了兩日。童自大正眼巴巴盼那和尚。忽家人來說。前日那和尚來了。童自大歡喜的忙忙出來接着。吩咐家人快備酒飯。知道他食量好。都是䠙蹄肥肉。大鵝壯雞。點心米飯。又是幾大壺玉蘭陳酒。儘他受用了一飽。然後問他道。師傅要用什麼東西。你只管說。那和尚要筆硯。童自大忙叫人在門口當鋪裡取了來\footnote{文人拿着一支筆胡寫亂畫。不知作了多少孽。他這樣財主家連筆硯都沒有。宜乎應享大福。}。開了一個藥單。童自大道。這件事我家人不在行。索性煩師傅去買罷。叫人取了五十兩銀子來。和尚笑道。何須用許多。十分之一足矣。拿了兩錠有五六兩。起身出去了一會。買了許多藥來。晚間。童自大也出來同他在書房中睡。到臨臥的時候。和尚取出一丸藥來。叫童自大用無灰陳酒服下。然後睡覺。過了一會。童自大覺得渾身骨縫中都滾熱得受用。下邊的陽物也熱腸(脹)得快活\footnote{與賈文物服藥時似同而不同。}。睡了一夜。次早。和尚叫煎了藥草水。叫他薰洗陽物。搓扯個不歇。有一個時辰纔止。又叫他用鹽滾湯服了一丸藥。每日早晚如此者七日。看那陽物〈具〉時。渾身靑筋暴綻。色若羊肝。一個龜頭些(紫)威威亮錚錚。形如染的雞子。約有七寸來長。一虎零一指粗細。童自大拿手捏着。左看右看。越看越得意。笑個不住。和尚又到街上將前次打的一把小銀刀取了來。到第八日早起。就不吃藥了。替他用麻藥把龜頭搽上。過了一會。掐着都不知疼。用手心揉着。揉了多時。散了血脈。然後用銀刀將馬口大大的割開。趕忙用靈藥敷上。用絹帕包好。先童自大還有些膽怯。到後來。見割得不但不痛。連血都沒有。他纔放了心。僧人見他陽具已成。然後將採戰的法傳他。如何採吸。如何運動。如何吐洩。童自大生平極蠢。此時竟聰明起來。就能領略。和尚無事之時。修合種子丸藥。又過了七日。叫童自大將陽物打開看時。那刀瘡蓋兒也掉了。那馬口就像一張小嘴一般\footnote{這才是櫻桃小口。}。叫他運氣試試。竟會一張一閉。把個童自大喜得倒在一張涼床上打着滾大笑。和尚道。施主的大功已成八九了。還要學熟方妥。第一是吸來的陰精要會運動行到週身纔妙。不然有一阻滯。恐生癰毒。爲害非小。又盡心敎了七日。童自大也虛心領敎。全然盡得其妙。又把修合的種子丸藥付與他。道。可依方服之。自有效驗。倘若婦人種了子。懷了孕。萬不可再採。不但墜了胎。還恐傷了孕婦。定要等生產百日之後。然後纔可採得的。緊\endnotemark[13]記緊記。又叫取了半斤燒酒來。和尚叫他吸了看。倒在一個碗內。童自大取出陽物。一吸一吸。頃刻而盡。和尚道。施主法已學成。你雖是盛德人。不用我多囑。切記着萬不可傷損婦人。你尋小時。若要女兒。定是二十歲上下的方可。若是少年寡婦。十七八歲也還不妨。七日之限萬不可少。倘若有十分肥壯婦人。年少身強。那樣好鼎器。五日也還可。你原是要圖益壽生子。若縱慾傷人起來。反要損壽夭子了。萬萬留心\footnote{和尚可謂叮嚀吿戒。}。他滿口應允。那和尚要作辭回去。童自大忙道。好師傅\footnote{師傅上加一好字。感之至也。}。離年不幾日。你也趕不到家。何苦在路上過年。你過了元旦去罷。和尚道。貧僧足力頗健。一日可行三百餘里。此處到敝寺不過千餘里。不用到除夕便到了家了。前日衆師兄去。我也要趕了去看看。多擾施主了。童自大見留不住。叫人捧出替他做的一身新綿衣服。一百兩盤纏。和尚道。蒙施主厚賜多了。綿衣貧僧拜領。銀子決不敢受\footnote{而今世上那裡去尋這不愛銀子的和尚。}。童自大再三強着他。道。師傅。承你這樣好情。應該謝你的。況且是我雞巴的恩人。越發該謝。定要求你收。那和尚見他話雖說得可笑。却倒是一片實心。便道。施主旣然這般諄諄下愛。貧僧夠盤纏回去就罷了。遂伸手取過一封打開。拿了數兩。別的定不肯要\footnote{前賈文物送道士百金而不受。今童自大道(送)和尚百金。先不受而後稍受。雖是遙遙一對。却是兩樣。仍係對而不對。妙。}。童自大甚不過意。忙叫備酒飯。家人掇了上來。和尚吃罷。起身作別。將送他的衣服裝入囊中。收拾完了。挑上肩頭。道了數聲多謝而去。童自大滿臉笑容。走進臥房。鐵氏正在那裡向火吃酒。見了。問道。你這些日子。每晚在外邊過夜。做些甚事。我聽見有一個會吃酒肉的和尚\footnote{此話奇。那一個和尚不吃酒肉。}。在這裡住着。你要與他做徒弟麼。你如今爲何這樣歡喜。童自大也不答應。只有嘻嘻的笑。鐵氏也好笑起來。道。你不像瘋了。問着話不說。只管笑甚麼。你想是吃了笑菌子了。童自大笑着道。我一些也不瘋。奶奶。晚上怕你要瘋呢。鐵氏道。我看你有些古怪。不要是當眞瘋了罷。那葵心蓮瓣看見主人公的光景有些可疑。釘釘的望着他。只見童自大笑着把衣服摟起。褲子扯開。把陽具取出來。像八蠻獻寶似的一手托着。向鐵氏道。奶奶。你看看這個寶貝。你可要喜歡瘋了麼。鐵氏定睛一看。失驚道。怎麼腫成這麼個樣子了。他道。你道是腫麼。到晚上試試看。鐵氏又見那馬口不同往日。用手〖扌扉〗開一看。見大張着。笑道。這是怎的了。好砛黃子。童自大道。說不盡這好處。等晚上試念(驗)了。再慢慢的吿訴你。鐵氏也歡喜得了不得。不忍釋手。捏着細賞玩了一會。若不因天氣冷。大約也等不到晚上了。也就不再問。同着他吃酒。那葵心蓮瓣看了這個稀奇物件。要近前細細的賞鑒一番。又疑(礙)着主母在跟前。料道今夜輪不到他嘗這新物。恨不得一口咬了下來。拿去取樂\footnote{咬下來就樂不成了。}。心中又喜又急。看看天晚。吃了晚飯。鐵氏等不得了。就上床脫衣去睡。童自大也要試新。趕忙也上來。將鐵氏兩腿分開。弄將進去。鐵氏也還不覺其妙。童自大運用起來。那馬口張開。在內中東咬一下。西啃一下。咬得他陰中癢癢酥酥。快活難當。只是格格的笑。咬了多時。那鐵氏摟得他緊緊的。笑道。我的裡頭要癢死了。鼻子內哼聲不絕。牙齒咬得格支支的響。童自大見了他這個樣子。更覺高興。然後一下咬住內中花心。如小孩咂乳一般。一陣咂。把那鐵氏樂得要死。渾身肥肉亂抖。就像發瘧疾寒戰的樣勢。連喉中聲氣也顫篤酥的。牙齒鬪得亂響。不多時。此(止)見他打了兩個寒噤。喉嚨格格響了兩聲。就身子動也不動。聲也不嘖。竟像癱化了。童自大覺得一股熱氣自尾閭穴直買(冒)天庭。樂不可言。方知這個妙發(法)果然奇妙。這鐵氏嫁了丈夫多年。何嘗經過這一番樂境。雖有他粗而且長的角先生。那是個死物。不過只塞滿了。挨皮擦肉。出進多番。也覺快活。今日遇着這大而且活的東西。怎不叫他受用得要死。鐵氏酥軟了好一會。醒過來。道。我從來沒有這樣受用過。裡頭的那個樂處。說不出來的那種妙法。渾身竟像打骨縫裡頭去了些東西一樣。遍身都鬆散了。這是誰敎你的這個好方兒。童自大把和尚傳的方法。並婦人要七日一輪。多則生病。這法還可以種子。若多買些婢妾。可以延壽。都對他說了\footnote{只有胖壯婦人五日也可以行得這一句。瞞了不曾說。}。鐵氏笑道。旣如此說。你買小老婆就討一百個我也不管。只要你有本事去做。只做定了例子。但是七日你就來同我弄一回。你若再有本事。在我肚裡種出個兒子來。就是十日我也等得\footnote{世間婦人未有不巴兒子者。看此憶起一事。也可謂之笑談。余友胡致還娶妻曾氏。將二十年。總不生育。曾氏常响(向)人道。我也不望長命百歲的兒子。只求養下一個會叫一聲媽媽。死了我也甘心。不枉我做婦人一生。}。童自大聽了他這話。喜不可言\footnote{開恩許買妾。自然該喜。}。次夜。又同他二位如夫人去試了一試。把一朶葵花心幾乎咬碎。把兩片蓮花瓣險些咂開。樂得他兩人次日還咧着大嘴。笑個不住。童自大雖學會了這件妙術。幾乎弄出一場大禍。若不虧樂府尹是個正人君子。縱不至於破產亡身。也要損一股大財。這是甚麼緣故。童自大賑濟流民的時候。內中有一個難民姓劉名弘。爲人奸狡百出。負義忘恩\footnote{這八個字是病症。世人犯者甚多。}。却生得漢仗魁梧。口舌便利。因他到處無情。以怨報德。受了人的恩會(惠)眨眼便忘。還是小事。有下石處。就想害那恩人。因此人人切齒。爲鄕黨所不容。人見他害人不曾害得。到處害了自己。衆人起了他一個混名。叫做劉大儍。他在蓆篷中吃了幾日飽飯。穿上了宦公子捨的綿衣。飽暖了又想高飛。他心中自商道。我的壞名。鄕人皆知。將來就是回去。也無安身之地。這童百萬是南京第一個富翁了。我何不投在他家看風使。或者還有個出產。定了主意。到了童家來求見了童自大。再四哀求道。小人已是將凍餓死的人了。蒙老爺活命之恩。無以爲報。如今不願還鄕。情願投在老爺府上。做個家奴。稍效犬馬之勞。雖赴湯陷(蹈)火。亦不敢辭。稍〔報〕天恩萬一。童自大是個誠實的人。見他說得如此懇切。也就留下。替他徹底做了一身衣服帽履之類。他終日小心慇懃。眞是一個滾盤珠。活動至極。童自大客(家)中的人。全是些算盤珠。撥撥動動的。從不曾見過這等活說人。心中着實相愛。一日。向他說道。我看你身材也好。又小心又勤謹。你在我家有甚麼出路。我改日看巧有好地方。舉薦了你去想一個出身。劉弘忙叩頭道。這是老爺天恩。若蒙老爺提拔。小心(人)得有寸進。粉骨碎身也不能報大恩了。童自大記在心裡。一日。樂府尹請鍾生同宦賈童四人小敍。〈說〉劉弘也跟了去。說話之間。童自大見樂公相待殷殷。甚是情厚。見劉弘在傍邊。忽然想起他的事來。童自大向樂公道。晚生有一事奉稟。樂公和顏悅色的道。有甚麼話。但請見敎。童自大因叫過劉弘來與樂公叩頭。說道。此人名叫劉弘。也是山東難民。他情願到晚生舍下來復(服)役。晚生見他小心慇懃。做事又能幹。晚生一個庶民人家。恐誤了他。意思要送到老爺府中。求大老爺收留使用。若果然慇懃妥當。求大老爺提拔他。就是大老爺的天恩了。不但他感恩。就是晚生也感恩不盡。樂公道。兄旣如此說。我豈不領命。明日叫他來。我留用就是了。童自大作揖道謝。抵暮回家。童自大取了五兩銀子與劉弘。道。你在我家這些時。也沒有甚麼給你的。你一到樂老爺府中。那裡就有錢使。這個你帶去盤纏。你到衙門裡。凡事要小心。不要說他府裡的幕賓事事要週到。就是到府中的管家也要圓活。禁不得衆人一歡喜。向主人一說你。就是造化了。劉弘叩首道。老爺恩典。敎導小人。小人敢不遵依。小人若稍有好處。必圖後報。童自大道。我也不圖你的報。但你投奔我一場。舉薦你的個好處。我就完了一番心事。次日。又親送到樂公署中。樂公收下。劉弘果然活泛至極。無處不週到。樂公有一個幕賓。是江西人。姓李名舞。樂公與他賓主甚是相投。眞是言聽計從。這李相公也善伺樂公之意。他見樂公常誇童自大的好處。說他一個貨値(殖)中人。竟有此大英雄手段。救濟若許流民。況宦賈二位。還是他鼓舞起來的豪舉。李相公也極力稱揚。贊不絕口。這劉弘見李相公是樂公心腹。要圖得他的歡心。強拿強做小獻勤是不消說得。他身邊有童自大給他的幾兩銀子。時常買些新鮮果品上樣細點來孝敬。誰知這李相公腹雖甚通。性極愛小。受他些小惠。喜愛他了不得。他府中還有一個大管家。姓鄭。幼眇一目。人順口都叫他鄭割(瞎)子。他做事伶透。也是樂公得用的人。劉弘見他〈他〉在樂公跟前說得話。諸事要仰仗他。遂買了一口豬。一罎酒。拜認他做娘舅。劉弘謅說他母親也是姓鄭。那鄭管家也是甚喜。時常叫他到家中吃酒吃飯。李鄭二人屢次在樂公面前說他的好處。樂公雖是信他兩個的話。多因童自大面上。也格外擡舉他。他站在高枝上了。過了些時。就拿出那中山狼的心腸來了。想道。童百萬算南京有名的財主了。放着這樣的肥主兒。何不在他身上想他一個道路。我如今下一個毒計。同李相公鄭舅舅商議。慫恿老爺拿他一個囮頭。弄他一主大大的錢出來。奉承他衆位。不但他們歡喜我。我至少也得一個小富。可以快樂下半世。不然不(替)人家踢門檻到那一日。他想定了主意。欣欣自得。向鄭瞎子說道。我蒙老爺擡舉。舅舅照看。無恩可報。如今有一〈個〉主財是樂得吃的。手到就可擒來。若弄到了手。老爺何止得一二十萬。就是舅舅。三五萬也是容易的。鄭瞎子大驚大喜。道。是那裡有這樣的好事。若果然老爺得了這大財。難道是好白了你麼。你且說是甚麼事。是誰家。劉弘道。就是童百萬家。他近日養着個妖僧在家裡。說是河南來的。藏在書房中傳法。每日不知做些甚麼。近來流賊四處搶劫。他的黨羽散在各處。做奸細的甚多。舅舅稟了老爺。只用把這妖僧拿來。做他是流賊差來的。約童百萬裡應外合。要想攻陷南京。就是他養活些流民。也是要圖謀不軌。這一個罪名他的性命還保不住。何況家財。他要想保得無事。三頭幾十萬銀子。怕他不拿出來麼。這事須開通了李相公同做方可。如今只算得三十萬。老爺得二十萬。那十萬舅舅同李相公分用。諒李相公再沒有不在老爺面前盡力幫襯的。他不強似做幾百年的主文相公麼。至於我。聽憑老爺舅舅尊意。多寡給我些就罷。便不給我也罷。我原不報銀子\footnote{眞廉。是滿心想坑人害人弄錢。却違心滿口說淸廉話。何世上此輩之多也。}。不過是我報老爺\footnote{眞義。}。同舅舅\footnote{眞賢甥。}。李相公的恩。叫做個借花獻佛。鄭瞎子被他說得心熱如火。忙同他去向李相公計較。李舞聽得可分數萬金。心中那喜裡那還說得出來。暗想道。牡丹亭傳奇中陳最良道。要腰纏十萬。除非是敎學千年。方纔貫滿。我辛苦做幕。背井離鄕。抛妻撇子。在此不但終日忙忙碌碌。還要伺東君顏色。只得二百四十金一年。此一舉得五萬。做二十多年的幕纔掙得來。何過(樂)不爲。得此回家。也就算榮歸了。做一個大富翁。何等受用。他的這計策雖毒。就明知是假。何妨弄假成眞。況他百萬財翁。便拿出三十萬來。只損了三分之一。在他不至於重傷。在我們便獲了大濟。遂滿口應承。道。這在我說。等老爺下來。你們大家在這裡幫襯說說。自然可成。他甥舅二人也心中暗喜。次日午間。公事畢了。樂公到書房中來。同李舞談了些公務。李舞就將童自大窩藏妖僧的話上達。樂公驚道。他果有此事。必定緊密的了。先生何以致(知)之。鄭管家在傍稟道。是劉弘向小的說的。小的因是地方上的大事。關係非小。不得不向李相公說。稟知老爺。樂公又問劉弘道。他家留了和尚。你如何就知是妖僧。劉弘還滿心希圖獻功。跪稟道。要是個好和尚。何妨明公正氣\endnotemark[14]的。他兩個成日關着書房門。在內中商議。不與人知道。不是想謀反是做甚麼。小的雖在他家一場。受過些須恩惠。今日蒙老爺天恩擡舉。事情重大。關係着老爺。老爺有地方的責任。小的穿靑衣抱黑柱。故不敢不說。恐負了老爺大恩。樂公大怒。不便呵叱李舞。罵鄭瞎子道。童財主做了賑救難民這番好事。我幾次要題請求個旌表。恐倒反玷了他的德行\footnote{樂公此心。誠可謂君子愛人以德。}。我敬愛他了不得。他〈你〉那種盛德人。可肯做這樣壞事。你這奴才。敢來無故陷害好人。到我跟前獻讒。喝叫家人打了一二十個嘴巴。又道。我只說你還用得。故此擡舉你。誰知也是見利忘義的壞人\footnote{見利忘義的人何止恒河沙數。焉得人人而痛撻之。}。吩咐攆下去馬房中養馬。再不許到我跟前。又罵劉弘道。你這沒良心。人面獸心的惡奴\footnote{罵得當。但恐世上不止劉弘一個。}。你也是個流民。他好意留養你。救了飢寒性命。就是他的大恩了。還恐怕誤了你。特送到我衙門裡來。托我擡舉你。也可謂恩情畢至了。你當子子孫孫感他的恩德纔是。你今日無中生有。倒反想害他的身家性命。你這惡奴心腸。不過想於中取利\footnote{眞靑天。洞鑒小人肺腑。}。你良心喪盡。禽獸不若了。我且問你。他有百萬之產。何求而尚欲爲逆。人家養外來的僧道也甚多。難道都是想通流寇的不成。你道他養流民是想謀反。你難道不是流民麼。但恐他不養流民。你也早矣凍餓而死。未必活到今日了\footnote{說得痛快。令他死而無怨。}。人說利令\endnotemark[15]智昏。就是你了。一處無恩。百處無恩。今日幸虧你自犯。不然焉知後來你不算計害我\footnote{劉弘何辭以辯。}。你誣陷良善。罪當反坐。本當立刻處死。姑念小人無知。從寬發放。傳了一個書辦進來。命行文上元縣。將劉弘重責三十板。即刻解回原籍。不計(許)時刻停留。登時去了。劉弘到縣中受了重刑。即時起解。寒冬冷月。又無盤費。走了幾日。便病故了。解差同地方報官驗過。抛於荒郊。餵了豬狗。可爲負心之報。押了劉弘去後。樂公怒猶未息。正言厲色將李相公說了幾句。道。先生是讀書君子。如何聽小人無稽之言。便欲害人謀利。我請先生來做西賓。原欲匡我之不逮。恐我諸事有差謬處。還要先生救正。今反欲陷我於不義。大非我延請先生之意了。李舞面紅耳赤。無言可答。此時恨無地洞可入。樂公氣忿忿的上去了。李舞自覺無顏。次日。辭辭樂公。試探其意。樂公也不留。將脩金送出。只得回去了。五萬銀子不見一分。掃了一鼻子灰。反討一場大沒趣。眞是。

\begin{quotation}

羊肉不吃得。空惹一身羶。

\end{quotation}

那鄭瞎子貪了些豬酒小惠。認了這一個好外甥。被他一陣說話得利慾熏心。獻了一篇讒。一文不得。弄做了一名馬頭軍。悔之無及。把那一隻眼也氣成了靑盲。越沒用了。樂公此事並不曾向童自大題起。後來童自大屢次到樂公署中。總不見劉弘。暗暗詢問他家人。那人將前事詳細說與。童自大心下大駭。感激樂公不盡。樂公病終之後。童自大因此厚贈贐儀。就是報他這件恩德。後來便見。再說童自大同妻妾都試過了些時。已是歲底。忙過了年。到上元節後。他着人把七老八少的媒婆叫了十數個來到書房中。拿果碟與他們吃酒。他陪着吃。衆媒婆道。老爺叫了我們來。有何吩咐。童自大道。我們請你們來替我尋小。衆媒人道。這是容易的事。憑老爺要多少都有。他道。我有個難題目呢。我有個三不要。衆媒婆道。怎麼叫做三不要。童自大道。我尋小。十四五歲的女孩子我不要。只要好小寡婦。這叫做一不要。就是小寡婦。或是瘦弱。或是暗疾的。我也不要。要那生得厚厚實實。胖胖壯壯。乾乾淨淨的。這叫做二不要。我只要二十二三以裡。十八九歲以外。十分老少我又不要。這叫做三不要。衆媒婆不解其意。都笑起來。道。別的也罷了。人巴不得要眞女兒。老爺爲何倒說不要。童自大笑嘻嘻的道。不瞞你衆位說。我的這東西雖不叫做十分大。却是個活的。那小女孩子禁不得。所以要小寡婦。那是破了的倒好。衆人聽了。都不好做聲。內中一個老媒婆。他奇(倚)老賣老。笑着道。誰人的㞠子不是活的。難道這東西也會死麼。況且活人身上的物件。怎麼得死。我就不懂得這話。童自大道。你們不知道。我這東西比不得別人的。連酒都會吃。要酒量小些的人。還吃他不過呢。所以說是活的。衆人聽說。只道是打趣他們吃酒。都笑起來。道。蒙老爺賞酒。我們領了幾鍾。就把我們比做老爺的那東西了。我們當是好話要的。還側着耳朶聽呢。童自大道。我說的是正經話。你們當說謊麼。叫了個家人來。道。你拿個碗去取半斤燒酒來。我試與你們看。不多時。取了一碗酒來。童自大叫他出去。衆媒婆不知其意。看他做甚麼事。他笑着道。你們不要見笑。我獻醜了。摟起衣服。扯開褲子。把陽物取出來。放在酒碗中。有幾個少年的媒婆羞得臉緋紅。背過身子去。幾個年老些的正要看這稀奇故事。看他怎個吃法。見見世面。都眼睛睜得多大。看着那金漆桌子腿一般的物件大張着馬口。果然一吸一吸。頃刻吃了半碗。都拍手打掌。哈哈大笑。道。這個作怪的東西。都實實不曾見過。怪不得老爺說是活的。會吃酒。眞乃好大量大根。小菜也不用。一氣就吃了半碗。那幾個少年的聽見這話。也顧不得了。都擠到跟前來。目不轉睛的看。見他張着嘴。一開一閉。不一時。把那碗酒全吃完了。有一調駐云飛贊他的厥物。道。

\begin{quotation}

此物蹺蹊。蓋世寰中少見之。口大非爲異。妙在能張閉。咦還有便(更)稀奇。酒吞滿觶。被底綢繆。自有別滋味。怎不敎。少婦魂消魄也飛。

\end{quotation}

童自大笑道。你們看見了。有這個緣故。所以我不要你小女孩子。他把褲子拽上。這些婦人眼睛裡的火都看得爆了出來。那兩個老媒婆道。實不相瞞。我們少年時走走邪路。那長的短的粗的細的也見過些。像老爺這個活的。會吃酒。不要說沒見過。連聽也沒有聽過。我倒聽見人說慈悲庵有個大姑子。原是個鄕紳的小奶奶出家的。他會吸男人的精髓。憑你甚麼精壯小夥子到他身上。幾吸便完帳。便請下馬。我想那還罷了。婦人下身的那張口原是會吃男人的。大約不過他的利害些。老爺這東西這樣個小嘴也會這樣〔吃〕酒。明日不知便宜那些有造化的小媳婦享用呢。幾〈緣〉個年紀小的媒婆見了這又大又活的罕物。好生動得火。嘴中說不出。心裡騷極了。這個把那個擰一下。道。你去試試看是個甚麼味道。那個把這個往童自大跟前一推。道。你急了就去試試罷了。又擰我怎的。嘻嘻哈哈笑做一團。滾做一堆。衆人心中都巴不得同他試驗試驗。嘗嘗這個異味。因彼此人多相礙。不好意思。臉上火噴噴一般。心中好不發急。他們一個個。

\begin{quotation}

上面口中嚥了好些唾沫入去。

下邊嘴內流出許多淸水出來。

\end{quotation}

衆媒婆大家起身。道。多謝老爺賞酒。我們打聽着了。再來回老爺的信。辭謝而去。衆媒婆替他傳揚。人人皆知童百萬是個絕大的活雞巴。會吃酒。這些小寡婦。就是他公婆父母不肯把他與人做妾。他聽見了這話。一心情願去做他的小。嘗嘗這活物件是甚味道。俗語道。〔初嫁〕憑爺娘。再嫁由自己。他自己願意。父母也沒奈他何。童自大跟着這些媒婆各處相看了許多。只揀\endnotemark[16]了十個。他暗算道。我聽見人說金釵十二。我家中有一雙。\endnotemark[17]\footnote{這一雙恐難當金釵二字。}帶這十個。豈不是十二了。奶奶獨當一夜。他們兩人當一夜。恰恰是七日一輪。遂將六間廂房收拾得甚是華麗。製了十分首飾衣裳。並房中床帳。箱櫃桌椅。擺設的香爐花瓶。鏡臺粉盒之類。件件簇新\footnote{雖是財主氣象。總離不得一個俗字。故妙。}。娶了這十個婦人來家。每人又買了一個丫頭與他\footnote{葵心蓮瓣有了丫頭。眞是樓上樓了。}。一邊六人住着。派定兩個一班。也將西屋做了一個官鋪。這些妾接着日子輪流上來伴宿。該鐵氏的這一日。他自己過去當値。鐵氏此後比(把)那先生砸得稀爛。撂在淨桶中。棄之\endnotemark[18]於糞坑之內。雖是鐵氏得新忘故。實在那先生空自長大壯觀。腹內空空無物。抛入糞中。在臭氣內潛身。也不爲過。童自大他採戰則戰。種子則種。四五年間。生得十多個兒女。他那個樂那裡還說得出來。鐵氏雖不曾生育。這姓(些)娃娃誰敢不叫他做娘。他看見大大小小一羣在面前。好不熱鬧。也喜歡得了不得。鐵氏今雖改變。毫無凶暴之氣。但童自大素常畏威攝(懾)服慣了的。每常敬他到十分的地位。今見他這樣寬恩。先畏威而後感德。竟尊他到二十分上。這樣(些)妾見主人公猶然如此。可敢有一毫膽大怠忽之心。都恨不得把他頂在頭上奉承。鐵氏見他衆人小心侍奉。也着實疼愛。妻妾過得甚是和美。話分兩頭。後再歸一。先那媒婆說慈悲庵的姑子。會吸男人精髓。他姓甚麼。是何來歷。聽我慢慢說來。且說那萬曆末年。城中有個顯官。姓吳名友。別號歸翁。生平貪鄙不堪。家資富厚無比。古人說。貪乃無後之相。一絲不爽。他家金銀紬緞。房產地土。無一不有。眞可富賽王侯。但只缺了一件。不要說沒有兒子。連想個女兒看看也不能夠。他夫人姓杜。那生性也就奇妒不過\footnote{姓不好。怪不得他。}。自己旣無所出。又不容丈夫娶小\footnote{不姓杜的夫人不容丈夫娶小者甚多。何況他姓杜。如何容得。}。吳友想兒心腸。暗地同丫頭們做那偷摸勾當。起先那些丫頭見主人要來同他做這樣風流樂事。可有不歡喜樂從。也還巴不得生出個兒子來。將來就是副主母。豈不榮耀快樂。誰知道有一個身孕。杜氏若知道了。髠髮熏目。截指剜耳。百般的慘刑無不做出。定至於死而後已。或有竟生下子女來的。杜氏明知是丈夫的骨血。寃說丫頭不長進。何處偷來的私孩。不但將孩子弄死了。連生孩子的娘也不想活。那歸翁在傍看着。連那護庇的話也不敢說一句。聽他施爲\footnote{辱翁曰。此等人豈眞無有。漢成帝就是前輩先生。}。後來這些丫頭們看見這個光景。大約這兒子難生。副主母也做不成。且留着命多活幾年。吳友要去高興。像強奸一般。死也不依。若使威凌逼。反喊得主母知道。不但有賞。且護庇着他。那歸翁惟有暗氣暗恨而已。亦無可奈何。這杜氏少年的時候還想生育。捐資建了一座慈悲庵。內中望(供)着送子張仙神像。着了家中七八個寡婦在內侍奉香火。世間但是富貴人家。你叫他週濟貧窮親戚。照看困苦朋友。他半個破錢也決乎捨不得。到了奶奶們拿去布施和尚道士。或是修蓋庵觀廟宇。成千成百。毫不吝惜。他都肯出手。這個慈悲庵是杜氏爲求子而建。越發不惜工價。費了數千金。果然蓋造好。內中廻廊曲檻。樓閣亭臺。異卉奇花。蒼松怪石。雖地方不甚大。却也無一〔不〕備。他老夫婦也時常來瞻仰禮拜。遊玩盤恒。不想供了二十多年。毫無靈感。仍舊是他夫妻兩個。並不曾添得一丁。正經杜氏建庵求子的。不曾生育。倒是看守香火的寡僕婦。有三四個年少些的。倒生了好幾個兒子。也不知從和(何)而來\footnote{張仙送來的。又何用說。}。却又棄之。吳友五十多歲。有人勸他姪兒中選一個立嗣。他一來捨不得家資付與猶子。二來還癡想自己生兒。到了六十多歲。他夫人杜氏纔嗚呼哀哉。大吉利市\footnote{歸翁造化。}。他吃了正夫人一生的虧苦。不敢續絃。忙忙娶了一個美妾。你道這個美妾姓甚名誰。後來曾生子曾生女否。下回便知分曉。

姑妄言第十七卷終



\endnotetext[1]{「給」原作「結」,據文義改;下文或同,不贅。}

\endnotetext[2]{「他道」原作「道他」,據文義改。}

\endnotetext[3]{此句原書右有夾批「妙在樂公口內求他」八字。}

\endnotetext[4]{此句原書右有夾批「慨然」二字。}

\endnotetext[5]{此句原書右有夾批「引前卜通一張以顯報應」十字。「引」、「張」二字,或爲「映」、「段」之誤。}

\endnotetext[6]{「遇着奶奶」原作「奶奶遇着」字,據文義改。}

\endnotetext[7]{「哼哼」原作「坑坑」,據文義改。}

\endnotetext[8]{「嘖」原作「噴」,據文義改;下同,不贅。}

\endnotetext[9]{「好」字原置「時」字之上,據文義改。}

\endnotetext[10]{「性命」原作「姓名」,據文義改;下文或同,不贅。}

\endnotetext[11]{「施主不知」原置「和尚道」三字之上,據文義改。}

\endnotetext[12]{「次數」原作「數次」,據文義改。}

\endnotetext[13]{「緊」字原置「的」字之上,據文義改。}

\endnotetext[14]{「明公正氣」原作「明正公氣」,據文義改。}

\endnotetext[15]{「利令」原作「令利」,據文義改。}

\endnotetext[16]{「只揀」原置「許多」二字之上,據文義改。}

\endnotetext[17]{「雙」原作「隻」,據文義改;批註亦同,不贅。}

\endnotetext[18]{「棄之」原作「之棄」,據文義改。}

\setcounter{footnote}{0}

\theendnotes

\part*{姑妄言第十八卷}
\addcontentsline{toc}{part}{姑妄言第十八卷}
\markboth{姑妄言第十八卷}{姑妄言第十八卷}

鈍翁曰。吳老兒好尋好做兒子。不曾尋得做得。被他們的元品妙琰(炎)把命一催。反尋到別人家去做兒又(子)去了。世間此等不自量老兒。正復不少。寫崔命兒之淫。非這貪鄙老兒的尖夫人。淫不至此。此尖夫人若不做尼姑。或亦淫不至此也。一爲貪人勸戒。一爲尼姑說法。再者。他們的元牝妙眼送掉了吳友猶其次。又斷送了無限少年。生我之門死我戶。世上看得被(破)者有幾。

佞佛之人往往受淫尼姦僧之害。而不知醒悟。即或知之。孽由己作。只得隱忍。蘭(藺)馥豈非榜樣歟。此一段並非謗佛。正是勸人好眞佛。虔心信佛。信心行善。不可被說假佛者哄誘。天堂不知何處。地獄先在眼前。所謂自貽伊戚。夫復誰怨。

司進朝一妻有妾。祁辛亦一妻有妾。司進朝請富新坐館。祁辛亦請何幸坐館。兩段事極相似。却舉勸行事以至結果又毫不相似。故爲妙也。寧可爲何幸之書呆。不可效富新之狡獪。

童自大之死命兒。一寫他得壽之由。二則將命兒諸人收拾。更把慈悲奄(庵)之陷坑塡滿。又接狐精一段。何意。童自大施恩賑濟一場。活了萬餘人。內中豈無一蒙恩受德者而報之耶。故寫蒙德報信。使童自大採得丹頭爲延壽之基。又留在二十四回中出首艾鮑艾復。庶不是生扭出此人也。

寫定計出於閔爲(慧)姑甘老姐者。總是作書者不肯漏去一個。即甘壽夫妻極此(無)要緊的人。尚要與他一個結局。若單提一老姐。便覺顯然。故陪出一個閔慧姑。以瞞看者之目。便不覺得。粗心浮氣之人看至(之)。烏足知此。

寫富新纔遇崔命兒。受了多少情愛。及得了雨棠雪梅。便負了命兒。受了司進朝\endnotemark[1]多少厚德。便偷〈娶〉廳〈空〉氏。以負進朝。纔偷空氏。就托故去偷龐氏。以負空氏。到後來偷娶龐氏三人時。鞏氏三個竟不一問。又負此三人。處處負心。纔寫他名字滿足。

富新負了司進朝。便接龐氏三婦負富新。富新因負心於司進朝而死。三婦亦因負心於富新而死。借這幾個男女。罵盡負心人。有(尚)不足爲妙。又借富新之負心。罵盡明末降賊諸文武之負心者。妙極。倘有負心之人見此。當亟爲改悔。不身罹橫禍而貽後人之笑罵也。

寫黑姑子授術與崔命兒。雖是因事敍事。却完結了第一回開首之人。

\chapter*{姑妄言卷之十八\\
第十八回 崔命兒害人反害己 童自大得壽又得兒\\
附 司公子漁色失便宜 傅典史負心遭橫禍}
\addcontentsline{toc}{chapter}{第十八回 崔命兒害人反害己 童自大得壽又得兒}
\markboth{第十八回 崔命兒害人反害己 童自大得壽又得兒}{第十八回 崔命兒害人反害己 童自大得壽又得兒}

話說那吳老兒見妒妻死了。娶了一個美妾。他父親姓崔。曾做過一任北京刑部司獄司司獄。同禁卒通同作弊。四六分贓\footnote{是官長之贓。官多而吏少。惟獨刑獄禁子得六而官得四。}。苦難獄中的犯人。刻毒難言\footnote{有此惡父。方生此淫女。掌刑獄者當着眼。}。雖掙了些家私。後被上司知道。革職回來。他這女兒生得十分標致。崔司獄夫婦愛之如命。故起他個小名叫做命兒。捨不得把他嫁與尋常人家。要選一個做官有錢的佳婿。誰知姻緣不湊。總無其人。到了二十多歲。吳老兒素聞其美。煩人去說。崔司獄雖知道他是要做小。因上邊汲(沒)有夫人。一心情願\footnote{婿雖官而有錢。未必甚佳。}。纔嫁了這個富貴全備的老漢。做了一位尖夫人\footnote{尖字新奇。}。你道何爲尖夫人。他要說是小。上邊又無正室。公然與大無異。要說是大。却又是娶來做小。在又小又大之間。所以有此美稱\footnote{第十回內。童自大說賈文物云。你是半大不小的個老爺。此處又有這又小又大之夫人。俱是奇稱。辱翁曰。然則楊貴妃亦尖夫人也。}。吳老兒那夜\endnotemark[2]同他交合之時。見他。

\begin{quotation}

櫻唇微綻。星眼生波。腰肢纖弱傍人傾。做盡千般婀娜。玉手揉荑挽繡襦。裝成萬種妖嬈。聽他鶯聲巧囀。不覺魂敎呼去。經他陰中微鎖。早已精洩難收。口內聲聲喝采。好個嬌娃。心中暗暗躊蹰。這回斷送。

\end{quotation}

還有一個江西月說他兩人。道。

\begin{quotation}

白髮蒼髯老叟。紅顏綠鬢嬌娃。枯藤纏繞嫩柔花。也算鳳鸞同跨。吳友心中自喜。命兒口內頻嗟。元紅可惜付之他。斷送老奴方罷。

\end{quotation}

吳友又叫媒人覓了兩個美婢。他道名花不可無美葉以襯之。旣有佳人。豈可無艷婢爲侍兒。得了兩婢。一個纔十六歲。小名做姐。一個只十四歲。叫做尋姐。這是他一個厚朋友見他納婢。替他取此二名。吳老問其大義。那朋友道。兄要他們。名曰相伴新嫂嫂。\endnotemark[3]無非也要圖他生子之故。尋姐者。要在他身上尋出兒子來。做姐者。要向他腹中做出兒子來。吳老見此兩字新鮮。從古來侍兒小名錄上並無此二字。就依他命了此名。那朋友向別人道。此老這樣年紀。納此少艾。做孽尋死耳\footnote{這老兒有了一個崔命兒。也就危乎危乎了。又添上兩個粉骷髏。不死何待。}。這吳老兒望子心切。二來守着那奇妒的老媼過了一生。今日得了這三個少年嬌艷。正合了兩句。

\begin{quotation}

杖藜扶入銷金帳。一樹梨花壓海棠。

\end{quotation}

這老兒不知死活\footnote{不止此老。世上不知死活少年亦不少。}。終日在他們身上做工夫。你想一個古稀將至的老翁。還濟得甚事。初時還勉強支撐。到後來。弄得腰也彎了。背也駝了。眼也花了。耳也聾了。\endnotemark[4]黏痰鼻涕。咳咳嗽嗽的。有些動不得了\footnote{他的油也將盡了。命也將完了。}。思量要遞個病呈寬限。那崔命兒二十多歲的嫩婦。纔嘗此道。正是慾火發動之始。不額外加徵就是他的恩惠了。可還容得躱避。不住嘓噥道。你要我們。原圖生兒育女。難道娶我來看樣兒的麼。還掙着命來來是呢。吳老兒道。我也巴不得呢。他不替我爭氣。叫我也沒奈何。我身子雖動不得。我有南鄕的田。北莊的地。家私盡夠你受用一輩子。崔命兒冷笑道。這纔是笑話。女人嫁\endnotemark[5]丈夫。只圖穿吃兩件罷。你說有多少莊田。你這樣大年紀。就不曾聽見人說的兩句話。古語道是。家有良田萬頃。不如肏進些須呢\footnote{命兒這兩句成語套得通甚。}。每夜替他百般搓弄。間或也還有略硬的時候。拿他將就應差。不想又過了些時。那陽物竟犯八法中的一條。道是罷軟兩個大字。起先用兩個指頭做篾片幫扶着。還塡得進去\footnote{入銷金帳旣用藜杖。此道時(自)然離不得指頭篾片。}。後來竟像一條大蚰蜒蟲。鼻涕般縮做\endnotemark[6]一團。此時不但崔命兒着急。連吳老兒也急了。原是要他們生子。取樂還在次之。如今把一個生子之具都沒有了。不能下種。如何望得收成。只得去弄些春藥來助力。雖不能堅舉。又微有些硬意。崔命兒見頗有應驗。日裡不勸他強飯加餐。到晚來便勸他春藥多用些下肚。或多搽些在陽物上。那老兒也只得惟命是從。他一個血枯精敗的時候。可禁得這大熱助火的東西常常不離。不上年餘。兒子還不見一些影響。早把個老\endnotemark[7]子弄做別人家的兒子去了。吳老兒素常守着這些財物。只知道自己受用。並不知骨肉親友是個甚麼東西。待族間極其刻薄寡情。曾有個朋友說個笑話給他聽。道。

\begin{quotation}

一個財主死後。尚未入殮。忽然\endnotemark[8]醒轉。伸了一隻手向兒子道。我偌大家〔私。〕死了不帶一文。我捨不得。你把元寶給我一個拿了去。那兒子將他㞠子一刀割下。放在他手中。道。你死了。銀子還拿得去麼。只好攥着個卵子走罷。

\end{quotation}

吳老兒聽了。並不悔悟。他在日親友都不上門。今日一死了。他沒兒女。是個絕戶。衆族人都要來分他的家產\footnote{這一種無子而更吝的人。天下極多。豈皆無心肝者耶。冥冥中自然有個定數。昔一富翁。家資巨萬而無子。又鄙吝至極。連衣食皆不捨。一日。忽悟。我旣無子。積了與誰。何不自己受用。夜夢一神怒謂曰。爾何敢想擅用官銀。一驚醒來。吝心復萌。又復不捨。死後。無一親族。家資入官充餉。可見有定數存焉。此輩人之產。焉知非族間人之福。他代爲聚積耳。}。對崔命兒道。你不曾生育兒女。若要嫁人家。東西絲毫拿不去。只好帶你隨身衣飾之類。你若是願守。少不得分一股與你爲養贍之資。房子是我們要的。你只好到慈悲庵去守。却要剃了頭出了家纔行得。不然。一個少女嫩婦住在那裡不便。這兩條路憑你的主意。那崔命兒如何拗得過衆人。明知道衆人要攆他罄身\endnotemark[9]出嫁。拿出家\endnotemark[10]二字唬嚇他。量他小小年紀。決不肯做那削髮披緇的苦事。他暗想道。我靑春年少的。本要去嫁人家。他們定然一絲東西不肯與我。我這兩年受用慣了。知道嫁個甚麼人家。不如且出家守着。後來再做計較。便道。我雖是與老爺做小。老爺也不曾把我薄待。也是夫妻一場。他骨肉未寒。我怎忍就去改嫁他姓。我如今情願出家。到庵中去守。但這兩個丫頭原是老爺買給我的。我要帶了去。衆人見他願出家。倒沒法了。只得依允。撥了些\endnotemark[11]佃房與他討租用度。又撥些田地與他。以供口糧。也有一二千金之產。分與他兩房老人家在外替他照管。餘者盡瓜分而去。崔命兒自己私房也將及有千金之蓄。衆人把吳老兒開喪殯葬之後。崔命兒同兩個美婢。並他的箱櫃器皿之類。也就搬到慈悲庵。衆人託請一個老尼替他披剃了。按宗門法派起了個法名。叫做信悟。那個大丫頭做姐也剃了。做了他的徒弟。法名元品\footnote{好做兒子不曾做出。做了尼姑。}。小丫頭尋姐做了徒孫。法名妙炎。此時庵中先那些僕婦。三四個老的死了。那幾個年小些的。也都老邁了。仍留他們在內服侍。崔命兒在庵中。雖夜間在被窩中沒得事幹。覺冷淸難過。日裡却穿吃不愁。庵中景致甚是爽心。倒也自在。一日。天色將晚。見一個老僕婦進來。道。外面有一個道姑要進來借宿。我們不敢做主。請問師太許不許。崔命兒道。旣是女流。又都是出家人。這有何妨。可請了來。不一時。進來了。向命兒打了個稽首。命兒也回了個問訊。讓了坐下。仔細把他一看。好個道姑。生得端端正正。白白胖胖。頭戴妙常巾。身穿水田服。明眸皓齒。淨襪涼鞋。潔淨可愛。命兒問他道。師傅法號。仙鄕何處。那道姑道。貧道賤名本陽。別號守雌。\endnotemark[12]揚州人氏。雲遊到此。無處棲止。敢借寶庵。暫\endnotemark[13]宿一宵。命兒道。但恐敝庵無甚管待。簡褻師傅。那道姑道。豈敢。若蒙師太見容。就是大慈悲了。說着。妙炎拿上茶來吃了。須臾天黑。點上了燈。送上夜飯。吃罷。元品也來陪着說了一會話。那道姑談論風生。着實投機。崔命兒道。師傅不棄。我們同榻一宵。說說淸話罷。那道姑喜動顏色道。但恐賤軀有汚尊榻。旣承見愛。敢不奉陪。譙樓鼓動。夜漏將沈。二人上床。各被而寢。那道姑逗他一句道。師太\endnotemark[14]法臘幾何。年少靑春。爲何就入了空門。崔命兒嘆了一口氣。道。我今年纔二十五歲。因夫主仙遊。故在此出家守節。因道。師傅。你今年貴庚多少。爲甚做了道姑。又出來雲遊。那道姑道。貧道與師太同庚的。也爲先夫沒了。無子。族中將產業占去。貧道發恨出家。無處歸着。所以四處遨遊。復長嘆了一聲。道。別人多少夫妻團圓相守過日子。我貧道年又不老。半路孀居。身子都無處着落。言\endnotemark[15]之令人傷心。崔命兒笑道。旣然如此。你旣無所戀。何不嫁了人去。夫妻熱鬧不好麼。他也笑道。實不瞞師太說。貧道生來命苦。別人嫁的丈夫。恩恩愛愛的幾年。就不幸中途抛閃。守寡也自甘心。我當日父母被人誤了。把我嫁了一個老漢。師太不要笑話。我雖不叫做標致。也不爲很醜。嫁了幾年丈夫。被窩中連一次遂心暢意的風流事也\endnotemark[16]沒有受過。我這樣小年紀苦守的是些甚麼。料道貞節牌坊也輪不到我的身上。我何嘗不想嫁。又恐爲人所誤。不如不嫁。還得自由。崔命兒聽了他這話。眞是同病相憐。也嘆了一口氣。笑着道。我住在這庵中內。總不見一個男人的面。倒也罷了。你終日在外邊雲遊。男女混雜。也動心麼。他道。師太。你看螞蟻蝨子這樣微物。也知個陰陽交媾之道。何況人爲萬物之靈。那有不動心的。間或見了風流少年。心中愛得要死。春心一動。徹夜無眠。日間連飯食都嚥不下。這是我以心腹相吿。師太不要見笑。命兒道。你我都是同病。況且這都是人之常情。有何笑處。據你這\endnotemark[17]樣說。必定有外遇了。可實吿。不須隱諱。他道。不瞞師太說。我當日嫁了那個老兒。一個餳如鼻涕軟如綿的物件。弄得不疼不癢的\footnote{鼻涕綿三字。貶之太過。果如此。如何還弄得。雖有指頭篾片。恐亦無用矣。}。我出家這幾年。雖不曾遇着男子。常同婦人們在一處閒話。俗語說。三個男人沒好話。三個女人講屌話。他們這個說男人的物件有多粗多大。那個說有多長多久。我想若遇了這樣東西。也不枉失節一場。若還是同老兒差不多。又不如不做這事了。或一時興動起來。可是陳妙常那一首西江月道得好。

\begin{quotation}

一念靜中思動。遍身欲火難禁。強將津唾嚥凡心。怎奈凡心轉盛。

\end{quotation}

那心只一動。那裡還按納得住。到了萬分忍不得的時候。尋女伴中兩陰相合。搧打一會。人叫做磨鏡子。將就解解罷了。命兒道。男女幹事。全要那物件放在內中纔有樂趣。女人對女人。光撻撻對着撻撻光。有甚妙趣。道姑道。師太。你沒有做過不知道。怎麼沒有趣。我覺得做起來。比那沒用的老頭弄的還受用些。你這麼一想。便知道了。婦人對婦人。雖少了那件東西。都精壯有力。亂揉亂揉。還有些樂趣\footnote{命兒當云。例(到)底不如肏進些須。}。同那老兒弄時。那物件軟叮噹。已是不堪。再動不得幾下。不是腰疼。便是腿疼。更覺難過。你不信。我同你試試看。你嘗着了這樂趣。纔知道妙處呢。那崔命兒一個少年寡婦。他是沒奈何出了家。那一日一夜不想此道。今聽了道姑這些話。火已動到十分。却不好應他。只笑着道。我到底不信這事有趣。那道姑見他雖不應承。却是也想試試的口氣。先自己脫光。鑽了到他的被窩。就替他褪褲。那命兒也不推辭。笑着任他脫下。他一翻身上來。兩件光撻撻的東西對着搧打一會。那道姑亂拱亂聳的。引得那命兒陰中淫水長流。叫道。不好了。裡頭難過得很。你下來罷。他道。不妨。等一等就有好處。他不搧打了。對着陰門一陣揉。揉得那命兒春心蕩漾。意亂神迷。正在難過的時候。忽覺得牝戶中有個極粗極大。又硬又熱的東西塞得脹滿。且頂在一個樂處。妙不容言。心中動疑。忙用手一摸。却是那道姑胯中一條肉根。纔要問他時。被他出出進進。橫舂豎搗。命兒從來沒有經過這樣美事。連哼還哼不過來。那裡還顧得說話。被這道姑足足弄了有半個更次。命兒也丢了有四五回。方歇住了。命兒喘息了一會。問他道。你旣是個男人。怎麼裝做道姑來騙我。該問你個甚麼罪。他笑道。任你怎麼用肉夾棍\endnotemark[18]夾。皮腦箍箍就是了。命兒笑道。說正經話。你端的從何處來。如何知道來尋我。他親了一個嘴。笑道。我敢騙你。我自幼得異人傳授。學會了個縮陽的法子。若縮了進去。同\endnotemark[19]女人的陰戶一樣。用着時。就伸了出來。因爲有這本事。不忍埋沒他。故此裝做道姑。大發慈悲。專救這些少年寡婦的苦難。我聞得你月貌花容。靑年孤守。心中甚是憐愛。又是那不忍。特來與你應急。你可感激。況你是應以尼姑身得度者。我即現道姑身而爲說法。那崔命兒聽了。笑吟吟伸手將他陽物一摸。沈甸甸。甚是粗大。他道。你縮縮我看。他定了一定。不多時。果然縮得一些也沒有。只剩緊緊一條縫兒。把個命兒歡喜得了不得。說道。像倒像個婦人的。只是少了個心子。摸了一會。又道。你再伸出來看。一霎時。又伸出來。硬幫幫。比先分外的雄壯。他又跨上身來。命兒見他伸縮了這一番。正有\endnotemark[20]些興動。欣然笑納。又被他弄了有許多工夫。又丢了兩度。命兒道。你且歇歇着。我有話問你。他也就歇住。却不拽出來。命兒道。當日我夫主在日。他高興的時候。至多工夫不過四五十抽就完了。動不得。後來只放進去就了帳。他急了。弄了些春藥來助興。還略堅久些。歇歇動動。也還熬得一盞茶時。你弄了這半夜。怎還不見你洩。他道。你一個休說。就是十個婦人。我輪流一夜弄到天亮。也是不得洩的。命兒道。我就不信怎有這樣堅久的東西。當日我夫主的求其硬而不能硬。今日你的又不得軟。天公生物太不均勻。何不兩分着些\footnote{命兒此話愚矣。世間人富者太富。貧者太貧。尚不能均勻。何況此物。欲其如一耶。}。也罷。我被你弄了這兩次。也來不得了。我一個。料道也敵不過你。你旣然在這裡。我那兩個徒子徒孫也瞞不得他。大家弄弄。一者免得口聲。二來試試你\endnotemark[21]的本事。那本陽先見那兩個妙尼。雖不能及命兒。但命兒如一朶牡丹。他兩個也還是兩枝芍藥。不到十分不及。此時正在想慕他們。思量尋了來做這事\footnote{這才正是尋姐做姐。}。但不便出口。恐獲得隴望蜀之誚。聽了命兒這話。滿心歡喜。忙答道。你的尊見極是。\endnotemark[22]命兒叫道。元品妙炎\endnotemark[23]都過來。原來命兒同道姑先在高興的時節。因夜靜了。命兒被他上邊抽出的哼聲。下邊搗出的淫聲。遠聆數室。那妙炎起來溺尿。忽然聽得。覺此異聲出自他師太床上。近前一聽。掀天\endnotemark[24]揭地的大幹。他忙去推醒了元品。同來竊聽。聽了許久。那元牝妙眼之中那種難過。是不消說。連兩隻腿都酥麻的動不得了。站都站不住了。只得蹲在地下。那陰中之水順着直流。聽得命兒叫他們。口中都答不出。只鼻孔中哼着應。却不見走來。原來他兩個竟酥癱了。動不得\footnote{前嬌花偷聽宦蕚侯氏交合時。也是酥麻了。動不得。忽聽見叫他替兩(他)。兩隻腿反硬起來。忙忙走到床前。那是未曾經歷之處子故耳。元品妙炎是知味者。熬了許久。忽聞此聲。故酥癱動不得了。妙。}。本陽聽得是在床後面答應的聲氣。將陽物拔出。忙跳下床。走去一看。見他二人披着件小衫子。光着屁股。蹲在地下哼。他一手抱着一個。上得床來。先將元品放倒。摸他的嫩牝時。淫水泛溢。連兩股都濕了。就弄將起來。弄了一會。看那妙炎時。急得爬起睡倒。有個要死要活的樣子。忙同他又弄了一陣。命兒看得騷興大發。伸手去他牝中。將陽物生拉出來\footnote{妙炎此可謂拔出眼中釘了。}。塡入自己戶內\footnote{這個塡字。又要用指頭做篾片了。}。大弄了一場。三人輪流。果然弄到天亮。\endnotemark[25]他還不曾洩出。大家歇了起身。命兒問僕婦們借了梳子\footnote{細極。他是光頭。無此物者。}。與道姑梳丫頭。大家淨面洗漱已畢。坐下吃茶點。命兒道。我師徒三個身子都付了與你。你却要情長。不要日久厭了。沒良心。撇了我們。又去別戀新人。他忙答道。我承你這番厚情。豈敢變心。遂設誓道。我若後來負了你師徒三位。另厚別人。粉身碎骨。死於官刑之下\footnote{後來道姑之死恰與此誓相合者。謂其罪適當如此死耳。非俗所謂犯了誓神也。}。命兒把這本陽留住。也就如他的性命活寶一般。如何肯放了他去。每日叫僕婦們上街買上品佳饌美酒供養他。每夜三人挨次同他大弄。兩三夜並不見他走洩。命兒問他道。你的話倒也不假。本事委實高強。你從來可曾遇着狠婦人把你弄丢了的麼。他道。我要洩就洩。要不洩再不得洩的。所遇的都是些少年寡婦。或是未嫁的處子。如何弄得我丢。只有接引庵\endnotemark[26]有一個姑子\footnote{不聞〈甚〉其名久矣。不知風騷尚如昔日否。}。黑黑胖胖。有四十來歲。是個辣手。我聽得人說他會採戰。我去同他試了一試。我却敵他不過。一夜定有兩三次走漏。命兒道。他怎麼個採法。本陽道。我這東西弄了進去。被他一口咬住。內中緊緊的裹住了龜頭。一陣狠咂。咂得骨軟筋酥。由不得就洩了。命兒道。他\endnotemark[27]這個法兒也肯傳人麼。那本陽道。這是他的養身祕法。如何肯輕易傳授與人。人若學會了這個妙法。同少年精壯男子弄聳。採了陽精。補益髓血。可以返老還童。髮白轉黑。延年益壽。你想想。這樣仙訣可肯傳與人麼。命兒道。這甚麼相干。他獨自會也不過如此。就傳與人也還是如此。難道別人會了就占了他甚麼去了不成。他的若肯傳我。我重重的謝他。你旣同他相厚。你去和他\endnotemark[28]說說看。本陽道。倒還有個機緣。我明日看看去。我數日前在\endnotemark[29]他那裡。他對我說他有自幼相與的一個厚\endnotemark[30]朋友\footnote{玉簪記那船家說陳妙常云。我老兒活了六十九。不曾見師姑與秀才做朋友。此老可謂愚甚。天下姑子能有幾個不同男子做朋友者。}。叫做到聽。\endnotemark[31]數年前煩他替人轉借了三十兩銀子做本錢。不想這姓到的\endnotemark[32]前年就死了。數年來本利絲毫未曾還人。這債主前日到庵中打鬧。問保人要這銀子。年分多了。本利滾算。該一百幾十兩。債家死了。保人代還一半。還該八九十兩。那債主勢力又大。他一個出家人。如何拗得過他。他正急得沒法。等我對他說。他若肯傳你。你替他還這宗帳目。看他可肯。你可有這項銀子麼。命兒道。他若果然肯盡心傳我。我竭力湊了與他。本陽道。先說過。你若學會了方法。先拿我採起來。就行不得了。命兒笑道。你是引進的恩人。怎肯採你。大家說得高興。又輪流大戰了一場。然後睡下。次日。本陽到接引庵對黑姑子說了。那黑姑子正在着急的時候。滿口應允。遂同本陽到慈悲庵來見了命兒。命兒\endnotemark[33]見他形容醜黑。心中動疑。讓他坐下。茶罷。姑子先開口道。方纔這位道兄說師太要學貧尼的秘術。可是眞的麼。命兒道。正是呢\footnote{此三字。疑而未決之辭。}。我聽得這位道兄說師傅的妙法可以返老還童。有許多妙處。故此想要拜求你。我看師傅的尊貌怎麼這樣老蒼。那姑子見他遲疑。笑着道。哦。師太疑心是假的麼。這有個道理的。採戰雖有補益。也要有那補益的東西。方纔見效。即如人參。名爲補藥。必定要吃下肚去。纔得見功。沒有只拿着看看就能補人的道\endnotemark[34]理。貧尼一來生得貌甚不揚。不能招攬少年淸俊。二來敝庵淺促。又人眼衆多。做不得這事。縱有奇方。做不來也沒用。要像師太這深房秘室。自己旣做得主意。況且這樣靑年美貌。指着元品妙炎道。又有二位師兄這樣好幫手做了招牌。何愁甚麼主顧不來下顧。只要\endnotemark[35]你學熟了。善於運用。一日雖十次。也\endnotemark[36]不爲多。越多越妙。然後纔見功效。命兒道。也要傳多少日子。那黑姑子道。像師太這樣聰明的人。不過三五日。盡得其妙。即不然。到七日。再無不透徹的了。命兒滿心歡喜。叫備齋。命\endnotemark[37]元品陪着。他到房中將私蓄取出百金。然後出來。同他們吃飯畢。攜了那姑子到內。將銀子遞與他。道。師傅。這是一百兩足紋。你拿去使用。我晚間候你來。你傳了我。若實在有好處。我還謝你。那姑子見了這兩大包銀子。歡喜異常。答道。蒙師太救了我的急。我若不盡心相傳。眞是畜類了。我回去還了人。今夜必來。拜謝而去。到了將晚。那姑子果然來了。吃了夜飯。命兒叫本陽過那邊屋內。同元品妙炎去睡。他同姑子共寢。敎導心法。古語道。

\begin{quotation}

世上無難事。只要有心人。

\end{quotation}

那姑子也盡心相傳。命兒更盡心領敎。三四夜就全得其奧妙。命兒問他道。師傅。你這個妙法。當日是甚麼人傳授你的。姑子道。這是我十八九歲時。遇着個陝西雲遊道士。蒙他傳我的\footnote{此一回將第一卷開首三人重復一提總結。去後不復見矣。}。師太學會了這個法子。只有一件要緊。却要留心。當日這道士再三囑咐我道。倘遇着有會採戰的男子。看他手段要利害。就忙迴避。若被他採丢了。不但將前功盡棄。還要傷了性命。這叫做崩\endnotemark[38]鼎。若保固得住。吸得過會採戰陽精。來得這一次。却也抵得每常千次的功效。補益却也不小\footnote{魚因貪。死於餌。人因貪。死於財。命兒實死在此一句上。}。那男子渾身精脈喪盡。也不能保全性命。他又曾說道。但是男子再採不過婦人。他是動。我是靜。以逸待勞。他是剛。我是柔。他外有形。而我內無形。不但柔能克剛。以無形而制有形。自然得勝的多\footnote{這幾句是崔命兒的催命符。}。然不可不防。命兒也聽了在心。那姑子辭了要回。命兒又送了他些禮物別去。命兒心中想道。今夜且拿這假道姑試試法看。到了晚間。對本陽道。這幾夜一箭雙鵰\footnote{倒是一刺雙蚌。}。也算你樂夠了。今夜過來同我睡。本陽道。你學會了麼。命兒道。他雖傳授了。不知法靈不靈。我同你試試看。他道。只許這一次。下回使不得的。命兒笑道。還不知驗與不驗。你就這樣害怕。說着。兩人上床脫盡。命兒叫他上身。\endnotemark[39]弄將入去。幾下送了個盡根。命兒運用起來。一下咬住。本陽覺得與那姑子無異。分外還裹得緊些。不多時。被他採去了。那命兒覺得丹田內一股熱氣。行遍週身。眞如醍醐貫頂。甘露融心。其樂無比。暗思道。這個妙訣果是精奇。且不要饒他。再採他一次。也不爲\endnotemark[40]過。兩手將他摟得緊緊不放。下面仍然咬住。本陽道。我洩了。你放了我罷。命兒也不答應。閉目運氣。更加力鎖採。約夠一盞茶時。只聽得本陽道。哎呀。不好。說了一聲。下邊又冒了。本陽着急道。你好沒良心。我舉薦了人敎了你。你倒不顧我死活。這是恩將仇報了。命兒摟住他。親了個嘴。笑道。我怎肯傷你。這算替我前日那幾夜報仇。笑嘻嘻\endnotemark[41]的放了一口氣。本陽見內中陽物鬆活了。連忙拔了出來。道。下次決不可如此。男人被鎖丢一次。比每常自己洩的三次還利害呢。命兒笑着同他相摟而睡。這本陽戀他三個騷而淫的美婦。到晚滾做一床。週而復始。輪次搏弄。命兒一夜定要採他一次。過了數夜。有些膽怯。旣同元品妙炎弄。又不得不同他弄。弄了又怕。心中一餒。但將陽物送到命兒牝中。就不能十分強壯。也不用狠採。只略鎖幾下。他就大洩如注。不到半月。漸漸支撐不住。心中還捨不得。又過了幾日。雖戀着那元品妙炎的嫩物。却甚怕崔命兒利害。性命要緊。顧不得了。那日。推往外邊走走。竟逃之夭夭。高飛遠走去了。\endnotemark[42]\footnote{本陽此走。罪有可原。昔有一笑談。有一國王。一日向寵臣道。宮中女子盡皆黃瘦憔悴。有何法以治之。那寵臣道。大王但任臣醫治。不過百日。自然痊癒。王喜允。此臣選壯健男子數百。納入宮中。未及三月。死者過半。而女子個個面上紅光飛舞。此臣請王遊宮。王見諸女大異向日。心中大喜。正贊獎時。忽見一處堆積許多死屍。驚問此臣。他對道。藥皆醫治了衆女。這都是藥渣兒。本陽他若不走。豈定待做藥渣而後已耶。}命兒當晚不見他來。還只說他別處有事。等了數日。不見踪影。方知他是鳥飛兔脫\endnotemark[43]了。命兒旣學會了這種妙技。可肯安靜持守。一心想弄些少年來做補藥。遂與元品妙炎商量了一個妙策。叫他二人做牽頭。他二人知道這件事是有樂無苦。自然喜諾效勞。命兒叫了幾個老僕婦來。吩咐道。我們如今在此。人口衆多。靠誰養贍。庵門成日關\endnotemark[44]着。也不是事。今後開了。聽人隨喜。倘或有緣。遇着個貴宦長者。做了護法。也有個指靠。這些老婦都是手下人。又聽他說得辭嚴義正。可敢不遵。竟把庵門大開。\endnotemark[45]慈悲庵中的華麗。左近居人皆知。誰\endnotemark[46]不要到庵中賞玩遊覽。因先是門常關着。又知是姑子庵。誰好敲門打戶進來走走。今見開了。就有閒人走到內邊看看。元品妙炎輪替在廂房中坐守。在窗洞中往外張。有那老年誠實的。便憑他去了。見有生得淸秀少年。穿得略乾淨些。就出來招攬。\endnotemark[47]慇懃扳答。但那些輕薄少年見了這樣姑子。又\endnotemark[48]在靑年。可有不想他臍下的那件妙物。或說句\endnotemark[49]風流話兒勾引。他便開門笑納。再不推辭。上樣的進與命兒。其次者他二人留爲自用。漸漸也就人來隨喜的多了。命兒大發慈悲。一槪布施。人經了他這妙牝。有老成些的知道利害。就得趣抽身。有那不知死活的少年。上面愛他的嬌容。下邊喜他的幹法。死死戀住。十人之中。四個成癆。倒有六個喪命。被他把藥汁吸盡。都成了藥渣兒了\footnote{這一種藥渣。世上甚多。}。行了數載。被他這一點美穴中。葬了多少少年。那元品妙炎雖不曾害了大人。他二人腹中的小娃娃。數年來後園中竟做了一個子孫窖子。暗暗埋在內中無數\footnote{吳老兒陰間可有了兒子了。}。起先那些老婦見他三人如此行事。較淫娼尤勝。雖不敢當面談論。背地也不知恥笑了多少。到後來看熟了。甚覺眼熱\footnote{不知是上眼熱下眼熱。}。不但不說他們的不是。反恨自己年老了。不得像他們這樣風流快樂。眞所謂近硃者赤。近墨者黑。命兒這一日\endnotemark[50]正在閒坐。要等個人來取樂。忽然一個僕婦進來。道。外邊有一個二十多歲的男僧要來掛搭。我回他是女僧庵。他說是淨過身的老公。沒有陽道。不妨得。叫我進來回師太。命兒想道。就是有陽道的也不怕他。何況沒有。我只聽見人說老公是割了陽物的。却從未見過。何不留他。看看是怎個樣子。遂道。你引了他進來。那老僕婦出去。同他來到房中。那和尚連忙施禮。命兒回了。讓坐。看他好條精壯健漢。暗道。這個人要是有陽物。倒是精壯。採他些。大有益處。可惜是個老公。遂問道。師傅是幾歲淨身的。他答道。貧僧十二歲淨身。今年二十四歲。淨過十二年了。命兒道。這割過也還長麼。他道。年年要修的。不修。一年也還長出一寸來。命兒道。師傅。你有幾年不修了。他道。貧僧有七八年來不曾得修。那老僕婦伸着舌頭。道。七八年沒修。就長出七八寸。阿彌陀佛。夠了夠了。衆人望着他大笑。那老婦自覺失言。紅了臉。忙忙走出。命兒笑着問道。師傅。你這重長出來的。可與先的一樣麼。那和尚道。自然是一樣。命兒道。可借出來看一看。那和尚見了這三個齊整姑子。腰中那小和尚久矣直豎。聽了這話。知有俯就之意。忙取將出來。命兒一看。果然約有八寸來長。原來這和尚是個賭錢吃酒養婆娘三者備矣的一位高僧。素聞命兒之美。又知他延攪英雄。故謅出這話頭。以爲進身之階。命兒見了。知他是個假話。心中喜道。從不曾遇這長大之物。且試試新看。遂走到床上坐下。那元品妙炎知局。即抽身出去。隨手將門帶上。那和尚忙到床前。替命兒脫了衣服。他也脫了。上床就幹起來。那和尚原\endnotemark[51]想來賣弄他的大傢伙好本事。並不知命兒的利害。興興頭頭。鼓勇盡入。欲施展他的威風。不想弄了進去。那命兒覺頂到至深處。甚有妙趣。又將陰中揎得〖阝少日小〗縫\endnotemark[52]皆無。領敎過無限的陽物。從未經此。被他一下咬住不放。一陣吞鎖。那和尚把持不住。就洩了。命兒喜他精脈壯盛。那裡肯放他。連夜飯都不吃。一夜之間。採了他七次。那和尚頭腦轟轟。一陣陣發迷。他腰眼酸痛異常。苦吿求饒。命兒纔放鬆了他。那和尚見了天色明。忙穿了衣服。臉也顧不得洗。垂首喪氣。抱頭鼠竄而去。權且按下。再說那假道姑自慈悲庵躱了出去。在那些尼庵道姑處借宿。偶聽得說有一個姓藺的人家酷喜僧道兩門。他便想去投托棲身。你道這是個什麼人家。這人姓藺名馥。妻于氏。家中甚是殷實的。他夫妻二人都有六旬年紀。他兒子名叫藺通。是縣衙中一個能吏。也還是胸中明白的人。媳婦強氏。女兒名喚佛姑。他夫妻二人一生好的齋僧敬道。禮斗誦經。斷酒除葷。持齋念佛。他兒子再三苦勸。決不依從。後來勸得次數多了。那藺馥\endnotemark[53]反責備兒子。毀僧謗道。不敬三寶。不能體貼父母的虔心。大是不孝。你在衙門中。豈不知法律。忤視父母尚然有罪。何況逆父母向善之心。其罪更當何如。那于氏更爲可笑。但見兒子勸他。便咬牙切齒道。孽障。你的欺道滅僧。後來定有惡報。天打雷霹(劈)。你看我老兩口子定有好處。就是你妹子跟着我們這樣持齋念佛。將來定然有福。夫榮妻貴。比你強百倍\endnotemark[54]呢。那藺通是個衙門中人。怎敢當父母責以不孝二字。知父母是勸不醒\endnotemark[55]的了。只得由他尊意。他那女兒佛姑已二十九歲了。被父母生拿活捉。叫他吃齋念佛。每日跟着燒香拜禮。他違拗不得。沒奈何。只得依從。心中老大不願。巴不得早嫁了人家。脫離了這苦難。因藺馥于氏要選個吃齋信佛的女婿。纔肯與他\footnote{此等佳婿。雖於僧道中覓之。恐亦難得。}。你想這愚而佞佛的人家。一時如何遇得着。所以只管躭誤了他的靑春姻緣。他那愁恨之心。雖不敢向着父母使出。那女子願爲之有家的\endnotemark[56]心。雖然如此。每日在風淸月朗之夜。或錦衾繡榻之中。搥床搗枕。短嘆長嗟。兩淚偷垂。咬牙切齒的暗恨。那本陽聽說藺家好道信佛。連女兒也叫佛姑。三十歲了。生得甚是齊整。在家吃蔬看經念佛。一心要選一個持齋的好女婿。本陽聽在心。暗想要替他女兒開一開葷。就到他家去化齋。于氏聽見是個道姑。忙叫請入內室。藺馥見他相貌端莊。語言穩重。就盤問些道經釋典。應答如流。夫妻二人滿心歡喜。\endnotemark[57]以爲是他心地虔誠。感動了活仙姑降世。盛齋款待。苦死要留在家中。長遠供養。晚上就叫他與女兒佛姑同臥。求伊夜間傳授女兒些釋道妙諦。以做將來成佛做祖的津梁。那道姑滿心歡喜。正合他來意。連聲應諾。藺馥于氏歡喜非常。以爲女兒若得了這個仙姑心印妙義。倘得正果。將來他夫婦這一對公母佛。一定到西方極樂世界去。不消說。且說這道姑同佛姑二人得在一處。以乾柴就烈火。\endnotemark[58]豈有不生燃之理。本陽見佛姑果然生得俊美。晚間上床。見他身上雪白皮膚。兩隻三寸的金蓮。換了一雙大紅睡鞋。好生動火。心中雖然愛。却\endnotemark[59]不敢造次動手。只好慢慢的引動了他的春心。纔可行事。住了兩日。熟了。無人處間或說兩句笑話兒勾引他。他三十歲的聰明女子。甚事不知。何事不想。但只是女孩兒家臉嫩。不好答應。只微微含笑。心中也巴不得問問過來人此道內中的妙處。一夜。本陽同他睡着。說道。\endnotemark[60]姑娘。你今年將三十歲了。別人家的女兒十四五歲嫁了丈夫。到了你這樣年紀。養過了七八胎。連孫子都差不多見了。男人的那件好東西。也不知受用過幾千回了。可憐你還不曾嘗着那味兒。你心裡不急麼。那佛姑笑着。不好答應。本陽又道。女人生在世上。只十五歲到三十五歲二十年的風光。夫婦快樂。過此便是半老佳人了。你白白的虛度到三十歲了。再捱幾年。豈不空過了一世靑春。虧你這樣空房獨守的不急。引得那佛姑抓耳撓腮。只是嘆氣。本陽雖知他情急。却不敢下手。漸漸假裝睡着。只聽得佛姑翻來覆去。不住聲長吁短嘆。本陽假做夢中顫着聲兒哼。身子不住往上攧簸。佛姑見他這個樣子。只道他是夢魘着了。忙推着叫他。他做那夢中驚醒的光景。連嘆了幾聲。道。可惜。可惜。\endnotemark[61]一場好事被你這不知趣的人打脫了。佛姑道。你睡着了。有甚麼好事我打脫了你的。他道。你是女孩兒家。吿訴你不得。就對你說。你也不知道那裡頭的妙處。佛姑先聽他說了那些話。心中已是很難過。正要老着臉細細問他。見他睡着。只得忍住。此時又聽得他說這個話。笑嘻嘻的儘着追問。本陽道。你這樣苦苦的問我。我對你說了。那時動了火。沒處發洩。却不要怨我。佛姑笑道。我不信就這樣的。甚麼好吃的果子。你就說得金綠綠的。你只管說。看我可動心。本陽道。你沒有嘗過。怪不得你。若嘗着了這滋\endnotemark[62]味。只怕要想死了你呢。我纔睡着了。夢見一個標致小夥子把我抱住。扯掉了褲子。挺着他那又粗又長的東西。鐵硬的塞在我這裡頭。本陽嘴裡說着。就一把將佛姑摟住。下身一陣亂聳。道。他就是這樣把我一陣亂抽搗。弄得裡面酸酸癢癢。那說不出來的快活。我渾身都酥麻了。正在受用。被你叫醒了。豈不可惜。佛姑聽了這話。心中火已引動。強笑着說道。我不信這東西就這樣有趣。你一個出家人。幹得這個事的。本陽道。你將三十歲。怎還說呆話。人生在世上。還有快活過這事的麼。你家老爹奶奶也是在出家持齋把蔬的呢。\endnotemark[63]要不幹這事。你打那裡來的。我聽得人說的一個古語兒。說給你聽聽。你就知道這件事的妙處了。

\begin{quotation}

幾個婦人偶然在一處說閒話。內中一個說道。我們大家想想。人生在世上。第一件快活的是甚麼事。大家想了一會。一個道。我想來極快活的事莫過於肏屄。衆人齊道。果然不錯。眞快活。又一個道。這一件是極美的了。可還有快活的麼。這一個又想了一想。搖頭道。沒有。沒有。要快活。除非再肏。

\end{quotation}

你想想。這是天下第一件快活的事。你沒有嘗着。所以不知道。不瞞你說。我十三四歲時在家做女兒。就同人偷着弄了。後來嫁了一個丈夫。那東西一點點子。甚不濟事。不上半年。他就死了。恐怕再嫁了人。又遇着這樣短小不濟的。豈不躭誤了一生。借名出家了。在外邊看有又粗又大好物件的精壯男子。相與幾個。也不枉爲人一世。我今年與你同年。不敢誇嘴。大大小小的。也見過有百十個。佛姑道。我聽見說弄還要疼。你怎倒要大的。不怕疼麼。本陽道。女兒破身。不過頭一次有些疼。後來就好了。越大越有趣。那小東西弄得不疼不癢。要他做甚麼。佛姑道。到那快活的時候。是怎樣的樂法。本陽道。男人的那東西弄了進去。抽上一會。弄得裡面似酸非酸。似癢非癢。心窩內都不能自主。就像是要死要活一般。四肢百骸。渾身經絡。都酥麻起來。這個趣眞形容不盡。有一個笑話。

\begin{quotation}

兩口子兩三日沒吃飯。他夫妻商議道。飯雖沒得米煮。我兩人的東西是有。何不高興一番當了飯呢。兩人就弄將起〔來。〕弄了一會。兩個俱洩了。頭迷眼花。昏昏暈暈的。二人道。原來這件美事不但可以當飯。而且可以當酒。

\end{quotation}

本陽對他說話時是臉對着臉。就借這個意。摟着親了個大響嘴。道。這樣美味。你後來試着了。纔知我的是眞話。又將他雙手抱住。嘴對嘴道。若同一個少年美男子共臥。不要說弄。就是臉挨臉。嘴對嘴。四隻胳\endnotemark[64]膊摟着。兩雙腿壓着。胸貼\endnotemark[65]着胸。股疊着\endnotemark[66]股。親親嘴。咂咂舌。也就酥麻得要死了。佛姑兒聽這些話。急得一臉火。牝戶中也就流出些淸水。心中撩亂。着實難過。把\endnotemark[67]他擰了一下。嘴中強着說道。我到底不信。本陽放開手。又說道。這件事定要親身歷過。纔知道有這些妙處。空對你說。你自然不信。胯襠中的一條縫兒。如何就樂到這樣地位。我雖然親身經過。過後想起來。還解不出這宗道理。何況於你。本陽同他說着話。伸着手將他遍身撫摩。\endnotemark[68]緊緊的兩個乳餅貼在胸前。身上又光又滑。摸見他褲子雖然穿着。却不曾繫帶子。趁勢一伸\endnotemark[69]手下去。摸着他那件鼓蓬蓬\endnotemark[70]光滑滑的寶貝。一條細縫。微吐着一點雞舌。水潤得潮潮的。笑道。你旣說不信。怎也動了心。淌\endnotemark[71]出水來了。佛姑也不迴避。任他摸。笑道。你說我。你還不知淌得怎\endnotemark[72]樣的呢。本陽道。不敢欺。我是見過世面的。不像你這樣饞。不信你摸摸看。佛姑正想摸摸經過男人的陰戶是怎麼\endnotemark[73]樣兒。聽說。就伸手一摸。短短的一大些毛。一條大縫。果然乾乾的。沒一點水。却有核桃大的一個大疙瘩。頂上微微有些黏涎浸出。驚問道。你這是甚麼。本陽因摸了他一會身子。又摸着那件妙物。說話時候嫩臉廝挨。脂香沁腦。就是鐵石人也沒有不動心的。忍不住突出一個龜頭。却死命的縮住。笑着說道。這是我從小生來的一塊努肉。先還小來。如今漸漸的大了。要狠努一努。竟努出三寸多來。佛姑道。你努了看看。用手摸着他的。果然努出有三寸多長一個光頭。佛姑道。男人的可是這個樣子。本陽道。雖不同些。我的這個也可以同女人弄得的。我同你做做看。就脫他的褲子。佛姑此時也心渾意亂。任他脫下。本陽也脫了。爬起來。叫他仰面的睡平了。把兩條腿揸開。他伏上身。先把他牝戶中抹了些唾沫。用指頭挖挖。眞是未經陽道的原物。緊揪揪。妙不可言。自己龜頭上也抹了。然後慢慢輕輕塞了進去\footnote{佛姑造化。今日開葷了。}。佛姑雖然疼痛。但他情急得很了。也顧不得。咬牙忍受。那本陽放將進去。就不是那個三寸多了。全身盡入。佛姑忍受不得。皺着眉道。脹疼得很。本陽道。你忍着些。到後來自有樂處。淺抽慢扯。弄了一會。佛姑雖覺得漸有意思。却因他的那努肉太大。撑得甚疼。說道。你下來歇歇再頑罷。我裡頭痛。不好過。本陽依他。拔出來。他枕傍有一條白紬汗巾。拿過來把牝戶揩一揩。拿上來一看。因他年紀大了。雖無猩紅點點。也還有些淡紅顏色。說道。被你弄出血來了。伸手去摸他的那塊努肉。竟成了鐵一般七寸來長一段巨物。大驚道。我說怎麼這樣疼。原來長得這般大了。你像是個男人來哄我的罷\footnote{肉食者鄙。他全虧吃齋吃聰明了。此時竟爾懂得。}。本陽捧着他的嫩臉。親了個嘴。道。親親的心肝。我果然是個男人。聽見你生得十分美貌。又年紀大了。耽誤着你的靑春。故此來同你做伴。又摟過脖子。親\endnotemark[74]了一個大響嘴。那佛姑也是求之不得的事。況弄已\endnotemark[75]被他弄了。還有何說。歡歡喜喜相摟相抱。睡了一會。重又弄起。這一次不比起先。佛姑雖然還痛。似可忍受多時。也稍得了些樂境。過了數日。他這塊又粗又長不軟不洩的努肉。弄得這佛姑竟成了一尊快樂自在佛。面上紅光飛舞。喜笑顏開。那藺馥同于氏見女兒大不同往日那苦面愁容。以爲是他得了個仙姑的妙諦眞詮。明心見性。纔有這番樂態。本陽因愛佛姑過甚。到那十分高興之時。把那菩薩甘露不由得滴幾點在他那兩片肉蓮瓣之中。這却弄出禍來了。過了幾個月。這佛姑年(眉)低語慢。腹大胸高。這些丫頭僕婦見他二人言語嘻笑之間。也見了些破綻。因〈是〉這位仙姑是老主夫婦供養活神仙。何敢輕言。這藺馥于氏只顧念他的佛。那裡知道女兒佛姑的腹中竟有了道姑的仙種。一日淸晨。一個丫頭在房中掃地。見被拖下地來。忙將被拾起。掀開帳子。送上床去。不想他二人脫得精光。道姑仰臥。佛姑騎在他身上。摟抱着鼾呼大睡。這丫頭動疑的悄悄將他下身一看。佛姑的陰門兩瓣。道姑的努肉長脫(拖)。忙走去吿訴自幼帶佛姑的一個老僕婦。這老僕婦近來見他二人的舉動。也有些疑。聽了這話。更留心伺察。夜間聽得床上笑語喁喁。那淫媾之聲。夜靜了。明明聽得。次日。冷眼看他。見佛姑穿着一件對衿小衫梳洗。乳大腰粗。雖然勒着抹胸。帶子放得大長。高腆着一枚鼓肚。約似乎有半載胎胚的樣子。那老婦見事體不妙。料瞞不住。不敢向老主夫婦說。悄悄吿訴藺通。這個藺通雖然心中恨甚。也還在疑信之間。那日道姑出來去了。叫人請了妹子到他屋裡來。着他妻子強氏按住一摸。果然一個大肚。還恐他是有病。扯下底衣。將他牝戶一看。兩片皮大張。已成了紫黑顏色\footnote{一女子嫁夫之後。一日。偶然低頭看見陰戶。大詫道。我一個通紅的花心。怎弄成了紫黑色。不意這樣一個小小光頭。竟會做染博士。}。強氏覺得比自己嫁久了丈夫的陰門色道雖同。其形狀似乎寬濶幾分。就盡情吿訴了丈夫。藺通氣恨塡胸。叫妻強氏留着小姑娘在房中。不要放他去。瞞着父母。到外邊等道姑回來。叫家人拿住他進來。審問妹子情由。那佛姑贓誕(證)俱明。遮飾不得。實吿是道姑的點綴。藺通出來。將本陽帶着。同到縣中來稟見知縣。素常着實愛他。他見了。求避迴了衆人。他跪下哭訴父母佞佛好道。以致惡棍假冒道姑姦淫了他妹子。求恩盡法處治。但求毋究妹子之事。恐張揚醜名。無顏在衙門中站立。叩懇天恩。做官的人聽見了這樣的事。可有個不髮指痛恨者。即刻升堂。帶進道姑。審問他是何處人。敢男假女裝。私入良家內室。他供是揚州府江都縣人。執定是女身。並無假扮情由。知縣大怒。命拶起來。敲了數十。他咬牙不招。知縣吩咐傳了兩三個穩婆來。互相驗看。都稟道。雖無陽物。却與婦人各別。再命剝去他衣服。將奶頭驗看。却與男子無異。這知縣是個明理的人。說道。這是縮陽法子。命取了些豬油用鹽蘸着。叫衙役擦在他胯下那縫中。叫牽了狗來舔咶。狗聞了那油味。一陣舔\footnote{閱此。偶想起火氏來。不知尚用此法否。}。狗舌最熱。不多時。那道姑忍縮不住。紫漒光鮮一條大肉棍突出。衙役稟知知縣。叫帶上來。怒罵道。你這個惡奴。也不知被你展(玷)汚了多少婦女。你罪萬劫莫贖。本縣要申文上臺。徒汚我紙筆。吩咐衆役。可拉下堂去。你們各持板棍。替我亂打。以死爲度。衆人也動了公怒。上前亂斫混打。頃刻之間。化成了一堆肉醬。知縣怒猶未息。叫拉出去餵豬狗。藺通看着事完了。官府退堂。叩謝來家。立逼着妹子自盡。做了個繩圈兒。繫在梁上。請君入套。不由那佛姑做主。他夫婦二人擡他上去掛上。看着吊死了。纔出了這口惡氣。然後去稟知父母始末詳細。請母親到房中去親驗妹子的尊腹同下體。那藺馥于氏是他兩口子自做錯了的事。抱怨不得兒子。這是敬僧重道持齋念佛的好報應。又說不出來。女兒又死了。要選個好佛的女婿。不增(曾)遇着。倒替一個假裝的道姑殉了死。自悔佞佛之愚。已無及了。生生自己坑了一個女兒。他夫婦痛哭了幾場。替女兒念了有幾千遍往生咒。藺通只說妹子病故。裝殮擡出。一火焚之。藺馥于氏念佛之暇。即互相抱怨說。誤留了這個假道姑。倒送掉個眞佛女。隱恨在心。不久雙雙下世。這話兒吹入崔命兒耳中。聞知道姑如此死法。心中大暢。道。這個負心奴撇了我。別戀情人。應了前誓了。一日。正在房中閒坐。見妙炎引進個美少年來。命兒將他一看。雖然穿着一身布服。却生得俊美非常。十分可愛。見他。

\begin{quotation}

面如紅玉。類漢室之韓嫣。膚若凝脂。擬晉時之衛玠。齒齊編貝。開口常噴荀令之幽香。唇賽塗硃。吐語一似秦靑之嬌囀。論丯姿。宗(宋)朝未必能強。說容貌。彌子或堪與匹。\endnotemark[76]體穿舊舊布衣。恰稱身材窄窄。髻挽\endnotemark[77]絲絲。黑髮。偏宜美貌翩翩。貪淫女自應魂迷。光頭尼霎時魄蕩。

\end{quotation}

命兒一見了。喜孜孜笑臉相迎。忙讓了坐下。心中暗想道。我每常自負。以爲自己是極美的了。疑婦女中尚無我之儔匹。不想他一個男子。竟這等標致。與我相形起來。我眞要拜下風了。兩隻眼睛盯在他的臉上。連眨也不眨一眨。倒把那少年看得頸紅面赤起來。元品隨即捧上茶來。吃了。就送上絕精的果點來。斟上佳釀相待。那少年也愛命兒的嬌容。吃着酒。不轉睛的看他。命兒笑吟吟相讓。飮過數杯。古語道。

\begin{quotation}

三杯竹葉穿心。兩朶桃花上臉。

\end{quotation}

又道。酒是色媒人。那命兒一見他時。恨不得把他就抱上肚子。此時吃了幾杯。淫情大動。鎖不住意馬心猿。他那徒子徒孫都是做慣了。早已走開。只他二人對坐。命兒見他年幼。恐他雖然有相愛之心。而無動手之膽。反拿話先勾引他。笑說道。你這樣個標致少年。在街上行動。不怕把婦女們愛死了麼。那少年可有不知局的。也笑答道。像師太這樣的俊龐兒。難道不怕愛殺了男子麼。彼此相視而笑。命兒按納不住了。拿了一杯酒。拉過椅子。走過來。同他並肩坐着。一遞一口的吃。那少年也就捏腕\endnotemark[78]搵腮。便相攜上床。上邊先做了個呂字。下面就做起串字來。這少年不但貌美。且腰中之物更美。這樣個俏小身材。竟有六寸來長一件的妙具。命兒愛他至極。不忍採他。任他高興。事畢之後。命兒緊緊的摟着他在肚子上。問他的姓名年紀。他說姓富名新。今年纔交十六歲。命兒捨不得放他起來。做出許多騷模騷樣。富新也十分愛他。又風流了一度。方纔穿衣而起。命兒同他攜手並肩共坐。又問他的家世。他將家中只有老母。貧窮度日。盡情相吿。命兒又愛又憐。取出二十金相贈。又在奩中揀出他向日關頭的一根金如意簪。替他關在頂上。道。這是我當年關髮的。今日贈你。切不可抛我腦後。叮嚀他常來走走。富新見他美情。也就領謝別去。此後一月之中。他也來五六次。這富新\endnotemark[79]就是賈文物的內姪。富氏的族\endnotemark[80]姪了。他父親亡後。虧得富氏與了他那三十兩銀子。過後又送了幾挑米幾挑柴與他助喪。賈文物去弔。又折了四兩奠儀。他母親\endnotemark[81]將他的父親殯葬了。將所餘者留爲盤費。自己仍前紡績。\endnotemark[82]以供薪水。他母親也因兒子生得太美。恐他年幼。被人引誘了去做龍陽。走了下流的道路。不容他出門寸步\footnote{孰知他桃花星照命。到底不能免。奈何。}。敎他朝夕溫習經書舉業。服滿之後。正値歲考。叫他去觀場。府縣都取了。到了道考。這宗師是個少年科甲。極喜少年玉筍門生。見他生得如\endnotemark[83]美女一般。問起只十六歲\footnote{這宗師要是福建人。便有些不妥當了。}。已自心喜。看他的文章也還明白。看少美兩個字。竟高取了。這學中朋友見他這樣個靑年。誰不想來親近。但他自幼被父母管敎着。不曾多會人。迂迂板板。從不喜同人談笑。衆人見他如此。疑他是少年老成。倒不敢同他兒戲。就有很羨慕他的。也只好看他兩眼罷了。他自進學之後。他母親就放鬆了些。也就時常出來走走。聽得人說慈悲庵有個絕色的姑子。又如何風流善戰。有美少年到那裡。皆欣然笑納。他一個少年情性。未免也就心動。問了慈悲庵的去處。走了來看看。不意蒙崔命兒相待爲腹上之賓。以臍下之美味相款。且格外垂情。又有朱提金簪之贈。他不但慕色。且又感情。時常走來相看。那日。正同命兒坐着說話。又進來了一個翩翩少年。這人姓司。雙名進朝。年方二十有二。他父親名司導。現任廣東糧道署按察司事。母親金氏。他家有萬餘之富。這司進朝是個獨子。父母珍愛。留在家中照管。他是一個恩監。他生性倒也還豪爽。腹中也還有些墨水。只有一樁毛病不好。別的都不甚愛。只在一個色字上專做工夫\footnote{世上富貴人家子弟。不在此字上做工夫者。能有幾人。}。他的妻子空氏。也是大家子閨秀。生得那身材容貌。也算得一個十全的女子。比他小一歲。那空氏。

\begin{quotation}

雖不能賽古時有名的美女。

也可以算今日無對的佳人。

\end{quotation}

他還四處訪求。娶了兩個美妾。一個姓龐。是揚州人。年方二八。一個是姓鞏。蘇州人。纔十七歲。又覓了四個美鬟。一名雨棠。一名風柳。一名雪梅。一名月桂。都是他朝夕鑽硏的。這四個丫鬟都學會了彈唱。內中雨棠雪梅更覺風騷。司進朝也分外鍾愛。他身邊有了這些家藏美味。心猶不足。尚各處尋覓野食。他又酷好男風。人將他的名字借音而改。都叫他做色精騷。他偶然一日同朋友們談及男色一道。內中一個道。我不知此輩是何肺腸。此事於正妻則不可。旣有妾有婢。那小官有的。他身上也有。不過同一糞窟耳。豈男子者又有別味耶。何必捨此而取彼。眞是捨近而求遠了。司進朝笑道。如此說。兄竟是門外漢。倘如尊言。自古就不該留下這一件名色了。雖男女之味相同。而其趣大相遠絕。難道古時候帝王宮中豈無美女之後庭。而取趙高。董賢。彌子瑕。龍陽君諸人耶。他雖相與了些後庭朋友。每以未遇一殊麗者爲恨。他聞命兒之名。相與久了。命兒因他是個大護法。常有餽遺。不敢過採他。要留下做個耐久朋友。\endnotemark[84]他也甚愛命兒。不時來同他做些樂事。他前在文廟中看迎學的那一日。見了富新。暗詫道。何物老嫗。生此尤物。不覺心魂飛越。無故不好去相親。不想今日在這裡遇着了這五百年風流孽寃。滿臉堆下笑來。彼此揖遜坐下。司進朝說了許多假親熱渴慕的話。又詢及家中尚\endnotemark[85]有何人。在何處居住。富新也一一相答。司進朝聽他只有寡母。心中越發\endnotemark[86]暗喜。命兒忙叫收拾茶果蔬飯吃了。因兩個都是心愛的厚朋。不好偏留。兩人都辭了出來。各自歸家。司進朝想了一夜。想出一個主意。次早就到富新家一拜。且要登堂拜母。回到家。忙吩咐預備下酒飯。不多時。富新來回拜。留在書房小飮。富新要辭了回去。司進朝道。弟極喜相與朋友。久慕長兄之名。不敢造次奉謁。昨得幸遇。故今早滟誠奉拜。又蒙\endnotemark[87]賜顧。豈有空坐之理。弟還有一事相商。屈駕片刻。富新見他美意諄諄。也就坐下。飮酒中間。司進朝道。弟近來爲家務縈心。學業都荒廢了。欲請一位朋友到舍下。彼此切磋\endnotemark[88]砥礪。做一番候\endnotemark[89]場工夫。弟想來。這除非得一知心契友。方纔有益。正無其人。若兄長\endnotemark[90]不棄。肯來賜敎。弟決不敢以異姓相目。竟如手足一般。老伯母一年薪水之費。並衣服等項。都是弟這裡供給。免分兄讀書之心。兄竟長在舍下下榻。或憶老伯母。間回府一看。兄長尊意如何。可肯賜敎否。富新家中貧\endnotemark[91]寒。聽見這話。心中也暗喜。答道。承老兄長雅愛。但弟年輕學淺。不足以談舉業。況弟也不敢自主。還得稟命於家慈。看家慈之意如何。弟再來復命。司進朝道。稟明尊堂。這是自然。要說別的話。就是兄過謙了。吃罷酒飯。富新別了回家。將此事向母親說了。他母親見兒子進學之後。常往外邊行走。正恐他遊蕩壞了。又知司家是富豪鄕宦。不但兒子去。可安心讀書。況又許送盤纏衣服。有何不肯。連連應允。富新次日復了司進朝的話。司進朝大喜。即刻封了二十兩銀子。又送了許多柴米小菜醃魚臘肉之類。擇了個日子。寫了個紅全柬。\endnotemark[92]請他進館。差人送至富新家來。他母親見了。喜出望外。他家升米束柴買了多年。今忽然得了這些東西。\endnotemark[93]眞是陡然富貴。忙忙的收這樣。收那樣。收完了。道了幾百個謝字。又忙叫富新拿塊銀子。押了五百文大錢來。送衆人做勞資。富新到了日期。到司進朝家中來。又是絕豐盛的酒席相待。飮到半酣。司進朝說道。承兄不棄。肯來賜敎。弟想來。你我二人皆無兄弟。何不結盟。做一個異姓骨肉。更覺親厚。富新道。弟一介寒儒。兄長簪纓世胄。何敢仰攀。司進朝道。兄一個讀書君子。如何也作此市井之見。古云。斯文骨肉。同在斯文一道。即如骨肉了。何況你我如此相契。富新道。兄旣不鄙寒賤。弟敢不從命。司進朝叫家人預備香紙。二人次早焚香設誓。異日寒盟。定遭惡報。拜畢。富新又拜了盟兄。兩人攜手同到內書房。這是外人到不得的地方。與上房咫尺相連。只一門之隔。司進朝道。我們如今旣做\endnotemark[94]了弟兄。便是一家了。賢弟今晚就在此下榻罷\footnote{富新到此。雖是身入泥塗。司進朝也是引賊入室。}。因叫小子道。快看酒來吃。隨即掇上酒肴。各飮了幾杯。飯罷之後將午。又叫拿了果碟來吃酒。吩咐小子道。傳到上邊。去叫雨棠雪梅帶了樂器來伺候。你們都出去。不多時。兩個丫鬟來了\footnote{這是兩個迷魂陣的先鋒。}。富新舉目一看。好一雙標致的艷婢。都是桃紅紗衫。石靑露地。紗比甲。月華裙。家常吊着桃兒。戴着茉莉花。金簪珠墜。下邊微露尖尖小脚。穿着白紗褶褲。大紅平底花鞋。不覺那魂靈兒竟鑽到他兩人身上去了。司進朝向他二人道。這是我的盟弟富相公。叫你們出來。每人唱個曲兒。敬一杯酒。那兩鬟見了富新這樣個美少年。也十分心愛\footnote{先是命兒見了愛。司進朝見了愛。此時兩鬟見了愛。後日空氏諸婦見了愛。至於流賊見了也愛。沿(冶)客(容)誨淫。不獨女子爲然。即如紅顏薄命。亦不獨婦人。自古來美男子雖多。或有知者。有不知者。如潘安衛玠。則無不知其美名耳。衛玠以怯病死。潘安以殺死。男子紅顏亦皆薄命。}。雨棠忙斟了一大鍾酒。笑吟吟雙手遞上。富新忙立起來接。雨棠用尖尖指甲將他手背輕輕一掐。兩人相視微笑。雨棠彈弦子。雪梅拍板。雨棠露皓齒。吐嬌音。唱一隻小曲。道。

\begin{quotation}

雨初霽。海棠嬌。賽過胭脂鮮俊。俏佳人摘一枝。試問郞君。你看這花容勝。還是奴容勝。郞君故意道。花容好。佳人聽說怒生嗔。將花揉碎灑郞身。夫君呵。今夜你就同花去寢。我再不與你相交頸。

\end{quotation}

富新聽得骨軟筋酥。見他唱完了。忙把酒飮乾。雪梅又斟上一鍾。他彈琵琶。雨棠掌板。雪梅開檀口。放嬌聲。唱道。

\begin{quotation}

雪裡梅花早放。南枝春光先透。忙向園中折一枝來。最愛香幽。試問丫鬟。我比梅花誰淸誰瘦。丫鬟說道。梅花雖瘦無煩惱。姑娘你。憔悴了花容爲郞愁。學只學白梅花。冰淸玉潔的無憂。他開放時。獨占名園。百花魁首。任着那浪蝶狂蜂去尋花問柳\footnote{二鬟所唱即以己名爲曲。妙甚。}。

\end{quotation}

富新聽他音韻悠揚。雖非繞梁裂石之音。那歌喉婉囀。幾不能自持。腹中又有了幾杯。忘其顧忌。見他兩人如左瑤草而右琪花。東顧西盼。兩隻眼睛直射在他二人身上。司進朝看了他那個樣子。忍不住暗笑。又見他臉上如桃花瓣相似。越增心愛。那兩個丫頭也望着他微微含笑。又敬了數杯。司進朝笑着道。富相公不是外人。你們把風流的曲兒再唱一個。不妨村俗。他二人笑着。同唱了一個三調彎兒。

\begin{quotation}

俏寃家。偶來到園中觀眺。猛見那花茵上一對貍貓。那貍貓不住貓\endnotemark[95]貓亂叫。公貓咬住母貓的頸。母貓回頭望公貓。一根竹子節節高。送與寃家做管簫。口兒噙着。口兒噙着。十指尖尖摟抱着腰。小嬌嬌喘喘氣兒再一遭。左眼兒觀。右眼兒瞧。觀定貍貓鸞鳳交。貍貓調情人心動。不好了。再看再看一會貍貓。俏寃家。你的銀紅褲兒濕透了。

\end{quotation}

那富新聽得只是嘻嘻笑。司進朝一來愛他的那個騷態。二來要引動他的春心。說道。你\endnotemark[96]們前日學的那疊落金錢有趣。可唱與富相公聽。\endnotemark[97]再敬一鍾。二人斟上酒。又唱道。

\begin{quotation}

花園裡去採花。花園裡去採花。㖿㖿哎哎喲。進得門樓撞見他。我的寃家㖿㖿。雙手兒摟抱到那花㖿花\endnotemark[98]枝下。奶頭兒在手裡拿。奶頭兒在手裡拿。㖿㖿哎哎喲。舌尖兒在口中咂。我的寃家㖿㖿。旣然不肯。你給我摸㖿摸摸摸罷。

\end{quotation}

唱畢。又每人奉了一鍾。富新也有了幾分醉意。掌上了燈。纔散了。富新這一夜翻來覆去。達旦無眠。將閉上眼。不是聽得絃索響。就隱隱像他二人唱。又驚醒來\footnote{此數語寫得入情入妙。非身歷者不知也。余幼時入學。圍棋無日不下。到臥時。滿眼皆是棋子。又驚醒來。不過此同一理。}。那司進朝帶着兩個丫頭進去。到了密室。遂將心愛富新。故騙他來家。要想採他後庭的那一朶木樨花。恐他不肯。要他二人去做個香\endnotemark[99]餌。引誘他動了心\footnote{起心原自不良。後日之妻妾被淫。亦難獨罪富新。言悖而出者。尚亦悖而入。又何況於此耶。}。慢慢遊說他。若肯了。許他二人交換。若事不成。倘先有私弊。決不輕恕。這兩個丫頭先見了富新。也眼中冒火。正想怎得這妙人兒相伴一宵。也不枉人生一世。今聽了主人這話。奉此美差。歡喜非常。\endnotemark[100]滿口兒應承道。相公放心。我兩個包管成了你的美事。司進朝心想富新。也動了火。就拿他二人要大弄。一來權當做他\footnote{到底當不〔得〕他。若當得他。又可不必尋他矣。}。二來做開手賞賜。司進朝次日又叫裁縫替富新渾身徹底做了兩套紗紬衣服\footnote{這也是蜘咮(蛛)絲的。}。連鞋襪都換了。更覺好看。司進朝同富新名說讀書。但司進朝要支撐門戶。親友家冠婚喪祭的事。並人情來往。都要他親身去應酬。回來家。就想陪吃酒。叫了丫頭來彈唱侑觴。富新一個少年。到了這個局中。也不過把書翻翻。那裡還看得下去。一心只想着那兩個歡喜的寃家。也巴不得司進朝來共飮。好同他二人親近。數日。都熟厚了。司進朝飮酒中間。或推有事出去。讓他們個空兒。好施前計。那兩個丫頭奉了主人之命。要成就主人之事。又是爲\endnotemark[101]着要成就自己的好事。豈不上心。見主人去了。便走到富新跟前。挨挨擦擦。這個讓酒。那個唱曲的奉承他。或互相調笑。富新先雖愛極。尚還不敢放膽。見他二人先來賜顧。可還肯做那假道學。也就涎着臉。先還用口說笑。漸而便用手捏腕摸胸。久之。連接唇咂舌。把那妙處都撫摩起來。二人俱笑而不拒。只是要做實事。他二人便推辭說道。你愛我們。我們難道不愛你的。但恐主人知道。不敢奉命。把個富新急得要死。常常求吿。他二人只以主人爲辭。一日。司進朝坐了一會出去。富新拉他二人到跟前。笑嘻嘻把肉具取出。脹得挺硬。向他二人道。你二位縱不憐我。只當可憐他。你看看。差不多要脹裂了。他二人一見。粗而且大。比主人的放樣了許多。心中喜得劈劈亂跳。眼中火星亂飛。說道。只有一個苦肉計可以做得來。你可肯不肯。富新道。你有甚麼妙計商量了看。雨棠道。我家相公酷好\endnotemark[102]男風。你要捨得後邊的那一件。就可以換我們前面的這兩件了。富新紅着臉道。一個堂堂丈夫。這事如何行得。豈不叫人知道笑罵麼。雪梅笑着道。而今世上半是此類。恐笑罵不得這許多。雨棠見說他不肯。心中一急。眉頭一蹙。又想了一想。道。你若肯依從了。還有一百二十分的好處呢。不但我二人屬了你。我家奶奶同兩位姨娘都有絕世之容。你若做了我們的主人外眷。我二人替你做個紅娘。引誘主母姨娘。他們若見了你。焉知不做了你的外室。你捨了後面的一個圓眼。就得了我們前面的五個扁窟。你便宜多了。你想想。好不好\footnote{此計出自雨棠。今日成就他二人。故是功之首。異日淫穢他一家。難免罪之魁也。}。那富新聽了這些話語。也有些顧不得。便道。話雖如是說。就算着依了你們行。一個朋友家。怎麼就好做這樣勾當。他兩個道。什麼相干。你果然肯。多吃幾杯酒蓋着臉。就不妨了。你對相公說要我們同在一處。大家混弄起來。越發不覺。只做過頭一次。後來還怕什麼。富新想了一想。實在心裡忍不得了。說道。罷\footnote{這一個罷字。古今來。千萬萬的人坑在內中。}。講不得爲你兩個。我捨了身子罷。他兩個見他依允。心花俱開。跑去向主人報功。司進朝歡喜欲狂。忙走進。向富新深深一揖。道。蒙賢弟厚愛。生死難忘。富新紅着臉笑道。弟不惜賤\endnotemark[103]軀以奉兄。兄亦當以此二美贈我。司進朝忙道。賢弟若愛。我何敢惜。忙叫取了酒來。斟了一鍾。雙手遞與富新。道。敬此一巵。願永諧盟好。富新也笑着接過飮了。司進朝就命兩婢挨着富新左右坐下。猜枚豁拳。飮夠多時。都有酒興了。富新被這兩個寃家挨在身傍。那裡還忍得。說道。酒止罷。司進朝已十分興動。不好催他。見他說止。忙道。旣如此。賢弟就請安寢。向兩個丫頭道。你兩個陪富相公同睡。富新也有了七八分醉意。一手摟着一個。同到床上。三人脫得精光。富新就把雨棠弄將起來\footnote{因定計是他。賞其功也。}。司進朝也脫了上床。抱住富新。笑道。得罪了。將他糞門並自己龜頭都抹濕了。款款頂入。司進朝的陽物只有一虎粗細。四寸餘長。富新雖係初時開荒。也不大覺受創。弄了不到半個時辰。司進朝早已吿辭。雨棠乍遇他這件偉陽。又有司進朝在他背後抽拽。兩人之力下杵。已被他弄丢了兩次。富新見司進朝完了下來。他探起身。看那雪梅。兩頰鮮紅似火。兩眼汪汪滴水。急得那樣子。又好笑。又可憐。忙將他放倒。大肆抽弄。有幾句說他四人。道。

\begin{quotation}

五體投蓆。腹背相攻。馬走吳宮。夭桃鬥紅。俱笑日兔奔月窟。摽梅含翠共搖風。搖風嬌影隨流動。鵲繞枝棲。笑曰。香浮隔岸。豐鴻來渚。道。瑤鳥鸞翔。衝破玉壺開妙竅。芳叢蝶亂。潛游金谷覓花心。此中適酣。彼亦大樂。兩男暢美於榻中。二婢消魂於枕畔。

\end{quotation}

兩人幹了一會。也就事竣。此後這兩個丫頭朝夕陪伴着他。總不上去。司進朝雖好色而力不及。旣外邊同富新盤桓。又要顧內裡去應付。三五日纔同富新弄得一次。却便宜富新同這兩個騷精每夜行樂。間或日間他們偶然高興。或遇着司進朝來。便做那柳\endnotemark[104]穿魚的解數。富新但同雪梅雨棠交媾。弄得他二人爽心的時候。便以空氏同二妾的事相懇。要他踐前日之言。他\endnotemark[105]二人要富新盡力。也極力應承。許他緩緩圖謀。造次不得。且說那司進朝這人。也是無美不愛。眞算得個色精。他家中雖算上了富新。過幾日定去看看崔命兒。賞鑒他的妙牝。命兒見富新許久不到他庵內。心中時刻想念。偶然同司進朝說話之間。問他一向可曾看見富新。司進朝不好說在他家做了男妾。但道。我約他在我家同讀書呢。命兒甚喜。托他下次來時約他同來走走。或叫他自來亦可。再三諄囑。司進朝應諾回\endnotemark[106]家。向富新說命兒十分記念。約他去走走。他也不答。後來司進朝要去看命兒。約他同往。他因戀着雨棠雪梅。日裡要做一番生活。決不肯往\footnote{頭一個。負心於命兒。}。司進朝怎好拉了他去。屢次如此。只得將約他不肯來的話復了命兒。命兒暗恨道。這樣一個少年。原來如此負心短倖。我初會你。就不惜厚贈。今日約你一會。都不肯來。這等無情無義的人。會他也無益。遂把他撇於腦後。再說雪梅一日有事上去。空氏叫他到跟前。問道。你\endnotemark[107]們兩個。相公爲甚麼叫了出去。況且相公又不常在外邊過夜。是甚麼緣故。雪梅只是笑。空氏再三追問。他纔把富新的話細細吿知。空氏道。這人怎樣個美法。你相公就肯把你兩個換他。雪梅屢受富新之托。借這意兒慫恿道。若說模樣。果然是少有的。不要說男人。若女人中趕得上他的還少呢。此時相公不在家。奶奶何不去張張。空氏聽說得高興起來。就同着雪梅往外走。剛到院子裡。迎頭遇見龐氏。問道。奶奶往那裡去。空氏又不好回來。又不好吿訴他。笑說道。你也同去看看。到了外邊。一個人也沒有。悄悄走到窗下。往裡面一張。見那富新之美。心中私愛是不消說。又渾身赤露。如一塊無瑕白玉。竟像放光的一般。他把個雨棠按在一張椅子上伏着。挺着六寸長多的一個大物。隔山取火。狠力着亂搗。搗得那雨棠受用得像臨死掙命的樣子。喉中格格有聲。四肢亂抽亂扭。\endnotemark[108]空氏龐氏看到這種光景。頭髮根一麻。遍體酥軟。幾乎癱在地下。見他兩人事完。富新拔出陽具。仍然堅舉。粗而且長。空氏龐氏益發酥了。心中雖戀戀不捨。又怕他出來看見。只得扶着了雪梅。一步步掙了上去。空氏到了房中。悄悄向雪梅說。叫他做媒。匣中取出個鴛鴦玉墜。床裡拿出一隻鳳頭繡鞋。用一條大紅縐紬汗巾包了。汗巾頭上還有一副金三事。一個同心盒。送與他做表記。又叮嚀了幾句話。若遇相公夜間出門不在家。千萬約他進來一會。雪梅接了藏好。纔走到院子裡。龐氏點手叫他到房中。手上攄下一對比目魚的金戒指。身上脫下一件喜相逢小紗衫。再三央及他轉贈。約他遇巧進來。雪梅也袖着。到了書房。向富新道。我纔上去。兩個妙人兒托我帶了幾件東西來送你。看你怎麼謝我。遂將幾種寶貝取出。富新一見。喜到百分。笑道。好姐姐。這是誰送我的。雪梅道。好自在性兒。輕容易就吿訴你。富新道。你不過是刁難我索謝。等我來奉敬。遂將他抱到床上。脫了衣褲。奮力謝了一謝。伏在肚子上。又問道。這端的是誰給我的。雪梅道。我纔不在這裡。你同棠姐幾乎把椅子都搖散了。這只\endnotemark[109]算得補我的數。謝禮我還不曾領情呢。富新笑道。罷了。說不得了。我再奉申謝敬。又竭力弄了一陣。雪梅纔吿訴他如何空氏問話。如何誘來張。恰遇龐氏一齊同來。怎樣張見他兩人幹事。回去叫送了這東西來。改日有空相約。把個富新喜得心窩亂癢。把住他親了十來個嘴。纔要下來。雪梅一把摟住。道。你不謝謝媒就想跑。富新道。該謝。該謝。正要抽動。雨棠將富新抱住。道。要謝先謝我。他若不見我們在這裡演武。還未必就動心呢。我的功有七分。你只有三分。如何僭得我的先。雪梅道。積陰隲的姐姐。你讓我這一會兒。我只略領領他的謝意。酒醉後來客。後邊有多少都讓你就是了。雨棠笑着放了手。富新又抽了一會。雨棠見雪梅像是丢了。就把富新生拉了下來。二人高興了多時。各整衣服起來。雪梅又說。主母同龐姨再三囑咐。東西要收好。若被主人看見。大家都有不妙。富新道。此處如何藏得。我送到家中收了再來。忙忙的回去收好。他母親要問他話。只匆匆答了兩句就跑來了。過了幾日。司進朝人家請去吃戲酒。有一夜不回。空氏得了這個空兒。叫雪梅約進富新來。以完心願。掌燈時。富新雪梅進來。到了房中。見空氏獨對銀釭。手托香腮坐着。忙近前一揖。空氏雖約了他來。但他一個少年嫩婦。忽一個驀生的男子走到身邊。而且還要做那件事。由不得滿面嬌羞。側身還了一福。\endnotemark[110]低頭不語。富新上前攜着他的嫩手。到燈前細看\footnote{第二。負心於司進朝。}。燈下看佳人。越覺美貌。情興勃勃。一把摟過脖子。就要接唇。空氏微微含笑。把臉略扭。富新越覺魂消。只見他。

\begin{quotation}

一段嬌羞。百般騷浪。一段嬌羞。兩頰微紅。雖是含羞而却帶喜色\footnote{話刻毒。}。百般騷浪。雙眼斜窺。雖作嬌態而實是勾魂\footnote{說透淫婦人心髓。}。面上似笑而非笑\footnote{寫生。}。口中欲言而不言。粉頸微扭幾回。硃唇略抿數次。知是他春心發動。難禁我淫興攻來。

\end{quotation}

他二人也無可扳談。相攜上床。富新替他寬衣解帶。他惟閉目徉羞。脫光了。富新在燈光之下將他渾身細細一看。宛如一團瑞雪。由不得遍體酥麻。怎見得他的妙處。有個七字令贊他道。

\begin{quotation}

妙。好。女喬。馬蚤。柳眉彎。櫻\endnotemark[111]桃小。眼波淫淫。腰肢嬝嬝。尖尖玉指柔。窄窄金蓮小。\endnotemark[112] 酥胸嫩乳團團。玉骨冰肌皎皎。動人情處不能誇。紅溝微綻眞奇寶。

\end{quotation}

這贊他不盡。還有幾句道。

\begin{quotation}

眼兒餳。\endnotemark[113] 唇兒笑。髮兒烏。容兒俏。乳兒僅僅一捏。腰兒剛剛一抱。腿兒白白光光。脚兒尖尖蹺蹺。腹兒軟軟如綿。臍兒小小一竅。看到胯下那一件。肥又肥。緊又緊。紅又紅。紫又紫。滑又滑。香又香的美物。眞個是盡皆佳妙。

\end{quotation}

富新看得興致倍濃。一下搗了進去。不歇氣盤桓了有半個更次。空氏乍經大敵。嬌聲嚦嚦。嫩體搖搖。富新如在仙界中快活。越加憐愛。歇了片時。又見那空氏口中微有聲息。腰肢咯咯款扭。富新愈覺興豪。越加用力。不多時。只見他渾身打了個寒\endnotemark[114]噤。用\endnotemark[115]手摟過富新脖子。度過舌尖來。富新知他樂極了。含咂了一會。空氏就將他緊緊的摟了兩摟。臀兒向上湊了幾湊。富新知他興尚未足。又大肆馳驅。盡力衝突。猛聽得空氏叫了一聲。哎呀。罷了我了。癱於枕蓆之上。富新見他這樣子。也不覺渾身一麻。一洩如注。伏了片刻。互相把舌尖咂了咂。下來相摟相抱。同臥了一會\footnote{他處贊美婦人只數句而已。此一段贊空氏。娓娓不休者何故。見得有此等美婦。司進朝猶不知足。而尚漁男色。致妻妾爲人所淫。亦自取之耳。}。這日。龐氏也知司進朝不回家。再三托雪梅邀富新赴約。富新同空氏弄了這場。心中記掛着龐氏。假說的司進朝回來要出去。空氏也心滿意足。體乏要睡。就放他起去。再三囑定後期。富新穿衣出來。又同雪梅悄悄到龐氏房中\footnote{第三。負心於空氏。}。他早已睡下。富新上床。掀開被摸他。尚穿着衫褲。替他卸下。自己也脫了。就弄將起來。龐氏的姿容雖不及空氏。而被底風流過之。也謅了幾句。說他二人這番光景。

\begin{quotation}

龐氏腹上馱着個美男子。一杵中撐。兩膝跪榻。忙忙的橫舂豎搗。富新身上壓着個俊嬌娃。兩片分開。雙足高蹺。急急的上送下迎。兩張嘴正正相親。四隻臂緊緊互摟。富新道。俏心肝。我愛他(你)百種風流。你的俏心兒。切莫要又向着別人去使。龐氏道。小寶貝。我同你千般恩愛。你的寶貝物。〔千〕萬不宜另做負心的奴。上一個笑吟吟。思舂破了他內中嫩皮。方纔意足。下一個喜孜孜。欲〖扌歪〗斷了他腰中硬棍。始覺興闌。

\end{quotation}

兩人直到\endnotemark[116]四鼓。方纔別去。富新見龐氏與他同歲。枕蓆上別有一種風情。更覺嘉喜。此後但是有空。便來同他兩個應酬一番。久之。鞏氏同風柳月桂也知道了。如何放得過。那一夜。司進朝有一個父執雪給事七十整壽。他送了禮去赴戲筵。富新同空氏弄了一度出來。就到龐氏處。兩人正在如此云云。鞏氏打聽得知。走將進來。一手掀開帳子見了。說道。好好。相公不在家。你們做的好事。我要不叫破了。後來連我也拉在渾水裡頭沒(麼)。富新驚得連忙拔出爬起。龐氏笑道。好姐姐。你不要假撇淸了。也來大家樂樂罷。鞏氏道。儂是弗稀罕事個。渠弗要拖人下水。龐氏知他口硬心軟。向富新使了個眼色。道。你就不求求姐姐。他肯依麼。富新忙下床。赤條條跪在地下。鞏氏見他渾身雪白。如月宮玉兔一般。腰間橫着一根玉杵\footnote{此玉大約紫玉的。}。一跳一跳。由不得都渾身軟了。笑道。要不看你的面。儂這一〖口么〗喝起來。大家子弗成。富新見他口鬆。起來一把抱住。放在床上。就去扯褲。鞏氏道。儂弗聲張罷了。你\endnotemark[117]倒敢做格樣事。嘴裡說着。任憑他脫下了。就弄起來。上身衣服也被龐氏替他脫光。富新便將他弄了一陣。猛擡頭。見風柳月桂站在床前。鞏氏是同他兩人約了來的。說道。儂罷哉。你同渠兩個耍子一歇。做個大家歡樂。富新見他兩個模樣也還不俗。就下床拉他二人按在\endnotemark[118]春凳上。每人都見了見。此後空氏同他六個人都做了一路。只有司進朝尚在鼓裡。一絲毫不知。還時常送柴米送盤纏與他老母。也混了年餘。忽一日。廣東有家人到來報喪。說老主在任病故。夫人差來接小主去搬靈。闔家大哭了幾場。一門掛孝。司進朝要去搬父柩。接母親。遂將家務事內托空氏。外托富新。又吩咐家人。小心聽服富新使令。如同我一般。不然回時重責。又囑雨棠雪梅好生陪伴他。纔起身去了。這時家中去了個大貓。該這些老鼠出來成精了。富新同這幾個婦人公然明弄到夜。夜睡到明的大樂。竟像親夫妻一般。毫無忌憚起來。那空氏也恐家人有口聲。將家資任富新揮霍。富新拿出那田完買齊的手段來。把不心疼的錢。家中男婦大小都沾厚惠。又拿出柳盜蹠的本事來。暗盜了這許多銀子回去。他這幾個男婦都弄昏了。也不想一想。這項銀子將來司進朝回來。作何開銷。富新也竟把司進朝的家私。當是自己的。任意施爲。毫不顧惜。這衆家人又得了重賄。心中不勝感激。背地念他一個權印的主人。比正經主人如此的厚恩。有幾個老人家賄雖受了。却心中不忿。道。我主人好容易掙來的家私。却被他如此撒漫。因一個是主人。一個是主人的盟弟。且又是極相契厚。況主人臨行之時。又切切吩咐要着實小心。故此不敢多言。他們也樂了有幾個月。司進朝回來了。將父親棺木寄在聚寳門外普德寺中。同母親到家。親友來弔唁者絡繹不絕。也無暇同富新相敍。富新同衆婦人熱鬧了幾個月。今日\endnotemark[119]一旦分開。那雪梅雨棠因老主母來家。自然要上去。只他獨自一個孤孤悽悽在書房中。還想司進朝來同做他那背後的生活。別的婦人不敢望了。得這兩名美婢來幹幹前面的事務。以消岑寂。不想過了幾日。司進朝事體稍暇。那兩個老人家將家中之事細細稟明主人。司進朝悔之\endnotemark[120]無及。去查點家資。少了三千餘金。問空氏。他無言可應答。只說家中盤費了。問作何項。使用許多。但睜目張嘴。頭低面赤。不能復答一語。司進朝同他大鬧了一場。他母親來問何故。司進朝又不好詳說。只說媳婦在家。把銀子不知花往何處去了。那金氏夫人把媳婦也就數說。責備許多不是。司進朝又叫那老人家出來辭那富新。道。家中老主沒了。小主要守制。也無暇讀書。富相公請回罷。我家相公要親自說。因無顏來相見\footnote{反說得妙。}。有那感激家人前來戲(獻)勤討好。將主人上邊鬧吵早即吿訴他了。他還癡心以爲司進朝或再不能忘情於他後庭。還想爲入幕之賓。今見家人來辭。知站立不住了。也有些心慚面愧。只得歸家。這却是古語道。

\begin{quotation}

分開了連理枝。拆散了鴛鴦伴。

\end{quotation}

司進朝將他父親安葬後。見這幾個婦人如眼中釘一般。由不得生氣。空氏係正室。不便驅逐。把兩妾四婢都叫媒人賣了。他待這空氏總無一毫善色。無一句好言。指東瓜罵葫蘆。指和尚罵禿驢。無一日不誚\endnotemark[121]帶他幾句。空氏忍氣吞聲。可還敢說一字。料道情人\endnotemark[122]今生不能見面。常要領敎丈夫的幾句臭罵。終日鬱悶。不久氣結而亡。那富新盜了司進朝之物。約有三千餘金。聞得他家賣妾賣婢。他愛龐氏風騷。雨棠雪梅是他久契。自己不敢出名。托人轉買來家。做了一妻二妾\footnote{第四。負心於鞏氏風柳月桂。}。却得了自在受用。叫做個。

\begin{quotation}

若要人不知。除非己莫爲。

\end{quotation}

久而久之。他們這些事轟揚得人人知道。雖怪司進朝好色所致。但這富新受他多少恩惠。他雖辱身。係他情願。並非司進朝強拿硬做。且酬之以二婢。也就罷了。決不該淫他的妻妾。盜他的家產。可謂負心之至。知者無不痛恨。司進朝父親有一個老友。做過一任給事。吿老在家。他姓雪名芳。是個極義憤的人。專好替人雪忿報仇。他也知道這事。新文宗是他的會場同年。他相會時。將富新的事一一說知。文宗訪了他一個劣行。將衣巾褫革。重責十板逐出\footnote{惜哉此股。此文宗大殺風景。}。富新無顏在家。拿了數百金到北京。做了個黑豆跳。又名飛過海。又叫活切頭。冒名頂替。叫做傅誼。得了陝西西安府富平縣典史。回家買了一房\endnotemark[123]家人。同了母親妻妾。雇了馱轎騾子去上任。剛過了潼關。不想遇着十來個流賊。縱馬蜂擁而來。他母親家人騾夫不必說。喪於刀下。擄了三個婦人。又要殺富新。內中一個賊酷好喜男風。混名叫做毛坑蛆。忙止住道。兄弟不要動手。那三個老婆給你們。這個小子留給我罷。那賊便收住了刀。有三個賊便將三個婦人抱上馬。同騎着揚鞭如飛而去。這些賊的規矩。十個人一架帳房。有一個小旗管領。那六個賊把馱轎棄了。拿騾子馱上了東西。翻\endnotemark[124]上馬。趕着頭口。放開了轡頭。飛馬攆去。只剩毛坑蛆押着那富新公同回營。多時方到。進了帳房。富新舉目看時。三個婦人已脫得精光。九個賊也渾身赤露。輪流了大弄。龐氏雨棠雪梅受用得嘻嘻哈哈。哼哼喞喞。全無一點悲苦之色。有幾句說這夥賊同這三個婦人。道。

\begin{quotation}

這賊人身逢少女。猶如餓虎吞羊。那淫婦心愛壯賊。好似渴龍得水。貪淫婦性情騷浪。本自愛耍貪歡。好色賊手段高強。眞是能征慣戰。糴的糴。糶的糶。沒一個肯將服輪(輸)。往的往。來的來。都一般辛勤出力。雖然小典史曾爲魚水之歡。怎似大強盜善作衝鋒之戰。

\end{quotation}

這毛坑蛆見他們高興。笑道。你們好快活。老子也該受用了。拉過富新。按在鋪上。扯下褲子。露出光臀。雪白如玉。嘖嘖贊道。老子做了這幾年的賊。也沒有肏過這樣好屁股。今日好造化。一面誇着。也不着一點唾沫。挺着鐵硬的大㞠子。往糞門裡就頂。狠命的二三下。搗個盡根。富新雖被司進朝破過。一則他陽物不大。二則有許多愛惜之心。不但用上若許津唾。而且輕輕款款的弄。怎當這賊拿出強盜的力量。且又蠻舂混搗。如何受得。疼得屁股亂扭。毛坑蛆道。我的兒。不要動。你好好兒的。我給你牛肉燒刀子吃罷。明日還給你馬騎。要不依我。我就是一刀。一面說着。大肆衝突。富新雖覺難禁。心裡尚思逃命。恐他行凶。只得咬牙死忍。搗了好一會。方纔事畢。那賊不住道。快活。快活。老子樂殺了。纔拔出來。富新再看那三個婦人。還同衆賊頑笑着弄呢。心中恨着道。婦人水性楊花。一至於此。我爲了他們。做了這些負心的事。今日落在這羅網中。他們各圖歡樂。連一毫顧惜我的心腸都沒有。這却難怪。三婦如何顧惜法。我何苦〔做〕這壞人。心下也深自懊悔。但已無及。諺云。不到黃河心不死。人不到盡頭。尚不知悔。有個劈破玉道。

\begin{quotation}

問君家。你緣何不到富平任。原來是天做對。不作(佑)你這負心人。把合家全結果在這賊一陣。妻妾爲賊嬲。尊臀被這賊途。這是你負心的下場頭。也勸世人。還是要好心纔把穩。

\end{quotation}

又有多時。衆人方纔完事。衆賊自有乾兒義子替他們煮飯燒肉。熱酒早已齊備。那九個賊擁着三個婦人。拿了一大盤牛肉。一餠(瓶)燒酒吃着頑耍。叫那毛坑蛆道。哥。你也大家來頑頑。毛坑蛆道。你頑你們的。我同我這兒子在一搭裡坐。也拿了些酒肉放在面前。把富新抱在懷中。看了看。心愛得了不得。親了個嘴。把酒喝了一鍾。遞在富新嘴上。道。乖兒。你也吃一口。富新那裡吃得下。推辭不飮。他大笑道。老子方纔肏得你不快活麼。你不吃。我自然久了的。你吃些。大家助助興。晚上我包你有半夜受用。富新先已被他弄得難受。聽見這話。知道晚上要受他的大創。不如醉個半死。捨了身子。憑他去罷。再看三個婦人。歡歡喜喜。同着衆人大飮。那雨棠雪梅又暢(唱)個曲兒侑酒。喜得衆賊笑着不住連聲贊美。這個抱住親個嘴。那個伸手到胸前摸摸奶頭。他三人毫不羞拒。富新心中又氣又恨。那毛坑蛆自己喝了幾鍾。又向富新道。乖兒。你看你的老婆倒那樣老練。你反這樣嫩氣。強叫他吃。富新也一氣喝了。那毛坑蛆大喜道。好乖寶貝。再一鍾。富新也吃了。毛坑蛆喝了一會。將富新的褲子褪下。不住撫摩他的嫩股。摩興翹然。把富新推起。頂入糞門。按他坐穩。抱在懷中。一遞一口吃着幹。衆賊看着笑道。哥在那裡又弄起來了。難道我們的本事不如他麼。向三個婦人道。你們快些吃碗飯。我們也動手。他三人道。飯我們是\endnotemark[125]不吃了。下語未曾說出。暗含着我們也弄罷。衆賊笑道。說的有理。吃飯甚麼要緊。我們趁興動手。忙搬去盤碗。大家一齊脫光。他衆人好弄。這一弄。足弄到刁斗三敲。方纔歇息。一連數日。衆賊也不論晝夜。遇興即弄。富新也被弄熟滑了。雖不覺得吃苦。但自己受用慣了。那裡禁得〔這〕等狼藉。滿心想得空逃了。一日。衆賊糧食完了。要出去打糧。因吩咐富新道。你看着帳房。這三個婦人雖是你的老婆。如今是我們的人了。我們不在這裡。你若瞞着同他們偷弄了。我們回來試驗出來。那就顧不得大哥惱。我們就一刀搧了你。毛坑蛆道。我的這個寶貝。比女人還嬌嫩害羞呢。他肯幹這樣的事。因摟着富新親個嘴。道。我去了就來。你不要囗我。大家騎馬而去。他們的乾兒養子都收拾口袋。翻上剗馬。跟着去了。富新見左右無人。問三婦道。我爲你們。今日到了這樣下場頭。你們一點〔都不〕顧惜我。成日歡笑快樂。我當日同你們何等恩情。一旦付於流水。你們就負心到這樣地步。良心也過得去麼。那三婦人一齊放下臉來。道。當日好好的在家罷了。誰叫你想做什麼官。帶累我們到這受罪。我們不抱怨你就罷了。你倒來抱怨我們。你看着我們這樣頑笑。不過是假道哄他們。還不知我們心裡怎樣苦呢。像吃了幾斗黃連水一般。富新道。這話難說。我見你們一頭拿來時。就一點愁苦也沒有。後來弄的那個樣子。你們自己也不覺得好不肉麻難看\footnote{富平縣寶村。一日土賊竊發。有一家夫婦二人。躱避不及。其夫鑚草垜內。妻爲三四賊所獲。按住姦淫。其妻哭喊叫罵。賊亦不顧。弄畢而去。賊平後。其夫欲休此妻。妻曰。我被賊强姦。非我之罪。其夫道。姦我不怪。我在草內看得明白。弄時你那屁股不住往上就。甚實難看。其情可惡。其妻無辭而被出。}。三婦道。我們各人的苦。各自知道。他(你)不聽見他們動不動就要殺。螻蟻尚且貪生。好死不如惡活。只得哄着他們。混一日是一日罷了。富新道。趁他們今日不在。我同你們逃了去罷。那三婦一來怕死。二來心裡那肯捨得去。說道。這樣大的營盤。我們鞋弓襪小。那裡有本事走得出去。與其被他們攆上殺了。撂得現天現地的。不如死在這裡罷。我們看那個人待你的情也不薄。比衆人待我們還厚呢。他方纔臨去還捨〔不〕得你。我們勸你將就住着罷。我們雖不是你的人了。到底是一塊土上來的。在一處也還親熱些。富新聽了這話。氣忿塡胸。話都說不出來。又恐躭誤了工夫。在那賊的囊中尋了些銀子。帶在身邊。拽開脚步而走。走了數里。見到處都是賊營。不知打那裡出去。只得亂撞。正走着。\endnotemark[126]只聽得後面喊叫。你是那營逃的人。不要走。富新當是那賊趕了來。知道性命難保。吃了一大驚便走不動。回頭看時。另是一個人。膽略壯了些。那人追到面前。喝道。你想逃往那裡去。富新頗有急智。他在賊帳中住了幾日。知道他們的營頭。答道。我不是逃走。我是右營左隊裡的人。我主兒叫我去打草。那人將他一看。富新本生得標致。又被一嚇。臉上通紅如兩朶桃花。那賊大喜道。我不信這話。你這樣個美人兒。肯捨得叫你去打草。又沒有馬匹鐮刀。明明說謊。我不管你是走不是走。且隨我回去。解下馬韁繩拴了。帶回帳房裡來。富新一看。也有八九個人在那裡豁拳吃酒。這個賊說道。行動有三分財氣。今日鬼使神差。我坐不住。出去走走。不想得了這個妙人兒來。衆賊一看。大喜道。造化。造化。大哥且吃一鍾賀賀喜。你就先上。我們托哥的洪福。大家嘗嘗美味。那賊笑道(着)一手拉着富新。坐在地下。一面吃着酒。一面看富新的龐兒。贊道。任你好婦人女子。有我這妙人兒標致麼。我耐不得了。且幹了再吃罷。就把富新按倒。剝光了。抱着屁股。弄將起來。富新此時見這十個惡漢子。知道此身斷不能活。嘆了一口氣。想道。我當日負了司兄。到今不但負了老母。且自身受報若此。一口氣往上一攻。遂昏昏迷迷。竟不知覺。過許久。漸漸醒轉。糞門中脹疼得要死。似裂了一般。一個賊還在背上搗呢。多時事完了。他如死人一般。身子動也動不得。伸手摸摸糞門。陽精津津流出。臟頭帶出數寸。心動悲慟。嗚嗚啼哭。一個賊怒罵道。老子們這樣心疼你。你哭甚麼。惱了性子。斫做三四段。富新不敢做聲。咬牙死忍。心中又悔道。前日那個賊雖然凶狠。還稍有情愛。這夥賊更惡。早知走不脫。又不如聽那三個淫婦的話。且住着罷了。昏一會。醒一會。到了次早。尚爬不起來。到了早飯時。只聽得外邊大喊道。我的孩子你們怎麼窩藏在這裡。快還我便罷。不然老子就去回將主。富新聽得是那毛賊的聲音。一驚。魂已冒出。你道這個賊如何尋了來的。他們昨晚打糧回來。遠遠見那三個婦人在帳房門外正盼他們呢。一見了衆人。笑容可掬。道。你們怎就去了這一日。叫我們眼都望穿了。那衆賊忙跳下馬。上前摟住。這一個親嘴。那一個搵腮。親熱了一會。然後說道。因去的遠。來遲來(了)。我們也記着你們呢。遂兩三個擁着一個。這個摟摟。那個捏捏。一個就伸手去摸摸下身。好生親愛。衆賊進了帳房。那毛坑蛆不見富新。忙問道。我的那孩子往那裡去了。三個婦人道。他要約我們逃走。我們捨不得衆人。他自己去了\footnote{三婦以前處沒奈何之地。還算不得負心。此數語乃負心之至。}。那九個賊同抱住他三個。道。好多情多義的心肝。不枉我們用力服事你。因譏誚毛坑蛆道。哥正同我們大家頑頑罷了。愛上了那小子的糞坑。今日人在那裡。還是我們這妙人兒知趣。毛坑蛆大怒。道。我這幾日爲你(他)把力氣都費盡了。他一點情也沒有。我去攆上殺了他。纔出得這口惡氣。見天晚了。只得忿忿的歇息。次日黎明。就騎馬四處去問。有看見的說道。昨日一個標致小廝被某營某人拴了去了。他故此尋了來。那夥賊見本主兒來找着了。沒得說。便道。昨日我去巡哨。知他是逃走的人。帶了回來。等人來認。毛坑蛆道。旣如此說。叫他來隨我去。衆人見富新動不得。假說道。他得了病。睡倒了。起不來呢。那賊走進帳房。見富新伏着睡在鋪上。一絲兩氣的。他大怒。把被一掀。見他精光的爬着。臟頭長拖。心中起火。罵道。你這沒良心的奴才\footnote{這一句罵得當。}。原來尋這樣快樂地方來了。氣忿不過。右手拔出尖刀。左手攥住臟頭。向糞門裡一剜。富新大叫一聲。早已了帳。毛坑蛆把他腸子扯出數尺。忿忿的向衆人道。讓你們受用。揩了揩手。揷上刀。出帳上馬而去。富新因這糞門做了多少負心的事。今日受了這番惡報。衆人將他屍骸拖出。抛於荒草之中。不在話下。那毛坑蛆回到帳房。向衆人說了一遍。都哈哈大笑。那三個婦人毫不動念。也嘻嘻的笑\footnote{忍心哉。後之受報者。因此二語耳。}。少刻。衆賊同三婦頑耍。毛坑蛆沒了對子。也挨了過去。衆賊道。我們幾個人分了三個。你一個人獨得〈子〉了一個。你占了多少便宜。此時你的情人就沒了。又想來攙我們的分兒。自己也過不去。倒是三個婦人說道。你一個帳房的好弟兄。大家頑頑罷了。那裡〔算〕得這些。就添他一個。我們也沒有吃甚麼虧。衆賊道。我們九個配你們三個正是數。添了他來。那一個肯讓。決必不依。毛坑蛆心中懷恨。本要殺了三個婦人。大家樂不成。見三婦有心到他。衆人不依。不關他們事。不忍下手。一肚醋氣。想了個主意。第二日。悄悄到小賊頭報知。說他帳房中有三個美女。且會彈唱。那賊頭聽說。忙親來一看。見了心愛得了不得。遂叫跟到他營中去。衆賊見是管主要。不敢違拗。心中雖十分捨不得。也無法奈何。又見這三個婦人雖然肯去。一步一回頭的望他們。越發難抛難捨。直等看不見了。纔回帳房坐下。大家不住嘆氣。這賊頭把三個婦人帶到帳房。也不等天晚。便輪流大弄。這賊頭就是當日水氏的奸夫勝叫驢李四。他因問徒逃脫。投了流賊。李自成見他力壯身強。放了他一個小頭目。管五十名賊。他的陽物雄壯。精力又雄壯。這三個婦人更自遂心。過了兩日。不想被別的賊頭知道了。要來分惠一個。李四如何捨得。那個賊頭見他獨享俱(其)樂。動了醋心。就到處倡揚李頭目帳裡有三個美人。三三兩兩。互相傳說。風聲傳到李自成耳中。傳出令來。叫這三個婦人去看。李四可敢不遵。即時親自送去。李自成一見大喜。問了許多話。知他兩個會彈唱。吩咐每人唱了一個。更加歡樂。叫他傍邊唱着侑酒。點上燈。同他三人一齊上床。李自成三人中更愛雨棠。就同他弄起。毫無涯際。李自成的陽物本來渺小。這三個婦人連日又弄得其大無比。李自成甚不洽意。拔出。向龐氏雪梅試試。亦復如是。興致索然。叫他三人下去。各自睡了。這三個婦人每日雖吃着美酒羊羔。那比小賊帳中牛肉燒刀固美。但那下邊竅中竟學敎門過年。如何過得。李自成醉臥帳中。衆婦女都睡熟了。他三人不約而同。一齊起來。悄悄而走出。到一個看內營的帳房中去行樂。那些賊正坐着支更。見他這樣標致的婦人。那裡還顧得死活。此時連大王都不怕了。便一齊混弄起來。你急我奪。雖是悄悄說話。未免有聲息外聞。不想被巡夜的頭目走來聽見。側耳一聽。聽得幾個悄說道。這是大王爺的美人。又不是你三個得來的。也讓我們大家嘗嘗。你怎麼只管獨享起來了。也不見答應。喘吁吁的弄得陰中一片聲響。又聽得婦人哼哼喞喞的淫腔。大驚道。好大膽。瞞着大王做這樣的事。我若不拿。定然貽累到我。遂同巡兵打開帳房。喝道。大膽的奴才。你們可做得好事。先那三人抽弄着。別人拉也拉不下來。被他這一嚇。竟一交翻在地下。那頭目喝叫衆將(賊)都精光的綁了。等到天明。稟了李自成。李自成大怒。命將衆賊都〔拿〕出去剝了皮。親問三婦道。你們好大膽。怎敢在我跟前作此勾當。三婦哀稟道。小婦人怎敢如此。我三人原是個官兒的一妻二妾。被營中搶了來。我丈夫生得比我們還嬌美。有一個留着他做小官。那九個人留了我三個。將如何姦淫。如何去打糧。富新如何要同他們逃走。恐走不脫不肯去。丈夫如何忿恨去了。後來怎樣把實話吿訴那人。那人氣忿。次日尋找着。把我丈夫殺了。昨夜我們正睡着。夢見我丈夫走了來。與在生時一樣。叫我們道。大王叫你們快去。我們便昏昏沈沈跟着走去。竟不知道。後來被人弄着。纔省(醒)了過來。不知如何到那裡去的。這明是我丈夫來報仇了。求大王爺憐察。李自成聽了這些話。說得富新如此標致。怒道。有這樣妙物不來上獻。竟公然私自留着。且又殺害。可惡至此。傳了李四來。吩咐道。你到前日這三個婦人那帳房中。查出收留殺害富新之人。即刻暫(斬)首。那毛坑蛆享用了幾日屁股。換去了一件吃飯的傢伙。大折其本。也是凶淫之報。李自成向三婦道。你夫妻四人被拿了來。你們得了樂處。就負了丈夫。今日在我這裡。又公然私出偷淫。本該碎屍萬段。據你們說。是你丈夫魂魄誘了出去。他來報寃的。這還情有可原。饒你們一個全屍。叫你們快活死罷。吩咐取三條板凳來。將三婦剝得精光。仰綁在凳上。屁股出在凳外。將兩腿彎用繩綑住。使牝戶大張。叫擡出營門外。傳令命守內營衆兵。輪流轉弄。以死爲度。那些賊得了這個美令。他畜生一般的人。知道什麼羞恥。大家挺起陽物。紛紛攘攘上前去弄。這個弄完了。那個就接上。起初這三個婦人還不覺得。後來漸漸腹脹如斗。受不得了。哀號之聲震耳。那些賊只是亂搗亂攮。又過一會。已經死了。衆賊愛他標致。還弄個不歇。直至小腹脹裂了。臭不可聞。方纔罷手。繳了令。命抛了出去。恰好撂在富新一處。他四人生雖析離。死後得在一處。眞個可謂不是寃家不聚頭了。富新已受了負心之報。這三婦又受了負富新之報。可見負心人不可做的。舉一推百。不但於此。即世間大小事。皆負心不得也。冥中負報最重。世人可不\endnotemark[127]愼歟\footnote{世上惟負心人最多。故此諄切以言之也。}。李自成見三個婦人死了。怒氣稍息。想了一會。忽命傳牛金星進帳。說道。方纔那三個婦人說他丈夫是個官兒。我營中的人旣拿了明朝的官來。爲何不解上來見我。竟大膽公然留着弄屁股。這等可惡。我如今正要收買人心。今殺了他一個不打緊。別的官兒聽見到了我們這裡要臊。臊了還要殺。誰還肯來投降。牛金星道。這人是個小官兒。還不妨事。若是大官。便不可了。李白成道。軍師差矣。古人說。兔死狐悲。物傷其類。大小總是一理。小官兒臊得。大官兒也就臊得了。這個名可是傳得出去的。牛金星道。大王只管放心。就是明朝的大官。旣背主來降。忠義全無。良心喪盡。他也就不怕臊了。大約像臣們要臊他。他或者還有些難意。若是大王爺之玉卵行幸。恐他們還求之不得呢。李自成大笑道。這是軍師過於奉承。孤家之德。或者還未必使衆人仰慕至此。牛金星道。臣非無據之言。敢欺誑大王。那太監杜勳。他也是個督師太監。八輿黃蓋。衣蟒腰玉。職分也不算卑了。齒過四旬。年紀也不爲幼了。只因他沒有鬍子。還裝嬌作媚。前日。同了十數個少年文武官兒。都是新來投降的。到臣帳中。說大王寶帳之內。美女固然衆多。恐無妖好狡童以薦枕蓆。他們情願以粗臀上獻。稍表歸順之誠。臣不識大王尊意若何。可愛這後庭之地否。故不敢上啓。以此言之。就臊臊也不妨。李自成喜道。他們來降。我還恐他們是不得已。尚怕他們不忘故主。心懷二念。旣肯怎(這)樣效忠於我。都該重膺封賞。你速去傳諭他們。孤家一人之雨露不能溥及。他衆人之情孤已心領。還叫他們傳揚開去。孤家極好此道的。倘或明朝的那些將相不怕臊的聞風而來。那時。孤家也說不得破些精力對付他們。萬一不能遍及。少不得叫你們來替我代勞。牛金星忙跪下叩首。道。臣預謝大王隆恩。李自成哈哈大笑。後來。各處的少年文武稍有姿色的。都歸之如市。久之。連那白髮蒼髯的大臣都來歸附。希圖一時之恩。便可長保富貴。南風之熾若此。亦千古來未有之異事也。那暗(時)有人笑道。

\begin{quotation}

餘桃一啖羞千古。斷袖相歡辱史書。

堪嘆明朝諸將相。賊庭囗欲泣前魚。

\end{quotation}

又有一作。道牛金星雖是個賊的軍師。竟有那知人之哲。能識那時文武的心腹。道他們。

\begin{quotation}

囗身旣降寇。何辭股獻之。

只貪一時寵。那惜萬年嗤。

\end{quotation}

還有四〔句打〕油嘆那時的臣宰。道。

\begin{quotation}

何以後庭寵。全忘故主恩。

南風緣大競。笑罵復奚論。

\end{quotation}

閒話且住。再說司進朝因這一番。此後大改前非。再不貪色。服滿之後。又續絃娶了個妻子咸氏。乃鍾生舅母之女。咸平之姊。十分賢淑。後來生兒欲(育)女。一家歡樂團圓到老。他父親司導所遺的官囊有二萬之外。儘夠他一生受用了。按下不題。且接前傳。崔命兒自從學會這採戰之術。行了多年。也葬送了無限貪淫的惡少在此牝戶之中。到此時。年已四旬之外。相貌還是二十來歲光景。較少時更加艷麗。他把男人的此道見過無數。因那鐵化同竹思寬來訪他。鐵化連火氏都敵不住。可還禁得他採鎖。不到一盞茶時。早已完了兩度。自覺抱愧。因力薦竹思寬的本錢雄壯。命兒是無不領納的。也同他試了試。誰知他的這陰戶會了採戰。竟能開能合。竹思寬如驢之具。竟容之有餘也。被他採了兩次。命兒也不過見他大而已矣。亦別無他趣。因道。我閱過多人。此物之大。要算你第一了。但不知世間可有會採戰的男人。同他試試。想定別有妙處。竹思寬道。鐵大爺的令妹夫童百萬。是有名會\endnotemark[128]採戰的。何不會他一會。就把他如何會吸酒詳細吿知。命兒聽了。喜不自勝。他就想遇採戰的人。要得他久蓄的精髓。今日忽聽見童百萬的陽物會吸酒。他想道。一定是會採戰的了。我何不會他一會。怎麼弄得他來。又想道。不如我去就他爲妙。設或把他採洩了。就有性命之憂。在他家中。還可爲辭。若在庵裡來。倘有長短。那就不妙了\footnote{滿心想害人。反害了自己。從古來不可勝數。又不獨命兒此事爲然。}。想定主意。叫了兩乘轎子。帶了妙炎往童家來。那童自大正在上房同他那些妻妾談笑戲耍。外邊說進來。道。有個慈悲庵的兩個姑子來會老爺。他近來肯行好事。聽說。只道是來化緣。就出來迎着。見前面這個姑子。穿一身華麗僧衣。靑旋旋一個光頭。配着雪白嬌嫩的俏容。只像有二十來歲。後面跟着的那個。也生得俊俏。却有三十多歲了。忙讓到書房坐下。吃罷茶。童自大道。師傅到寒舍來。有甚麼見敎。那姑子微微的一笑。却不答應。童自大見他這個騷態。心中甚愛。不轉睛的望着他。問道。\endnotemark[129]師傅。你笑的甚麼。有話只管說。不論你化甚麼。我都肯。那姑子道。檀越廻避了衆人。童自大吩咐衆人都出去。命兒笑道。我不是來化東西。我聞得檀越能\endnotemark[130]採戰。可是眞麼。童自大聽見問他這話。心喜非常。忙站起。走到命兒跟前。道。我也略知些。師傅。你莫不是要試試麼。命兒道。我正要來請敎。童自大笑道。承你美情不拒。我們試一試是甚妙的事。一面叫那個姑子去閉門。他將命兒摟着。同到床上脫光。命兒將他陽物一看。大張大馬口。比別人的大不相同。心暗喜。童自大見命兒渾身白軟柔嫩。似一堆新綿。胯中那件妙物生得更飽滿有趣。興致大發。陽道大舉。輕輕一下。就揷了入去。童自大並不知婦人會採戰。他弄進去。一頂盡根。正想運氣咬他。顯顯手段。不想反被他內中一下咬住。動也動不得。咂將起來。童自大從未經此。甚覺得受用。憑他咬咂。咂了多時。他心中快活。也就吐了幾滴。命兒見他精出。以爲畢事。定然大洩。忙用力採吸。却又沒有。如此數次。他力也就費盡了。他並不知童自大是可採可吐的。只說一洩便不能止。只顧用力。雖然自己十分用力。但人的精力有限。一鼓作氣。再而衰。三而竭。大小總是一理。童自大却覺他內中咂得一陣鬆似一陣。後來漸漸咬不住了。就像沒牙齒的老兒放了塊硬肉在嘴裡。只好亂咬。却降不動。童自大覺物鬆活。他却咬將起來。一下咬住了花心緊咂。命兒被他咂得渾身一陣陣的發麻。先還咬着牙關忍住。約有一個時辰。只聽得他道。不好了。我要死\footnote{童自大的陽物。又是他的催命兒了。}。說了這一聲。陰中一陣滾熱流出。童自大張開馬口儘着吸。他也只當每常\endnotemark[131]婦人的一樣。吸幾下就盡了。誰知這次越吸越多。吸了多時。覺與平時大不相同。渾身上下骨縫中。精氣無處不到。後來覺得充滿了。採吸不盡。他內中還流個不住。再看那姑子時。像死了似的。倒吃了一驚。連忙拔出。叫道。小師傅。快來看看你師太是怎麼樣了。那妙炎正在帳外看他二人熱鬧。見說。忙來一看。見命兒目閉口張。只出冷氣。不覺哭起來。道。你把我師太弄死了。童自大也着了急。摸他身上溫熱。胸口劈劈的跳。叫道。不妨事。你快度他幾口氣。那妙炎忙對了他的嘴。儘着度氣。度了有兩頓飯時。纔醒了過來。眼中墮淚。長嘆了兩聲。道。我一時誤信人言。今日命喪你手。童自大道。你是怎的了。命兒道。我也會採陽補陰。當日師傅傳我這個妙法。他說若採得採戰男子的精來。一個可抵千人的功效。又說男子決不敵婦人。我誤信了這話。聽得人說你會採陰補陽。我希圖得你的陽精補益。誰知反被你採去了。童自大道。我的要吐就可吐。我洩些與你何妨。命兒道。我渾身精脈已盡。不過數日苟活。還要你那一滴子濟得甚事。嘆了幾聲。道。罷了。我也傷的人不少。一報還一報。今日死乃其分\footnote{辱翁曰。此妖物也還有自知之明。}。童自大倒急得沒法。命兒叫妙炎扶起他來。看那褥子上的陰精。白亮亮如水銀一般。也不知有多少。向童自大道。你看。此皆我之骨髓也。童自大看了。也覺詫異。與別的婦人所出之精大不相同。妙炎替他穿上衣服。坐了一會。定了定神。擡了轎子回去。第二日就伏枕不\endnotemark[132]起。一個美貌嬌尼。一夜變成了個雞皮老禿。閉眼就見他二十年來弄死的這些少年來索命。嘴中胡說亂道。哀求一會。吿饒一會。捱到了七日上。猛然大叫了兩聲。氣絕而亡。身體乾枯。竟是一個人臘。喪事畢後。元品妙炎將命兒的家私二人八刀\endnotemark[133]不知逃奔那個厚友\endnotemark[134]去了。命兒的田地房產。仍爲族人分去。後來這個慈悲庵敗落。成了衆人遊戲的公所。你\endnotemark[135]說當日那接引庵黑姑子說男子再採不過婦人。謂以柔克剛。必然得勝。今日童自大是個蠢物。怎麼命兒倒喪在他手中。有一個緣故。要是那聰明會採戰的男子。他知道這婦人也會採戰。心中防他。恐自己洩漏。却又想採他的陰精。自然要閃躱騰挪。這些的婦人也就該留一番心待他。相持久了。靜自然勝動。男子自然敵婦人不過。童自大被命兒咬住。咂得快活。忍耐不得時漏掉了些。命兒見他如此。只說他是個夯漢。也不知他能吐能採。儘着力吞鎖。不想力有盡時。反被他一採。收納不住。走了個乾乾淨淨。雖然是人事。這也是他害了多少少年的報應。他若不死。將來流毒貽害還了得麼。那接引庵的姑子。虧生得黑醜。人不愛他。他旣不能傷人。人也不得害他。扯了個直。即如楚國的養由基。射了一輩子神箭。手中不知殺了多少的名將。後來反被吳國的兵卒射死。人生世上。恃着這件本事傷人。將來定以此自殺。總是一個循環的道理。這童自大眞是蠢人有蠢福。自從採得這一番之後。精神加倍。面貌生光。大不同往昔。\endnotemark[136]一日。他一個朋友來探望。說了一會話別去。他送了出來。到大門外。那朋友去了。他纔待轉身。忽有一個人走到跟前。跪下叩頭。童自大忙扶起。道。我的哥。你是誰\footnote{人雖改變。說話仍不脫市井氣。故妙。}。打那裡來。怎認得我。那人道。小人有件機密事。倒有些疑心。來和老爺說。童自大忙同他進來。到書房中。把門關上。讓他坐。那人道。小人如何敢坐。童自大再三不肯。道。我同你又沒有甚麼上下。坐了好講\footnote{他雖名自大。却毫不自大。宜有厚福。不似他人腰中略有幾文。便眼眶如燈盞。是(世)人皆不在目中。}。那人辭讓再三。纔敢在旁坐下。說道。小人係河南人。名叫蒙德。向年同家眷逃難到此。蒙老爺恩養了大半年。救了性命還鄕。小人朝夕感恩。無可爲報。今爲尋個親戚到這裡來。今日是葛城起身。誰知太早。走了二十多里。看月色時。只將半夜。前後不見個人影。小人心中一時害怕。爬在一棵樹上坐着。等個伴好走。坐了一會。忽然聽見兩個九尾狐狸走了來。捧着人的骷髏。頂在頭上。對月下拜。叩了幾個頭。變成兩個女人。一個穿白。一個穿靑。小人嚇了一跳。聽得穿白的說道。我的功行已成。再漏得一個有福的陽精。大丹就成滿了。那穿靑的說道。這倒是件難事。那有福的人斲喪過了。精氣有限。就得了也是無益。那裡輕易遇得着一個童身未走的元陽。那穿白的道。也不在這些。我聽得有個童百萬。他是個福人。又生來老實。決看不破我。不怕洩了機關。若得了他的。也就好了。那穿靑的道。你如何得見他。穿白的道。我到他家。說是個寡婦。求他週濟。見了面。見景生情。我這樣美貌。難道怕引不動他。那穿靑的道。你幾時去。穿白的道。今日月滿之夜。又是黃道良辰。挨晚些到他家。故意捱黑了。他若留我。我便宿下。這就更妙極了。正說着。遠遠有人來。就不見了。小人聽見是說老爺。飛星趕來報知。恐今夜着了妖精的手。老爺可防備他。盡小人一點報恩之心。童自大驚道。我的哥。虧你來說。不然被他吸了我的精脈。怎麼處。你在我家住着。等他來過了。我謝你。蒙德道。小人還要去尋親戚。改日再來見老爺罷。童自大道。你是必要來的。他應諾而去。童自大不知狐狸來是要漏他的陽精。只道是要害他的性命。心中想道。這妖怪好可惡。我與你無怨無仇。怎麼想來算計我。想個法兒處治他纔好。想了一會。想不出主意來。又恨又怕。走了上去。衆妾見他面色改變。問他緣故。他把上項話說了。內中一個妾姓閔。小字慧姑。生得面白身肥。指尖足小。性格溫柔。齒牙伶俐。敏慧異常。他聽了。笑道。這是老爺造化到了。怕的是甚麼。童自大道。這是送命的造化。免勞照顧。慧姑道。我當日在家中聽得父兄們說。任他怎麼得道的仙狐。酷好的是燒酒熏雞白煮蛋。老爺何不預備下。把他灌得大醉。他動不得了。古語說。慢櫓搖船捉醉魚。那時老爺却去採他。他是千百年修煉的丹頭。老爺若採得了。可成半仙之體。豈不是大造化。童自大道。你說的固然有理。我到底有些膽怯。又一個妾姓甘。叫做甘老姐。就是那甘壽的女兒。已長成了。生得身肥體厚。百媚千嬌。甘壽熊氏年老無子。情願將女兒與他爲妾。圖養老送終。這老姐也甚是聰明。接口道。老爺何必膽怯。看局面行事。他果然大醉了。只管放心行事。他若不肯吃酒。多叫幾個家人在外間上夜。怕他些甚麼。況且老爺方纔說那報信人的口聲。那狐狸他並不是要害你。不過想得數點陽精。助他的丹道。採得他的是萬幸。萬一不然。就洩些與他。也無害於事。童自大聽了。歡喜贊道。能幹女子強如懵懂男人。你兩個人的主意見識妙極。比我竟還通幾分。就依你們這樣行。出來吩咐家人。買了上好乾燒酒燻雞白煮蛋。又叫家人都吩咐了。正是。

\begin{quotation}

準備窩弓擒猛虎。安排酒食弄妖狐。

\end{quotation}

童自大不住的在大門口走進走出。他聽得甘老姐說不是要害他的命。他倒反巴他來。看看是怎個樣兒。將晚時。遠遠望見一個穿白的婦人來了。由不得那心中亂跳起來。只見那婦人走到跟前。拜了兩拜。童自大把他一看。竟是個天仙的面龐。俗語道。若要俏。須帶三分孝。一身的縞素。更覺些妖嬈。有幾句古語。借來贊他。道。

\begin{quotation}

施朱則太赤。敷粉則太白。加一寸則太長。減一寸則太矮。眞有沈魚落雁之容。閉月羞花之貌。世間美婦那能儔。天上垣娥堪與匹。

\end{quotation}

他生平也沒見過這樣美人。心中一愛。把那怕飄到東洋大海去了。忙答禮。道。奶奶。你從那裡來。那婦人嬌聲細語道。我來尋童老爺的。童自大道。我就是。這門口不便說話。請到裡面去。那婦人見說。喜得笑吟吟的進來。到了書房中坐下。婦人道。我先夫姓胡。我姓白。先夫歿了。又無親戚可靠。聞得老爺是位慈悲好善的人。特來求吿資助些盤費度日。花言巧語。說了許多。也不能盡述。又做出千嬌百媚的妖態。要迷惑童自大留他。那知童自大心中明鏡一般。知他都是鬼話。滿口答應。有有。不要說要我資助。就是叫我養活一輩子。我也肯。但你請放心。少刻。點上一根通宵大燭。童自大越看越愛。暗忖道。婦人中如何有這等標致的。怪不得他會纏人。我也顧不得了。弄得他過來是造化。弄他不過。洩些與他去。有何妨。且快樂一時是一時。遂涎着臉。笑嘻嘻望着他的臉。道。天晚了。你將就在這裡宿一夜罷。要甚麼。明日都有。若不嫌棄。我便奉陪。那狐精以爲童自大落在他的彀中。心中暗喜。不想反入了人的圈套。他喜孜孜啓一點硃唇。露兩行碎玉。嬌滴滴的聲音說道。怎好攪擾老爺府上。又故做嬌羞之態。掩口微笑道。陪倒不敢奉勞。童自大也笑道。主人可有不陪客的禮。不怕簡慢麼。若不稀罕就罷了。那狐精笑着把眼一瞟。做那勾人的態度。童自大吩咐。快看酒來。不一時。捧上一大盤燻雞。一大盤煮蛋。兩碟\endnotemark[137]秋油。四碟小菜擺下。將燒酒斟上。童自大道。天晚了。沒有甚麼款待的。將就用些。這狐精雖能變化。那裡知道人心裡算計他。他酷喜的是這幾件美物。見了正投所好。欣然同飮。童自大先替家人說過的。他鍾內是白水。陪着他鍾鍾吿乾。這乾燒酒其味甚甜。吃着不覺。過後却利害。那狐精見主人吿乾。他以爲自己酒量甚大。也〈不〉想把主人弄醉了好行事。主一鍾。他一鍾。鍾鍾不辭。看看後來有些醉意了。酒能亂性。他竟忘其所以。也不等主人讓。就着菜。吃得好不臊皮。約吃了有三四斤。有些支撑不住了。童自大還恐他是假醉。又親自拿着杯酒送〈道〉到嘴上。他竟伏在桌子上睡去。童自大見他是眞醉了。叫家人擡他到床上臥下。悄悄吩咐家人都要醒睡。我若叫你們。都要答應。衆人應諾。他進去。先自己脫了。然後替他脫盡。此時興發如狂。也顧不得是妖怪了。挺起陽物。一陣亂搗。然後運氣混咬起來。在內中大張馬口。一下咬住花心。含着力咂。那狐狸多時方醒。身子軟癱。急得只是亂扭。童自大吸了個盡情。看那狐精時。反昏昏睡着。童自大得了丹頭。精神頓旺。心中大樂。只見那狐精哭起來。童自大假驚〔道。〕你爲甚麼。他道。實不瞞你。我是一個千年老狐。費了多少苦功修煉。已經將成正果。只想得你有福的人一點陽精。我就成了仙丹。便可脫去皮毛。誰知一時圖貪口腹。把幾百年功夫。一旦送在你手。你旣得了我這些精華。可以延年却病。但苦我的工夫枉費了。童自大反倒可憐起他來。〔道。〕你旣然要得我的精。我洩些與你。何如。〔他道。〕我的大丹已失。此時就你〔洩〕些。也無濟於事。你旣有盛意。雖無大益。也還有小補。那(童)自大便上他腹來。抽弄了一會。道。你快些鎖。我要洩了。那狐精用力咬住。一陣亂咂。童自大一股陽精冒出。那狐精閉目凝神收吸盡。不覺紅日照窗。一同穿衣起來。童自大見他悵悵不樂。叫取酒來與他解悶。他微笑了笑。復長嘆了兩聲。道。

\begin{quotation}

貪此一杯物。失却千年寶。

昨日何散(歡)欣。今朝倍煩惱。

\end{quotation}

又嘆了一聲。這東及(西)害人非淺。起身扭(拉)住童自大的手。囑道。你有大福。須當固愛。作別要去。童自大要取些金銀相送。他笑道。那銀錢不過糞土之物。要他何用。出門。恍惚不見。童自大覺得氣爽神豪。心中大喜。過了兩日。那蒙德來探聽這事。童自大細細吿訴他。又取了三十兩銀子送他路費。那人領了。拜謝而去。童自大因這一番慷慨。因而得這兩次仙丹。後來竟活到百年之外。不想這樣一個愚蠢的人。竟得多福多壽多男子。可見人生在世。不可不做好事。人生幾何。胡不自省。自從〔宦蕚〕與賈文物〈即〉幫童自大做了一番好事之後。妻妾三人各舉數子。賈文物的四位美妾也都各產佳兒。可見天之報施不爽。正是。

\begin{quotation}

人間私語。天聞若雷。積善之家。必有餘慶。

\end{quotation}

閒話少敍。此回專言童富翁。下段獨表宦公子。端的宦公子是賢是愚。是善是惡。聽我細細敷行(衍)。便知他的詳細。



\endnotetext[1]{「司進朝」原作「司朝進」,據上下文改。}

\endnotetext[2]{「夜」字原無,據上海優生學會排印本(以下簡稱「排印本」)加。}

\endnotetext[3]{「相伴新嫂嫂」原作「相婢新如嫂」,據排印本改。}

\endnotetext[4]{「了」字原無,據排印本加。}

\endnotetext[5]{「嫁」字原無,據排印本加。}

\endnotetext[6]{「做」原作「個」,據排印本改。}

\endnotetext[7]{「老」原作「兒」,據排印本改。}

\endnotetext[8]{「忽然」原作「然忽」,據文義改。}

\endnotetext[9]{「身」字原無,據排印本加。}

\endnotetext[10]{「家」原作「嫁」,據排印本改。}

\endnotetext[11]{「些」原作「細」,據排印本改。}

\endnotetext[12]{「雌」下原衍一「之」字,據排印本刪。}

\endnotetext[13]{「暫」原作「整」,據排印本改。}

\endnotetext[14]{「太」下原衍一「年」字,據排印本刪。}

\endnotetext[15]{「言」字原無,據排印本加。}

\endnotetext[16]{「也」字原無,據排印本加。}

\endnotetext[17]{「這」字原無,據排印本加。}

\endnotetext[18]{「棍」原作「根」,據排印本改。}

\endnotetext[19]{「同」原作「替」,據排印本改。}

\endnotetext[20]{「有」原作「在」,據排印本改。}

\endnotetext[21]{「你」原作「他」,據排印本改。}

\endnotetext[22]{「你的尊見極是」原作「你的令尊見是極」,據排印本刪改。}

\endnotetext[23]{「炎」原作「焱」,據上下文及排印本改。}

\endnotetext[24]{「天」原作「然」,據排印本改。}

\endnotetext[25]{「亮」字原無,據排印本加。}

\endnotetext[26]{「庵」下原衍一「的」字,據排印本刪。}

\endnotetext[27]{「他」原作「你」,據排印本改。}

\endnotetext[28]{「他」字原無,據排印本加。}

\endnotetext[29]{「在」原作「他」,據排印本改。}

\endnotetext[30]{「厚」原作「原」,據排印本改。}

\endnotetext[31]{「到聽」原作「道聽」,據第一回改;下文或同,不贅。}

\endnotetext[32]{「的」原作「底」,據排印本改。}

\endnotetext[33]{「命兒」二字原無,據排印本加。}

\endnotetext[34]{「道」字原無,據排印本加。}

\endnotetext[35]{「要」原作「好」,據排印本改;下文或同,不贅。}

\endnotetext[36]{「也」字原無,據排印本加。}

\endnotetext[37]{「命」下原衍一「兒」字,據排印本刪。}

\endnotetext[38]{「崩」原作「烹」,據排印本改。}

\endnotetext[39]{「身」原作「床」,據排印本改。}

\endnotetext[40]{「爲」字原無,據排印本加。}

\endnotetext[41]{「嘻嘻」下原衍一「嘻」字,據排印本刪。}

\endnotetext[42]{「了」字原無,據排印本加。}

\endnotetext[43]{「脫」字原無,據排印本加。}

\endnotetext[44]{「關」字原作「開」,據排印本改。}

\endnotetext[45]{「開」原作「門」,據排印本改。}

\endnotetext[46]{「誰」原作「罪」,據排印本改。}

\endnotetext[47]{「攬」原作「覽」,據排印本改。}

\endnotetext[48]{「又」字原無,據排印本加。}

\endnotetext[49]{「句」下原衍一「把」字,據排印本刪。}

\endnotetext[50]{「日」下原衍一「內」字,據排印本刪。}

\endnotetext[51]{「原」原作「愿」,據排印本改。}

\endnotetext[52]{「縫」原作「逢」,據排印本改。}

\endnotetext[53]{「馥」原作「佛」,據排印本改。}

\endnotetext[54]{「倍」原作「輩」,據排印本改。}

\endnotetext[55]{「醒」原作「惺」,據排印本改。}

\endnotetext[56]{「的」字原無,據排印本加。}

\endnotetext[57]{「喜」下原衍一「想」字,據排印本刪。}

\endnotetext[58]{「以乾柴就烈火」原作「以爲乾柴烈火」,據排印本改。}

\endnotetext[59]{「却」下原衍一「然」字,據排印本刪。}

\endnotetext[60]{「道」下原衍一「這」字,據排印本刪。}

\endnotetext[61]{「惜」原作「措」,據排印本改。}

\endnotetext[62]{「滋」原作「茲」,據排印本改。}

\endnotetext[63]{「呢」原作「泥」,據排印本改。}

\endnotetext[64]{「胳」原作「〖月谷〗」,據排印本改。}

\endnotetext[65]{「貼」原作「貽」,據排印本改。}

\endnotetext[66]{「着」字原無,據排印本加。}

\endnotetext[67]{「把」原作「托」,據排印本改。}

\endnotetext[68]{「摩」原作「麻」,據排印本改。}

\endnotetext[69]{「伸」原作「陣」,據排印本改。}

\endnotetext[70]{「蓬蓬」原作「逢逢」,據排印本改。}

\endnotetext[71]{「淌」原作「倘」,據排印本改;下同,不贅。}

\endnotetext[72]{「怎」原作「這」,據排印本改。}

\endnotetext[73]{「怎麼」原作「這個」,據排印本改。}

\endnotetext[74]{「親」原作「嘴」,據排印本改。}

\endnotetext[75]{「已」字原無,據排印本加。}

\endnotetext[76]{「匹」原作「四」,據排印本改;下同,不贅。}

\endnotetext[77]{「挽」原作「捥」,據排印本改。}

\endnotetext[78]{「腕」原作「捥」,據排印本改。}

\endnotetext[79]{「富新」原作「富心」,據上文改;下文或同,不贅。}

\endnotetext[80]{「族」原作「內」,據排印本改。}

\endnotetext[81]{「親」字原無,據排印本加。}

\endnotetext[82]{「績」原作「續」,據排印本改。}

\endnotetext[83]{「如」原作「姑」,據排印本改。}

\endnotetext[84]{「友」字原無,據排印本加。}

\endnotetext[85]{「尚」原作「的」,據排印本改。}

\endnotetext[86]{「發」原作「喜」,據排印本加。}

\endnotetext[87]{「蒙」原作「奉」,據排印本改。}

\endnotetext[88]{「磋」原作「嗟」,據排印本改。}

\endnotetext[89]{「候」原作「後」,據排印本改。}

\endnotetext[90]{「兄長」原作「長兄」,據排印本改;下文或同,不贅。}

\endnotetext[91]{「貧」原作「病」,據排印本改。}

\endnotetext[92]{「柬」原作「束」,據排印本改。}

\endnotetext[93]{「這些東西」原作「這西東些」,據排印本改。}

\endnotetext[94]{「做」原作「借」,據排印本改。}

\endnotetext[95]{「貓」原作「貍」,據排印本改。}

\endnotetext[96]{「你」原作「那」,據排印本改。}

\endnotetext[97]{「聽」字原無,據排印本加。}

\endnotetext[98]{「花」下原衍一「花」字,據排印本刪。}

\endnotetext[99]{「香」原作「秀」,據排印本改。}

\endnotetext[100]{「非常」二字原無,據排印本加。}

\endnotetext[101]{「又是爲」原作「還是末」,據排印本改。}

\endnotetext[102]{「好」原作「氣」,據排印本改。}

\endnotetext[103]{「賤」原作「錢」,據排印本改。}

\endnotetext[104]{「柳」原作「都」,據排印本改。}

\endnotetext[105]{「他」下原衍一「他」字,據排印本刪。}

\endnotetext[106]{「回」下原衍一「來」字,據排印本刪。}

\endnotetext[107]{「你」原作「他」,據排印本改。}

\endnotetext[108]{「亂抽亂扭」原作「亂亂抽扭」,據排印本改。}

\endnotetext[109]{「這只」原作「只這」,據排印本改。}

\endnotetext[110]{「福」原作「幅」,據排印本改。}

\endnotetext[111]{「桃」原作「唇」,據排印本改。}

\endnotetext[112]{「金蓮小」原作「小金蓮」,據排印本改。}

\endnotetext[113]{「餳」下原衍一「餳」字,據排印本刪。}

\endnotetext[114]{「寒」字原無,據排印本加。}

\endnotetext[115]{「用」原作「尖」,據排印本改。}

\endnotetext[116]{「到」下原衍一「有」字,據排印本刪。}

\endnotetext[117]{「你」原作「做」,據排印本改。}

\endnotetext[118]{「按在」二字原無,據排印本加。}

\endnotetext[119]{「日」下原衍一「個」字,據排印本刪。}

\endnotetext[120]{「之」字原無,據排印本加。}

\endnotetext[121]{「誚」原作「捎」,據排印本改。}

\endnotetext[122]{「料道情人」原作「人情科道」,據排印本改。}

\endnotetext[123]{「房」原作「方」,據排印本改。}

\endnotetext[124]{「翻」原作「騙」,據文義改;下文或同,不贅。}

\endnotetext[125]{「是」字原置「我們」二字之上,據文義改。}

\endnotetext[126]{「走着」原作「着走」,據文義改。}

\endnotetext[127]{「可不」原作「不可」,據文義改。}

\endnotetext[128]{「會」原作「爲」,據排印本改;下文或同,不贅。}

\endnotetext[129]{「問道」二字原無,據排印本加。}

\endnotetext[130]{「能」原作「于」,據排印本改。}

\endnotetext[131]{「常」字原無,據排印本加。}

\endnotetext[132]{「不」原作「一」,據排印本改。}

\endnotetext[133]{「八刀」原作「分力」,據排印本改。}

\endnotetext[134]{「友」原作「文」,據排印本改。}

\endnotetext[135]{「你」字原無,據排印本加。}

\endnotetext[136]{「昔」原作「惜」,據排印本改。}

\endnotetext[137]{「碟」原作「蝶」,據文義改;下同,不贅。}

\setcounter{footnote}{0}

\theendnotes

\part*{姑妄言第十九卷}
\addcontentsline{toc}{part}{姑妄言第十九卷}
\markboth{姑妄言第十九卷}{姑妄言第十九卷}

鈍翁曰。宦實家庭訓子一番說話。可抵得一篇過庭訓。乃父旣發此〇〇心。兒子雖不肖。冥冥之中自然亦化爲好人。這一回內。只算得宦蕚一本紀善錄。宦蕚行了許多好事。而報恩者並無多人。只向小娥一個。故此又特特夾寫鮑德一殷(段)。伏下回報德之案。不然施者施之不倦。而報其恩施者竟無其人。豈個個皆無良心者耶。施恩者雖不望報。而報恩只小娥一女子。太把男子漢說得不堪了。故不得不寫此一段。咸平棄妻。鍾生婉轉成就。然終歸功於宦蕚〇〇〇〇〇〇〇〇〇宦氏父子。事有賓主之分。看者須知。至於劉太初此等好人。豈〇〇〇有棄妻之咸平除名。而有不棄妻之劉顯得中。一是警醒世人。一世(是)完劉太初父子好處。卜孝伍氏此等兒媳。在今日不少。焉得霹靂。個個震之。以快人心。一夕話上有兩句。取來贈卜孝夫婦。道。有朝豁剌一聲響。打殺兩個直娘賊。闕氏之子媳不孝。得宦蕚收留。有此恤老憐貧之善人。越顯忤逆不孝之惡子。雷之一擊。適當其罪。

貧寒無儔匹之人。焉能有棺葬父。欲典子以送終。此孝心即可感於神明。宦蕚纔發一點好心。出門便遇見孝子。可謂兩不相負。贈〇〇〇銀。雖是宦蕚做的一件好事。亦韓無儔孝行所致。宦蕚初次出門。頭一個便是寒無儔匹的。可見那時民窮財盡。天下窮人而無吿者多也。

賣菜一生之苦漢。能孝養八十餘之老親。可謂難得矣。宦蕚要作好事。自然從孝字起。所以第一個遇送死之孝子。次即遇養生之孝子。又接寫一欲賣身救父之孝女也。

一貨郞逢賴銀之鄕親。本錢焉得不畢。但賴盈實非賴銀。特貧病耳。宦蕚今日濟之。後食其報。故知其非無恥賴銀之人耳。貧做負恩人一語。可爲註脚。後本賴盈報信。鮑德報德。同在一處。恐人眼光看不到。故此處寫賴盈之後。接寫鮑德也。

嗟乎。貧儒爲妻所棄而不能留。權老兒因貧而不能勸女不〇〇〇〇〇苦。一至於此。姓權者。權離而終合也。司富向爲宦蕚之師傅。今又爲權氏之師傅矣。繆氏始終處處點醒權氏悔心。眞妙人妙舌。不愧姓繆。

向惟仁向日有錢。便可爲人。一旦貧窮。竟至賣女。嗟乎。錢之爲錢。至於此乎。權氏因夫貧而欲棄夫。咸平因妻貧而欲背盟。雖是寫世風囂薄。總是爲錢字放聲一哭。

與利爲徒之人。尚知父母妻子爲何物。若非宦蕚。則父母將塡溝壑。妻子不知更屬何人。此又受圖利之害者。無錢旣不好。有錢又不好。將奈何。然亦在人有善處之方耳。少年沒父。幸得老母巴巴竭竭撫養成人。安得尚有錢娶媳。吉家女將三十。亦難怪親家之急。宦蕚慨然使二姓得完婚配。恩德厚矣。宜乎吉氏之尸祝也。

單于學翟疊峰一段。一則見謔之一字未免觸鬼神之忌。善於謔者。尤不可也。故至於妾婢淫人而死。甄字有堅貞二音。謂雖有堅貞之妻。亦難免賊道之汚以自殺。可謂警戒世人之至。二則謂世間僧道之流。皆如蜂蠆之賊。不可不遠避而緊防之也。

此一回內寫向小娥之孝。平淑姑之貞。甄孺人之烈。可爲閨中師範。

\chapter*{姑妄言卷之十九\\
第十九回 宦公子積德救嬌娃 向惟仁報恩酬愛女\\
附 鍾刑部婉轉成表弟 宦司空慷慨嫁淑姑}
\addcontentsline{toc}{chapter}{第十九回 宦公子積德救嬌娃 向惟仁報恩酬愛女}
\markboth{第十九回 宦公子積德救嬌娃 向惟仁報恩酬愛女}{第十九回 宦公子積德救嬌娃 向惟仁報恩酬愛女}

話說宦實父子一日間家庭閒話。宦實偶然嘆道。天地間再不可以貌取人。當日尼父道。以貌取人。失之子羽。絲毫不謬。我當日看這童家賢姪。不過蠢蠢然一個癡肥財主。你們都還笑他鄙吝。誰料他去年做了這一番仗義的事。可是那看財奴自了漢做得來的。偌大京城。多少財主。可有一個及得他這一場好事。你同賈家賢姪雖然也幫他施捨了些。只算得個碌碌因人成事。這番功德是他倡議。十分中他獨得八九。你與賈家賢姪只算得一二。我的家私雖不能與他相匹。也不爲不厚了。古人說。積書與子孫。子孫未必能讀。積家產與子孫。子孫未必能守。不如多積陰德。存此方寸地。留於子孫耕耳。這是眞正藥言。我如今已八旬的人了。你正在強壯之時。何不力行善事。非爲好名。但願將來得個好子孫。我也可以含笑入地了。宦蕚聽了。悚然道。父親明訓。兒敢不力行。此後但是可爲的善事。自當行之。以承老父之意。那宦實連連點頭。道。你果能如此。就是我幹蠱之子了。我宦遊四十餘年。雖家資殷實。並未曾貪婪酷虐。刻薄屬吏小民。是我一任布政。十載戶曹。又掌工部數年。是分內所應得之物。我靜夜自思。在宦場中不敢說淸廉二字。也還沒有甚麼壞處。到了臨末一着。因得失心重。依附魏公。當日若非鍾親家。今日我身家性命不知作何局面。至今撫心內愧。你若做得一番好事。人念其子而原其父。若掩得我當日之醜。也不枉我生你一場。那宦實殷殷敎訓。宦蕚聽了父親這些話。時刻在念。一心一意要尋些好事做。忽然想起他姑父劉太初來。道。凡事自然先親而後疏。我這姑母同老父同胞兄妹。因我當日少年無知。得罪了他。至今總不上門。後來老父親去請他。他也不肯一到。薄有所贈。又堅拒不受。那年老父爲事之時。他老夫妻忙來叫我急尋門路相救。可見他並不是沒有親情。皆因生性狷介之故。他家中至今一然囗貧。我何不送五百金去與他。不但全骨肉之情。也可救他的貧乏。但恐他不受。奈何。又想道。不要管他。且送了去看。遂取出五百金。命家人宦有識送去。這劉太初名和。江寧縣學庠生。家貧。以授徒爲業。寧甘凍餓。不肯枉道求人。他同宦實作諸生時。就娶了他妹子。就(不)意才高命蹇。走了幾科不中。他竟棄了這領靑衿。自從見宦實做了顯官。未免眼界略大。宦蕚又是有名目無親友的呆公子。那裡認得這窮姑父姑母。他就絕跡不履宦門。今忽見內姪送了五百金來與他。力揮不納。宦有識回來說道。小的雖是個下人。素知劉姑太爺的性情。曉得他是絕不肯受的\footnote{果然有識。不負其名。}。但老爺吩咐。不敢不去。宦蕚道。你再送了去。放在他家門口。你逕回來。宦有識領命。到他門口放下。叫道。姑太爺。我們大爺又叫我送來了。撤身就走。劉太初大呼。叫他拿回。宦有識飛走不答。劉太初只得自己拿着攆了一會。直攆到宦家門口。放下。不顧而走。家人進內說了。宦實父子不勝慨嘆。劉太初寧甘淡薄。絕不求人。誠所謂薑桂之性愈老愈辣者也。在今日。如此公不慕勢不貪財這等心胸之人亦鮮矣。按過一邊。且說宦蕚一日偶然想道。我旣要做好事。但終日坐在家中。外邊事一些也不知。那好事如何飛了來尋我。況我父子雖發了此心。外人不得知道。就有知道的。見我家侯門似海。誰敢敲門打戶的來尋我。我不〔如〕每日在街上閒走。遇可行者即行。豈不爲妙。也不跟多人。只帶兩個小子。身邊揣着銀子。騎兩頭驢兒跟隨他。自己乘了一匹馬。任馬所走之。也不認定到何處去。頭一日出門。正走着。只見一個棺材鋪門口。有兩三個人在那裡講話。內中一個頭上包着白布。披着蔴。些(在)哭哭啼啼的哀求。那賣棺材的道。如今買賣艱難。賒一半。現錢一半。還是照着本錢。就算我的情了。如何白拿了去。這個帶孝的儘着哭吿。那旁邊的一個只是嘆氣。宦蕚跳下馬來。上前問那嘆氣的道。是爲甚麼事。那人見他是個貴介樣子。忙道。這個帶孝的是我一個緊鄰。姓韓。叫作韓無儔\footnote{一個送死的孝子。}。他家中窮寒得無比\footnote{此所謂寒無儔也。}。他父親前日歿了。今停了兩三天。總弄不出個棺材來。我看着心中甚是不忍。這個掌櫃的是我的朋友。同他來賒口材。掌櫃的看我的薄面。定要一半現銀。如今何處得有銀子。我又手內無錢。要有錢時。也就幫他做了這一件好事。宦蕚道。棺材要多少銀子一家門。倒講明白了。掌櫃的也憐〇〇〇〇〇〇〇〇〇〇〇〇〇〇〇〇就是這一個松木兩併。價錢是〇〇〇〇〇〇〇〇〇〇〇〇〇〇這多大事\footnote{富貴公子視此三兩銀子如氈〇〇〇。孰不知貧窮人如少一文錢。尚〇〇〇。}。〇〇〇〇〇〇〇〇〇〇〇兩。遞與掌櫃的。道。都是紋銀。你收了〇〇〇〇〇〇〇〇〇〇〇〇〇做好事。可肯少了小人的。何用稱。就接過〇〇〇〇〇〇〇〇〇〇〇〇頭。宦蕚拉起他來。道。你棺材雖有了。擡錢〇〇〇〇〇〇〇〇〇〇道。蒙老爺天恩。得了棺材。且裝了我父親不暴露着。再做區處。我有個十來歲的兒子。典幾兩銀子。發送他老人家罷了。宦蕚聽說。心中甚慘。又敬他棄子葬親這一點孝心。又將銀子稱了十五兩。對他道。古人說。冠婚喪祭。稱家之有無。這銀子你拿去用。五兩趕着就把你父親葬了罷。死者以入土爲安。我看你也很窮。這十兩銀與你作本錢。尋個小生意做。也可養家餬口。韓無儔儘着叩頭。道。老爺賞了一具棺木。就是莫大之恩了。何敢又當這樣厚賞。宦蕚道。不必多講。快雇人擡材回去。料理你的事去罷。韓無儔見這樣施恩。也就叩謝領了。宦蕚上馬。韓無儔拉住小廝問道。這位老爺貴姓。小廝與他說了。衆人方知是宦公子。都贊揚他的恩德。韓無儔葬了他父親。領着十一歲的兒子。到宦家門口叩謝。送他的兒子與宦家爲僕。宦蕚那裡肯要。因見他好個乾淨孩子。反與了他二兩銀。兩疋布。他父子叫了幾十聲恩人。拜謝而後去。再說宦蕚那日與了韓無儔銀子棺木。心中甚樂\footnote{這一個樂字。便寫得善心充滿。}。又走了一會。只見一個人急得兩頭亂跑。口中叫道。是那位積陰隲的好爺們。若拾着了。賞還了我罷。可憐我是個窮漢。口裡叫着。眼睛急得多大。兩淚汪汪。像瘋了一樣。宦蕚心疑。叫小廝叫過他來。問他是甚麼緣故。那人搥胸跌脚的道。小人名字叫作蔡繹生\footnote{一個養生的孝子。}。是個賣菜的。我家中有個老爹。八十多歲了。病了一個多月。我在家守着伏侍。不得出來賣菜。連兩千文本錢都吃光了。我老爹這兩日略好些。想個鴨子煮口湯喝。又沒有一個錢。沒奈何。我把一件小襖脫下來。當了一百五十文錢。指望買與病人吃。或者就好了。他老人家若好了。我出來借兩千印子錢。賣着菜。還買把米度命。不然再守幾日。一家子全要餓死。我把錢同當票子拴在一處。揣在懷內。不想走急了。到了鋪子裡看了鴨子。摸錢時。才知打襖破處掉去了。不旦(但)我窮人好容易掙一件襖穿。沒了票子。日久了。他如何肯認。宦蕚道。這是你自不小心。票子不拴在錢串上另收着。如何得丢。蔡繹生道。老爺。那當票我拴得緊緊的。如何得丢。因是錢掉掉了才沒了他。他如今還拴在那錢串上呢。旁邊人聽他說這蠢話。由不得都大笑。宦蕚道。你如今在這裡跑着叫甚麼。蔡繹生道。當票同錢掉了也罷了。他搥着胸說。如今我家老爹現沒得吃。眞叫我苦死了\footnote{好孝子。聞此話而不動心者。其人必不孝。}。我所以在這裡求吿。或者有慈悲的爺們拾着。賞還了我罷。不然把當票子拿去。單賞了我的錢去買鴨子。再不然賞我一隻鴨子。他把錢同票子都拿去也罷了。宦蕚道。人千人萬的走。知道誰拾了。況且知是在那一處掉的。這是望梅止渴的事。你空叫有何益。他道。據老爺這樣說。是沒用的了。搥搥胸。望天叫一聲道。天爺爺。苦死我老爹了。掉了兩點淚。才要走。宦蕚道。你站着。叫小廝稱了五兩銀子與他。道。我憐你一點孝心。這銀子給你買鴨子與你父親吃。趕着去贖了衣服穿。剩下的留着做賣菜的本錢。他〈他〉眼睜睜的望着。不敢用手接。宦蕚道。你爲何不要。他道。老爺請收起來。不要同我小人們頑笑。宦蕚道。我好意給你。同你頑甚麼。他笑道。老爺當眞都賞我麼。宦蕚道。旣與你。如何不眞。他笑嘻嘻才伸手來接。又連忙縮回。看着宦蕚。只是笑\footnote{形容得妙極。一生賣菜之人。同人爭一文錢。費多少唇舌。今宦蕚給銀五兩。實是夢想不到。疑天地間無此等事。非寫其呆態也。}。宦蕚叫小廝塞在他手中。他見果是眞了。接過來。叫道。我的恩人老爺\footnote{他叫這一聲。抵得做官的幾百個德政碑。}。我看天底下也沒有你這樣第二個好人了\footnote{實心稱贊。非比他人假奉承語。}。等我老爹病好了。同到這個地方來與你老人家磕頭罷\footnote{刻舟求劍。有人行之。不可笑他此語。}。我不認得你府上在那裡住。說了。歡喜得跪倒在地。叩了十來多個頭。宦蕚叫小廝拉。也拉不起來。直等他叩得興足了。才爬起來。把那銀子看了看。叫旁邊一個人道。你擰我一下看可疼。還是做夢是醒着呢。旁邊人說。大靑〔天〕白日裡做甚麼夢。你快做你的事去罷。他道。不是夢。難道竟是眞。哈哈笑道。好老爺。好人。好人。好老爺。欣欣而去。宦蕚也就回家。在馬上也自得意。道。這兩件雖算不得大好事\footnote{宦蕚此想。不脫膏粱氣味。他以爲銀子用得少。算不了大好事。孰不知全人之孝。濟人之急。乃天下第一大好事也。}。也算發了一個市\footnote{這才眞是開市大吉。}。不枉出來一場。到家歇息。他但無事。就出來大街小巷的走。那一日。見許多人圍着在那裡看。宦蕚也催馬上前一望。只見一個人打着一個人。拳頭脚尖齊上。口中侉聲侉氣不住的罵。那個捱打的也不敢回手。只用手遮攔。這個動手的只是打。宦蕚看了動疑。叫小廝拉他過來。要問他的緣故。他那裡肯依。只是掙着打。宦蕚喝道。你這人好沒道理。打死人不要償命的麼。好意勸你。要問你話。怎這樣牛。殺人不過頭點地。他就有萬分不是。你打着。他不敢回手。就罷了。還要怎樣。你丈(仗)着漢子大行凶欺負他軟弱麼。那人見宦蕚裝束像個官長。責備他不是。方歇住手。向宦蕚道。老爺不知內中的情弊。俺打死這沒良心狗娘養的。情願替他償命。宦蕚道。你們爲甚麼大事。就這大的仇恨。那人見問。便恨恨道。老爺請聽言。事情雖小。叫作殺人可恕。情理難容。俺是山東人。俺名字叫作畢本。因家鄕荒亂。到了者(這)兒。又沒多大的本錢。只有十來兩銀子。做個貨郞。掙個饝饝吃。住在一個店裡。指着那捱打的道。這個沒良心〔狗〕娘養的。他叫作賴盈。也是俺一搭兒的人。同在店裡住着。他得了病。俺與他非親非故。看鄕親面上。替他請醫生吃藥。俺早晚得閒。還扶侍他。他身邊又無有一個大錢。俺旣照看他一場。只得替他擔着。他病了幾個月才好了。後來算了算。連藥銀店錢就該着六七兩。他身上又沒件衣服。寒冬冷月。只得又替他賒了幾個布同棉花。通共該八兩多銀子。這項銀子沒處出。他求俺替他借幾兩還了人。他去傭工掙了來還。俺一來看他還老實。二來是俺的首尾。只得向俺絨線鋪主顧哀求。俺作硬保。借了十兩銀子。纔還了人。剩下一兩多些。他留下盤費。原說定出去傭工。掙的多。陸續着還他本錢。就不能還本。年年淸他的利錢。也還可以行得。誰只(知)這沒良心〔狗〕娘養的。不知在那搭兒裡去了三年。躱得影兒不見。鋪子裡主顧依不得了。問我保人要。要打要吿。算起本利來。該他十七八兩。剛剛把俺的本錢作了去。我爲他連〔累〕一場。水也沒喝他一鍾。如今倒弄得我這半年來當了個乾淨。無穿少吃。我這條命不是被他坑送了麼。今日要不是撞着他。他還躱着呢。因此我情願打死這沒良心的。替他償命。老爺請說。叫人可惱不可惱。說了。又要掙着去打。宦蕚叫小廝拉住了。道。這怪不得你惱。必定有緣故。那里人的良心就喪到這個田地\footnote{宦蕚是以君子之心度小人。世上人喪良心者。猶不止此。}。等我問他。叫那捱打的過來。問道。你這人眞沒良心。人爲了你一場。你倒把他的本錢弄乏了。坑了他\footnote{賴盈當云。他的名字不好。原叫畢本。與我何涉。}。你就沒銀子還他。也該見他的面。怎麼還躱着呢。賴盈道。老爺上裁。人心都是肉做的。承他這樣的情。可還有躱着的理。我時運不好\footnote{這四個字。把天地間多少英雄豪傑才子能人屈死了無限。何況於賴盈。}。又是病枯了的人。做生意沒本錢。只好去傭工。但用一點力。就傷着了。定要病幾天\footnote{病魔專凌窮漢。余亦受此大累。}。人家都不肯雇。東走西撞。總弄不着一個錢。連口也餬不過來。人說不看吃的看穿的。老爺看我身上這個樣子。就見得我不是說謊了。因沒臉面見他是眞。何曾是躱着呢。如今他就打死了我。也沒得說。宦蕚向畢本道。他這話也像眞。若果然如此。情還可恕。畢本道。老爺不要聽他。這都是鬼話。俺只打殺了他。才出得這口氣。宦蕚道。不消。我有個道理。叫小子稱出十兩銀子來。宦蕚遞與畢本。道。這算你替他借的那十兩銀子的本錢。利錢算你倒運賠了罷。拿去還做你的貨郞。且餬日子。畢本道。甚麼話。他該銀子。怎麼叫老爺還。這個我不敢受。宦蕚道。我不是替他還銀子。如今世上人。至親骨肉在一個錢上還刻薄不過\footnote{不意宦蕚一貴公子。竟能洞悉世情。}。你同他不過是個鄕里。又非舊識\footnote{這一句又露出公子本像來了。豈舊識便有情義關切耶。}。你就在他身上用這一番的厚情。像你這樣的人。也就是難得的了\footnote{千眞萬眞。}。如今他負了你。不但你寒心。後來不肯做好事。就是別人。看見施了恩就遇着沒良心的人。反害了自己。誰人還肯學。我如今送你這銀子。見得好心還有好報。他雖負你一般。遇着我還了你。你後來或者還肯行好。就是傍人看着。也還肯發善心\footnote{宦蕚此語。直欲將這一片婆心充滿宇宙。使人人皆做好人。行好事。是聖賢心地。}。畢本還要推辭。旁邊有認得宦蕚的人。便道。這位宦老爺。去年捨了你們那裡來的鄕親萬把多件棉襖。搭了幾百間大棚與他們安身。成兩萬家銀子都捨了。可稀罕這點子。你受了罷。畢本忙道。原來就是救我們敝省的大恩人。我也有許多親戚受過恩惠。小人有眼不識泰山。慌忙要下跪。宦蕚拉住。道。多大事。不必多禮。又叫過賴盈來。道。你病與不病。我也不得知道。古人說。要飯吃靠天。有一種不知事的人道。黑心人倒有馬奇(騎)。熱腸人偏沒飯吃。這話信不得。世上事。何曾沒有沒良心的壞人享着榮華富貴。這不過是眼前花。焉知他後來不男盜女娼。子孫絕滅。好人雖目下貧苦。又焉知他後來沒有好處。要看這兩種人的收圓結果。才定得好歹\footnote{宦蕚這一番話。以聖賢爲心者。自然謂之有理。以刻薄爲事者。未免罵其迂呆。世人只圖眼前受用。身後那管他有結果沒結果。}。你把良心掏出來。以前事不必題了。你明年儘力去掙。不能全還。一年還他一兩。七八年也就把利錢還完了。你若掙的多。多還他些更好。果有良心。天必不負你的\footnote{不意此君竟成了個道學先生。}。你今生不還他。等來世變騾變馬塡還好麼\footnote{話雖有些和尚氣。然亦是理之所必至。此一段借宦蕚之口。欲勸醒世上沒良心之人耳。但恐忠言逆耳。沒良心者不但謂汚耳。反恨其饒舌。}。衆人道。宦老爺說的是好話。你聽着。賴盈也叩頭道。謝宦老爺。宦蕚把他拉起來。見他甚是襤褸。打開銀包。拈了有三兩來的一個派州錁兒與他。道。這銀子與你買件衣服穿。做個小買賣度着殘冬。開年去想方法。賴盈又叩謝了。就將那錠銀子雙手送與畢本。道。這是老爺賞我的。你請收了算利錢。我凍餓死也沒的怨。畢本道。這是宦老爺行好與你度命的。我如今何肯要你的。宦老爺同我們一個陌路。就這樣施恩。我同你到底是鄕親。那利錢我也不問你要了。只當我害病吃了藥了。要神天保佑。托老爺的福。我在這貨郞上。再去慢慢的掙罷。說着。就在腰中順袋裡取出他的借約來。當面撕掉了。道。從此畧(撂)開手罷。宦蕚見他二人如此。心中暗道。德能感人。我這幾兩銀子就把兩個人都化了。欣然乘馬而去。正走之間。到了一個店門口。見一個大漢。生得豹頭環眼。頦下一部虬髯。六尺四五身材。三十八九年紀。在那裡背叉着手。白眼望天。不住長吁短嘆。宦蕚見他凜凜一條大漢。像有十分心事一般。又見那店主在一傍陪着笑臉說話。覺有緣故。勒住繫韁。把馬蹄放慢了些。聽得那大漢道。俺這樣的男子漢。是少你的飯錢的麼。等俺的親戚來。自然一齊開發你。那店主陪着笑。道。怎麼敢說爺上少飯錢。但小店本錢短少。供應不來。求爺多少給些。以便預備爺的酒飯。那大漢道。俺身邊若有銀子。何用你說。實在難爲你。我豈不知道。但俺此時在客邊。何處去設法。復又長嘆了一聲。道。

\begin{quotation}

在家千日好。出外一時難。

\end{quotation}

宦蕚想道。看這人的相貌。是個塵埃中的英雄。定非落魄之人。趁他在窮途。何不結交他一番。遂下馬走到跟前。拱手道。尊兄高姓。貴處那裡。爲何在此長嘆。那人見他氣宇軒昻。也拱手道。小弟賤姓鮑。山東泰安州人。請問貴姓。那店主道。這位老爺是我們這裡有名行好事的宦老爺。那人道。聞名久矣。敝省的人常稱述三位的大德。不想今日在這裡幸會。宦蕚道。何敢當尊兄過譽。那人道。尊兄不嫌蝸陋。請到小寓坐一坐。宦蕚正要問他話。說道。弟正有事請敎。遂攜着手同到店裡一間客房內。重復作揖。然後坐下。宦蕚問道。尊兄有何貴幹。到此又有何事縈心。浩然長嘆。方才這店家說甚麼飯錢。不妨細細見敎。那人嘆了一口氣。道。小弟賤名鮑德。寒家雖不敢稱爲富足。也還有幾十頃地。將就也還過得。我家姑母年老寡居。只有一個家表兄。姓辛名同。自前歲販了幾千金貨來在貴處發賣。曾有信寄回。說在評事街行裡住着。不意他三年不回家。姑母憶兒成病\footnote{人家父母見兒遠出。無不望其速回。無奈兒子一去。將父母忘却。古詩云。慈母手中線。遊子身上衣。臨行密密縫。意恐遲遲歸。凡人子遠遊。當將此四句念熟。}。\endnotemark[1]恐差家人不的當。命弟前來叫他回去。弟來時也還帶了幾十兩金路費來的。因見途中貧苦無食的人甚多。傷心慘目。弟以爲到了這裡。尋見了家表兄。自然就有盤費了。遂將身邊的銀子三錢二錢的都散了貧人。僅存了些須路費。不想到了這裡。找到行裡去問。說在此住了將二年。又往湖廣去了。弟要往湖廣去尋。又不知他在那一府。又沒有路費。只得在這店中住着等他。一住三個月。杳無音信。弟又食量頗雄。一日酒飯肉菜之類。非三腥不能飽。前月有些衣服都賣了。打發了他的店錢。這個把月。實在沒處設法。又在異鄕。舉目無親。向誰吿貸。也怪不得店家瑣碎。他能多大本錢。復大笑。拍着肚子。道。倒被賤腹裝了他十來多兩在裡面。叫他如何供應得來。弟欲回不能。欲住不可。故不覺發嘆。不意驚動尊兄。宦蕚笑道。原來是爲這些微小事。弟若早遇尊兄。台駕也回府久矣。向店主道。鮑爺差你多少飯錢。店主道。額定三錢銀。到今日正四十天。共該紋銀十二兩。令小人如何擱得住。所以纔大膽開口向鮑爺說。宦蕚道。我從不曾聽見南京的店錢三錢一日。你不許欺生。店主道。小人開着店。怎麼敢欺生。別人每日只五分銀子。鮑爺一日要用肉五斤。酒十壺。這兩樣就是二錢五分。一日還得二斤米飯。油鹽小菜靑菜豆腐之類。算起來小人還是白伺候。一文錢還不得落哩。宦蕚向鮑德道。兄眞英雄也。他大笑道。弟所謂酒囊飯袋耳。何足爲道。宦蕚吩咐小廝。你稱十二兩銀子給店家。就叫店家快去叫一乘轎來。送鮑爺到我家去。那店主得了銀子。歡喜非常。鎖在櫃內。飛跑叫轎子去了。宦蕚因向鮑德道。這店中非尊兄住的地方。可到舍下去。別有商議。把行囊都發了同去罷。弟先到舍下恭候。鮑德道。平(萍)水相逢。怎敢當尊兄如此過愛。宦蕚道。我輩相遇。何必故作這套語。鮑德道。尊兄旣是豪傑舉動。弟亦不敢作腐頭巾的虛套了。宦蕚起身作別。吩咐一個小廝等着同去。鮑德同到店門口。宦蕚一拱手上馬。道。耑候尊兄的大駕了。他到了家中。就吩咐預備下酒飯。不多時。鮑德到來。讓到書房坐下。小廝們把行李也搬了進來。坐下茶罷。須臾就送上酒肴。二人對飮。鮑德是個豪爽的漢子。在店中每日那種飮食。不過充飢而已。就是那酒。也不過只算得潤喉。因囊中乏鈔。不敢大嚼。今到了宦家。見杯盤擺烈(列)。烹炰精美。況宦家的酒量素常善飮。又不是寒酸主人。也不謙讓。傍若無人。豪飮大啖。宦蕚見他這種的氣槪。倒也少見。慇懃相勸。酒飯吃畢。天色將晚。宦蕚叫取一副新鋪蓋來。鋪上與他睡\footnote{與下回宦蕚到鮑德家對看。如何相報之速也矣。}。留住了數日。無非大酒大肉相待。徹底做一身新衣\footnote{眞可謂賢主佳賓。這一身新衣。與司進朝替富新所做那一身新衣。兩人之心胸行事。何啻天淵。}。他所談講的。俱是談兵說劍武藝中的話。宦蕚雖不懂其中的妙處。倒也聽得津津有味。氣爽神豪。一日。宦蕚陪他飮酒之間。說道。弟喜得遇兄。本欲屈留些日子。但尊兄離家久矣。恐府上同令姑母懸望。目今趁初秋天氣。正好走路。尊兄還是回府。還是在這裡住着等令表兄呢。鮑德道。弟欲回久矣。自無路費。連日承兄見愛。又不敢啓齒。家表兄知他到何日才來。弟歸心似箭。也不等他了。只到行裡說下個信便是了。宦蕚道。尊意旣如此。明日即爲兄送別。鮑德大喜道。弟承尊兄過愛。我也不效那妄說感恩戴德的虛話了\footnote{丈夫語。}。但願異日得相晤暢聚爲樂耳。弟此時就往行中說個信來。宦蕚道。對他說。令表兄來時。竟請到舍下來住就是了。鮑德喜道。這更妙了。去不多時就回來了。宦蕚次早備酒飯與他餞別。他的行李也收拾完了。小廝捧出五十兩銀子來。送他作路費。鮑德道。何必用許多。一半也就夠了。宦蕚笑道。兄忘了前日之事了。途路間寬裕些好。設有不敷。又將奈何。他也笑着收了。宦蕚又吩咐一個家人道。你拿十兩銀子。送鮑爺過江。到浦口雇了騾子。看着起了身。來回我話。又叫備兩匹馬來。親自要送。鮑德道。不勞尊兄罷。宦蕚道。弟不敢留兄者。恐尊府懸望耳。然而惜別之心。哽咽於胸。送兄一程。多聚一刻。稍慰一刻鄙心。鮑德長嘆道。弟生平交人多矣。不意貴介中有尊兄這等俠腸義氣漢子\footnote{此語雖是誇宦蕚。却將貴介中人一筆抹殺。}。撫膺道。銘刻於我心矣。二人上馬。一路說着話。到了下關過浮橋。同到江口下馬。二人握手。依依不捨。鮑德上了擺江船。家人搬上了行李。那個送的家人也上去了。臨開船時。宦蕚道。尊兄長在途保重罷。鮑德道。尊兄請回罷。此身不死。容圖異日相會\footnote{感之至。一語勝千萬言。}。宦蕚看他的船去遠了。上馬悵然而返。正走着。將到三彈樓。見幾個人在那裡說笑道。那裡去看戲。這就是眞戲文了。那戲子們唱爛柯山的崔氏逼嫁。還沒有他這樣眞正行徑呢。宦蕚正勒馬要問。衆人齊笑道。朱買臣出來了。宦蕚看時。只見一家門裡一個破衣巾的文人。送出一個老兒來。也戴着一頂爛方巾。穿着一雙紅不紅紫不紫的沒後跟的破鞋。氣忿忿向那人道。我們家不幸。生出這樣不成器的女兒來。賢婿也不必氣惱。或留或休。任你的意思。我總不管。我像沒有生他的罷。宦蕚聽得有些詫異。忙下馬向那老兒同那人拱拱手。他兩個連忙還禮。宦蕚道。請敎府上有甚麼事。那老兒搖頭道。羞愧死人。我不能出之於口。指着那破衣巾的道。尊駕請問他。宦蕚看那貧士時。

\begin{quotation}

頭上爛爛一頂巾。以飯糝做補丁。而腦油浸透。脚下舊了兩隻襪。以黃泥爲漿粉。而脚底對穿\footnote{有人作謎云。天不知。地知。人不知。我知。是何物。他人不解。問是何物。彼笑云。我襪底有一洞耳。此貧生襪底對穿。宦蕚想當然耳。}。面皮黃皺。肉味豈止三月不知。顏色鏖糟。浴水料道六時不見。身上衣補空萬千。常穿不時之服。室中竈塵灰堆集。或煮飢後之餐\footnote{或字好。也是想當然。}。昔年買臣後身。今日妻休貧士。

\end{quotation}

宦蕚向那人道。請敎。那人道。賤姓平。就是齊家治國平天下的平字。賤名儒。乃汝爲君子之儒\footnote{開口酸腐之氣衝人。描寫迂腐措大。入骨三分。}。忝列庠序。這一位就是家岳。小弟自二十歲畢婚。今已十七年矣。賤內與小弟同庚。小弟一介寒儒。只靠筆耕餬口。不意兩年來。年成荒歉。沒人讀書。這硯田也就荒蕪了。去歲還將就苟延。到了今年。就力不能支。三旬九食竟是常事。在當初。竈下以不舉火爲奇。近日竟以舉火爲奇。眞正是空如懸罄。家徒四壁。古人云。啼豐年之飢。號六月之寒。不意此二語竟是爲小弟而設。不想賤內忍受不得。竟有個要別抱琵琶之意。原也怪他不得。冬日則飮湯。夏日則飮水。終朝枵腹。如何過得。他去意甚切。小弟多年伉儷。何忍分離。意有不捨。再四苦求。其如他塞耳弗聽奈何。賤內執意不回。小弟不得已求了家岳來。以大義責他。以好言勸他。他決意不從。適間反以不遜之言挺撞了家岳。所以家岳忿怒而去。宦蕚向那老兒道。令愛要去。不過是因令婿貧窮之故。老丈若可養活得女兒女婿。就可相安了\footnote{世人因女婿貧窮之故。連女兒皆棄而不顧者甚多。宦蕚作此言者。或疑及此。然見這老兒行徑。不問而知其窮。尚作此語者。方不脫是個公子本色。}。那老兒嘆了口氣。道。先生。先生。非我唐突得罪。你這眞是何不食肉糜之言了。我們當初弄了一頂爛頭巾戴在頭上。以爲是功名的一個進步。何等興頭。誰知吃他一生的大累\footnote{初進學時是頂簇新的頭巾。因你不能上進。把他戴爛了。頭巾不怨你足矣。如何反怨他。}。當初指望飛騰黃甲。脫却這蓋皮。就可以耀其祖而揚其宗。封其妻而蔭其子。大其居而改其門。華其身而充其腹\footnote{王恩是八其翰林。他又是個八其措大。}。不想毫不如意。其如命何。老學生自十五歲遊庠。鄕試過二十餘次了。那朱衣老先生在暗中。他那尊頭就不肯略點一點。那柳汁比金子還貴重。就不肯灑一滴在我寒士身上。拿輕不得。負重不得。不稂不莠。行動又要惜三分臉面\footnote{這老兒宜乎貧寒至此。偌大年紀。不知世務。世人但顧臉面。焉有不受窮者。}。家中釜甑生塵。兒啼女哭。眞有乞丐所不堪者。老學生今年虛度七十有五了。豈但三月不知肉味。孟夫子曾云。七十非帛不暖。五十非肉不飽。老學生比五十又多了二十五年。成年累月還不知何者爲肉。昔日聽得一笑談。

\begin{quotation}

一貧士終年食菜。一日。有人以羊肉餉之。夜夢五臟神云。羊踏破菜園了。

\end{quotation}

老學生今日求其踏破菜園而不可得。至於衣服。不要講衣帛。請看我這鶉頭百結。捉襟露肘的樣子。求寸布如異錦之難。其寒家之境況。可想而知了。自給猶無所措手足也。而況於女兒女婿乎。當日古人有一個淸江引。正合了老學生的近況。道是。

\begin{quotation}

三更半夜睡不着。惹得我心焦躁。趷蹬的響一聲。儘力子嚇一跳。原來是把一股脊梁筋兒窮斷了。

\end{quotation}

此乃我學生今日之謂也。宦蕚又問平儒道。你令政旣不願相從。就勉強留下他。也未必相安。終日吵鬧。也非常法。平儒道。小弟豈不知此。其如此哀不忍何。宦蕚道。迂。迂。眞迂。因見隔壁有個茶館。說道。二位請到那裡坐坐。我有話相吿。那老兒道。豈有此理。老先生駕臨敝地。豈有反客爲主之事乎。雖有欲奉屈之心。其如囊中無此力何。宦蕚道。不用謙讓了。請進去罷。二人進內。一同坐下。老兒道。請敎老先生貴姓。宦蕚道。我姓宦。老兒道。得非大司空宦老夫子令公〔子〕麼。宦蕚笑道。正是。那老兒復鞠恭(躬)道。眞今日翩翩之佳公子了。久仰。久仰。老學生翁婿何緣幸會。宦蕚笑道。多承謬獎。料道他們都是空腹。要了幾碟點心來。讓他二人吃了一會。道。我看你翁婿二位讀書一場。一窮至此。倒甚爲惻然\footnote{天下讀書之窮人何止億兆。惻然不得這許多。昔有一人云。天有富我心。賜我一塊金。方圓四十里。裡外不空心。余謂雖此一塊大金。猶不足以資給之。}。我此時就算資助你些。勸他留下。但不能常繼。用度完了。舊性復萌。仍然要去。又復奈何。我有個主意。你一位是他的令尊。一位是他令夫。我如此如此替你化他一化。將來能完全你家室之好。你二位說。可行得麼。平儒還有不忍。口中不住咨嗟。倒是那老兒道。宦老先生君子人也。何傷乎。他之尊意。可謂妙極而無以復加矣。賢婿把這不肖女總如棄了一般。何不聽其所謂。倘能革心改面。豈非爾室家之慶乎。平儒想了一會。嘆道。哎。小弟騎虎之勢。也出於無奈了。悉聽尊裁。還要求老先生稍加姑息。不宜督責太過。宦蕚叫小廝拿過銀包來。打開。撚了一錠約有三四兩。送那老兒。道。爲先生一肉一衣之敬。又拿一錠與平儒。道。權爲薪水之資。等你令政悔心之時。我再送來與你。那時或可相安了。設或惡性不改。我替你另娶一房。此等婦人終棄之亦可。問那老兒道。老先生。你恐怕還有愛惜不捨之心麼。老兒正色道。豈有此理。我老學生今雖窮乏。當初先祖權副使也是有名人焉。此等不肖之女。已在七出之外了。辱我儒門之父多矣。尚何惜乎。老先生雖將他鼎烹斧銼。我學生猶不過而問焉。何況於化惡爲善耶。但旣承賜茶。又蒙厚惠。何以克當。誠所謂却之不恭。受之有愧了。宦蕚道。不必過謙。請收了罷。我回去。就有人來。他翁婿深深一揖。道。承愛了。大家同出了茶館。宦蕚別了他二人。上馬來到了家中。將權氏的事吿訴了侯氏。侯氏又是那好笑。又是那恨。宦蕚道。我因他們想起一個笑話來。

\begin{quotation}

一個人家請了一個先生。窮得很。他要回拜東家。沒人拿帖。叫他老婆扮作家人隨去。到了那裡。賓主甚是相投。款待酒飯。定要留宿。那先生辭不脫。只得住下。東家叫兒子陪先生睡。叫館童陪那家人睡。次日。先生回去了。其子向父親道。老先生倒好。只是窮得很。昨晚脫衣服睡覺。連褲子都沒有。那館童接口道。他那家人。不但沒褲子。窮得連雞巴都沒有呢。

\end{quotation}

這個笑話正好贈那平秀才。侯氏又笑了一陣。宦蕚吩咐家人叫了個媒婆來。如此如此對他說了。叫小廝領他到平家去。到了他家。此時平儒受了宦蕚的計策。躱在外邊聽信。那媒婆走到裡面。向那婦人道。這就是平奶奶麼。權氏道。我如今不是平家的人了。你是那裡來的。媒婆道。我是南京城裡第一個有名做媒的趙大嫂。人都叫我趙老實。城裡的張富翁。李財主。程閣老。宦尚書。這些大老財主家中。我沒一家不走動。聽得說這裡奶奶要嫁人家。又賢慧。又會當家。如今有一位財主鄕紳要娶一位奶奶續絃。托我來說。那權氏一臉的笑。道。我雖說要改嫁。又沒有口風出去。怎麼人就知道。媒婆道。這位財主要尋位好奶奶久了。托的人甚多。他同你這一位街坊姓甚麼甚麼呢。我就忘了。他兩個是好朋友。聽得他說。故此才煩我來。奶奶。你旣翻身一場。不要錯過了這樣的好人。家中穿紬着緞。揷金戴銀。使奴喚婢。你到了那裡。眞是飯來張口。水來濕手。受用一輩子呢。權氏滿心歡喜。笑道。他家姓甚麼。媒婆道。他姓賈。滿城中誰不知道賈鄕宦家。權氏道。這也等我那倒運的漢子來。對他說明白了着。媒婆道。你不要癡了。一面摹旗。一面擂鼓。只要你心肯了。我回他一個信去。送了衣服頭面來。等你家相公回來說一聲。就走上了轎子。還怕他拉回你來麼。權氏道。他這樣個大人家。也不行財下禮。難道就是這樣烏嘴烏面的擡了去。媒婆道。你是自己做主。要下禮做甚麼呢。擡了來仍要擡了去。況且你是有丈夫的。那時驚動了街坊鄰舍。閒言雜語。攔阻起來。反倒不妙了。權氏道。你的主意也是。但恐我那倒運的漢子不肯放。怎麼處。媒婆道。他要留你。你就叫他拿好衣服來你穿。買東西來你吃。怕他不叫你去麼。權氏道。就依你說。幾時可行呢。媒婆道。打破頭。趁熱揉。俗語說。停留長智。過後又怕生枝葉。要去就去。你主意要決了。今晚就去做新人。早一刻。不受用一刻麼。因走到跟前。附耳低聲道。說這賈老爺有名的大陽物。笑道。你夜裡被窩中更受用呢。我總成你這樣好去處。過了門。十兩媒錢。一分也少不得的呢。權氏歡天喜地。反再三囑托道。我在家同那倒運的坡(扳)倒身子。講個決斷。你今晚千萬的要來接我。那媒婆道。我知道。還用你說麼。平儒在外面見媒婆去了。便來家。權氏放下臉來。道。我不是你的人了。我今日晚間就要去的。你要留我。就去買綢緞來替我做衣服。買好飮食來供給我。不然。你要強留我。不是你死。就是我亡。這苦日子我實在過不得了。平儒道。你到底往那裡去。我同你將二十載的夫妻。你就忍得撇我麼。權氏冷笑道。古人說。酒肉的兄弟。柴米夫妻。沒穿少吃。我同你就是陌路了。還講甚麼恩情。有兩句古語說得好。

\begin{quotation}

將軍不下馬。各自奔前程。

\end{quotation}

我的去處不勞你管。大約自然比你府上強些。平儒道。你旣主意已決。諒也不能留你。也有兩句古語。道是。

\begin{quotation}

心去意難留。留下結寃仇。

\end{quotation}

你去是去。但只是你後來或有不得意處。千萬還來尋我。權氏夾臉唾了一口。道。啐。你替我發這樣好利市。難道別人家還有不如你的。我就死了。也不再上你的門。你可曾聽得說。回爐的燒幷(餅)不脆麼。正說着。那媒婆夾個氈包進來。道。轎子來了。權氏向平儒道。你快寫休書給我。不要誤了我的良辰。那平儒也不作難。寫了休書。權氏又叫念與他聽。無非是養贍妻子不過。任憑改嫁的話。權氏又叫他打了手印\footnote{老作家。}。收了。渾身徹底換了衣服。戴上首飾。向平儒道。你生平可見過這些東西。歡歡喜喜。頭也不回。上轎而去。有四句說他二人。道。

\begin{quotation}

平儒今日被妻休。崔氏當年醜已留。

何是琵琶貪別抱。雎鳩不肯在河洲。

\end{quotation}

因這權氏。有一調駐雲飛嘆世人夫婦。道。

\begin{quotation}

夫婦恩情。結髮髫年到百齡。舉案齊眉敬。全仗家豐盛。哎囊罄沒分文。難逃怨恨。口縱無言。勉強身相順。試看那實在心安有幾人。

\end{quotation}

那權氏被轎夫一直擡到宦家。下轎時。媒人不知何往。只見四五個婦人叫他出轎來。擁他入內。到了上房。宦蕚同侯氏高坐。衆婦人道。與老爺奶奶叩頭。權氏興抖抖來做財主奶奶。忽然見這個光景。心中鶻突。衆婦人又道。你見了老爺奶奶怎麼還站着。好不知規矩。還不快叩頭。他見丫鬟僕婦左右圍繞。尊嚴得了不得。不由得雙膝跪倒。還疑是哄他來做妾。叩了頭起來。宦蕚對司富道。這個婦人萬刁萬惡。嫌貧休夫。被他父親賣到我府中來。交與你名下收管。叫他做各種活計。磨磨他的刁性。若稍有完(頑)劣。拿皮鞭着着實實的打。拉了去。把衣服換了。衆婦人拉他過去。換了一身舊布衣服。他此時已入圈套。悔之無及。又帶了過來。稟道。換過了。司富就帶他到廂房內。道。你就跟我在這裡住。就派了些活計與他做。說道。都定有日限的。遲誤了。十個皮鞭。他一心打點來做奶奶享福。今到了這個光景。又不知是甚麼人家。又不知是如何來的。聽說是他父親賣了他來。想道。我一個出嫁十多年的女兒。父親如何賣得我。我丈夫怎又不說。不明不白。心中又悔又恨。那媒婆不知從何而來。今又不知何往。暗暗哭了一會。夜間悄悄起來上吊。不想司富他們都是商議過了的。有心防着他。一聲喊叫。救了下來。到次早。稟了宦蕚。宦蕚大怒。叫了十數個僕婦。將他按倒在地。剝去衣服。只剩一衫一褲。大皮鞭細竹條。自頸至踝。足足打了數百。侯氏再三說情。方纔饒了。吩咐一個僕婦繆氏監管着。餓他三天。不許給他飯吃。那權氏渾身打得如菜花蛇樣。擡了去。放在床上臥下。皮膚無處不痛。想起當日雖窮。丈夫何等憐愛。今日受此苦楚。是自己尋來。只好自怨。那心腸也就悔了兩分。那繆氏私自拿東拿西與他吃。待他甚是親熱。悄悄勸他道。你旣到了這裡。揷翅也飛不出去。人說螻蟻尚且貪生。你怎麼尋此拙見。討這一場苦吃。寧在世上捱。莫在土裡埋。焉知日後就不捱出個好日子來。你不要呆想。你死在這裡。不過像死了個螞蟻。誰還可憐你麼。你耐心守着。少長缺短。悄悄對我說。我照看你。權氏感激不盡。好了起來。不是做針指。就是漿洗衣裳。雖不叫他上去伏侍。也沒有一日得閒。自從捱過那一場肥打。也不敢再想尋死了。看見別的婦女都忙忙碌碌。終日做活。久之也就慣了。宦蕚憐平儒是個貧士。時常週濟他。後來開義學時。轉托梅生約到他家。考了考他腹中學問。也還頗通。就請了他做先生。在館中敎學。這是後話。一日。宦蕚在家。門上傳進來說。有一個姓辛的山東人要見。宦蕚知是鮑德的表兄了。忙走出來迎着到書房。相揖坐下。宦蕚看他面白黃鬚。狼腰虎背。細條身材。也好一個相貌。他動問鮑德的信。宦蕚將店中偶遇。接了來家。留住了數日。並打發起身回去的話說了。道。去了兩個多月。大約久矣到家了。辛同再三致謝。宦蕚又道。尊堂在家懸望。兄也當速回才是。湖廣這一次的買賣定然是得意的了。他蹙額道。去的時候生意倒也甚好。聞得貴處米價湧貴。在湘潭販了幾千兩銀子的米下來。不意途中遇了張戲(獻)忠的賊兵。搶掠一空。小弟落在水中。性(幸)喜自幼頗知水性。逃得性命。只剩孑然一身。行囊俱失。虧得別船一個老客見憐。帶了下來。昨晚才到。且到舊行家看看可有鄕親在此。問個家信。他言舍表弟曾來過。臨去時留下信。若小弟來時。叫到尊府來問。故此特來驚動。宦蕚道。旣如尊言。歸途盤費何以設處。辛同道。爲今之際(計)。沒有別法。除非向舊行家借貸些須。還不〔知〕他可肯慨諾。宦蕚叫家人取了三十兩銀子來。說道。本要奉留盤桓數日。恐尊堂得了令表弟的信。越發盼望。些微路費。可以到府了。今日尚早。就請渡江。雇了頭口。星夜回府罷。到家致意令表弟。容圖後會。辛同道。蒙尊兄盛情。愚弟兄言謝不盡。小弟也不敢假作謙辭。竟拜領大德了。就此拜別。小弟即刻長行矣。宦蕚留他吃了酒飯。送到門外而別。倏忽秋盡冬來。大雪初霽。宦蕚出門。要遇好事做一兩件。信着馬蹄。緩緩而行。大街小巷串了一會。走到一條僻靜巷內。見一個人兩眼哭得紅紅的。身上穿得甚寔(是)單寒。打門內送出一個人來。含淚囑道。事求速些爲妙。那人道。我知道。明日定有回信。拱拱手去了。這人又掉了幾點淚。嘆了一口氣。擡頭望望天\footnote{望望天。妙甚。欲開口吿人。無門可訴。欲吿之於天。奈天又高而難聽。只得嘆氣望望而已。寫盡窮人苦楚。}。慘慘悽悽。折身進去。宦蕚想道。這人雖穿得襤褸。形狀舉動像個正經人。定有萬不得已的事。方這樣傷心。我問他一問。或有急難。我何不救他一救。遂打着馬進他院中來。那人來到房門口。正要推門進去。聽得後面馬蹄子響。回頭一看。却認不得。見他肥馬輕裘。又跟着兩個小廝。忙迎了過來。問道。老爺尋誰。宦蕚下了馬。一拱手。道。就是來〔尋〕你。那人驚道。素不曾拜識過尊顏。老爺下降。有何吩咐。宦蕚道。且到你屋裡去講。那人道。寒家不堪得很。故此不敢奉讓進去。恐屈了尊。宦蕚道。這有何妨。那人見說。只得推開門。讓了進去。宦蕚到裡邊一看。果然不堪之甚。兩門透風的房子。四面牆上大洞小眼。頭頂上還有幾個天窗。朔風凜烈。刮得颼颼聲響。大嚴冬天到屋裡。連個火星兒也沒有。兩張破板床上。鋪着兩床破草簾。還鋪着破竹蓆。連被也沒有一床。床上蹲着兩個婦女。還有兩個孩子。都穿着稀爛的衣服。肉都露出在外邊。抖抖的戰。那人掇過一張破竹椅。撣淨了灰。讓宦蕚坐下。宦蕚道。你也請坐了好講話。他謙讓了一番。然後拿了一條三隻脚的板凳坐下。宦蕚道。兄貴姓。他道。不敢。賤姓向。賤名惟仁。不敢拜問老爺上姓。宦蕚道。我姓宦。向惟仁道。想就是去歲捨衣服救窮人的宦大老爺了。宦蕚笑道。怎麼這點小事人都知道。向惟仁道。久仰老爺大名了。老爺是貴人。下臨賤地。有何吩咐。宦蕚道。我才在門口過。看見兄送出那個人去。滿面慘容。必有萬不得已的事。特來相問。向惟仁但低頭嘆氣。一時不便回答。宦蕚道。兄何妨從實吿我。不須隱諱。向惟仁道。承老爺殷殷下問。只得要直稟了。寒家當日也還可以將就過得。做着千金的買賣。向日也曾爲過人。連年運氣不濟。做着的就折本。連舊房子也賣了。尋了這兩間破屋棲身。數年不曾修葺。越發倒敗了。因前歲借了阮大鋮老爺府上銀五十兩做本錢。又遇着這兩年年程荒歉。人口多。就吃掉了。如今三年整。本利該他百金。終日來索。沒得還他。他的管家看見小女生得乾淨。回去說了。阮大爺要拿小女去學戲。准算本利錢。小人怎肯把親生骨血送去做這樣下流的事。苦苦不依。他前日惱了。把我送到縣中追比。我求人保了出來。限十日內還他。老爺請看寒家這個光景。開門七件事。件件都斷了。煙火俱無。一家都是不久的了。可還有這百十兩銀子還人。沒法。怕受凌辱。要尋一死。二來不忍見家中這個樣子。死了。眼不見爲淨。就罷了。說到此處。就哭起來。宦蕚道。不必傷心。有話且講。他擦了擦眼淚。指着床上那女兒道。我這個小女。他說小人一死。如水桶散了箍的樣。一家人都是要死了。他情願自己賣身。不論爲妾爲婢。但求多得幾兩銀子。還了阮府。倘餘剩下些。叫小人做個小買賣。帶着他母親兄弟將就過活。小人生他一場。指望嫁一個好人家。與他去完他一生一世的事。怎麼忍心賣他與人爲奴作婢。雖然顧了一家。豈不把他坑死了。又哭起來。道。他見小人不肯。倒要尋起死來。說除了此法。一家都是要死的。他不若先死了。免得眼見難過。小人只得依他。尋人說合。就是小人方才送出去的。那是個官媒。他說有個過路的官兒要買妾。只要人物生得好。倒不惜身價。來問小人可捨得賣到外路去。小人還不忍。是小女說。倘本地人出不上價。他白捨了身子。仍舊救不得父親母親兄弟。只求多得幾兩銀子。就是外路去。也說不得了。況且在本鄕本土。或有好歹。恐父母知道。反要傷心。一狠百狠。遠遠的去。只當死了。割斷了肚腸。倒還好些。小人思量他這些話也說得有理。只得依了他。養他一場。落了這樣個下場頭。怎不叫我做父母的心中像刀割的一般。怎不悲慘。說着。越發悲慟。宦蕚道。好孝女。好孝女。難得。難得。請你令愛來。我問他一問。向惟仁叫他女兒道。我兒。過來見了宦老爺。那女子羞羞慚慚的下床來。走到面前。拜了一拜。宦蕚把他一看。雖然穿着一件破補丁藍布衫。一條鋸齒邊的破裙子。好個標致端莊的女子。有一首一斛珠的詞兒以咏其美。道\footnote{石崇在雙角山以一斛珠換得綠珠美人。曲牌名因此而起。今以爲詞贊佳人。合拍甚妙。}。

\begin{quotation}

曉霧輕籠。晴山淡掃妝雖草。舊敝衫裙偏覺好。朱顏旣妙。那用梳妝巧。海棠夢裡醉魂消。柳葉簾前體態嬌。桃花面上含悲悼。試聽纖喉。上花鶯聲小。

\end{quotation}

一點脂粉也無。全是天然本質。眞是秀色可餐。若再裝飾起來。可稱個十全的佳人了。但只是臉上寒毛都凍得直豎豎的。眞令人可憐。宦蕚問他道。小姑娘。你今年十幾歲了。他朗然答道。癡長十六歲了。宦蕚道。我纔聽見你令尊說你這一段孝心。誠然可敬。但與人做妾。也是一件大苦的事。若遇了不賢慧的大妻。一日也難過。你這樣個嬌生慣養的柔軀。倘不幸遇了那樣悍妒之婦。豈不斷送了。你年紀小小的。可曾想到這上頭麼。他答道。我何嘗不知道。我當日聽得家母舅講書。殺身成仁還要去做。何況捨身救父母兄弟。也說不得了。今日且救了一家。後來就到那個地位。就死也瞑目了。強似今日眼睜睜看着這個樣子。肝腸痛裂。一刻也是難過。眞是生不如死之時了。也就淚隨言下。宦蕚先就想要救他父親。今聽他說了這番話。激出一段熱心來。道。你這樣孝女。我若不救援你。空做鬚眉丈夫。枉在世上爲人了\footnote{枉在世上爲人者。恐十有八九。}。叫小廝拿過銀包來。內中約有十數金。遞與向惟仁。道。這幾兩銀子。你今日就去買些柴米炭火。再買幾件棉衣服來。你一家大小穿上。你去回那媒人。也不必題我的話\footnote{行好不欲人知。方謂之陰德。}。只說你遠處來了個親戚。助了你百金。不賣女兒了。再約了你當日借銀子的保人。明日早飯時等着。我明早到你家來。與你一份銀子。你拿去還了阮家。就淸白了。向惟仁道。蒙老爺天恩。小人也不敢假做推辭。但一家來世變畜生補報罷。遂叫他妻子空氏同女兒並兩個兒子道。快來叩謝恩人。他一家歡天喜地。忙過來跪下叩謝。宦蕚一手拉住了向惟仁。那妻女二人又不好伸手去扶。急得只叫快請起來。衆人叩完頭站起。宦蕚道。我是救孝女的。與你們無干。何勞道謝。說着。就出來上馬而回。次早。帶了銀子到向家來。下馬。向惟仁聽見。忙開門讓進。到了房中。與昨日大不相同。幾萬個補丁的窗子也糊亮了。地下一個瓦盆燒了一盆大火。鍋內熱氣騰騰。一家都穿上了棉衣。床上疊着兩床舊布被。忙讓了宦蕚坐下。那女兒也就走到跟前站着。宦蕚看他時。穿了一件紫布棉襖。靑布背心。白布裙子。比昨日體面了許多。說道。天氣冷。小姑娘你請到火盆跟前坐着去罷。向惟仁道。老爺天恩。小人一家今日都到了天堂了。今再要說冷。可就眞折福了。宦蕚叫小廝拿那兩封銀子來與他。道\footnote{此書之細。令人容易看不出。銀子則銀子矣。而曰那兩封銀子。不過是一句話。就不知那者。還有這也。後來又取兩封。一與向小娥。一與向惟仁。方悟那字之妙。}。這是一百兩紋銀。你拿去還他。你保人約下同去不曾。向惟仁道。昨日就約定了。他在家中等。宦蕚道。如今人壞的多。還你的文書時。須看明白。不可被人哄了。向惟仁道。蒙老爺吩咐。小人知道。宦蕚又叫小廝把包內的碎銀子拿了有三兩多。遞與他。道。把這銀子你另外拿着。恐怕他拿廣法馬兌你的。就要個大加三。那時少了。爲這一點子又爭論。仍不得淸楚。向惟仁道。老爺的恩典。想得這樣全美。宦蕚道。你去了快來。我還等你回來說話。那向惟仁剛跪下要叩謝。宦蕚拉住。道。不消多禮。你去罷。他拿着銀子忙忙的去了。那女兒篩了一鍾茶。纖纖玉手奉與宦蕚。宦蕚欠身接着。道。又勞動你。吃罷。他接了過去。便道。天氣冷。老爺來的早。恐還不曾用飯。我家備有一杯水酒。老爺不嫌棄。請用一杯。宦蕚道。我怎好叨擾。他道。我一家吃的穿的都是老爺的。這還是老爺擾的是自己。等我們父子有得孝敬老爺的。日子就好過了。說着。就去將燙酒的壺放在火盆上。他將靠南窗的一張抽屜桌子擦淨。說道。老爺。請過來坐罷。宦蕚站了起來。他忙把竹椅掇過。靠桌正面放下。開了抽屜。拿小菜碟兒。宦蕚一眼看見抽屜內有些舊書。問道。這書是誰念的。他笑着答道。是我小時念的。宦蕚道。原來你也從過師。怪不得這樣知道孝順。通文達禮呢。他道。老爺取笑。我知道些甚麼。當日我母舅敎館。帶着我念了幾年。因家寒。搬到這裡來。那時就不念了。我才得十二歲。今年也撂下將四年了。說着。讓宦蕚坐下。酒也熱了。他斟了一杯。雙手捧着。笑盈盈遞上。道。這街上沒有好酒。老爺將就用一鍾避寒罷。宦蕚忙接了過來。道。小姑娘。你去坐着罷。叫我的小廝來伺候。他道。我一家蒙老爺莫大之恩。就終日爲奴爲婢。也是該當的\footnote{辱翁曰。此時已有願到他家之心了。}。何況在寒家。理當服侍的。他母親把鍋揭開。原來是大葷館裡買來的四品上好美肴。怕冷了。蒸在鍋內。並一盤果饀狀元糕。端來擺上。宦蕚道。你何故費這些事。他道。家寒沒有甚麼敬的。買的現成東西。恐不可口。老爺休怪。宦蕚讓坐。他再三不肯。宦蕚道。你不坐。我也不吃了。叫小廝將板凳拿過來。放在橫頭。讓他坐了。又叫小廝拿了杯筯來。斟了一杯。讓他吃。宦蕚又問起來道。你當日讀到甚麼書。他道。讀過四書詩經。皆念完了\footnote{宦蕚當問他可曾讀過人之經。}。宦蕚道。你撂下這幾年。也還記得麼。他道。我時常翻翻。也還認得。宦蕚將抽屜拉開。順手拿出本書來一翻。中間夾着許多字倣。打開一看。寫得甚是秀美。覺得比自己的強好些。看見臨了寫着向小娥習。問他道。這是你的名字麼。他笑道。我母舅說古時浙江有個孝女叫作曹娥。要我也孝父母。故起名叫做小娥。正說話之間。向惟仁回來了。將文書遞上與宦蕚。道。蒙老爺大恩。小人的銀子還了來了。又跪下來叩謝。宦蕚一把拉住。道。你只管這樣。倒叫我不安。讓他坐。家中再無第二條板凳。就同女兒一凳坐着。忙敬了宦蕚一杯。飮過。又讓了兩筯菜。宦蕚將那文書遞與他。道。這一張紙幾乎坑了你令愛。快快的燒掉他。向惟仁接過。送入火盆內燒了。宦蕚對他道。你這令愛原來又識字通文。我看他眞是萬中選一的女子。他也不小了。你替他尋個好女婿要緊。不要貪圖豪富。若配得個詩禮人家的子弟更好。不然。就是買賣人家。只要揀個誠實的女婿就罷了。古人說。相女配夫。萬不可錯配了人。誤了他的終身\footnote{宦蕚說此一段擇婿良方。眞愛惜小娥之至矣。}。叫過小廝來。把那兩封銀子拿出\footnote{所以先兩封有那字也。}。先拿着一封。對向惟仁道。這二十兩銀子是送你令愛的。他也大了。你替他做幾件衣服。該置辦的甚麼妝奩小器皿並鞋脚之類。也替他備下些。等有人家。到出嫁時。來對我說。少長缺短。我再幫你。向惟仁忙叫女兒拜謝。宦蕚不肯。止住了。又拿過一封。對他道。我看你家中一無所有。何以度日。這是五十兩銀子。你做個生意。將就過日子罷。向惟仁道。蒙老爺昨日賞了銀子。今日又替小人還了債。已救了一家人的性命。使小人夫妻子女白骨再肉。眞是重生父母。天高地厚之恩。已是殺身難報。今又賞了小女。恩已過厚了。如何又敢領這厚賞。宦蕚道。救人須救徹。你不得這項銀子做本錢。家中將何以爲生。不久又是昨日那個光景。不如我不救你了。你收了。不必多辭。宦蕚與向惟仁眞是。

\begin{quotation}

濟人須濟急。救人須救徹。

不如拿雲手。網羅誰解結。

\end{quotation}

向惟仁道。老爺天恩。替小人慮得如此週到。小人一家粉身碎骨也難報涓涯萬一。又叫妻子大小來叩謝。宦蕚立起身。道。你要這樣。我就去了。向惟仁忙道。小人遵命。老爺請坐。他父女讓着宦蕚吃酒。向惟仁道。老爺明見萬里。洞察小人肺腑。剛才若不是多帶那幾兩銀子去。事還不能完。饒說把那都添上了。他還道少。費了多少唇舌哀求。才肯依了。因嘆了口氣。咳道。老爺施恩的又過於太厚。他刻薄的又太覺利害。宦〔蕚〕道。阮大鋮不知殺過多少大臣。何況這些微利害。說着話。又吃了數杯。就不吃了。向惟仁道。大淸早。小人也不敢多敬。請用飯罷。送上飯來。吃畢。撤去與小廝們吃。宦蕚吃着茶。向着小娥道。前日有個人送了我幾隻湖筆。幾匣徽墨。我用他不着。改日送來與你寫字。不要丢住了可惜。小娥笑道。我會寫甚麼。不過是亂揚。玷辱了那好筆墨。少刻。兩個小廝吃完了。宦蕚起身。道。多擾了。向惟仁道。老爺空坐受饑。怎敢當個擾字。他父女同送了出來。宦蕚道。外邊冷。小姑娘。你進去罷。那小娥竟有個依依不捨的光景\footnote{古云。女爲悅己容。宦蕚之於小娥。可謂憐惜親愛之至。小娥一慧心孝女。旣感救父之恩。又感憐己之德。安不心爲之死。}。宦蕚去後。向惟仁隨後就到宦府叩謝。回來。他夫妻感謝。念之不盡。道。天地間怎有這樣好人。我們的造化。救了我一家性命。若不是他。此時父南子北。不知成個甚麼光景了。望着女兒道。這都是你一點孝心。感動天地鬼神。所以才遇了這位大恩人。若是沒有神靈。怎麼可可的我送出媒人去。恰巧就遇着他。二來也是你一點造化。小娥總不作聲。低着頭尋思。向惟仁道。你不作聲。想甚麼事呢。小娥忽然道。女兒想來。蒙他這個恩德。生生世世是再報不盡的。我當日原是捨身爲父母。今日何不將我送與他去。也可報他萬一。不強如賣到他鄕外府。父母兄弟不能見面麼。向惟仁大喜道。你說得有理。我早有這個心腸。只說不出口來。恐兒女抱怨。好說外人倒救了你。我做父母的又把你送去作低伏小。你主意旣如此。我與你置幾件衣服簪棒之類。我夫妻同送你去。向惟仁到街上做衣鋪中。買了幾件紬絹棉夾衣服。紬裙背心之類。又到首飾樓上換了數樣簪環。又買了些零剪紬子回來。趕忙做小襖中衣。新鞋褶褲等項。數日完備了。叫兩頂轎子來。他母女二人坐着。囑兩個兒子看家。他跟着同到宦家來。宦蕚不在家中。門上人說了進去。侯氏叫嬌花嫩蕊領着僕婦們。接了他母女進來。向上就要叩頭拜謝。侯氏忙忙攙住。讓他坐下。空氏道。小女是送來服侍奶奶的。如何坐得。侯氏問起緣由。空氏細說起女兒要賣身。蒙宦老爺救他。並與銀子。救了一家子患難。今女兒情願來服侍的話說了。侯氏看那小娥。生得模樣又好。舉動又端莊。着實愛他。定要他坐。說道。就是留你。我也不肯看低了你。況你此時還是客。那有個站着的理。小娥道。雖蒙奶奶開恩。我怎麼敢。侯氏定然不肯。他方把杌子挪在背後坐着。侯氏笑道。你過來好說話。小娥道。奶奶的恩典。這裡坐就儘夠了。侯氏倒把座兒橫過來。和他一長一短的說話。心中十分相愛。那向惟仁也在前廳守候。不多時。宦蕚回來了。向惟仁上前復又拜謝。宦蕚拉住。道。你的禮數太多了。你來有甚麼話說。可坐了講。向惟仁不肯坐。將他夫婦親送女兒來與他爲婢的話說知。宦蕚道。怪道我才進來。看見大門外有兩頂轎子。原來是你家的。你這一番的舉動。把我一片好心都沒了。難道我是看上你的令愛才做這番事的麼。向惟仁道。這出在小人夫婦並女兒心中。稍報大恩萬一的意思。宦蕚決定不肯。他苦苦哀求道。老爺不留下。小人一家寢食也不安。就是小女他一心情願。也不肯中止的。宦蕚倒沒法起來。道。也罷。你且請回。再作商議。他方才去了。宦蕚進到內中。他母女都過來見了禮。侯氏道。他如今送了女兒來。你的意思怎麼樣。宦蕚道。這如何行得。他父親剛才在廳上熬了我這一會。我活落話兒回他去了。我當日一點好心救他。不忍把他女兒與人作妾。我今日若要了他。不如當日不救他了。可成個人做的事。侯氏道。這也是他夫妻父女一點好心。你留下罷。他母親在這裡儘着哀求我。我想來。雖然說你一點好心腸救他。此時若是你去要他。那就不成個人了。他送了來。也還與理無礙。我看好個有福的孩子。我心裡很疼他。你不要當我吃醋。故此不要。宦蕚道。你雖然如此賢德。但這事萬萬不可。我若留了他。把以前一片熱腸盡付流水了。那空氏見不肯留他女兒。跪在地下纏着苦求。宦蕚叫嬌花拉着他。那裡肯起來。一轉身。小娥也跪在地下。忙叫嫩蕊攙他。也不肯起來。侯氏笑道。你看他母女這樣眞心實意。你留下罷。宦蕚沒奈何了。便道。你請起來。我留下就是了。那空氏方才站起。小娥也就站起。侯氏叫拿酒飯來款待他母女。小娥不肯同吃。侯氏再三再四叫他在桌橫頭坐着同吃了。空氏起身道謝作吿辭。宦蕚叫他把女兒帶回。他那裡肯。說道。老爺。大人口裡無戲言。方才旣留下。此時如何又叫我帶去。宦蕚見他不肯。只得把小娥留下。打發一個小廝送了空氏回去\footnote{細極。此等處。他小說所不能及在此。似此雖極沒要緊的事。亦必定寫得有道理。向惟仁先回。小娥留下。單叫空氏同轎夫回去。可還成個大家行事。着小廝送去。方成體也。}。到晚間。宦蕚叫丫頭們西屋裡鋪了一張床與小娥睡。他仍同侯氏共臥。侯氏道。你怎不去伴新人。宦蕚道。你當我要這女子麼。方才是被他父母纏得沒法。只得留下他。過幾日。送他回去。我旣救他。如何又肯要。你這樣賢慧。我要尋小時。那裡尋不出來。怎肯把這個孝女拿他作妾。侯氏聽了此話。心中也着實敬他。暗暗贊他的好處。次日。宦實老夫婦聽見了這些話。也心中甚喜。暗道。我兒果然竟成個大好人了\footnote{兒一變至於好。}。可見做好人也不在乎讀書\footnote{宦老此言迂甚。豈讀書者便是好人耶。有大通的人偏用其才。那心地比不讀書者更壞。古今來不勝屈指。}。他與童家賢姪都是一竅不通的。所作所爲都是那大通的人所不能爲。不肯爲者\footnote{不能爲。其罪猶可言也。不肯爲。則罪不可言也。}。心中暗喜。這小娥一些也不裝生。每日絕早起來梳洗了。就到侯氏的跟前。好不慇懃小心。侯氏倒着實心愛。捨不得他。每每勸宦蕚留下。宦蕚執意不依。他也沒法。宦蕚替小娥做了兩套衣服。侯氏又與了他幾件頭面戒指之類。過了幾日。那日宦蕚又拿了十數兩銀子。請過小娥到跟前。說道。你住了這幾日。沒甚麼送你的。這是兩套衣服。幾件首飾。你拿了穿戴去罷。這是十來兩銀子。你拿着。後來出嫁時。添着買些嫁妝。又是兩帖筆。兩匣墨。道。這是我前日許你的。我今送你回去。替他拿他的包袱都包了。那小娥道。我父母送我來服侍老爺奶奶。如何又叫我回去。宦蕚道。小姑娘。你是讀書明理的。我爲你一場。你雖然要做個感恩報德的好人。倒叫我做個貪淫慕色的壞人麼。你心何忍。那小娥起先來時。所慮者恐侯氏不容。不能相安。今見大奶奶疼愛他無比。一心要在這裡。忽見宦蕚叫他回去。但他是個女孩兒。怎好賴在人家要與他做妾。只得聽他。不由得淌下淚來。宦蕚見他這樣戀戀不捨。心中也甚難過。對他道。承你父女這等好情。我家奶奶又如此賢慧。我難道是鐵石心腸。當眞不愛你麼。只是理上行不去。故此忍心割捨。你不要哭。好好去罷\footnote{宦蕚愈憐愛之甚。則小娥愈感之深。更不肯去也。}。叫僕婦替他拿着衣包。宦蕚站起。親自送他。他又與侯氏叩頭。侯氏扶起他來。心中十分難捨。也有個墮淚之意。那小娥哭哭啼啼出去。上了轎。宦蕚叫跟他的小廝送了去了\footnote{常跟他的那小廝送去。妙妙。別人認不得他家也。此等細處。我不題出。輕易看得出否。}。宦蕚隨後也就出門。侯氏在房中坐着。心內想。這幾日這個孩子在跟前說話嗑牙。倒好不解悶。這樣個牛心的人。定要打發他回去。可惜我錯了。我前日該帶他上去見了公婆。求公婆留下。諒他不敢不依。正在思想着。只見門上人進來說。向家娘兒兩個又來了。侯氏又驚又喜。喜的是他來。驚的是他去了又來何故。叫人忙去接了進來。他母親哭對侯氏道。方才小女到家。說蒙奶奶恩典。疼他了不得。如今老爺不要他。他今生決不嫁人。情願出家持齋念佛。保佑老爺奶奶。打開頭髮要剪去。我把剪子搶得快。還剪下一綹子來。在袖中拿出與侯氏看。又道。我夫妻再三阻他。他決不依。沒奈何。只得又同他來。求奶奶勸勸老爺留下他罷。侯氏把小娥一看。見他頭髮挽着個䯼在頭上。兩隻眼睛哭得通紅都腫了。心中甚是不忍。道。我勸過多少。他不肯聽。叫我也沒法。我有個道理。我帶了你母女去求老太爺太太。若他老公母倆做了主。就不怕他不依了。那空氏好生歡〔喜。〕侯氏就帶着到公婆屋裡來。他母女二人叩了頭。侯氏將這宦蕚不肯收這女子。自己怎樣再三勸着不依。並他女子要剪頭髮出家的話。詳細說了。如今要求公婆勸兒子留下。他方不敢違拗。纔可救得這個女子。宦實心中甚喜。兒子的好事不消說了。這個女子如此賢孝。又知恩報德。已屬難得。媳婦又這樣賢慧。更爲可喜。便道。我前日聽得兒子不肯留這女子。我心甚喜。這正是理所當然。你旣如此賢德。這女子又如此賢孝。我成你兩人之美。吩咐家人道。叫了你大爺來。侯氏道。他不在家裡。宦實吩咐一個僕婦道。看你大爺來家。叫他來。又向侯氏道。把這孩子叫他梳洗了。他母女連忙叩謝了。都歡歡喜喜同侯氏回房。他母親辭了回去。侯氏吩咐僕婦們拿水與小娥沐浴了。叫他換了一身新衣。看着他梳洗。梳頭已畢。與他戴上許多珠翠。下午時。宦蕚回家。到了內中。見小娥又在屋裡。滿頭珠翠。遍體羅綺。打扮得嬌嬌滴滴。正才要問。只見個僕婦向前道。太老爺問了老爺好幾遍可曾回來。請快去。有要緊的話說呢\footnote{省筆法。}。宦蕚忙到父親房中。那宦實就將小娥怎樣要剪頭髮出家。誓不嫁人。並媳婦賢慧的話說了。便道。他來求我。我看那孩子甚有造化。你留下他罷。宦蕚的意思還有些不肯。迫於父母。不敢違拗。低着頭不作聲。宦實見兒做難。解說給他道。你當日救他。是一番的好心。今不收他。他果祝了髮。不是你反害他了。你的心。天地鬼神已知。又是我的父命。再不可推諉了。宦蕚道。兒救他時。不忍以孝女與人做妾。今日自己反拿他做小。於心何安。宦實道。媳婦大賢。你把他處於妻之次。妾之上。禮酌乎中。也就罷了。宦蕚只得應允。侯氏知道了。忙叫人替他收拾床鋪。新被褥新枕頭帳幔。當晚就預備酒筵。叫他二人合巹成親。這一夜。兩人綢繆恩愛。可想而知。不用多說。次早。廟見之後。拜見宦實老公婆。待他之禮。比侯氏稍殺。吩咐家人都叫二奶奶。稱嬌花嫩蕊爲姨娘。小娥拜見侯氏。以妾禮自居。侯氏不肯。只受他兩禮。同嬌花嫩蕊以姊妹相敍。這小娥孝敬宦老夫婦是不消說得。他敬這侯氏也到十分。侯氏也愛他如妹妹。他待這嬌花嫩蕊如嫡親姊妹一般。先他二人見小娥後來居上。還有些妒心。見他如此。倒反親厚起來。他待下人一團和氣。眞是闔家和美。這宦蕚疼他到了至極地位。連宦實老夫婦同侯氏也疼愛他了不得。鍾生知親家娶了副親母。約會了梅生賈之(文)物童自大到他家賀喜。宦蕚留飮。彼此閒談之中。宦蕚忽想起。問鍾生道。昨日小价在尊府門口過。回家說見兄送了幾位客出來。不知府上有何事。鍾生道。正是呢。弟有一件事要同長兄商量。還要求老伯做主。府上今日有喜事。且過數日。再來奉懇。宦蕚也不再問。大家共飮。日暮方散。宦蕚見鍾生說有事同他父子商議。恐有甚機密話。在稠人廣衆之中。故不好說得。因此不問。次日。即到鍾生家來。一來謝昨日往賀。二來要問這事\footnote{如此關切。方不愧至親二字。今日有此等人否。}。你當鍾生同宦蕚商議的是甚麼勾當。鍾生的母舅早故。一個表妹嫁了司進朝。還有個表弟。名字叫作咸平。二十一歲了。新進了學。他母親要替他畢婚。他父親在日。同他的一個厚友。姓韓名仕的。自襁褓中就結親。定下他的女兒淑姑爲媳。與咸平同庚。他二人因係相契。只過了一個小定。原約到臨娶之日行聘即娶。不意兩親家數年相繼而歿。因兒女尚幼。故未婚配。今惠氏見兒子大了。意欲完成。咸平少年。才學也還〔可以。〕但只有些輕薄好勝。他知岳母寡居貧寒。不願就這門親事。向母親道。我們這樣人家。要尋何等門當戶對的親家不得。爲甚麼要娶這樣寒透了骨的女兒。兒子是決不要的。惠氏道。這是你父親在日。你襁褓中就定下的。怎麼講不要的話呢。咸平道。當日又不曾行茶過聘。父親不過是一句口頭話。如何就做得准。惠氏道。小人兒家。不要說這樣的話。古人說。寸絲爲定。你爹爹同你丈人知心莫逆。故此結下這親。雖未下大聘。已行過小茶。怎麼說是口頭話。咸平道。不管定與不定。兒總不願這門親事。就是母親定要替兒娶來。兒也決不與他同房的。不是姻緣。也難強合。惠氏到底是婦人家見識。心中暗想。兒子旣一心不願。倘強娶到家。他夫妻若不睦起來。豈不誤了終身大事。只得央人婉轉去向親家母說。兒子執定不願。恐誤了兩家的兒女。親家有令愛。何怕沒人來求。那韓寡婦聽了這話。知是女婿憎嫌他家貧寒。大怒道。這小子如此沒良心。後來焉得長進。他旣不願。難道我把女兒押上他家門去不成。要悔便悔了罷。那人復了惠氏。誰知這淑姑自幼從父親讀過幾年書。列女傳中歷來這些閨媛賢淑節烈的事。常講說與他聽。他都記在心裡。今日見咸家要悔親。母親竟賭氣依了。他向母親道。父親在日。時常敎訓孩兒說。女子之道。一與之醮。終身不二。女〔兒〕自幼已許咸家。生是咸家人。死是咸家鬼。他家負義棄兒。兒豈敢背禮他適。兒願今生永侍膝下。若要兒改事他姓。兒便不能侍奉母親。只得就隨父親同遊於地下了。寡婦聽了女兒這話。心中着急。先因氣頭上回了咸家。此時怎好又去說把女兒還與他家的話。況女婿不願。怎麼強得。左思右想。去請了族中幾位人來商議此事。內中也有三四位秀才怒道。這狗畜生\footnote{是秀才罵人的話。}。才進了學。就如此輕薄狂妄。我們到學道處呈他一狀。說他嫌貧棄妻。看他那頂頭巾可戴得穩。內中有一個老成的搖着頭說道。這使不得。我家要同他斷絕了這門親。自然是該這樣去做。不但滅了他的威風。也可出出我們的惡氣。如今我家的女兒旣然還要嫁他。這一吿了。越發成仇。後來就難收拾了。須要想一條萬全之策方妙。想了一會。道。有了。鍾員外是他的親表兄。此人是個道學先生。我們何不同去會他。把這事請敎於他。看他做何主意。他若推脫不管。那時只得到學臺處鳴鼓攻之。求學臺斷合了。衆人齊道。有理。遂同到鍾生家來。鍾生雖不甚會客。聽見有學中的朋友來會他說話。素常又知是親戚。忙忙出迎到廳。揖罷坐下。詢其來意。衆人把咸平寒盟並淑姑矢貞的話。詳細說了。鍾生躊躇了一會。說道。舍表弟年幼無知。諸位尊親不必介懷。他旣不願。就強而後可。夫妻一倫。白頭相守。若不和美時。實在兩誤。弟有一個鄙見。須當如此如此行之。再無不妥。衆人大笑道。老先生高見妙極。成全了兩姓之好。不但生者銜恩。死者亦戴德矣。辭了出來。回了韓寡婦的信。他母女歡喜不盡。那日鍾生向宦蕚要說的就是這件事。次日宦蕚到了鍾生家。先謝了昨日的厚情。並問及有何事相商。鍾生將咸平棄妻淑姑自矢的話。詳細說了。道。舍表弟少年無知。今日弟若不爲彼完成此事。不但他靑衿難保。且將一生的人品喪盡。先母舅只此一子。焉忍坐視他沈溺不救。況豈不誤了這韓家賢女的終身。弟思了一策。懇吾兄婉達老伯。權忍(認)作義女。弟稍備些須妝奩。弟去與家舅母商量。假爲舍表弟作伐。完成之後。老伯再說破。以正言敎之。彼必不敢再萌別意了。宦蕚喜道。君子人成人之美。長兄旣有此美意。弟當玉成其事。況令表弟之不願者。嫌彼之貧故耳。弟備妝奩賠了他去。便把一天好事都完了。鍾生道。豈敢又破費長兄。使弟更不安了。宦蕚道。你我兒女至戚。何必還說此客話。弟在他人猶不惜。況於親戚乎。辭了回家。稟知父親。宦公喜允。遂差了兩個僕婦到鍾生處。一同差人接了淑姑來家。宦公見他雖裙布荆釵。好一個端莊的女子。滿心歡喜。認作了女兒。替他做紬衣製首飾。那如吹灰之易。不用說得。鍾生一日到舅母家來。作揖坐下。咸平也陪着。鍾生說了些閒話。然後向惠氏道。表弟已經成立。韓家的令愛也大了。親事也該完成。以畢終身大事。惠氏道。這門親事你兄弟不願。已經辭退了。鍾生佯驚道。這是甚麼話。舅舅在日。替表弟自幼定下的。今日如何講不願的話。不但棄妻爲不義。且背父命又是不孝了。舅母如何順他胡做。那韓家雖然家寒。族中有許多秀才。倘一時動了公憤。到宗師處吿起來。不但功名不保。後來何以見人。況且人家若知道這件事。誰家的女兒還肯同我們結親。我們去退親之時。他家如何回復了來的。惠氏道。他母親別無多說。也竟依了。鍾生道。造化。造化。這是他韓府上的人盛德。若略要動氣。何以處之。向咸平道。表弟少年。才得一步。這樣負心的事。可是做得的。咸平面赤耳紅。無言可答。鍾生又道。如今事已至此。悔亦無及。但你也時不可待。我宦親家有一令妹。乃宦老伯之愛女。我爲表弟作伐去求。何如。但恐無大賠送。未必中你之意。咸平聽得說宦府的女兒。便道。承老表兄下愛。弟安敢尚萌別念。但恐宦府閨秀。未必肯下嫁寒門\footnote{嫌貧之人自然慕勢趨富。聞得宦府之女。又自揣其恐寒微不敵。故作此語。小人之心胸大都如是。}。鍾生道。我若去說。十分中有八九可成。允與不允。我再來復信。作別回來。次日。又到舅母家中。到房內向惠氏道。恭喜舅母表弟。我昨日到宦府去提親事。一說便成。只打點行聘。就可以娶。咸平母子歡喜非常。擇日行聘。到吉期迎親來家。合巹之時。咸平覷見好個女子。暗道。到底是大家閨秀。不但美麗。而且穩重。比寒門小戶的女兒。自是不同。要是前日不拿定主意。要娶了韓家的女兒來。不知是怎個寒乞的樣子呢。他心中那個樂。眞說不出。又見賠送的嫁妝雖不爲十分豐厚。件件俱備。且還有一個使女爲媵。更自欣喜。出去陪待賀客。到晚人散。忙忙進來。要同新人做一番親熱。不想房門緊閉。咸平不知何故。心中疑訝。輕輕敲門。內中一個宦府遣來作伴的婆子老僕婦隔門說道。姑娘吩咐不許開。姑爺今晚且請在書房暫宿一夜。明日等我家太老爺同鍾老爺家老爺同來說明白了。再做商議。咸平驚道。百事俱已完成。還有甚麼商議的。煩你去求姑娘。不要誤了吉期。那伴婆又說道。姑娘說。聞得姑爺自幼定下人家一位閨女。嫌他寒貧。遂背盟棄擲。今我家的姑娘。妝奩菲薄。恐姑爺日後憎嫌起來。又想抛棄。豈不自誤。除非同家老主衆位共同面講過。才敢放心。咸平又是那愧\footnote{良心幸還未死。}。又發急道。這是甚麼話。你家姑娘一個千金小姐。怎比得那貧士的女兒。不要說有這些賠事。就是絲毫沒有。我也不敢憎嫌。因道。恐你姑娘不足憑信。我跪在這裡發誓了。跪下道。我異日敢負初心。人神共殛。那伴婆去了一會來開門。道。姑爺記着這句話。咸平忙走到房中。見新人在床上。背燈而坐。深深一揖。道。賢妻爲何如此多心。我蒙岳父大人不棄寒微。又是家表兄作伐。可敢萌一毫別念。遂上前解衣就枕。成就了百年姻眷。次早。雙雙拜了家堂老母。這日單請宦公同宦蕚鍾生三位喜筵。宦公到來。坐下茶罷。向咸平道。賢婿旣不棄小女。已結百年之好。令岳母處也該去拜謝才是。咸平道。岳母尊前。小婿昨日就叩謝過了。宦公笑道。非老妻之謂也。此女非老夫親生。乃我故人韓氏之女。即賢婿前日之所棄者。我撫爲螟蛉。故令表兄作伐。已完宿緣耳。咸平方知是他的舊妻。羞得置身無地。鍾生正色責他道。吾弟始博一領靑衿。便做這等負心無義的事。視古人不棄糟糠之婦者。寧不自愧。前日韓府上許多令親。都是三學中朋友。同到我家。\endnotemark[2]要動公呈到學臺處呈狀。若此事一行。不但你功名不保。連一生的人品都喪盡了。蒙宦老伯不忍見你少年破敗。故有此義舉。吾弟此後當洗淨前心。宜爾室家。倘再萌不肖之念。我們都要動公忿了。那咸平羞愧難當。說道。弟知罪也。蒙岳父垂慈。長兄憐愛。弟安敢尚有別意。長兄陪岳父舅兄坐坐。我此刻就往岳母處謝罪。宦公道。賢婿且住。我知令岳母孀居。並無以次親人。賢婿何不接了來。同令堂老親母一處相伴。不但不失親親之誼。就可以挽回前愆了。咸平連連應諾。他知岳母家寒。恐沒有衣服。問母親要了一套衣裳包了。叫了一乘轎子。親去謝罪迎請。韓寡婦見女兒已嫁了。他家女婿又如此盡禮。前憾盡釋。欣然同來。宦公衆位日暮方散。咸平次早去拜韓家族中諸親。就下帖請男婦吃會親的筵席。衆人知他連岳母都接了家去養活。還有何惱。盡來赴席。無一個不誇宦家喬梓同鍾生的好處\footnote{誇他三人的好處。正反映咸平之不好處。此乃是不罵之罵也。}。另日又請宦公父子鍾生司進朝。內裡請艾夫人侯氏向氏嫩姨嬌姨錢氏戴氏並司家姐姐。惟宦公老夫妻辭了。別的男女都到。咸平也忙了數日。才淸楚了。他夫妻相愛。甚是和美。咸平每每自愧前失。那年正値大比。有兩句古語改兩個字。就是他今日了。道是。

\begin{quotation}

榜名盡處是孫山。咸平更在孫山外。

\end{quotation}

咸平自恃才高必售。孰知落第。心中悶悶不悅。夜間夢見他父親道。我祖宗積德三世。你今科已榜上有名。因你有棄妻一事。已經革去。幸賴鍾家賢甥成全了你。你若再行好事。下科尚有可望。榜上第六十三名劉顯。他有不肯棄的好處。就是頂你的了。說畢。慘然而去。咸平一驚醒來。不勝痛恨。此後他夫妻之情更篤。權且按下。你道劉顯是誰。他是劉太初之子。宦蕚姑母之兒。他當日同鍾生梅生司進朝咸平都是廣先生的門人。廣先生敬太初是個今之古人。不趨炎熱。不貪名利。不降志。不辱身。知他後嗣必昌。廣先生有個女兒。倒叫梅生去向劉太初說。願把女兒與他爲媳。劉太初也識廣先生是個盛德君子。一諾無辭。劉太初家寒。無以爲聘。惟一言爲定。廣厚德後來運捷。中了進士。歷仕做到吏科給事中。因參了閣臣楊嗣昌。崇禎大怒。要將他革職議處。吏部同都察院再三執奏。說科道兩衙門若以言事問罪。是鉗言路之口矣。才將他降了廣東潮州府潮陽縣典史。廣先生原是個窮儒。又做了幾年淸官。宦囊蕭索。女兒尚小。一個兒子廣沛。還在童稚。不能留在家中。只得同老夫妻一起帶往住所。到任三載有餘。就病故了。他這女兒因見父亡母老弟幼家寒。離鄕數千里。父親骨櫬並家口何日是個歸期。朝夕啼哭。竟把雙目喪明。他母親租了幾間房子住着。聞得房主要往南京貿易。寫了一封書子寄與女婿。托他來接家小。又恐女婿是個寒士。未必找尋得着。因想起丈夫舊日的學生。內中只有司進朝的父親做過司道。還是個有名的鄕紳。易於找覓。又寫了一封書與他。一則托他轉付信與劉顯。二則托他向衆門人吿助。叫女婿來接。這房主憐他家是個好官。今日流落異鄕。竟不負所托。到南京尋着了司家。將書投了。司進朝看過。方知先生已故。先將劉家的書信差人送去。即親到梅生鍾生槩(曁)向日同窗的朋友處。說了先生訃音。又將師母的來信都與衆人看了。他首倡助銀百兩。衆人公分十兩二十兩不等。同他的湊了有二百餘金。鍾生感先生昔日相愛之情。送五十金。宦蕚知道表弟去搬丈人的靈柩。要厚贈他。恐那迂姑爹不受。拿了一百五十兩來付與鍾生。同他的湊作二百。只說他送師母的途費。共有四百餘兩。交與劉顯。鍾生見他孤身遠行無伴。叫鍾用同去。劉顯感之不盡。辭別了父母同衆友。帶着鍾用。雇船去了。一路無話。到了潮陽。接了岳母一家。搬岳父靈柩回來。到了家鄕。因岳母無家可歸。將他隔壁有賣的一所房子買了。與岳母居住。將岳父安葬在廣氏祖塋。還剩有百餘金。交與岳母收了。此時他夫婦年俱二十以外。劉太初煩原媒梅生去向親家母說要完成兒女的姻事。廣夫人說女兒雙瞽。不可以奉箕帚。情願叫他家另娶。他令愛也執意不嫁。願伴母親終身。劉太初父子決定不肯。說道。當日承親家厚愛。將令愛作配小兒。不要說瞽目。就是有惡疾。也不敢寒盟。劉顯也說。若他的令愛不嫁。我也終身不娶。寧可絕嗣。爲宗祖之罪人。不敢負義。爲名敎之罪人\footnote{有是父方有是子。}。梅生〈枚〉往返了數次。廣夫人母女見他父子如此。不得不依。婚嫁之後。一夕。劉太初夢到一公署。進內看時。上面坐着一位貴人。如塑畫文昌帝君的形像。傍坐許多官員。私問傍邊吏役。說是帝君同各府的城隍。查各府今科舉子賢否姓名。好定榜上奏天庭。劉太初大驚。方知是神道。在榜(傍)竊聽。上面帝君一名一名點去。是何處人。那府城隍便將他家善惡細呈。或勾或換。也說不得許多。忽聽得點到第六十三名咸平。係應天府上元縣人。傍坐一神起立。道。此人嫌貧棄妻。應當革去。雖虧他表兄完成。但起心不端。當壓一科。那帝君便一筆勾去。說道。可舉一人來替。那神又稟道。江寧縣庠生劉和父子。不肯以原聘之媳因瞽而不棄。正同此案。乞將伊子劉顯頂補。見那帝君提筆寫了兩個字。像是換了名字。劉太初心中一喜。醒來却是一夢。又驚又喜。不敢說出。果然到放榜之日。劉顯中式第六十三名。咸平素常同他相厚。又是自幼同窗。那日來賀。他將自己父親托夢向他父子說了。劉太初也把自己所夢對咸平細說。方知舉頭三尺有神靈。坐客個個驚異。咸平自怨自艾。矢心向善。下科果然得中。仍是六十三名。更以爲異。此是後話。不必多敍。再說宦蕚同小娥成親之後。叫小廝拿着二百兩銀子。他親到向惟仁家謝了他送女兒之情。並吿訴他不以妾禮相待。位居大奶奶之次。向惟仁夫妻歡喜不盡。宦蕚又將二百兩銀子送他買房子住。向惟仁推辭再三。宦蕚不肯。他方受了。他正戀新婚。上馬歸家。到了一個人家門口。聽得裡面一個婦人嚎啕大哭。又是幾個小孩子悲啼。一個老兒嘓嘓噥噥個不住。街上站着幾個人。嘆息不〈不〉已。他下馬向前相問。那衆人道。這家姓利。他兒子往湖廣做買賣去了。三年總沒個音信回來。他父母都老了。他撂着老婆兒女五個。又沒得穿。又沒得吃。老兒又老了。沒掙載。一家常常捱餓。老兒說湖廣流賊正多。必定是兒子歿了。要媳婦帶着兒女改〔嫁。〕媳婦又不肯。說沒有得丈夫的實信。如何行得\footnote{賢哉此婦。宜乎得遇宦蕚相救。}。那老兒終日吵吵鬧鬧。媳婦哭哭啼啼。眞是沒法的事。宦蕚想了一想。問道。他兒子名字叫作甚麼。是那一年去的。內中有一個道。他叫作利老大。誰知叫甚麼名字呢。又一個道。我少時同他念過書。他學名是個陞官圖的圖字。又一個想了想。道。他是那年八月裡去的。我爲甚麼記得。因指着他拉着的那兒子道。他頭兩日在我家吃過小子滿月的酒。第三日才起身去了。小子三歲了。他去了不到整三年。宦蕚問明。上馬到了家中。着人請了鄔合來。把適才利家的話吿訴與他。道。我想要救他這一家。除非寫他兒子的一封假信。內中封幾兩銀子做個憑據。方可解救得。故請你來寫寫。就煩你送了去。如此如此說。你還在行些。對答得來。他滿口答應。道。大老爺做這樣陰隲好事。晚生當得效勞。把書寫完。念與宦蕚聽。宦蕚喜道。寫的好。即取了十兩銀子封在書內。火上烤乾了\footnote{其細至此。}。叫先跟馬的小廝領了鄔合去。不多時。到了他門口。聽得裡面還嗚嗚的哭呢。鄔合上前敲門。敲了半晌。只聽得一個老兒咳咳〖口敕〗〖口敕〗扶着柺出來。問道。是誰敲門的。鄔合道。是送家信來的。那老兒聽見送家信。忙把門開了。問。大爺是送甚麼信的。鄔合道。你老人家就是利老爹麼。那老兒道。不敢。我就是。賤姓利。大爺請裡邊坐。到了房內坐下。鄔合道。我姓鄔。往湖〔廣〕做買賣去來。遇見了令郞。偶然間說起來。都是鄕里。他的生意十分連年茂盛。賺了大錢。捨不得撇下。不能就回。我的事完了要回家。他托我帶了一封信十兩銀子來。袖中取出遞過。道。你老人家收了。那老兒聽得兒子有信回來。又說在外賺了大錢。已是歡喜之極。又聽得帶了十兩銀子來。又如死了又還魂的一般。喜得屁滾尿流。笑得滿臉眼淚。向鄔合作謝。道。多謝大爺遠遠帶來。誰肯。聽見媳婦還在那裡哭。叫道。你還哭甚麼。兒子煩人帶了信同銀子來了。還不來謝謝這位爺呢。那媳婦眞像得了命的一樣。眼淚也沒擦乾。忙走來拜謝了鄔合。問公公道。信上怎麼說。那老兒哈哈大笑。道。我喜歡昏了。信還拿在手裡。忘了看呢。又遞與鄔合。道。我不識字。就煩爺念念與我們聽罷。只見那老婆子聽得兒子有信。也拄着柺。滿頭白髮。不住搖頭磕腦。戰篤酥的。口中喃喃念着佛。也來聽。謝了鄔合。坐下問道。爺貴姓。爺是好人。爺怎麼認得我兒子。就肯替他帶了信來。那老兒道。這位爺貴姓吳。你不要說熟話。且讓吳爺念了信着。鄔合拆開念道。自從前年八月離家。外面生意甚好。所以戀住。至今不得回來。屢屢要寄幾兩銀子回家。因無的當人可托。今有鄔大爺還鄕。特煩帶信問安。並銀十兩盤纏。明年三四月間一定回來。不必記掛。媳婦好生孝順公婆。看視兒女。餘不盡悉。他一家聽了歡喜是不用說。向鄔合道謝了又道謝。那老兒道。老爺貴姓鄔。我當是姓吳。年老了。耳朶背了。那婆子同媳婦絮絮叨叨。問長問短。哭一會。笑一會。問了好些話。鄔合含着笑隨機應變。含含糊糊的答應了幾句。恐露出馬脚來。忙忙的起身作別。那老兒送着說道。爺再請坐坐。我取壺酒來敬爺酬勞。鄔合笑道。多謝罷。不必費心。老兒道。多謝爺盛情。簡慢爺去。窮人家連茶也拿不出一鍾來。爺又不用酒。等我兒子回來。到爺府上叩謝罷。鄔合別了回來。又復了宦家的信。宦蕚甚喜。果然到了次年三月。利圖滿載而歸。闔家歡喜。到晚間。夫妻上床接風之後。講起別後家常。他妻子從新眼淚鼻涕的哭訴。公婆如何不見音信。逼他改嫁。正要尋死。虧得帶了銀子同信來。才好了。若再遲幾日。今生已是不能相見了。利圖聽了。茫然道。我並不曾帶甚麼銀子同信來。婦人反吃驚道。是去年冬天。一個姓鄔的帶來的。利圖次早問父親要了那封字兒看。不知從何而來。問父親可曾問這姓鄔的住在何處。那老兒道。我只說你必定知道。所以就不曾問。他一家都疑是菩薩神道就(救)他。那裡知是宦菩薩做的好事。倒焚香化紙。三牲五果的叩謝神恩\footnote{若果心虛。宦蕚必定醉飽。何以知之。狄仁傑早朝。面有醉容。武后問曰。卿素不飮。何得有酒色。狄仁傑道。昔臣在秦州。百姓德臣。建立生祠。或今日醉臣耳。}。却說宦蕚臘月初旬那一日。風微日暖。他騎着馬各處走了一會。到了一條小巷內\footnote{前寫向惟仁在一條僻靜巷內。此寫巴氏在一條小巷內。此是何意。要知熱鬧處房子貴。窮人住不起耳。}。見一個院子裡一個老婦人\footnote{大臘月院子裡可是說話之處。豈非漏空。若在屋裡說。宦蕚何由得見。極難下筆。方悟着開首風微日暖四字之妙。}。指手畫脚哭着說叫。一個二十多歲的小後生扶着他勸。有幾個男人站着聽。宦蕚疑必有原故。打馬進去。下馬。衆人看見。忙來迎着道。老爺有何貴幹。宦蕚道。我才打這門口過。見這位老奶奶哭哭說說。是爲甚麼事。那老婦一腔苦楚。見宦蕚問他。答道。我先夫姓穆。我姓巴。我四十歲上守寡。指着那扶他的後生道。這是我兒子穆富。那時才五歲。我娘兒兩個。家中沒一點甚麼。巴巴竭竭的守到如今。他二十八歲了。還是他爹在日。就定了一個吉家女兒作媳婦。是同年生的。吉家催了幾次叫娶。我兒子在銀匠鋪裡做徒弟。一年的工銀只夠娘兒兩個吃穿。可還有銀子娶媳婦。親家發了幾次話要悔親事。虧了媳婦賢慧。抵死不依\footnote{窮人之無力娶妻者甚多。而宦蕚惟力助穆富者何故。因重在此句話上。}。如今親家那裡來說。女兒大了。不拘怎麼。趕年底下亂歲的日子接了來罷。老爺你請想。人家這樣好話說了來。我們還怎麼回得他。如今就是做幾件布衣服被褥。轎子水酒零碎使用。至少也得十多兩銀子。況且俗話說的。新人進了門。還要費一條牛錢呢。那裡不要錢用。此項從何處來。沒法了。請了他們來。指着兩個人道。這是我兒子的親叔叔。又指着那二人道。這是我兩個親兄弟。求他們幫助幫助。大家都一毛不拔\footnote{大約都是楊朱的高弟。}。老爺。你叫我一個老寡婦何處去折騰\footnote{勿謂老寡婦沒處折騰。即小寡婦一有處折騰。便不妙矣。}。怎不叫我傷心。宦蕚向他衆人道。列位旣是至親骨肉。也該多寡幫助些才是\footnote{至親骨肉貧窮無力者何足責。有擁重資坐視而不顧者不知幾許。宦蕚或未之知耳。}。衆人道。老爺在上。我們都是窮家小戶。俗話說。風吹了下頦去。連嘴也趕不上。一碗飯還奔波不過來。如何幫得起這些銀子。就是些來小去幫補些。還吃力呢。實在力量不能。並不是捨不得。要有銀子藏着。至親骨肉的喜事不拿出來幫助。就男盜女娼。留着一家銜口買棺材釘。宦蕚向巴氏道。他們發這樣惡誓。大約都窮。也怪不得他們了。你方才說十多兩銀子夠你絞纏媳婦了。你母子就不要添件衣服。古語說。寧添一斗。不添一口。娶了媳婦來。柴米油菜炭火那樣不要添些。這又得幾兩銀子。巴氏道。這十多兩。千難萬難。還沒個影兒呢。再要這樣算起來。一輩子也娶不成了。只好得一步進一步。宦蕚道。我替你打量。有三十兩銀子就富餘了。那巴氏倒反笑起來。道。拿我老婆子賣了娶媳婦。也沒人出三十兩銀子。宦蕚叫小廝拿過銀子來。稱了三十兩與他。道。這成全你兒子媳婦罷。那巴氏眞做夢也想不到。忙同兒子跪下拜謝。道。老爺的天恩。叫我母子如何補報。宦蕚道。你老人家請起。我憐你寡婦孤兒。媳婦又賢。故此成你美事。豈望你報。又笑向那四人道。不用你列位出錢。看是至親。幫幫他好事罷。衆人道。這是當然的。何須老爺吩咐。巴氏道。老爺貴姓。量我母子也不能報恩。只每日燒香叩頭保佑罷。宦蕚笑道。你問我姓做甚麼。不必記心。遂上馬。與他四人一拱而去\footnote{古人云。臣父(不)淸。畏人知。臣淸。畏人不知。宦蕚可謂他人行好。恐人不知。自行好。惟恐人知。優劣便見。}。內中有一個認得他的。道。這是有名行好的宦大老爺。衆人方知他是宦公子。後來巴寡婦娶了兒媳婦來家。知是宦公子成全了他夫婦。那吉氏果然賢慧。立了個牌位。一家早晚燒香保佑他。不題。再說一日臘盡春回。陽和布暖。他夫妻三個早飯罷。宦蕚道。忙忙碌碌過年過元宵。誤了我好些善事。今日晴爽。且出去看看。遇着有好事。做他一兩件。帶了小廝出門。轉彎抹角。打馬正走。見前面一簇人圍繞着。不知看甚麼事。他催馬上前。進內看時。見一個老婦掩面悲啼。一個婦人抱着個孩子兒喲肉喲的不住拍拱(哄)。一個凶暴壯年小夥子在那裡大罵道。我拿着飯白給你這老殺肉的吃。做甚麼事。把個孩子跌得恁個樣子。遂了你的狼心狗肺了。不住的大叫大罵。你道這少年姓甚名誰。他罵的是甚麼人。他姓卜名校。是卜通的一個族弟。十歲喪父。虧他母親闕氏。績麻紡線。養他成人。他自幼無父敎訓。闕氏只此一子。未免嬌縱太過。他並不知母親是何物。如同奴婢一般。任情呼使。稍有違誤。輕則大罵。重則掄拳。闕氏被他降服慣了。叫東不敢往西。他尚不遂心。無日不見敎幾句。他到了十三四歲。在外邊挑個菜擔子。每日掙幾文錢來幫補。這闕氏口挪肚趲。積了十數年。湊得十數金。卜校到了二十五歲。替他娶了個媳婦伍氏。這伍氏好吃懶做。生性憊賴。與這卜校眞是天生一對。地長一雙。也並不識婆婆兩個字是個甚麼東西。他一日惟有高坐。悶了來同鄰舍家婦女們去閒嗑牙。睏了睡上一覺。便是他的事務。一日燒茶煮飯。掃地關門。無樣不是闕氏去做。他此時年也老了。一日到晚來服侍兒子媳婦。稍有閒空。也要歇息一會。不能紡績了。專靠兒子度日。好不好便不許他吃飯。因此越發怕他無比。卜校生了個兒子。這日是他週歲。他丈人丈母舅子送了些魚肉酒麪來。闕氏忙了半日。整治款待衆人。兒子媳婦陪着大吃。吃完之後。衆人散了。闕氏收了些殘湯剩水。將就吃了些。卜校伍氏這日未免起得早。又陪着衆人吃了幾杯早酒。醺醺然要睡午覺。把孩子交與闕氏。抱他在門首。坐在一條矮凳上。哄他頑耍了一會。那孩子就睡着了。闕氏有年紀的人。又辛苦了一早起。不覺舂了個盹。失手把那孩子就掉在地下。把額上油皮跌破了些。那孩子喳的一聲大哭起來。闕氏驚得慌忙抱起。卜校伍氏正睡得受用。夢中聽得孩子哭起來。一驚醒。夫妻從床上跌跌滾滾跑出房外。見闕氏抱着孩子替他揉頭。那伍氏連忙接過去。看見跌塌了有指頂大的一點油皮。抱着說道。我的兒囉。心疼死我囉。我就知道叫這老殺肉的抱着不好。果然跌得恁個樣兒。却趁了你的心了。就同我們大人有仇。拿着恁點孩子作踐。也不當家。明化化的神道的眼睛看着你呢。我的兒喲。嚇壞了你囉。嘴對着嘴。啐呀啐的替他收驚。儘着拍哄。一面嘴裡不住的咒罵。那卜校那裡還依得。將闕氏打了兩拳。還不住跳着大罵。宦蕚問人是甚麼緣故。他那鄰舍有不忿的。將他家事向宦蕚細說。宦蕚聽說他罵的是母親。心中大怒。騎着馬到他跟前。喝道。你這人好沒道理。一個母親。那是罵得的麼。卜校看了看。要是別人。他也就動粗了。因見宦蕚體統尊貴。不敢放肆。說道。他就是我母親。他該跌我的孩子麼。宦蕚道。你養的。你就知道心疼。你是他養的。倒不心疼他。你別的不知道罷了。你想想他十月懷胎。三年乳哺的恩。可是忘得的。況且你從小無父。他養活大了你。替你娶妻生子。你今日不能孝敬他。倒打罵他。你不怕天雷劈腦子麼。卜校哈哈大笑。道。天高高的。那雷也管不着我們這些閒事。至於說十月的懷胎是他的恩。那有甚麼恩處。你道他好意懷我的麼\footnote{奇想。描寫逆子心腸口角。妙甚。}。復笑道。那是他兩口子圖快活。朝死裡弄。誤打誤撞。把我弄在肚裡。他不懷着怎麼樣呢。又不是私孩子。他肯用藥打掉了麼。說他三年乳哺。他養下我來。圖我醒眼。給他解悶。他不給我吃。難道餓死我不成。況且奶是他身上出的。還費了他半個錢麼。他就不給我吃。他怕脹得疼\footnote{愈想愈奇。}。宦蕚聽他說了這些話。又是那氣。又是好笑。駁他道。我聽得你從小沒了父親。不虧他養活你麼。卜校道。我十歲上老爹才死了。我吃的穿的都是我爹的。他那有本事掙錢養活我呢\footnote{阮籍云。禽獸不知有父。猶知有母。人生天地間。不知母者。禽獸不若。卜校之謂。}。我十三四歲就賣菜。掙了錢回來養家。就算他養了我二三年。我今年也養了他十幾年。還扯不得直麼。宦蕚又道。你的妻子是那裡的。難道不是他替你娶的麼。卜校道。這話越發出奇了。他旣有本事養兒子。不替我娶老婆。他好戲(意)替我娶呢。他圖我養兒子替他傳代\footnote{眞是這話越發出奇了。}。我的兒子是個寶貝一樣的東西。他不小心的抱着。頭上的皮都跌塌了。要他做甚麼事。拿飯養狗也替我看看家。這樣老沒用的。白拿飯給他吃。是爲甚麼。那闕氏先怕兒子打。不敢回言。此時見宦蕚在跟前問話。諒他不敢動手。哭着說道。我雖老了。做不得甚麼。不拘到那裡去替人家燒鍋掃地。也掙得一碗飯吃。再不然沿街叫化。也還舒心些。你不要我。我去就是了。何苦一日打打罵罵的。卜校大怒道。你要去。你當是我要留你麼。一手拉着他的膀子。一手掐着脖子。往外一搡。一交跌得老遠。罵道。夾着你的老屄走。再要上我的門。把胯子踢揸了你的。宦蕚大怒道。反了。反了。天地間那裡有這樣的事。忙叫小子們快把那媽媽扶起來。宦蕚正要發作。只見那婦人向卜校道。你叫他往那裡去。知道的是他壞。不知道的還當是我做媳婦的挑你着(容)不下他呢。再者。他別的做不得。留他在家裡服侍使喚也罷了。你攆了他去。這些粗夯活計。我是不會做的。卜校道。你放心。世上有累死人的活計麼。死了王屠戶。還連毛吃豬。他去了。不拘甚麼事。我都一攬干包。全全做的。你只管先坐着受用\footnote{他不能孝母。却能孝妻。眞孝夫。然而世上恐此等孝夫不少。}。叫他去。且落得寃家離了眼睛。宦蕚先聽得媳婦要留婆婆。還當是好意。以爲兒子不孝。媳婦若賢慧。還打算勸他母子和好。不想後來的話是要留下當奴才的意思。忍不住笑道。這樣的禽獸\footnote{他夫妻只算得梟獍。如何及得別的禽獸。}。同他一般見識做甚麼。又問他一句道。你的母親你當眞不要他麼。卜校道。漢子家說話。可有三心二意的。說不要就不要了。宦蕚見闕氏還在地下哭。向他道。老媽媽。你不要哭了。我府中家下人有幾百。何爭你一個。你到我家去。一點事也沒有你做的。一年穿吃不用你愁。我都給你。你老了的時候。我買棺材發送你。這樣不孝的奴才。你稀罕他做甚麼。叫小子送他老人家到家去。那闕氏見宦蕚收留他。滿心歡喜。也不哭了。還要進去取他的破衣舊被之類。宦蕚道。不消了。你到我家。怕沒有麼。小子們領着他去了。宦蕚忿忿然也上馬而去。旁邊看的衆人無不嘖嘖贊他的好處。闕氏到了宦家。宦蕚吩咐管家婆司富替他做了一身衣服被褥之類。命每日好生管顧他的飯食。那闕氏受了一生的苦楚。還要受兒媳的凌辱。今日忽來飽食暖衣。一毫的事也無。終日高閒自在。感恩無旣(際)。每日早晚當天叩首。保佑宦恩人福壽綿長。子孫繁衍。又求吿蒼天。不孝兒媳早賜報應。他這一點虛心。上蒼豈不鑒察。他過了些時。身子閒不過了。幫這家漿洗漿洗。幫那家抱抱娃娃。衆家下婦人見他活動些。沒一個不憐愛他。這個替他做鞋脚。那個送些東西吃。其樂無比。終日惟有嘻嘻說笑。一點憂愁煩惱都沒有了。但想起兒子媳婦來。氣恨不過。就當天叩一陣頭。咒罵幾句。且說卜校自攆了母親去後。他果然慇懃之極。當日闕氏在家。他一毫也不相幫。如今一應的事都是他做。總不驚動伍氏。伍氏惟有抱着孩子頑耍。他忙忙收拾了還要去賣菜。十分勤快。間或伍氏懶動。或身子微有不快活。晚間回來連淨桶都是他倒\footnote{他原說過一攬干包。}。闕氏養他一場。也不曾受這樣服侍一日。如此過了月餘。他夫妻二人坐着偶然閒話。伍氏抱着那孩子頑耍。道。老婆子去了這些時。倒覺得眼睛淸靜些。像拔了肉中刺一般。卜校道。我只巴不他死。他偏不死。就像我眼裡疔瘡。如今去了這些時。眞是拔去眼前丁(釘)了。伍氏道。只怕那人家留他住厭了。又送了回來。怎麼處。卜校道。他還想回來麼。今生不能夠了。可是人說的。醃韮菜入不得畦了。他要來。我不說別的。只說他雖然年老。到底是個婦道家。到人家去了多少時。知道養漢沒養漢。肯留着玷辱家門麼。他自然站不住。少不得去尋頭路。伍氏笑道。你好頭好算計。二人說話之時。正天淸日朗。忽然一陣暴風。烏雲陡暗。雷聲隱隱。他二人還不覺得。那雷漸漸在他房頂上轉響。那卜校伍氏也就有些心驚肉顫。忽一陣硫磺氣。一個大悶火光大亮。一聲霹靂。震地驚天。把他兩間房子並家中所有燒得精光。一牆之隔鄰家絲毫未動。將他三人提到街心。衣服皆不知何去。卜校燒得烏黑。身上批了四個大紅字。有認得的說是不孝逆子四個字。那孩子也燒焦了。父子死在兩處。那伍氏震死了好一會。重復醒了過來。赤着身子。渾身皮肉皆被雷火燒糊。雖還未死。却動不得。睜着兩隻大眼睛。並不一眨。嘴裡〖口么〗〖口么〗喝喝。那街上來看的人擁擠不動。那伍氏上下無一絲遮身。有看不過意的。脫件布衫撂了。替他蓋着下身。他震得瘋瘋癲癲。將他夫妻忤逆不孝的事。從頭細述。他父母知道了。擡回家去。一到了屋裡。便渾身疼得要死。叫喊連天。擡到街上。又歌又笑又哭。向人訴說他夫妻的這些妙處。身上便不覺疼。夜間擡進屋裡。就疼得亂叫。他父親沒奈何。只得搭個小蓆棚在街上。叫人守着他。他也總不吃東西。便溺遍身汚穢。過了七日才死了。他父親買了口棺材裝了埋葬。剛葬了。忽一個大雷將墳擊開。棺材劈得粉碎。那屍首越發燒成一塊烰炭。他父親不敢再埋。棄了回家。倒不如卜校沒人收葬抛棄了的省事。這是忤逆不孝的兒子媳婦的樣子。人生世上的罪。可還有重似不孝的。古云。

\begin{quotation}

萬惡淫爲首。百行孝爲先。

\end{quotation}

豈可不自爲警省。有一調駐雲飛\endnotemark[3]感嘆世間的兒女。道。

\begin{quotation}

父子深恩。富貴場中間有人。若得兒孫順。須是親榮盛。噫親老更家貧。尚何尊敬。忤逆多般。陌路還猶可。嘆那孝字。而今有幾人。

\end{quotation}

那宦蕚知道了此事。滿心暢快。道。天地神祇靈應至此耶。闕氏聽得兒孫媳婦被雷擊了。媳婦又是這樣死法。不但毫不悲戚。忙向天叩了有數百個響頭。就有好傳新聞的刻出勸世文來賣錢。傳得通國皆知。後來闕氏老故。宦蕚殯葬了他。做了一件全始全終的好事。此係後話。再說宦蕚偶然一日道。我這些時不曾到城南去。今日去走走。遂乘馬帶着小廝走到了油房巷口。見一家出殯。十分熱鬧。有許多紳衿步送。那內中有宦蕚認得的人。下馬喚住。問他是誰家。那人說是單于學的妻子。你道他妻子死了。爲何有這些人送。這單于學他心地倒也豪爽。但性情酷好戲謔。他雖不能稱作大通。也還不是一塊白木。他家資富厚。娶妻甄氏。是個儒家之女。生得端莊秀麗。識字知文。不悍不妒。眞是個四德兼全的賢婦。又有三個妾。一個姓紅。一個姓黃。一個姓白。單于學把他三人比作三種牡丹。紅氏稱爲一捻紅。白氏稱爲玉樓春。黃氏稱爲姚黃。還有兩個通房艷婢。一名花鬚。一名花蕊。這幾個雖算不得絕色佳人。也都還有幾分的姿色。單于學恃着有一根成文的陽具。在這些婦人中晝夜鑽硏。猶不滿意。還在外邊眠花宿柳。因作喪過了。那陽物進了陰門。未及交鋒。早已敗衂。他當日戲水氏時。雖說不濟。也還有十來抽的本事。後來不知自檢。還恃勇前驅。竟弄成了個自反而縮。任你百般摶弄。總伸不出來。他是個在此道中用功的人。而且家中擺設着這些花枝般的嬌妻美妾艷婢。終日眼飽肚飢。如何過得。心中着急。四處尋人醫治。費了許多銀錢。吃了無限藥餌。薰蒸洗泡。無樣不治過。全然無效。偶然聽得人說有個外路來的道人。姓翟號疊峰\footnote{謂如蝶蜂之賊也。}。在街上賣藥。自誇善能壯陽固本。有養龜妙術。單于學聽見這話。猶如天上降下一位眞仙來救他一般。尋到他寓處。求其救治。敦請了來家。許他重謝。誰知這賊道是個淫壞不堪的惡物。他不知在何處學來的許多的異方。與人治病。頗有奇效。更有幾種極惡的方兒。說起來令人切齒。但有人請他到家。他見有婦女。狡計多端。定要被他淫汚了才罷。他有一種末藥。名爲自送佳期。不拘酒中飯中茶中。暗暗與婦人吃下。便陰中深處熱癢難當。任你抓撓摳挖。再不能止。定要同男子交媾之後。方才止得。不然就摳爛了也是無益。更有一件藥物。也是製成的麪子。名爲美女自解裩。將些須放在淨桶中。婦人去小便。熱尿一衝。那藥氣一蒸。更加利害。陰中不但奇癢。且要浮腫得翻將過來。非陽物洩去火氣。斷不能癒。他這賣春方的人。小戶人家用他不着。請他的自然都是鄕紳富室。姬妾衆多之家。他住久了。買通了他家狡童奸婢。便暗暗下手。或有那正經婦人。雖癢死不肯辱身的。他還有一種迷藥。也是細末子。不拘飮食中與人吃了下去。便昏昏沈沈。四肢動不得。口中說不得。任他淫媾。那大人家婦女。深房邃室。他如何得見。就行此惡術。他只先勾上了一個或是貪淫的僕婦。或是那好弄的丫鬟\footnote{大約丫鬟無有不好弄者。}。便替他做事。他也奸過無數良家婦女。他不但有好春舌可以鼓動好淫婦人。且自己養得那龜有七寸餘長。又粗又久。可以通宵不倦。所以貪淫婦人經過他一次。死心塌地戀着他。不想這單于學該倒運。請了他來家。細道病原。求他醫治。他道。貴恙乃少年時斲喪太過。陽氣虛弱之故。非一朝一夕可以奏功。必須靜養百日。早晚服藥調理。還得兩個少壯婦女。常常按摩丹田湧泉二穴。子午卯酉四時。兩處呵氣食頃。使他少年壯陰之氣上下齊攻。引陽氣歸於腎經。百日之後。不但堅舉。且大勝往昔。須得居士到外邊來住。等貧道看着他們作爲方可。單于學大喜。連聲道謝。若大癒後。許其重謝。就吩咐取兩副鋪蓋到書房中設下。那三間書房是一明兩暗。東一間他同道士睡。西一間作丫頭的臥處。小廝們都打發出去。叫了花蕊花鬚來服侍。須臾。送上酒來。二人對飮。翟道見了兩個丫頭。好生動火。吃完了酒飯。翟道開了一個藥單。叫打了藥來炮製丸藥。無非是參苓。桂附。肉蓯蓉。淫羊藿。虎脛。鹿茸之類。又叫單于學仰臥在榻。翟道敎那二婢如何搓抹。如何呵氣。那兩個丫頭雖然騷浪。到底是少年女子。見道士在傍看着。未免有些羞澀之態。單于學道。翟道爺是有德行誠實君子。你們羞甚麼。他二人只得依方呵摩。到了三鼓子刻。又叫起二婢如前作用。過了一宿。次日。這賊道有些按納不住。見兩個丫頭呵時。不住望着他微笑。那丫頭也紅着臉。低着頭笑。翟道越發魂銷。想道。今晚下手罷。他到了酉時。看着單于學做完了工夫。掌上燈來吃酒。飮了一會。翟道推辭不用。單于學斟了一杯。親奉與道士。道。我敬老師這一杯。翟道正中心懷。接過飮乾。暗將那迷藥入了些須在內。也斟了一杯回敬。單于學那知就裡。忙雙手接來。也一氣飮乾。翟道道。兩日二位姐姐也辛苦了。每人也用一杯。將單于學的杯同他的杯滿斟了。也暗入了藥。遞與二婢。他兩人不肯接。道。我們不會吃。單于學道。道爺賞你。怎麼不吃。二人只得接過吃了。翟道道。酒止了罷。居士安歇養神要緊。單于學依他。便各自去睡。那二婢也往西間去了。約有一個時辰。翟道知藥性已發。悄悄下床。走過西屋。種火上前點上燈。見着那二人時。在一張床上並枕而臥。將被掀開。見他都穿着衫褲。以便夜裡起來服侍主人。翟道替他都脫光了。燈光之下。見二人體白如玉。又拿燈照看他二人的陰戶。眞個可愛。麈柄突興。就爬上花蕊的身上。弄將起來。那丫頭似夢非夢。矇矇矓矓。心中雖覺有人弄他。却動不得。說不出。他自從主公陽痿之後。有多半年不嘗此道。今遇着這又粗又大又久的妙具。且戰法高強。眞樂到不可言處。翟道弄了一會。又到花鬚的身上去弄。週而復始。足足被他弄了一夜\footnote{蝶蜂所採者。花之鬚蕊耳。故二婢先爲其所淫。}。五更藥力將解。他才回到東間去睡。天亮時。兩個丫頭醒轉來。各人自思夜間之事。難道是做夢。却像有人壓在身上一般。覺得胯中濕漉漉的。伸手一摸。淫液淌了兩股。連褥子都濕了一塊。心中甚是疑惑。忽然想起睡時穿着衫褲。此時如何脫得精光。越發吃驚。兩人互相細問夢中情景。所遇皆同。猜測不出。只得起來。忙梳洗了。到主人處。以待卯時摩呵。那單于學也到日出方醒。見翟道在床上打坐。說道。昨夜失眼睡着。誤了子時的工夫了。翟道道。日間卯午酉三時行得到。也就罷了。夜間不但居士勞頓。即他二位起倒也甚辛苦。可以不必罷。居士倒不如夜裡安臥。養了神氣更好。此時翟道放個屁。單于學都是要欽此欽遵。也就信以爲實。午時又摩呵了一陣。單于學覺得渾身通暢。不覺睡去。花鬚花蕊也偷空去西屋裡閒坐。想起昨夜的事。又受用又動疑。花蕊問花鬚道。我夢見的有多長多大。與爺的雖差不多。却一次的功夫抵得他幾十次。你覺得怎樣。花鬚道。我同你夢的一般。不但長久。又弄得在行。下下皆中癢筋。我們今日夜裡睡醒着些。再要夢見。明明白白的受用一會。不強似昏昏沈沈的麼。花蕊道。不要講折福的話。夜間要做這個樣的夢。也就是造化了。正說笑着。那翟道見單于學睡着了。走過來要調戲他二人。見了低聲笑道。我有一件疑惑的事來問你二位。我昨夜夢見到這屋裡來同你二位睡了一夜。你們可曾夢見麼。兩個丫頭正疑惑這事。聽了便道。我們也夢見來。道爺你細細說來看可對。翟道笑道。我說了。你二位不要見怪。我夢見走過來。你二位都穿着衫褲。我替你們脫了。輪流着弄了一夜。指着花蕊道。你的身子瘦怯。兩個小小奶頭貼在胸前。下身微有幾根氄毛。大大一個花心。裡面倒乾爽。抽着緊緊的。甚覺有趣。又向花鬚道。你比他胖好些。奶頭雖大。却圓緊緊的好。底下好件寶貝。眞像個饅頭一般。緊緊就就。指頂大的一個花心吐着。弄在裡頭。肥得有趣。抽得一片聲響。弄到天將亮。我忽然醒來。却在那邊床上。你說奇不奇。你們夢見的是怎樣。我說的可對不對。兩個丫頭見說的一絲不錯。笑道。你說的是。倒是我不信怎有這樣的奇夢。翟道道。大約是我該同你兩個有緣。故此就做了這夢。就一隻手拉着一個在懷中。道。你二位要不棄。我今夜來同你們圓圓夢。何如。那兩個丫頭只是嘻嘻的笑。也不答應。翟道知他心肯。就每人親了個嘴。兩隻手便伸到兩人胯下去摸。二人故意用手遮掩。翟道笑道。夢中弄了一夜。此時還怕甚麼羞。他兩個就笑着鬆了手。道士扯開褲子摸了摸。笑道。好兩件寶貝。今夜我有福消受了。花蕊道。你夜裡過來。倘我家爺醒了。怎麼處。翟道道。我有一種瞌睡藥。人若吃了。一夜睡到天亮。遂在腰中取出個小葫蘆來。倒出有數錢。道。每次用四五分就夠了。用紙包好。遞與他。道。晚上吃酒時。放在你爺的鍾內。包管他大睡。咱們好放心行樂。花蕊接過來。扎在汗巾頭上。翟道道。怕你爺醒來。我過去了。晚上你兩個脫得光光的等我來圓夢。笑着走了過去。兩個丫頭巴到天晚。主人吃酒之時。就依着賊道行事。飮畢。單于學睡了。翟道忙走過去。爬上床。往被中一鑽。那一對小妖精果然脫得光光的等着。翟道到花鬚身上就大幹起來。弄了一會。又同花蕊去弄。把這兩個淫婢弄得嘻笑不住。做了一夜整工夫。方才歇手。如此者兩三夜。把兩個丫頭弄得不但心花俱開。一片心爲他死都肯了。翟道見熟了。遂問他內中的事。奶奶多少年紀。還有何人。兩個丫頭就把詳細奉吿。說奶奶姓甄。生得如何標致。年紀三十二三。只是性情古板。從不輕言妄笑。還有三位姨娘。都才二十之外。各各風流美貌。內中惟有紅姨娘生得更好。那浪樣兒。不要說男人看見心愛。連我們看着都愛得了不得。翟道道。你奶奶姨娘都這樣靑春年少。你爺的陽物沒用了。他們不着急麼。花蕊道。奶奶是不好這樁的。當日就是爺好的時候。也是十日半月纔同睡一夜。別的姨娘他們怎麼不急呢。那白黃兩個還好。只急在心裡。顯不出來。那紅姨娘只急得要死。坐也不穩。睡也不安。一日長吁短嘆的抱生怨死。這些時連茶飯都減了。瘦了好些。他要夢見你。眞要快活死呢。翟道摟住他兩個。每人親了個嘴。道。好心肝。你們要把奶奶姨娘總成我弄上了。我生死不忘你們的恩。我每夜下力補報你。他兩個笑道。不知足的。有了我兩個。又想他們。你若是有了他們。還肯戀我們麼。你請休想。翟道道。你若不替我上心。我明日各自去了。大家弄不成。我來替你爺治病。原是圖你們。不然。我儘着住着做甚麼。你們當是我稀罕你爺的謝禮麼。那兩個丫頭愛他如命。恐拂了他的意。若去了怎處。笑道。他們雖然着急。知道他們心裡是怎麼樣。我們的話怎麼敢出口呢。若一時惱了。對爺一說。我們活活要死是不消說。就是你也不好。翟道道。不用你們說。只依着我行。包你他會來\endnotemark[4]尋我。花鬚道。你有甚麼妙法。翟道附在他兩人的耳上如此這般說了。就把一包藥付與花蕊。兩個齊笑道。你這牛鼻子。原來有這樣偷婦人的妙方兒。奶奶那人料道不肯。不是好惹的。且下手弄三個姨娘。等你弄到手。再作商議。翟道喜道。我且先送了謝儀着。把兩個丫頭每人痛痛的狠弄了一陣。次日。花蕊晚間上去。悄悄把那藥放在他三人的淨桶內。臨睡。他三人各小解上床。不多時。陰中忽然奇癢。說不出那種難過。只得用指頭摳挖。越挖越癢。眞癢得要死呢。那紅氏忍不住哼聲不絕。白氏隔床問道。姐姐。你怎麼的了。紅氏道。說不得。今日這東西作癢得很。癢到命裡頭去。不知甚麼緣故。白氏道。這也就奇了。我也是這樣的。眞要死呢。黃氏道。實在古怪。我也同你們一樣。要說是病。難道三人害一樣的病不成。紅氏道。哎喲。受不得了。叫丫頭弄些熱水來洗了看。叫起丫頭。點上燈。燒了熱水來洗了一回上床。〈時〉不一盞茶時。那藥氣經了熱水。比先更癢得利害。不住的摳。皮都幾乎摳塌。癢尚不止。只得忍着疼摳到了天亮。各低頭一看。腫得翻着。好像一朶翻心石榴。三人商議道。這個病又不好對醫生說的。要像這樣起來。兩三日就要送命了。前頭的那道士說他會治百病。叫花鬚問問他看可有好方兒醫治。正說着。恰好花鬚走了來。原來是翟道叫他上來探信。紅氏見了。說道。你來的好。昨夜我們三人忽得了個奇病。下身偶然癢起來。今早時看看。都腫翻了。活活的要死。你不要說是我們。只說是下人得了這個奇病。問問那道士可有甚麼方兒治得。不要叫你爺聽見。問了。快些來回信。花鬚假意去了一會。進來道。問了那道士了。他說婦人家這病是沒有藥醫的。這是男子離久了。慾心甚熾。一團的邪火攻在那裡。除非是同男人狠狠的弄兩下。火毒一洩。即時就好了。姨娘們等爺的病好了。請他腰裡那醫生一治就好了。三人齊道。我們連一刻也捱不得。你爺昨日說道士說要一百日才得好。我們捱到那時好死去。連盡七都過了。花鬚道。別的醫生請得出來。這種雞巴醫生可難尋。街上又沒人割下來賣的。只好忍着罷了。紅氏道。我們要死在這裡。你還說笑話兒呢。你替我們想個方兒救命才好。花鬚故意想了一想。道。我倒想出個妙法兒來了。不知姨娘說可行得。紅氏忙問道。甚麼妙方兒。花鬚道。道士說定要人弄了才得好。我想外邊的生人進不來。沒有個叫家下人來治的理。那道士也還精壯。到夜裡等爺睡着。我悄悄同他進來弄。弄到五更。我帶他出去。可不妙麼。白氏道。行不得。倘或你爺知道了。我們還想活麼。紅氏發急道。眼下就要死在這裡。那裡還顧得這些。且醫好了再處。就是他知道了。死也還得幾日。你們不作罷。我是顧不得了。向花鬚道。你到夜裡留神些。我開了院子門等你。只怕你爺夜裡睡了再醒了。尋那道士呢。如何是好。花鬚道。姨娘請放心。道士製了些藥酒給爺臨睡時吃了。一夜到天亮才能醒。紅氏道。旣是這等。好姐姐。你千萬不要誤了。我實實的要死呢。黃氏笑道。人說。丫頭作媒。自身難保。一個生叉叉的人。你怎好就向他說。你像是先同他有一手兒了。花鬚笑道。實不相瞞姨娘。我前日同蕊姐也得了這個病。眞要死呢。虧這道士替我兩個一醫。即刻見效。白氏道。這也就奇了。怎麼我們都害這一樣的病呢。紅氏一面哼着。一面笑道。那道士的東西比爺的怎麼樣。花鬚道。大小都差不多。工夫長得利害。又硬得怕人。就像一根短鐵棍。把我兩個整整的弄了一夜。第二日幾乎爬不起來。他還說不曾足興。紅氏向黃白二人道。你們聽聽。這樣的好東西。還裝腔做勢的怕死呢。你們不罷。且讓我快活一夜着。他二人笑道。你自己且不要拿穩了獨享。等他來再看罷了。花鬚道。三位姨娘在一處住着。二位就玉潔冰淸。誰人肯信。落得大家受用。黃白二氏笑道。倒不知道你會說媒。少不得依你。讓紅姐姐占先就是了。紅氏望着日頭道。天爺。你快些黑了罷。慈悲救命要緊。花鬚出去了。他三人巴到天晚。把院子門房門都虛掩着。澡牝上了床。側耳聽聲。等那道士。起過更一會。只見那門輕輕一推。他們住的是東廂房。這日是初八。月正照着。紅氏忙把帳子一掀。見是三個人進來。心中喜得如獲了異寶。聽得花鬚低聲道。他來了。那翟道就上床脫衣。鑽入被中。摸紅氏時。不曾脫褲。替他褪下。再摸他陰戶。腫得多大。暗暗含笑。就用陽物一頂。紅氏哎了一聲。道。慢些。疼得很。道士也不理。往內使力。一下進去一半。紅氏又哎喲了一聲。翟道又一送到根。沒稜露腦的抽。先紅氏因陰門摳破了。被他搗得疼。抽一下哎喲一聲。抽了數十下之後。內中之樂無窮。把哎喲兩個字就變成個哼字。少刻。連哼字都沒有了。只鼻孔中如母豬呼子。不住的吼吼的響。弄了多時。紅氏丢了數次。他自從跟了單于學數年。所經者十數抽而已。何嘗遇此大敵。此時不但內中之痛癢全消。另有一種說不出來的快活。身子也弄軟了。說道。你讓我歇歇罷。還有兩個人呢。你都替他們醫了去。翟道得不的一聲就抽出來。花鬚花蕊兩個坐在床沿上聽梆聲呢。見他下床。就送他到白氏床上去。道士上去摸時。却是兩個。原來夜靜了。他兩個聽得道士同紅氏弄的那聲息。明明白白。幾乎心中急死。黃氏恐道士到白氏床上再弄這些工夫。如何捱得。遂走來同白氏共臥以俟。道士把他兩個都脫光了。先到白氏身上。一面弄着。一面伸手去摸黃氏的牝戶。將白氏弄了一會。就到黃氏身上弄。如此轉換。弄了將有一個更次。只見紅氏精光着爬上床來。道。怎麼你兩個占住他。不放到我那裡去了。我們大家到一處來罷。見道士正同黃氏弄呢。他生拉到身上來。又弄了一會。才一家一度相輪。聽得外面已五鼓將盡。只得放道士出去。囑道。我們但是叫他兩個去請。你千萬就來。道士應諾。兩個丫頭同他出去了。這三個婦人在極癢之時。遇了道士這硬大之物。只弄得渾身骨酥筋軟。次日精神了許多。紅光滿面。你看我。我看你。不住的嘻嘻笑。這一夜。道士在書房同二婢弄了個滿心暢意。以報其成就之恩。次日又約了進來。仍是四個同床。弄過了一遍。道士道。承三位姨娘不棄。小道感激不淺。不是小道貪心。我常要進來陪伴三位。恐上房的奶奶知道。非同兒戲。除非連他一網打盡。方保無事。姨娘們尊意如何。紅氏笑道。誰說我們是姨娘。定是兩個丫頭賊嘴吿訴你的。你方才說的話固然是。但奶奶的性格比不得我們圓活。誰敢去捋虎鬚。翟道道。小道自有妙法。昨日三位姨娘不是小道的妙法。怎得來親近玉體。白氏問他原故。他把同二婢所設之計細細說出。紅氏笑着將他擰了幾下。罵道。原來是你這個賊道弄的鬼。幾乎把我們癢死了。翟道笑道。不是這一癢。怎得有後來的受用。黃氏道。要想刮上奶奶。除非把他的夜合兒弄上了。在內中行事才中用。翟道道。我有些末藥。明日姨娘們不拘誰給他茶酒吃。入在內中。他吃了下去。下身更癢得利害。再煩位姐姐去一勾。不怕他不上我的路。叫過花蕊來。托付與他。明日如此行事。次日早飯後。他三人同花蕊正在算計夜合。要了乎(壺)酒來。低聲說笑。只見夜合笑嘻嘻走了來。道。我才見姨娘們要了酒來。就不賞我盅吃吃麼。衆人正算計他。恰好尋上門來。就暗下了藥。斟了一杯給他。他接過來。一口吃了。又給了他一盅。他呷了。道。我夠了。多了臉紅。怕奶奶罵。就走了去。花蕊留心看着他。不多時。見他走到後院子裡去了一會。才走出來。少刻又去。來回如走馬燈一般。花蕊知是藥的緣故。就悄悄隨他到了後院。見他坐在一塊搥衣石上。褪了褲子。低頭看着。拿手摳呢。花蕊低聲道。夜合姐。你做甚麼呢。夜合擡頭見是他。忙扯衣服蓋了。笑道。姐姐不要笑話。我今日要死了。花蕊道。你是怎麼的了。他道。不知甚麼緣故。我下身癢得要死。摳了這半日。差不多要爛了。也不得好。怎樣的呢。花蕊道。我會醫。夜合道。你不要說謊。你又幾時會做醫生呢。你只會替爺拔水罐子。那裡會醫我這個。花蕊道。我是正經話。我時常也是這樣的。爺給了我個假㞠子。搗一陣就好了。夜合道。好姐姐。你就是我的親媽。你借給我用用。花蕊道。那是我救命的寶貝。怎肯借給人。你夜間到我屋裡去。我替你醫醫還使得。夜合道。我在奶奶房裡睡。怎得下去呢。花蕊道。等奶奶睡着了。你悄悄下去。不過一會兒就好了。若奶奶知道問你。只說肚子不好。在屋裡上淨桶怕熏了奶奶。就瞞過去了。夜合道。不中用。你是哄我。你在前頭伺候爺呢。怎得進來。花蕊道。有鬚姐在那裡是一樣。我旣許了你。定然進來。他道。好姐姐。你醫好了我。替你磕頭罷。夜合夜裡聽得甄氏睡熟。悄悄起來。輕輕開了房門。到西廂房門上一摸。果然是掩着呢。走進去。悄悄叫道。姐姐。你在那裡睡呢。花蕊下床拉着他。道。你上床脫光了等。我就來。夜合忙上床脫光仰臥。只見一個人上床來。爬上身。摸着他陰門。往裡就頂。夜合道。好好。就是這樣狠狠的就好。果然就狠搗了數十下。他叫道。我的娘。好東西。眞是個寶貝。我摸摸你是怎樣拴着的。弄得這樣好。比爺的強大了。伸手一摸。竟是連根生的。驚道。姐姐。你原來是個男人。緊緊摟住。道。我早知道你是男人。就不癢也早來尋你了。你是這樣個東西。爺怎麼同你弄來。哦。我知道了。想是肏屁股。又道。不是。不是。我記得你熱天洗澡。我看見是同我一樣的扁貨。這是幾時長出來的。花蕊在床脚頭笑道。說夢話的。不要嚼蛆了。我可憐見你。替你請了給爺治病的道爺來救你。他才不作聲。那道士一陣大弄。夜合道。好道爺。我也沒甚麼酬謝你的。捨着這東西。憑你弄罷。道士附耳道。這算不得。還要尋個別的謝我。夜合道。可憐我有甚麼。還有一個屁眼。你若不嫌棄。說不得我忍着些。也憑你受用。道士道。我不愛後面的。還要一個前面的。夜合笑道。我一個人那裡來的兩個屄。要有兩個倒好了。巴不得送你。得兩處受用。道士道。你沒有。你奶奶身上有。你送了我。就是謝我了。夜合道。我倒肯。恐他未必肯。道士道。只要你肯。他自然就肯。夜合道。我不懂得你的話。道士道。我有一點末藥。只要你明晚上倒了他的馬桶。放在裡面。等他用過。自然就肯了。不要你管別的。況且他要肯了。你也得長久快活。夜合道。我巴不得的呢。別的我做不來。你把藥交付我。道士又弄了一陣。放他起來。穿了衣服。遞了藥給他。再三囑咐。那丫頭被他弄得千肯萬肯。欣欣上去了。道士同花蕊到東廂房。向他三人說了。大家歡笑了一會。又各弄了一陣出去。次日。夜合依着道士行事。甄氏睡下。不多時。陰中癢得難當。想道。我從來沒有這樣。況我又不曾動淫心。怎得如此。我只秉住心睡着了便沒事。睡了一刻。那內中如千萬蟲子在裡面爬鑽。癢得實實難受。由不得也就摳摳。直到天明。不曾合眼。次日。雖說不出口。那面上的火。一陣陣上攻。癢得連飯都吃不下。夜間仍復如是。要吿訴丈夫請醫生來治。自料這話難向醫生說。只得死忍。又捱了一夜。第二日。夜合向花蕊道。用了藥兩日兩夜了。總不見他怎樣。只是夜間在床上有些聲聲氣氣的不睡。虧他忍得。難道是鐵的不成\footnote{昌氏倒是鐵的。若經此藥。更忍不得。}。花蕊又吿訴了賊道。翟道笑道。我給他一個雙掭燈。看他可還忍得。又取了些藥遞與花蕊。道。你悄悄交與夜姐。叫他不論茶酒中給他吃。花蕊付與夜合。夜合到甄氏要茶吃時。將藥與他吃下。過了一刻。前癢未退。後癢又加。這却癢得要死了。先兩日是陰門內癢。還摳得着。這一癢在內中深處。指頭摳不着了。急得坐立不安。下身只是扭。兩眼睜得多大。咬着牙死捱。丫頭們見了那樣子。吿訴了翟道。他夜間進來時。笑對紅氏三人道。奶奶雖然不說。也實實難受了。此時大約我去。諒他也不拒。但恐一時有變。明日再送他一個瞌睡蟲。暗暗去救他一救罷。又把迷藥付與花蕊。叫他遞與夜合。明晚給奶奶吃了。夜間起來開門。不要誤了。次日。甄氏一覺睡去。明明一個男子奸他。要推。手擡不起。要叫。口又叫不出。要掙。身又動不得。急得心中要死。約弄了半夜。方才不在身上。天明醒來。陰中已不癢了。想道。難道是夢。我又不心邪。如何有這樣惡夢。要說是眞。此人從何而來。門又關着。從何而入。難道是妖怪。我無一點苟且之心。妖自何興。解說不出。只得罷了。次夜無事。第三夜。他貞心不昧。雖然口啞身禁。心中頗明。隱隱覺得夜合息息索索起來開門。少刻。就有個人替他解帶裩淫媾起來。心中雖怒急。總不能一展動。半夜去了。到天明醒起來。忙看房門時。又是拴着。小衣仍穿得好好的。但陰中覺有些不淨。想了一會。已悟了幾分。道。這事夜合必有緣故。這幾日花蕊花鬚時常同他交頭接耳說笑。定是他三人同謀。我若正顏厲色的問。他們決不敢承認。須得用言語詐他。才可得眞情。早飯後。叫了夜合到跟前。假做笑容問他道。這兩夜我覺得有個人在床上同睡。你必定知道是誰。可實在吿訴我。夜合似有驚懼之色。答道。我不知道。甄氏鑒貌辨色。知是他了。笑道。小奴才。你還瞞我怎麼。我昨夜明明聽見你開門放了他進來。還說不知道。這件樂事是婦女們求之不得的。我還惱麼。那人這樣暗暗的來。我不得明白受。可惜錯過了。旣然那人愛我。你定知情。說明白了。明明的約他進來同我會會。我還要賞你擡舉你。難道反有怪你的麼。那夜合不過是個蠢婢。那知主母心事。便笑嘻嘻的。還不肯說。欲言不吐。甄氏笑道。有話就說。怎麼吞吞吐吐的。夜合道。來同奶奶睡的。就是爺留着醫病的那道士。甄氏心下一驚。笑道。他怎麼就愛上了我呢。是誰來托你替他開門的。怎麼來時我又說不出。動不得。你細說了。我才明白。那丫頭已經說出口。料瞞不住。見主母一團和氣。滿心還想獻功。便將花蕊如何托他兩次用藥。見奶奶不動心。後又用了兩次迷藥。他才來了兩夜。甄氏道。他有甚麼好處到你。你就肯替他做事。夜合想沾翟道餘波。趁着主母歡喜。索性說出。免得後來吃醋。又將花蕊怎樣哄他去醫病。到廂房裡奸淫他也說了。甄氏呆了一呆。忖道。這惡道連我也放不過。可有放過他們三人的。又問道。你三個姨娘可同這道士有賤(奸)沒有。夜合道。這個我不知道。除非問兩個花姐姐。甄氏道。你去叫了花鬚花蕊來。他去了一會。那兩個丫頭。夜合已將前話對他說了。放心大膽的走來。甄氏笑道。你這兩個壞丫頭。道士旣然愛我。你兩個何不對我早說。做這暗事怎麼。今夜你兩個同他早些來。我同他會會。但恐怕你姨娘們知道。不好意思的。花鬚道。奶奶請放心。姨娘們早同他打做一家了。甄氏道。他們怎得上手的。花鬚也將用藥的話說了一遍。甄氏道。你們夜間常上來。不怕你爺醒來尋問麼。花蕊又將用藥迷他的話相吿。甄氏道。你們去罷。晚間千萬早來。我等着呢。兩個丫頭到東廂房。向紅氏三人說了甄氏的話。大家喜笑。以爲得計。甄氏見兩個丫頭去了。嘆了口氣。滴了幾點淚。取過筆來。寫了一張柬帖。折了壓在桌子上。午飯也不吃。將他的舊鞋裹脚並行經之物包作一包。帶了夜合到了後院。挖了個深坑埋了。夜合見他如此。不測其意。臨晚叫舀了一脚盆水在床後。他將牝戶着實挖洗了一會。嘆恨道。不意此爲賊所汚。死了還是個不白之鬼。恨了幾聲。起來徹底上下換了一身新艷的衣服。頭上緊緊扎了個觀音兜。把右手大袖捲起。拿一根大紅絲帶。叫夜合替他扎緊在肘後。那花蕊花鬚出去時。已對翟道說了。那翟道喜不自勝。打點一副精神來對付他。花蕊恐主母變卦。上來探信。見甄氏如此裝束。到廂房笑向紅氏三人道。每常還說奶奶怎樣古板呢。看他今日。比我們還浪。一個偷漢子。還打扮得像新娘子一般。他三個笑道。他兩個上床。還不知怎樣肉麻。晚間老道上來時。你知會我們一聲。大家去張張。花蕊答應。又去了。日落之後。甄氏叫夜合掌上兩根大燭。單于學的祖父在嘉靖時曾做京營游擊。那時倭寇臨城。他得了一口好倭刀。又輕又快。寶藏了三輩。日日懸在壁上。常常吼哨。甄氏取了下來。輕輕拔出。攥在手中。光芒奪目。見夜合在床後鋪他的鋪。甄氏走到他背後。怒從心起。惡向膽生。將刀揚起。儘力向脖子一下。雖然他的力小。因恨極了。刀又利。已砍得那頭伶仃將斷。一交跌倒在地。甄氏出來。在靠桌子的一張椅上坐下。將刀放在背後。等他三人。定更後。翟道同兩個丫頭興興頭頭歡歡喜喜的走了上來。花蕊忙知會了紅氏三人。三個忙跟了來張。窗眼內見他三個進了房。那甄氏一臉的怒色。面貌鮮紅如血染的一般。坐在椅子上。動也不動。他三個還以爲是他假裝羞怒之色。要道士竭力賠禮之意。只見那賊道到跟前。叫了聲。奶奶奉揖了。一恭到地。只見甄氏的手一揚。一道亮光如閃電一般。那道士已撲在地下。花鬚驚得呆了。哎呀一聲。只見甄氏手中的刀起。劈面刴來。花鬚仰跌倒了。花蕊才回身要跑。被甄氏搶一步趕上。後心一搠。刀尖從前胸穿出。撲的便倒。回身見那道士還掙夾(扎)。後心一連搠了幾刀。紅氏三人嚇得魂飛膽喪。兩腿都驚木了。要跑又跑不動。又恐他出來要殺。心中亂跳。連渾身都軟了。沒奈何。用手搬着窗欞站着還張。只見甄氏那臉越紅。柳眉剔立。好不可畏。他仍還坐在椅子上。不出來殺。心才略放了些。那甄氏手拿利刃。怒還未消。已想到須將那三個淫婦也殺了。才出得這一口惡氣。但他一個嬌怯的婦人。猛性殺了四個人。也就軟了。忽然心中一回。道。他三個固該殺。但被妖道淫婢所惑。情尚可原。所可恨者。他不能死耳。他三人張着甄氏。見他口中嘓嘓噥噥說了幾句。低頭沈吟了一會。忽然長嘆了一聲。大聲道。原難。原難。將手中刀向項下一橫。鮮血直噴。他便倒在椅背上靠住不動\footnote{此一段夾寫甄氏動手。紅氏三人張看。敍着甚妙。甄氏說原難。原難。要知非說紅氏三人當死之難。乃謂受藥時難忍。故爲所淫耳。即所謂尚可原者耳。}。他三人嚇得越發要死。你攙我。我扶你。跌跌爬爬。滾到廂房。三人擠作一床。各人扯了被蒙頭蓋上。渾身篩糠打戰。不在話下。次早。單于學醒來。不見了道士。以爲他去出恭。還不以爲意。叫了兩聲丫頭。又不見答應。以爲他們有甚麼私事。忙穿衣起來。到西屋去看。並外邊去尋。不見了三人。疑是道士拐這二婢去了。大呼家人查看門戶。皆局鎖甚嚴。心中甚疑。到上房來。見院子門大開。更覺可駭。走到東廂房一張。不見動靜\footnote{妙。先疑三妾或有原故。}。再看了西廂房門。又是鎖着\footnote{二婢決無約他進來之理。然不得不疑到此。妙。}。疑道。難道道士竟在上房不成\footnote{却不道怎麼。}。但我妻子不是淫賤的人\footnote{有此一句。方見甄氏平日之貞。}。走上去。見房門也開着。遂幾步搶了進去。一眼先見甄氏一身鮮血。右手持刀擱在膝上。面貌如生。怒氣勃勃。急到跟前看時。頸上痕深寸許。喉已兩斷。道士撲在他跟前。身上血痕遍滿。兩婢也殺了。到床後一看。夜合也被殺死。單于學急渾了。一眼看見桌上有個帖兒。忙取過一看。寫道。

\begin{quotation}

妖道淫婢合謀。以術魘我。汚我淸白之軀。今手刃之。以雪其恨。痛此身已辱。無顏再事君子。冥冥中未免遺憾耳。永訣良人。傷心泣血。願郞自玉。勿以賤妾爲念。辱妾甄氏絕筆。

\end{quotation}

單于學看了。放聲大慟。紅氏三人聽見。只得起身上來。也就假哭。單于學哭了一場。問他三人可知情。他們恨不得多生出幾張口來。說得自己身上乾淨。連說了幾十個不知。單于學連柬帖拿着。親到縣中去報。那知縣是他認的老師。也不委屬員。親自帶了忤作來驗。見了甄氏奶奶好好坐着。面色不改。十分驚異贊嘆。忤作驗了。報道。殺死道士一名。腦後刀傷一處。背搠刀口七處。大約係行強奸。故被殺死。砍死丫頭一口。腦後兩瓣。搠死丫頭一口。胸口對穿。床後殺死丫頭一口。頭顱伶仃將斷。大約係三人同謀。引入道士。故一時怒殺。甄氏係自行刎死。兩喉俱斷。知縣見他那遺字。知他已被淫汚。無處查考。又不肯汚了烈婦的名。向單于學道。令政英氣凜然。我自然呈報上臺。表請旌獎。可即殯殮。道士同三婢屍骸。應該置於極刑。已死勿論。即行抛棄。以飽鳶鳥豬狗。稍伸烈婦之恨。說罷。回衙去了。單于學即命家人將道士三婢抛出。棄於荒郊。殯殮甄氏。將那口刀裝在棺中爲殉。不用細說。知縣申報了上臺。上本啓奏。奉旨甄氏賜贈孺人。建坊。大書四字。

\begin{quotation}

香閨烈士。

\end{quotation}

出殯下葬時。甚是熱鬧。那些鄕紳士夫。文人墨士。都作了挽歌詩詞來弔奠。知縣佐二都親來燒紙。甄氏雖被賊道所汚。死後之榮倒也不小。紅氏三人自那日嚇破了膽。日夜心驚肉顫。疑心生鬼。但合眼便見道士同那三婢血淋淋在面前。又見夜合罵道。都是你三個淫婦下藥我吃。害我到這個地步。快還我的命來。他三人愈加驚怕。前已嚇破了膽。今又夜夜夢見衆人索命打擊。竟嚇得瘋瘋癲癲。兩目直視。叫道。夜合打我們還罷了。你兩個弄藥來害我三個。才捉弄奶奶的。怎麼你也打我。家中婦女聽見他人人如此說。就借着口氣問他始末。他三個將花鬚花蕊如何替道士用藥害他。因而成奸。又如何勾引夜合。後來又用藥害奶奶。詳細說出。衆人方知這些緣由。過了數日。三人相繼而歿。單于學年過三旬。尚無子嗣。自甄氏死後。大悔少年之非。改過自新。再不貪淫。他將那道士的藥早晚服下。買了二婢。還行那摩呵之法。果然到了百日。陽具竟硬了些。可以動作。他感〈之〉甄氏之死。不忍再娶。就把這二婢收在跟前。後來竟各生子女。單于學因貪淫兩個字。好好的妻妾弄得如此落場。幸而改過。始得血嗣未斬。古云。福善禍淫。豈不然哉。宦蕚聞知了詳細。着實贊嘆。上馬而回。正走着。又見許多人在那裡圍住着\footnote{江南風俗。街上勿論有大小事。即圍上無限的人看。所以謂之呆鵝頭也。}。宦蕚也打馬擠了進去。見一個二十多歲的少年。滿臉滿身是血。口中道。像我這待哥哥。也就夠了。反這樣不公平。倒下死手打我。一個大漢一臉橫肉。疙瘩麻子。黃鬚白眼。上身赤剝着。惡狠狠拍着胸膛道。我打了不怕你。你只管去吿。一個老者背着臉向那大漢道。你這奴才。這樣凶惡。難道官府衙門都沒有王法處治你的麼。那大漢道。老叔不要偏心。都是你姪兒。不犯着擡一個滅一個。冷竈裡一把。熱竈裡着一把。手掌看不見手背。勸你老人家將就些罷。不要太做絕了。揸手舞脚。一跳八丈的。那老兒怒起來道。你欺負兄弟罷了。難道敢打我叔叔麼。轉過身來。宦蕚素常認得他這人。姓曾名好義。字公道。是個年高有德的人。宦蕚忙跳下馬。你道他所遇這人所爲何事。要知詳細。下回便見。

姑妄言第十九卷終



\endnotetext[1]{批註「將此」原作「此將」,據文義改。}

\endnotetext[2]{「我家」原作「家我」,據文義改。}

\endnotetext[3]{「駐雲飛」原作「駐飛雲」,據文義改。}

\endnotetext[4]{「會來」原作「來會」,據文義改。}

\setcounter{footnote}{0}

\theendnotes

\part*{姑妄言第二十卷}
\addcontentsline{toc}{part}{姑妄言第二十卷}
\markboth{姑妄言第二十卷}{姑妄言第二十卷}

鈍翁曰。秉公道之人。在嫡親姪兒跟前。亦爭不去。誠可太息。爭家禮者。越行不得。倒不如凶惡而爭家財者。還得便宜。然便宜雖然占去。而殺才之名已布於鄕黨鄰里矣。

因二十金之故。便致父子割恩合氣。蘇季子貧窮則父母不子之嘆。千古同然。

薄氏之薄。大約已非一日。方器生之氣。亦未必今日方纔氣生也。今値方生氣之時。恰遇宦蕚。得其解囊一贈。氣者不氣。薄者不薄。銀之爲銀。眞通神之物也。此寫薄氏欲去而未去。前寫權氏〇〇〇〇〇〇去而仍是未去。妙。

詳寫刁桓父岳之結局。非無味之贅筆。亦是勸人做好人之〇〇〇〇揮欺寡婦孤兒。謀奪其職。刁千戶夫婦終日醺醺。〇〇〇〇〇〇〇只取快一時。生此等子女。以至滅門出醜。悔〇〇〇〇〇〇〇〇〇男子之身已終。只剩一母氏寡居苦守。

爲殮乃必至之苦情。幸鄰居一有美一有〇〇〇遇宦蕚而〇〇〇使屍骸不致暴露。子女皆有所歸。宦蕚之陰功固大。而聖人里仁爲美之言。不可不知。

丱角之交。因些微小利。以至性命相搏。恐此人面獸心之朋友世不乏人。

勢敗奴欺主。古今一轍。沒奈何之懦主遇無天良之惡僕。將奈何。向小娥所勸。宦蕚所行諸善事。一則見小娥之才。二則總是惡(要)宦蕚做到一個絕頂的好人。

瓊州府知府焉得還窮。其窮者。因有沒福之子故耳。其子沒福。家業一賭蕩盡。幾至流爲餓殍。雖有後而實沒得後矣。所以子名牧福。父名牧德厚也。屈攀桂仰氏旣屈於下僚。而仰攀富貴之上司以爲榮。得一沒福之婿。只圖目前之熱鬧。不慮兒女之終身。何其愚也。若不遇宦蕚。其女尚可言哉。可爲攀高結貴者戒。幸其女名紳姐。故屈而尚有能伸之時。後隨父之通州也。

屠四。刁桓。曾嘉才。與衆賭棍同此一結。不但了去衆人。且見放賭者。好賭者。局賭者。一遇廉明官府。如魑魅之見皎日。自然盡化爲烏有矣。詳寫曾嘉才之妻女子媳者。因一賭字。以至家破人亡。可見賭字大害。一至於此。貪賭之流見之。亦知稍警醒否。作者之意是要勸諸人不可如此。切勿錯會起來。竟去效顰。不但負作者之心。眞成一大笑話矣。寫宦蕚在賈文物家豪飮。非謂其量宏也。特寫其大醉後。尚能有不平之鳴。與裸婦同臥。猶能自持。較坐懷不亂尤難。總是要將他高擡到十二分地位。

趙酒鬼與正傳雖無涉。寫賭字之害已畢。更寫一酒字之害以做襯耳。宦蕚代衆窮黎還拖欠。雖是一片熱腸。然對知縣所說的話。仍然膏粱公子氣味。故妙。他雖心地變好了。如何便能一旦貫通到無所不知的地位。仍帶三分呆氣者。寫公子不得不如此。看者要知作者之心。因要寫公子之呆。非作者之有呆筆也。看者勿被作者又笑其呆。

宦蕚之美事敍完。而用兩個同心報德之人以終之。妙絕。先用一開首之賴盈報信。總結上文。更妙而又妙者也。

兩回大書。受宦蕚之恩德者多矣。無不領而謝之。只頭一個劉太初竟却而不受。出人意外。有衆人之受。方完宦蕚之善心。有太初之不受。方顯其高節。

宦蕚失身在泰安州。妙甚。泰安者。太安也。以爲至此安然無慮矣。不意反致被盜。人生快意處常失意。亦同此類。

宦蕚領回官誥。雖與積德事無關。這兩〔回〕書將宦蕚善事寫完。見冥冥之中亦報其德。使祖父受朝廷之恩榮。恐人看不出。故寫途遇鮑德。又爲寫一報德同心之人。直送他到盧溝橋也。

\chapter*{姑妄言卷之二十\\
第二十回 受恩百姓男婦感洪仁 積德賢郞父母膺上壽\\
附 屈氏一意捨身報恩 宦蕚兩番坐\endnotemark[1]懷不亂}
\addcontentsline{toc}{chapter}{第二十回 受恩百姓男婦感洪仁 積德賢郞父母膺上壽}
\markboth{第二十回 受恩百姓男婦感洪仁 積德賢郞父母膺上壽}{第二十回 受恩百姓男婦感洪仁 積德賢郞父母膺上壽}

話說宦蕚見了曾公道。忙下馬近前。舉首(手)道。公老\endnotemark[2]爲甚麼動怒。他一看。認得是宦公子。忙舉手道。失瞻得罪。尊駕往那裡去來。宦蕚道。偶從此過。見公老在此說話。故來聽聽。這二位是誰。有甚麼事。以致你發怒。曾公道道。老爺。你是位貴公子。明理的人。見的又多。你就評評這個是非曲直。這是我兩個舍姪。指着那個大漢道。這是我前頭先嫂生的。名字叫做曾嘉才。指着那一個一臉血的小後生道。這是我先兄續絃的先嫂生的。名字叫做曾嘉禮。大的這個奴才。小時不知花了先兄多少銀子。先兄當日還有幾千金過活。單替他娶媳婦。就花了七百多銀子。前年先兄臨危時。請我到跟前。替他二人分家。房產地土一樣均分。只有一千兩銀子。先兄是極公平的。說道。大的若論起來。這銀子他一分也不當得。他用過何止千金。今日若單給小兒子。人未免說我偏心。這銀與大的三百兩。小的七百兩。他雖然分的多些。他還不曾娶媳婦。要論起。大的當日娶親。就差不多用了七百兩。這只算與小的娶親的銀子。家私還不曾分着一個錢呢。去年大的這奴才。又刻薄。又不長進。龍天不佑。把一分家私就輸得精光。着了急。來同這小的鬧。說他多分了銀子。小的還知道些人理。請了我到他家。他道哥哥輸光了。看着他那樣子也過不去。把他父親多與他的那二百銀子與了哥哥。這却均分了。說了個斷絕。此後再不許胡鬧。當初。先繼嫂問他娘家要了個小丫頭服侍。後來先嫂去世。這丫頭就歸到小舍姪跟前。至今也生了兩個孩子。大的這沒廉恥的奴才。不好鬧銀子了。要來分這丫頭。小的說。不要說我這丫頭是母親問外祖家要來的。就是父親銀子買的。今日跟我兄弟養了兒女。哥哥也不好賣了分的。大的決定不依。說。你要留這丫頭。該多少身價。要兄弟沖出那一半銀子來與他。小的急了。說。你當日娶嫂子費了七百兩銀子。也該沖出一半來給我。他沒的說了。說兄弟把嫂子比了丫頭。又賴他說要賣嫂子分銀子。把兄弟打得頭破血出。老爺你請想。天下可有這樣不公平的事。我來說他兩句。他還往着我跳。老爺你請看看。他那氣象可看得。我定要送他到官。處治這奴才。纔出得這口氣。曾嘉才翻着眼睛瞅着他叔父。道。我勸你老人家將就些兒罷。不要太做出來給人看。我知道你老人家衛護他。鵓鴿兒揀旺處飛。他是有錢的姪兒。自然該心疼的。你老人家要送我到了官。料道沒有我的死罪。我出來不打死他。也不是人娘養的。拚着替他償了命。大家撂開手。那時你老人家也沒有偏的了。那老兒越發怒起來。上前要拿頭撞他。宦蕚拉住他。道。令姪那種氣質。叔叔都不認得。人倫都沒了。可是同他講得理的。公老。你是盛德的人。不必與他較量。若經了官。徒傷骨肉之情。知道的是他理虧。不知者還道是你偏護。這種人不睬他就罷了。那曾嘉才自幼不孝不友。俗語說的。天是王大。他是王二。毫無忌憚。人背地起他個混名。把他的名字改了一個字。都叫他曾殺才。他聽見宦蕚說了這幾句話。那裡還依得。因見他樣子體面。還不敢十分動粗。只氣狠狠的白瞪着眼。望着宦蕚道。我各人家的事。用不着你費心。別扯騷蛋子。老廖怎麼死了的。操心死的。一個鼻子三眼。多出了一口氣兒。一條褲子三條腿。多了你這個管。這纔是賣蘿蔔的跟着鹽擔子。鹹操心。旁邊看的人認得宦蕚的。齊都喝道。你這人紅了眼。人也認不得。這是宦大老爺。說的是好話。你滿口胡說的是甚麼。他聽見是宦公子。也就軟了三分。不敢再說。宦蕚聽了他說那幾句可惡的話。心中大怒。又回想道。這樣不孝不友的下流奴才。我同他一般見識做甚麼。冷笑了一聲。問他道。你到底要你兄弟多少銀子。他道。那丫頭爛不濟也値五十兩。我該得廿五兩。宦蕚叫小廝稱出廿五兩銀子來。對曾公道道。公老。我看你小令姪還是個孝弟知禮的人。我與那凶徒這銀子。替你小令姪解了兄弟之仇。又向衆人道。列位親翁皆在這裡。這個惡人不是我沒本事處治他。我今要處了他。他方纔罵了我。人不知道的說我小器。我如今倒給他這銀子。此後他再來與兄弟打鬧。叫他兄弟去對我說。我送他到衙門裡。替曾家除了這一害。叫小廝將銀子撂與曾嘉才。宦蕚道。曾老不必生氣。也請回罷。曾公道道。寒家不肖的事。倒破費老爺。同着嘉禮作揖謝了。宦蕚向衆拱了拱手。上馬而去。那曾嘉才拿着銀子。披上衣服。敞着胸。欣欣得意也去了\footnote{是個下流無恥的人。潑皮形狀。}。宦蕚正走着。見一個老兒拉着一個小夥子。許多人在那裡勸。宦蕚看那老兒有些面熟。一時想不起他的姓來。問他道。你老人家好面善。你爲甚麼事。那老兒認得他。答道。宦老爺。我是葛子恩。你貴人不認得我了麼。這是我那不長進的兒子。叫做葛器。我一生一世苦掙了廿兩銀子。我兩口子都年老了。留着做棺材本的。他毆死毆活定要借去做生意。去了幾個月。不知在外邊怎樣嫖賭。花光了回來。說是折了本。這樣不孝的奴才。我定要送官處死他。宦蕚道。你老人家有幾位令郞。葛老道。這一個就足夠了。我還禁得有幾個。宦蕚道。你旣然只這一個。要送了他。後來老了。靠誰發送。他道。我死了。靠這奴才。還有本事掙口棺材與我麼。不過是狗拖豬啃。不如今日送死了他。我且出這一口氣。沒有他。我倒罷了。古語說。街死街埋。路死路埋。倒在陽溝裡。就是棺材。我也顧不得這些了。宦蕚問葛器道。你怎就花了你父親的銀子。叫他這樣的恨怒。割恩絕義的。葛器道。老爺。這事寃屈死人。我又不嫖又不賭。如何會花。時運不濟。兩三次生意做不着。就折得個精光。我家老爹和我合氣。咬住這麼說。叫我沒得辯。只得憑他老人家罷了。宦蕚叫小廝稱了廿兩銀子做棺材本。〔道。〕你父子好好的回去罷。那老兒笑嘻嘻的道。怎敢當老爺賞。一面推辭。一面就納之於袖了。葛器叩謝。宦蕚拉他起來。他父子二人歡歡喜喜。一點怒氣也無。和和氣氣說着話回去了。宦蕚騎上馬正走。忽見一家門口站着一個三十多歲的漢子。氣憤憤的。臉脖子脹得烏紫。靠在門枋上。內中一個婦人潑聲潑氣的大罵。宦蕚勒住馬。問那人道。你姓甚麼。爲甚事氣得恁個樣子。那人正受了一肚子髒氣。沒得訴處。聽見問他。往內指着道。老爺請聽聽。宦蕚側耳聽時。那婦人罵道。窮忘八。人家嫁漢子原是圖吃圖穿。叫我成日熬淸受淡的。你旣沒這屄本事養活老婆。留我做甚麼。你與了我休書。像我這樣的能幹老婆。不是說大話。怕嫁不出好漢子來麼。三隻脚的蟾尋不出來。像你這兩隻脚的漢子。要無千帶萬多的很呢。嘴裡罵着。把桌子板凳打得一片聲響。宦蕚聽了。問道。端的爲甚麼緣故。那人嘆恨了一聲。道。小人叫做方器生。這婦人是我的妻子薄氏。成日家橫草怕拈。豎草怕動。只是要好的吃。小人開了個小酒店。蘇碟小飮。就在這巷口。倒好來。每日無移的賺錢數銀子。一日除日用之外。還有多的。每晚有剩下的葷菜拿回來。又帶兩壺酒與他消夜。一句閒話也沒有。小人前因病了。兩個來月就把本錢花用了。如今不做買賣。沒得給他吃。終日這樣吵吵鬧鬧的。剛纔吃飯。他要買些熟肉吃。家中又沒一個錢。連飯碗都摔掉了。罵了這半日還不住。宦蕚道。你這酒店也得多少本錢。方器生道。桌凳壺碗鍋竈器皿傢伙都是舊有的。不過買些雞魚蝦笋香腸肉鮓什件肫肝之類。酒是擡兩罎賣兩罎。四五兩銀子就夠了。宦蕚叫小廝稱了五兩銀子給他。他不敢接。宦蕚笑道。我給你做本錢的。你收了。我還有話說。遂下馬。附着他的耳朶說了幾句。方器生謝了。拿着進去。宦蕚輕輕躡足跟了去。在窗下竊聽。那方器生到了房中。薄氏罵道。倒運鬼。背時鬼。你今日晚上沒有肉與我吃。我明日早起捲捲拍拍屁股。各人尋好漢子去。你不要見怪。方器生把銀子往桌子上一丢。說道。不要罵了。等我明日發市。開了鋪子。寫休書與你另嫁就是了。那薄氏正罵着。一眼見了銀子。一臉的笑。忙跑到跟前。道。好東西呀。你是那裡的。方器生道。你是要去的人了。管我這閒事怎麼。那薄氏笑嘻嘻的道。你有了銀子。大風大雨的。我望那裡去。方器生道。你婦人家好見短。見我沒掙頭。就要嫁漢子去。見了銀子。就不去了。那薄氏笑着道。你道我當眞要去麼。恩恩愛愛的夫妻。往那裡去。不過是激你的意思。不虧我這一激。你肯弄這銀子來麼。不說買些好肴打兩壺好酒來謝謝我。倒還說我的不是。怪不得人說男人沒良心。還是我婦人家的心腸好。哈哈的大笑。方器生又是那生氣。又是那好笑。便道。你吵鬧了這些日子。此時見了銀子。就說這些鬼話。薄氏笑道。你一個頂天立地的漢子。難道自己的賢德妻子那(拿)假話激你。都聽不出來麼。你今後開了鋪子。有得酒肉我吃。看我可做聲。再要吵鬧。就舌頭上長個碗大的疔瘡。你不聽見人說。八十歲的媽媽嫁人家。不圖生長只圖吃麼。況且嫁丈夫圖的是甚麼。原圖上下兩張嘴都有肉吃。又笑個不住。道。不要講閒話。且快拿錢。把銀子買些酒菜來。我替你道喜。那宦蕚忍不住好笑。出來上馬。又走到一條街上。見兩個人廝揪廝扯。打得頭破血出。口中祖宗父母無樣的那硶惡言語都罵了出來。就像有殺人的寃仇一般。要以性命相搏的樣子。宦蕚不知他們有甚麼大仇恨。恐內中傷了一個的性命。忙叫小廝將他二人分開。叫了一個到跟前。問道。你兩個人姓甚麼。有甚麼寃仇。就到這樣死命相打。那人氣狠狠道。我姓任。因家中開個小麪鋪。人都順口叫我做任麪。指着那人道。他姓壽。名字叫做壽新。是我的緊鄰。我兩個自小兒光着頭就相好。還拈過香。磕過頭。拜過弟兄。對天發誓。願同生同死。有官同做。有福同享。做了這些年的好朋友。連臉也不曾紅過。我家賣八鮮麪。鱔魚麪。那殘湯剩水。他也不知擾過我幾千次了。今日同他出來閒走走。前面人走腰裡掉下一百文錢來。我先看見。就拾了起來。他說無義之財應該均分。我不分給他。他就揪着我打。要同我拚命。老爺請評評看誰的是。誰的不是。宦蕚先當有多大的事。聽說只爲一百文錢。笑了笑。叫過壽新來。道。你們旣是好朋友。這一百文錢能値幾何。就到這樣地位。他雖刻嗇。你也太覺小器。壽新道。老爺好輕巧話。一百文錢我應得五十。紅糙米買得二三升。夠家中一日過活。他憑着甚麼理該一個人獨吞。他說我擾過他幾千回殘湯剩水。我家賣熟牛肉。那剩下的骨頭骨腦。他也不知擾過我多少擔數了。這沒良心的想吃獨食。叫他一家子吃了打脊梁上過。我同他兌掉了這命纔罷。我也認不得這樣的朋友了。宦蕚道。你們不過是酒肉相交。原算不得朋友。事體不大。我替你兩個解了仇恨罷。叫小廝取出一百文錢來。遞與壽新。道。你兩不必再講。各自去罷。壽新接錢在手。滿臉是笑。道。倒多謝老爺了。向任麪道。我們多年好朋友。不要爲這點子事薄了面皮。這位老爺給了我一百文。你也是一百文。我兩個打個平火。和好了罷。不要給人看着我們爲這小事。薄囂囂的笑話。任麪笑道。老弟。你說的是。好朋友到底是好朋友。打鬧的是甚麼。兩個人摟肩搭脖。嘻笑而去。因這兩個人面獸心的人。有一調駐雲飛感嘆世間的朋友。道。

\begin{quotation}

朋友交情。道義當年尚有人。近日相親敬。勢利胡廝混。哎一遇事來臨。相推不認。腹笑心誹。反面無情甚。看而今。友道場中沒一人。

\end{quotation}

宦蕚見他二人去了。又是好笑。又是可嘆。打馬正走。見一個襤褸不堪的人。拉住一個體面騎馬的道。我沒吃沒穿。你可憐見我。多少幫補我些。不但是你的原(厚)情。也只當積陰隲。那人馬上道。你快放手。不要胡纏。我要不看情面。打你一頓好鞭子。那窮人拉着不放。哀求道。你不看我。也想想我去世的老爹情面。你忍心看着我餓死了麼。那騎馬的道。你餓死了。干我屁事。我各人有事。還不放手。揚起鞭子來要打。這窮人只得放手。他打馬而去。這人跌足切齒道。天地間有這樣沒良心的人。求老天看着他罷了。宦蕚看見必有緣故。叫他到跟前。問他詳細。這人滴淚道。我姓穆名鼐。也是世家子弟。因無營運。坐食山崩。一貧至此。方纔這騎馬的姓吳名天良。他祖父在我家當了幾輩子家奴。先父在日。念他十數年的勤勞。就把一家白放了出去爲民。他原是鳳陽府人。就回他故鄕去了。不知幾時他發了財。在鳳陽總督標下鑽謀了一員承差官。不知有甚事。差了到這裡來。我今日遇見他。求他資助些須。他不但一文捨不得。反使勢要打我。老爺你說。世上可有這樣無良心天理的人麼。宦蕚聽了。寔(十)分恨怒。見他貧寒可憐。叫小廝稱了五兩銀子給他。他再三稱謝而去。宦蕚一面走着。不勝長嘆道。都不過爲些銀錢。父子夫妻弟兄朋友主僕皆不相認。世風至此。眞堪墮淚。一路嘆息而回。又一日。他到了一家門首。舉目一看。眞是桑戶繩樞。茅簷草舍。蕭條景狀。鄙不堪言。聽得裡面一個女孩子聲氣。哭得十分哀慟。又不好進內去問。勒馬等了一會。只見兩個人打裡面出來。嘆氣連聲道。可憐。可憐。看這個樣子。眞乃傷心。說不得我們行個好。弄碗飯給他度着命。宦蕚忙下馬問道。是甚麼事。可對我說說。那二人看了他一看。答道。這家一個寡婦姓毋。他男人叫做終聲。早歿了。他從小守着一個兒子一個女孩兒。不肯改嫁。今年兒子十八歲了。女兒是十六。這幾年靠着兒子賣焠燈。他娘女兩個在家做針指度日。這毋寡婦已死了五六日了。家中一個錢也沒有。棺材也買不起。他有個小叔在鄕裡雇與人家做長工。他兒子終小大去尋他叔叔來弄棺材。去了這幾日。還不見來。就來了。還不知可有本事弄口棺材來不能。這婦人孤苦伶仃守了這十來年的寡。死了連棺材也沒有。現天現地的撂着。豈不可慘。幸虧天氣涼。若是夏天怎處。他家這個女兒。日夜守着娘屍哭。家中一顆米也無有。我二人是他左右緊鄰。纔來看看。商意(議)弄碗飯度他的命。故此說傷心。宦蕚聽了。甚覺慘然。道。你二位同我進去看看。二人同他入內中。見死屍放在門板上。那個女子坐在地下哭娘。宦蕚道。小大姐。不要哭了。你起來。聽我說話。那女子也就住了哭聲。站起來。宦蕚叫小廝稱了十五兩銀子。對他道。你不必傷心了。這銀子與你。就煩這二位替你母親買口棺材裝殮了。等你哥哥回來。就擡去埋了罷。多的銀子。你兄妹兩個做件衣服穿。買些柴米度日。又對那二人道。他母親死了。這個孩子無依無靠。他叔父要來不消說了。倘不來。就煩你二位替他尋個好人家嫁了罷。不然。靠那裡過日子。那一個道。小人賤姓凌。名居美。倒有一個小兒。這個女孩子我素常知道他很好。不出言不出語的。做一手的好針線。只是不敢做這門親。恐他叔叔後來有閒話。〔宦蕚道。〕只問這女孩子情願不情願。他若願意。你只管做了。若他叔叔有後話。我姓宦。你來尋我。我與你做主。他二人方知是宦公子。宦蕚又問那一個道。你貴姓。答道。賤姓梅。名仁。宦蕚道。我做主婚。就煩你做個主媒。那梅仁說。老爺旣有此美意。小人情願做媒。因對那女子道。這是你的造化。遇見了老爺這位大恩人。凌大哥的兒子凌保。是你常見的。你若情願。就過來謝了老爺\footnote{好。這人善於做媒。這女子肯與不肯。如何好答應。叫他拜謝。願與不願便在其中矣。}。那女子也正在無處歸着的時候。今得了婆家嫁丈夫去。有甚麼不願。就過來叩頭。宦蕚道。不消。請起。又對那凌居美道。等他母親棺材一出去。你就接了他去罷。凌老也稱謝了。宦蕚方回去。凌居美去買了棺材來。把那毋寡婦裝殮了。這女子是他的兒媳。自然不同。回去叫了婆子來同他做伴。送茶送飯。好不應心。那凌保也來幫着照看。替他家買柴糴米。燒火挑水。凌居美又忙忙買布替兒子媳婦做衣服被褥。收拾房子床帳。又過了兩日。終小大方回來。說。尋了叔叔幾日。找不着。不知何處去了。問起棺材來歷。凌居美同梅仁把宦蕚事對他說了。那小子正慮妹妹無處倚靠。見有了人家。也甚歡喜。凌居美把銀子遞與他。道。十五兩銀子。除買棺材並換錢買柴米等項。共用三兩五錢。這是十一兩五錢。你可收了。宦大老爺叫剩的與你同妹子做衣裳穿。如今你妹子旣與了我家做媳婦。衣服是樣都是我做。這銀子留着你做本罷。那小子也就接下。次日。雇人將他母親擡了去。與他父親合葬了。凌居美煩了梅仁的娘子送了衣服來。叫那女孩子洗了個澡。通身換了。接到家中。與兒子成了親。第二日。凌居美帶着兒子凌保同終小大到了宦蕚家叩謝了。再說那宦蕚捨了棺材銀子。這日到了家中。在侯氏房內。小娥也同坐在一處閒話。宦蕚喟然嘆道。如今的人。不但鰥寡孤獨無衣食的甚多。死了沒棺材的也不計其數。我遇着的就施捨了。我遇不着的却怎麼樣。我想了一個道理。我旣行好事。不如開個大棺材店。專捨棺材。各處貼了報子。但是沒有力量買棺材的人家。就來擡去。這豈不妙。小娥道。老爺安心做好事。可行的也甚多。不止這一件。宦蕚〔道。〕我一時想不起。有見不到處。你有何高見。只管說來。小娥道。譬如捨棺材的這件事。人旣連棺材買不起。定是窮到極處了。雖然捨給他一口棺材。擡錢又出在那裡。何不每捨一口材。再與他一兩銀子做擡錢並埋葬工價。再者。人家有祖墳地的不消說。擡去埋葬了。或沒有墳的。或是外鄕來的人。又叫他何處去尋地。老爺再買幾塊義塚地。有沒地者。願葬只管來葬。不願的也不強他。這豈不是一個陰功做到底。宦蕚大喜道。想得好。就是這樣做。他又道。這是爲了死的。旣做好事。要一視同仁。生的也要爲。如今人窮財盡的時候。貧人很多。無歸的人也不少。何不再蓋一所大養濟院。凡是無倚靠的人。或年老無子。或疲癃病廢者。都養活着他。終年給以衣食。這可不是養老了。如今人爲窮了抛下小男碎女的甚多。再蓋一所育嬰堂。雇些有乳的婦人。收留人家抛棄的嬰兒。養大了。有沒兒女的人要去養活。就與他領去。這不是慈幼了。這兩件陰功莫大。還有一種病人。困窮了沒錢吃藥捱死了的也不少。再開一座大藥鋪。修合各種應病的丸藥。施濟貧民。也算得一件好事。宦蕚道。你是讀書大通人。見得到\footnote{雖帶三分奉承。却是自己覺得不甚通。自愧不如語。}。再想還有甚好事說來。我一並奉行。你也有一半功德。小娥道。這是我成全老爺做個全美好人。我有甚麼功德。要說好事可做的甚多。也說不盡。只在性長。遇着就做。力行不倦方妙。若半途而廢。就把前功盡棄了。即如修橋補路。冬夏捨茶湯捨衣服。那一件不是事。強如齋僧敬道。做那無益的事萬倍。還有一個濟貧的法子。叫做不費之惠。拿十萬金開一座當鋪。多的不當。富的不當。專當與窮苦百姓。成兩的就不當。只當三錢五錢的。只要一分利息。夠房租工銀那就罷了。雖不賺錢。却不得折本。窮人却沾了多少恩惠。還有一件要緊的事。如今討飯吃的先生甚多。只認得一本百家姓。公然就去敎學。偏有這些瞎東家。只圖省束脩。也不管好歹。就送子弟去讀書。白花費了多少錢。念上幾年書。連一個字還不認得。我聽得說有一個姓張的。名字叫做東旭。是人家的一個逃奴。他領着一個兒子。無可餬口。到了一個村中。誇他大通。會敎學。拿班做勢。裝出那假斯文的樣子。那村中有個姓馬的。就做領袖。替他糾合了一二十個學生念起書來。這姓張的雖認得幾個字。却不多。敎得別字連篇。可憐一村的人竟沒一個知道。有一讀書人在那村中過。在他學房中歇脚。聽他敎一個學生的書。道。伯牛有疾。子問之。自庸執其手。又敎一個。在下位。不拔上。這人大笑而出。遂替他哄傳。稱他爲拔上先生。牖字認不得還罷了。連援字都認不得。就公然去敎學。豈不可笑。他這樣不通。敎了幾年。竟還發了財。眞是異事。老爺如今開幾個義學。延請先生宿儒。設帳一年。厚資館穀。人家的子弟不計脩金厚薄。即窮無力者。只管來念。雖不能保得個個做秀才中舉中進士。再沒有個一字不識的。成就人家多少子弟。這件陰功却也不小。雖然使這些混帳不通的先生討吃無路。原是他自己作孽。也怨人不得。況他不知坑了人家多少兒子。就餓死了他。天理當然。也不爲罪\footnote{何不叫此等先生也來入學讀書。}。宦蕚此時一心要行好事。二來又是新來的次夫人善意。二善相合。他就力行起來。騰了幾間閒房子。接了向惟仁一家過來。請他掌管當鋪。兌出十萬金來做本。一分行息。專當與窮民小戶。每年送他勞金二百四十兩。又叫了鄔合來監管養濟院。育嬰堂。棺材店。義塚地。各處事務。支放銀錢。給散糧米。一年也與他一百二十金酬勞。又開了七八處義學。煩梅生請了幾位老成在庠的通儒。平儒也在其內。每位一年脩金五十兩。撥人承應。一日三餐上好供給。敎訓生徒。招攬有志上進者來念書。他又買了千畝良田。將族中這些窮戶。凡係同祖傳下者。不論親疏遠近。一年按人口大小給以衣食。有力者不在其內。又置了五千金佃房討租。爲這些人婚嫁死葬之費。就選了兩位年高族長。一正一副。掌管出入。他把諸事都安排停妥了。自己還在外邊尋着好事做。勇猛力行。全無倦怠吝惜之心。一日淸早。到了上元縣衙門口。見有帶枷者數十人。繩拴者約有百餘人。內中還有婦人。都有差役帶着。宦蕚不知是甚麼緣故。心中動疑。上前問那些差役道。這都是些甚麼人。爲了甚麼事。差人認得是宦蕚。忙上前答道。這是本縣管下各鄕各啚的排年里長。拖欠錢糧。拿來追比的。宦蕚道。爲何有枷的。又有拴的。差人道。枷的是早拿來的。今日到限。帶來打比較。拴的是新纔拿到的。見了本官。少不得都要枷責。宦蕚道。他們這幾個窮百姓。能欠多少錢糧。就這樣的枷打。差人道。欠戶多得很呢。萬人還不止。拿不得這許多。這都是爲頭的。追比着他們。好叫他催徵。宦蕚又道。一戶也該多少。差人道。這個不等。也有欠幾錢的。還有欠幾分的。成兩的少。雖沒有甚麼多欠。總起來銀數就多了。宦蕚道。他們欠的旣不多。何不完了。了却一件事。差人道。人戶多了。這都是那窮苦極了的百姓。無衣無食。要一個錢也是艱難的。如何得能夠完官。宦蕚道。怎麼又有婦人。差人道。他丈夫躱得沒影。小人們空回要受責罰的。不得已纔帶了婦人來抵搪繳批。宦蕚聽了這番話。又看見這些貧民形狀。甚是不忍。激出一腔義氣來。道。甚麼話。爲民父母。不能體恤民情。這樣的窮百姓。還拿來胡敲亂打\footnote{這却是呆公子。不知做官的苦。}。一個良善好民。又不曾做強盜。做窩主。爲何拿人婦女\footnote{余謂話雖是呆公子。心却是大菩薩。}。都替我放了。我替他衆人一力全完。衆差人不敢一(不)依。都把項上的繩子解了。衆人聽見說他一力代還。跪在地下。響頭磕得震耳。那些帶枷的也兩手扶着枷叩首。宦蕚道。你們起來。我會了知縣放你們。衆人歡呼踴躍。一個個歡歡喜喜。不像先那樣愁眉苦臉的了。宦蕚催馬到衙門口。道。進去對你們本官說我來會他。那陰陽生往裡飛跑。頃刻。儀門大開。陰陽生回道。請老爺馬進去。宦蕚昻然直入。進了儀門。見知縣在甬道旁拱候。原來這知縣的祖與宦實是會榜同年。他還算宦蕚的年姪。宦蕚忙下了馬。他讓進後堂坐下。門子送上茶來。吃罷接去。知縣見宦蕚滿臉怒容。〔道。〕老年叔尊面爲何有不豫之色。宦蕚道。我纔在衙門外。見許多窮百姓。一個個披枷帶鎖。問起來。說是拖欠錢糧的甚麼排年里長\footnote{這的的確確是公子話。他不知排年里長是何物。}。衆人該錢。拿着他們枷打。也忍心麼。況且說這些欠戶。連衣食都沒有。爲民父母的。還該可憐他纔是。就是這些排年里長。也未必都是有錢的人。別人沒得與他。他未必能夠代還。就打殺了他也沒用。這不是屈棒打平民麼。那知縣通紅了臉。滿面愧容。道。老年叔見敎得極是。小姪也是無其(可)奈何。目今軍需緊急。一時應付不到。上臺就要參處。在他衆人還易於爲力。不得不加箠楚。小姪不但沒有這些銀子替他們代償。況從來可有家中馱了銀子來做官的呢。旣從事簿書。自己的功名要緊。仁慈惻隱四個字就提不起了\footnote{有命的話。}。宦蕚道。這些男人還罷了。怎連人家的婦女都拿了來。知縣道。這却小姪不知。回顧傍邊吏胥。一個稟道。因他男人逃避。故將家屬拿來。知縣怒道。本縣不曾吩咐。如何擅拿人婦女。少刻到堂上重責。宦蕚道。也不必責罰他們了。方纔鎖着的人。我叫都放了。可把那些枷着的都釋放了。我亦許了衆人。替他們代還。可算起了共欠多少。叫人跟我去取。知縣道。老年叔凡事要三思。雖然是老叔一片熱心。但他們欠的多着呢。恐還不得這許多。宦蕚道。大丈夫一言旣出。駟馬難追。我旣許了他們。他們頭都磕了。豈有反悔的理。只將正數查淸。不要加火耗就是〔你〕的盛情了。任憑多少。我都力償。知縣喜得滿臉堆着笑容。說道。老年叔這一番菩薩心腸。小姪爲民父母者已不勝愧殺。再想圖火耗。眞狗彘不如了。老年叔這一場義舉。免了貧民多少比較。陰功無量了。吩咐六房書吏相幫去算。又命將衆人的枷都開了。知縣讓宦蕚到書房中吃了便飯。等到將午。戶房來稟。通細算淸。共欠一萬七千有零。宦蕚道。甚麼零不零。叫人跟我去取一萬七千兩來就是了\footnote{連知縣的考成俱完全了。大有行取之望。}。知縣道。正是。大數足了足矣。些微零頭。那就易於開銷了。宦蕚道。我替他們還了銀子。你給他們個執照。不要把我的這項錢弄在夾曾(層)裡去。知縣道。豈有此理。少不得都給衆人紅票去。小姪還各鄕各啚出示曉諭。使衆百姓知道老年叔這番恩德。宦蕚起身。知縣送到丹墀中。讓宦蕚乘馬而去。到了大門外。衆百姓果然枷都開了。又跪下叩謝。宦蕚道。你們共欠一萬七千兩。我都替你們還了。方纔知縣說給你們紅票做執照。你們領了。都回家去罷。衆人又歡呼拜謝。宦蕚同着一個戶房。知縣的兩個管家。還有二十多個衙役。拿着籮筐扁擔到了家內。上去將前話稟知宦實。宦實極力贊美。宦蕚在箱中搬出三百四十封銀子。叫家人運到廳上。查點明白。交付縣中衆人而去。他回到房中。向侯氏小娥說。都不勝欣喜。誇不絕口。次日淸早。聽得大門外人聲鼎沸。家人忙進來回道。有幾百男子婦人。手拿着香在外叩謝。宦蕚出到門外。衆人見了跪下。齊呼道。蒙老爺天恩。救了我們窮苦百姓。少捱了多少棍棒。願老爺壽高百歲。子子孫孫代代八座。罷了。宦蕚喜笑道。你們請起。我請太老爺來看看。這是他老人家的恩典\footnote{功歸於父。好。}。宦蕚忙進去請了父親出來。衆人看見。又都跪下叩謝。宦實大喜。命每人賞錢一百文。衆人口中宣揚着佛號。高呼大慈大悲救苦救難的宦菩薩。鼓舞而去。少頃。知縣親來拜謝年伯祖同年叔。待茶而去。第二日。宦蕚飯罷出門。方到門外街上。跪倒百餘人。也是荷枷帶鎖。大叫道。求老爺天恩。一體救拔小民罷。宦蕚問甚麼人。原來是江寧縣排年里長。聽見宦蕚救了上元縣的欠戶。故此都來乞恩。宦蕚道。你們都起來。等着我回了太爺。帶你們同去。復翻身進來。下馬到內邊。向父親說了。宦實道。同一窮民。何分厚薄。該多少。你也替他們還了罷。宦蕚領了父命。笑吟吟出來。跨上馬到外邊。招呼衆人同到江寧縣來。這知縣昨日聽得上元縣的欠戶宦公子替還了。將二萬金舊欠全完。嘆道。寅翁好造化。遇這位積福的善人。省了多少心力。脫了多少干係。考成十分全完。榮陞在即。偏我就遇不着。正想時。忽報宦公子領了本縣這些排年里長來了。知縣喜得屁滾尿流。嘴中忙叫道。快請。快請。如飛的到儀門外接着。讓到迎賓館坐下。叩其來意。宦蕚把〔替〕衆人還欠項的事說了。那知縣笑容可掬。左一恭。右一恭。贊了又贊。謝了又謝。多時算淸。共少一萬二千有餘。江寧縣的百姓比上元縣略富庶些。宦蕚也如數還了。衆百姓也焚香叩謝。這上江兩縣數萬欠戶。自從宦公子替他們還了這宗拖欠。免得提心吊膽。如釋重負。男婦大小無不感念。望空叩頭保佑的也不計其數。眞是家誦戶祝。凡相遇着。提起一個宦字。就感恩誦德不已。這宦公子的美名。却也就幾幾乎傳遍闔京了。話不繁言。宦蕚一日高興。到城北一帶走走。人烟稀少。盡是園圃。見一座墳墓傍邊有三間小房。一個獨院。左右無一居鄰。聽得內中一個婦人聲音。喊叫救人。宦蕚心驚。道。此處荒僻。莫非有人做甚不公不法的事麼。忙跳下馬來。進入院中。大喝道。房中甚麼人喊叫。只聽得喊着道。是那一位。快些進來救救人。宦蕚忙叫了一個小廝同到房中。見一個少年婦人吊在梁上。一個老婦抱着兩腿。往上搊着。見了宦蕚。叫道。老爺積陰功。幫着救一救。宦蕚叫小廝相幫搊住。問道。你家有刀沒有。老婦道。那桌子上有把剪子。宦蕚拿了過來。把繩子剪斷。同着將那婦人擡放在床上。替他捏着喉嗓。叫那老婦道。你摸摸他的心口可還熱。那老婦摸了摸。道。還熱呢。宦蕚道。不妨。你快去燒些熱水來。那婆子去了。宦蕚此時也顧不得嫌疑。將那婦人抱在懷中。抹胸度氣。不一會。喉中漸有聲響。纔把繩子解去。那婆子也拿了水來。忙灌了幾口。那婦人嘔出一口痰涎。纔透過氣來。就哽哽咽咽的哭。宦蕚見他已救活。心纔放下。叫那老婆子扶他坐着。然後下床來。坐在凳子上。將這婦人一看\footnote{這一句便寫出菩薩心腸。聖賢肝膽。先只忙忙以救命爲事。並不看其奸(妍)媸。此時見救活了。方纔一看。}。有二十一二年紀。生得十分美艷。一身雖都是紬絹衣服。却補補衲衲。舊而且破。不堪之甚。有一調秦樓月說他道。

\begin{quotation}

香馥馥。眼中一個人如玉。人如玉。荆釵裙弊。苦寒裝束。嬌羞緊把眉兒蹙。千般隱恨縈心曲。縈心曲。滿肚愁腸。淚痕盈目。

\end{quotation}

看他房中雖然都是破爛之物。却是個舊家光景。知是大家子孫敗落下來的。宦蕚道。府上貴姓。尊夫在那裡。有甚麼傷心的事。如此靑年。爲何就尋這個短見。婦人見問。越發哭得傷心。宦蕚道。不必悲傷了。有甚麼話。可吿訴我。我或者出得些力。也不可知。那老婦道。這位老爺是你救命的恩人。奶奶你有苦楚。何妨說說。到了這個田地。你還瞞甚麼。那婦人纔要說。看見宦蕚的小廝在傍。欲言又止。宦蕚會意。叫小廝道。你到外邊去。小廝出去了。那婦人一面流着淚。一面說道。我家公公姓牧。名字叫做牧德厚。婆婆聶氏\footnote{是極。不是作了孽。如何沒得後。生下這等好賭下流的兒子來。}。公公在廣東瓊州府做過一任知府。掙有十數萬金\footnote{廣東謂廣州府爲睡十萬。瓊州府爲坐十萬。潮州府爲跑十萬。瓊州知府雖掙餘十萬。禁不得兒子一賭。奈何。}。只生我丈夫一個。名字叫做牧福\footnote{沒福之人。雖留下百萬。又奚益哉。}。從小不知管敎。任他胡做非爲。我爹爹姓屈。叫做屈攀桂。母親仰氏。我因是我爹爹得官那年生的。叫做紳姐\footnote{造化。虧這個小名好。}。我爹爹就做瓊山縣知縣\footnote{公公做窮知府。老子又做窮知縣。宜乎兒女受窮。}。是他的屬官。因仰攀他家的富貴。把我嫁與他家做媳婦。不幸公婆染了瘴癘。一齊病故在任上。我隨了丈夫扶柩到這裡來。只三四年間。把銀子紬緞。金銀器皿。首飾衣服。並房產地土。一色等項。賭輸了個乾乾淨淨。家人賣的賣了。走的走了。指着那老婦道。只剩下這老兩口。賣是沒有人要。他是公婆手裡舊人。也可憐見。他們所以捱死捱活的跟着。連房子也沒得住。搬到這墳上來住。如今吃的也沒有。穿的也沒有。他還只是賭個不住。當日有錢。還同的是體面些的人賭。如今窮了。那略像樣些的人都不同他賭了。就同那些光棍屎皮辣子不堪的下流人賭。該了七八個人的銀子。成日上門上戶的打鬧。時常被人村辱不堪。他一些也不知羞愧。新近又輸了一個甚麼刁公子的五六十兩銀子。每日叫小廝們上門來打罵。這個壞良心天殺的。不知幾時看見了我。說到這句。臉就緋紅。大哭起來。宦蕚道。不必哭。有話說完了。有甚麼事。我替你做主。那屈氏道。刁家那斫頭的起了一片壞心。他對我丈夫說。叫我同他做那不長進的事。若依了他。還叫我那不成人的丈夫寫張典我的文書與他。不但他的幾十兩銀子不要。該衆光棍的銀子他都替還。我男人先還不肯。這姓刁的串通了這些光棍。終日打罵。在街上把他凌辱不過。我男人急了。竟應允了他。許他明日來。他替還了衆人的銀子。我就算他的人了。叫我陪他睡。今日來對我說。我也是好人家的兒女。怎肯幹這樣醜事。所以纔尋自盡。不想老爺又把我救活了。我早晚是必死的。辜負老爺這片好心。說完。放聲大哭。宦蕚大怒道。刁家這奴才。我素常知道他的名字叫做刁桓。一個麻臉。幾根黃鬍子。混名叫羊肚石。這奴才萬惡萬刁。他老子做着個千戶。多大個官兒。他公然在外邊做這些惡事。誘人家賭博。又想騙人家妻子。這奴才同一個慣開賭場的姓屠的勾連。坑了人家多少子弟。你放心。我替你報這個仇。我明日如此如此設法救你。屈氏忙忙下床來拜謝。宦蕚道。不消。不消。你丈夫在那裡。屈氏道。他怕有人來打鬧。躱在一個小庵裡。離這裡有一里多路。〔宦蕚道。〕我有一句話。你不要惱。屈氏道。老爺有話。只管請說。宦蕚道。如今把你們這場事弄淸了。設或你丈夫又輸了別人的。把你又要典與人。我如何得知。又怎麼來救你。除非叫你丈夫把你典了與我。我替你做了主。他纔不敢又生他想\footnote{看至此。未有不疑宦蕚心愛此婦。故以恩結之。看到後來。竟大謬不然。愈見其聖賢心腸。豪傑氣象。作用眞不凡。}。你心裡酌量。可行得麼。屈氏想了一想。道。罷。老爺救了我一命。再替我出了這口氣。我應該報答的。強如捨身子與那樣奴才。宦蕚道。須得把你丈夫尋來。當面說明方可。屈氏道。家中沒人去尋他。怎麼處。宦蕚指着老婆子道。他的老頭子呢。屈氏道。他雖六十多歲。因見家中沒得吃。每日早起。雇與人家做小工。掙三分銀子。買升米買個柴來家度命。宦蕚道。他不在家。怎麼樣呢。那老婦道。我認得。等我去尋。宦蕚道。你尋着了。把我先說的話不要吿訴他。看走了風。衆人知道了。那老婦道。我知道。忙忙的去了。宦蕚問屈氏道。你家柴米。這個老兒去掙了。家中日用油鹽菜蔬並冬夏的衣服。這些零碎盤纏出在那裡。屈氏見問這話。紛紛落淚。道。可憐一碗飯還不得飽吃。還說甚麼菜。幾個鹽花就〔是〕下飯的菜了。成個月連油星兒也不見。燈是久不點的。有月的日子多坐一會。無月之日早早便去睡了。至於衣裳。好的准了賭賬。與人去了。賣也賣了些。有不値錢略像樣些的。都當了日用。剩下破爛的。當賣不得。拼拼補補。遮體罷了。宦蕚道。你身上這件衫子好像百家衣。太難爲情。把你當票拿來我看。屈氏在一個舊拜匣裡\footnote{舊拜匣。妙。好的賣是賣掉了。}。拿出一包票子來。約有百十張。宦蕚道。你可認得票子上這種字是些甚麼東西。逐張念與我聽。屈氏道。我都有字記在後邊呢。原來這屈氏寫得一筆好字\footnote{此寫屈氏認得字。非誇其聰明。江南當票上別有一種字。不然。宦蕚旣認不得。屈氏又記不得許多。將奈何。故說他認字。便益於查耳。}。他遂一張一張的都念與宦蕚聽。宦蕚把他穿得着的衣服。並幾件丁香簪棒被褥之類。都把票子接過來。別的仍叫他收起。將這些票子本利一算。該二十多兩。宦蕚道。我若把銀子與你。怕你丈夫又拿了去賭。我替你贖了來罷。你家這個老頭子。明日以後不必打發出去了。留着家中使喚。你家柴米我都送來。屈氏嘆道。我們有甚麼補報老爺的。老爺這樣的恩情到我。宦蕚道。我憐你是宦門之女。嫁了這樣不成器的丈夫。故動了一點慈心。豈望你報。正說着。那老婦同牧福來了。老婦路上已將屈氏上吊。虧這人救活。並將要典他的話。對他說了。他一進門。就與宦蕚深深打恭道謝。宦蕚看他有二十四五年紀。好一個齊整少年。也穿得襤褸不堪。暗嘆道。可惜這樣個人品。却做這樣的下流事。那牧福問道。請敎老爺貴姓。宦蕚道。我賤姓宦。牧福又深深一揖。道。原來是宦老爺。晚生何幸得遇。只見屈氏柳眉剔豎。杏眼圓睜。粉面通紅。向着牧福道。我已是吊死了。蒙宦老爺救活了我的命。如今許替你應那姓刁的同衆光棍的賭賬。你早想要把我典與那刁姓的。你如今寫文書。就典與宦老爺。那牧福低着頭。紅着臉。不做聲\footnote{此所謂無羞之心非人也。人雖下流。此心幸未喪盡。故後尚能自新。}。宦蕚道。這憑你願與不願。也不強你。屈氏又道。你把〔我〕典與老爺就罷。若典與姓刁的。我叫你人財兩空。牧福道。你不用着急。旣蒙老爺救了你。又肯替應欠賬。自然該的。還有何說。就取了紙筆。親筆寫了一張將妻典銀的文書。夫妻同畫了字。遞與宦蕚\footnote{充古好因男色而棄妻。牧福因好賭而典妻。勿謂作書者過言。余親見江寧有一妓曰卓二官。係揚州人。厥夫酷好嫖而無資。因命妻接客。得他人之嫖金。以作己之嫖資。不知此輩人心腸是何生法。}。宦蕚道。明日他們說多昝來。牧福道。說是早飯後來。宦蕚道。等他們來。你留他們坐着。我自有道理。說了。就吿別上馬而回。到了家。叫小廝送了一擔米兩挑柴一千錢到牧家去。他然後到府尹衙門來。會見了樂公。樂公一見便道。年兄前日替兩縣窮民代償拖欠。這一番義舉。不但萬民銜恩。就是兩縣也受德不淺。誠所謂惟大英雄餘本色了。我學生不勝敬仰。宦蕚道。這是家父憐念小民的一點慈心。晚生遵而行之。何敢當老先生過譽。樂公詢其來意。宦蕚便說。有一牧舍親。他令先尊曾莅任太守。他年幼無知。被衆光棍誘賭。將家私輸盡。並惡棍刁桓夥同賭局屠四。勾他輸了銀子。希圖奸騙他妻子的話說了。道。求老先生重究。以警刁頑之輩。牧舍親一家生死皆銜恩德矣。樂公生平極恨的是賭博。又是個嫉惡如仇的人。聽說刁桓的這些壞處。勃然大怒。命傳番役到了面前跪下。吩咐道。你們衆人明早同宦老爺的管家。去將那些賭博光棍全拿來。若走一個。重處不貸。再將開賭場姓屠的。一幷拿到。衆人應諾下來。宦蕚也就辭了回家。叫衆番役到他家中。道。明日你們去拿人。那姓刁的並衆光棍身邊都帶着銀子。你們只管搜了去用。拿到衙門動刑時。加力打那廝。我過後知謝你們。叫家人待他衆人酒飯吃了去了。次早。衆番役約了宦家小廝領路。同去拿人。再說那刁桓他常來牧家走動。久矣看上了屈氏。不想牧福剛剛輸了他銀子。他是光棍中的魁首。遂約了衆人。終日在他家打鬧。料道牧福不得不走這條路。今見牧福把屈氏典與他。滿心歡喜。他預先都與衆光棍說明。牧家那裡來的銀子他都代還一半。向着牧福只說全還。衆人見牧福窮到這個地位。這項銀子也有八分置於度外的了。今得一半。還有何說。遂一同八九個人說說笑笑而來。好生得意。那刁桓滿心今日要與屈氏做新相知。穿了一身新衣。搖搖擺擺。都到了牧家。方纔坐下。那知這些番子們在左近四散看着。見這一起人進去。知道是了。哨了一聲。同走了進去。不由分說。都套上了鎖。帶到天井中拷吊起來。這些番子都受了宦公子之囑。將衆人先打了個下馬威。然後都在房簷上高高吊起。那衆光棍還受得些苦。這刁桓他是個嬌養子弟。如何奈得。殺豬也似的叫。身邊帶來還人的銀子。盡行奉送。衆光棍身上有帶着賭本的。也都傾囊相贈。方放鬆了。帶到衙門中來。正値午堂。樂公略問了幾句話。每人三十大板。一面大枷。刁桓係爲首光棍。屠四係開賭之人。各加責十板。衆人俱枷號一月。限滿問徒。一個個都打得血肉分飛。帶到通衢示衆。那刁桓他是好人家子弟。只因生性好賭貪淫。遭此羅網。他如何禁得這等重刑。只枷了三五日。就嗚呼哀哉。死於枷眼之內。正是。

\begin{quotation}

未遂奸淫身已喪。因貪賭博命橫亡。

\end{quotation}

且待我把這刁桓的來歷細說一番。他父親是個世襲的衛千戶。家中頗覺富足。一生惟有杯中之物是好。終日沈酣。與麴糵生爲友。他妻子尹氏。亦同此癖。夫妻二人自淸晨起來。每人捏着一個杯。直到臨睡時。方纔放手。他二人在酒字上做了工夫。到色字上毫不介意。因此一生只生刁桓一個。這刁桓生得一臉指頂大黑麻子。自十五六歲上。便長出數撮黃鬚。麻子疤上不長。只在那空〖阝少日小〗處長將出來。揸揸巴巴。長得奇形怪狀。人見他那尊容。取其形似。都稱他爲羊肚石。他自幼貪淫好賭。刁頑之極。他乃尊終日昏昏醉夢間。不但不管敎。而且不知。任他在外胡做非爲。刁千戶有個上司暴指揮。名字叫做暴如雷。也是世襲前程。這職役原是他哥哥長房頂襲。他哥哥艱於得子。後來年老方生一子繼名。叫做觀音保。他哥哥死後。該觀音保承襲。他欲謀此職。買出本族作證。說他哥哥並無子息。這個姪兒是個螟蛉。本姓闕。名映寶。祖宗制例。異姓不許襲替。應該他胞弟承襲。族中人貪他賄賂。都具了甘結。他各衙門都打點了。觀音保幼小。寡母不能與爭。只得讓了他。他自得了官。屬下這些千百戶的便宜。他個個占盡。是不消說。本管的那些窮衛丁。他倣(放)賬盤利。刻薄無比。雖掙了一分好家私。却也無人不唾罵。無人不飮恨。他又性如火烈。鞭撻衛卒。凶暴非常。因而怒氣傷肝。到五十歲外。便成了雙瞽。只得退了前程。在家閒住。他白占了姪兒功名。自己又無子。遠房不准承襲。把一個世代功名白送掉了。他妻子亡故。只留得一女。他要想續絃。人都知他刻薄。且性子起來。專好打老婆。他前妻因此氣死。又瞎了兩個眼睛。誰肯嫁他。只得買了個丫頭在身邊答應。他這女兒生得更是可笑。一個臉歪在一半。因出痘瘡。又壞了一隻眼。那瞎眼要是閉着倒還罷了。他却沒有黑睛。只雪白的一個眼珠子。疊暴在外。如鑲嵌上的一顆大珍珠一般。人聞其形。也贈了他一個美號。稱爲海螺杯。這海螺杯姑娘之名。人人皆知。竟沒人來求親。直捱到靑春將及四八。猶然閨中待字。他忍耐不得。竟自己尋起佳配來。他家有個小廝。是個海南的黑鬼子。雖係異類。因自幼養大。頗通人性。名字就叫小鬼子。海螺杯就看上了他\footnote{同氣相求。海螺杯原也是海裡所出。}。暴指揮家中奴僕因主人暴戾。都逃走乾淨。只剩了老邁兩口不能遠走。在家中以供炊爨。小鬼子是外國人。也還老實。二來他那面貌無處可逃。在家以應灑掃差使之役。暴指揮閉着雙眼。毫無一事。酷聽鼓兒詞。常養着一個姓夏的瞎先兒在家。專一說書。那通房之〈之〉婢。時刻守定瞎主人扶持。寸步不離。海螺杯或在父親房內聽說一回書。倦了到自己房中睡一覺。他先胡胡塗塗。倒也過了。一日晚間在他父親房中。聽說西遊記上陷空山無底洞老鼠精那一段鼓兒詞。忽然引動春心。便十萬個金剛也降伏不住。走向房中去睡。上床脫光。用手摸着牝戶。不住長嘆道。人家女兒像我這樣大。不知生了多少娃娃了。要是十三四歲得了早子。都見孫兒了。我還不曾嘗着人間的滋味。心中着急。將枕頭摟在懷中。亂聳亂拱了一會。越發難過。翻來覆去。一夜無眠。天色纔明。聽得小鬼子在堂屋裡掃地。心中想道。我實在有些過不得了。把這小廝應應急罷。低低叫了兩聲小鬼子來。那小子聽見。推門進來。走到床前。暴氏問道。老爺起來沒有。小鬼子道。還關着門。像是還睡呢。暴氏道。你關了房門來。我叫你做甚麼。那小子關了門。又到床前。暴氏掀開帳子。道。我的小肚子疼。你上床來替我揉揉。那小子上床。蹲在床沿上。暴氏仰臥着。把被掀開。露出一個光肚皮。同胯中那條細縫。叫道。替我揉。那小廝嘻嘻的笑。伸手去摸。直摸到那條縫上。用指頭一勒一擦的動。暴氏笑道。你的可是這樣的。他笑道。我的不是這樣。暴氏道。你也拿給我摸摸。遂伸手到他褲襠裡去。那小子十六七歲了。已知識大開。一個半大陽物也自挺硬。暴氏摸着了這件寶貝。那裡還忍得。指着陰戶向他道。把你的放在這裡頭試了。那小子聽說。喜得忙脫了褲子。就上身來。暴氏用了些唾沫。捏着他龜頭。對了自己的門戶。說道。你往裡送送。那小子往裡一下。進去了大半。你道他一個處子。如何這等容易。一來那小子的陽物不大。二來情急得很了。先被摸勒了一會。也有些津津水出。所以不覺煩難。暴氏雖不見樂趣。也覺內中有些意味。抽了一會。恐他父親起來。叫那小子出去。囑託他每夜等老爺睡了。悄悄到房中來同宿。小鬼子滿口應諾。此後每夜約那小廝來相伴。權且按下。那暴指揮也不知他令愛奇醜。偌大年紀尚無人來求。心中也按(暗)急。他一日衣服上掉了根帶子。叫女子給他釘。海螺杯答道。我年老了。眼睛花。看不見了。暴指揮聽了這話。知是女兒年長無偶。嘆了口氣道。我知道是我的不是。是我的不是。愈加着急。偶然想起刁桓來。他也廿八九。尚未娶妻。因他父母只在酒杯上做工夫。故將兒子的姻事蹉跎下了。今日若將他二人配合。豈不合了兩句俗語道。

\begin{quotation}

破磨對瘸驢。歪鍋配斜竈。

\end{quotation}

眞是一雙兩好。遂叫夏瞎子去探刁千戶的口氣如何。並說自己無兒。將來家私都是女兒女婿的。夏瞎子去探。刁千戶雖知他女兒醜陋。一來是舊上司。扳了親。圖體面。二來貪他的內囊。滿口應允。遂成了這門姻眷。迎娶之日。新人進門。夫妻合巹。彼此一看。眞合了古人的一副絕對。那刁桓恰是。

\begin{quotation}

麻臉黃鬚羊肚石。倒栽蒲葉。

\end{quotation}

那暴氏恰是。

\begin{quotation}

歪腮白眼海螺杯。斜嵌珍珠。

\end{quotation}

兩人一見。各各氣生。你道是何緣故。暴氏素常以爲。他這歪臉暴睛。是千古美人圖上畫不出來的妙容\footnote{二語令人絕倒。然而實在千古美人圖上決無此等妙容。}。眞要算絕代佳人。滿心思想嫁一個賽潘安強衛玠的丈夫。不想今日嫁了這樣個醜驢。較之小鬼子。那不過黑些。論起形容來。刁桓比他尚還不及。如何不氣。那刁桓雖然醜態可憎。他是專在婦人身上用工夫的。瞞着酒鬼老子。偷出銀錢嫖妓女。養私窠。偷野食。這些淫婦人只貪他幾個錢。那管生得醜俊。他閱人甚多。婦人中從未見這樣奇美的怪相了\footnote{語甚新趣。}。這是終身配偶。朝夕相對。如何過得。焉得不氣。兩人各氣在心頭。却發洩不出。晚間上床。刁桓少不得要做些成親的圈套。扯扯拽拽。那暴氏攥住褲腰。死也不放。亂蒯亂抓。刁桓也並非高興。不過是虛應故事。見他如此。也就放手各睡。過了數日。兩人並不交談。那刁千戶夫妻只知吃他的酒。那裡知道兒子媳婦的這些瑣事。一日夜間。刁桓有了幾杯酒。忽然興發。想到。他雖然貌醜。或有件好物。也不可知。況他這樣門扇大的肥身子。其物必肥。且我從來所遇的婦人都是破物。他到底是女兒。自別有妙味。果然有個好美窟。夜間吹了燈看不見。尚可盤桓。有了幾杯酒。忽然高興。再想終日相守。沒有個只有夫妻之名而無男女之實的道理。這一回想。把他的醜忘了一半。就伸手去摸暴氏。那暴氏已是知味的女子。起出(初)嫌丈夫醜陋。各睡了數夜。那心也有些忍不住了。想道。當日同小鬼子私偷。原不像意。我大着他十四五歲。已生得下他來。況他年幼。此物自然渺小。今日他是將三十歲的大漢。必定此物也雄壯。旣明公正氣嫁了丈夫。放着美食在傍不吃。何苦擔飢。只閉着一隻眼。人說眼不見爲淨。憑他去弄去。且快活一時是一時。正然想着。見他來摸。假裝睡熟。等他解開了褲帶。將摸到那要緊的去處。方纔用手來掩。刁桓趁這意兒。褪了他的褲子。一翻上身。還以爲他是處子。拿出憐香惜玉的手段來。用了些唾。輕輕款款。做那蜻蜓點水之勢。不想只略往裡一送。如蛇鑽窟窿一般。一下全身入去。方知這位醜美人。是合了連環記上那鎖南枝曲子上的兩句。道是。

\begin{quotation}

靑靑柳。嬌又柔。一枝已折在他人之手。

\end{quotation}

遂興致索然。連忙拔出睡下。心中氣忿忿的。要聲張起來。不但礙着丈人是父親的舊上司。且又想妻子的東西雖醜而破。他陪嫁的私囊却富而厚。只得忍住。旣好氣又好笑。這樣的婦人還有甚麼人愛。肯同他私偷。眞不可解。那暴氏見丈夫弄了進去。比小鬼子的大有不同。內中塞滿。以爲定有大樂。心中私喜。不意他忽然拔出睡下。知是嫌他不是原封了。大掃高興。那忿恨之心又說不出。次早起來。彼此都是一個惡狠狠的面孔。先前二人只是彼此嫌醜。尚無恨心。今日又加上這一番。怨怒自然越發加倍。不到半月。兩人終日言語相激。竟致反目。初而罵。繼而打。不想那刁桓生得瘦怯。反沒有暴氏壯實有力。被他摔倒。一屁股坐在頭上。拳頭如擂鼓一般。打得刁桓披頭散髮。滿地亂滾。喊叫救命。刁千戶夫妻正在醉鄕。聽見了。吃了驚。跌跌倒倒的跑來拉開了。刁桓賭氣走了出去。竟不回家。暴氏哭了一場。將陪嫁之物一一收起。絲毫不發。過了幾日。刁千戶叫人找了兒子來。勸他進房。兩人相見。怒目而視。不但恨他。前日被他打寒了。竟有幾分懼怯。晚間雖也同床。却兩頭各被而睡。此後刁桓終日在外。或在賭場。或在妓館。常不在家。手內無錢。到家中要尋些須。爲嫖賭之資。暴氏也知他在外走這狹邪道路。便罵道。都是我家賠來的東西。倒不得你拿去嫖賭。刁桓見他識破機關。東西又藏得沒影。只好等父母醉臥。偷些私蓄出去行樂。滿月後。暴氏回家去住對月。他熬了這一個月了。還拿小鬼子來解渴。住了些時回來。仍舊斷了葷味。心中說不出的苦。一日。夏瞎子來看姑奶奶。暴氏想道。這瞎子雖沒眼睛。㞠子是有的。何不在他身上尋一番樂境。主意定了。留他說書。到晚不放他回家。這晚刁桓恰好未回。刁千戶一則醉生夢死。不知防閒。二則知是親家翁家中的長遠主雇。媳婦留他說書。有何不可。便叫在堂屋裡鋪了個鋪給他睡。到夜間人靜。暴氏悄悄到外間瞎子的榻上去就敎。那瞎子旣看不見他的好醜。且又是三十多歲無妻的一條壯漢。婦人的這件美物。是他求之不得的寶貝。可有推辭之理。公然鸞顚鳳倒起來。不意那瞎子竟有一具壯觀的陽物。暴氏喜出望外。再三叮囑。夜間要常留他不便。恐公婆疑心。姑爺是日日不在家的。你不妨日間源源而來。公婆只知痛飮。不管閒事。家下沒有多人。遇便即可行樂。夏瞎子一面笑着。一面不住聲答應。果然那夏瞎子竟不爽約。過兩三日就來走走。暴氏見沒人。掩上門。到床上就做一番。如此多次。一日。二人正在綢繆之際。忽然刁桓回家。推門進來。一眼見了。大罵道。沒廉恥的淫婦。你在家做女兒偷漢子。到我家來還偷。我同你了不得。我前日就疑心甚麼瞎眼的人愛你。同你偷。原來就是這瞎奴才\footnote{寃哉。寃哉。眞是寃殺傍人。笑殺鬼子。}。又罵瞎子道。你這瞎奴才。敢膽大做這樣的事。我把你送到官去講。夏瞎子正同暴氏做得好。將入佳境。忽聽得刁桓聲音。唬得一翻身滾下床來。光着屁股滿地亂爬\footnote{亂爬。妙。旣唬癱了。又看不見。}。又被刁桓在光屁股上踢了兩脚。又不敢叫。就地亂滾。暴氏雖是個淫醜的惡婦。今做這勾當。被丈夫撞見。不但自己覺愧。心中也有些膽怯。遂急出一個主意來。一骨碌爬起。說道。你不稀罕我。難道叫我守一世活寡不成。你在外頭嫖得。我在家裡也嫖得。我同你好講。你若聽我。以後我的東西任你拿去嫖賭\footnote{錐心入耳之言。刁桓那得不聽。不意此婦有此急智。}。我也不管你。你也別管我。各人幹各人的事。要是這樣便罷。不然。要死要活我同你做。我不怕你這樣子。我也不願活在這裡呢。刁桓心中本有幾分怯他。所以先見時不敢上前去打。聽得他這番話。倒心中情願。暗喜借此挾制着他。不愁嫖賭之費。說道。罷了。罷了。從今後。你是你。我是我。說了這一句。反走出去。暴氏見他去了。餘興未已。下床拴好了門\footnote{太小心。}。扶起了瞎子來。還要他終局。誰知那瞎子被這一嚇。把個陽物縮得只剩些軟皮\footnote{掃興。}。暴氏與他再三摶弄不起。只得放他回去。這日。刁千戶夫妻飮得醄然大醉而臥。兒子媳婦這一番大鬧。他竟不知。次日。暴氏見刁桓進來向他要私房。因要他賣路\footnote{這才眞是買路錢。}。放瞎子往來。只得給刁桓些私蓄。刁桓自此因手頭充闊。越發在外日夜嫖賭。他在屠四家與牧福相識久了。一日去尋他。無心中見了屈氏。眠思夢想。要算計他。因想出這個惡主意。勾了牧福。贏了他這項銀子。諒他沒得還。不怕不走這條路。拿妻子做當。孰知天道難欺。剛剛遇了宦蕚。他投入法網。送了性命。刁千戶見兒子死了。媳婦無出。送回暴家。任他改嫁。暴氏回到家中。不想嫁人。只同夏瞎子小鬼子二人輪流作樂。後來夏瞎子同衆夥計飮酒。多了幾杯。偶然失口。說出這段佳話。內中有個古瞎子。一個眞瞎子。留了心。次日公分請他。求他介紹。不然便要聲張去稟暴指揮。夏瞎子醉後失言。悔已無及。不敢拒他二人。恐有禍患。只得婉轉向暴氏說。自說感佩厚情。恐獨力不能報效。要薦賢自代。不知肯容納否。孰不知暴氏寬容大量。久有延納豪傑之心。因恐瞎夫撚酸。不好啓齒。今見他說這話。眞是入耳之談。一諾無辭。夏瞎子見他慨允。向暴指揮說。門下有兩個同伴。說得古詞甚好而多。特特舉薦來孝敬恩上。指揮甚是歡喜。就叫領了他二人來。說了半日。果然可聽。晚上留下。同夏瞎子一處起臥。那一夜暴氏竟悄悄開門下去。四個人滾做一床。輪流做了個通宵之樂。後來有人知道。編了四句歌兒道。

\begin{quotation}

三男一女一隻眼。一個陰門六個卵。

父夫作孽女妻償。正是天公有巨眼。

\end{quotation}

傳得人人皆知。只有暴指揮還在睡夢中。竟不知道。小鬼子雖是個化外的人。見暴氏如此不堪。便不肯同臥。暴氏屢屢強他。他推却不得。偷了些東西。不知逃往何所。後來暴指揮死了。他族中的人惱他刻薄。又見暴氏醜名難聽。無不掩耳。沒一個上門。暴氏獨掌了家私。更覺快心。常養着這三個瞎子。日夜作樂。後來被他寡伯母同觀音保並族中人公稟了官。差人夜間到他家。三瞎一女在床。光光的鎖了。只給了一件上衣穿着。次日帶到衙門。恨三瞎朋淫職官之女。每人四十頭號大板。一面重枷。都送了性命。暴氏本當重處官賣。念他祖父門第。免究。只攆了出去。家私房產入官。暴氏無人肯收留。他到了卑田院。做了衆丐之妻。暴指揮刻薄了一生。掙了個家私。却生了這個好女兒。替他出醜。人生行刻薄者何益。刁桓思謀人妻。未得沾身。不但自己送了性命。妻子落了這個下場頭。天處高而聽卑。淫賭二事。若能永戒。必不上干天譴。即酒之一字。亦當知節。刁千戶夫婦若不終日醺醺。或兒媳猶不至此也。刁千戶雖是酒徒。還無過惡。後來他房中有個使婢。叫做蓮房。刁千戶一時酒後高興。來同他點綴了一番。露滴蓮房之中。竟生了一個兒子。得繼後嗣。閒話且住。再說那些光棍枷滿一月。帶到衙門。樂公一生最惱恨是賭博。都問滿徒三年。這幾個人中。剛剛曾嘉才也在其內。他性凶貪賭。前次去騙兄弟。打鬧了一番。宦蕚替曾嘉禮給了他那二十五兩銀子。他欣欣得意。不暇歸家。就走到屠家賭場呼么喝六。不到半日。一送精光。過了幾日。見別人大包的銀子。成袋的銅錢。都在那裡大擲。他看得眼中冒火。心裡急得像滾油煎的一般。再要去騙兄弟。又無可尋之因頭。況宦公子又說過他再要去囮騙放肆。定要處治他。他雖是個賭棍。豈不懼王法。不敢復萌此念。竟把三間住房賣了。租了一間房子。有個小院。他一妻一子一女一媳擠着住下。他把房價又輸了。將家中床桌杌凳之類。凡値數十文之物。無不賣了賭去。一家全打地鋪。連吃飯就把地當了桌子。他家中虧得妻子同女兒媳婦做些針指度日。他兒子二十多歲。倒是個顧家的人。每日下苦在外做些小買賣。每晚掙三四十文回家。貼補母親度日。曾殺才沒法了。想出一條妙計。到一個相熟的藥鋪中。說要配老鼠藥。買了些砒礵。藏在身邊。到家中不住的嘆氣。他妻子道。你今日怎不要去了來。嘆的是甚麼氣。他道。我如今這麼個樣子。還賭甚麼。悔也遲了。我從以後起。誓再不擲骰子。捱這窮日子罷。他妻子道。你此時是沒有錢的話。恐怕有了錢。又不是這話了。曾殺才道。我也是個堂堂丈夫。說一是一的。先是心昏。賭了這些年。弄得傾家蕩產。還不灰心。眞連人味兒也沒了。你不信。弄壺酒來。我當天起誓\footnote{昔劉伶戒酒。誓云。天生劉伶。以酒爲名。一石而醉。五斗解醒。婦人之言。切不可聽。我代曾殺才誓云。天生殺才。有錢還來。婦人之言。安可聽哉。可確哉否。}。他兒子聽了。喜歡得了不得。說道。爹果然要戒了賭。別的不能夠。我就頭拱着地。每日掙飯來養活爹。冬夏好的不能。粗布衣裳我也包着有得穿。只要爹的心拿得穩。就是一家的造化了\footnote{好兒子。此等殺才。如何有福留得此子。}。曾殺才道。呆孩子。我恁大年紀。難道還不知世事麼。你母子們只管放心。那兒子笑容可掬的道。爹旣這樣說。我去賒壺酒來。替爹戒賭。飛星般去拎了一大坐壺酒來。他先斟了一碗。遞與父親。曾殺才假誓道。我此後再要耍錢。定遭官刑。不得好死。說了。把那碗酒一氣飮乾。再斟上。他叫妻子女兒媳婦都吃了些。壺中所有。他獨飮了。還剩下一碗。他暗暗將砒礵着上。向兒子道。我自幼受用慣了。一點事是不會做的。只好在家閒着。家中的事。全靠你去苦掙。將就捱這窮日子罷。這碗酒與你酬勞。他兒子喜笑道。爹放心。養我一場。別的沒本事。連碗飯都掙不來。還成個人麼。爹的酒不夠。請用了罷。我不吃。曾殺才道。我不吃了。這是我給你的。大家吃些。後來好同心協力的過日子。他兒子遂接過。幾氣吃下。收過了壺碗。不多時。面色發紫。叫肚裡疼。先還用手摀着。次後肚子疼緊。站不住。蹲在地下。他娘與妻子忙來攙扶。他忽然滿地打滾。口中大叫道。疼死我了。他母妻那裡按得住。只見滾了一會。嘴同鼻耳一齊冒血。氣絕而亡。他母妻妹子放聲大哭。只說他偶得暴病。那裡疑到是老子毒死了他。那殺才也假意在傍跌足嘆氣。他窮得這個樣子。那裡還有錢買棺材。拿了一片墊睡的破蘆蓆。找了兩條糟繩子。這一口斜紋軟棺材。加上金箍三道。就是他送終之具了。殺才自己背去。棄於城外亂葬岡上。他這媳婦娘家一個親人也沒有。只過了三日。殺才說家中無有飯吃。打發媳婦轉嫁。他婆媳那裡拗得過他。他串通媒人。賣與人做小。得了身價三十兩。瞞着妻子到賭場。三日不歸。絲毫無剩。銀子沒了。就想到女兒身上。有一個過路的官府要買丫頭陪嫁閨女。他帶人暗暗相了。講明身價四十兩。來擡人時。他母女纔知。哭得肝腸寸斷。眞是眼中流淚。心內成灰。生生拆散了去。他妻子怨恨塡胸。纔〔想〕到兒子之死。是他所爲。日夜哭泣。只剩他一個。孤孤悽悽。柴米俱無。傷心欲絕。曾殺才輸背了氣的人。把銀子拿到賭場。一日到晚。連快也不曾擲一個。越急越下住。越下住越貼臭。白亮亮一大包。輕輕又屬了別人。他心中想到妻子。一狠百狠。女兒媳婦都賣了。那老婆還留他做甚麼\footnote{想的甚有理。何不想到自己這樣殺才。還留他做甚麼。}。托媒人要賣他妻子。四十多歲了。一家要娶他續絃。只出財禮銀八兩。他急等銀子去賭。只得依了。他那妻子忿恨入骨。毫無留戀。大罵一場。上轎而去。他把賣妻之銀。又被六塊骨頭送去。這却沒得想頭了。房子退還原主。罄身挨到屠家來棲身。說道。四叔。你家中也沒人。我身子又沒家\footnote{此語趣。}。留下我相幫罷。屠四欣然應允。他就頂了竹思寬的衣鉢。屠四先有竹思寬相幫。到後來郝氏贅了他去。家中如拆了左右手一般。可還有這等下流的人肯到他家來做長工。年來屠四那半嬸半妻之通氏。因要生產。他是個寡婦。孕從何來。不敢去叫收生婆。屠四只得自己替他收接。不想娃娃橫在肚中。母子俱斃。那非弟非子的那個孩子\footnote{奇聞。}。沒了娘。無人照看他。屠四只顧得照管拈頭。那裡還有工夫去顧到他身上。飢一頓。飽一頓。得病死了。今得了曾殺才來。好不殷勤。又四叔長四叔短叫得震耳。屠四樂不可言。留他在家相幫。曾殺才過了些時。見沒有大油水。不過食粟而已矣。就入在衆光棍黨內。今遭了這一場官刑。枷滿問徒遠去。在路腰無一文。乞食前往。又値炎天。棒瘡腐潰。走了幾日。便死於路上。解差報了地方官。差人相驗。給了回文自去。將他屍骸抛棄荒郊。作爲老鴉喜鵲的口糧了。這是好賭的結局。却是眼前的活報應。那屠四是窩家。受刑旣多。枷號又大。家中並無一親人照看。也死於枷內。他的家私房屋無主。地方呈報入官。遣人淸查。他多年積了竟有二三千金之蓄。人屠戶屠四叔姪開了一生屠(賭)局。坑了人家無限不肖的子孫。雖聚多金。自己又不得受享。今日到了這個下場頭。有何益處。這叫做。

\begin{quotation}

天網恢恢。疏而不漏。

\end{quotation}

不必細說。再講牧福他正陪人坐着。眼巴巴望宦公子來替他還銀子。突然見一羣如狼似虎的公人走將進來。把這些人都拿去上鎖。他嚇得魂都沒了。鑽在床底下去躱。又聽得拷吊了那一番叫喊連天。他面目失色。渾身抖顫。衆人去了。他還不敢出來。屈氏笑道。你旣好賭。又怕的是甚麼。這是宦老爺替我們除害。要是拿你。床底下是躱得掉的麼。你出來罷。那牧福如夢方覺。纔放了心。爬將出來。滿頭滿臉。一身全是灰。屈氏替他撣着。說道。宦老爺今日必定來。你可預備些酒肴謝謝他。大遠的路。叫人家餓着肚子來回的走。也不好意思。那牧福定了一會神色。拿了宦家昨日拿來的那吊錢。帶着老家人到街上買了些酒肴果品回來。他道。我往庵裡去。屈氏道。你不等他來謝謝。又去怎麼。今日料沒人打鬧了。你還躱甚麼。牧福悄悄向他耳邊道。出這些力。又送這些東西。原是爲你。恐怕他要說甚私房話。我在家不便宜\footnote{牧福言至此。以爲屈氏之身決不能保矣。後日見宦蕚竟保全之。實出望外。}。那屈氏紅了臉。不好做聲。牧福去了不多一會。宦蕚乘馬而來。屈氏讓了進來。坐下拜謝了。就拿上酒來吃。說笑方纔拿人的這些話。正說着。那小廝驢子上馱了兩個大包袱來。送到房中。宦蕚叫放在床上。屈氏去打開。查了件數。宦蕚看看都是半新不舊的紬絹衣服。紗羅裙褲。並紬緞被褥\footnote{諺云。貧了富。還要穿三年布。富了貧。還要穿三年綾。他是富了貧者。故家中尚無布衣也。情景眞妙。}。宦蕚笑着道。你此後留着穿罷。再不要當了。屈氏道。這算你給我的。他如何當得我的。況家中又承你送了這些柴米。有飯吃就罷了。宦蕚道。你就把衣服換上罷。屈氏滿心以爲宦蕚未必放得過他。定要同他如此如此的。也不避他。便去掩上門。到床上破皮脫下。露出那團乳酥胸。竟是一塊無瑕的白玉。下邊穿着一件破夏布小衣。還有幾個大補釘。他換了一條半新廣紬小衣。兩條嫩腿猶如玉柱。一雙小脚實賽金蓮。宦蕚看得明明白白\footnote{屈氏少年婦女。焉能老臉至此。今寫他如此者。非謂屈氏之無恥。乃寫宦蕚見此等之美軀。竟能不動心之爲難耳。}。此時正是五月初旬。天氣正熱。屈氏穿了一件白線紗衫兒。縐紗裙子。上着石靑廣紗背心。耳上戴上金丁香。頭上關了兩根簪子。更覺得十分俏麗。他把別的衣服都收在一個大舊皮箱內。疑他酒後要高興。把床上褥子也鋪好。蓆子拭抹個乾淨。被也疊了\footnote{此處寫屈氏以爲宦蕚決定如此。孰不知竟不然。實出意外之想。}。然後來共坐飮酒。宦蕚讓他吃了幾杯。見他雪白粉腮。襯着微紅。此時也熟滑了。說說笑笑。兩隻媚眼生春。眞個是。

\begin{quotation}

巧笑倩兮。美目盼兮。

\end{quotation}

令人魂消。幾不自持\footnote{極贊屈氏是傍筆。高擡宦蕚是正意。}。宦蕚秉住了心。雖同他說玩說笑。總不動一毫邪念。吃了一會。叫小廝來。拿過了銀包。打開。拿了有四五兩散碎銀子。遞與屈氏。道。你留着陸續換了盤纏\footnote{余先見宦蕚送屈氏柴米時。只給錢一千文。甚疑。每與他人。或幾兩。或幾十兩。今與屈氏何其少也。至此方悟爲一片深心。先送錢一千者。暫時用度。恐多了。牧福又拿去賭輸。今覿面。暗與他四五兩。叫他留着陸續盤纏。其意可知。}。我過些時來看你。又把昨日典他的文書。在銀包內拿出付與他。道。這個你也收了。却不要與你丈夫知道。屈氏道。你爲甚麼不收着。怎交給我。宦蕚笑道。我要他做甚麼。或燒了。或留着。都憑你。起身而去。那屈氏滿擬他必然有一番動作。身子料保不住。見他不動而去。倒也猜詳不出是甚意思。晚上牧福歸家。夫妻上床。牧福道。他今日同你怎麼個意思。那屈氏道。只吃了一會酒。說說話就去了。連戲言也不曾說一句。牧福那裡肯信。道。這話哄娃娃也不信。他不是貪圖你。爲甚麼來。屈氏道。你旣然把我典與他。我的身子就是他的了。比得我私自做甚壞事。瞞你做甚麼。牧福到底半疑半信。此後宦蕚或半月或一月來看他一次。定留些銀子與他盤費。無柴送柴。少米送米。牧福但見他來。必推故避出。到冬來。又替他做了一身絲紬綿衣。連牧福並老家人兩口都做了綿衣。待這屈氏十分親厚。只是不及於亂。屈氏暗想道。他在我身上可謂百般用情。怎再不見他做甚事。是何緣故。他是好心人。大約是恐我不願。所以不敢妄動。我受他這樣厚情。除了此身之外。拿甚麼報他。等他再來。我去就他。再無推辭之理。一日。宦蕚又來。他是預備下的乾菜菓子好酒等候他來。一到就拿上來同飮。吃過幾杯。這屈氏與他親厚了半年。來往多次。雖不曾做那貼皮貼肉的事。却情孚意合。竟像夫妻一般。此時又有了酒蓋着臉。竟一屁股坐在他懷中。同他一遞一口的吃酒。吃到後來。屈氏少年婦女。一來要捨身報他。二來三杯落肚。坐在男人懷裡。未免烘動春心。拿嘴含着酒到他口中。宦蕚也笑着嚥了\footnote{昔有二人。論魯男子柳下惠之事。一曰。閉戶不納易。坐懷不亂難。一曰。旣坐懷。可以不必及亂。此易爲。閉戶不納者。誠難也。孰難孰易。諸君共評之。}。宦蕚知他是感情。故俯身來就。心中雖十分愛他。倒有二十分憐他。只是嘴中說笑。連手也不敢伸去在他身上摸一摸。吃了多時。宦蕚恐酒多心亂。把持不住。留下一錠銀子給他。忙起身別了回家。屈氏見他去後。疑道。這眞奇了。我這樣就他。他難道是鐵打的心腸。就不略動一動。要說他沒有那東西\footnote{這一想。是水窮山盡的想頭。}。我前日問他。他家中妻妾四五個。又都生有兒女。要說嫌我貌醜。我也還不是甚麼東施嫫母。這事眞令人不解。我旣然同他如此親厚。還怕甚麼羞。改日竟摸他一摸。看有陽物沒有。便可釋疑了。又一日。宦蕚來看他。天氣冷。屈氏同他並坐在火箱上飮酒頑笑。二人並肩疊股。合盞而飮。屈氏做盡媚態。撒嬌撒癡。睡在他懷內。說道。要說你不愛我。我看你疼我的心腸。百般俱盡。要說你愛我。我同你親厚了半年。總不和我沾身。是甚麼緣故。宦蕚只是笑。也不答應。屈氏見他不答。倚着酒意。忽伸手到他褲襠中一摸。宦蕚雖然不肯淫汚他。但這個千嬌百媚的美人倒在懷中。又做出十分嬌態。雖鐵石人也沒有不動心的。那根厥物。其硬如鐵杵一般直豎\footnote{寫得愈見其堅忍之難。}。不隄防伸手來摸。見他摸着了。笑着忙用腿夾住。屈氏先還疑他或沒此物。所以不做這風流樂事。今摸着了。不但有而已矣。且竟是放樣的分外粗大。唬了一跳。連忙縮回手。說道。你旣這麼動興。再不見你同我怎麼的。到底是甚麼意思。再三追問。宦蕚道。你起來坐着。我對你說。屈氏起來坐下。宦蕚正言厲色的道。我起初憐你。救你一場。我怎肯又淫汚你。我要做了這傷天理的事。與刁家那奴才又有何異\footnote{眞豪傑。}。我同你親厚者。一來憐你舉目無親。所仰仗我。若不與你這樣假親熱。我資助過你幾次。你未免心就不安。你少長缺短。怎好常問我要。你以爲身子屬了我。一家纔好靠我養活。二來我若同你做些苟且的事。我圖了一刻風流。豈不壞你一生名節。況你丈夫。今日他窮。出於無奈。敎你做這無恥的事。倘後來他有了好處。他不怪自己不成人。反責備你是失節的婦人。後來你夫婦如何相守。再者。我同你若做了淫媾的事。設或有了孕。生下來弄死了。豈不有傷天理。你家若留着。是我亂了你牧家宗祧。我如何當得這大罪過\footnote{眞菩薩。}。我若收了你去。又有你本夫些氣脈。我淸白人家。怎肯養個雜種\footnote{眞丈夫。}。三來我看你丈夫的人品。目今雖不成器。你牧家祖宗當日或稍有積德。他若能改過自新。將來或者還不終於流落。古人云。人人有面。樹樹有皮。況天下事再瞞不得人的。我若同你有私。後來叫他怎麼擡頭做人見人\footnote{眞聖賢。}。四來我正要煉我的心。雖不能到聖賢地位。也正要借此打磨個鐵漢子\footnote{眞鐵漢。}。所以百般堅忍。我今日雖然說破。你不必多心。此後我還照常養活你們。那屈氏聽了。忙跳〔下〕火箱。兩眼流淚。雙膝跪倒。說道。恩人。你這一番心腸待我。眞叫我粉身碎骨也報你不盡了。我每常感你的恩。不過想以賤軀相報。今日恩人旣這樣說。斷不及於亂了。但你活我之恩。與生我者並。我也無可報答。我認你做個恩父罷。不盡之恩。生生世世爲犬馬補報。說着。就叩下頭去。宦蕚忙起身拉住。道。你請起來。旣如此。我同你認做兄妹就是。屈氏道。我認恩人做父。還是過分。怎敢說兄妹。恩人若不稀罕我做女兒。下次我也不敢受一絲毫恩賜了。宦蕚見他心眞話急。也就受了他四個頭。認了父女。且說那牧福。他問過屈氏數次。屈氏回他宦蕚並不曾沾身。他心中不信。道。他我非親非故。他若不圖這些兒風流勾當。他爲何肯這樣竭力照看。這日。他在外邊偶然回來。見院子裡拴着馬。知是宦蕚在房中。天氣冷。他兩個小廝在廚房中烤火。牧福纔要避出。見院子裡沒人。心中想了想。悄悄到窗下來竊聽他二人舉動。看每常屈氏的話可眞。聽了宦蕚的這些說話。汗流浹背。報(赧)愧無地。暗想道。他倒這樣憐愛我。我自己反不惜皮毛。禽獸何異。我素常疑妻子是誑言。誰知他竟是這樣一位盛德君子。忙忙跑了進來。也流着淚。向宦蕚跪下叩頭。道。恩人。你恩德如天。我是不成人的料。無答報之日。我祖父陰靈也感恩人的恩私。今日恩人這樣的大恩。憐念我。保全我夫妻名節。我從此若不改過。眞是畜類不如了。宦蕚拉住。道。你果然能改過。替你祖宗父母爭口氣。勝如報我了。我別的不能。一年衣食我照舊供給你。他夫妻二人又叩謝了。宦蕚歸家。那牧福感恩無地。後來竟果然戒了賭\footnote{此一部書中寫好賭者多人。而能改過者。只戴遷牧福二人。足見人之趨於下流者易。改過上進者難。}。每每恨旣往之非。常常暗中流涕。屈氏次日雇轎子。老家人隨着。到宦家來。拜見宦老夫婦爲祖父母。拜侯氏爲恩母。向小娥爲次母。宦老問兒子他來拜認的緣故。宦蕚先述他二人父母的履歷。次及他丈夫不肖的話。後說因兒濟他的貧窮。故他感恩拜認。宦實也就信了。屈氏恐埋了宦蕚的好處。感恩的心重。竟不避羞。當着衆人。將他捨身報恩。宦蕚堅拒。不亂始末原由。細細吿訴\footnote{嬴氏在縣堂不避羞直訴者。恨入骨髓。屈氏對衆人不避羞細吿者。感入肺腑。其理一也。}。宦實大驚異道。我不過只說兒子變成了好人。行些善事。誰知竟造到坐懷不亂的地位。眞跨竈之子了。老夫婦喜歡不用說。侯氏小娥閤家大小。無一個不贊揚他的好處。宦老夫婦也憐念屈氏是好人家兒女。與了許多的東西。侯氏是恩母了。越發不用說得。留了酒飮(飯)。小娥也有所贈。屈氏竟滿載而歸。四時八節時常接喚。宦蕚月月不斷與他送柴送米。添補衣服。宦蕚間或到他家來。竟像嫡親父女。連戲話都不說了。屈氏敬他如親父一般。那牧福借妻子的光。也認了翁婿。過有年餘。屈氏的父親屈攀桂陞了南京通州知州。到京城來見上臺。找尋着了女兒女婿。見女婿家業蕩盡。要帶他夫妻同往任上去。屈氏雖不好對父母說那捨身的話。只說窮極尋死。遇宦恩父救了命。如何照顧一家衣食。如何接喚如嫡親父母一樣。如何宦老夫歸並恩母疼愛與東西的這一番週濟。詳細說知。那屈攀桂感激不已。登門拜謝。送了許多廣東土物。宦蕚也送下程請酒。兩下親家稱呼。仰氏同女兒也拜謝艾老夫人。親母侯氏向氏。然後纔一齊往任上去了\footnote{屈氏隨父母到通州。此後伸而不屈矣。}。那宦蕚一日在賈文物家拜壽。鍾生童自大鄔合都在那裡。賈文物備了極豐盛的酒席款待。並無一個外客。飮酒中間。鍾生笑向宦蕚道。我與長兄忝在至戚。同飮亦多次矣。總不曾見長兄一大醉。但恨弟一蕉葉量耳。不能奉陪。長兄約略也能飮多少。宦蕚見鍾生贊他的量。一時豪興大發。哈哈大笑道。弟不敢瞞親家說。酒色二字中。弟可稱一員驍將。酒之一物。弟自幼即能豪飮。醉亦有之。然而酊酩則未也。酒後性剛則有之。若云酒狂亂性則未也。至於能飮多少。倒從不曾較過。賈文物正想讓他酒。遂道。大哥尊量。弟亦不能窺其底際。今日弟之賤降。承衆位光臨在舍。鍾兄又欲見吾兄之量。何不一較之。將舍間所有之觥盞。大哥各飮一杯。何如。宦蕚道。賢弟取來。我吃了看。賈文物叫家人進去將大小各樣杯斝皆取出來。擺滿了一張大几。內中有一個金鑲沈香桶。約盛五六斤。又一個雕花大面爵。可盛四斤。訢(其)餘則金杯玉盞。瑪瑙。琥珀。玳瑁。犀角。象牙。海蛋。海螺。竹根。倭漆。螺鈿。銀爵。或大或小不等。童自大看了。吐舌道。哥。你這些東西得好兩千銀子纔製得來。叫我就不做這呆事。吃酒只要酒好。就是磁杯也吃得醉人。何必費這些閒錢\footnote{他此話。富貴人論之。定謂其吝而呆。道學人論之。誠至理也。以精金美玉爲器。而貯以柴茅村釀。能使之佳否。}。鄔合道。賈老爺是素富貴行乎富貴。老爺所說是成家守業的話。各人志向不同。如何一例論得\footnote{篾得通。兩家都奉承到。}。鍾生見拿出許多酒器來。笑道。若論這些酒杯。將盛百斤。如何吃得。但憑宦長兄盡量而止。我輩相契。不過適興而已。豈必強之以難。宦蕚聽了。立起大呼道。親家以我不能也。可自大至小篩來。家人忙將大香桶斟上。那是個沒奈何放不下的尖底。家人捧着。他以嘴就酒。數氣吸乾。道。何如。鄔合贊道。大老爺尊量。眞如滄海了\footnote{久不聞他諛語了。此處略點綴一二句。方不脫本色。}。宦蕚連道。斟來。斟來。他大者兩三氣。小者一氣一杯。席上十六碗菜未曾上完。他竟將几上所列盡皆飮畢。却一筯菜也不曾拈。大笑對衆人道。我之量如何。童自大說。哥。你不要怪我說。你也不像吃酒。竟像灌老鼠洞。這些酒差不多夠我洗個澡的了。笑道。要是幾年前。我見你有這大量。也不敢請你。幾時到我家。我雖沒有二哥這些好杯。我拿大碗也敬你這些酒。鄔合道。大老爺海量。眞天下無敵了。晚生看老爺興猶未足。門下家寒屋窄。不敢屈尊。今借賈老爺美酒。做個借花獻佛。下席來將那大香桶篩滿了。跪下奉敬。鍾生道。宦兄之量固宏。然酒亦足矣。可以不必罷。宦蕚此時的酒已有十分。聽見鍾生這話。他笑道。親家以我鼠量已盈耶。遂道。拿來。家人雙手持着。宦蕚對鄔合道。你起來。我飮。鄔合道。晚生特敬。如何敢起。求上過了。宦蕚大笑。也站起來。兩三氣飮完了。道。乾。請起。鄔合纔起來。那宦蕚也覺太過了。就靠在椅背上動不得。鍾生見他醉了。說道。宦長兄今日飮興大豪。也似乎過了。且在榻上小憩。若何。宦蕚道。親家以我醉耶。我特酒滿耳。我也不吃一點東西了。我仍躍馬而回\footnote{醉人不服醉。寫得逼眞。兄可與知者道。}。小廝們快牽馬過來。衆家人牽馬到。鍾生還要勸他。他起身下廳。到簷前一拱。道。恕不陪了。一躍上馬。呼道。我不醉也。得罪了。大笑鞭馬而出。走了不到數箭地。他酒湧上來了\footnote{寫酒亦有層次。先是酒滿。還不大醉。後一躍上馬。酒便上湧。然後方醉。妙。}。在馬上東晃西晃。家人忙上前兩邊扶住。前面一個攏着轡頭。慢慢的走。正走時。只見一個酒鋪門口圍着許多人。宦蕚道。是爲甚麼事。我進去看看。家人忙分開衆人。讓他馬進去。衆人認得他的多。又見他醉醺醺。都閃開了讓他。到了裡面。只見三四個人拉着那賣酒的往外拖。那人緊緊的扳住門枋。死也不放。說道。就是送我到官。也許我分辯分辯。容緩兩日。慢慢的設處。你拉我去怎的。宦蕚見了。喝道。爲甚麼。快快的放了。那幾個人也認得他。忙放了手。宦蕚叫那賣酒的問道。爲甚麼事。那賣酒的道。小的兩年前因沒本錢。問阮大老爺家借了十兩銀子做本。五分行利。月月不少。今兩年多。利錢也打過十幾兩了。這幾個月生意遲些。利錢交不上。打發這幾位大叔要把小的送到縣裡去處治。連本錢都要追。小的一時如何還得起。正在哀求他列位〈位〉緩兩日。他們不依。不想驚動了老爺。宦蕚聽了大怒。吩咐家人道。把這些放肆的奴才拿住打。衆家人見主人醉了。可敢不依。上前拿住。阮家三四個惡僕見他人多勢衆。又素知宦公子的名大。跪下道。老爺天恩。小的們奉主人之命。不敢不來。與小的們何干。宦蕚雖然酒醉。心中還明白。遂問那開酒鋪的道。你方纔說借他多少銀子。連本利共該多少。他道。本銀十兩。欠五個月利銀。共十二兩五錢。宦蕚哈哈大笑道。我當該多少。對阮家的人道。多大事。你家主人這樣要緊。你們叫甚麼名字。一個道。小的名字叫龐周利。他兩個一名盛〖考口〗。一名司敷\footnote{忙中伏下一筆。看官須牢記。}。宦蕚道。你三個明日拿了他的文書。同他到我府裡去取。又問道。該多少。賣酒的道。十二兩五錢。宦蕚道。我替你還他。饒這惡奴們一頓好打。你們是誰家的。答道。小的們是阮老爺家的。宦蕚對家人道。饒他去罷\footnote{寫他的話重復瑣碎。活是個醉人。活是說酒話。}。家人放手。那三個人爬起。飛跑而去。宦蕚此時覺酒越湧上來。有些把持不住了。說道。扶我下來歇歇再走。家人忙扶了下馬。到鋪坐下。那賣酒的見他攆去了阮家人。又許明日替他還銀子。心中快活不過。走到面前。道。這個去處。不是老爺坐的。請到小的房中坐一歇兒罷。宦蕚立起。就扶着他肩膊進去。吩咐家人道。你們在外邊伺候。衆人應諾。賣酒的扶着他。一步一踵走到房內。靠着桌子一張柳木椅上坐下。出來對他妻子道。難得宦大老爺解了這場禍。我不敢近前。你篩一杯茶送去。婦人是個蘇州人。頗有丯韻。長䠷身材。細白麻子。走路俏生生的。雖是布衫布裙。却十分乾淨。就是房中。雖無甚擺設。即床帳桌椅。也都一塵不染。他便篩了一鍾茶來。宦蕚醉眼迷離。道。放着。那婦人將茶放下。宦蕚道。那賣酒的是你甚麼人。婦人嬌聲嫩氣答道。那是儂家丈夫。宦蕚乜乜斜斜向他道。有你這樣個人。還愁無錢使麼。復大笑向他道。我是你甚麼人\footnote{此數語寫宦蕚已愛此婦之甚。而後來竟能堅持不亂者。所以更爲難得也。}。那婦人紅了臉。不敢答應。宦蕚此時已醉到十二分了。受不住了。道。我醉得很。我要睡睡。婦人道。老爺不嫌床鋪醜。請安歇安歇。那宦蕚就站起。摟住他道。你扶我床上去。那婦人沒法。又不敢得罪他。扶他到床上。他此時也忘其所以。只當是在家中。伸脚叫婦人替他脫襪子。只得替他脫了。他自己將衣服脫了。道。拿過去。那婦人也接了。搭在椅背上。他只穿一衫一褲睡下。婦人又拿被與他蓋上。然後出來。誰知他丈夫在窗洞中看得明明白白。遂拉住他妻子商議道。宦老爺雖許明日替我還賬。但是他醉話。不知醒了怎麼樣。我看他有些愛上了你。你陪他睡一夜。若同他厚上了。還愁沒吃沒穿的麼。那婦人抿着嘴笑道。這擠噶行得。儂若同他睏。他乘了酒興。還饒得過儂麼。這事儂弗會子幹個。他丈夫笑道。你又來說假話了。我每常覺得你會得很呢。要他不饒你纔好。你想。我們銀子沒得還。阮家把我送到了官。打了板子。還要追比。這房子是租的。連家私翻過來也不夠還他。那時弄得家破人亡。不如你捨了身子救一救罷。人家的老婆。瞞了丈夫。還要去尋野食。這是我叫你去救兩口子性命。怕甚麼羞。那婦人笑道。命雖救了。怕你的頭要綠哉。他丈夫也笑道。如今正經人家。那男人暗戴綠帽的不知多少。何況於我。頭雖綠了。不強如一頓板子打得通紅的血屁股麼。婦人笑道。你怕屁股痛。不難爲儂了。他丈夫道。但放心。你一點也不痛的。就是弄破了。我尋個皮匠替你縫戞兩針。還是照舊。二人笑了一會。那賣酒的又道。他一個大老官的性子。須你去就他纔好。你留心些。我到外邊照看他那些大叔們去。那婦人也未嘗不肯通融。見丈夫雖然這樣說。却不好慨允。那心中早已依了。見丈夫出去。他笑着進來。看看天晚。收拾完了。他蘇州人的此竅。無日不洗幾次的。那不必說。領了丈夫的命。也就上床。脫了上下衣服。掀開被。與宦蕚同衾共枕而臥\footnote{此亦與屈氏相同。婦人未必無愧心至此。蓋欲高擡宦蕚耳。}。看那宦蕚時。酣呼大睡。他有一番心事。不但睡不着。也不敢睡。到有四鼓。宦蕚醒了。心中想道。我昨日在賈兄弟家吃酒回來。到一個酒鋪中來。幾時來家。就不知道了\footnote{是個大醉後醒時光景。古詩有云。獨憶卸冠眠細草。不知誰送出深松。此數語在此詩中畫(化)出。}。覺得那被硬邦邦的。用手摸了摸。竟是布\footnote{大約宦蕚生平此是頭一次試新。}。心中說道。我家中如何有這被。這是那裡。見傍邊有一個人睡着。還疑不知是妻是妾。問道。你是誰。那婦人明醒着。不好答應。以爲等他高興之後再扳談不遲。問了數聲。他總不答。宦蕚伸手去摸。在他身上猶不覺。摸到了那妙處。覺得與妻妾之物大不相同。他此時酒雖未十分大醒。心內已明白。忙縮回手。問道。你是甚麼人。一連問了幾聲。那婦人料道隱瞞不住。只得答道。昨日老爺醉了。在我寒噶要睏。儂丈夫蒙老呀許還阮噶印子。無恩可報。故叫儂來服〔侍。〕宦蕚聽了。忙坐起來。道。豈有此理。你丈夫在那裡。婦人道。渠在外面同衆位大叔們睏呢。宦蕚道。我的衣服在那裡。婦人道。外面早得極。老呀再安歇一會兒罷。宦蕚道。那裡有這樣的事。你快遞與我。那婦人知他是不肯如此的了。忙穿了衣服下床。黑影裡在椅背上摸着了他的衣服。遞過去。宦蕚一面穿着。說道。快叫你丈夫點燈來。那婦人出去叫他丈夫。把前話向他說了。那人跌足抱怨道。我就說你不在行。把事弄壞了。他這一醒。決不肯認賬。婦人也啐了一口道。臭忘八。他弗肯個。難道叫儂攥住渠的不成。他丈夫只得點了燈來。宦蕚正色向他道。我一番好心。許替你還銀子。你倒做這樣的事。幾陷我於不義。那人忙跪下道。小的怎敢。蒙老爺天恩救拔。無可報答。所以想出這個法子來。宦蕚道。叫我的家人備馬。我馬〔上〕回去。婦人道。外面鑼纔四擊。又無月色。老呀回府。栅欄雖不敢阻。黑了弗好走個。宦蕚宿酒尚未十分醒。也怕路黑難行。便道。燒茶來我吃。那賣酒的忙忙去了。這婦人羞羞慚慚站在傍邊。宦蕚笑道。多謝你的美情。承你俯身相就。我想來也非你之本意。不過因貧窮所使。我雖不敢淫汚你。同宿半夜之緣。我也憐愛。明早叫你丈夫跟我去取。我與你五十兩銀子。除了還阮家。剩下的做個本錢。夫妻好好度日。以後這美人計萬不可再用。你婦人家一失了身。爲終身之玷。再悔不來了。那婦人忙紅了臉。跪下叩頭。宦蕚道。起來。起來。那婦人忙到廚下向丈夫說了。歡喜無限。燒上茶來送上。也叩頭謝了。他二人說話時。宦蕚家人皆在窗外潛聽。見主人如此。無不贊嘆。後來大家常常說及。鍾生知道。嘆道。不想他當日一個匪人。以爲改過已奇了。何期造到聖賢地位。可見蓋棺論定四字。方能定人之終身。賈童二人知道。皆自以爲不及。宦蕚坐到天明。叫那賣酒的跟了他家去。給了五十兩銀子。他叩謝而回。他夫妻因此而成家。供着他長生牌位。後來生了兒女。兒子的小名便叫做宦大。宦二。女孩兒的小名也喚做宦大姐。宦二姐。以誌不忘宦公子的恩德\footnote{受宦蕚之恩者多矣。而獨寫此賣酒人感之更深者。何故。謂保全人家婦女名節。其恩德更厚。借此意以警世間人耳。}。宦蕚數年來。他也不知救了多少窮苦患難。若要全記起來。眞可汗牛充棟。人背後編他兩句謠歌。道。

\begin{quotation}

昔年呆公子。今日善菩薩。

\end{quotation}

久之。傳遍闔城。這些小孩子都聽熟了。路上遇着他。就齊聲相和的唱起來。他聽見了。也自覺得意。越肯做好事。他一日出門。任着馬蹄行去。在梅生家經過。他下馬進去相探。梅生留坐。便酒小飮。正飮着。聽得隔壁人家一個老嫗一個婦人的聲音。哭得甚是悲哀。宦蕚問道。這家有甚麼傷心的事。哭得如此悲切。梅生笑道。這家一個兒子。有名叫做趙酒鬼。因醉死了。一個是他老母。一個是他妻子。古人說。幼婦哭夫。老母哭子。都是極悲慟的。宦蕚道。此人如何就到醉死的地位。兄試道其故。梅生道。說起來倒也是個笑話。可以佐酒。兄慢慢消飮。聽弟細說。以助一笑。二人一面對酌。梅生一面細談他的妙處。你道這趙酒鬼如何是個笑話。他父親倒也是個本分的人。家中也還有一碗飯吃。三十歲上纔生了趙酒鬼。這酒鬼娶得有妻。也生了一子一女。他自幼好酒。先還瞞着父親。私下偷吃。到了十八九歲娶親之後。也不避父親了。竟無時無刻不飮起來。後來糟透了。飮則必醉。他父親也罵過不計其次。他聽熟了。不但當是罵他吃酒。竟像罵着勸他吃酒一般。再醉得利害。到了三十多歲。父母六旬外了。他但天明起來。便到酒鋪中去吃。當日淳于髠是一斗亦醉。五斗亦醉。一石亦醉。他則大謬不然。雖好飮而量極不濟。一鍾亦醉。一碗亦醉。一壺亦醉。他的飮法亦奇。大約是讀過飮中八仙歌的。他內中摘了兩句。道是。

\begin{quotation}

道逢麴車口流涎。飮如長鯨汲百川。

\end{quotation}

他無錢時。三文沽得四兩燒酒。一口飮之。若有錢時。沽得一斤半斤。也是一氣飮下乾無滴。多寡總是一醉。他更有一件妙處。把劉伯倫酒德頌中兩句。學得爛熟。你道是那兩句。是。

\begin{quotation}

幕天席地。任意所如。

\end{quotation}

他但醉後。不拘街上路傍。放倒頭便是一覺\footnote{他也是從劉伶死便埋我句中學來。}。一日大淸早起。他吃得東倒西歪的回來。他父親見了。不覺嘆了兩聲。說道。孽障。酒誰不吃。也有個時刻。或午後。或晚間。消閒無事吃些也罷了。大淸早睜開眼就吃得恁個賊樣。我知道你那是吃酒。明明是作死。他哈哈的笑道。老爹。你有年紀的人了。怎還不知道理。一個吃酒。有甚麼時候。古人說。開門七件事。柴米油鹽醬醋茶。可見這酒是不等開門就要吃的。我聽見人念李太白的一首酒詩。我拿他當了聖旨。我念給你老人家聽。

\begin{quotation}

春若無酒花作羞。夏若無酒風生病。

秋若無酒月徒明。冬若無酒雪沒興。

早起無酒懶下床。晚間無酒睡不定。

一時無酒便有災。因此把酒當性命。

\end{quotation}

我續了他兩句。道是。

\begin{quotation}

世上若有同心人。幾句良言便相贈。

\end{quotation}

老爹你說。可通不通。我講個道理給你老人家聽聽。人家說早起甌一甌。強如做知州。這酒從淸早晨吃起。慢慢的自然就醉到午後下晚了。你道我作死。當日彭祖活了八百歲。你看他不吃酒來麼。世上的老頭子難道都是不吃酒的。那月子裡的娃娃。同娘肚裡的孩子。就死了。那也是醉死了的不成\footnote{他這一番說。實在他的令尊沒得答。}。我雖吃酒。還有個檢點。不像別人死貪着他。倒街臥巷撒酒瘋。我有個耍孩兒唱與你老人家聽聽。遂高聲大唱道。

\begin{quotation}

勸爲人酒莫貪。吃了他就發癲。行凶撒潑欺良善。雙親不識高聲罵。兒女相扶打幾拳。妻兒不敢傍邊站。勸人生休貪美酒。不飮他倒也淸閒。

\end{quotation}

他父母聽了。又好笑。又好惱。罵道。奴才。你旣知道這個曲子。你又望死裡貪他怎麼。我管你死不死。只可惜我白養了你這樣大。他道。我死只塡了我的坑。與你老人家毫不相干。你倒不吃酒呢。你的鬍子頭髮就不該白了。有了幾歲年紀。那滴溜都碌的葡萄話。不知打那裡來的。叫人入不上耳。復哈哈大笑道。

\begin{quotation}

三杯和萬事。一醉翻筋頭。

\end{quotation}

哎呀。快活快活。一步一跌的往房中睡覺去了。他父親不由得生氣。罵了幾句。飯也不吃。到房裡也就睡了。這趙酒鬼一覺直睡到次日天明方醒。渴了要茶吃。他妻子倒了一鍾茶與他。說道。你也三十多歲了。吃杯酒越發連尊卑都不認得了。昨日老爹勸你少吃酒。不過是疼兒女的好話。你嘴裡胡說亂道的。把他老人家氣了一日沒吃飯。睡倒在床上。一個六十多歲的父親。養你一個獨子。不能孝敬他。反倒叫他生氣。你心裡也安麼。你也現有兒女。將來不怕學你的樣兒麼。趙酒鬼道。放屁的話。我從來是極孝順的。除了吃兩杯酒。別的再沒壞處。況酒吃在人肚裡。又沒吃在狗肚裡。我可敢沖撞他老人家。這不過是你想勸我斷酒。拿這不孝的名來壓枉我。你當我不知道麼。他妻子道。你當我說假話。你過去看看老爹可有病沒有。你再問問奶奶你昨日說些甚麼話來。他道。我不信。我吃酒從來也不會醉。就有三分酒意。心裡像明鏡一般。再不胡塗的。他妻子道。你自己說的明白。三杯落肚。天也不知多高。地也不知多厚呢。你還知道甚麼。他道。當眞的。旣是這樣。我這酒還吃他做甚麼。我從今就斷了。再也不吃他。妻子道。你那有本事斷。你要斷了酒。除非狗就不吃屎。此時說斷。停會見了酒。喉嚨一癢。好又想開酒。酒鬼道。甚麼話。你把我看得半個錢也不値。你當我愛吃酒麼。我不過適興而已。漢子家說話。一言旣出。如白染皀。說不吃就不吃。甚麼要緊。我再要吃酒。如同吃脖子上的血一般。我今日同你打個賭。看我可有本事斷沒有。他妻子聽他說得斬釘截鐵。滿心歡喜。忙去向公婆說了。他父母雖信不過。想他或者戒了。也不可知。心中也暗喜。趙酒鬼果然虧他竟戒了一日。是平生所未有的事。到了次日。老早出去。下午時分。他吃的醉得不堪。一身臭泥。滿頭滿臉都是。帽子也沒有了。一個姓扶的朋友攙着送了他來家。說道。他不知在那裡吃得恁個樣兒。跌在溝裡倒浸着。幾乎淹死了。幸喜我看見。救起他。送了回來。他妻子謝了那人。扶着他進房。渾身臭不可聞。抱怨道。昨日賭咒發怨(願)說不吃了。今日越發醉的恁個樣兒。酒鬼大怒。跌跌舂舂。夾臉就是一拳打去。短着舌頭罵道。我肏你娘的眼。我吃脖子上的血。與你甚相干。那婦人見他打來。忙一躱閃開。不曾打着。他打了個空。失了一失。幾乎跌倒。越發怒起。兜襠一脚。正踢在那要緊的地方。那婦人一手揉着〖毛必〗。蹲着哎呀哎呀的叫。他那一兒一女見娘如此大哭。叫道。奶奶快些來。爹爹把媽媽踢壞了。酒鬼怒道。肏你多嘴的娘。一個一脚。踢得兩個孩子滿地亂滾。那婦人心疼兒女。怕打壞了。忍着〖毛必〗疼。掙起來。一隻手拉着一個。彎跑了出去。他便橫倒在床。頭向裡。脚拖在床沿下。酣呼大睡。次日醒來。叫他妻子。那婦人只得一瘸一跛的走到他跟前。他問道。你好好的怎麼瘸了。他妻子道。你昨日撒酒瘋。把我同兩個孩子都幾乎踢死了。還問怎麼。他大笑道。這是那裡來的鬼話。我前日戒了酒。昨日只吃了一杯。又不曾醉。好好的撒甚麼酒瘋。拿這沒影兒的話寃賴我。他妻子道。你不曾醉。你這一身臭泥是那裡的。你的帽子望那裡去了。要不虧扶大爺送了你來。大約也淹死在溝裡了。他看了渾身的泥。咂嘴道。這又奇。這又奇了。纔沒得話說。他妻子見他滿身滿床無處不是臭泥。心裡固然氣惱。又看不過。燒了水來。叫他洗了。渾身換了衣服。他又出去了。累得這婦人把被褥都重拆洗過。他父母知他是個勸不醒的了。說也無益。任憑他去。一日。深秋天氣。他又多了一杯。套學古人的詩句。略略改頭換面。古詩云。

\begin{quotation}

醉臥松竹梅林。天地借爲衾枕。

\end{quotation}

他在街上就高臥起來。竟一覺放開天地。穩的大睡。忽然下起雨來。雨雖不大。連綿不住。渾身淋得精濕。他在醉鄕深處。全然不覺。有一兩個認得他的。走來推叫。那裡叫得醒。大雨下着。人都怕濕了衣服。各人都自顧去了。他睡了多時。身上被冷雨一逼。也漸漸醒來。打了兩個寒噤。睜眼一看。原來睡在這樣一張大土床上。爬了起來。慢慢的一步一步的掙了回來。他妻子嘆了幾口氣。又把濕衣替他換了。放他睡倒。拿被替他蓋好。到了半夜。渾身熱如火炭。次日便不能起床。懨懨睡倒。延醫調治。藥都不受。服即吐出。茶飮(飯)都不吃。終日只飮數杯。他母親守着他。哭了幾場。他也心酸落淚。過了幾日。倒也覺得好些。飮食稍稍略進。他母妻喜得了不得。勸他道。你這一回若逃得出命來。眞是死裡逃生了。此後酒再不可吃了。留着命多活兩年罷。酒鬼道。我難〔道〕是死人麼。經過了這一回。還不知道。前日見奶奶往着我哭。我心酸得要死呢。又過了十多日。竟可以扶杖而起。也將有廿多日。一滴也不曾沾唇。一日偶出。大醉而歸。病復大返。却待斃了。他妻子坐在床沿上。流淚嘆道。每常爹媽說了你多少。我勸過你幾千百次。你總不聽一句。今日到了這個地位。丢得父母年老。妻兒幼小。你也放得下麼。他悔也無及。一言也沒。只長嘆了幾聲。滴了些淚。還要了一碗酒吃。便奄然而逝。他父親雖有這兒子。每常生氣。似有如無。見他死了。墮了幾點淚。也就撂過。他母親只此一子。焉得不慟。他妻子見公婆年邁。兒女幼小。自然哭得傷心。梅生是緊鄰。盡知底理。詳細向宦蕚說了。一(不)禁大笑。作別而回。宦蕚行了好事多年。越發勇猛精進。竭力行善。小娥數載連生三子。都好個齊整相貌。那宦老夫婦後來雙雙活到百歲。一日無病而逝。人皆以爲奇異。都稱他訓子積善之報。宦蕚夫婦同小娥家私越富。皆享期頤之壽。兒孫滿目。個個孝順。這都是冥冥中暗酬他的陰德。正是。

\begin{quotation}

欲享遐齡須積德。要生好子定存仁\footnote{閱至此。以爲宦蕚之事終之言矣。不意後面還有數段。眞寫得好。即如前面已行到水窮山盡。忽然一轉。又見奇峰突起。令人眼界倍新。}。

\end{quotation}

此是後話。且說那權氏在宦蕚家磨了二三年。雖有衣有食。無一日一時得暇。時常自恨自愧。那繆氏又常冷言冷語的點他。道。做婦人的。不管窮富。守着一夫一妻。將就度日子。就是造化。得享福呢。是命好。受窮呢。怨自己命不好。俗語說。命裡只該八合半。走遍天下不滿升。爬得高。跌得重。我們在人家當着個奴才。雖不愁吃穿。伺候主子。深不是。淺不是。一日提心吊膽。巴不得做個窮百姓。無拘無束。吃口涼水也安心。何等快樂。我聽見說你當日的丈夫還是個相公。就是窮些。誰不叫你一聲奶奶。你今日到了這裡。趕得上誰。人都知道你休棄丈夫。誰眼裡還有你。你如今可悔麼。權氏也無言可答。惟有眼淚鼻涕的哭。一日。侯氏生辰。有鍾奶奶戴姨娘梅奶奶賈奶奶童奶奶鄔大娘都來拜壽吃戲酒。撤席以後。正本兒點了爛柯山。朱買臣前逼。後逼。癡夢。潑水四齣。繆氏同權氏也在傍邊看。看到逼嫁的那個樣子。繆氏笑着悄悄的相(向)他道。你當日同你家相公吵鬧着要嫁。想也就是這個樣兒了。那權氏羞愧無語。繆氏道。一個漢子這樣跪着哭着苦留他。他還不肯。好個狠心的淫婦。笑道。丈夫這樣心疼。就窮死了何妨。怎就無恥到這個田地。權氏想起在平家。雖無穿少吃。丈夫也極恩愛。今日到此。有誰動憐。不住擦淚。那心又悔了幾分。繆氏冷眼看着他。看到癡夢那種醜態。繆氏笑着嘆道。你看崔氏這淫婦。當日耐一耐窮苦。今日何等的榮耀。大約他此時不知怎麼心悔呢。又看見張木匠出來那關模。笑道。揀漢精的娼婦。嫌丈夫窮。就該嫁個官兒做夫人奶奶去。還嫁了個木匠。你也就像他了。鄕宦財主嫁不成。嫁到人家來當奴才。羞得那權氏眞無地縫可入。又看到潑水那一齣。繆氏道。你看看這個淫婦。與其今日跪在馬前這樣出醜。何不窮的時候忍一忍。今日也是香車寶馬。何等受用。也怪不得他。他沒這個福。那權氏越深自後悔。聽那朱買臣唱道。

\begin{quotation}

恁娘行福分低。恁娘行福分低。做夫人做不得。恰纔是夫唱婦隨。舉案齊眉。你享不起。繡閣香閨。翠繞珠圍。蠢婦你年將四十。羞答答。薦誰行枕和蓆。

\end{quotation}

繆氏道。將四十歲的老婆。後面的光陰也就有限了。旣跟着丈夫苦了多年。就窮死了。也有個好名。何苦吵吵鬧鬧。到了人家。還是這個樣子。反落了萬代罵名。這是何苦。就算嫁了個財主。男子漢的心腸。見他嫌窮棄了前夫。一個活人妻。也就不把他爲重了。那權氏正是三十七歲出來的。聽了年將四十這兩句。又羞又恨。由不得泫然泣下。又聽得唱道。

\begin{quotation}

收字兒急忙疊起。歸字兒不索重提\footnote{蠢婦。你可記得當初拍掌的時節麼。}。我慘哭哭。雙眸流淚。的溜溜。雙膝跪地。那時節。求伊阻伊。實望指你心回意回呀。要收時。把水盆傾地。

\end{quotation}

繆氏笑道。這癡淫婦。水如何收得起來。與其今日求他收回。何不當初不要鬧出。我聽得說你的前夫雖不曾做官。這三年來得了美館。比當日大強了。又笑道。你幾時也去潑潑水。求他收你回去。免得在這裡受罪。權氏忍不住跑了回房。上床拿被裹着頭暗哭。此夜他一心痛悔欲歸。不敢出口。只把心腹話吿訴繆氏。時常流淚。那司富說了數次。他仍墮淚不止。司富一日大怒。拉到宦蕚的跟前。道。這老婆作怪。這幾日無緣無故。動不動就淌眼淚的哭。說着他總不理。要打幾下才好呢。宦蕚問他道。你好好的哭甚麼。他不敢答應。宦蕚怒道。他大約是想漢子了。這樣無恥的婦人。我上邊也用他不着。可將他配一個馬夫。叫他幫着漢子羣裡去煮料。㔍草的養馬的司婦就拉他道。跟我去。他跪下哭道。老爺就打死我也罷。我不願去。宦蕚道。你旣不願。你心裡要想怎麼樣。他欲說又不敢。只含着眼淚不作聲。繆氏在傍使了個眼色與他。道。老爺問你。你有話就說。怎麼含着骨頭露着肉的。權氏叩頭道。老爺奶奶的恩典。把我賞回前夫。就是萬代的天恩了。宦蕚道。你還想回去。只恐怕你到了他家。又想要跳槽。權氏道。我一念之錯。到如今悔已無及了。若得跟了原夫。就餓死也不敢再生他想了。宦蕚道。你當日賣到我家來。今日諒你丈夫那裡有銀子贖你。我爲甚麼白放你去。除非打一百皮鞭。一則戒你不許再效前番的舉動。二則算我的身價。你要受得。我就放你去罷。你怎麼說。權氏欣然道。老爺恩准我回去。情願領打。宦蕚叫取了皮鞭來。登時取到。宦蕚又問道。你果然願打麼。權氏道。願打。就爬在地下。宦蕚笑道。權記着你這一次。向司富道。帶他去罷。他當日的衣服換了來。司富遂叫他跟了去。宦蕚又吩咐去請平儒。權氏仍換了向日來的那衣服。帶了幾件首飾。又帶了來。宦蕚侯氏同站了起來。讓他坐。他不知是那裡的帳。那裡敢坐呢。睜着兩個大眼睛\footnote{他此時眞是睜着兩個大眼睛做夢。}。望望宦蕚侯氏。又望望衆人。宦蕚笑道。你請坐了。我有話對你說。司富拉他坐下。宦蕚把當初遇見他父親丈夫。說他要休夫改嫁。我知你夫家甚窮。就叫他強留下你。也不能相安。故商議了這個計策。弄你到我家來。磨磨你的性子。叫你後悔。你想一想。你就另嫁了人。一個活人妻。還有人敬重麼。我憐你夫妻。不忍看着你們拆散。故想出這個法兒來。你今旣然悔心。要歸前夫。是極美的事了。你原夫在我家敎了三年學。家中也不像那樣貧寒了。你此去安分守己。同丈夫一心一意的過。再有不肖的這念頭。恐就不能再容你了。那權氏聽說了。如夢方醒。見是成全他夫妻這一點好心。又羞又感。跪倒痛哭拜謝。侯氏忙忙親自攙起。又勸了許多的好話。還贈了他些衣服零碎物件。他又拜謝了司富繆氏衆人\footnote{司富只算得大座師。繆氏方是嫡親房師。}。外面來說。平相公來了。宦蕚出去道。恭喜。尊夫人已悔過了。遂將來歷。着兩個僕婦。一個做惡。一個做好。如何點醒他。今已悔悟。又將如何試他的詳細吿訴了。道。先生今日同回。可謂珠還合浦了。平儒揖而又揖。謝而又謝。宦蕚吩咐叫兩乘轎子來。又叫請出權氏。他夫妻一見。不覺大慟。雙雙拜謝。轎已到了。讓他夫妻上轎同回。隨後送了一桌菜一瓶酒去。平儒請了丈人來相會。權氏又羞又喜。一家深感宦蕚成全之德。念不置口。他夫妻後來甚是和美。白頭偕老。平儒敎了幾年學。得了兩百銀子束脩。雖不能豐厚。也不像當年無衣無食。一貧徹骨了。按下不題。且說宦蕚的大舅子侯敏。十數年來已升到太僕寺正卿。帶一封信來說。朝中四路發兵。太僕馬匹發盡。兵餉不繼。無從採買。兵部太僕寺公奏。奉旨新開捐納事例。內有一款。凡係革職內外文武大小官員。一品者捐馬二百匹。二品者捐馬一百五十匹。三品者捐馬一百匹。以下遞減。每匹折銀一百兩。准復祖父封贈。本身誥命。如捐復職者加倍。老伯何不趁此捐復祖父封贈。亦絕好機會。宦公父子商議。宦公道。我之封誥可有可無。我做官一場。祖父的封贈一並消去。深爲可恥。今去捐復了。也是一件美事。須你親去同你大舅商量行事。宦蕚答應了下來。遂差人先去雇船。尚書正二品該捐一百五十匹。着六個的當家人押銀一萬五千兩。從水路進京。先期去了。他自己帶了五千金。打旱路起行。要到京中托他大舅打點料理。收拾明白。擇吉日起身。衆家人要帶鳥鎗弓箭腰刀之類。宦公知道。問道。你們帶這些東西做甚麼。衆家人道。帶着這麼些盤纏。路上好防盜寇。宦公笑道。好不知事。你們帶着兵器。明是吿訴人帶着銀子了。古人說。投鼠忌器。若路上不遇着小人是萬幸。倘若遇着了。那都是亡命之徒。你們就同他敵得過麼。銀子失去小事。還要送了性命。你們不許帶一件器械\footnote{眞是老誠之見。}。即不幸遇賊。竟全送與他。我也還不窮在這幾千金上。只保你小主平安回來就罷了。衆人可敢不遵老主的命。鍾生梅生賈文物童自大治酒餞行。臨別之日。送至江口而回。宦蕚帶了十數個家人。雇了騾子進京。一路平安無事。一日。到了泰安州地方。離城尚有四十多里。一片荒郊。杳無人跡。有幾句道那時的境況。

\begin{quotation}

十里俄驚務(霧)暗。九天倏覩雲昏。八方民舍斷朝煙。七級浮屠無夜火。六翮飛禽爭投棲於別羣。五花頭踏盡潛避於州堂\footnote{此位州尊可謂畏賊如虎。}。四野牛羊皆沒影。三齊百姓悉無踪。兩下來人俱說此間行不得。一聲唿哨果然草莽有強徒。

\end{quotation}

正然走着。突遇一夥土賊。有五六十人。托鎗拽捧。蜂擁前來。也有拿着割麥的搧刀。有拿着劈柴的斧頭。頭上都裹着花布手巾。腿繃赤脚。一床藍布單被子拴在一根竹竿上做了旗號。敲着兩三面破同(銅)盆作了金鼓。圍了上來。手中亂舞。脚下混跳。口裡喞喞喳喳。只叫留下買路錢\footnote{確乎是一起烏合土賊行徑。}。衆人見了這些樣子。又好笑。又好惱。面面相覷。赤手空拳。寡不敵衆。可敢同他相抗。將所攜的五千金全然劫去。還將鋪蓋行囊。扛的扛。背的背。一轟去了。宦蕚同衆家人。一個個垂首喪氣。問了家人可還有剩的盤費。這個說還有兩餘。那個說還有三四兩。共算算。還剩有二十餘金。夠作盤費。可以到京。又走了廿餘里。到了一個大村莊中。約有千餘人家。覓了一座店歇下。店主見他們沒有行李。不肯留。宦蕚就坐在店門口。吿訴他午間遇了這夥賊劫去。店主道。近來土賊竊發。各處都有。多少不等。盡是餓民哨聚。地方官又不敢申報。來往的人吃了他多少虧。近來客人們都知道了。三二百結夥同走。方保無事。你們怎麼也不問一問。就冒冒失失撞了來。可惜失去了一主大財。主僕們商議還是報官。還是走路。宦蕚道。據店主說。四處都是賊。報了官。去拿那一起的是。知道是誰劫了去。只管守着。豈不耽誤了大事。忍着撂了罷。到京尋你大舅爺商議。再作區處。但只是沒有行李。恐路上盤詰瑣碎。正在躊躇。只見一個人走進店門。向着宦蕚納頭便拜。道。恩人方才吃驚了。宦蕚連忙扶起。看了看。不認得。問道。尊駕是誰。面荒得很。怎麼認得我。又何以知我遇賊。那人笑道。老爺不認得小人了。小人名叫賴盈。那年該了賣貨郞姓畢的十兩銀子。蒙老爺替小人還了。又賞了小人一錠盤費。小人想。一身是病。在外沒用。就趁那銀子做路費。回來兩年。病倒好了。今年又遭了流賊。只剩了一身。又値年程荒歉。只得入了賊夥度命。老爺的天恩。小人是時刻想念着。方才在那裡見了老爺就認得。因同衆人在一處。小人不敢認。特暗暗跟了下來。老爺可報了官。多着些官兵。小人領了去。靠那些毛賊中甚麼用。所失的東西。一去就可奪回。宦蕚大笑道。今日晚了。我們明早同到州裡去。正然喜笑。只見門外一陣有三十餘人。都騎着馬。個個彎弓揷箭。臂鷹牽狗。簇擁而來。宦蕚正要問店主是甚麼人。只見爲首的那個彪形大漢。一眼看見他。忙跳下馬來叫進來。道。這不是南京的宦恩兄麼。宦蕚忙站起。細細將他一看。原來是鮑德。他一把拉住宦蕚的手。道。恩兄幾時到的。那陣風兒吹了你來。這兩年想殺俺了。若不是我今日出來打圍。幾乎錯過。如今往那裡去。宦蕚將上京有事。適間遇賊被劫。並賴盈才來報信。明早要去報官的話相吿。鮑德笑道。恩兄放心。包在弟身上取來。還且請到舍下去再講。宦蕚眞是他鄕遇故知了。無限的歡善。叫拉出馬來。同他並奇(騎)而行。到了他門首。好一所大宅。門外都是合抱的大柳樹。圍牆數仞。四角四座看家樓。進了大門下馬。二門內方是大廳。兩邊刀鎗兵器揷滿數架。兩人揖罷坐下。鮑德道。自從別後。無一日不想念恩兄。我屢屢要南去一會。因連年荒歉。盜寇縱橫。不敢離家。今日甚麼風吹得恩兄到這裡來。叫小廝。快去請辛大爺來。你說南京的宦老爺來了。宦蕚道。令姑母安健麼。令表兄府上在那裡。鮑德道。家表兄那年承恩兄資助盤纏。鮑(兼)程星夜來家。家姑母一見。病就好了。近來着實康健。每常感念恩兄不盡。宦蕚道。多大事。爲何尊兄這樣掛齒。使弟不勝汗顏。不一時。辛同到來。深謝向日之情。少頃。拿上酒肴來。雖不比宦蕚家烹調味美。他都是豬羊鵝鴨燒煮着。大盤堆砌饅首薄幷(餅)米飯粉湯。也十分的豐盛。鮑德同辛同陪着。又吩咐家人款待宦老爺的管家同賴盈吃。他主僕上下都吃畢了。請宦蕚到小齋內坐。又擺上菓品醃臘下酒之物。讓了坐下。鮑德向他道。弟有些須小事。今晚不得奉陪。家表兄在此相伴。宦蕚道。尊兄只管請便。鮑德去了。辛同陪着飮了一會。宦蕚不用了。榻上已鋪設下簇新的衾枕\footnote{與前鮑德到他家一對。}。辛同吩咐下人。管家們都給他們鋪蓋\footnote{細。}。答應俱有了。然後二人對面兩床睡下。宦蕚着了辛苦的人。又因心中歡喜。多飮了幾杯。一覺直到黎明方醒。忽聽得外面人聲洶洶。馬嘶犬吠。宦蕚驚問辛同道。此是何故。辛同笑道。大約是舍表弟回家來了。宦蕚道。令表弟何處去來。還未說了。只見鮑德箭衣紮袖。頭裹包巾。腰懸鐵鐧。如天神相似進來。哈哈大笑道。幸不辱命。宦蕚忙起來看時。許多人搬進銀子搭連並鋪蓋行李。所失之物。一件不少。問鮑德道。尊兄效三鼓奪崑崙之法。請敎在何處得來。鮑德笑道。弟與家表兄在此處頗有個聲名。我這村中有二千餘家。老幼不算。健壯男子將有三千人。農忙時耕種。閒時操練武藝。做古制寓兵於農之意。衆人尊我兄弟二人爲首。悉聽調度。器械皆是我給他們。他等齊心守護莊村。一聲有賊。我二人一個領衆殺賊。一個統人守護。不要說這些土寇。就是些少流賊。也不敢到我這裡來\footnote{伏後點燈子敗去。}。這左近的毛賊。我也不去傷他。他也不敢來犯我。昨日晚間別了恩兄。帶着賴盈。我領了幾十個人。有二古(鼓)將盡。到了那裡。衆賊正然好睡。將一個個綁了。追問這項銀子東西。他們聞知是我朋友的。他等磕頭賠罪。雙手送還。一絲不〈不〉少。弟也便饒了他等。宦蕚謝道。非兄大力。此物已屬他人了。但只賴盈是不能回去了。鮑德問他道。你可肯在我這裡。要是肯住下呢。我替你安個家。也很容易。賴盈忙叩頭道。蒙老爺天恩收留。小人的大造化了。宦蕚梳洗了。要到辛同家去相拜。辛同辭道。不敢勞尊駕罷。宦蕚道。不但有老伯母在上。就是尊兄。也沒有個在此一會的理。竟不到府上。辛同同鮑德陪着他。也不騎馬。三人步行。同到了辛家。重又作揖。托他稟候老伯母。他老母請到上邊去一會。深謝了一番。坐了片時出來。就留酒飯。宦蕚要辭行。鮑德笑道。恩兄好容易得來。至少也住十日。宦蕚將捐復祖父封誥的話相吿。恐誤了日期。他二人道。旣爲此大事。不敢苦留。兄回來時。在此多住幾日罷了。宦蕚道。這不敢許。弟或水路回去。或又走他道。怎敢失信於尊兄。他二人道。罷了。兄今日住一日。明早送別。宦蕚見他二人情意殷殷。不好再辭。也就住下。吃畢酒飯。辛同留住他下榻。他每人以二百金爲程儀。宦蕚再三辭謝。道。弟所帶盤費儘夠用了。不敢勞二位尊兄費心。他二人知他帶的銀子多。也不相強。午間備席共飮。鮑德道。兄旣遠來。才會得一日。就要別去。何以爲情。向辛同道。近日賊寇公行。我要保家。去不得。今宦兄攜着重資前往。我又不放心。恐前途有失。奈何。老長兄帶幾個孩子們。護送他到盧溝橋再回來。方才放心得下。辛同欣然道。我明日同去\footnote{此行用辛同送去者。彼二人皆受過宦蕚之情。鮑德奪回行李。已報之矣。故辛同遠送。以報向日之情耳。作者一筆不肯易下。}。宦蕚是驚弓之鳥了。見他說送了去。說道。承二位尊兄如此見愛。眞朋友而骨肉了。一宿晚景不題。次早約到他家。吃了酒飯起身。宦蕚臨行。給了賴盈一百兩銀子安家。他要推辭。宦蕚不肯。他叩頭領了。鮑德同賴盈送了廿餘里方回。辛同帶了七八條健漢。都帶着弓箭。騎着壯馬。直送到盧溝橋後。方作別回家。宦蕚言謝不盡。兩人分手。宦蕚進了京城。到他舅子家住下。他二舅子侯捷也相會了。一番親熱接風。不必細說。托他打點。錢能通神。自然明白。家人押的銀子也到了。交了進去。仍將昔日追出的官誥給還。宦蕚見旱路的賊多。要從水路回去。他素常聽得鍾生說戴氏的父親在張家灣開大船埠頭。他叫人先去問着了。說了詳細。此時戴良老做(故)了。正是戴遷主家。他久矣接女兒的信。知他的外孫定的是宦尚書的孫女。宦公子的女兒。不勝欣喜。今聽得他來到。忙叫請了來。酒飯相待甚農(濃)。次日。又戲筵款待。宦蕚甚是不安。煩他雇了兩隻麻溜船。要圖趕快歸家。戴遷又送了許多下程食物。煩他帶信與鍾生。又帶了些東西送兩個外生(甥)。宦蕚謝了他上船。晝夜兼行。月餘到家。宦公見請了誥命回來。心中大喜。宦蕚說起遇賊劫去。正在進退兩難。虧得賴盈報信。鮑德奪回。辛同送至都門。詳細稟知父親。宦公嘆道。俗云。行好自有好報。做好人何嘗吃虧。可笑世人不肯行好。奈何。宦蕚取出戴遷的信。同帶來之物。差人送到鍾生家去。鍾生同賈文物童自大梅生又來賀喜接風。熱鬧了十多日。過了月餘。一日。鍾生來對宦蕚道。賈兄做了一件豪舉。我們竟不知道。昨蒙聖恩。特授兵部職方司員外。他到舍下來問弟當受不當受。弟才得知。宦蕚道。請長兄細說其詳。鍾生自首至尾吿訴了。宦蕚道。可惜這場義舉。被賈弟一人做了。我們少不得大家約同公賀。你道賈文物做了甚麼義舉的事。平白地就得了官。且看後文。便知分曉。

姑妄言第二十卷終



\endnotetext[1]{「坐」原作「作」,據書前目錄改。}

\endnotetext[2]{「公老」原作「老公」,據下文改。}

\setcounter{footnote}{0}

\theendnotes

\part*{姑妄言第二十一卷}
\addcontentsline{toc}{part}{姑妄言第二十一卷}
\markboth{姑妄言第二十一卷}{姑妄言第二十一卷}

鈍翁曰。〇〇〇〇〇〇〇〇〇〇〇〇〇〇全眞。然皆頗有影〇〇〇〇〇〇〇〇〇〇〇〇〇〇〇〇〇〇祿乃見於史册〇〇〇〇〇〇〇〇〇〇〇〇〇〇〇〇〇〇〇賊攻城掠地〇〇〇〇〇〇〇〇〇〇〇〇〇〇〇〇〇〇〇〇〇〇〇〇〇〇〇〇〇〇〇〇〇〇〇〇〇〇〇〇〇〇〇〇〇〇〇〇〇〇〇〇〇〇〇〇〇〇〇〇〇〇〇〇〇〇〇〇〇〇〇〇〇〇〇〇〇〇〇〇〇〇〇〇〇〇〇〇〇〇朱珠兒實有其人。並非捏〇〇〇〇〇〇〇〇〇〇〇他三人稟史司馬之語。眞破〇〇〇〇非紙上談兵者也。

聽說捐俸。汲斷金幾乎急斷筋。傅聖(勝)係富甚之大臣。無視國家之事。一毛不拔。反訴許多苦惱。聽得借庫帑。牛騂又十分牛心。都是此等臣宰。如何不把明朝天下送去。

賈文物之捐貲。實由於鮑信之之鼓勵。賈文物救衆之功固大。鮑信之慫恿之功亦不小。賈文物旌之以官。理固應然。鮑信之亦得受職。不爲過也。

闖賊連破洛陽汴梁二事。俱載正史。一字不謬。然正史猶未若是之詳。看之令人髮指。

正史載裁驛一事。實倡於毛羽健。而成於劉懋。此罵羽健身爲龍陽。妻淫家僕。猶不足盡其罪。或謂此雖係罵羽健。故及其妻。但不當辱及溫體仁。然有說焉。體仁初入閣時。民間即謠云。豬遭瘟。瘟(溫)乃國姓。謂朝廷之用溫相也。其實體仁不但庸懦不堪。且壞了許多大事。罵〇〇〇〇亦不爲過。羽健以悍妻之故。流禍於國家。承〇〇〇〇〇〇〇〇〇〇〇〇〇其罪小。逢君之惡其罪〔大。〕裁驛疏上。乃劉懋一力奏准。其罪浮於羽健。故後身被殺。妻配賊復淫於人也。

此一部書中。一個人有一個小傳。有先敍來歷而後敍其事者。有前後敍事而中段敍其來歷者。有事將敍完而末後始出來歷者。有敍他人之事內中帶出此人來歷者。種種不一。非細心觀之。不能見也。即如大方家作文字。或兩大比。或三股。或散作。或八股。非如小學生初開筆。如板上釘釘。起股。中股。後股。束股。板板六十四。一定而不可移之死規矩也。

敍毛氏之事多矣。至此方細出始末。不但其文有參差先後之妙。更足見其不肯遺漏一筆。

\chapter*{姑妄言卷之二十一\\
第二十一回 史司馬爲國憂民 賈進士捐貲殺賊\\
附 李自成萬惡滔天 鮑信之一心奮義}
\addcontentsline{toc}{chapter}{第二十一回 史司馬爲國憂民 賈進士捐貲殺賊}
\markboth{第二十一回 史司馬爲國憂民 賈進士捐貲殺賊}{第二十一回 史司馬爲國憂民 賈進士捐貲殺賊}

話說那賈文物做的是甚麼義舉。他竟是爲國爲民的一段熱腸。因自成這個惡賊。向年兵犯鳳陽。斬陵木。燒寢殿。殺官吏。縱罪宗。搶劫一空。大有所獲。他心猶未足。直殺到沿江一帶州縣。有覬覦南京之意。那些官軍聞風而逃。可憐那老弱黎民盡塡溝壑。子女玉帛車載馬馱。屠戮之慘。眞不忍言。因鳳陽是祖陵要地。四處官軍兵馬雖然十分害怕。少不得要求恢復。援兵四集。那些流賊因婦女衆了。輜重多了。也不暇來攻取南京。他原不要城池地土。聞知此信。攜着紅裙翠袖。囊着白鏹黃金。方談笑鼓舞而去。這些逆賊見地方旣富庶。守備又無人。來往自由。好生樂意。時時刻刻還想來擾亂一番。我且把這瞎賊的出處說個明白。看官方知他的來歷。然後再說他的那些慘惡。以見那時生民塗炭。我們大家唾罵他一番。稍洩當年那些人的怨氣。他祖籍係陝西延安府米脂縣人氏。世居於此。他父名李守忠。他家七八代前的一個祖宗家甚富饒。一生酷喜齋僧養道。數十餘年力行不倦。人皆稱他爲李善友。年將衰暮。忽一日。有一個邋遢道人。臭味難聞。到他家來化齋。李善友毫不憎惡。欣然款待。齋供已畢。道人向他道。貧道素知老居士樂善不疲。後世子孫必有大貴人出。貧道四處雲遊。離此二百餘里。萬山之中有一塊福地。老居士百年之後可卜吉於此。將來定有興者。李善友歡喜無限。邀請這個道士同往去看。道人也不推辭。李善友備了行李頭口。到了那山脚下一村中居宿。原來這村中有許多李姓之人。李善友敍起宗譜來。都是一族。尚在服內。更加歡喜。次日。同道人入山點了穴。道又(人)道。葬時須起造一大圈。內設大鐵缸一口。滿貯燈油。若鐵燈不滅。李氏當興。李善友深謝了道人。仍約他回家厚贈。那道人笑道。我爲居士擇此善地者。報生平之善行耳\footnote{閱此。古云陰地不如心地。善哉言也。若此道人所點之地不佳。塚中枯骨亦何以安。異日伐塚時。腦中有龍。屍骨皆靑。亦異地也。若謂佳穴。塚中枯骨猶然暴棄。子孫死於法者幾盡。所佳者何在。昔日朱文公見一惡人葬吉地。嘆曰。此地不發。是無地理。此地若發。是無天理。後此墳被水沖去。可見不如心地也。吾思道人點此一穴。並非無因。豈爲李闖謀耶。須反觀之。}。豈圖報耶。遂拂袖如飛而去。李善友追之不及。衆皆驚異。以爲是神仙點化。李善友歸家。便將此事與兒孫說了。再三囑其死後如法安葬。又過了十數年。李善友老故。子孫遵他的遺言葬下。後來他祖間聽得說這是一塊福地。都想沾些餘福。李善友的墳居中。週圍竟葬了有十數處。傳到了李守忠。他是弟兄二人。他哥哥名叫李守義。長他有三十來歲。生了一子一女。子名李自達。比李守忠倒還大了兩歲。李守忠在縣中當了一名捕快。他生性暴戾。凶惡無比。却手段高強。數百里內的強盜小賊。無一個不是他的門下。年年納奉。月月餽金。他到了三十餘歲。尚無妻室。一日。有一個相士偶然遇見他。嘖嘖稱異。道。我閱人多矣。未有見君相貌之奇者。李守忠問他緣故。相士道。他人之相。窮通壽夭應在一身一世。而君之尊相。應在後人。將來定生貴子。但須積些福德。則異日貴盛無比。他聽了這話。暗合他祖上的傳言。他此時囊中所積也有二千餘金之貲。遂辭了差使。因想貴子尚還無母。央了一個姓連的媒婆尋親\footnote{媒婆而曰姓連。何意。謂男女一姓恰巧皆托他一人而聯合也。}。就將相面的〔話〕吿訴了他。要娶一個有福的妻子。好生貴子。那時有一個名妓姓苟。老鴇死了。是他自己當家。也三十餘歲了。在風塵中歷了將二十年。個中滋味已經嘗盡。意欲從良。尚還未決。一日。有一個番僧到他家來偷嫖。這苟氏閱歷之人雖多。從未嘗見過此凹目凸鼻捲鬚環耳的異物。欣然留宿。交會之後。這番僧相(向)他道。我看你骨格淸奇。後來定生一個貴兒。不可在這風月場中。錯過了可惜。須嫁一貴夫。以圖下半世受享。苟氏聽了。正合他向來從良之願。也煩媒人替他尋覓好夫。這媒人恰好就是李守忠所托的連氏。連氏便將相士說李守忠的話相吿。苟氏滿心願嫁。連媒婆又走去向李守忠也將苟氏當生貴子的話說了。李守忠見他兩人的不約而同。無限歡喜。就擇吉行聘。娶了過門。一個貴屌。一個貴陰。無夜不造作一番。想生貴子。誰想造了數年。貴種已將下盡。而貴子毫無影響。李守忠一夜向苟氏嘆道。我同你這幾年來貴種下了無數。貴精去了一盆。並不見過貴子的影兒。眞是可惜。苟氏笑道。便是貴子。也不過是偶然的一次貴種遇着。若你次次下的都是貴種。我的這一塊陳媽媽。竟是一張百官誥了。二人大笑了一場。那一年。他到了四十歲。尚還烏有。他夫妻着了急。一同商議齋戒沐浴。往西岳華山金天大帝廟中去求子。燒香回來。一夜。夫妻正然睡着。同夢見金天大帝領着一個沖天冠赭黃袍的皇帝。向他道。此破軍星也。賜汝爲子。他夫妻夢中驚喜拜謝。醒來。彼此相述。深以爲異。忙起來洗沐了。焚香叩謝。他二人得了此夢徵。每夜越加下力。你看他好造。直造得力盡精疲。那苟氏腹中果得了孕。他二人見有應驗了。心中歡喜。益發用力。直造到十月滿足。生下了一個兒子。就是李自成了。李守忠因夢中見他穿着黃袍。故起個小名叫做黃來兒。他夫妻疼這兒子如同至寶。到了七八歲。便生性憊懶。在街上同一般大的小孩子廝打相鬥。無日不然。此時李守義夫婦並兒子李自達俱亡故過。女兒已適了人。媳婦也改嫁了。只存一個孫兒。名叫李過。比自成只小一歲。他二人雖是叔姪。竟做弟兄相呼。相幫着在外生事闖禍。李守忠要送他二人上學去讀書。他兩個聽見了。便躱得不知去向。李守忠驚得幾死。四處找尋了來家。再也不敢重題此話。到了十五六歲時。他叔姪二人俱好嫖好賭。李自成自幼是他父母驕縱慣了的。百依百隨。只有要上天的星。那摘不下來的。就沒奈何。除此以外。力可爲的。無不聽其所欲。他要銀錢去嫖賭。李守忠可敢攔阻。任他揮霍。李自成酒色財氣四個字無一不好。於色字又分外重些。他生性雖然凶惡。却帶幾分呆氣。那李過凶暴與叔叔無二。還加奸狡些。李自成因常在外生事闖禍。人替他起個混名。叫做李闖子。李過力量粗雄。更是頑劣。人也贈了他一個混號。稱爲一隻虎。李自成常在這些妓女人家走動。他的一個陽物生得渺小無對\footnote{豈其然。游夏流便是對子。何謂無對。}。只三寸來長。大指粗細。這些妓女們就編了幾句口號嘲笑他。道。

\begin{quotation}

李自成。李自成。他的㞠子笑殺人。硬了只有拇指大。軟了好似細麻繩。

\end{quotation}

久之。他知道了。心內含愧。不敢再去嫖妓。想道。這些淫婦。他經過幾千百個漢子。自然嫌我的細小。先也還不肯自信。後來但是到出恭的去處。或是浴堂之內。他留心看別人之物。實在也沒一根像他這樣小巧的雅致了。方以爲然。自忖道。我這東西實在難看。我只娶個眞正閨女做了老婆。他只見過我一個。自然就不憎嫌我了。又想道。就是娶了人家的女兒來。如何知道他是眞的不是眞的。忽然悟道。有了。我常聽見人說。女孩子初次破身。定然要疼。只看他疼不疼。便知道了\footnote{難爲他好悟心。}。主意拿定。問他父親要老婆。李守忠見兒子在外胡行不休。久想要替他娶個媳婦。或可收攬住他。不知兒子心中如何。不敢開口。今聽他要娶。滿心歡喜。就央媒說合。替他娶了個姓屈的妻子。倒是個眞正女兒。成親之夕。因他的陽物太微。那女子也不覺艱難。竟容下了。李自成見他並無苦楚之態。疑心道。不好。這不是女兒了。却又十分拿不定。想道。是了。要是眞女兒。自然認不得㞠子\footnote{眞愈想愈奇。}。等我問他。看他認得認不得。就明白了。因捏着陽物。問那女子道。這是個甚麼東西。那女子含羞不答。每夜叮問。過了數日也熟了。那女子見他只是問。聽得瑣碎了。笑道。這不過是個雞巴。你只管問甚麼。他大詫道。你旣是女孩兒。如何認得雞巴。定然不是眞的了。起來對父母說。媳婦是個破罐子。要休了回去。李守忠先也不肯。禁不得他成日在家大鬧。李守忠不得已。叫原媒送了媳婦家去。那屈老兒不知是那裡賬。雖兩家費了許多唇舌。也還疑女兒或有差謬處。只得隱忍罷了。李自成親托媒人。要替他尋個眞正女兒。媒人四處打聽。又尋了一個的的確是黃花閨女了。娶過門數日。仍是如此。又把女兒退回。這女子的父親名字叫做韓淵。也是個有頭臉的人。心中不忿。吿到縣中。拘了李自成去問。他執定說不是處女。故此不要。知縣沒處查考。只得向韓淵道。夫妻是白頭相守的。他旣不願。強合了。你女兒在他家也難過日子。不若你把女兒留下罷。那韓淵見官府說得有理。心中雖含寃恨。只得罷了。兩家打了一場官司出來。李自成把媒人抱怨個不休。說他不打聽眞實。兩番誤了事。媒人心下甚疑。走去問那兩家女子道。怎麼成親之夜不說。定過了幾日。方說是破的。是何緣故。那女子含羞帶忿。細述其由。媒人不覺大笑。方知其中之故。那時有一個妓女也姓韓。生得頗覺俏麗。雖纔二十多歲。一個陰戶。其寬鬆無比。自小肚子上。以至股溝之下。一片長毛布滿。幾幾乎無門可入\footnote{毛氏有此異牝。不想此婦亦然。豈惡賤(賊)奸臣之妻。陰戶皆如是耶。}。而且交合之時。淫水長流。涓涓不息。內中其冷如冰。有那嫖過他的人見他這一件出奇之美窟。贈了他一個雅號。稱爲韓松泉。謂其又寒又鬆。又謂淫液如水之多也。這韓松泉之名一出來。下顧者甚少。只有縣中一個衙役。姓蓋名君祿。他的陽具有七寸餘長。棒槌粗細。別的妓女遇了他。皆逡巡畏怯。弗能大飽其欲。惟這韓氏不畏怯。他常來嫖這韓氏。兩人正是天生美對。蓋君祿之陽具旣雄。便不覺他的深鬆。況他是個無妻的光棍漢。得遇婦人之物。那裡還擇好歹。韓氏之寒與水。彼皆不較。惟取其勇於受敵而已。二人甚是相厚。一個願娶。一個願嫁。但蓋君祿心雖要娶。却囊中無物。不能替他贖身。他的老鴇見女兒主顧甚少。要將他轉賣。央煩媒人尋覓售主。這媒人就是替李自成說親的那人。這媒人想了想。笑道。我把這件美貨總成了這呆孽障罷。遂向韓氏道。你媽如今要賣你。我想你門戶人家的女兒。不是賣去仍做此事。便是與人做小。如今有一個好人家却是娶正妻。我總成你去受用。只是一件。若是男人問你他那東西叫做甚麼。你咬牙根只說不認得。要緊要緊。又將先那兩個女子的事向他說了。韓氏笑着應允。這韓氏心雖戀着蓋君祿。奈身不能自由。暗暗同蓋君祿商議。等嫁到李家之後。叫他假認作表兄妹。可常常來往。得空以遂私情。蓋君祿喜諾而去。再說那媒人來向李自成道。這一回實實尋着個眞女兒了。模樣又好。却財禮要厚。李自成滿心歡喜。一心要娶。他父親是不敢拗他的。娶了回來。成親之時。李自成弄了進去。韓氏全然不覺。見他在肚皮上一動一動的。知是弄上了。裝出許多的苦楚樣子。叫疼叫苦不休。李自成以爲是眞。連忙拔出。韓氏還叫苦不住。李自成道。我已拔出來了。你如何還叫疼。韓氏道。我是眞正女兒。你的太大了。我空着還是疼的呢。李自成越加歡喜。過後把陽物問了他幾十次。他只說不認得。李自成暗道。這纔是個好女兒。因笑對他道。這叫做雞巴。那韓氏暗忖道。好的我不知見過多少。稀罕你這個雞巴。忍不住失笑。李自成問道。你笑甚麼。他不答應。問之再三。他含笑說道。我長了這樣大。今日纔知道叫做雞巴。我往常當是男女一樣。原來是恁個樣兒麼。李自成愈加歡喜。十分恩愛。原來韓氏做妓女時。李過也曾嫖過他。他兩人頗有情愛。李過恐叔叔見了佔了他的去。不曾與李自成知道。所以李自成不曾見過。自從韓氏嫁了過來。二人一見。都是舊相識。豈不認得。但韓氏是嬸母了。李過不敢提起舊情。這韓氏因李自成物旣微而本事又不濟。有個溫溫舊帳之意。一日早起。李自成還在睡覺。韓氏張見李過在後院背着臉溺尿。他悄悄走到後面。伸手去將他陽物一捏。李過倒嚇了一跳。回頭一看是他。嘻嘻的笑道。你如今做了嬸娘。高枝兒上去爬了。還肯想着他麼。韓氏摟着他親了個嘴。一手攥住陽物。說。〔沒〕良心的。我當日同你何等相厚。你要我的陰毛做表記。我還拔了一大把送你\footnote{一大把。趣。方見其毛之多。昔一人嫖妓。費將千金。臨別。謂其妓曰。我同你相厚一場。你將陰毛送我兩根做表記。妓即拔而與之。別後。此人一日想起。將紙包打開一看。忽被一陣風吹去。其人滿街亂跑尋覓。忽見一皮匠正上鞋幫。口含兩根猪鬃。其人誤認。大罵曰。我費了千金。纔買了這兩根毛。你拿來含在嘴裡。李過得了韓氏這一大把毛。若被風吹去。若遇見賣刡子者。定然寃罵一場。呵呵。}。我來了這些日子。你竟不睬我一睬。李過道。我如何敢忘你。巴不得同你親厚呢。一來不知你心中如何。二來我叔叔性氣不好。怕他知道。你旣有此好情。我有個妙策。我今日哄叔叔到外邊去。灌醉了他。夜間等他睡熟。你可到我外邊來。便可成就好事。韓氏喜諾。此時一腔火氣本要洩一洩。恐李自成起來。只蹲下身。將他陽物含住。咂了幾咂。各自散了。這日。果然李過同李自成出去。抵暮爛醉。李過扶了他回來。進房放了他睡下。他家是三間正房。東屋李守忠夫婦住。西屋李自成住。李過在堂屋中打鋪。到了夜間。韓氏見李自成沈睡。悄悄開了房門出來就敎。二人多時未會。且韓氏這些時被李自成弄得不疼不癢。淫情蓄到十分。今日相遇李過。一團鬱火全要洩在他身上。一度不已。兩次不休。足足搗了大半夜。怕李自成醒來。只得分開。如此者多次。李守忠夜間也聽見了些聲息。恐兒子性氣凶狠。不敢做聲。推聾裝啞。任他二人快樂。那韓氏是做妓女的人。有何厭足。自嫁到李家來。那蓋君祿依他前策。假認做表兄。常來探望\footnote{昔有一吏與門子私語。官見之甚恚。吏曰。他係書辦之表弟。語家務耳。官因出一對。曰。表弟非表兄之表子。能對則恕之。吏對曰。丈人乃丈母之丈夫。官笑而釋之。此則是表妹乃表兄之表子。當以何對之。}。李守忠夫婦一來有年紀了。照管不得許多。二來也以爲他們眞是兄妹。並不防閒。那裡知他們裡頭有彎兒帳。李自成是遊手好閒的人。時常在外。那蓋君祿同韓氏得空便敍起舊來。時常做那鳳倒鸞顚鴛鴦交頸的事。一日。他兩人正在房中高興。不意李自成同李過撞了回來。見房門關着。推開進去。一眼看見那蓋君祿正在將完未完酥麻的時候。一見了他。越發嚇軟了。動不得。竟癱在韓氏肚子上。李自成大怒。腰中拔出短刀。將蓋君祿肋上背上幾刀戮死。韓氏嚇得發昏。生了個急智。連道。殺得好。殺得好。他竟強奸我呢。李自成怒道。旣是強奸。你爲甚麼不叫。韓氏道。我要叫來。恐鄰舍家聽見。丢了你的面皮。李自成明知是假話。心中本捨不得殺他。又値李過在傍邊。也恐李自成殺韓氏。聽了這話。一把攥着李自成的手腕。說道。聽嬸娘的話。與他不相干。不要屈了人。就將刀奪下。李自成借這意兒。也就鬆手。只將韓氏打了幾拳。把陰戶狠狠的擰了幾下。那韓氏擰得亂叫。李過看着心甚害疼。忙勸住了。李守忠聽得鬧。走了過來。見奸夫殺了。不曾殺媳婦。他當年曾在衙門中站過。知道事體。向李自成道。你單害了奸夫。是要償命的。你旣捨不得殺媳婦。你在家中住不得了。孫子在傍見死不救。到官也有大罪。你叔姪快快逃躱出去。我替你們擋官司\footnote{眞是老牛䑛犢。}。遇有恩赦。再圖歸計。那李自成也顧不得父母了。忙捲行李。要了些盤費。同着李過逃往甘州去了。李守忠同地方上報了官。知縣追問他兒子的去向。他說。殺人之後。避罪在逃。不知何往。知縣怒道。人殺在你家中。你明明縱子行凶放逃。如何賴得。命將他監禁。要他兒子。韓氏無辭抵賴。打了二十板。發與官媒領賣。仍是那舊鴇兒買回。又吃舊窩邊的草去了。那李守忠此時要有幾百銀子上下打點。也還可以保得沒事。因一分家私被兒子花盡了。力不能爲。又因有了年紀。到了獄中。心裡旣記掛兒孫。衆人知他當日在衙門中掙了一股大錢。不知他是空了。只疑他捨不得。又遭了些磨難。心中氣忿。不數日而亡。生了這樣個好貴子。一日不曾受享其福。先帶累了老〔子〕拖了牢洞。那老婆子見丈夫死在牢中。兒孫逃得不知去處。\endnotemark[1]又不知何年何日纔得回來。媳婦又官賣了。孤孤悽悽。回想當時在䘕衏中何等熱鬧。若不圖生貴子。今日仍當一個老鴇。安得寂寞如是。悔恨當日誤聽番僧之言。一至於此。憂憂鬱鬱。不久吿斃。他家親人只有李過的姑娘是他們的親姪女。主持着將房子賣了。把他夫妻埋葬。再說李自成叔姪東逃西躱。數月身〔無〕所歸。那時流賊蜂起。他也就入在黨內。你道那時天下奠安。流賊之起。始自何時。一旦就遍於陝右。此賊衆因起於裁驛夫。驛夫之裁。倡於御史毛羽健。成於科臣劉懋。你道他二人是何來歷。因何事故便釀成了國家這樣大禍。他二人是兩姨兄弟。俱是南京人。毛羽健的父親字曰毛褒\footnote{老子的名字叫毛包。不意他令愛的此道也是毛包。奇。}。倒也是個世代科甲。生了一子一女。子即羽健。女即阮大鋮之妻也。這毛褒中過一榜。做了一任敎官。後陞浙江湖州府烏程縣〈縣〉知縣。他一個姐姐嫁了韓門。姐夫早亡。只有一個外甥。名韓繼壽。毛褒將他母子二人帶往任所。這劉懋是他兩姨之子。幼無父母。也帶了他來。因是老婆面上的親。待他如同親兒一般。此時韓繼壽已十八歲。毛氏十六歲。劉懋十五歲。毛羽健十三歲。倒都如親兄弟姐妹一般。這毛氏同毛羽健姐弟二人。生得一個模樣。女雖不比王嬙。男雖不如宋朝。都生的粉團〈似〉也似的一個白臉。淸淸秀秀。稱得起一個俊男美女。就是那韓繼壽劉懋。俱生得乾淨可觀。不似那三家村放牛的牧豎。他三人同窗讀書。劉懋羽健兩人夜間又同榻。這韓繼壽年紀大了。知識大開。就看上了表妹。毛氏雖十六歲的女孩兒。他天性中帶來的有一種淫念。而且骨頭中又生滿了騷髓\footnote{此奇語也。從未經人道過。}。自以靑春二八。這瓜該破得很了。見父母尚未與他擇婿。他便暗暗相中了表兄。要把這瓜叫他破一破。那韓繼壽日日上來看母親。他兄妹各有私心。遇着無人處。便打牙犯嘴。互相調笑。打得火熱。初則口皮頑戲。後來偷空竟肚皮相貼。便成了那件風流事。也偷了多遭。那瓜已成了兩半。久之。毛褒也知道了些風聲。說不出口。在毛褒的意思。也想學賈充的故智。將錯就錯。把女兒配與韓繼壽。不但遮了醜。且完成他一對少年心願。不想韓繼壽一日正同毛氏在床上放着帳子高興。正做到妙處。誰知一個貓攆老鼠。從頂篷上掉了下來。剛剛跌在銅臉盆上。噹啷一聲響。把個韓繼壽嚇得一攛。從毛氏肚皮上直滾到地下。他一個少年人。血氣未定。正在斲喪之時。受了這一嚇。便得了個心悸的病。或坐或臥。即飮食之時。聞得微有聲響。猛然一驚。跳得老高。百藥無效。遂成怔忡而死。他母親只此獨子。痛哭是不消說。毛氏也不禁悲慘。暗暗飮泣。這一節事。劉懋毛羽健也都知道。一夜。他兩人同臥着私語。劉懋道。世間事也奇怪得很。一個男人一個女人。人生面不熟。只把這一段肉送到肚裡去。便親熟得了不得。你看韓表兄同表姐兩個那般親熱的樣子。還了得麼。你年小不覺得。我常冷眼看他兩個眉來眼去。好不肉麻。我想你我兄弟兩個。要是把我的送在你肚子裡。你的再送在我肚裡。豈不更加親厚。他兩個雖厚了一場。韓表兄生生的嚇死了。要是我兩個厚起來。一些驚怕也沒有。豈不長遠快活。毛羽健也高興起來。笑道。旣如此說。你先給我弄弄。我也給你弄一下。劉懋道。我比你大。自然該先讓我弄起。毛羽健道。先後總是一樣。就讓你先來。劉懋將他扶起。伏在枕上。也學用了些吐沫\footnote{不知自何處學來的。}。弄了進去。問道。你覺怎樣的。毛羽健道。不覺怎樣。只悶杵杵脹得慌。劉懋弄了一會下來。毛羽健也照樣去弄。他年紀小。陽物如指。不知不覺就弄了進去。也抽了幾下完事。他兩個睡下。相摟相抱。親嘴咂舌。親親密密。勝似夫妻。權且按下。那時溫體仁尚不曾入閣。還是尚書。他〈孰〉是烏程人。此時因吿病在家。他有一個女兒。生得貌甚不揚。他一心要選一個美婿。本縣中宦家子弟雖有。皆不中他的意。一日。偶然見了毛羽健。他便十分心愛。煩人對毛褒說要他兒子爲婿。毛褒見一位尚書要同他做親家。心中雖十分私喜。嘴中連說了幾個不敢仰攀。溫體仁再三央人來說。不計品級高下。家貲貧富。只要圖個好女婿。毛褒眞出望外。就忙忙去拜謝了。毛羽健已十五歲。溫體仁要他當年完姻。毛褒也一諾無辭。原來這溫小姐貌旣陋而心更淫妒。已十九歲了。嫁時妝奩之富。是不消說得。賠了八房家人。八個丫頭。八個小廝。到了署中。竟把他的縣衙塡滿。毛羽健見他的媵嫁那些婢婦侍奉小姐那尊貴的樣子。由不得就勢怕起來了。臥在身傍。心膽畏怯。況他與劉懋親厚已久。身在此而心在彼。捱過了幾日。便躱往書房中。同劉懋共宿。這溫小姐自以爲尚書貴女。必定嫁的顯宦之子。方稱佳配。不想嫁了個知縣的乃郞。那知縣署中寒酸的樣子。如何入得尚書小姐之目。心中十分不悅。因見毛羽健淸秀可愛。比自己尊容強了許多。還略有可解。況且毛羽健同劉懋幹慣了的後庭。頗知交合中的奧妙。溫小姐因此將就罷了。不想纔得嘗到趣味之時。忽然見他出出(去)睡。疑必有故。他的乳媼丈夫也姓溫。是溫體仁遠房族弟。因家中貧窮。典身到他家來做乳母。有兩個兒子。一個名叫溫世幸。纔十四歲。生得齒白唇紅。伶俐乖巧。溫氏着實疼愛他。出進不忌。那夜叫他去打聽姑爺在外邊做甚事。溫世幸出來。見人靜了。就蹲在書房窗下竊聽。聽得床上二人笑語。一個道。你好沒良心。我兩個好了二三年。今日你得了新鮮美物。偏我去受用。就把我忘記了。又聽得姑爺道。我怎敢忘你。他新來乍到。我脫不得身。故此今日纔躱了出來陪你。此後聽得二人氣喘吁吁了一會。那一個道。你同新人弄。大約比這個還快活了。又聽見姑爺道。雖然又是個味兒。但我有三分怕他。弄得一點興頭也沒有。以後便不做聲。又聽了一會。只聽得酣呼鼻息。知是睡着了。上來回小姐的話。見臥房門已關。不敢去敲。立在窗下。時已三鼓。月色正午。丫鬟們都睡熟了。溫氏心中氣惱。不曾睡着。二則也等溫世幸的回話。見窗外有個人影。知是他來了。披衣而起。即走來開門。一看。果是溫世幸。遂叫他進來。悄悄問他。那小子從頭細稟。溫氏知是他表兄弟二人幹那樁事了。不勝忿恨。怒道。他旣如此無恥。我也可以效法。遂叫溫世幸上床。脫衣共寢。原來這小子也常同人幹後庭。他那根厥物比毛羽健的還強壯些。且進退有法。分外在行。溫氏甚覺得意。事畢之後。悄悄放他出去了。此後得空。不時寵幸\footnote{所以名溫世幸也。}。次日。毛羽健進來。溫氏不似往常。便另是一副面孔。同他話也不說一句\footnote{淫婦心腸另是一種。自己同小子弄前孔則無羞愧。丈夫同人弄後庭則發怒。摹仿入神。}。晚間到了床上。溫氏把昨夜小子聽的話說了一遍。道。你也是個宦家子弟。做這樣下流無恥的勾當。還想來同我沾身。把個毛羽健羞得要死。此後夜間再也不敢離他。他只好日間在書房中同劉懋敍敍舊情而已。這毛褒做了十多年的窮敎官。陞了個知縣。烏程地方頗富庶。他貪婪無比。將地皮幾乎捲盡。被上司廉訪着了。參他個貪酷。幸虧得溫體仁在內替他一力維持。只革職回籍。到了家中。阮大鋮的父親知他宦囊富厚。聞得他女兒又標致。要求了爲媳。阮大鋮同毛羽健劉懋同案進學。見其弟美。知其姐姐必佳。心中也喜。那毛褒雖知他乃愛的鮮花已被採過的了。沒有個將破女兒養在家中一輩子的理。聽得阮家求親。欣然允諾。他娘恐女婿試出不妙。甚是憂心。南京人有個惡俗。嫁女之夕。岳母交一幅白絹與女婿取元紅。他娘知女兒是久沒這件的了。絹幅不敢交與女婿。弄了些紅花水。希圖臨上轎時染得斑斑點點。與女兒帶在身邊。詐充了去。不想那日他家因備喜筵。染紅綠果品。剩了一碗槐花水。丫頭們看見那碗紅花水。也以爲是剩的。就放在一處。毛氏的娘再三囑咐他道。你自己做的事自己知道。夜間成親時須要十分遮掩。倘被女婿〔看〕出。不但父母無顏。你一輩子也擡不起頭來。毛氏點頭會意。到了上轎之時。他娘去染那白絹。不暇細看。放在碗中蘸蘸。誰知蘸的那是碗槐花水。忙忙遞與女兒藏了。阮大鋮成親之夜。去脫毛氏的衣服。他那裡肯。死死的攥住。阮大鋮先見他新人貌美。已心愛情急得了不得。此時不過以爲他室女害羞。再三替他強脫。毛氏被他纏了一會。一來也有些興動。二來前後總免不得。成敗在此一舉。也就任他脫去。到了交合之時。他做出萬分艱難之態。也不像行房。竟像剮他一般。那叫苦畏避。眞說不出\footnote{吃了他令堂敎導的虧。俗所謂敎的曲兒唱不得。}。阮大鋮倒反動疑起來。道。我也聽見人說過。女孩兒破身雖有些痛苦。那裡就到這樣地位。事畢之後。拿起喜帕一看。恰合了古詞上的兩句。道是。

\begin{quotation}

不見不見。還你一方白絹。

\end{quotation}

他這帕上不但不見點點鮮紅。而且東一塊西一塊。全是黃斑。阮大鋮大怒。罵道。沒廉恥的淫婦。你同甚麼人私偷。不知弄過了多少回數。今日矯揉造作。裝這個樣子來哄我。起來穿了衣服。快快替我回去。我不要你這樣淫賤婦人。那毛氏尚有何辯。赤着身子下床跪着哀求。道。是我一時不長進。做了壞事。如今旣到了你家。求你開恩。包涵了罷。只容我占個正室的虛名。以全兩家體面。要娶妾討小。任你尊意。你這一攆了我去。不但我一生不得人。連我爹娘的臉面都沒了。你只當積陰〖阝少日小〗(隲)罷。阮大鋮見毛氏雖非處子。心中固惱。因毛褒知道女兒內中的東西破壞不堪了。把外邊的東西賠了個十分成文。約有數千金。阮大鋮自幼貪婪\footnote{毛氏是骨頭裡面帶來的淫髓。他也是骨頭裡帶來的貪癖。}。他心中想。這一攆了他去。果然兩家都不好看。且這些妝奩斷無留下之理。少不得仍要還他。豈不可惜。況毛氏生得甚美。赤身跪在地下。像一個粉妝成玉琢就的人兒一般。臍下那條細縫。內中雖寬濶了些。而外面鼓蓬蓬。甚覺可愛。心中就動了幾分憐惜。只見毛氏家來伴姑娘的一個老僕婦推門進來。道。姑爺。你兩口子今晚百年的頭一日。不歡歡喜喜的睡覺。吵鬧些甚麼。見毛氏精光的跪在地下。說道。可憐。可憐。我家姑娘一個嬌生慣養的閨女。你忍心這樣作賤他麼。阮大鋮冷笑道。你家姑娘好個閨女。那東西被人弄得像皮袋似的。是個閨女的媽了。那婆子道。阿彌陀佛。姑爺不要枉口白石(舌)的。我家姑娘同奶奶娘兒兩個終日唇不離腮。那裡有這樣的事。不要屈了人。阮大鋮將那帕子撂與他。道。你看看你家姑娘的喜帕。他接過來。燈下一看。許多黃跡。半晌說道。哎呀。這是怎的來。姑爺。想是你太狠了些。把姑娘的苦膽弄破了罷。阮大鋮又好笑。又好惱。那老婆子也跪下。道。姑爺看我的老臉面。將就些罷。就是眞正黃花女兒。方纔經你這一下。也就破了。你只當是你弄破的。也就不氣惱了。那喜帕上管他是紅的黃的。也不過頭一次有一兩點子紅。後來都是白的。你也只當是弄第二次。還氣惱甚麼。我記得我當初嫁老伴兒的時候。倒是眞正女兒。頭一回一點紅星兒也沒有。他也並不曾說甚麼。姑爺。我勸你息息怒罷。阮大鋮一來聽了他這話。不由得好笑。二來他的心先也就有些回了。見他苦求。借意兒也就收科。向毛氏道。他老人家旣這樣說。我且饒過。你在我家。若再有絲毫錯處。那却休怪。起來罷。那婆子連忙站起。扶起毛氏。一面替他披上衣服。一面說道。姑爺好說。我家姑娘年幼。一時間做錯了。那裡有個只管錯的理。哈哈的笑了一聲。向毛氏道。你這樣小小年紀。那裡這樣順便的食就撈到口裡。我活了七十多歲。還沒有遇過這樣巧宗兒呢。毛氏又羞又氣。把他盡力一搡。那婆子一路跌去。幸得門枋子扶住。說道。我好意來勸鬧。你倒幾乎把我推跌死了。咳咳〖口敕〗〖口敕〗。走了出去。過了兩年。阮大鋮毛羽健劉懋鄕試同中了。次年。又同中了進士。選了庶吉士。後來毛羽健得了御史。劉懋得了戶科給事。阮大鋮得了工科給事。這毛羽健同劉懋不但是兩姨弟兄。而且彼此又是後路夫妻。契厚得了不得。今到了宦場中。凡事彼唱此和。兩人一心。那時陝西有些飢民作亂。特差毛羽健去監察着撫鎭剿撫。他到了陝西。沒有管頭了。他受了醜妒婦人多年的挾制。今日始得自由。娶了一個美妾。嬖愛之甚。他的那些家人多是溫家的媵人。素常只知有主母。不知有主公的。況此事可敢隱瞞。當新聞一般報知溫氏。溫氏在家有溫世幸做了寵童。毛羽健雖在可有可無之間。但醋氣難按。一聞此信。帶了溫世幸同家人婢婦。星夜乘船而來。沿途聽得是欽差監察御史的夫人。敢不應命。也不及報聞羽健。溫氏到了署中。方纔知道。美人藏匿不及。只得相見。溫氏作了一場威福。將那妾立刻遣出。毛羽健見溫氏來的速。不及預防。心中恚甚。不敢怎樣夫人。遂遷怒於驛遞\footnote{古謂。怒其室而作色於父。此羽健之謂。}。倡爲裁驛夫之說。特疏啓奏。謂驛夫一裁。一年可省帑金數十萬兩。崇禎發九卿科道會議。衆人皆以爲不可。而劉懋現在戶科。一力舉成。謂毛羽健爲國省費。竟奏准了。驛遞一裁。閒人千萬。倚驛遞爲生者無從得食。相率爲盜。遂致滋蔓。闖賊得以招集之。流毒中夏。那覆宗夏。兩人首禍。萬死不足贖。而實酸於一婦人。女禍之酷。伏於枕蓆。可不懼乎。且說李自成他生來有些膂力。性子又莽戇。膽子又大。到處爭先。所向常勝。先還是個強盜中的大哥哥。後來兵馬多了。聲勢衆了。就公然稱起王來。他說項羽當年自稱爲霸王。他因自己混名叫闖子。竟自尊爲闖王。那時天下奠安了二百餘年。將不成將。兵不成兵。他帶着賊衆。從不據地方。只流來流去。故此人稱他流寇。他到州城府縣。只搶擄殺戮一番便走。把些城池被他攪得粉碎。各省親王宗室。以及文武官員。兵民老幼。被他殺得幾無噍類。且把他的惡處略說幾件。便知他的萬惡。同那時人民的苦楚了。他破了鳳陽。殺戮之慘。天地皆黑。或縛人的父親丈夫看着。叫人淫他的妻女。淫過了纔殺。或拿着人父。使淫其女。以爲戲笑。然後殺之。或把懷孕的婦人脫光了。大家賭猜他腹中是男是女。以爲輸贏。拿出紂王的陳樣來。割腹驗看。一試不中。又剖一個。一日之內。這些孕婦死得不知其數。又將火鍋煮油。把小孩子撂在內中。看他跳躍啼號。頃刻化爲枯骨。以爲笑樂。又將〔人〕縛在地上。生刳其腹。裝上米豆。餵他的戰馬。又取人血和米麥煮粥。以飼馬騾。使他腹壯而能衝敵。擄來的子女千百。臨行不能帶去。盡皆殺了纔去。或攻城之時。把殺了的人間着蘆葦薪木。堆在城下。縱火焚燒。那穢氣煙焰薰逼城上守禦的兵卒。無不仆倒。他陷鳳陽之日。劉(留)守朱國相同兩個姓陳的千戶忿戰而死。別的文武官員死的死了。走的走了。逃個乾淨。把皇陵樓殿燒個灰燼。燔松三十餘萬株。殺守陵太監六十餘人。縱放高牆有罪的宗人九十一名。焚留守公署司府廳五百九十四間。焚鼓樓龍興寺六十七間。燬兵民廬舍二萬二千六百五十二間。知府顏容暄囚服避在獄中。被賊搜出。先杖而後殺。並殺同官六員。武官十一人。殺生員六十六名。殺陵牆班軍二千二百八十四名。殺高牆看軍一百九十六名。殺精兵七百五十五名。殺操軍八百名。圍六合縣時。把小孩子聚上數百。四週圍堆上柴木。放起火來。聽其哀號。觀其奔逃。少焉俱死。臭不可聞。以爲暢快。攻城之時。將婦女們千百尋(成)羣。脫得精光。向城大罵。婦女稍有羞愧。即亂刀刴在城下。攻破六合之日。聚城中兵民將要屠殺。忽有令免死。每人刴一手。衆人大喜得饒命。爭先伸臂。沒一個叫痛苦者。故六合的沒手者甚多。他刴手則不殺。刴的時候。伸右手與他刴了便罷。若先伸左手。刴去了。仍要刴去右手。你道他慘毒不慘毒。他攻破江浦。一日早間。他把一個婦人在東門外寸磔。原來這婦人被擄。李自成要淫汚。被他把臉打破。李賊恨他不過。不令他速死。故碎磔於城外。對衆以辱之。待我把這烈婦的事蹟表白一番。也顯一顯他的貞烈。賊破江浦。進城之時。有一個小賊頭姓獻名勤。因他生得身粗項短。綽號叫做縮頭龜。他到了一家。見一個美婦正在那裡上吊。他上前解救下來。那婦人痛哭罵道。賊奴。你不殺我。解我做甚麼。縮頭龜笑道。大王爺正要尋個美人取樂。傳下令來。道有獻美人者受上賞。你這一去。定有造化。我也有重賞。那婦人罵道。萬剮的賊奴。我一個淸白良婦。豈肯從賊。你快殺了我便罷。縮頭龜要去拉他的手。那婦人哭罵着。一頭向地下要撞去。縮頭龜眼快。搶上前一把抱住。那婦人千賊萬賊的罵道。我一個淸白之軀。你敢拿賊手來汚我。那縮頭龜由他罵。兩手扯住了他兩隻手。叫兩三個小賊在後面推的推。〖扌送〗的〖扌送〗。到李自成的處所來。李自成在縣署中住着。正擄了些婦女來。在那裡飮酒作樂。看那一羣女子並無一出色人物。都不中意。忽聽得報說獻勤獻功。得一美女。滿心歡喜。叫快些進來。遠遠見三四個人推着一個女子。獻勤拉着。雖然頭髮散亂。滿面淚痕。那一種風流標致。自不能掩。到了跟前。獻勤方放了手。那婦人便坐在地上哭叫道。賊奴。你快殺我。你快殺我。我不順汝。李自成滿臉堆笑。問獻勤道。你是那裡得的這件活寶貝。獻勤忙跪稟道。臣無心到了一家。這婦人正在那裡上吊。臣見他生得好。特救了下來。獻上大王。李自成大喜道。妙哉。妙哉。你出去聽賞。那獻勤叩了個頭。道。謝大王爺。走了出去。那婦人不住聲只是哭罵。李自成笑道。美人。你不要破口。我今日得遇你。也是前緣。你姓甚麼。那婦人罵道。潑賊。我一個淸白姓字。怎肯對你賊說。你是何等賊奴。敢向我說個有緣。你快殺了我便罷。李自成有了些酒興。心愛極了。任他大罵。也不動怒。笑道。你不要呆了。你從了我。享用天大的富貴。孤家後來得了明朝的天下。你就是一位貴妃了。可不好麼。那婦人道。你這賊。明日被天兵拿住。碎屍萬段。身子不知餵豬餵狗。你敢妄希天位。還想甚麼富貴。你這樣淫惡潑賊。上天也不容你。李自成和顏悅色的道。美人。氣是好忍的。你罵也罵夠了。今日我同你成了好事。包你就一點氣也沒有了。向衆婦人道。替他換了衣服。梳洗了來吃酒。那婦人道。賊奴。我梳洗的是甚麼。換甚麼衣裳。坐在地上。那裡肯起來。李自成道。不梳洗也罷。你們扶他起來。過來坐着。衆婦上前攙住。那婦人是個嬌怯女子。如何拗得過。被衆婦女擡了起來\footnote{擡字。妙。活畫出一烈性婦人樣子來。}。要他近桌子。他那裡肯。只亂掙亂扭。李自成見衆婦人拉不過來。便親自起身。要伸手去拉他。那婦人見他來拉。忙把手一縮。柳眉剔豎。粉面通紅。喝道。賊奴。不要無禮。你不殺我麼。罷了。看見傍邊一個婦人手執着一把金酒壺。他猛力掙脫。一手奪了過來。夾李自成劈臉一下。那闖賊不曾提防。被他打個正中。面上的〔血〕打得直流。壺中的酒淋淋漓漓弄了一頭一身。李自成大怒。罵道。好潑婦。敢來打我。喝叫一聲。綁去砍了。兩邊帳下親隨答應一聲。上前綁定。正要帶了出去。李自成道。這惡婦若是一刀。便宜了他。明早剝得精光。到城外東門橋上碎碎的割他。叫萬人看他的巴子。辱這惡婦一辱。纔出得老子的這口惡氣。那婦人不哭了。反大笑罵道。惡賊。你就對衆剝光辱我。我得一死。便顯淸白之軀。這有何害。我生不能啖汝之肉。死當追汝之魂。李自成叫帶去監守。明日行刑。衆人將婦人帶去。次早。在橋上剮的就是此位烈婦\footnote{古人稱罵賊者。僉曰顏常山張睢陽。看此烈婦。又何遜於二公。}。李自成叫取過鏡子來一照。看見臉上打破一塊。血流滿面。一時忿恨起來。遂遷怒到獻勤身上。喝令叫獻勤來。那獻勤正等着領賞。聽得叫。他忙欣然走入。李自成罵道。這樣的潑婦人。你獻他來做甚麼。把我大王的臉都被他打破了。好生可惡。綁出去替我砍了。衆人一擁上前。綁出門外。一刀兩段。把一個獻勤的縮頭龜弄做了個齊肩斷頭鬼\footnote{獻勤的看樣。}。有一首打油道那時亂離的光景。不勝酸鼻。

\begin{quotation}

萑苻寇起弄干戈。兵火盈城布網羅。

宋子齊姜遭玷辱。亂離情景可如何。

\end{quotation}

\footnote{此與宮人紅袖泣。王子白衣行。一樣淒楚。宋徽宗在五谷城。一日偶到一酒肆。見一番婦領一女子。各席唱曲要錢。番婦稍遠。那女子問道。官人像是東京人。想也是被擄到此了。徽宗點首。亦問道。你是誰家女子。被陷至此。那女子泣下答曰。我慈懿太后姪女也。不幸至於此地。一天子一太后姪女遭亂離至此。又何況於閭閻之女耶。}\endnotemark[2]再說李自成殺了獻勤。坐了一會。氣略消了些。把這婦女中選了一個。拉到床上去同睡。他的陽物本來渺小。此時又着了氣惱。其軟如綿。硬不起來。叫那女子去咂。那女子尚是個處女。羞愧難當。方纔看見那婦人的一段烈性。也就感動了幾分。心中想道。同是一個女身。他便那樣激烈。視死如歸。我們此身何苦爲賊所辱。不過是一死。有(何)足懼。想到此處。倒不羞了。縮下身去。一把攥住陽物。就放入口中吮咂。想道。我一下咬掉了他的。這賊死了。替衆人除根。也不枉一死。遂下力咬了一下。一來他小女子心慌膽怯。二來要是硬或倒咬斷了。因他是軟皮。不曾咬斷。只咬了幾個牙齒血印\footnote{古云。聖天子百靈擁護。他是一個草頭王。也定有凶神惡鬼爲護衛。陳守備一箭。只射瞎一隻眼。不能貫顱而死。何況此小女子。能死之耶。}。李自成痛入心髓。把那女子一脚踢下床去。心中恨極。床頭拔出腰刀。一揮兩段。一連數刀。砍做幾截。可惜這兩個貞烈婦女。失傳他的姓氏。李自成忙拿刀瘡藥擦了陽物。養息了數日。方纔起兵而去。賊退後。土人憐他二人之節。甚敬之。因不知其姓氏。不敢報官表請旌獎。只私建了一祠。額曰雙烈以俟(祀)之。此二女較明朝降賊諸臣。寧不啻天淵耶。後來闖賊領衆攻打汴梁。自己扮作游騎。雜於衆賊之中。到城下來覘探城池的高深。有官兵認得他模樣的。指說與總兵陳永福的兒子。他素稱善射。暗發一箭。射瞎了他一隻眼。此後人纔稱他李瞎子。他攻破洛陽。殺了福王。將王肉同鹿肉煮熟了。又將王血同鹿血和酒。宴飮衆將。名爲福祿宴。闖賊巡營嚴密。部下再不能逃。有逃走者謂之落草。拿回寸磔。他連營百里。竟日不能過。所以再逃不脫。禁衆賊不許藏金銀。私帶者斬。精兵許帶妻子。生了兒女。不許留養。每人許收男子十五以上〔女子〕十四以下爲使從。爲之打草餵馬。安營下寨。汲水煮飯。照管馱橐。多者三四十人。至少者也有十數人。過城市不令住屋。總在帳房中居住。一名賊兵要好馬三四匹。冬天用綿褥墊着馬蹄。恐其怕冷。剖人腹用爲槽。故此他的馬鋸牙如虎豹一般。到處下營之後。即令兵士射前。日晚方罷。每夜四鼓都要飽食聽令。所過崇崗絕坂。飛騰直上。不許傍越。惟有黃河阻轡。許用船渡淮泗涇渭。衆兵翹足踞馬背。或抱鬣緣尾。呼風而前。馬蹄壅遏。水爲不流。淺不盈尺。步兵搴裳徑涉。臨上陣時。列馬兵三萬名三堵牆。前面者但回頭返顧。後面者即殺之。戰久不勝。馬兵佯敗。官兵一追。他預伏伉健步兵。飛鎗三萬。擊刺如飛。馬兵復回圍上。官兵則無孑遺矣。他攻城的號令一到即降。不焚不殺。守一日殺十分之三。守兩日殺十分之七。三日全屠。雞犬不留。殺了的人束其屍點灼。叫做打亮。攻城將陷。着步兵萬人周圍城下。馬兵巡哨於外。有縋城者一個也跑不出去。張獻忠每破城之日。尚留一面與人跑。到了這瞎賊破城。竟是俗語說。滾湯泡老鼠。死在一窩。各營將校所獲。美女珠玉爲上功。騾馬者受亞賞。得弓矢鉛銃者又爲次。瞎賊多覓蘄黃人爲奸細。或爲醫卜。或爲星相。或爲緇衣黃冠。或爲乞丐戲術。或爲挑肩買賣。或爲皮鐵雜藝。分布各處。覘探虛實。又沿途邀截赴京舉子。說透打合。爲之夤緣中式。以作內應。故此攻破城池的那日。雲合響應。一呼咸集。人都不知從何而來。他又叫人四處謠言唱道。

\begin{quotation}

開了門。迎闖王。闖王來時不納糧。

\end{quotation}

以此語鼓惑愚民。後來闖賊聲勢益張。朝廷密旨命陝西巡撫汪喬年查訪他親屬。米脂縣邊大受拿獲得李自成族人拷問。供稱他祖墳瘞地離此二百餘里。在萬山之中。聚塚十六。中一塚是他始祖。相傳此穴是仙人所點。有鐵缸點燈壙中。說道。鐵燈不滅李氏興。邊知縣親領人役到那墳上看了。叫人掘開。內有螻蟻數石。火光尚熒熒然。頗(剖)開棺材。骨皆靑黑色。黃毛遍身。腦後有錢大一穴。內有四寸來長一條靑蛇蟠在中間。頭上有角。見了日光飛起。高有丈餘。以目迎日色而吞咋者六七顧。眼射日尚不能開。復落了下來。邊知縣將那蛇烘乾並頭骨呈報。巡撫汪喬年又送到京中。上呈御覽。李自成之射瞎眼睛。舉事無成。還虧破了他這風水。崇禎十一年。經略洪承疇督師孫傳庭大破闖賊於潼關\footnote{李自成之在潼關。原張獻宗(忠)之在穀城。彼時若殺之。如屠一豕。竟縱之去。後皆不可復制。以致君亡國破。雖彼時督師之重臣愚庸誤國。然實有天意存焉。非人能謀也。}。自蜀之楚。往依張獻忠。獻忠不納。復走商雒。依老〖犭回〗〖犭回〗。在營臥病半年。病癒後。老〖犭回〗〖犭回〗授以百人。走穀房。會同諸賊。出文堦。此後不可復制矣。到了崇禎十四年上。風聞得流賊過了潼關。順河南一路搶殺而來。殺戮之暴。更甚當日。洛陽已破。福王被害。現今賊衆攻打汴梁。也就有許多百姓紛紛的攜妻帶子逃往南京來。那逃難來的衆人。好生傷慘。有幾句說他們。道。

\begin{quotation}

人民逃竄亂紛紛。覓弟尋兄。男婦慌張哭啼啼。抱兒挈女。父呼子。子呼父。悽慘堪憐。妻喚夫。夫喚妻。悲傷難聽。十室九空。村中並無居住之人。千辛萬苦。路上惟聞失家之慟。夜月淒淸。幾點靑燐照野。夕陽滲(慘)淡。數堆白骨塡途。風聲鶴唳。盡疑惡賊來追。膽戰心驚。惟慮微軀不保。正是寧爲平安犬。果然莫做亂離人。

\end{quotation}

各處居民都晝夜惶惶不安。一日數驚。那時天長六合江浦三縣。有十數個仗義的豪傑。一個姓慕名義。一個姓林名忠字報國。便是梅生姑母之子。一個姓尚名智。這三個又算衆豪傑中的巨擘。俱猛勇絕倫。智謀足備。因見時政日非。奸邪當道。素知朝廷專任太監。便不肯出仕。情願棲身草莽。他三人中。林報國更身長力大。膽壯心雄。自幼習學了一桿渾鐵鋼鎗。十分純熟。他生得豹頭環眼。虎鬚倒豎。令人望而畏之。他後來又遇了一個異人。傳授了兩口刀法。可以在萬軍中如入無人之境。你道他這刀法是何人所授。數年前。他有一個朋友要往京中貿易。馱了數千金貨物。聽得人說山東一帶路上到處有響馬土寇作祟。恐途間有失。煩他保護同往。他笑道。我常聽得沿途這些鼠賊坑陷過往客商。十分利害。都道他們手段高強。弓馬嫻熟。並無人與敵。我正要想去試試這夥盜賊的本事。看是如何。因未得其便。今趁此去會他們一會。遂欣然收拾了弓箭器械同往。一路平安無事。到京住了數日。賞玩了長安風景。欲整歸鞭。別了那朋友。假鋪宣武門外\footnote{俗稱爲順城門者是。}。將行前夕。忽値大雪。只見一美少年。披狐裘。佩雙劍。策蹇驢。倉皇投宿。其狀如美婦人。光艷奪目。甫入店。即呼主人家索燒刀子一斗一生彘肩爲餐。主家意多同侶。如數具之。及昏。無一人至。乃熟肉暖酒進之。少年拔劍切肉。豪飮大醉。須臾過半。林報國初窺其風流雋逸。心已暗異。及見其飮食粗豪。益爲驚怪。乃上前拱手。從容詢其姓名。問其行\endnotemark[3]狀。那笑(少)年注視良久。笑道。亦我輩中人。遂讓了坐下。說道。俺姓朱。無官名。乃山右太原人氏。我母夢神人授赤珠一顆。光照四壁而生我。因名珠兒。十歲就學外家。歲暮解館。遇白髯老人攝入深山。置萬仞懸崖之間。授飛走擊刺之術。期年。身輕如葉。可於屛風上行。水波上立。能飛劍斬人於五百步外。百發百中。年十三歲技成。仍送還家。時母已故。父爲豪家所賊。俺因痛忿。飛刺仇人於市中。自首於吏。吏受豪家金。欲致俺以大辟。因而遁跡浙東。與會稽貴公子姜堯相善。後吏以貪酷誅。俺遂歸省丘隴。而姜亦南遊臺雁。値山賊卒起。道阻不得歸。賊帥素知其材。欲強留之。姜堯不屈。謂賊道。吾父子受國深恩。恨書生力綿。不能操戈殺爾。寧從爾耶。若等逆天反叛。滅族之禍。翹足可待。而欲人陪戮西市。誰其肯之。賊帥怒。即縲絏軍中。罵道。俟吾先下兩浙。定江東。然後殺豎儒。俺今欲馳往救之耳。林報國道。彼旣陷賊中。將何策以拔之。珠兒舉劍示之。道。我有此君。賊雖多。其奈我何。語畢。遂滿引\endnotemark[4]邀報國共飮。報國道。我明早亦南旋。苟不棄。聯轡可乎。珠兒笑道。吾騎日走八百里。非君騎可及。且吾前途期會要客。尚多逗留。於中道相會。君可兼程而進。吾所宿旅舍。壁間必繪一鷹。下寫月日。驗之即知吾所過也。如不及。則於淮陰市酒肆中覓之。遂各就寢。明晨並轡出彰義門里許。珠兒於驢背上拱手道。吾先行矣。即策蹇如飛。轉睫失所。林報國日行百餘里。數日始抵高唐。見旅舍壁間果有繪鷹。讀其識。乃出都之夕也。詢之逆旅主人。云。畫鷹客\endnotemark[5]於此信宿。候其侶不至。已去八日矣。始信其八百里之言不謬。及抵淮陰。果於市中酒樓得之。握手大笑道。我候君兩\endnotemark[6]旬餘矣。今乃至耶。即呼酒共飮。報國心羨其驢。嘖嘖不置。珠兒道。君愛之乎。我與君易之。報國謝道。我何敢當。明日早起。與珠兒整轡同發。珠兒乘馬。報國乘驢。同出店門。驢竟不行。珠兒又躁不可待。乃於馬上語報國道。君不善乘。我不慣於汝乘。請先驅。於蜀岡相候。遂加策飛馳如電。報國見其去。若鷙鳥逐爵。勁弓出矢。不禁色然而駭。盡力加鞭。終不可及。乃信步而行。及抵江都。珠兒已於蕪城俟兩宿矣\footnote{蕪城在江都縣蜀岡上。}。因吿報國道。行道遲疾。存乎其人。非在騎也。果得其道。雖淹蹇疲乘。日可千里。況良騎乎。於是報國知其果有異術。再拜求敎。願以師事。珠兒識其誠。許之曰。吾受姜氏恩。今姜子爲賊困。急急欲往救。今則不能。大約在春燈之夕。當造君授之。遂別去。馳入賊壘。脫姜之繫累而出。賊帥遣鐵騎追逐。箭發如雨。不能中。珠兒復飛劍斬數十賊下馬。賊帥大懼而退。送姜堯歸會稽抵家然後歸。新正元宵。果至報國家中。報國拜之爲師。求授武藝。遂傳十八般兵器。於雙刀更極其妙。珠兒授之乃去。此時慕義林報國尚智三人。聞得流賊的消息。遂約齊了衆人。聚在一處商議。慕義道。我們沿江一帶。旣無深山老谷可逃。又無猛將雄軍可以禦敵。不是抛家棄業逃竄他鄕。就是妻離子散被賊殺戮。向年此地被賊殘害。慘不忍言。至今數載。瘡痍未復。我們如今不若在衆人之中。齊集好漢。自相爲保。與其東逃西躱。尚不能求生。不若盡力殺賊。在死中求活。衆位尊意如何。林報國道。這事非同小可。若行得來。不但上可盡忠報效於國家。下可竭力護庇於鄕黨。須要衆人努力同心。方可做得。若弄個虎頭蛇尾。豈只貽害身家。而且反爲賊笑。尚智道。這事我久矣有算於胸中了。但我們要分頭去做。行得來時。自然是妙的了。若做不來。趁早中止。再想頭路。衆人道。願聞妙策。尚智道。我們三縣不下有十數萬戶。十分貧苦的算不得。只將略殷實並可以稍有餘者。擇出三萬餘家來。十戶公養一人四季衣糧食用。每一人一年給以五十金。十家派來。每家五兩也不爲過。強如做賊來全全送他拿去。還要貼上妻子。這三千人却要操練嫻熟。激以忠義。每縣駐紮一千。如長蛇之勢。賊攻一處。兩下救援。只有死進。再無生退。智信仁勇嚴五個字。缺一不可。訓練了這一枝兵。都是精強力壯的。況又是父子兄弟。同心協力。如背指相連。豈懼他甚麼賊衆。岳侯以五百背嵬軍破兀朮十萬鐵浮屠。何況三千子弟兵不能敵數萬烏合之鼠輩耶。這些賊人。傳說他凶勇異常。因是那些畏刀避箭的將官。領着那從未操練的兵士。被他殺怕了。聞風膽碎。遇賊便逃。還聽得官兵常常全軍覆沒。並不是臨陣殺傷。都是見賊就跑。自相踐踏。死者過半。那跑不動者。或自刎。或跳崖。或投水。又去一停。所餘無幾。再被賊趕上一殺。故此就無孑遺。這些流賊從不曾遇着勁敵。竟也目中無人。以爲自己如何梟勇。前聞賊寇湖廣。以五百賊兵橫一大纜。敺漢陽漢口數百萬軍民男婦老幼自投於江。江水爲之不流。這幾百萬衆俯首就死。竟無一個奮槌一擊之人。故此他把官兵越發不足介意了。我們這些鄕勇。一年吃着衆人供給。又免了自己差役。況都是骨肉相連。不但爲了大衆。且要自保身家。若齊心協力。我輩親冒矢石。奮勇前驅。率領着衆人。痛殺他幾場。使賊聞名喪膽。魂夢皆驚。再不敢垂涎我們的這幾處地界。你列位道好麼。內中有一個姓國名守的。是林報國的妻兄。說道。兄籌畫得甚妙。但還有慮不到處。如今這些贓官汚吏。他見了賊固然會縮頭潛逃。見了百姓他却會任情魚肉。見了我們這番舉動。反要想起我們的錢來。是怎麼處。若要給他。我們做這番義舉。如何肯送錢與這些贓(賊)胚。若不給他。他倒誣賴我們要舉兵應賊。那纔有口難分辯。賊不曾殺得。他人不曾爲得。反先喪了身家性命。林報國道。兄說得有理。且還有一說。這三千人旣要操演敵賊。若無盔甲器械。如何行得。再製這些物件起來。越發驚人耳目。況且這一項銀子又從何出。難道又好在這三萬戶科派不成。尚智道。諸兄不必多疑。議論多而成功少。弟都早已安排定了。這都是後一着的事。一步一步往前進。如今只要這三萬戶肯齊心供給。果然內中挑得出三千義勇來。自然又有道理。衆人道。人都稱尚兄爲智囊。眞正不錯。我們依他主意。各人分頭行事。看人心向背如何。再做商議。尚智道。事不宜遲。可行不可行。都速來回信。好別做計較。衆人應諾。慕義回江浦。林報國回天長。都分頭而去。這尚智就是六合縣人。他家中親丁子姪也有二十多人。約有千金家產。他疏財好義。一縣盡聞其名。他家中把耕牛宰了四五條。殺了十數個圈內的豬。窨着的酒起出數十罎來。把合縣的鄕約(紳)保正總甲地方排年里長。並縣中有頭腦的些人。請了有百十多位。在場圃中席地而飮。飮酒中間。衆人問道。尚兄今日約我們這些人來。有甚麼話說。尚智道。我請了衆位來。有一件大事相商。當日我們這一帶地方遭流賊之害。到如今七八年了。還不曾復舊。縣中沒手的人將及一半。見之令人痛心切齒。近日見河南逃下來的那些男婦傳說這夥惡賊河南八府已殘破了七處。僅存汴梁未下。又想到這裡來搶殺。我想衆人沒有個坐着等死的。當年賊來倉卒。一時逃躱不及。被他殺害了多少。如今旣然知道了風聲。自然都想攜家小避難。就算逃得性命。賊去了再回來時。家中房產已成灰燼。所有家私糧食牲畜俱蕩然一空。倘或途中遇了賊寇。不但父母妻子被他殘害。而且自己的性命亦不能保。何況於所有之私蓄。如今我的愚意同衆朋友商議了。我們六合同天長江浦這三縣地方。是一條邊窵三犄\endnotemark[7]角。相隔都不遠。倒是可守可戰之地。我們在這三處挑選三千精壯。這三千人。每一人得十家供給。每年一家出銀五兩。十分窮的不在數內。却在這些窮戶中挑選精壯。免他丁役。我們挑足了。操練出來。三縣互相救應。儘力殺賊。不但替朝廷做了地方保障。又還保護了自己身家。且又報復前仇。你列位道好麼。衆人道。事是極好。但恐官府瑣碎。不是兒戲的。尚智道。鼓可是瞞着打得的。只怕衆人不肯齊心。若把底下明白了。少不得到上司處去稟明了方行。我們下邊的話未經說明。還不知衆人可戮力同心。冒冒失失先稟了上臺。底下一時做不來。豈不是欺弄官府。衆人道。尚兄想得週到之極。我們大家去商量定了。再來回話。尚智道。還有一說。列位總甲每位須製兩本冊。把那情願出供給的寫在一本上。那些窮戶中有精壯少年願出力的。也另注了姓名在那一本冊上。不妨多些。於中再加選擇。這是大家的義舉。且都是自己有益的事。目今人心俱在惶惶。只在列位說得委婉。大約事有可爲。却是強不得人的。衆人去了四五日。都來回信。道。我們合縣當年吃了流賊大害。近日聽風信。所過地方不但人口遭殘。連雞犬都不留。千里俱無人畜。衆人正在驚慌。聽了尚大爺這番作爲。也都願意。冊子都注明白。出供給的。城中連各鄕各啚。約有一萬餘家。有力量稍次的。我們將兩家並算一戶。窮戶中精壯少年。也有一千四五百願出力的。尚智心中大喜。道。只等他那兩縣的信來。果都像我們縣中這樣仗義。就大事可成了。等他們有回信時。我再通知列位。衆人別去。又過了三四日。慕義林忠都來了。道。衆人聽見我們是爲衆的事。倒都齊心向義。都造了草冊來了。衆人將三縣殷實戶口一算。共有三萬四千多家。精壯人名一總也有五千一二百人。尚智道。夠了。我們這就做第二着了。如今南京兵部尚書史可法應天府尹樂爲善這二位老爺。都是憂國憂民愛人愛物的好官府。我們同去見他。具個手本。把這些詳細說明。他見是保障地方護持衆命的事。再無不依的。還有一說。這些盔甲器械還要求他賞給。每人得銀十兩。支散三萬金。以成這番義舉。衆人道。這恐不能。他若聽見要這些銀子。一時不准起來。倒把好事弄崩了。尚智道。凡事要慮首慮尾。愼始愼終。這事自有一個道理的。我們此時不但沒有這項銀兩。就有所出。但製辦軍裝器械。不是我們百姓做得的事。我們這事旣成了。保護城池人口。須等流賊剿盡。方可解散。不是一朝一夕就罷得的。這兩位好官可保得住他常在這地方上麼。他設或陞遷病故。換了個壞心的來。拿捏我們私造兵器。豈不吃他的大累。如今求官給下來的東西做了把柄。不但可杜後患。就是目下尋是尋非的官吏。也免他許多妄議妄想的。衆人道。尚兄想頭。可謂十全之極了。事須緊速。不可躭延。我們急忙同去。若到臨渴掘井。就無濟於事了。遂大家起身。渡過江來。到了城中。尋店安下。備細寫了兩個手本。前列慕義尚智林忠名字。後開國守武備等二十餘人姓名。次早先到府尹衙門來等。開門的時候。單他三人進去。跪在丹墀。樂公見他三人儀表非俗。慕義方面大耳。圓扇長鬍。林忠豹頭虬髯。尚智白面長鬚。正有些驚異。呈上手本。樂府尹看了。喜動顏色。道。你們都是忠義豪傑。快請起來。叫上堂來。問道。事非小可。你這三縣人都齊心麼。三人答道。這是上爲朝廷。下保身命的事。衆人都願意。若蒙老爺恩准。就可以刻期舉行的。樂府尹道。這是爲國爲民。是極好的義舉。本府焉有不准的。但須關會兵部纔可。且這三萬金也非細事。還費商量。他三人道。小人們另備有手本到兵部投遞。先稟明了老爺。然後去投。但這三萬兩銀子不得不求恩給。如今養這三千鄕勇。非厚給以衣糧。何以得他死力。每人一年支五十兩。三千人每年須十五萬兩。在這三縣小民。也就算竭力得很。他固然是要保身家性命。不得不出。若十分多了。力便不能。這一項銀子再無從裁派。是以不得不求恩賞給。樂公道。你們說得有理。且去投了兵部的手本。我再會史老爺公議。計較出個法則來。他三人謝了出來。又到兵部。正値史公散了出衙門來。他三人攔轎跪下。呈上手本。史公也正見飛報流賊的羽檄交至。甚是緊急。他是本兵。正在憂慮。接過這手本來看了。甚是歡喜。復翻身又回衙門中來。叫他三人到面前。道。不意草莽之中。有你們這些忠義之士。但三縣人多。賢愚不等。這事是出在各人舉義。又強不得他的。衆人可肯齊心麼。答道。衆人一來替朝廷保障地方。二來向日大受賊害。如今也求各保父母兄弟妻子身家。都肯力行。只求老爺天恩准行。並賞給盔甲器械之費。就可立舉。但聞得流賊聲息甚急。求恩速行方妙。恐緩不濟事。那就空成畫餅了。史公道。每縣添設這一千人。在何處屯紮。答道。每縣原有一名指揮。領官兵鎭守。如今於縣城相離不遠。相視地宜。星夜築一大堡。四週環以深濠。開南北二門。內中滿建草房。不但可以屯兵。且可爲縣中犄角之勢。況衆人家口衆多。一城屯聚不下。一聞賊信。聚在一處。城堡各一半。方可保護。不致疏虞。史公道。你們雖想得是。但你們原是爲保障地方。還是在城中守護爲是。衆人道。小人們都曾慮過。屯兵自然是城中有個防守。但臨敵事宜。機不可失。應戰則戰。應守則守。恐爲地方官一時掣肘起來。倘一有失。反誤了數十萬生靈性命。二則城中狹小。存不下這些人口。史公道。每縣旣添設一千鄕勇。自然將你們議幾個統領督帥。不然何以爲軍中司命。可行可止。都在你們。如何又聽地方官的鈐制。這兩件事都要兼行。城中一半兵。堡中一半兵。築堡存人家口。也是一件要緊的事。當速行之。諸事我都准行。也還要啓奏。表你們這點忠義之心。正說話之間。當堂投進鳳陽總督報警咨文。史公忙接過一看。內中道。

\begin{quotation}

流賊晝夜緊攻汴梁。四路援兵不敢進逼。周王吿急文書募人縋出者數次。諸將帥皆袖手旁觀。竟無半籌可展。汴梁似不能守。恐汴城一破。賊兵乘勝南來。不但京城當戒嚴守備。即鳳陽乃皇陵要地。恐兵微將寡。不能守禦。貴部職司本兵。亦當思調何歷練老成之將。統素常訓熟之兵。以爲聲援。倘有疏虞。皆有攸責。云云。

\end{quotation}

史公看了。半晌無語。忽發聲道。鳳陽馬督有報警文書。說恐賊不日南來。你們當作速料理。你們如今共有幾個人在這裡。答道。手本上有名的都在這裡伺候。史公道。都傳進來。我看一看。傳呼衆人到丹墀下叩見。史公吩咐起來。兩邊站立。定睛一看。一個個腰細膀濶。體大身強。果然都是英雄氣象。怎見得。

\begin{quotation}

那尚智身長力大。腹隱珠璣。不但有決機制勝之才。且能具驚人潑戰之勇。林忠豹頭虬髯。衝鋒破敵何難。慕義狼腰虎背。斬將搴旗甚易。國守白面長鬚。銀鎗出衆。武備細腰濶臂。金斧稱奇。其餘的都是干城猛將。一個個眞乃草莽英雄。

\end{quotation}

史公心中大喜。道。目今事不可緩。只留你三人在此等候下落。他們衆人都打發回去。如挑兵築堡建房等事。非旦夕可成者。分頭料理。當速爲之。三人又稟道。老爺明見。今日就着他們回去。還求給一執照。方敢行事。史公吩咐書辦寫了個執照。硃批了。用了印。給與他。衆人叩辭。史公道。別的先去罷。你三人在此。我還有話說。他三人站下。史公道。你們這些人中。也要得千餘匹好馬。纔可禦敵。那流賊的馬多。我們若全是步卒。怎麼相持。這個你們可曾想到麼。這項銀子又出在那裡。你手本上的三千人。用三萬兩製甲胄兵器也夠了麼。尚智答道。小人都算過了。那萬惡流賊說起來令人髮指。聞得他餵養馬匹。到一處地方。把老弱男婦剖開胸腹。剮去臟腑。以人血拌草豆餵馬。以人腹爲馬槽。那馬臕壯力強。見人都〔有〕吞噬之勢。我們雖有馬匹。如何敵得過他。如今一千人中有一百多馬就夠了。不過要探聽事機。傳報軍情。以及追奔逐北之用。這一項銀子也都想到。如今三千人只用三萬戶養贍。目今三縣共有三萬四千餘家。擇力量稍次者剔出。命他十家出一匹好馬鞍轡。不過三千餘金足矣。永免供應。諒他也自情願。這有四百來〔匹〕馬就儘夠了。至於盔甲器械。如今純用步卒。不用鐵盔鐵甲。那又重又夯。不過好看壯膽而已。流賊全仗弓矢。那盔甲連箭也抵擋不住。用之何益。古人曾說。他甲在身。我甲在心。如今只製黃布綿甲。通身盡畫虎紋。又輕又穩。禦敵時用水濕了。箭旣不能透入。穿着又伶便。又可用力。頭上俱做黃布虎頭包腦。厚厚大大的。不但護住了頭項。且使那賊的馬不但不敢咬嚙人。他見這些虎頭繞躍。人身上盡是虎紋。自然心驚。馬一驚跳起來。馭之不暇。何能更使兵器。至於我兵所用器械。不用他物。一千人中。二百大砍刀。以二百長鎗隨之。用片刀者低頭專斫馬足。長鎗上刺賊人。兼護刀手。二百連耞棍。亦以二百鈎鐮鎗隨之。連人帶馬一齊力打。鈎鐮鎗上可鈎人。下可鈎馬。又可直刺。以護棍手。賊兵從未經過這種戰法。亦一制勝之道。還有二百鄕勇。一百馬兵。皆持長柄大刀。臨陣或衝隊。或追敗兵。隨時調用。那一百弓弩手。帶同衆百姓。預備磚石滾木。金汁灰瓶。護守城池並堡子。愚意若此。求老爺上裁。史公大喜。道。你這一番議論。眞經濟之才也。可惜屈於草莽。果能爲國建功。何慮不爲朝廷柱石。你們且歇息去。我會同衆官商議出這項銀兩來。給你們去製辦。他三人辭了出來。値樂府尹來會史公。史公接了進去。到後堂坐下。史公就叫書辦將方纔他三人那手本拿來。遞與樂公看。樂公接過。展開一看。道。他三人也曾到敝衙門來。他說要到老先生這邊來呈報。不知老先生准行否。史公道。這是他衆人的義氣。又不費朝廷錢糧。得了這枝父子兵捍禦殘寇。不但說護庇了數十萬蒼生。且保住了朝廷城池。可有不准他的。如今但躊躇這三萬金無出耳。樂公道。弟見他衆人這段好事。心中也甚喜。我們都有地方重任的。得他們保護住了。我輩旣免守土之責。且使黎庶免遭無限慘毒。是極妙之舉。也就是爲這三萬金煩難。無處措處。弟之愚意。或守道庫中\footnote{京師守道即外省布政。}。或兩縣庫中。雖不能足數。且湊些出來。看差多少。再來會老先生商議。古云。苟利社稷。專之亦可。支用了的。然後題本。就朝廷見罪。爲了百姓。便棄了這功名。又何害也。不想傳了守道同兩縣問起來。都說四處經餉隨到隨解。尚且不敷。庫中竟是空空如也。眞令人寒心。弟因實無措置。特來請敎。當是如何畫策。況這事情甚急。又躭延不得日子。却是怎麼處。史公想了一會。道。弟今請了各部並各衙門衆位老爺來公同計議。要大家肯爲國爲民。捐俸幫助。更爲義舉。萬不然。我二人問司農庫中借出三萬金來。先給他們用去。然後公同啓奏皇上。就有責備。我二人力認罷了。若因此而獲罪。榮莫大焉。樂公搖首道。捐俸一節。萬萬不能。還是借庫。或尚可行。然大農司未必有如此擔當。也還在兩可之間。史公笑道。遽伯玉恥獨爲君子。先生太藐視一切了。樂公自愧失言。無可回答。史公差衙役各處分請。不多時。陸續都到。讓了坐下。茶罷。史公道。奉請列位老先生到此。有一要事相商。衆官道。請敎。史公道。近接各處塘報。並鳳督來文。流寇猖獗。慘毒異常。自河南一路攻城掠地。又想來寇逼京城。目今六合天長江浦三縣。有許多忠義之志(士)。自爲廩食。奮勇編伍。爲朝廷保護地方。捍禦流寇。所需者盔甲器械。他們爲頭十數人。特到大京兆同敝衙門兩處。求給三萬金。以爲製刀鎗甲胄之用。弟想這些草莽百姓還有忠君愛國之心。難道我輩食朝廷重祿享高位的反不如他們。寧不自愧。故此請衆位老先生來。不拘多寡。捐俸力助。倘能成此義舉。也是一件爲國爲民的好事。不知列位尊意若何。衆人先聽見他們爲史公所請。以爲是吃酒。不知是做甚有錢的事。都欣欣然而來\footnote{此二語乃作者譏貶衆人之意。}。忽聽說要捐俸。眞掃天下人之大興。都都像啞巴一般。默默然無語。你看着我。我看着你。總無一人回答。內中也有幾個尚義的。肯拿出些來。但銀數多了。多出捨不得。少出不濟事。聽衆人聲口如何\footnote{這幾句回護得妙。不然。豈衆人中皆無人心者耶。然而語中猶帶刺。更妙甚。}。見這些人都金口三緘。他也就閉口藏舌。內有一個國子監祭酒。名叫做汲斷金。是福建福州府人。聽得要捐俸。急得眼睛睜得有燈盞大。臉脖子通紅。結結巴巴。半日掙出幾句來。道。這固然是好事。奈敝衙門是個冷草(竈)。連飯都沒得吃。假一年的俸祿。衣食還供不上。如何有得幫助做這一事。衆人也就接口道。弟輩與大司成都是同病。心有餘而力不足。奈何。又有一個禮部尚書姓傅名勝。係江西南昌府人。家中有巨萬之貲。世稱豪富。却鄙吝無比。他道。學生待罪禮曹。終年連一個大錢也沒得進益。連買太\footnote{大。}臟恰\footnote{吃。}的錢都冒有。還要助甚麼俸。況我敝衙門只管僧道儀註。這些募兵捐俸的事情問我不着。這是本兵部同戶部的責任。老先生何不問大司農借。何苦扳扯我們\footnote{此原是史公本意。今却出在傅勝口中。妙。}。史公不覺怒起。面紅耳赤的道。我輩朝廷臣子。反不如那些閭閻義士。捐俸之議。不過是上爲朝廷之封疆。求其永固。下救黎民之塗炭。拯拔生靈。而諸君竟無愛上恤下之心。難道朝廷是我一人之君麼。衆人見他發急。語語關着朝廷。難以回答。都如箭穿雁嘴。鈎搭魚腮。口也不開。史公見衆人不做聲。沒奈何。向牛尚書道。如今事在燃眉。先生庫帑借三萬金出來。且給與他們。弟上本啓奏。若是皇上不認。弟願破家賠補。如何。這戶部尚書名牛騂字日新。就是牛質的族兄。他姓牛。那生性也就是一條蠢牛。答道。目今軍需緊急。倘一時征調錢糧。何處設法支應。若朝廷見罪起來。如何了得。這斷難從命。這是傅老先生自己捨不得。拿着本部推諉。老先生如何認了眞。問庫裡借起。傅勝發急道。我一個閒曹。是那裡來的錢。你管着戶部。不拿出來。倒扳扯我。牛騂道。我雖管戶部。是朝廷的銀子。豈是我的私囊麼。若拿出用了。朝廷不認。且有擅專之罪。那時怎麼處。先生府上之富。甲旋江右。人所共知。借出這三萬銀子來。如氈上去一毛耳。傅勝越發急得臉脖子發紫。說道。我家雖有幾個錢。是祖宗留下來與子孫的。並不曾叫助兵餉。況朝廷的臣子不是我一個。爲甚麼叫我出。若是我的家事。那就講不得了。這是朝廷家的公事。少不得要問貴部要。牛騂道。雖是朝廷家的事。若有旨意。我自然應付。今私自講借。後來恐弄到我身上。我怎麼敢發。他兩個只管爭競起來。傅勝纔要開口。樂公道。老先生且止言。向牛騂道。史老先生尊意。不過暫挪一時。我二人擔着。少不得連名上本。即皇上不認。弟同大司馬公賠。這算是因公挪用。決不貽累於老先生。牛騂道。怎麼貽累不着。銀子現在敝衙門庫中。守者不能辭其責\footnote{辱翁曰。這却是實情話。}。二位先生要做這忠義之舉。弟却不能以身家功名奉陪。做這迂濶之事\footnote{眞是牛心。}。二公請想。還是軍需要緊。還是這未定濟否之瑣事要緊。史公更怒起來。道。爲朝廷保守封疆。何爲迂濶。要說軍需要緊。這難道不是爲朝廷出力麼。牛騂道。二位老先生旣說朝廷不認。願傾家賠補。與其獲罪而後賠。何不今日竟慷慨任之。且使朝廷聞知。更見二公忠義。豈不簡捷更妙。衆人附和道。牛老先生這一論。眞痛快妙極。雖聖人復起。不易斯言也。樂公此時也忍不住了。便大聲道。諸位老先生皆食祿仕朝。難道只我二人是朝廷臣子麼。我二人並不是捨不得家貲。但此是一時立等要用。目下措辦不及。恐緩不濟事。若可以爲。早已自行。又何必請列位來計較。更何必向老先生苦懇。牛騂冷笑道。二位做忠義豪傑的人。志向自然與人不同。弟輩碌碌。原不足與議。就立起身來。冷笑了一聲。道。奮不顧身者自是聖賢。而明哲保身亦非迂濶。衆官也就起身。道。牛老先生所言有理。我們且別過。不要誤了二公的正務。汲斷金極贊道。列外\footnote{音位。}先三\footnote{音生。}。尊演細疾\footnote{言是極。}。瓦們且棄\footnote{我們且去。}。遂大家鼻中冷笑而去。史樂二公送他們去了。復坐下。史公長嘆道。弟先以爲老先生尊言太過。此時看起來。眞是朝廷之上。朽木爲官。殿陛之前。禽獸食祿了。恨聲不已。復道。汲黯矯詔發粟。眞鐵漢。眞忠臣。何古今之不相及也若此。樂公道。此輩庸人。不足與較。且相商此事要緊。爲今之際。尊意若何。史公道。弟此時怒激於中。竟不能想出一條道路來。且事在匆忙逼迫之時。又不能從容緩議。實在沒法。又叫書辦將鳳督馬的來文與樂公看了。史公道。事將奈何。先生有何高見。樂公道。弟倒想了一策。尚不知如何。此時傳了兩縣來。命他傳諭合城大鋪戶。百金以下本錢者不必論。三五百金以上十數萬金以下者。叫這些人明早都〈早都〉到敝衙門。屈老先生的大駕。也到敝署去。我二人以婉言勸之。激以忠義之氣。那三小縣窮民一年出數十萬\endnotemark[8]養兵。難道這一個大京城兩縣鋪家湊不出三萬銀子來。史公想了一想。道。老先生此想雖妙。但恐未能。樂公道。老先生何以見得。史公道。那三縣的人豈都是一心向義。專爲捐助朝廷的。他要顧身家性命。保護父母兄弟妻子。不得已而出者居多。況是大勢使然。十家有七八家出了。那兩三家就不得不出。且每年一家只出五金。力還易爲。這三萬金要一時拿出。他自己又無急難。如何肯捨。況這事又不是強逼得人的。且堂堂臣宰尚猶如此。而何況於閭閻之小民乎。樂公道。弟也想到此處。偌大京城難道沒有四五千大鋪戶。每人不須十金就夠了。恐也還易舉。史公道。若做得來。是極妙的了。先生請回。今日趕着命兩縣去傳。明早弟到貴衙門來。樂公作別而去。史公也回家去了。樂公一到衙門。就傳了兩縣吩咐了。約於明早飯時齊集衙門。到了次日。史公老早就來了。吃了便飯。到巳刻。兩縣進來稟各鋪戶到齊了。呈上兩本冊子。兩縣各開地方鋪家的名字。二公看了。恐衙門丹墀窄小。人多站不下。遂同步到大門外來。把上項的事說了一遍。並要他們樂助這宗銀兩。說了許多的忠義的話。又道。這也不強你們。但出在你各人心裡。願出多寡。就註在各人名下。說完。吩咐兩縣叫把那花名冊那(拿)與他們親自去寫。他二公進來。兩縣吩咐書辦拿了冊子叫衆人去寫數目。二公在後堂閒話。外面傳進一角文書。係亳州知州金蘇的申文。書辦拆開呈上。樂公看道。

\begin{quotation}

南直隸亳州知州金蘇爲懇恩旌獎節烈以勵人心事。流寇大隊盡駐汴梁。其游賊四出劫擄。民間子女多遭淫掠。職所屬離城百里。有一節義村烈婦余氏。係何光衛之妻。年十七。適光衛。今始十九。聞賊將至。知其地賊所必經。烈婦即以針線密縫衣褲。預爲死計。明旦寇至。乃抱幼女同從姪女唐氏婦走避。道遇賊。即投水中。旣沒復浮。仰見唐氏婦尚竚溪畔。乃大呼曰。汝欲出醜耶。可速下。於是唐氏婦亦投水死。三旬寇退。光衛歸家。循溪十餘里得烈婦屍。尚緊抱幼女。而唐氏婦附焉。時値盛暑。已經匝月。兩屍面色如生。毫無腐穢之氣。見者無不驚嘆。地方呈報到職。據實通詳。祈恩旌獎。亳州之地正當孔道。賊若南侵。決不捨此而出他途。今旌獎二氏之貞節。不但使婦女聞知。捨淫就義。亦可激勵男子。奮忠義之心。或可守此彈丸之地。云云。

\end{quotation}

樂公看了。遞與史公看畢。嘆道。一鄕僻女子能知死於節烈。而鬚眉男子食朝廷之祿。反俯首從賊。搖尾乞憐。是何心哉。樂公即吩咐本房做本。題請旌獎。到午後。兩縣送進冊子來。二公翻開一看。許多當鋪紬緞鋪金珠鋪都是一兩二兩的居多。三兩五兩的還有些。一個十兩的也沒有。翻到後邊小鋪戶來看。盡是一兩。或見一個錢米鋪鮑信之。註着助銀一百兩\footnote{眞是空谷足音。不得不驚。}。二公驚訝道。多少大鋪家連十兩的也沒一個。他一個錢米鋪能多大本錢。肯出這些。必有緣故。叫他進來。衙役出去傳呼。鮑信之隨了進來。跪下。二公道。你起來。他便立起。樂公道。近前來。他走到跟前。樂公道。兩本冊內上。兩縣約四千多人名。十兩的並無一個。你有多少家私。就肯捐出一百。鮑信之又跪下。樂公道。不必跪。起來講。他站起。道。二位老爺。今日之舉。不過是忠君愛民的事。又非自己要入私囊。小人但恨本錢少。鋪中不過三幾百金的局面。若家私大。就助一千二千也該的。況素知流賊的凶惡。恨不得殺盡了他。以除衆害。小人雖是小民。也有些忠義之氣的。但恨力量不能。二公聽了。嘆道。若人人皆如你心。何事而不可爲。叫書辦將冊內銀數一算。通共不足萬金。史公道。這尚不足三分之一。奈何。樂公道。這銀子如今且不要他們的。倘事做不來。豈不像騙百姓的銀子用。且叫他衆人回去。等用時再來傳諭。不用就罷。兩縣出來吩咐了。衆人散去。鮑信之也去了。史公道。這事怎麼處。樂公道。此時急也無益。且稍緩再爲設策。史公道。做官到底是貪婪的好。若我輩在宦途不爲不久。職也不爲不尊。而竟毫無私蓄。要有宦囊。何等便易。何必費這許多周責(折)。樂公笑道。不然。那種肯聚歛宦囊的人。他未必肯來做這些事了。況且我們今日就算這件事做不來。上不愧於朝廷。下不慚於百姓。較之貪鄙吝嗇者。又覺此中稍安。今日上托聖天子之福。倘這數十萬生靈不當膺鋒敵(鏑)之苦。或另有機緣。亦未可料。史公長嘆了兩聲。作別去了。却說鮑信之回家。正打賈文物門口過。想道。久不見老爺了。我順便進去看看。到了門首。賈閽進去說了。賈文物正在書房中。聽說。叫請他來。鮑信之進來。作揖坐下。賈文物道。許久不到。今日往那裡去來。鮑信之道。一向窮忙。失於親近。今早府尹樂老爺傳到衙門中。纔回來。賈文物道。傳你有何事。他遂將史樂二公勸慰幫助的那些忠義的〔話〕說了。便道。這些奴才。整千整萬銀子的本錢做着大買賣。都只助三兩二兩。一城的鋪子。連十兩的也沒有一個。門下激起一點義氣\endnotemark[9]來。我就寫了一百兩。雖知他也無濟於事。也盡我這一點鄙心。愧一愧這看財奴。但恨我窮。我若有十多萬的家私。叫我獨認。我也肯。想這一番義舉。若能救幾十萬人性命。豈不比童老爺那年施粥賑救萬數人的功德更大。比宦老爺代償拖欠的仁慈更廣些麼。我看史樂二位老爺見湊不出銀子來那個急法。他也不過是憂國憂民的念頭。門下雖有尚義之心。而無助銀之力。奈何。賈文物聽了。尋思道。他多大本錢。倒有此義氣。我前日算算我的家私。數年累積也將有二十餘萬了。宦哥童弟他兩人做了多少好事。獨我不曾。我何不獨行這一場義舉。忠君愛民。其功也不在他二人之下。主意定了。便道。罷。這一件事我獨任了罷。我今日齊了銀子。明早去親見樂公。你明日早來。拿我個手本。到兵部稟知史公。也史(使)他歡喜歡喜。鮑信之慫恿道。老爺若做了這一件美事。自然要上達天聽。那就朝野馳名了。門下明日早來效勞。遂別了回去。賈文物到了房中。帶着金銀珠玉四個妾。搬出六百封銀子。堆在一處。富氏問其故。着實歡喜。道。這是救人的好事。應該做的。況去了這些。也還窮不着我家。我每常會着宦家姆姆。童家嬸嬸。無人不贊他們丈夫的好處。我臉上好沒光彩。今日你做了這事。我也添了多少體面。賈文物見富氏這樣興頭。分外鼓舞。次早。賈文物起來。寫了兩個手本。鮑信之也來了。付了一個與他往兵部去投遞。叫家人拿了一個。坐轎到府尹署中來。門上認得是本官相契厚的。連忙傳進。樂公請入後堂。坐下茶畢。賈文物方說道。聞得老先生與大司馬史公有爲國爲民的一番事。所少者不過三萬金耳。竟無一個仗義之人。以成二位老先生義舉。以救百姓。晚生深爲扼腕。晚生雖非富翁。願力任此。助三萬金。以全二位老先生美事。樂公大喜。道。三公可謂樂善不罷\footnote{音疲。}了。但這三萬金非細事。急等要用。年兄可曾打點。約料幾時可得。賈文物道。老先生這邊。晚生可敢孟浪。都預備齊了。方敢來奉吿。此時若用。就可取來。樂公更大喜。道。君子之所爲。衆人固不識也。我此時同年兄去會會史公。也使他歡喜。趁今日尚早。還可行事。賈文物道。晚生已着人稟知史公去了。樂公道。旣如此。年兄且在此寬坐。等貴使的回信。叫了個衙役來。吩咐道。你飛星到兵部衙門去。看見賈老爺的管家叫他來。衙役稟道。不知賈老爺管家貴姓是甚麼。小的好去問。賈文物道。就是昨日在此的那個鮑信之。差役應諾去了。樂公問道。這鮑信之竟有一腔義氣。原來是貴紀綱。賈文物道。他非晚生家人。不過在舍下走動就是。二位老先生這一番事。也是他昨日在貴衙門回去。到寒舍說的。晚生方纔知道。不講他二人閒話。且說鮑信之到了兵部。値史公在大堂上坐着。因這一項銀子尚無影響。一來賊信甚緊。二來他是個做大人的。興抖抖准了呈子。又給了執照築堡挑兵。這件事人人皆知。今爲沒有銀子。忽然罷了。如何行得。心下十分作難。眞是。

\begin{quotation}

一心分(粉)碎萬民憂。兩眉愁鎖無錢恨。

\end{quotation}

正在躊躇。忽見門官進來稟道。有一個助餉的人在外面稟見。史公聽了甚喜。而又詫異。叫快傳進來。須臾。鮑信之隨了進來。跪下。史公認得是昨日助一百銀子的那人。只道他送了銀子來。便道。你上來。他起來走到公座傍。史公道。你送銀子來了麼。若全城都像你這等仗義。何消本部慮得。方纔門上人來稟說有人來助餉。本部正在疑惑。那裡有這等好人。原來還是你。鮑信之稟道。小人不是送銀子來。諒那些須。濟不得二位老爺甚事。便把賈文物的稟帖呈上。道。小人昨日回去。見了這賈進士。說起老爺與樂老爺二位這樣爲國爲民的心腸。竟無一人肯於體貼。賈進士一時仰體二位老爺龍心。力捐三萬兩。以成美事。他不敢造次來稟見。着小人先來稟知。史公大喜。復大笑道。不想名敎中竟還有這等義氣漢子。眞令這些庸奴愧殺。你如何認得他。鮑信之道。小人是他門下。小人也是蒙他的恩德提拔起來的。史公道。你東人如此古道。無怪乎你纔有這種義氣。他有此等高情。我先到他家去拜謝。就起身叫搭轎。鮑信之道。小人來時。賈進士見樂老爺去了。此時恐不在家。不敢勞老爺大駕。史公道。他旣在樂老爺處。我就往那裡去拜他。且還有事同樂老爺商議。你也跟我去。便上轎起身。吩咐到府尹衙門來。此時府尹的衙役正在門口等鮑信之。見史公去會本官。如飛的報信去了。樂公正與賈文物敍話。衙役來稟道。小的正在兵部門口等候賈老爺的管家。不見出來。史老爺來會老爺了。少頃。聞得史公到了。樂公同賈文物出來接着。史公問樂公道。這位就是賈年兄麼。樂公道。正是。史公上前。一把拉住了手。笑道。年兄這樣高德厚義。學生竟不曾識荆。眞是俗吏了。賈文物道。久仰山斗。未敢進謁。今得瞻韓(仰)。何幸如之。攜手同進後堂。賈文物一揖。就下一跪。史公忙抱住。道。怎敢動勞。學生該拜謝纔是。作了揖。史公道。學生要到府的。因貴門下說年兄在此。特來奉拜。賈文物一恭到地。道。何敢勞老先生玉趾。晚生反得罪了。史公問樂公道。老先生與賈年兄素常相識麼。樂公道。相契久矣。弟當日到任之初。正遇兩省流民飢寒待斃。弟竟束手無策。將他三人如何救拔了這萬餘飢民的話。說了一遍。史公道。前番的事。人皆敬仰。自不必說。今日這一番高誼。不但學生佩服。這些買賣中人何足道。使各衙門諸公聞知。都該愧死了。賈文物道。些微小事。何敢當老先生過譽。史公因見鮑信之在傍。問賈文物道。這人是貴門下麼。賈文物道。他開個小錢鋪。常在舍間走動。史公道。年兄讀書君子。還有一說。不意他一個經紀中人。竟肯這等仗義。却是難得。又問道。年兄所云之物。幾時纔得齊備。樂公道。賈年兄英雄作用。已經預備下了。要用就可取來的。史公喜道。妙極。妙極。旣承盛情。早一刻得一刻之濟。賈年兄在此坐坐。煩盛使回府發了來罷。賈文物道。還得晚生回去照看。就着鮑信之押來。晚生不來復命了。史公道。旣如此。不敢留。亦不必復勞大駕。容日再拜晤罷。賈文物吿辭。他二公要同送出來。賈文物再三道。老先生請留步。怎敢勞動尊步。樂公道。老先生請坐。我送罷。賈文物道。二位老先生商議正務要緊。晚生托庇帲幪久矣。何必拘此。樂公道。旣如此。遵命了。只送到大堂後邊。一揖而別。賈文物出來。鮑信之也隨了去了。二公又坐下。史公笑道。先生竟有先見之明。學生弗如也。樂公道。老先生何以言之。史公道。老先生昨日說上賴聖天子之福庇。若這數十萬生民有救。自有機緣。不意就遇賈年兄這等豪爽義氣。豈非老先生之先見。他這一番好處。定要上達聖聰。倘有恩綸。庶可稍報他這種盛德。樂公道。老先生尊意極是。他雖不望報。若朝廷肯加恩於他。亦可鼓勵後人。史公道。今大事已濟。可即吩咐他們領去。但只兵無主將。何以行得。弟的意思。將他爲首三人。先委他三個守備職銜爲總領。其餘手本上爲頭的人。三營設九員千總。十二員把總。俟有功之時。再行題請實授。一來可堅他仗義之心。二來鼓舞他衆人的義氣。老先生尊意若何。樂公道。此舉允合人心。當理是極。史公顧左右道。尚(慕)義等三人在何處。可去傳來伺候。衆人稟道。現在衙門首。不多時。鮑信之進來稟道。銀子到了。請二位老爺示下。放在何處。史公道。就放在堂上。二公同出堂來。坐下。吩咐傳慕義三人進來。慕義等進來。跪下。史公叫起來。近前。道。銀子有了。你們應買甚麼。到這裡領去。作速製辦。早早預備。我看你三個人。不但義氣可嘉。智勇亦爲一時之傑。本部委你三人三個守備職銜。統領衆人。三處本部起三個營名。以便識認。慕義所轄就名爲義勇營。林忠爲忠勇營。尚智爲智勇營。新築三堡。亦以此名之義勇堡。忠勇堡。智勇堡。三人跪下道。蒙老爺天恩。但小人們尚未絲毫報効。怎敢就蒙委職。史公道。幾千人沒有統帥。如何有紀律。再給千總箚九張。每營三員。一爲中軍。二爲左右翼。把總箚十二張。每營四員。爲分汛游擊。你將前本內有名的好漢。量材補授。我給你們空名箚去。只管塡上申文來就是了。明日早堂。到我衙門領箚。俟候有功。題請實授。三人就叩謝了。又向樂公叩謝。復又稟道。倘有賊至。小人們只管拼力迎敵。守城之責。還是地方官的事。各有分任。不得互相推諉。推諉恐其誤事。史公道。說得是極。三縣城守指揮的名字叫做甚麼。你們可記得。答道。一個叫做裘道饒。駐天長。一個叫做卜濟世。駐六合。一個名叫做聞則陶。駐江浦\footnote{恐那時的文武官。無一個不是求盜饒。不濟事。聞賊逃者。恐不只三指揮耳。}。史公道。也是明日在衙門行文與他。他三人各自耑管守護地方。稍有疏虞。軍法從事。慕義等又跪稟道。小人們雖各統一營。還求老爺差一員文官。同心協力的共事。恐地方上有甚麼事。即小人等或有功罪。也便於申報。小人們只管得營務。史公對樂公道。這也是他們謹愼處。恐地方上文官有不肖之心。妄爲詳(佯)報。要個監理之意。老先生着甚麼官去好。樂公道。各官皆有職事。若使不得其人。倒僨了他們的事。因叫過鮑信之來。道。本府看你是個忠義好人。我擡舉你。給你一個照應職銜。一輪四個月。分駐三堡。他們有功有過。你俱據實呈報。俟他們建功之日。我也題補你。鮑信之忙跪下。道。念小人一介小民。毫未效力。怎敢蒙恩委職。史公道。這是樂老爺愛你這一點忠義之心。委了你。好同他們共事。只要你協力同心。就算補報了。不必推辭。謝了就是。鮑信之向二公叩謝了。樂公道。你也是明日早堂領箚。你名字這個之字不好。去掉了。只叫鮑信。你同慕義等三人明日都備了官帶。領箚之後。押着銀子。就同他們一齊起身。慕義三人又稟道。還要採買一應當用物件。尚求寬限二日。史公道。使得。該用多少銀子。到樂老爺這裡支用就是。鮑信又稟道。三萬銀子製辦軍裝。非同小可。求老爺諭縣。撥夫搬運。差營並領兵護送。方保無虞。二公笑道。他就是個做官的樣子。想得是。吩咐書辦行文知縣。撥夫擡運。委城守把總一員。兵五十名。押送了去。臨期齊集。勿誤。尚智又稟道。這挑選的三千鄕勇。要求老爺恩免他本身丁差。樂公道。這是理當。你們這冊移到本縣開徐(除)。叫他申上來就是了。吩咐完。史公也作別去了。次日。四人在兩衙門領了箚。尚智等三人係老虎補服金帶。鮑信是鵪鶉補服角帶。都紗其帽而圓其領冠帶着。兩處叩謝了。各人分頭行事。梅生同鍾生到他們寓處。攜酒盒來拜賀。斟鍾要請他三人。三人說有公務緊急。苦苦辭了。只到鍾生梅生家一拜謝。連話也不能多敍。就吿別採買各項去了。鮑信一個買賣人。忽然得了一個八品職銜。眞是平地一聲雷。把錢鋪也收了。南京繁盛地方。只要有錢。百事一呼而集。他就投了三四個家人。買備了冠帶圓領。領出箚來時。就乘兩人轎到了家。燒了天地祖宗喜神香紙。就有許多親友男婦拿果盒來道喜。他堂弟鮑復之同妻貞姑都來稱賀。那含香眞是喜從天降。公然間奶奶起來。心中暗暗感激賈文物。虧他少年時沾了他些貴氣。今日攜帶他夫妻俱得了好處。鮑信又到賈文物家來拜謝。賈文物見他做了官。也着實歡喜。道。這是史樂二公的恩德。何故謝我。鮑信道。不是托老爺的洪福提攜。晚生焉能到此。數年門下之恩。以俟將來報答。賈文物待他也自不同往日。要留他酒飯賀喜。他辭道。晚生一則要幫他三人買辦東西。二來家中還要料理料理。行期匆迫。也不能再來叩謝了。賈文物見他有事。也不強留。兩日內。他們買辦完了。辭了史樂二公。一齊起身。當日就到了江浦。鮑信雖是個委署職銜。却是上臺差官。知縣衙官少不得都來接拜。他把兩處東西交與知縣。指揮又撥兵夫送往天長六合去了。慕義林忠尚智各到了家。着人連夜督築堡子來。星夜製辦盔甲器械。招買馬匹。不日完成。會同鮑信將箚副按名塡補。申文去了。又將三千壯丁造冊。送縣開除。又挑選了幾十名力壯身強的好漢。委充百總管隊總旗小旗同營頭目。又沿途立了烽火一處。有警烽火一起。兩處就到接應。慕義三人要顯自己威名。他本營軍士稱爲飛虎軍。林忠稱爲猛虎軍。尚智稱彪虎軍。諸事料理停妥。聞得汴梁被賊放水衝沒。亳州亦爲賊有。鳳陽各處報急文書傍午於道。他三人知流賊不久要來。皆磨拳擦掌以待。再說史樂二公約會題上本去。先說慕義林忠尚智同三千鄕勇自備資糧。保護地方。俱權委守備千把職銜。並委鮑信照應監理。後將甲子科會試中式舉人賈文物助銀三萬。製辦軍裝的話。詳細奏上。又道。乞恩優敍。以鼓後人仗義之意。云云。崇禎看了甚喜。着吏兵二部會議具奏。兩部議了上去。慕義等忠義可嘉。俟剿賊建功之日。題請實授。賈文〔物〕捐貲爲國。着免其殿試。賜二甲進士出身。超補南京兵部職方司員外。鮑信俟贊功一並題補。奉旨依議。就有報子星夜下來。分頭去報。報到賈文物家來道喜討賞。賈文物雖然歡喜。想道。鍾兄是有大見識的人。我去請敎他該受不受。就到鍾生家來會着。將損(捐)貲殺賊並授職的恩旨請敎他。鍾生道。兄意如何。賈文物道。因此不決。故來請敎。忝在瓜葛。多年契厚。甚勿隱諱。鍾生道。這樣高遷大喜。弟本不當勸阻。旣承問道於盲。不敢不以忠言相吿。但兄此番義舉。耳其名者。無不稱揚敬仰。若因此而得官。與資郞何異。不受的更高。賈文物喜道。幸得請敎高明。不然幾乎自誤。遂回家推病不至。及至部文到時。史公差人來道喜。他已推病久了。不願受職。史公強勸他數次。斷不肯應命。史公同樂公親到他家中來苦勸。他婉言再四回覆。二公更敬他高尚。只得奏云。賈文物因久病未及殿試。蒙特恩賜進士出身。代題叩謝天恩。不能受職。崇禎正在缺餉之時。要鼓舞人心。批旨道。賈文物俟病痊之日到部供職可也。又報了下來。賈文物復來請敎。鍾生道。聖主之恩。爲臣子者不可過拂其意。兄但受虛名。不去到任。這又何傷。他纔受了。雖不曾到任。已是欽賜二甲進士超授的五品京職了。誰不來尊奉。親戚朋友賀者塡門。鍾生把前事向宦蕚說了。約會了梅生童自大。叫戲擺酒來賀喜。賈文物又還席道謝。外邊官家。內邊堂家。也熱鬧了十數日。史樂二公都有花紅羊酒來作賀。賈文物特席奉請。又約鍾生宦蕚童自大相陪。閒話按下。且把流賊攻打汴梁的慘毒。聽我細述。崇禎十四年正月二十二日。賊兵飢困。圍困河南府。福王常洵在內。河南八府惟汴梁與洛陽未破。李自成就食無所。志在必得。攻擊甚勁。舁各府大將軍砲環城密布。迅發如雷。三日後。賊勢稍殺。傍晚。總兵王紹禹叛兵內應。洛陽失陷。衆賊入城馳殺縱火。喊聲大震。福王及世子由松\footnote{即弘光。}與鄭太妃俱縋城走。福王軀腹肥重。不能遠行。黎明猶藏附郭民居。被賊兵搜執。牽入城內\footnote{王字之上從未見有牽者。福王被牽。其王爲何如王哉。已如羊豕等。無怪乎爲衆賊所烹而食了。}。舊紳大司馬呂維祺亦被執。遇見西關。王哀呼道。先生救我。呂維祺道。我命亦在頃刻。但名義甚重。王毋自辱。欲再言之。已迫牽去。福王見了自成。詞色悚怖。泥首乞命。李自成縱橫肆惡。數責其罪。傍有一個賊將。撫王肌。垂涎叫道。這樣一塊好肉。大王何不殺而食之。自成點首。那賊遂將福王殺了。稱重三百六十斤。臠分肢割。與囿中之鹿同烹。列賊臚食。謂之福祿酒飯\footnote{唐封道弘軀肥股大。李勣戲之云。爾殿斟酌坐得即休。何須爾許大。余謂。福王之軀略胖即休。何須爾許大。徒供賊人飽食。福王爲賊所啖。衆所共知。弘光即位之後。不思殺賊復仇。惟以漁色爲事。可謂天理良心喪絕喪盡者矣。}。呂維祺罵賊。氣節不稍挫。賊怒殺之。那時所在震動。巡撫李仙風出戰河北土寇。汴梁城守副總兵陳永福往洛陽收輯殘破未回。二月初九日。賊乘汴兵盡出。疾走三晝夜。十二日抵汴梁。辰巳時。有\endnotemark[10]馬賊三百僞稱官軍到西關。居民紛紛入城。午未時。步兵及大營隨到。巡按下令築門守。因賊攻西城。祥符縣知縣王變領衙役兵登城堵禦。巡撫高名衡同衆官分守各門。周藩承奉曹坤。左良史。李映春。率周府勇士八百人登西城守禦。下令民間有能出城斬一賊者。賞銀五十兩。能射殺一賊者。賞銀三十兩。射傷一賊或磚石擊傷者。賞銀十兩。百姓持弓矢刀槊者。紛紛登城。先是城垜口用桌面門板蔽砲矢。仍然打透。官兵手足不能施。生員張堅獻懸樓式。用大柏木三根。上排橫木十餘根如筏。其廣可跨〈可跨〉五垜或三垜。出垜外四五尺。每樓容十人。賊臨城下。官兵從上用火罐砲石擊之。樓堅厚。砲石不能入。又高出。能蔽身。官兵得施展手足。推官黃澍督造。一夜成十五餘座。分置城上。先是賊穿城六孔伏其下。官兵城上擊之不及。今從懸樓擊之。無不中者。怒賊甚。雨射終日。箭揷城垣如蝟。賊以四十八人舁一大雲梯。將抵城下。官兵放大砲擊之。俱死。隨發萬人敵火罐。悉燒之。並燒死紅甲賊首一人。宗室生員朱之滄縋城誘賊與言。斬之而回。賞銀五十兩。陳總〔兵〕在洛陽聞賊攻汴梁。兼程兩晝夜赴援。十六日夜至兩關。三鼓。由孤魂壇穿城營進小西關。砍死賊無數。遂統騎兵至城下。巡按令伊子陳德看眞。開水門放入。步兵貪取賊兵所遺騾馬。次早尚在小西關接戰。被傷被(頗)多。一兵登屋。手殺七賊。賊不敢近。被賊亂箭射死。西城有石〖石袞〗十八層。賊見而懼。遂不敢攻。十七日。闖賊雜衆賊中至城下窺視。有識之者指示。陳總兵子陳守備射之。中左目下。深入二寸許\footnote{此一箭不能殺此賊。豈非天乎。明朝當興。郭英無心一箭射殺陳友諒。明朝當亡。陳守備有意一箭不能射殺李自成。誠天數也。}。抱頭驚擁而去。闖瞎子之名自此始也。賊常出挑戰。陳總兵發兵出迎。至濠各退。賊欲誘官兵深入以擊之。官兵亦以賊衆我寡不中賊計。一着藍甲賊首忿恨躡退。爲陳兵所斬。十八日黎明。賊前鋒西向逡巡終日。至夕陽遁去。時傳左兵將至。又傳保兵渡河。賊解圍去。破密縣。又走登封。此次闖賊因乘汴梁空虛。來攻其不備。他帶領精兵不過三千。脅從之衆也不過三萬多人。賊去後。知縣王變督衆修葺城垣。晝夜兼工。十日吿竣。各官募兵添設營伍。防賊再至。知縣王變創立社兵。八十四地方立八十四社。擇民家有一二千金產者出兵一名。或兩家出兵一名。萬金產者出兵二名。巨商亦然。每社社兵五十名。擇殷實素行員生爲長副領之外。選總社五人。按五所五門。各置一人統之。凡四千二百不餉之兵。諸上臺時加獎勵。無事則團練習藝。有事則登陴守禦。三月二十三日未時。賊七騎飛奔曹門。貼僞吿示二張於栅上。守關兵追之莫及。是夜。賊大營到。闖賊屯土堤外應城郡王花園內。小曹操羅汝才屯繁塔寺。知賊必來攻東城。王知縣半夜遣人召李光壂爲右所總社。統社兵各照汛地防守曹門至北門。巡按任濬。巡撫高名衡。副總兵陳永福。同衆文武派守各門。二十四日。督師丁啓睿領兵三千。自南陽赴汴。就濠邊築壘防守。賊至。一戰輒敗。兵悉降賊。北門月城爲賊所據。有上至瓮城者。守北門回營。加銜都司李耀率數十回兵。各持大柳椽。躍過瓮城。盡擊賊落下城。王知縣急擲火盡焚之。曹承奉率周府勇士用土築門。至其半。門上有二孔。有賊來拆門者。從孔中鈎住。斬其首。賊遂不敢近。撫按下令。民間有男子一人不上城者斬。賊驅難民負門千餘掘城。城上用磚石擊死甚衆。照賊擊去。磚石不能擊者。擊以柴加烘藥下燒之。賊自出。火燒晝夜不息。自曹門至北門。環垣十餘里。次日。賊攻東北愈急。社兵有殺賊者。即報開封府總社紀功。東北角賊掘一大孔。用大砲攻城。傷兵頗多。城上用一大砲殺賊更衆。賊拆城開二丈餘。大砲十餘並放。步賊先登。馬賊繼之。官兵亦放大砲十餘。步賊至半途者。一擁而下。死者無數。每夜對攻數十次。至晚稍歇。汴梁謂佃戶爲牛人。此時稱爲牛兵。一夜鼓。巡撫發珠帖。令黃推官速撥牛兵三百赴援東北角。崇禎十五年正月初一日\footnote{去年二月十二日攻城起。至今已將一年矣。而四路竟無援兵殺賊。尚成何世界。是何軍政。亡國景況一至於此。可嘆。}。賊用陰門陣。驅婦女赤身濠邊。望城叫罵。城上點大砲悉倒洩。城上急用陽門陣。令僧人課(裸)立女牆叫罵。賊砲倒洩\footnote{昔明有一帝。見宮內豢豕。謂侍臣曰。宮闈之中。蓄此何用。命悉發光祿。後一夜。宮中獲一怪。索豬狗血厭之。而夜深。豬不可得。帝嘆曰。祖宗法自有深意。向之蓄豬。焉知非爲此。所謂寧可備而不用。不可用而不備也。余嘗謂和尚一敎。亦世間可有可無之人。比閱至此。破陰門陣亦大有用處。亦不可少之。然而大有疑焉。男人皆陽具。何故不可破此陣而必用和尚。愚意度之。豈以男子陽物微。不足以敵盛陰。因和尚上下兩光頭。以二陽而破一陰乎。殊不可解。俟高明敎之。一元子曰。三敎一體。賢愚不一。智者當自悟。作此批者。愚而且蠢。無味。}。賊又剜城。城上分中掘透其孔。以磚石長鎗擊刺。賊不能存。後賊不剜直穴。更傍剜小穴以避之。賊伐柏墊數臺。長十餘丈。廣五丈餘。高可三丈。上容百餘人。放大炮攻城。城上用方木長丈餘。廣厚二三尺。築一方臺。高出柏臺三丈。置大砲擊之。柏臺之賊悉死。生員張爾猷獻懸砲石式。立長柏木三如鼎足。懸大砲其上。望柏臺擊之。連斃數賊。保定總督遣兵扮乞丐送蠟書來。云大兵即至。巡按任濬傳示城頭。羣情愈定。丁督師兵三千先旣降賊。闖賊恐爲內應。誘至老營點名。俱縛手斬擲蓮花池\footnote{殺得好。殊快人心焉。}。賊在曹門北心字樓下掘一巨洞。我兵城上掘透。賊在內死據。兵莫能入。巡撫懸二千金置洞口。上硃書。有能奪此洞者賞。朱呈祥領百餘人。先用柴懸入洞中之半。加上烘藥。隨以多柴塡燒。極熱。賊不能存。乃灌水百餘斛。帶短刀跳入\footnote{所謂重賞之下必有勇夫。}。容兵五十餘人。凡三十六洞。俱以兵守之。賊晝夜竭力剜城。盡爲官兵之用。於是人心愈奮。一夜。三更大雪。任巡按令選奇兵五百。由水門銜枚出。傳令總社。約以暗號。奇兵過濠外。分數處砍入賊營。賊衆驚起。奇兵退走濠內。賊躡足追來。各洞兵齊出。斷賊歸路。奇兵復回。合殺一處。斬賊首七百八十三級。數十賊頭持刀驅其負門。持短撅入原掘洞口。官兵在內奮擊。不敢近。欲另掘。又被懸樓磚石擊走。回至濠邊。持刀賊乃盡殺之。屢驅屢殺。於是終日死者萬餘。陳總兵守大洞口。連日與賊戰。賊齊放大砲百餘。步賊隨砲聲上城。城上放砲。連倒洩三五尊。陳總兵置一大砲於胯下。命速點。大呼道。忠臣不怕死。砲竟不倒洩。百砲萬努(弩)齊發。打死衆賊。成了一堆虀粉。賊砲中傷官兵亦多。官兵愈加奮勇而前。對陣處無一線之〖阝少日小〗。急取王府及各寺廟門千餘。添築城牆。添一層。打透一層。築至七層乃止。賊又於東北角之南。陳總兵汛地之地北。貼城牆外壁剜一穴。約廣丈餘。長十餘丈。每日以布袋運火藥於內。約有數十石。置藥線兩根。長四五丈。粗如斗。是日。馬賊千餘。俱勒馬濠邊。步賊無數。巳時點放。藥煙一起。迷如深夜。天崩地裂聲中。大磨石百餘及磚石皆迅起空中。碎落城外。可二里餘。馬步賊俱骨肉如泥。間有人死馬驚逸者。城上城內未傷一人。此眞天意。非人力也\footnote{雖云天意。然亦理勢使然。洞口向外。如放砲同。自往外洩。不往內坐也。}。賊如是有退志。賊意懈。攻打俱緩。惟砲聲未絕。十五日。老營賊五鼓拔營。攻城之賊未動。午時。賊馬飛奔。呼衆賊速走。自西北往東南。揚塵蔽日。十六日。巡按命啓門。遣黃推官王知縣往視賊營。周視賊營中。牛驢頭皮腹肺。間以人屍。臭穢滿營。內外廣八九里。長二十餘里。以繁塔寺爲聚糧之所。糧深三尺。賊所遺婦女二千三百餘人。悉歸城下。因收月城內。禁民兵掠奪。俟其親屬認領。次日除〈除〉領去外。尚存三百餘口。悉送尼庵。每日人給麥一升。黃推官王知縣張伴讀總社李光壂出城遍視。自曹門至北門十餘〔里。〕賊凡剜三十六處。幾爲平地。屍橫遍野。斷髮滿地。死傷者不下十萬。令地方掩埋。十日未畢。十九日。馬丁張賀四將領兵三千。自汝寧府來赴援\footnote{這三千人好造化。幸遇賊去。若早來幾日。未必得保生全。}。悉令沿濠結營。看守修城。修完。仍遣之去\footnote{此三千兵只算得來監工。豈算救援。}。此一次闖曹二賊合攻汴梁。精賊約有三萬。脅從之衆有四十餘萬。攻城死者幾半。二賊到朱仙鎭點閱精兵。除亡外。中傷者二千八百〈餘〉七十餘人。俱以方桌仰舁而去。左良玉兵至杞縣。號十萬衆。賊甚懼。故聞風解圍遁去。左兵二日追至郾師白沙河。與二賊連戰十八日。屢次俱勝。左鎭見賊衆不能撲滅。只殺跑了他。解了汴梁之圍。便引兵回保襄陽去了。二賊走至項城。殺西兵三千。汴梁賊方去。黃推官李光壂同知縣率人運磚燒灰。竭四十晝夜之力。躬視版築。城垣一新。賊之偵者見金城如故。疑有神助。任巡按高巡撫合疏奏李光壂功勣。奉旨持賜拔貢\footnote{賞太輕。}。王知縣行取進京\footnote{此庶幾可。而黃推官亦有大功。恩賞竟無。}。李光壂辭總社。不許。闖曹二賊連陷十七州縣\footnote{有一笑談。一人誤中流矢。請外科看之。此醫以鋸鋸去箭桿。索謝。其人曰。鏃猶在內。奈何。外科曰。那是內科的事。與我無干。左帥是當時馳名大將。將來殺賊。只解了汴梁之圍。便回保襄陽。縱賊屠此十七州縣。豈此城池非朝廷之疆土耶。揆其意曰。襄陽係我所轄。汴梁旣解。各保地汛要緊。此十七州縣。非我之屬也。亦與外科鋸箭同意。}。三月二十二日。寇睢州。賊入城搜掠財物。未殺一人\footnote{此城人何幸。}。二十七日。攻陷歸德府。夷其城。殺戮甚慘\footnote{宋獻策即歸德人。爲闖賊之心腹。視其屠桑梓之中。不出一語相救。眞忍心哉。此賊也。}。四月。合土賊袁時中抵杞縣。屠其城。闖賊欲袁賊先攻汴梁。袁賊懼。夜半拔營東去。闖賊追至亳州界。連戰敗之。復歸圍汴。二十八日。諠傳賊將至。衆官悉登城守禦。五月初二日。賊頭哨先到。馬賊徘徊堤上。步賊於堤外曳枝揚塵。作疑兵之狀。次日。賊老營兵到。屯閻李寨。距城二十里。闖賊屯其中。衆賊頭目環營其外。縱廣約十五里。曹賊屯橫地鋪。相連不遠。賊後隊俱到。堤上賊馬往來不斷。時有游騎下堤。將至城而旋。步賊下堤割麥。或數十百人爲一羣。官兵亦出城爭割。賊東兵西。兩不相値。偶然卒遇。兵多賊即走。賊多兵亦走。數日麥俱盡。僅存堤邊之麥。十三日。左鎭及楊丁二督帥領大兵援汴。前鋒至朱仙鎭。賊遣三千騎往探。賊將堤上未割之麥盡行焚燬。左總兵屯營朱仙鎭。率大軍收服土寇劉扁子等。連營四十里。號四十萬。闖賊三千偵騎俱被擒斬。十六日夜。闖賊踉蹌移營馳拒左兵。賊知偵騎被殺。心中怕甚。盡棄營中器物而去。次日。難民自西南來。說賊已夜遁。陳總兵選健卒往探。果是空營。滿載遺物而歸。賊遺麥豆甚多。魚雞鵝鴨豬羊之數。及金銀器物床帳車輛衣服。無不盡備。其精好者。皆爲兵有。民日擔糧二回。數日。兵民約得麥豆二萬餘石。二十三日。丁營將官楊維城自朱仙鎭逃回。至西城下叫門。縋城上。說丁兵失利。左鎭南去。賊將復至。巡撫賞酒食。與公文令投丁督師處。次日。賊塘馬先回營中。諸物已盡。惟有豆麥\footnote{當日在城諸公知賊必然復來。何不即運麥豆入城。亦大失着也。}。兵民往取。見賊馬奔回。二十五日。闖賊復回閻李寨老營打糧。賊三二百爲羣。走五十里外。惟曹門外只二十里。懼土兵黨一龍截殺。不敢前。六月初四。城中有一個藿(霍)賣婆引一少婦。假做採菜出城。送至闖賊老營。霍婆向賊說王府中事。闖賊大喜。給金四錠。重四十兩。元寶兩個。囑他若送王府宮女一名到營中。給銀一千兩。霍婆進城。有恐懼狀。都司張吾銳搜筐中。得金銀呈上。巡撫審問明白。寸斬於市。遂禁婦女出城。城中乏糧。各官多方糴散。推官黃澍結義勇大社。豎大白旗於曹門上。大書。汴梁豪傑願從吾游者立此旗下。郡王鄕紳士民商賈無不願入。四方豪傑及土著智勇之士悉至。約得萬人。刑牲祭關帝。與衆飮血酒定盟。製旗五百餘面。每人給社票一紙。凡腰中繫無憂縧者。皆大社中人也。器械逐名領給。旗號按五方色。整齊鮮明。揚兵城頭。謁見巡撫。巡撫悅甚。郡王鄕紳總社及各頭目俱下馬飮三爵。給銀牌一面。周城四十里。人馬絡繹。旌旗蔽空。衆官稱賞不已。初。賊中有一賊將獻計掘河灌城。闖賊遂用千餘人掘河上流。使逆流而上。水勢緩。高不過五寸。三日流滿海濠。闖賊恨水不能淹城。反將海濠注滿。廣處四五丈。深三丈餘。雖欲攻城。不能飛渡\footnote{此獻計賊將是合城人救命王菩薩。瞎賊始終不能進城者。此濠之力。}。又撥萬餘人取土塡故道。因殺獻謀賊將\footnote{若遇說因果。必謂此賊證西方。}。七月初七日。寅時發兵。黃推官領總巡督陣門外。逐賊至土堤外。斬首四十一級。生擒十二賊。奪馬九匹。布帳器械百餘件。射殺三百餘人。土堤賊敗。大營賊喊聲近。收兵進城獻功。巡撫賞銀三百兩。自此每日出城。往住有小捷。次日。陳總兵置酒宴勞將領。以牛酒飯餅大饗士卒。五鼓。出擊賊營於土堤上。盡殺窩鋪中二百餘賊。割其首。收其布帳食物。此後各營或交戰。或擊營。無日無之。十三日。得河北檄。云十四日援兵渡河。城中整兵接應。次早。東北角烽火連起。未見船隻人馬。總兵劉澤淸過河擊賊。兩日皆捷。營中忽自驚擾。仍退還河北\footnote{劉澤淸亦算當時名將。而乃用兵是此。其彼自知。}。汴梁外土城。去城五里。在土堤上。闖賊遣衆削平如壁立。前此猶間留一段。至此盡剗取掘深坑。以防出入。留一二小路。晝則下去城哨探。夜則以草塞之。週圍俱步賊。每夜發喊鳴更。火光不斷。馬賊俱在大堤上。曹門將官夜刦賊營。被賊斷雙手。衆兵舁回。曹門外南北隅有葦坡數十頃。兵民日出割葦。賊亦割以飼馬。至是賊用毒煙燒三日三夜。城上見煙即起。聞氣臭知有毒。各含檳榔甘草。置大鋼(缸)百餘於城頭。滿貯水及甘草解毒之藥。煙毒不能傷人。賊移三營於曹門外。正東土城外三千賊紮一營。名新營。東北土城外紮二營。僞副將羅賊都司張賊帥領。有壯丁五百人。各負麥三四斗。自城西孤堆過河。夜走大堤外。經賊老營被擒。盡去雙手。驅至西門外。望城跪拜。投濠死者半。進城者半。闖賊斷手必至囗部。曹賊只斷手指一半。間有斷中三指者。猶不至爲廢人。城中製車營布帳。八月初一日。於東鹽坡列成陣勢。願爲前驅者三千餘人。擇初三日出師。車營內安大帳房。巡撫上坐。總兵僉坐。餘以次列坐。細閱車營。適有卒於城外生擒一賊。至極肥大。即磔車營前。黃推官稟巡撫道。今城中十兩銀易麥一升不得。乘此時人尚有力。猶可縱使。推官願以車營出城取糧。不用官軍一人。只義勇大社兵足矣。城以外。推官與李總社任之。但祈總鎭發火器手四百。城上左右救援。總鎭微笑不答。巡撫問李光壂道。道路豈無崎嶇乎。汝能熟識乎。光壂道。自北門至河上。大道如砥。路傍草莊被賊前已燬盡。有大樹百株。令健兒上樹遠瞭。賊來某處。即大呼某處有賊。巡撫道。砲揚起放無力。也(七)里遠。能擊死賊乎。光壂道。揚頭大砲七里外恐不能傷命。中軍營甫抵河上。每車取一人。得二千四百人。倚河爲背水陣。信砲到城上。城上放砲以四里爲的。河邊放砲擊三里。遣善氼者踰河請援。河北兵有不飛渡來者乎。河北兵直抵濠外紮營。連放兩日夜大砲。賊不能近車營。河北兵有不盡渡乎。河北兵渡。則糧亦不多運乎。不戰功成。賊惟喘喙遄遁。賊未至時。壂曾詣河上閱視。此路並無坑穴。兵法云。知己知彼。又曰。得地利者必勝。此之謂也。巡撫道。西兵前有信。八月出關。中秋前後可到。吾兒前月初四日進京面聖請援。料今已到河北。且再俟半月如何。衆皆默默。黃推官拂袖出帳外。抗聲道。事不可爲矣。莫若盡焚其車。澍跳入火中做厲鬼以殺賊。吳知府出慰道。半月亦不爲久。姑待中秋未遲。黃推官道。此時人有日食半餐者。猶可用力。若半月後。盡成餓莩。能驅餓鬼而用之乎。無論中秋及重陽。亦無援兵也。巡撫聞而不語。乘馬上西城\footnote{巡撫雖老成之見。恐如馬謖置之死地而後生。不意置之死地而竟死也。然而勢有不同。今獨守窮城。束手待斃。何不聽之使去。在死中求活。圖僥幸於萬一。有何不可。而半籌莫展。誠碌碌無能之輩也。}。各官回汛地。竭二十晝夜之力。竟成畫餅。城中糧盡。婦女數十萬。晝坐衢路。夜即臥地。死者不可勝數。黃推官見之惻然。於東嶽廟施粥三日。城中人相食。有誘而殺之者。有羣捉一人殺而分食之者。每擒獲一輩。輒折脛擲城下。兵民競取食之。至八月中九月初。父食子。夫食妻。兄食弟。姻親相食。不可問矣。有老夫婦二人商議。欲食兒婦。此婦聞知。跑回父母家中去。云公婆欲食。故逃回。其父母私議道。我家骨血。爲何便宜人家。遂將女殺而食之。命民間報牛馬驢騾充餉。送到城上給價。每兵分肉一斤。准糧一升。五日俱盡。開五門放婦女出城。先聞闖賊有令。窩鋪中藏匿婦女者斬。故放出三萬餘口。任其所之。有持數升糧復進城者。人無可食。吃牛皮以及皮襖。又取藥肆中山藥。茯苓。蓮肉爲上。次則何首烏。川芎。當歸。廣桂。芍藥。白木。地黃。黃精。門冬。蓯蓉。兔絲子。車前子。又其次榛子皮。杜仲。川烏。草烏。柴胡。白芷。桔梗。蒺藜。無不食之\footnote{諺云。有福之人無病也服藥。此時城中諸人無病服藥。不知有何病何福。}。城四隅有鹽坡。水深三四尺。忽生纓絡草。鮮嫩可食。男婦入水。手隨採隨食。水綿本不堪食。亦強吞之。水中小紅蟲他時取以飼魚者。皆縫紗布爲囊取之。名曰金魚子。入葱油炒食。味似魚子。每斤賣八百文。後至三千錢絕無矣。屋上瓦松每斤賣二百錢。後至一千二百亦無矣。糞堆中有〖虫喿〗。肥白寸長。積一二年者愈多。悉掘食之。食盡食膠泥。有騎馬過者。人羣食之。拾其糞。炒淡黃色。用水吞之。人食藥材。面目浮腫。有婦女在街頭賣藥酒。用甘草廣桂煮湯。如黃酒色。一錢一杯。飮之立癒。一車報理刑張客藏茶甚多。往視之。獲八百包。每將弁給十斤。兵一斤。以滾水漬去汁。曝乾爲末。入麪少許。作餅食之。城中白骨山積。斷髮滿地。路絕行人。神號鬼哭。天日爲昏。間有一二人枯形垢面。如同鬼魅。棲牆下。敲人骨吸髓。自曹門至北門。兵餓死者。日三四百人。夜則城頭寥寥。處處鬼叫。官府與諸郡王將校。旦夕北面而哭。家將謝廷璽領大社兵出城探賊。巳時點兵。未時收兵。並未見賊。此時大社兵也殘廢無多人。惟右翼程丹領南兵尚有千人。日夜登城。北望號泣。人盡枵腹。不能負戈。城頭奄奄殘喘。不能動履。一老農住曹門下。藏麥一窖。生員張爾猷訪知其家。到彼。向他道。汝有麥不敢食。不敢賣。埋之何爲。我爲汝起送城頭。活官府郡王。其功甚大。更爲汝留少許自食。老農點首道。在竈前。盡發之。得三十二石。送巡撫一石。守道五斗。諸郡王將弁分食五日。陳總兵家尚有黃黑豆數石。潛令人撒於街衢及空閒處。次晨。餓民見而食之。羣相訝曰。上天雨豆。救我殘黎。有拾至半升者。此次闖曹二賊合圍汴梁。步賊十萬。馬賊三萬。脅從之衆近百萬。瞎賊素知汴城富足。意欲困破。以圖擒掠。今久圍不開。心中忿恨之甚。恰値連連陰雨。河水大漲。十四日夜間。令衆賊將黃河上流握(挖)開數處。那溜水一瀉而下。城中遠遠聞得水聲。正在驚慌。十五日黎明。水至城下西南。賊俱遠遁。東北賊溺死無算。十六日。水大至。黃推官坐城下。李光壂與張爾猷抱土率兩營兵塞門。水從〖阝少日小〗入。勢不可遏。水聲如雷。曹門水高丈餘。進門輒南下。是時南門先壞。北門沖開。至夜。曹門東門相繼淪沒。一夜水聲如數萬鐘齊鳴。十七日。天黎明。滿城俱成河。止存鐘鼓兩樓。及各王府屋脊。相國寺寺頂。周府紫禁城惟夷山頂皆乾地。逃水者滿集。十八日。黃推官遣善泅家丁李用柳體直二人過河請救。泛一木水上。三晝夜始達土堤。監軍道。王爕得推官手書。連夜督二十餘船。自乘小舟。從北門揚帆直入。高巡撫黃推官各乘船到紫禁城上。見周王。抱頭痛哭道。請王北渡。宮眷五六百人同行。百姓有在城頭屋角樹杪者。俱漸次渡河北。到了柳園。煮粥食難民。眞古今來未有之苦。亦古今未有之厄也。這惡賊因城高固。池寬深。急不得下。屢次進攻。城中守禦甚嚴。倒反傷了許多賊兵。心中恨毒。決開黃河放水一淹。百萬生靈盡爲魚鱉之食。先是城中聽得賊營傳言。開城之日。不但雞犬不留。掃箒也刴三刀。因此兵民困守。至死心猶不變。被這惡賊放水一沖。幾無孑遺。瞎賊雖出了他的惡氣。但躭誤了許多日子。又一無所獲。他自己的人馬也被淹死了無數。一片汪洋。無處存紮。遂統大隊乘勝破了亳州。那知州金蘇也不知是死了。也不知是逃了。竟無影響\footnote{驚酥了的人。自然是嚇死了。還逃往〔何〕處去呢。}。被這些惡賊將一座城池並周圍數百里之內殺搶一空。且說那時陳州守將姓岑名繼彭。賊素憚他的威名。圍汴之日。恐領兵救援。先差一隻虎李過。同李公子李岩。帶領驍將數員。賊兵一萬。進攻陳州。綴住他這一枝人馬。這陳(岑)總鎭的夫人係楚藩的郡主。幼好兵法。天授神勇。左右婢妾皆佩刀侍立。年十五時。善穿楊神箭。又善雙劍。能飛斬人頭於數十步外。然有柔情。對左右從未有疾言遽色。十七。攻書法。有衛夫人之逸。楚王絕愛之。留心擇配。遍顧羣下。無一當者。時岑君方弱冠。以善騎射補營幕忠顯校。奉帥命入府啓事。楚藩見其氣宇不凡。遂以郡主妻之。及流賊犯境。勢甚猖獗。郡主授岑君野戰法。率壯士五百。大破賊衆。擒賊首鐵棗兒。黃標。胡盧等。論功擢陞副將。旣而張獻忠大舉入寇。又連大敗之。晉銜總戎。坐鎭陳州等處地方。河南沿邊一帶左右不遭流賊蹂躪者。與有力也。瞎賊大隊攻打汴梁。李岩李過進圍陳州。岑君嚴督民兵多方守禦。郡主常授其計。屢出奇兵殺賊。或親率婢妾數百人沖突賊陣。所向無敵。無不披靡。賊衆畏之。李岩向李過道。敵兵猛甚。不可力敵。徒傷兵馬。但設長圍困之。他糧盡援絕。其城不攻自破。堅圍年餘。城中乏糧。樵疏路斷。援兵竟無一至。城下士卒枵腹。不能執戈。賊衆探知。率衆力攻。內不能禦。城遂破。値岑君大病垂斃。郡主即呼家衆。整頓馬匹器械。郡主以帛束岑君。親負之。率署中男女五百餘人。上馬舞雙劍前導。賊衆見其勇莫可擋。盡避其鋒。遂突圍出走。李過不捨。領兵馳逐。郡主命家衆發預制連機弩。一發四十九矢。賊皆應弦而倒。李過不敢復追。引衆回去。郡主捷走百里之外。乃休息人馬。查驗男婦。無一失者。蓋素日純練之精也。李報到亳州。瞎賊知陳州已破。岑君已去。見無後患。心中大喜。便想去取南京。傳下號令。各營且在亳州養息。差了一員心腹大將。官拜權將軍前鋒都統。名喚史奇。綽號一堵牆。領本部兵前行。試看江沿守備如何。並探聽南京軍政如何。快來飛報。這個賊將生得黑面虬髯。力雄性惡。素常自誇梟勇。所向獨前。他乘黑馬。穿黑甲。慣用一桿黑纓鎗。有幾句贊語贊他的形象本事。道。

\begin{quotation}

面如黑鐵。眉間露兩道凶光。聲若巨雷。胸次隱一團殺氣。射響箭以爲生。身長臂大。騎劣馬而作寇。力壯膽雄。腰下囗弓張開處。官軍落魄。手中鐵矟尖到時。百姓魂飛。黑凶神自天降來。瞎闖賊前生惡子\footnote{開首托生多人。惟卜多銀史奇點明來路。}。

\end{quotation}

他領着〈着〉前隊三千黑甲哨兵。殺奔前來。探視風聲。想要攻搶南京。那賊的兵勢好生利害。眞是。

\begin{quotation}

轟天黑天。掣電奔雷。喝水成冰。驅山開路。川岳爲之震動。草木盡皆披靡。深林處。虎豹也潛形。村舍中。雞犬全沒影。

\end{quotation}

史奇來了。不知尚智用何高着禦敵。要知勝負如何。須聽下回分解。

姑妄言第二十一卷終



\endnotetext[1]{「去處」原作「處去」,據文義改。}

\endnotetext[2]{批註「袖」原作「神」,「唱曲」原作「曲唱」,據文義改。}

\endnotetext[3]{「行」原作「形」,據陳鼎《留溪外傳》卷十七《劍客傳》改。}

\endnotetext[4]{「引」字原無,據陳鼎《留溪外傳》卷十七《劍客傳》加。}

\endnotetext[5]{「客」字原無,據陳鼎《留溪外傳》卷十七《劍客傳》加。}

\endnotetext[6]{「兩」字原無,據陳鼎《留溪外傳》卷十七《劍客傳》加。}

\endnotetext[7]{「犄」原作「特」,據文義改;下同,不贅。}

\endnotetext[8]{「十萬」原作「萬十」,據文義改。}

\endnotetext[9]{「義氣」原作「氣義」,據文義改。}

\endnotetext[10]{「有」字原置「辰巳時」三字之上,據文義改。}

\setcounter{footnote}{0}

\theendnotes

\part*{姑妄言第二十二卷}
\addcontentsline{toc}{part}{姑妄言第二十二卷}
\markboth{姑妄言第二十二卷}{姑妄言第二十二卷}

鈍翁曰。岳忠武云。爲將之道。智信仁勇嚴。缺一不可。誠至言也。余閱此回。方悟尚智諸人命名之由。夫爲將者。無智不足以料敵。故尚智爲首。有智而無義不可以馭衆。故慕義爲次。智義全矣。非有一片忠君愛國之心。上不能以報朝廷。下不足以勵士氣。故林忠又居其次。忠雖居三。而實爲智義之首。智義忠備矣。念念不忘朝廷。始足以報國也。三者俱全。尚何敵之不摧。所以屢戰屢勝。諸人爲江北之屛籓。而賊爲之喪膽矣。作者猶恐看者不能會其意。又加一鮑信。特拈出此信字。見智義忠信悉具。爲將之道備矣。看官勿以稗官而忽之也。

屎棋遇常勝之高着。已不能支。何況更逢國手。焉得不滿盤俱空。到狼狽不堪之地。敗逃而去。猶爲萬幸。

李自成自恃兵威。以牛爲軍師。帶了些羊馬狗猴猿鹿獐狐豬。一羣畜類之將。又統的是些羊囗之賊。兼程前來。想敵智義報國之虎軍。眞是驅疲獸而鬥猛虎。多見其不知量也。其敗衄不亦宜乎。寫高傑邢氏。雖獎他棄逆從順。得膺天寵。正是寫李自成壞處。連妻子也不與之同心。又見彼一男兒。猶不如婦人之有見識。又接寫楊氏之私李錦。瞎賊之自詫。總不過是罵他王八。欲辱他之至。然而他三妻皆是實事。非作者寃罵之也。

史奇再來。眞是不知死活。必死於國守之手而後已。寫彼恃匹夫之勇。一旦身名俱喪。誠盜賊而愚者也。

屢屢描寫官兵之醜態。雖是過於形容。然實有八九。枉言者一二耳。亦可供閒中一笑。

姚澤民一死。了却姚廣孝公案。及找及第五回內以完前孽一語。勞正游夏流二人一劫同歸。癆症者不復憂其再發。游於下流者亦更無可下矣。

兪一鳴之女媳一段。不可笑兪春姐之愚蠢不及刁氏之刁滑。以我論之。刁氏之滑終露馬脚。反不如愚蠢之兪氏尚有本心在焉。

\chapter*{姑妄言卷之二十二\\
第二十二回 李闖賊恃勇敗三軍 史兵部加恩酬衆將\\
附 興平伯殺流賊 澤國公完舊孽}
\addcontentsline{toc}{chapter}{第二十二回 李闖賊恃勇敗三軍 史兵部加恩酬衆將}
\markboth{第二十二回 李闖賊恃勇敗三軍 史兵部加恩酬衆將}{第二十二回 李闖賊恃勇敗三軍 史兵部加恩酬衆將}

說話史奇奉了瞎賊之命。領着一隊賊兵。遇城不攻。只沿途搶劫。殺奔前來。到了六合。這次大非昔日之比。堅壁淸野。四境村落中千室萬宅皆空空如也。不但不能搶幾個婦女來取樂。連那豬羊牛馬雞豚鵝鴨酒米之類。想搶些來肥嘴也不能夠。這一羣賊見無東道主人。心中大怒。離城十數里歇了一夜。第二日淸早飽餐。乘着一股銳氣。想來攻城。殺個快活。一來醒脾。二來洩忿。不意到了城下。遙見城門大開。以爲人都逃盡。是座空城了。心中來搶殺的興頭一懈。那銳氣就減了幾分。衆賊還想先到城中。尚可擄些餘剩之物。各縱馬加鞭。正要長驅而入。突然一聲砲響。尚智領着中軍千總繆策。右軍千總滿福\footnote{上智之主軍。又有妙策滿腹。一無謀之屎棋。那得不滿盤皆輸也。}。率着一枝虎頭軍。衝出城來。身上都穿虎紋綿甲。有四五百人。片刀大棍。長鎗鈎鐮。上打人身。下砍馬足。鎗刺鈎鈎。勇猛無比。這羣賊從來十處九處再沒人敢同他對敵。他並不提防這個小縣中竟有人出來廝殺。正是錯愕。起先見他人少。又是步卒。還不介意。不想到了跟前。他也不站隊伍。一味野戰蠻打混斫。從沒有經過這種殺法。措手不及。正遮攔不住。又被那些虎頭亂繞。人身上又是虎紋。馬也繞得眼花。驚得亂跳。衆賊旣要馭馬。又要對敵。正勉強抵鬥。軍少賊多。還掙着支持得住。只見後面一陣聲起。喊殺連天。是那堡子裡分屯的四百兵。一員左營千總姓國名守。白面長髯。銀盔素甲。粉白馬爛銀鎗。如一團瑞雪相似。同着左隊把總卓高。右隊把總常勝。都穿白甲乘白馬。從後面又蠻斫混打起來。史奇同衆賊有些站不住了。偷空就跑。尚智領一百馬兵。持大刀趕殺。命步卒隨後追來。那賊騎的都是健馬。跑得飛快。尚智率衆正追不上。遠遠看見旗旛招展。兩路兵來。流賊正跑之間。看見了。以爲是他家發來接應的後隊到了。把馬倒慢了些。要等他們到來。好一齊殺回報仇。誰想到了跟前。都是虎頭軍士。這是慕義林忠探聽得賊兵來攻六合。他二人各帶了八百名精壯。如飛來應援。正遇賊兵敗走。阻住去路。此時史奇同衆賊要跑。却跑不掉了。只得掙命迎敵。賊衆所恃全是弓箭。他衆人綿甲護住了身子。身上輕。脚下快。一到賊隊前。齊發一聲喊听(叫)打起來。衆賊弓箭無所施展。史奇正在危急。尚智馬步兵又追上了。也喊了一聲。上前一裹。四面夾攻。史奇心正驚慌。左望右望。瞅空兒要跑。早被國守看見。一馬衝到背後。大喝一聲道。黑賊休走。一鎗刺來。史奇回頭一看。叫聲不好。將身一閃。被國守一鎗攮在左肋的甲上。國守急撇回鎗。因用得力猛。把史奇一扯。晃了一晃。幾乎栽下馬來。嚇得他魂飛魄散。恐第二鎗又來。忙伏在鞍上。打馬而逃。那三千流賊。被這些鄕勇也有片刀斫做兩截的。也有大棍子打出腦髓的。也有長鎗刺洞心窩的。也有鈎鐮抓斷手足的。只剩得千餘逃去\footnote{前一回看衆賊之凶惡。不勝恨忿至極。看至此。胸中稍覺一舒。}。國守還要去追。趕盡〈盡〉殺絕。尚智道。不必窮追。且收兵回去。到了城中。一面着人收賊抛棄的器械。一面查點賊首。查明了來回報。共殺賊一千八百餘級。器械若干。馬匹若干。鮑信忙備公文。差人連夜到南京史樂二公處報捷去了。數年來從未聞有此一場大戰而勝。史公聞知大喜。遣官飛馬往京師報聞。再說尚智命衆人都到城中暫且歇息。先令犒勞慕義林忠的軍卒\footnote{好。}。然後治酒席與衆官賀功酬勞。飮酒之間。尚智道。闖賊若得知這一場敗衄。數日內大夥必到。這一次却非今日之比。他來定有數萬人馬。我三千步卒。寡不敵衆。須以良計破之。二位協力成此大功。一則不枉這一番義舉。再者仰報史樂二公知遇之恩。三則使逆賊再不敢正眼覷我地方。林忠慕義齊道。兄有何妙計。我三人同功一體。敢不尊令。尚智道。賊聞敗信。他必憤怒前來。趁他喘息未定。我領兵衝他前隊。二位不必遠去。只在十數里之外養精蓄銳。不住探聽。賊一到來。將欲交鋒。弟素知林兄武勇絕倫。領本部兵橫衝他的中堅。斷他做兩截。慕兄後面殺來。擾他的後隊。與他個三面接應不暇。必然取勝。然此非血戰不能取勝成功。今只激勵衆人。臨敵我等身先士卒。大家齊心併力。何愁不以一當百。衆千把總領了令。率衆出城。分頭屯紮。尚智又向林忠慕義道。但恐賊兵來緩。他銳氣未洩。難以爲敵。須得用一詐降計。誘賊星夜奔來。人困馬乏。庶可成功。鮑信道。三位都立過功了。這一功讓我爲之。遂修了一道降表。其內中之大略云。

\begin{quotation}

前大兵臨城。臣本擬迎降。尚智倔強。恃匹夫之勇。挫辱王師。今尚智偶得小勝。妄自誇大。反欲首臣。心懷二念。臣素知大王天威。四海咸懼。大兵若來。蕞爾小邑。定成虀粉。臣料尚智決不敢攖大王之鋒。若聞大駕親臨。必然遠遁。祈大王星夜直下。出其不意。使彼逃避不及。臣率合城百姓內應。求恩賞賜保全。獲得尚智。獻於軍門釁鼓。上則盡臣仰歸聖主之誠。下可雪陷臣功名性命之恨。云云。

\end{quotation}

差了一個心腹鄕勇。叫做伊策。這人善於行路。一日可步走三百多里。雖快馬亦不能及。故此差他送去。又囑咐他如此如此。不可誤事。後來成功。定有重賞。伊策去了。隨後着探馬沿途打聽。權按過一邊。再說流賊做了這些年的快活賊。逢州過縣到處。官兵遇着就跑。尚恨爹娘少生了兩隻脚。他並不曾張弓隻矢。費一點力氣。要攻城就克。金帛子女。只揀着上好的收了。其餘棄的棄。殺的殺。何嘗吃過這樣大虧。這一回傷折了許多人馬。逃脫的還有小半着傷。一個個抱頭鼠竄而逃。又恐後有追兵。星夜前奔。史奇被國守一鎗。幾乎喪命。魂夢皆驚。眞果是騎豬而竄\footnote{唐武懿宗形質鄙猥。武后命之爲將。大敗而逃。有人作古風譏之。起句云。長弓度短箭。蜀馬臨階騙。中有騎豬向南竄。武后云。懿宗有馬。何故騎豬。對曰。騎豬者。夾豕走也。武后大笑。今史奇亦是夾屎而走也。}。領着敗殘卒衆。到了大營。自縛請罪。報與闖賊道。臣領兵到了六合。不想城中出來一羣士卒。猛勇無比。三四處救應的人馬。四面圍裹殺來。以致大敗。三千人馬只剩得千數回來。失機之罪。自知當死。但聽大王天恩。李自成大怒道。多少大府州縣。尚不敢當我兵鋒。聞風非逃即降。這一個小縣。前番經我殺寒了心的。尚敢如此可惡。問道。你可曾探聽這領兵的將官是誰。是何名姓。史奇道。臣沿途拿得逃民詢問。說這人姓尚名智。是個鄕勇頭兒。近日南京兵部新委了他一員守備。同一個姓鮑的文官。協守六合。自成越怒道。這等的無名之人。何足掛齒。我不殺盡了這些人口。踏碎了這座城池。也出不得我胸中惡氣。正在發怒。忽營門外賊將進來稟道。獲着一個奸細。他說是齎降表來的。要求見大王。現拿在外面候旨。瞎賊命帶進來。他懷中拆開衣縫。取出降表來呈上。瞎賊看了大喜。宋獻策接過看了。說道。他戰勝而後降。恐內中別有詭計。瞎賊大笑道。我素聞爾名。前日破歸德時。我不喜得城而喜得汝。今日何作此迂腐之儒言。孤行兵久了的人。何嘗不想到。諒這一個斗大小縣。他雖有十面埋伏。孤何懼哉。他詐降做甚麼事。況戰勝者尚智也。投降者鮑信也。他一個文官怕死來降是實。何用多疑。瞎賊就不曾想到是誘他速去。要疲困他的人馬。那伊策聽了瞎賊的話。心下暗喜。忙叩頭道。大王天恩。明見萬里。不枉小民萬死一生前來投順。瞎賊命賞了他一個元寶。吩咐道。你星夜回去。對你本官說。我大兵到時。就開門接應。只殺士卒。百姓一人不戮。凡係百姓之家。門上都寫順民二字爲號。成功之後。我得了鳳陽。就陞他知府。叫他城中預備下糧草等項。候我兵到食用。你可快快去罷。伊策叩頭謝恩而去。瞎賊問史奇。此處離六合有多少路。答道。有五百餘里。此時已未末申初時候。瞎賊報仇心急。傳令老營人馬不要動。都留在亳州休息。只選揚武營二萬多精兵。全是馬軍。限兩夜一日趕到六合。遲了恐尚智聞風逃去。不得報仇。此時連夜起馬。後日淸晨到彼齊集攻城。遲誤者斬。又吩咐史奇以每常功勞將功贖罪。免死革職。帶罪圖功。史奇謝了恩。瞎賊選了數員武藝精強的賊將。放砲起兵。他此來想一個縣城中。能有幾個兵馬。先因人少。故官軍偶爾得了勝。這次若知他的人多。決不敢出戰。他命驍將制將軍苟捷綽號東郭盧爲先鋒。以偏將軍侯矯綽號滿山飛爲副。帶領四千人馬爲前部先鋒。着權將軍胡爲羣綽號九尾仙爲左翼。以偏將軍羊委綽號髯參軍爲副。帶領四千人馬繼進。瞎賊自統中軍。領六千人馬。同着軍師牛金星副軍師宋獻策。並護衛將軍馬雷綽號千里足做第三隊。第四隊也是四千人馬。着權將軍章黃綽號麝香囊帥領爲右翼。以偏將軍朱繼溫綽號剛鬣猴爲副。着制將軍兼五路救應使祿奔綽號百花將領四千人馬爲合後。以偏將軍袁滑綽號福緣君爲副。傳令不必運送糧草。只可帶乾糧。後日破城之後。自有食用之物。衆人得令。這些賊到處搶擄慣了。在汴梁躭誤了年餘。久聞城中富甲天下。都以爲一攻破了。金銀還在次。先得美女來取樂。不想一水淹得精光。毫無所得。今聽見去攻城。拿穩是一到就破的。好生樂意。況是當年得過大利的地方。旣無猛將強兵爲敵。且有子女玉帛可搶。是朝暮盼羨的去處。此來興頭得了不得。大家大刀闊斧。長箭輕弓。騎着健馬。連夜奔馳。你道他們的利害。

\begin{quotation}

旌旗蔽日。殺氣喧天。開山斧閃爍生光。流星鎚蓓蕾出色。棗木槊狼牙棍。猶聞磕腦之腥。偃月刀丈八矛。還帶殺人之血。蹂躪得地上草不生。薅惱得夢中鬼也怕。

\end{quotation}

離城約有數十里。又傳下令來。道。若離城不遠。不必定隊。一齊擁上。便去攻城。先入者賞。退後者斬。如有人開門接應。只殺兵卒。不許害一百姓。門上有兩個大字。勿得擅入\footnote{兩個大字。妙甚。兩個字者。順民也。但衆賊不識字者多。故云兩個字耳。若說順民二字。亦無不可。細思之。便覺不通。足見此書之妙。一字不肯苟且下筆。}。那瞎賊領着這些牛羊馬苟侯袁章祿朱胡衆驍將。以爲這一到了。四面圍攻。城中又有內應。前日的那些兵將如甕中捉鱉。一個也走不脫。意氣洋洋。傲然自得。瞎賊的軍令極嚴。行軍傳令。不敢稍誤時刻。天將黎明。這些賊衆人馬兩夜一日不曾大飮食。腹中也有些餓了。又奔馳得有些困乏。離城還有七八裡之遙。正走着。見對面遠遠座頭起處。一隊兵馬到來。這是伊策連夜回來報了李自成兵來的信。尚智領衆出城等候。以逸待勞。賊兵看見。纔往中軍飛報。闖賊正在要立隊時。那枝彪虎軍已衝到面前。只聽得一聲喊。如天崩地塌。刀棍齊施。鎗鈎並舉。這些賊正措手不及。又飛報李自成。瞎賊聽了大怒。催各隊兵一齊快上。衆賊纔縱馬上前相迎。忽然又聽得一聲喊。只見一隊兵從中衝來。刀鎗在左。棍鐮在右。把賊兵衝做兩截。原來是林忠的一枝猛虎軍。李自成正分兵迎敵。後面又一片喊聲。慕義率着飛虎軍殺將入來。三員千總當先。中軍武備。左營全藝。右營殳禮\footnote{妙。慕義之人。而又武藝全備。自然能除李賊也。}。奮勇听(斫)殺。此時流賊不知當有多少官軍。心中一慌。自然隊中就亂了起來。站脚不住。只是想跑。李自成見勢頭凶猛。也有些着忙。突見一彪人馬殺到他跟前來。原來是一員女將。只見他。

\begin{quotation}

金冠束髮髻。銀甲罩嬌軀。一瓣紅蕖挑寶鐙。更顯得金蓮窄窄。兩彎翠黛拂秋波。越覺那玉流沈沈。嬌姿嫋娜。慵拈針黹好輪鎗。玉指靑葱。懶畫鳳鸞騎劣馬。漫道佳人多猛烈。果然閨閣大英雄。

\end{quotation}

你道此人是誰。他就是林忠的妻子國氏。乃國守的胞妹。生得天姿國色。且又英勇異常。慣使一桿家傳的黎花鎗。坐下騎一匹火炭赤兔馬。臨隊當先。較他乃兄還利害幾分。自嫁了林報國。眞是一對英雄夫婦。琴瑟和諧。相敬相愛。這日同丈夫來殺賊。匹馬單鎗。爭先直闖入賊的大隊。鎗到處。那些賊紛紛落馬。李自成見了。又驚又喜。喜的是見了這樣一員標致女將。眞目所未覩。驚的是女人中有如此英雄。比他當日的邢夫人還加倍利害。忙叫章黃朱繼溫二將去迎。兩人縱馬挺着兵器。剛纔對面。只見鎗尖如瑞雪一般。章黃眼睛一花。嗓管上鎗尖早着。翻身落馬\footnote{臨陣章黃。自然要死。}。那朱繼溫吃了一驚。纔回馬要跑。一鎗早中了後心。透出前胸。國氏回手一帶。也栽於馬下\footnote{朱繼溫成了朱遭瘟。}。李自成驚得呆了。正然着急。只見林報國領着猛虎軍直搗中堅。同着中軍千總熊羆。左軍千總猛如虎。右軍千總斑豹。將賊衆衝開。奮力殺進來尋李自成。這林報國兩臂有千斤之力。使一條渾鐵鋼矛。所向無敵。賊衆披靡。如入無人之境。遠遠望見一個金盔繡甲的人。知是瞎賊。直奔了他來。衆賊見了。都來救護主公。上前一裹。將林報國圍住。林報國棄了鎗。拔出雙刀。如風飄瑞雪。雨打梨花。只見一團光亮。衆賊紛紛墜騎。無人抵敵。國氏又看看殺到面前。李自成見勢頭有些不好。料難取勝。領着些護身的驍將。衝開條路走了\footnote{忠心報國之人。領着熊羆虎豹之將殺賊。自如催(摧)枯拉朽耳。}。這些賊先就想跑。因他瞎王在陣中。只得死命站住迎敵。不敢動步。今見他先跑了。誰還肯戀戰。一齊喊了一聲。四分五落。鞭馬而逃。這些虎軍鄕勇見賊敗了亂跑。也分頭追趕。殺得好不興頭。林報國夫妻率領衆賊(軍)追了有數里。追趕不上。方領衆而回。他們這一陣。好一場廝殺。怎見得。

\begin{quotation}

殺大將連人帶馬。追小卒棄甲抛鎗。棍中頭顱。腦頂天庭俱粉碎。鈎傷手足。毫毛筋肉盡分張。丈八蛇矛。恰似蛟龍探爪。虎頭軍士。猶如猛獸驅羊。愁雲黯黯屍橫野。殺氣騰騰血染場。這惡賊以爲殺遍中原無敵手。誰知道今朝到此膽魂消。

\end{quotation}

尚智下馬暫歇。向衆人道。賊雖敗去。未曾大傷。也因是兩夜一日奔馳了五百餘里。人困馬乏。我們是出其不意。攻其無備。故僥倖了一陣。兵法云。百里而趨者蹶上將。正此謂也。若等他歇息再來。養成銳氣。那時衆寡相形。未免難與爲敵。衆賊今日這一場敗走。越發困乏了。可着人打聽賊營離此多遠。我們連夜去劫寨。他必不防。若再成此一場大功。賊必不敢復來。衆位休辭勞苦。衆人道。大家的事。又是將主軍令。焉敢辭勞。國守道。這瞎賊的軍師牛金星同宋孩兒兩個。素常聞人傳說他皆善於用兵。恐有準備。尚智笑道。古云。知己知彼。百戰百勝。瞎賊自猖獗以來。所向無敵。誰還在他目中。我兵今雖小勝。他諒我人少。決不敢去劫營。故此我欲去耳\footnote{兵驕者敗。瞎賊之謂。他二人不可無此一番議論。一見國守之能。二顯尚智之智。}。衆人皆以爲然。俱各飽餐暫歇。到了日晚。尚智約會了衆人。三營齊發。人盡含枚。馬皆勒口。慢慢而走。只見探事的鄕勇來報道。賊的大營離此將四十里。尚智吩咐道。離賊營十里之外再探賊可有準備。若無備時。命衆軍且稍住。吃些乾糧。喘息一會。到三鼓時。等賊睡熟。我同慕兄四面斫入。林兄同尊嫂各領兵埋伏在數里之外。俟賊敗走。斷他的歸路。雖未必擒得瞎賊。也殺他個膽寒。衆人齊道。遵令。却說李自成敗跑了有三四十里。打聽並無追兵。吩咐安營。將晚時。敗兵都到了。他傳了衆將到跟前。道。今日之敗。是我大意了。以爲他不敢出來。故不曾防備。二來我的人馬都困乏了。因此敗了一陣。命查點折了多少人馬。五營中查了一會。來回復道。還有二萬來人。折了不過頭二千名。李自成道。我看他不過四五千人\footnote{四五千人。妙。對陣之時。三處殺來。是似人多。約略之辭耳。若俗筆云將三千人。豈李自成曾替他點兵耶。}。我四個對他一個。還怕殺不過他麼。傳令各營。打草餵飽了馬。人雖沒有帶糧米。把帶傷的馬宰了。同着帶的乾糧。飽吃一頓。睡他一夜。明日五鼓。再各飽餐。好去報仇。臨陣之時。不必站隊。一味野戰。認定四五個人戰他一個。再無不勝之理。殺他個片甲無存。一個也逃不脫。定要把這座城池踏平了纔罷。軍師牛金星道。恐他今夜乘勝來偷劫我們的營寨。大王不可不防。李自成大笑道。軍師何不智之甚。用兵之道。要知己知彼。人素聞我的軍聲。誰不膽怯。今日他僥倖得勝。自以爲萬幸了。焉知他不疑我是詐敗。恐我連夜攻城。他自守不暇。有多大膽子敢來捋虎鬚。劫我的營盤。他若果有膽量。見我敗了。何不來追。這就可見他的膽懦。只管叫孩兒們放心去睡。養息精神。明日廝殺。衆賊聽了這個令。都是乏倦了的。心中好生快活。吃飽了。倒下頭。也不管天南地北。都放心高臥。不意到了半夜。衆虎軍到了他營盤外面。悄悄四圍拔去鹿角。闖進重圍。喊聲大震。殺將起來。衆賊睡得正濃。夢中驚醒。人不及甲。馬不及鞍。黑影中連兵器都摸不着。只顧逃命。這些鄕勇見無準備。心中一喜。勇力倍加。如虎入羊羣中。混斫混殺。星光之下。只認着沒虎頭的斫戮。這些賊四處亂撞。自相踐踏。李自成見黑影中難以交兵。又是夢中驚醒。也就慌了。打着馬。帶了些親隨。馬兵在前衝開一條路。瞎賊在中。牛金星宋獻策緊緊跟住。死命撞出。奔逃而去。直殺到天明。眞果是屍橫遍野。血流成渠。這些賊兵殺的殺了。跑的跑。尚智道。我們快上前去接應林兄的兵馬。這些鄕勇得了大勝。心中歡喜。一些也不覺辛苦。越發興頭。如風魔的白額大蟲一般。聽說往前接應。皆奮勇爭先。如飛而去。不上數里。早遇見林忠領衆奏凱而回。問他李自成下落。答道。我正設伏等候。李自成帶領着二千多敗兵逃了來。被我攔住。他見沒路了。死命相持。被我立斬了四員賊將。雖他的人多。因着了驚。又有一小半沒有兵器。被我衆軍也殺了許多。正殺時。他的敗兵陸續到了。約有萬餘人。我見他人多勢衆。兵法云。歸師莫掩。窮寇勿追。只得放了他一條生路。隨後又趕殺了一陣。賊去遠了。我纔領兵回來。衆人聽了大笑。查點所殺賊人有七八千個。所獲盔甲器械無數。再說李自成被林報國殺敗。攆了一程。見他收兵回去。纔放了心。正然走着。忽聽一聲砲響。看時。兵雖不多。爲首一員猛將殺奔前來。瞎賊急看時。你道他怎生打扮。

\begin{quotation}

頭戴三叉紫金冠。身披爛銀鎖子甲。襯着那雪白素羅袍。袍上織着金翅鵰。左邊袋內揷雕弓。手右壺中攢硬箭。手中搭枝\endnotemark[1]丈二桿鎗。坐下騎一匹赤兔渾紅馬。那馬好馬。眞是。掣斷紫絲握玉轡。火龍飛下九天來\footnote{國氏裝束前已贊過。今又重寫一番者。因李自成中箭。特爲此雕弓硬箭四個字耳。}。

\end{quotation}

李自成認得是那員女將。心下大慌。馬雷恐傷了瞎主。只得上前迎敵。只聽得嬌聲一喝。道。逆賊慢來。馬雷背上早中了一鎗。負痛逃命。李自成也落荒而走。國氏放下了鎗。拔出寶雕弓。搭上狼牙箭。認着瞎賊射去。不意匆忙。把頭低了些。中了瞎賊後股。晃了兩晃。幾乎墜馬。乃忍痛加鞭。飛馬而逃。國氏見去遠了。也不窮追。只趕殺賊衆。大獲全勝而回。到營中將前事說了一番。衆人無不稱贊。尚智又差人打聽李自成的下落。次日回報。已連夜回亳州去了。衆人方收兵回來。鮑信申文備言一連兩陣。斬獲賊首萬級。賊將數員。並所得之物。彙報了功。其敍功文內云。

\begin{quotation}

兩次得勝。皆林忠夫婦功爲第一。林忠斬將搴旗。追奔逐北。親冒矢石。鼓勇爭先。其妻國氏臨陣。先斬賊將二員。賊首喪膽。得獲全功者。國氏先威之力也。後分兵埋伏。又刺傷賊將。李自成中箭。幾爲所擒。殺賊之功。無如國氏。尚智運籌帷幄。身先接戰。慕義繞賊後隊。亂彼軍心。夤夜劫營。逆闖奔逃。皆二人之力。功爲次。衆千把總俱有斬馘之功。驅馳之勞。又爲次。

\end{quotation}

尚智三人也申報。

\begin{quotation}

獲此大勝。乃鮑信詐降誘賊之力。賊兵兩夜一日奔馳五百餘里。人困馬乏。因此得以成功。云云。

\end{quotation}

史兵部見報。大喜。連夜報捷。奏請恩賞。樂府尹亦題奏鮑信參贊畫策。誘賊成功。崇禎皇帝屢年遣將調兵。花費了多少錢糧。如石沈大海一般。從未見報一場大捷。今日見報殺了〈殺了〉這些賊衆。且又不費國家一分糧餉。聖心嘉悅。奉旨。

\begin{quotation}

前暫委者。皆着實授。慕義林忠尚智加授都督僉事職銜。林忠外加軍功二次。妻國氏封英杰夫人。其鄕勇之家。並出供給戶口。蠲免本年錢糧。鮑信陞應天府經歷司經歷。仍駐三縣。該本知道。

\end{quotation}

部文到了南京。史公差官傳了。慕義衆人都到城中來。謝了史樂二公。這一回是實授了。與前自是不同。二公見他們屢著功蹟。替他面上爭光。便着實優待。都賞了花紅羊酒。又設席賀功。又傳諭三縣。與他們各建衙門。都有衙役執事等項。鮑信又稟。詐降虧伊策入虎穴。求恩旌賞。樂公賞了他一個元寶。史公見他是一條好漢。賞了一張外委把總箚副。就做鮑信的羽翼。專一打探軍情。飛報緊急事務。伊策叩謝了。鮑信到了經歷司的任。這一次家中比先分外熱鬧。開席唱戲。請慕林尚三人並衆千把總。連賈文物也送了賀儀來。他特設席請賈文物。拜謝提攜之恩。約了他堂弟鮑復之相陪。含香也特請富氏並金銀珠玉四位姨娘。也約了貞姑並他一個小姑。是鮑復之的親妹\footnote{貞姑上吊時已曾出名。}。同來陪侍\footnote{此小姑。鍾自新之妻也。鍾自新尚未出現。此女已提過兩筆。想作者著書時。早已通篇想到矣。}。再說李自成領着萬數敗殘人馬。逃奔了百餘里。眞是風聲鶴唳。草木皆兵。聽得後面沒人追趕。纔放了心。查點部下衆將。只剩得軍師牛金星宋獻策馬雷侯矯袁滑五人。其朱羊苟祿章胡六將俱死於衆虎軍之手矣\footnote{牛金星宋獻策係賊之文官。不必論。但許多賊將盡皆授首。只逃得三人回來。而馬雷亦曾中傷。只侯矯袁滑平安而回。可見世人不但狡滑者處處佔便宜。即做賊者。亦狡滑者得便宜。}。一場掃興。數日跑到亳州老營。衆賊將接着。請罪道。沿途飛報。雖聞大王失機。因無王命。不敢擅動。李自成道。這兩場敗衄。是我恃勇欺敵之過。太藐視他了。衆將道。諒此小地方何足介意。以我之衆。投鞭可以斷流。長江可以騎渡。臣等帥領老營人馬前去復仇。寸草不留。毀城塡池。以出大王之氣。李自成道。不消了。我所爭者不在這彈丸之地。如今幹大事要緊。我的兵將前攻汴梁。將及二年。這幾個月又勞苦了。今在此休兵一月。四路劫些糧草財貨。且回陝西厲兵秣馬。等強壯了。奪了北京。那時江南一帶自然入我掌握之中。不怕他飛上天去。衆賊將道。大王高見。非臣等所能及也。瞎賊遂拔了八根令箭。差一隻虎。紫金梁。蝎子塊。闖塌天。滿天星。點燈子。混世王。顯道神等八員賊將。往正東。正西。正南。正北。東北。西南。東南。西北。各鬮一處地方。每人領馬步精兵三千。八路分頭搶擄。限一月繳令。一隻虎鬮着淮泗一路。這一隻虎名李過。就是李自成的親姪。當日同他一同逃出來的。他領了三千卒兵到了徐州地界。將衆卒分做十數股。二三百一夥。四散搶劫。他獨守老營。只留了三十多名小卒使用。這些賊向來無人敢敵。操大了膽的。往各縣各鎭各鄕村去搶擄。那時徐州鎭有姓高者名傑。混名叫做翻山鷂。因他身矮又稱他爲高矮子。他生得五短身材。膂力出衆。使一根四十斤的鐵棍。所向無敵。他當日落在賊營。與李自成李過。三人曾結盟爲弟兄\footnote{叔姪結盟。只有水滸傳中鄒淵鄒潤。此書李自成李過。他書更無所見。}。李自成見他是一條直性漢子。托以心腹。李自成的妻子邢氏生得美艷異常。古人有兩句話正好贊他。道是。

\begin{quotation}

比花花解語。比玉玉生香。

\end{quotation}

這樣一個嬌媚婦人。却能騎劣馬。善使雙刀。又足智多謀。時常參畫軍機。十中八九。衆賊兵將都懼他幾分。他甚有恩到人。人却又感念他。所以李自成愛他不啻至寶。他却視瞎賊如贅疣。李自成聽得張獻忠的妻妾子女爲官軍所獲。監在襄陽獄中。他因不時出外四處流殺。恐老營有失。知高傑的武藝高強。遂將邢氏托付與他照管。命他執掌內營事務。那邢氏這樣個伶俐婦人。被李自成搶來做妻子。是無可奈何從順的。豈肯終爲賊妻。他見高傑身雖爲賊。頗有忠義之心。算計要同他歸順天朝。一時不敢出口。又見高傑相貌堂堂。方面大耳。虎臂熊腰。一表非俗。也較李自成強了許多。又有些心愛。遂想了一個主意。一日。瞎賊領衆他出。邢氏趁這空兒。請高傑到內帳來商議軍務。他們皆以叔嫂相稱。說了一會。看上酒來對飮。高傑雖是一條好漢。却免不得酒色二字。他酷好杯中之物。邢氏有心。叫侍婢們頻頻相勸。高傑也談笑痛飮。不多時。便入了醉鄕深處。隱几而臥。邢氏叫幾個心腹侍女擡他上床。脫了衣服。邢氏也將衣褲卸盡。與他共枕同衾而臥。那高傑一覺直到五鼓方醒。猶在半酣。見傍邊睡着個婦人。一陣鬢雲香氣。沁入腦髓。用手一摸。體滑如脂。再摸到那消魂之處。即鐵漢也忍不過了。他也不知是誰。更不問所從何來。一時高興。一翻上身\footnote{是個翻山鷂。}。就抱着雲雨。一個是能征的女帥。一個是慣戰的將軍。兩下綢繆。不肯便住。高傑使慣了鐵棍。此時他那肉棍也像鐵的一般。奮勇長驅。那邢氏好像後西遊上的那長顏姐姐不老婆婆。被小行者一頓金箍棒搗得意亂心迷。那玉火鉗那裡還架得住。把一個邢氏被他搗得骨軟筋酥。癱於枕蓆之上。天色漸曉。高傑定睛看時。方知是邢氏。到了這個局中。也講不得名分了。見邢氏面頰緋紅。微微含笑。雙眸略閉。氣喘吁吁。心愛得了不得。復逞威風。又是一場大戰。邢氏每常同李自成交歡。他那短而小。小而快的本事。須臾吿竣。何嘗經過大敵。此時被高傑弄得四肢癱軟。嬌聲婉轉。求他罷戰休兵。高傑見日上三竿。也就雲收雨散。到了晚間。邢氏又請他進來。對飮了數杯上床。二人乘着酒興。這一齣非同小可。

\begin{quotation}

一個鐵棍馳名。一個雙刀出衆。鐵棍馳名。托(把)雙刀搗開兩半。雙刀出衆。把鐵棍箍成一束。那鐵棍進出無休。這雙刀收放不定。正戰時。那鐵棍如經火煉。漸漸軟來。罷戰後。這雙刀一似水磨。涓涓流出。使鐵棍的。將鐵棍收入囊中。用雙刀的。把雙刀夾攏皮內。說不盡他二人千般恩愛。形不出他兩個萬種風流。

\end{quotation}

事畢後。邢氏枕上勸他道。你我二人情同伉儷。雖死亦不忍分拆了。此事若大王回來知道。性命定然難保。即使不知。也要把恩情打斷了。況你全身武藝。何不貨與皇家。倘爭得個功名富貴。顯身揚名。耀祖榮宗。封妻蔭子。名垂竹帛。留個好名於後。不枉天生我材。但爲人在世。爲甚麼托一個父母淸白之軀。陷於賊黨。使萬世唾罵。況古來爲逆賊的人。可有個善始善終的麼。你若頓然歸順朝廷。不但轉禍爲福。你我又可永爲夫婦。若不早決。恐一事露。那時想脫其禍就不能了。你心下如何。高傑被他提醒了。如夢方覺。答道。你乃金玉之言。我亦有此心久矣。因不知你的心腹。不敢啓齒。旣然如此。事不宜緩。次日。遂同邢氏明公正氣成了夫妻。殺牛宰馬。待宴合營兵將。衆人素服他兩個威德。並無一人背言背語。三日後。他夫妻傳齊了衆將卒兵。邢氏說道。大王叛逆朝廷。恣意屠殺。天怨人怒。目下雖屬苟延。終久定然喪敗。你我都是朝廷淸白黎民。被他擄掠到此。異日一敗。徒死無益。我今已嫁了高將軍。同高將軍商量棄逆從順。歸順天朝。你們有願隨去者。便一同歸順。如有不願者。我也不能強逼。衆人齊聲道。我們在此從賊。因他托(把)我們家中殺盡。無處可歸。也是沒奈何的。誰不願爲良民。情願隨夫人將軍同去歸順天朝。邢氏聽得異口同音。不勝大喜。即同高傑率領着守老營的三千人馬。投順了天朝。有四句打油贊〈贊〉刑氏。道。

\begin{quotation}

莫欺閨閣更無奇。明眼娥眉自可兒。

能配英雄歸帝室。致令芳譽萬年垂。

\end{quotation}

那時聖心大悅。先賜了高傑一個參將職銜。便命他領本部人馬殺賊立功。他在賊營久了。深知賊營虛實。故旗開得勝。馬到成功。屢建奇勳。不數年。加封了興平伯。掛總兵印。統領精兵五千。鎭守徐泗宿亳十四州縣。刑氏也封了一品夫人。一日。各地方來報。說一隻虎李過領賊兵三千到本州界內。分作十數處。四散搶劫。老幼被殺。婦女全擄。金帛糧食毫無留剩。十分凶暴。高傑笑道。這該死的賊奴。他佔住了我的亳州。我因兵少。不能分兵去同他爭奪。他反敢來攪我的地方。他在別處橫行搶殺慣了。官軍不敢與敵。故此大膽。分兵四出。今日公然到我地方上來放肆。且叫他吃我一場大虧。纔知道我的利害。遂傳了六員將官。給了六根令箭。吩咐道。衆將各領人馬五百。探聽何處有賊。即往撥滅。他旣顧子女玉帛。又素常藐視官軍。你們但齊心用力。自無不勝。倘逡巡畏怯。稍有挫衄。定按軍法。如殺盡一處。亦不許再去。即回來繳令。若貪功違令。雖得功亦斬。他六處受傷。也就膽喪了。我兵回來。且養息銳氣。防他來復仇。我以逸待勞。一鼓而破之。我所轄境內。以後便可安枕。衆將領令去了。這些賊也有三百多一羣的。也有二百多一夥的。到處逢人便殺。遇物即搶。只留着少年女子作樂。此數百里之內。竟無雞犬之聲。他們的馬匹都馱着財物糧食。婦女們都是步行。鞋弓足小。一日走不得二三十里。衆賊也緩步而行。沿途搶得食物甚多。慢慢的同着婦女們說說笑笑。其樂無窮。被官軍打探明白。如風馳電驟般趕來。衆賊的馬又馱着寶貨。要棄了跑又捨不得。正在兩難。官軍已到。賊少軍多。圍裹上來。如砍瓜切菜。這夥賊中有顧命不顧東西的。也還跑掉一二十個。其餘盡做無頭之鬼。子女財帛盡數奪回。還有一兩起賊在村中住着。正同婦女們飮酒歡呼。都吃得醺醺大醉。忽見官軍殺到。一個個手足無措。惟是引頭受戮而已。六處皆得全勝。各回繳令。高傑命四處傳諭。叫被難的鄕民來認妻子家貲。無主者犒賞士卒。且說那李過正在營中。見賊兵近處的陸續送到婦女金帛。源源而來。他好生樂意。雖沒有粉黛三千。又不止金釵百二。左顧右盼。欣然自得。正選了幾個上等的婦女飮酒作樂。忽然來報有六處人馬大敗而回。李過聽得吃了一驚。叫進來問時。說各處盡皆滿載而回。金帛婦女無數。因不曾隄防。被翻山鷂部下的兵突然衝來。皆爲所殺。所獲之物盡皆奪去。一千五百餘人僅逃回一百多名。李過聽了。一場掃興。急得暴跳如雷。大罵道。這沒良心的矬賊。我們大家結拜一場。大王以心腹待你。托妻寄子。你把夫人都拐了去。那不礙着我的。倒還罷了\footnote{妙。極是良心話。却是賊口中語。}。今日又傷我這些人馬。我同你誓不兩立\footnote{一個大王的夫人反不如部下的賊。}。命四處的兵齊到營中。查明了數。只剩一千六百餘人。帶傷者却有一半。李過越氣得腹內生煙。留下百餘人看守這些婦女財帛。帶着一千五百人揚武耀威。正奔徐州。到了城下。見城門緊閉。城上並無一人守禦。靜悄悄的。李過怒道。這廝縮頭藏頭。我就罷了不成。叫衆賊喊罵。罵了多時。喉嚨都叫乾了。總不見一人答應。他愈加忿怒。喝叫衆賊道。他旣然不敢出來。我們難道就饒了他麼。你們大家接肩爬城進去。看他往那裡去躱。衆賊不敢不遵。二來也只當他畏縮。故一齊下了馬。拽起衣服。放心大膽。帶着利刃。便齊往上爬。剛爬到半中間。一聲砲響。鼓角齊鳴。城上旌旗密布。劍㦸如林。滾木壘石灰瓶如雨點般打將下來。衆賊急忙退時。已結果了七八百個性命。李過氣忿塡胸。還催着叫上。衆賊料想爬不上去。誰不惜命。正是你我推諉。忽又一聲砲響。南北二門大開。兩枝兵馬齊出。呐喊搖旗。直奔了來。那賊兵見不是勢頭。呐一聲喊。上馬就跑。李過止遏不住。只得也隨着跑。又見幾個敗殘的賊迎面跑來。道。將軍。不好了。大營被翻山鷂襲破。把我們的全殺了。只剩了我們幾個逃得性命。特來報信。李過同衆賊又吃一驚。心慌無主。只得往前奔走。又跑了數里。遠遠望見旗幡招颭。一枝兵馬擺在面前。擋住去路。李過此時也有些膽怯。前有攔阻。後有追兵。又沒處跑。只得領着人馬上前。遠觀不曉。近看分明。只見寶纛旗下爲首一員大將。正是高傑。金盔蟒甲。玉轡雕鞍。身坐白馬。手持鐵棍。威風凜凜。氣槪昻昻。不像當日爲賊的樣子。左右簇擁着許多將佐。雄糾糾好不威武。他仇人相見。分外眼紅。大罵道。你這矬賊。負了大王。拐去夫人。今日旣傷了我的人馬。又還攔我的去路。昔年口血未乾。你不怕鬼神殺你麼。高傑大笑道。逆賊。順天者存。逆天者亡。古云。君非而友是則順友。友非而君是則順君。我一個堂堂丈夫。淸白之體。昔日陷身逆賊。不得已焉。我今日荷蒙聖恩。身爲大帥。坐鎭一方。只知殺賊而已。昔日之盟。何所爲據。聖人云。要盟也。神弗聽。爾知道麼。我看昔年一日之雅。快下馬投降。我待你以不死。若不知止。只怕你此刻就作無頭之鬼了。李過怒極。罵道。別人怕你。我不怕你。今日同你迸(拼)個你死我活。正說着。官軍隊裡一員將官將令旗一揮。鼓聲大震。衆兵呐一聲喊。上前四面一裹。箭如飛蝗般射來。從賊紛紛落馬。李過心慌。東衝西突。想尋出路。無奈如鐵桶相似。正在危急。只聽一棒鑼聲。官兵就停住了箭。又一員將官將令旗一展。衆軍撒開了。讓出一條大路。李過見空。打馬如飛而跑。見高傑立馬在前。用鐵棍指着道。饒汝一命。以全向日之誓。勸你叔叔早早歸降。不失王侯之位。如或執迷。恐噬臍無及。則悔之晚矣。李過知是高傑放他一條生路。也不敢再罵。也沒得話答。只低頭鞭馬而去。跑了十數里。回顧衆軍。只有四百來人。正走之間。只見個土坡上一員女將。束髮冠金鎖甲。手執兩口雁翎刀。坐下一匹桃花馬。打一看時。正是邢氏。有幾句贊道。

\begin{quotation}

雲鬢堆鴉。恰襯桃花之面。金蓮簇鳳。偏宜湘水之裙。星眸略轉而微露凶光。鶯語乍聞而中藏殺氣。容嬌力壯。知爲善武之姬。性巧心靈。信是能謀之婦。不意閨中柔婦女。能爲陣上猛將軍。

\end{quotation}

他貼身簇擁着有三四十個女卒。都是頂盔貫甲。手執器械。遠遠有百十名將卒圍護。聽得邢氏嬌聲嫩氣的叫道。李過。你認得我麼。你看我歸順了朝廷。今做國家命婦。何等榮耀。你們爲賊的有何好處。何不歸降。自取富貴。李過大罵道。無恥的淫婦。你撇了我大王。同高矮子私自逃來。你還不識羞恥。敢向我饒舌。邢氏道。我棄逆從順。何恥之有。我叫你這不識時務的逆賊立刻作刀下遊魂。那李過見他左右的不多。向衆賊道。你們奮力上前。若擒獲了這淫婦。不但可報仇雪恥。且大王定有重賞。衆賊也圖僥倖。就喊了一聲。齊向山坡上奔來。馬快的先到了。山坡下一聲響。天崩地塌。都攧入陷坑中去。後面的急收住馬時。已下去了二三百個。李過正然錯愕。邢氏背後一聲喊。兩枝人馬自山坡後分兩翼殺出。李過顧不得衆人。打馬先逃。逃得出去時。只剩得殘兵二十餘個。一同去了。這是高傑夫妻定的妙計。只殺他個膽寒。却不傷他的性命。他是瞎賊的姪兒。若殺了他。李自成定然全營來報仇。不但怕衆寡不敵。就殺個平手。未免損傷人馬。況且殺他個罄盡回去。使賊營中知道。自然膽怯害怕。這也是先聲奪人之意。他夫妻得勝。率領着人馬。正是。

\begin{quotation}

喜孜孜鞭敲金鐙響。笑吟吟齊唱凱歌回。

\end{quotation}

高傑回城。犒賞了將士。又差夜不休星夜探聽闖賊的消息去了。再說李過帶了二十來個殘兵。連夜奔到亳州。見了瞎賊。說到了徐州。不料高傑在彼鎭守。出其不意。被他將我人馬殺戮殆盡。所擄金帛子女皆被奪回。李自成大怒道。這負義忘恩的矮賊。我恨他深入骨髓。常恨遇他不着。今日狹路相逢。如何放得他過。這正是。

\begin{quotation}

踏破鐵鞋無覓處。得來全不費工夫。

\end{quotation}

吩咐衆將道。留下一半人馬看守老營。等他們的七路人馬回來。同着固守。俟我得勝來時。一同回去。其餘將卒盡隨我去赴敵。衆卒得令。次早放砲起營。高傑探事的夜不收(休)打聽明白。星夜回來報信。高傑差人飛馬賫文到南京兵部處。報賊來有數萬之衆。乞發援兵。內外夾攻。方可取勝。史公見了。連夜檄靖南伯黃得功\endnotemark[2]火速應援。這黃得功算疆場第一員名將。他有萬人無敵之勇。每常上陣殺賊。匹馬當先。左腿上夾一鐵鞭。右腿下夾一鐵鐧。手執鐵鎗。腰跨兩張硬弓。兩個從人背二百枝箭做兩壺。緊隨身畔。離得賊遠。便左右開弓。箭如連珠一般。從不虛發。近則用鎗。殺得性起。便棄了鎗。一手執鞭。一手執鐧。直入賊隊。兩手齊打。賊人紛紛落馬。見他無不膽寒。賊軍中起他個綽號。稱他爲黃闖子。有個醉翁子小令贊他。道。

\begin{quotation}

面赤如重棗。虬鬚飄裊裊。神梢沒遮攔。千軍視等閒。屢戰威聲烈。踏碎沙場月。駿騎一聲嘶。衝營逐電低。

\end{quotation}

向日流賊八大王張獻忠蹂躪滇黔楚粤一帶地方。他有一個龍陽小將姓張。容如美女。力雄萬夫。臨陣當先。所向無敵。賊中稱他爲小張侯。官軍將卒聞其名者。無不遠避。他常聽得人說黃闖子之名。笑道。我恨不遇彼。若遇見。必活擒之。黃得功聞得此語。勃然大怒。領兵特尋了去與他對敵。他聽得黃得功來了。心中大喜\footnote{欺敵者必敗。此賊之謂也。}。次早列成陣勢。匹馬往來弛騁。索黃得功交戰。黃得功聞知。滿飮數斗。披甲上馬。馳出營門。見那小張侯正耀武揚威。在那裡賣弄。他一聲大喝。縱馬直衝到面前。那賊措手不及。被他活挾而回。餘賊喪膽。抛旗撇鼓而逃。被衆兵趕上。殺了個罄盡。所以黃得功的聲名。流賊聞知。無不亡魂喪魄。他將那〔小〕張侯拿了來時。到中軍帳坐下。笑道。你每常誇嘴。說要生擒本帥。今被我拿來。你有何說。那賊低頭不答。黃得功素知他驍勇。要勸他歸降。做一員佐將。說道。本帥素知你是一條好漢。你若肯歸順。將來富貴不小。他也不答。黃得功怒道。我聽得張獻忠愛你。常置你於腹上共臥。若順了本帥。本帥亦以此情待你。你意如何。小張侯只是低頭不答。黃得功叫左右帶去鎖禁看守。這賊竟數日不食而斃。黃得功雖惱恨他。又憐他是個賊中忠義好漢。命將他埋葬了。史兵部久聞他的名譽。特補他滁和總鎭。奉旨加封侯爵。統轄滁和廬無(蕪)爲(各)州十一州縣。他這日見了兵部的來文。又是鄰郡有事。即點精兵三千。前往赴敵。史公又令慕義林忠尚智速回。各領兵卒緊防三處疆界。倘有賊兵。隨機剿戮。他三人領命去了。且說李自成領着數萬人馬。到了徐州地界。不一日。到了城下。見城中守禦甚嚴。滾木礌石堆滿。却不出來交鋒。李自成傳令。叫衆賊在城下辱罵。罵了兩三日。他總不偢不倸。只當是不曾聽見。總不出來。闖賊心中大怒。正造雲梯。要想攻城。忽聽得報馬來報。黃闖子領兵來救援了。已在五里外安營下寨。闖賊素常怕的是黃公。聞得此報。心中便吃了一驚。這黃得功安營歇了一夜。次早乘着銳氣。帶領人馬前來索戰。闖賊傳令各處俱出迎敵。兩下擺成陣勢。闖賊遙望他的兵馬不多。還不介意。對壘多時。兩無勝負。猛聽得連珠砲響。背後三面呐喊。官軍蓋地蜂擁殺來。原來是高傑先因遜他的鋒銳之氣。故堅守不戰。今過了三日。知他銳氣漸消。正打點要同他見個勝負。聽得黃得功兵到。同賊交鋒。他心中大喜。自領了二千健卒。從背後殺來。命兩員將官各領兵一千。分左右衝突。這些賊數年在各處打降。官兵見了。不是疾走如飛。就是束手待殺。他殺現成的慣了。今見這些軍將與別處大不相同。奮勇長驅。竟一鎗一刀的要來對敵。就有些膽怯。況且高傑當年在他們營中時。翻山鷂的利害人人知道。這黃闖子的威名遍於賊中。聞名喪膽。每常偶然相遇。就遠遠的避開。正今日竟同廝殺。已懷着鬼胎。因他先聲素著。俱恐頭顱不保。若只遇他一個。還可勉強抵敵。今他二人在一處。前後夾攻起來。不由得心中害怕。正分頭迎戰。那黃得功見了高傑領兵四面殺來。如虎添翼。越發鼓起他的威風。大吼了一聲。猶如半空起個暴雷。右手持鎗。左手執鞭。帶領着隨身鐵騎。衝入賊陣。他標下的衆將見主帥爭先。焉敢落後。一齊奮勇殺上。把賊兵衝作四分五落。站脚不住。高傑見賊衆驚慌。也催兵混殺。自辰至申。闖賊看他的人馬漸漸喪失。知不可敵。遂率領衆將。招呼士卒。敗逃而去。高黃二帥見他的賊衆尚多。也各收兵回營。高傑到黃得功營中相會。謝了他救援之德。商議道。我兩人部下不足萬人。賊有數萬之衆。難以撥滅。若只力敵。恐受傷者多。須如此如此行之。不但此圍可解。我兩家的兵馬又不得折損。主意定了。兩人分頭行事。高傑回城傳令。城中只留下一千兵。命合城百姓皆給以盔甲。各執旌旗器械。都上城守護。托邢氏帶領文武督帥。自己暗暗領兵。連夜去了。那李自成敗回營中。怒道。我自行兵以來。未嘗屢敗。前敗於六合。今日又在此失機。這一口氣如何得出。李岩道。勝敗軍家之常事。大王何必介懷。今日因四面受敵。故此傷折。明日將衆兵養息一日。後日同他見個輸贏。我們的軍馬多他數倍。用更番之法。再無不勝之理。李自成道。何爲更番之法。李岩道。將我們的兵馬分作三隊。先出第一隊對敵。約兩個時辰。第二隊上去。將頭隊換下入營暫歇。又兩個時辰。第三隊上去。又換下第二隊歇息。又將第一隊換第三隊。輸流換隊接殺。雖連戰三晝夜。人馬亦不困乏。在我甚逸。彼則甚勞。人之精力有限。他能以一隊熬得過我三隊麼。他即欲分兵。則人少而不敢。此晉三駕疲楚之法也。李自成聽說。大喜。次日休養了一日。第三日早。衆賊埋鍋造飯飽餐了。備馬披甲。打點廝殺。衆賊將領着頭隊賊兵出了營門。揀寬濶處擺下隊伍。遙望黃得功營中微有煙起。靜悄悄不見一些動靜。遂掌號擂鼓。呐喊連天。直逼將過去。仍是如此。離營不遠。上高處瞭望營內虛實。見虛揷旌旗。原來是一座空營。忙報知李自成。差人去探聽。探事的回報。果然一人也無。李自成尚持疑不信。又遣兩員將進去看實了。然後親到營中去看。見糧草堆積。各帳房中兵士的衣服行囊全然未動。甚是動疑。再敎人到城下打聽。報說比前日防守更嚴。女牆邊士卒布滿。宋獻策說道。黃闖子忽然棄營而去。彼素知兵。以臣愚意度之。莫非爲圍魏救趙之計麼。李自成猛省道。此或有之。正說着。只見探馬飛來報道。黃闖子直搗亳州。暗襲老營。斷我們的歸路。已去了一日一夜了。李自成聞報。心下正在慌張。或(忽)又有數騎來報。左良玉知汴梁已失。自襄陽領四十萬大兵前來復仇。瞎賊聽得愈慌。恐老營中沒有大將。抵敵不住。傳令即刻拔營。連夜回救。奔馳了一日一夜。到了盱眙縣界。忽聽得背後砲響。回頭一看。見是高傑的旗幟。呐喊追來。衆賊無心對戰。且戰且走。後面追兵也不甚力戰。只是追趕。又走了數十里。一聲砲響。一彪軍擋住去路。當先一將笑喝道。認得黃將軍麼。衆賊看時。果然是他。闖賊見斷了他的歸路。無可奈何。只得催兵上前混戰。後面高家的兵漸漸追上黃家的。也戰不甚利。殺了一會。閃開一條大路。衆賊趁勢衝出。黃得功同高家步卒趕了下來。這些賊跑了兩日一夜。都不曾造飯。雖吃些乾糧。都人疲馬乏。看看日暮。正在飢渴之時。思量要紮營暫息。忽然一派鼓聲震耳。一枝人馬衝出。只見高傑領着將卒撞入隊中。揮刀亂砍。李自成驚道。此處如何又有這矮賊。料難迎敵。奪路先跑。這些賊只顧逃命。往前直奔。落後的皆被誅殺。高黃二帥統兵趕了一程。天色已晚。賊去遠了。纔收兵安歇。次早遣人打聽。回報賊兵都回亳州去了。二人連勝二陣。斬賊首萬餘。奪得軍器無數。差官露布往南京報捷。他二人回兵到了原營。高傑請黃得功同諸將進城。設宴賀功酬勞。遣官賫牛羊酒來犒勞士卒。黃得功次日辭別。領兵回鎭去了。這就是他二人定的妙計。二人領兵。連夜啣枚疾走。使賊不防。到百里之外二百里之內伏兵休息。故意聲張。假說去襲老營。使賊聞知。不敢攻城。定然星夜回救。又令人四下謠言左將軍自襄陽領兵到來。瞎賊素常怕他。自然不敢稍羈。奔馳回去。他二人以逸待勞。先伏一枝兵。假打高傑認(旌)旗在後追趕。他自然跑得更快。次後黃得功迎面衝他一陣。使他一驚。料不敢戀戰。也不敎官軍力戰。恐賊着急。拚命亂殺。致傷士卒。所以放條路與他走。衆賊見有生路。那裡還肯迎敵。但只隨後追趕。等他跑得倦極了。高傑又伏兵在前衝來。使他驚疑。心中自然越慌。不敢駐足。此乃高黃二帥見賊兵衆多。不能盡殺。不過驚他速去。保全了官軍不傷。寇圍得解就罷了。要是李自成不知兵。他們就不是這等相待了。那些賊兵有衝散了的。或三五十一羣。或百十人一夥。四處尋路歸營。有路經六合天長江浦邊界上的。都被尚智林忠慕義生擒的生擒。斬首的斬首。脫逃者無幾。探得賊去遠了。他三人又親到京中去獻俘。史公大喜。再說李自成見高黃兩家追兵已回。心中略定。不想迎面又遇着一枝人馬。風捲而來。你道是誰。起先林報國三人在京中領了史公之命回來。各整壯兵去守疆界。國氏知道了。要領兵去截賊歸路。林報國道。史公吩咐只叫各守邊界。不可遠離。如何違得。國氏道。古云。將在外。君命有所不受。史公之令乃持重之意。恐諸君兵少。不令遠去。懼賊衆多。倘或有失。未嘗不欲殺賊也。我雖女流。視賊奴烏合之衆如同螻蟻。但一舉手便成虀粉。國氏執意要去。林報國阻他不住。只得任其所爲。國氏便傳集衆壯兵。吩咐道。我如今要去斷賊歸路。你們可敢同我去麼。衆人都知他的驍勇。誰不願立功。盡歡答道。我們都情願隨夫人去。國氏道。不消你們齊往。還要人同我夫主沿邊殺賊。只須三百人跟去足矣。即挑選了三百名壯漢兵卒。將家中己囊取出三百金。每人給銀一兩。預備乾糧。俟有功回來。再申報請賞。衆人無不感激。次日。領衆前往。先差兩名健步前去打聽賊的消息。好做準備。去了兩日。回報賊衆被高黃二將軍殺得大敗。自盱眙一路逃來。不久就到。國氏命衆人飽餐。等候廝殺。李自成被高黃二帥殺得抱頭鼠竄而逃。正走着。前面又有兵攔阻。初見不多步卒。心猶不懼。遠遠望去。爲首一將。頭戴銀抹額。銀甲鮮明。鋼鎗耀日。坐下火炭馬。好似一朶紅雲托着一團瑞雪。又是那樣嬌嬌滴滴美賽姮娥。雄雄糾糾猛如項羽的那員女將。驚得幾乎墜騎。向衆將道。這人惹不得的。逃命要緊。衆人聽說俱慌。各要顧命。四散奔逃。國氏一眼認得瞎賊。飛馬追來。衆賊將少不得要〈殺〉保護主公。一齊上前攔擋。只見國氏鎗法展開。惟見一團光亮。繞得眼花。連人帶馬都看不見。衆賊將早被他刺死了幾個。又中傷了幾個。心慌膽怯。瞎賊已跑遠了。衆賊也就縱馬逃命。國氏見賊衆難追。只命收獲馬匹並器械什物而回。查點隨去之人。不曾損折一個。鮑信又細細申報史公。史公大加贊美。將高黃二帥的大功。並尚智林忠慕義同國氏的勤勞功蹟。一並申奏崇禎。皇上大喜。加高傑黃得功少保。賞給尚智等三人金幣。加封國氏一品夫人。衆將士皆有恩賞。那時衆人將國邢二位夫人稱爲女中兩絕。但邢夫人有大遜國夫人處。邢夫人雖有勇有識。但殺氣英風太露。國夫人生得如一朶嬌花。迎風欲顫。而殺賊的本事勇猛無匹。眞塵寰中少有之女子也。且說李自成帶了敗殘人馬。忙忙如喪家之犬。回到亳州。喘息定了。命查隨去之兵折了多少。賊將回報。人折不多。只喪了萬餘人馬。軍資器械盡行失去。李自成敗了兩場。一來有些懼怯。二來又着了一口暗氣。覺得身子不爽。一意要回陝西。你道這瞎賊如此凶惡。還受什麼氣。他自從邢氏隨高傑去了。聞得西安府長安縣有一姓楊的女兒。有賽楊妃之名。差人去硬奪了來。那父母怎敢違拗。闖賊見了。果然好個絕色女子。那楊氏生得。

\begin{quotation}

臨風欲舉。似飛燕之輕盈\footnote{論這輕盈。果賽楊妃。}。弱態堪憐。類王嬙之嬌媚。秋波一轉。能消鐵漢之魂。丯韻半天。可奪淫人之魄。衣間惹一種幽香。非蘭非麝。臉際砌十分春色。疑玉疑香。盈盈十五芳年紀。恰是楊妃未嫁時。

\end{quotation}

還有毛詩上碩人章的幾句。正好爲他寫照。道是。

\begin{quotation}

手如柔荑。膚如凝脂。頷(領)如蝤蠐。齒如瓠犀。螓首蛾眉。巧笑倩兮。美目盼兮。

\end{quotation}

這幾句還不足以盡其美。那偕老章道。

\begin{quotation}

玼兮玼兮。其之翟也。鬒髮如雲。不屑髢也。玉之瑱也。象之揥也。揚且之皙也。胡然而天也。胡然而帝也。

\end{quotation}

你道這樣的美人。雖石人見了也動心。鐵人見了也相愛。何況這個瞎賊。他得了這楊氏。眞比破了幾十座城池。搶了幾千馱金寶還快樂。他當日娶韓氏時。還是無賴貧窮的時候。見了他。就以爲天姿國色。十分心愛。如獲異寶。不想那韓氏是個風塵妓女。一心只有那蓋君祿。與他是掛名的夫妻。原不甚相愛。後來得了邢氏。雖比他美過數倍。但邢氏是勉強從順他的。李自成雖百分愛他。他心中却不愛這瞎賊。也不過只有夫妻之名色而已\footnote{趣而妙。韓氏是掛名夫妻。邢氏是名色夫妻。見這等惡人。連妻子皆不與之同心。}。況且他是個女中英雄。雖俊龐可喜。然那眉目之中凜凜有一種殺氣。相愛中又有些可畏\footnote{諺云。鬼怕惡人。瞎賊如此之惡。竟還有可畏之人也。}。今得了這楊氏。其美又出於邢氏之上。只有可愛。而無可懼。眞正心中愛的要死。但楊氏這樣個嬌而美。美而少的婦人。伴着這等一個凶暴的反寇。他只知有屠害生靈的惡腸。那種有軟玉溫香的手段。到夜間。興之所至。拿他像應差一般。蠻抽蠻扯。不過幾下。了其事而已矣。那裡知道甚麼溫存。何以謂之憐惜。闖賊因前番托高傑執掌內營。圖他保護妻子。不期連妻子都被他竊去。方知此物不是亂托得人的。他有個族姪叫做李錦。以爲是自己的姪兒。再無妨於事的了。叫他管內營事務。但照管內裡的夫人侍妾。孰不知李錦也是從小兒做暴賊的人。只知風高放火。黑夜殺人。書本兒也不曾摸過。知道甚麼叫做倫常禮義。他一見了這楊氏。就一片心神注在他身上。那知這楊氏自幼以爲生得如此美貌。將來定嫁一個俊俏兒郞。不想得了這樣一位外貌旣不揚。內才又不濟的尊夫。那心中如吃了幾擔黃連水一般。淹心的苦也說不出口。那眼淚只好暗暗的往肚裡落。這瞎賊雖然愛他。但成日要去攻城掠地。調將遣兵。做那流賊的伎倆。被窩中的事也不過是名色而已。楊氏見李錦纔二十多歲。一條精壯漢子。又還生得面白唇紅。雖算不得美男兒。較他令叔也高了許多。就有個要把他做壓寨小郞君的意思\footnote{名色甚新。}。無人處常拿話勾引李錦。那李錦是一個伶俐滑賊。何所不知。兩人眉來眼去。都懷相愛之心。一日。左右無人。李錦笑向李氏道。夫人是聞名的賽楊妃。當日楊貴妃是配唐明皇。唐明皇排行第三。人都稱他爲李三郞。我也是第三。如今合營中都叫我做小李三郞。賽楊妃自然該配小三李郞纔是。怎麼倒配了小李三郞的叔叔呢。又笑道。我聽得人說。當初楊貴妃是唐明皇兒子的媳婦。被公公拿了去做老婆。今日夫人是嬸娘。何必(不)配了姪兒。翻一翻案。替當年楊貴妃報報仇。楊氏也笑道。你想要配我。那是萬不能夠的了。我也聽得說。當日楊貴妃有一個乾兒子叫做安樂(祿)山。他明爲母子。暗做夫妻。只瞞了唐明皇的耳目。你如今是姪兒。比乾兒子又還親些。也只好瞞了你叔叔。我們暗效鸞儔罷了。他二人暗訂佳期。但內帳中侍兒羅列。難以偷期。這楊氏不但沒有邢夫人的膽量。且終日守着瞎賊。沒處下手。攻汴梁時瞎賊被射傷了眼睛。疼得晝夜號呼。一連數日。衆婦女在傍伺候。皆不敢合眼。那日。瞎賊眼疼略止了些。沈沈睡去。那些侍女熬了幾夜。也都趁空東倒西歪的睡着。這李錦每日黎明假意進來請安。希圖得空。好同楊氏了一了心願。孰知楊氏也有心。這早見衆人都睡了。他便獨坐。以候李錦。少刻。李錦潛步而來。見瞎賊睡熟。左右七顚八倒的都在夢鄕。見楊氏獨坐。他也不敢說話。笑向楊氏。用手往後帳中指指。楊氏含笑點頭。兩人同到後帳榻上。解珮露湘妃之玉。齊眉點漢渚之香。這一個竭力頻抽。以伸向來渴想。那一個盡情迎送。以償日久相思。楊氏見李錦外邊的威勢雖不及叔叔的萬分之一。被底的本事強如他叔叔的十倍。李錦見楊氏的標致。以及遍身的滑膩。妙是不消說的。且那一種風騷比外邊擄來的婦女大不相同。兩人的恩愛那裡說得盡。你看他二人好一番樂境也。

\begin{quotation}

賽楊妃金蓮高舉。喜孜孜。眞是那被底鴛鴦。小三郞玉莖忙舒。笑吟吟。堪賽那水涯鸂鶒。這個道。當日是明皇私媳。到今日。你小三郞翻案做來。那個道。昔時乃楊氏偷兒。到今朝。你大嬸娘依樣畫出。這個道。安祿山當初看見我軟溫新剝雞頭肉。我今日竟得嘗你雞頭肉。這肉好肉。那個道。李三郞昔日道他信是胡兒只識酥。我此時竟弄得你便只是酥。可酥不酥。這個道。你歇歇罷。你那瞎叔叔比不得老三郞大雅。肯容我錦繃兒擡那胖子。那個道。且慢慢着。這些小侍兒比不得瘦梅精吃醋。且等我助情花遂你嬌心。弄多時。這個哼喞喞。哎呀了一聲。已遍體酥麻。那個喘吁吁。完賬了一句。已全身壓下。已成彩鳳雙飛翼。交付靈犀一點通\footnote{通篇即以明皇楊妃事實。實好。}。

\end{quotation}

他二人恐人醒來撞見。忙忙的雲收雨散。整衣而起。此後他二人的情愛雖濃。然不能再赴陽臺之樂。這次瞎賊去攻打徐州。他二人得了這個空〖阝少日小〗。色膽如天。也顧不得衆侍兒十目所視十手所指了。竟公然就交鋒起來。一個是托嬸娘權當了嬌妻。一個是把猶子暫充了夫主。日夜大幹。這豈非瞎賊同部下淫掠婦女的現報。楊氏同他商議。這些侍兒可以威制。可以恩結。還有瞎賊的幾位如夫人。恐他們吃起醋來。洩露口風。非同小可。要做個一網打盡之計方妥。那李錦仗着他力壯陽強。何樂不爲。楊氏婉轉說合。這些賊中婦女有何愧恥。都欣然領納。他二人見無後患了。無夜不春風幾度。忽一日。他兩個聽得探馬飛報來說。瞎賊失了機。不久就要回來。此後不知何日又纔得空兒相聚。一日之內要做三五夜的勾當。把後來的都要預支。不想徹夜瘋狂。到五鼓反睡着了。原不防瞎賊回來的速。誰知瞎賊敗了幾陣。星夜奔回。大隊還在後面。他先領了十數騎即回老營。衆賊將還不知他回來了。他已到了內營。就往內帳裡走。那李錦同楊氏正還摟抱而睡。有一個侍女起得早。聞知瞎賊回來。已進內營。忙進帳將他二人推醒。說道。大王進來了。他兩個聽見。如雷震癡了一般。急忙爬起來時。那瞎賊已到了帳中。見楊氏李錦同在床上。慌手慌脚穿衣。心中大怒。思量要殺他二人。一個是愛姪。一個是嬌妻。捨不得下此毒手。但大聲說道。這也甚奇。當日那韓氏私通蓋君祿。次後邢氏又同高傑去了。今這楊氏又與姪兒相偷。三人前後一轍。我這樣一個漢子。緣何是一個大烏龜的命。因此着了一口暗氣。伏枕數日纔好了些。把李錦逐出。此後再不許他進內帳來。那瞎賊見這次用兵不利。毫無興頭。因聚衆將商議要回去。衆賊都辛苦了一年。不但一無所得。且損了無數。都興致淡然。贊成其意。此時那七路搶劫去的兵馬都回來繳令。惟北路去的點燈子\endnotemark[3]領着敗殘的數百賊衆大敗而回。你道是誰殺敗了的。他帶了人馬向北而行。過了多少城池村鎭。都是他們殘破過的。皆荒涼無人。他直到了泰安州地方。見一座村坊。約有數千家。人煙輳雜。景象富庶。心中大喜。一齊踴躍直奔了來。尚離數里。見一塊平陽大地。都到了這處。正要埋鍋造飯。吃飽了好去擄掠。忽然一聲響亮。如天崩地塌。陷了一個大坑。把二千餘賊盡塡於內。這點燈子在後壓陣的。幸得不曾陷了下去。見了目瞪口呆。看所剩人馬不上數百。正在癡呆之際。聽得兩處呐喊。見那村中左右分兩股兵殺來。約有二千多人。他不敢抵敵。領着殘兵。星夜奔回。這是辛同鮑德探知流賊到來。他學當年大同總兵郭登做的攪地雷。保護本村。不想流賊果然吃了這場大虧。敗逃而去\footnote{古云。識者有時有。英雄無日無。尚智諸人得遇史公。便得享皇恩。受爵祿。而鮑德辛同之謀勇。不在他諸人之下。無識之者。不過終於一草莽英雄而已。自古及今。文有經濟之才。武有治亂之勇。無人提攜。老死於牖下者。不知幾許人焉。惜哉。}。李自成見連連失利。遂傳令次日拔營。再說史奇他敗了那一陣。雖免死立功。他心中不肯服。忿忿不平。道。我經多少大敵。旗開得勝。馬到成功。今遇此小去處。反遭了這場大敗。功名還是小事。有何臉面立於衆將之中。若不大建一場功勞。何以掩得前番之醜。今見瞎賊要回陝西。他忙上前跪下。道。臣隨大王多年。曾立過微功。前次失機。蒙大王免死。但臣有何臉面與衆將爲伍。今願大王賞臣三千人馬。臣去攻破鳳陽。屠此一城。上張大王天威。下雪小臣前恥。李自成道。我大兵盡去。你孤軍在此無援。何以保得必勝。史奇道。臣雖一個大字不識。曾聽得人說。謝玄以萬餘弱卒。破苻堅數十萬雄師。臣以鐵騎三千。何愁一座孤城不克。況臣等跟隨大王。尚欲混一四海。以奉大王高登大寶。臣等還望列土分茅。若此一城不能攻克。尚畏首畏尾。何以橫行天下。臣前次失機者。一時出於大意。今若不能破鳳陽。願甘軍法。瞎賊大喜道。你這一片雄心壯膽。就可以直吞鳳陽了。何愁不克。你挑四員偏將並三千人馬前去。早早立功。我到潼關歇馬。等你的捷音。史奇叩頭謝恩。選了四員稗將。一名終嚴。一名童智。一名金從政。一名伏順。又選了三千勁兵。辭了瞎賊。洋洋得意。殺奔鳳陽府來。李自成遂傳令起行。大隊盡回潼關去了。且說這史奇他是個一勇之夫。胸中絲毫算計都沒有的。仗着他力大身強。自以爲英雄無敵。他前在六合遇了國守。吃了他那一鎗。魂都嚇走。今次不敢去惹他。想起鳳陽人都不濟。他想來施些威。破了城。可以名利兼收。他就不曾想。當日得勝是隨了李自成的大隊。人多勢衆。又遇着都是怕死的官軍。聽見流賊兩個字。不但大人魂夢皆驚。還可以止得小兒夜哭。何況見了面還有個不跑者。那文武官員更有好笑。當日岳忠武說。文官不愛錢。武官不惜死。天下自然太平。此時大謬不然。文官拚命要錢。武將愛錢怕死。流賊還在數百里之外。他駞着宦囊。帶着妻妾。拿出那楚狂接輿的身分來。遠遠的趨而避之。這些軍民見官府都竊負而逃。大家也就相率而遁。跑得快的跑掉了。跑得慢的。年少婦女悉爲衆賊之妻。老弱男婦咸作溝渠之鬼。所以這些官軍。不說他自己學會個棄甲曳兵而走。反說得這些賊竟是無敵於天下的。吏(使)這賊衆看慣了。不知是官軍懦弱。也公然以爲他是喑嗚叱咤。千人自廢。一個個都是蓋世無敵的眞正好漢。這史奇不知死活。竟大膽領着三千孤軍。要來攻屠鳳陽。以爲內中定無國守。向年瞎賊屠鳳陽時。姚澤民爲先鋒。他爲副先鋒。兵馬一到。城池立破。不但殺得臊皮。而且搶得快活。他此時還當是前番一樣。一團高興。蜂擁而來。誰知這一次不似前番了。當年因太平日久。人不知兵。素常聞流賊之名。如雷震耳。一聞他們來到。都心膽墮地。屁滾尿流。夾屎而遁。後來流賊滿載而去。恢復了地方。崇禎把這聞賊先逃的將帥也殺了幾個。又將城池修得堅固。添兵防守。如今聽得李自成大隊已去。只有幾千人來。膽又壯了些。雖不出來對敵。却也不敢聞風而遁了。鳳陽總督馬士英少不得率衆堅守。一面雪片文書到南京兵部。飛報賊情。求取救兵。且說史奇領着人馬。離鳳陽尚有數十里之遙。差人飛馬去探看城中可跑盡了。不曾移時。探馬回報說。城中緊閉。防守甚嚴。像是不曾逃躱。史奇大怒道。我們的名。那一處聽見不膽碎心裂。況此處又是我們向年殺怕了的。今日何敢不走。是誰借了些膽子給他麼\footnote{膽都可借。奇聞。}。對衆偏將道。這是天意。該我們建這場大功。發這次橫財。故此他們不曾遁去。我們快些上前。這一破了城。且搶幾個婦人快樂。衆賊聽說得興頭。大家如飛一般。齊催馬到了城下。見城上週圍都有人防守。史奇道。不要怕他。你們爬城。他見了自然要跑。衆賊也想得奪了城。圖內中的金帛婦女。大家下馬。就往上爬。城中兵卒見了。手慌脚忙。火砲齊施。木石並下。先爬到城半中的。傷了有二三百個。衆賊連忙退回。史奇料道不能爬進去。只得離城數里下了營寨。次日。差了兩員賊將。到城下來勸城中官員百姓投降。如開門投順。一個不殺。不然破城之日。寸草不留。馬士英同衆官商議。不敢惡言回答。恐激怒了他盡力來攻。如何抵敵。只婉言回覆道。將軍兵到。我們應該就降。但朝廷法度森嚴。恐後來加罪。請將軍先破了南京。我們自然歸順。那賊將回覆了史奇。史奇怒道。他諒我們不敢攻城。好話勸着不依。我們再齊心併力去攻。不怕攻之不下。遂造了幾座雲梯。推到城下。城中也防備甚密。見雲梯剛到城下。連發大砲。將賊衆又傷了數百。史奇見不能攻進。回營大怒。又差了賊將到城下說。你們旣然不降。可出來打降。見個輸贏。城中衆人總不敢答應。只是堅守。賊將見沒人答應。只得回營復命。史奇大怒道。料他也不敢出來。我們且往別處去搶擄一番。只留下數百人守營。餘衆分作七八路。到數百里之內。逢人便殺。婦女盡擄。金帛糧食都運到營中。一則取樂。二則爲久困之計。且說馬士英求救文書到了南京。史公見文書一日數至。諒必事在緊急。遂會同了衆文武。在午門外公議誰人領兵前去救援。這些公侯伯都督衆武職勳臣。一個個覿面相覷。沒一個出頭答應。史公見這般光景。知是畏刀避劍。明哲保身的大將。意欲派幾個去。料他們不敢不遵。恐到了那裡。喪師逃敗而回。倒折了天邦銳氣。故作色道。諸公食朝廷重祿。祖孫相繼者二百餘年。閒時談兵說陣。何等威風。今聞寇至。便束手無策。本部今日不是姑息諸公。不遣領兵前去。但鳳陽祖陵要地。恐到那裡無用。反誤了大事。衆武臣一個個羞得面紅耳熱。却不敢應承。樂公道。雖無將可遣。但救兵如救火。不可遲緩。慕義等乃屢勝之師。須遣他們去。庶可成功。老先生尊意如何。史公道。愚意正注在他三人。先生此言。正合愚意。但恐他衆步卒已經兩次奔勞。喘息未定。又命遠去救援。未免疲鈍耳。此時慕義等正在城中。史公命傳了他三人來。道。適間連接飛報。流賊大隊已回潼關。今分兵一枝來寇鳳陽。本部的意思。要你們去應援。你們心下如何。他三人齊聲應道。卑職上蒙朝廷天恩。又荷老爺提拔。雖赴湯蹈火。亦所不辭。旣受皇家爵祿。這殺賊報功乃武臣分內之事。安敢辭勞。史公大笑道。衆武臣都要似你們這般心胸。那些流賊早已撥滅盡了。奈何都是些慵懦之夫。以致天下四分五裂。令人可嘆可恨。衆勳臣心下暗想。他這些話。明明道着下官。只好忍氣吞聲。誰敢回言辯駁。史公又道。但你們部下都是步卒。前次奔走勞苦了。可在京營中挑選幾千兵馬前去。若得建功回來。本部自當力薦。他三人稟道。蕞爾小寇。何須京營人馬。卑職等三千步卒。留六百以守三縣城堡。只帶二千餘前去。足以剿滅那些逆賊。史公道。我知爾等足能辦事。但此行係應援地方的公事。都要給他們的行糧纔是。沒有個替朝廷出力。還叫他自備口糧之理。他三人道。這是老爺天恩。這些兵卒自然感恩。效死以報。史公向戶部尚書牛騂道。這些兵將。就是前日老先生所說弟迂闊之事。不急之需的那一起人。不但連次立功。且今日又去殺賊。老先生可肯給他們糧餉否。若老先生恐這些人沒用。怕枉費了帑金。就煩舉出一位將領來。督兵前去。牛騂滿面羞慚。答道。此係軍需緊事。老先生有文到敝衙門來。該用多少。敢不應付。史公向他三人道。你們到我署中。今晚關下錢糧。明日就都回去提兵。黑夜前往。三人答應了出去。史公心有所觸。莞爾而笑。樂公道。老先生何故失笑。史公道。弟偶然想起這捐餉的賈進士來。他雖得中科甲。又未蒞仕。食朝廷俸祿。他這項銀子應留與子孫享用的了。況又不曾爲朝廷掌管庫帑。並無官守。就力助三萬金。以今日人情論之。未有不笑其迂呆者矣。故不覺失笑。那傅勝牛騂明知史公是譏誚他。却做聲不得。惟有低頭含愧而已。衆官散去。史公回衙。把他三人又鼓激了一番。都賞了馬匹鞍䩞銀兩紬緞。行文戶部。關了一萬五千兩銀子。每鄕勇賞給銀五兩。又發牌文。凡經過地方。州縣官供給糧草。次早。慕義林忠尚智都辭了回來。將銀子分散了。衆人感激不盡。聽得要去剿賊。他們本是屢勝之師。心雄氣壯。無不踴躍歡喜。他三人商議了一番。每營留二百兵。一員千總領一百兵。幫城守指揮守城。一員把總領一百兵。同衆百姓守堡。三處總交與鮑信監督。不時輪流查核。他三人即日起身。先差伊策探聽鳳陽消息。叫他星夜回報。衆人走了三日。伊策回來報道。流賊領兵的賊將。就是前次我們殺敗的一堵場(牆)史奇。今領了三千人馬來要攻鳳陽。已經兩次攻城。城中守禦甚嚴。傷了數百卒兵。賊將十分忿怒。令他部下賊衆各鄕村搜尋少年婦女。拿來行樂。其老幼男婦盡殺之。以洩忿氣。左近地方焚蕩一空。城中只是堅守。沒一個敢出來對敵。尚智笑道。這賊不知死活。此來定然授首。他欺鳳陽無人。故孤軍而至。我以計破他。如摧枯拉朽耳。此處離賊營還有多遠。伊策道。還有一百餘里。尚智向林忠慕義道。賊衆酷殺。以逞凶心。我們不可不速援救。以保百姓性命。但此賊連次未得便宜。如今是忿師了。他城下失利。聽得有救兵來。他必奮死甘心。於我當設計誘之。先錯(挫)其銳。二人道。遵兄嚴令。努力共殺此賊。以甦百姓之命。尚智道。我引本部兵先行。他不知我們來應援。定大膽領兵來敵。我也假裝他處懦卒。便佯敗誘之。彼必放膽來追。林兄伏於數里之外。俟賊過後。見他隊伍一亂。以砲爲號。便從賊後衝來攻擊。我率兵掩回。前後夾攻。自無不勝之理。賊兵一出。他諒城中不敢輕出。營中必定空虛。慕兄從大寬轉。暗襲賊營。若襲破了。放起火來。亂他的軍心。二人依計。次日。緊走了一日。紮營安歇了一宿。天色黎民(明)。衆人飽食了前進。離賊營不遠。緩緩而行。且說這史奇在李自成面前說了些大話。又立了軍令狀。領兵前來。滿擬一到就破了城。搶殺一番。好回去獻功。不想城門緊閉。攻了二次。倒反傷了幾百人。還攻不開。怎麼回去繳令。自己領了一枝孤軍。屯兵於堅城之下。恐外面援兵四集。心中又怯又怒。着賊兵四散到各處去搶擄。一則出氣。二則且弄些婦女來營中散悶。此時城中若有好將帥。趁此時領兵剿戳。何愁不勝。又何愁衆賊不抱頭鼠竄而逃。無奈這城中官軍畏賊如虎。見賊不來攻城。私心竊喜。感激了不得。可還敢出來惹他。那外面跑不掉的婦女。被賊拿到營中取樂。將老幼百姓殺得屍橫遍野。血流成渠。在城官員未嘗不知。生怕自己的頭顱不知落在何賊之手。那裡還顧得百姓。即如當年嘉靖年間。倭寇蹂躙浙西。來了七個倭子。直犯南京。那時城中猛將如雲。謀臣似雨。還有數十萬京衛兵。嚇得把十三門關得緊緊的。竟無一人敢出。被他在官道上混殺了一番。傷了無限的人。晚間回去離城三十里板橋地方一個財主家。淫其婦女。大醉而臥。一夜而去。七個倭寇。怕到這個地位。又何況三千流賊乎。末世的兵將。說起來可發一笑。這一日。史奇正在營中。心中發悶。飮了一飽早酒。乘着酒興。把十數個婦女都叫脫光了。圍繞着他。揀了三個上好的。三面放下三張椅子。叫他三人仰臥在上。做那三仙出洞的款式。這個身上抽幾抽。飮一杯。那個身上抽幾抽。飮一杯。正在週而復始取樂的時節。忽營門傳鼓。報有援兵到了。他正做得有趣。聽了這話。阻了他的高興。心中大怒。穿衣到了前帳。發令道。不要等他到。我們上前去迎敵。殺他個怕。他自然退去。再回來取樂。城中料想不敢出來。只留二百人守營就夠了。吩咐畢。披甲持鎗。扳鞍上馬。領了二千多賊。如飛般迎了來。遠遠望見些官軍。也無盔甲。各擔着行囊包裹。扛着旌幟刀鎗。慢慢的走。忽見他賊兵一來。回身就跑。史奇大笑道。這一種兵也敢來禦敵。今日殺他個罄盡。也出出我連日的悶氣。便催兵快攆。衆賊縱馬趕了有數里之遙。看看趕上。那些人把行囊全撂了。空身四散而逃。這些賊看見。顧不得攆人了。爭先混搶。史奇催着前進。這些做賊人見了東西。性命都顧不得。誰還遵他的軍令。就要殺也殺不得許多。史奇正發急。衆賊正搶得高興。忽聽後面一聲號砲響。一彪兵馬搖旗呐喊。從背後殺來。衆賊忙回頭一看。見是一起虎頭軍。只得回身迎敵。內中有前次吃過虧的賊。吃了一驚。就亂擾擾有些不定。大家互相傳說他們的利害。古語說。先聲奪人。衆賊心中一怯。就奮不起威來。被他殺到跟前。沒有個束手待斃的理。少不得要去抵敵。忽又聽得喊聲震耳。一枝兵又從面前殺回。又一看時。不是先那些人了。也是虎頭軍士。史奇部下幸得都是挑來的賊中好漢。也還勉強敵住。遠遠望見老營火起。煙焰沖天。不但捨不得搶擄的東西。還有那心愛的活寶在營中。心下大慌。又是一急。就有些擋不住了。這史奇連日被婦女掏虛。今早又吃了一飽老酒。正在那裡高興。忽然來打降。先拿穩走來一殺就勝。便回營作樂。誰知兩三處的人馬只管廝殺起來。由不得昏頭昏腦。正死力支持。忽見國守挺鎗在前。林報國持矛在後。殺將入來。史奇前次在他手中的敗賊。心中大慌。道。這個寃家。如何又來到這裡。料抵敵不住。就落荒而走。國守見了。緊緊追去。這些賊見沒了主帥。又聽〖口么〗喝投降者免死。誰不惜命。也就倒戈棄甲的降了數百。跑了有千數。殺了有數百。尚智鳴金收軍。紮下營寨。同林報國二人坐下。衆人報功。不多時。慕義也領兵到了。坐定。說。賊營果無準備。殺的殺了。走的走了。奪回了許多婦女。其餘糧草輜重。一並焚燒。尚智大喜。吩咐另撥些帳房中\footnote{不但精細。且見經濟之才。}。也安頓了。然後查點將士。內中不見了國守。心下着驚。正要遣人四下去尋。忽報國千總回來了。傳進來問他時。國守道。史奇那廝被千總單騎追去。幾乎趕上。他營中逃出來的有數十人。同着一員賊將。把他救了去了。千總孤身。不敢窮追。所以回來。尚智向林忠慕義道。今日一戰。賊已喪膽。明日再奮力大殺一場。早早奏功回去。以付史公之望。吩咐衆人歇息。再說史奇逃了下去。營都沒了。要想逃回。見人馬折了個乾淨。恐李自成殺他。只得同敗殘賊衆在空處下馬屯住。坐在草地上。叫人四散招呼餘黨。到了日將沈西。那些賊將賊兵知他頭目尚在。又聚攏了。查了一查。還剩了一千二百人。此時帳房也無。鍋也無。糧食一點也沒有。連乾糧都在營中燒掉了。左近又是搶擄盡了的。遠處去搶。天又晚了。只得把馬放於野地啃草。衆賊也就將帶傷的殺了些。敲出火種。尋了些爛柴草來燒吃了。連鹽也沒有。衆賊無不惶惶。內有一個稗將終嚴。向史奇道。此處屯不得人馬。恐敵人知我們露宿在此。夜晚兵來。何以敵彼。不如連夜回去。大王去尚未久。我們星夜趕上去罷。此處一樣俱無。可還是個屯兵的光景。史奇不好說怕李自成見罪。便大怒道。勝敗兵家之常。你如何敢慢我軍心。腰間拔出刀來。定要殺他。衆人力求道。旣然不退。明日自然要去復仇。用人之際。如何自損羽翼。求將軍饒恕。叫他竭力報効罷。求之再三。方纔饒了。此時史奇何嘗不知終嚴是好話。但他各有心事。進退兩難。只是仰天嘆氣。尋思道。我好命蹇。處處遇見國守這個寃家。深悔道。我來差了。我來差了。眞是。

\begin{quotation}

碁有一着錯。滿盤俱是空。

\end{quotation}

今日回不得。住不得。叫我如羝羊觸籓。進退兩難。數載之功。喪於一旦。復又嘆了幾聲。道。

\begin{quotation}

出師未捷身先死。常使英雄淚滿襟。

\end{quotation}

心中悶悶不樂。再說那終嚴勸了史奇一番好話。正是見可〈可〉進而難退的美意。不想果是忠言逆耳。幾乎被他殺却。退後退(遂)約了童智衆人。說道。我們當初都是良民。被賊把家中殺擄盡了。沒奈何。跟着他做賊。這幾年我們殺的人也夠了。今日這光景。有個要給人殺的樣子。你看衆人三五成羣。交頭接耳的。軍心已〔散。〕還中何用。老史叫做矮老兒往深井裡跳。死活也不知道。這個局面還掙着命要廝殺。眞是揷標賣首。活得不耐煩了。我們與他同死無益。不若今夜暗暗差人去投降。約他明日淸早領兵來。我們歸順天朝。且顧眼前的性命。我們都是一身一口。又無父母妻子可戀。你們列位尊意若何。金從政道。蜂蠆入懷。解衣自救。我們顧不得他了。伏順道。列位言之有理。你看翻山鷂歸順了朝廷。何等榮耀。我們如今服順了。一刀一鎗在疆場上掙個功名。便是死了。也有個好名。強似做賊。都謹依遵命。終嚴見衆人同心。大喜不勝。遂差了他一個貼身賊奴叫做莘福。前去投降。附耳吩咐。如此如此說話。不可有誤時刻。那莘福掩掩藏藏。暗暗偷走出去了。再說尚智等看着衆人都吃飽了飯\footnote{似此閒筆。都要晉(留)心。此見得與士卒同甘苦。方得其死力也。}。輪班歇息。刁斗嚴明。有一更多天。營外報有人求見。尚智命搜檢明白\footnote{細防刺客也。光武之待大鎗(搶)銅馬。推心置腹。固妙。而後來囗岑皷(彭)却又受此害。奈何。}。帶了進來。問他來意。莘福將衆人情願投降。明日天明兵到就投戈拜倒。並那些賊的行景。詳細說了。尚智大喜。命帶去賞他酒飯。慕義道。恐他是詐降。不可不防。尚智道。他降。我明日也要領兵去。就是不降。也要領兵去。到了那裡。他降了更省力。如不降。不過是多一番殺戳。據我看來。降是決定眞情。人心已離。誰不惜命。那史奇是瞎賊的一員心腹猛將。若能殺了他。不但使彼奪氣。亦折他一臂。但只要防他的出路。叫過國守來。道。史奇畏你如虎。他明日見人散了。定往長河衛一路逃去。你同卓高常勝領三百軍士。伏在左近。或生擒。或梟首。不可放他走脫。你三鼓領兵先去。傳令合營。四鼓飽餐五鼓動。天明要到賊處。不可有誤。吩咐已畢。歇不多時。都起來埋鍋造飯。吃飽了。打點停當。尚智向林忠慕義道。古云。受降如受敵。我們分作三路去。陸續起行。我今先往。他若是詐謀。我陷在伏中。慕兄即在外衝突。我二人內外夾攻。不愁不勝。林兄再四圍踩看何處兵厚。即奪勇衝之。一二千毛賊。何能擋我三枝義兵。命昨夜來投降的莘福做了嚮導前行。〈路〉天色黎明。離賊不遠。却說衆賊在露天之下蹲了一夜。衣服露得精濕。昨日又沒有吃飯。又冷又餓。身上都有些好不自在。又想起前日在營中吃着酒肉。同衆婦女歡笑。何等興頭。今夜在此受這悽惶。好生難過。聽得遠遠的呐喊。四路殺來。都左張右望。有些驚慌。史奇跳起。忙叫衆人披甲備馬。此時兵不望將了。一個個佯佯不睬。催了幾遍。這四個賊將向着衆賊道。我們留着這件吃飯的傢伙罷。這個樣子還殺甚麼。不如大家投降。救這窮命罷了。衆賊正想要四散逃命。聽得這話。同聲大喊道。我們情願跟着投降。史奇見局勢不好。看看兵馬漸近。領着心腹數騎。飛奔長河衛一路去了。尚智兵纔一到。衆賊抛下器械。一齊拜倒。大呼願降。尚智把終嚴等撫慰了一番。不多時。林忠慕義的兵都到了。一面安營。一面差人進城。報與鳳督並守陵太監。尚智知道衆賊昨日未食。吩咐給與糧草\footnote{好。}。衆人歡呼若雷。又命人去將賊營所擄婦女。並看營的兵。都搬了來。待稟鳳督。出示招人將婦女領回。再說那史奇帶着七八個小賊逃去。見後面無人追趕。遂放心往前奔走。暗說道。國守。國守。你若早先在此伏下一枝人馬。我史奇萬無生理了。不想剛到了長河衛。見前面擺開百餘虎頭軍。一員銀盔白甲的將官大喝道。史賊。你想逃往那裡去。史奇一見是國守。魂不附體。帶馬往斜刺而逃。那跟的幾個賊見勢頭不好。顧不得主人了。下馬拜降。國守率兵攆了下去。史奇要尋生路。只剩孤身。傍邊連做眼的也沒一個。急得要死。面前卓高又領着虎軍擋住。常勝又從傍領軍圍住。正在急。不料國守一騎馬飛近跟前。大喝了一聲。史奇剛回頭一望。那根鎗已進後心。栽下馬來。國守將他首級梟下。奏凱回來獻功\footnote{可笑史奇不自揣。是死於國守之手而後已。}。此時鳳陽城中之危方解嚴。鳳督馬士英發了許多豬羊牛酒出來。差了一員推官。一員指揮來犒軍。尚智令千把總守營\footnote{細防降賊。恐其有詐。}。他三人進城。參見拜謝。並稟奪回婦女一槪查明交付等情。鳳督大喜。又待酒。回營。尚智一面遣人賫史奇的頭顱。飛馬往南京報捷。一面回軍。數日到了京城。命衆軍各回安歇\footnote{細。}。他三人同到京城來見史公。並交這些投降軍卒器械。史公大悅。大加獎譽。細細題奏崇禎。皇上見他三人救了祖陵要地。只二千多兵。不但把賊殺的殺。降的降。而且斬賊一員大將。面諭兵部將慕義林忠尚智皆陞遊擊將軍。加都督同知職銜。賜正二品服俸。林忠仍帶軍功二次。千總國守斬賊有攻。着陞守備。加都督僉事。其隨經(軍)有功人員。皆着加一級。兵卒每人賞銀十兩。其投降賊將。着南京兵部尚書史可法量材擢用。以鼓餘賊向化之心。所降賊兵。願歸農者。給牛地。入籍爲民。願爲兵。分派各營充伍。賈文物鮑信俱着加一級。報到了南京。欽遵而行。他三人俱是正三品武臣。便是古之通侯了。又有兼銜。俱穿猱獅二品補服。更覺軒昻熱鬧。正是古人說的。

\begin{quotation}

識者有時有。英雄無日無。

\end{quotation}

他衆人若不遇史樂二公。不過一鄕農而已。焉可以資格論哉。且說鳳督吿示通衢。傳諭各處百姓來認妻女。有父兄丈夫來認者。即着領回。如家人被殺無遺者。擇人匹配。有一個百姓名叫兪一鳴。他的個女兒是立春那一日生的。叫做春姐。婦刁氏。俱被賊擄去。聽得官府出示。招人去認眷屬。他以爲兩個之中得一個回來就算萬幸了。不意女媳俱存。好生歡喜。領了回家。那兪一鳴見女兒媳婦在賊營多日。雖知定非全璧。此係遭了大難。不足責備。見他們受了這一番驚恐。得了性命回來。悲喜交集。偶然同女兒說話。問問賊中的景況。道。聞得賊人凶惡異常。他營中也還像個人麼。是怎麼個光景。這兪春姐眞愚蠢得出奇。答道。賊營裡穿衣吃飯。與我們過日子一樣。只有幾件不同些。我們住的房子。或是瓦的。或是草的。他們住的都是矮矮小小的布房子。吃飯睡覺都不用床桌。總是在地下。我們在家吃飯是豆腐鹹菜。他那裡頓頓吃肉。我見這裡家家都是一夫一妻的。他們一間小布房裡。四五個漢子娶一個女人。還有一件。夜間睡覺也不同些。我們從小枕頭是枕着睡的。到了那裡。他把枕頭墊在我屁股底下過夜。兪一鳴聽見這話。知女兒是個蠢材。喝一聲道。唗。兪春姐道。他把我兩條腿直豎豎的扛在肩膀上。肚皮壓得死緊的。中間還用個大釘子閂着。兪一鳴見他說的不成話。罵道。胡說。兪春姐道。爹。你是鄕下人。沒有見他們的那個厲害。他把舌頭塞在我口裡。腰裡像搗碓一般那樣大力氣。他還着一個在後頭推我。弄得我上氣接不得下氣。心裡像要死也似的。哼不出來呢。還說甚麼。要像在家裡這樣閒着。不論怎樣。就胡亂說出來了。兪一鳴怒道。放屁。放屁。他見老子連說兩個放屁。他倒反發起急來。道。爹。你好不知人的死活。倒說說的好聽。他四五個人。一夜輪流着上上下下的。那兩個卵子像雨點一般往下打。連番(糞)門都撞腫了。還放甚麼屁。要是你老人家到了那裡。恐怕拿轆轤還壓不出屁來哩。那兪一鳴見他說得更不入耳。自己倒沒趣。佯佯走開。他那個媳婦刁氏嘴舌便利。自己誇得他冰淸玉潔。並未爲賊所汚。這是沒有對證的話。憑他去說。他村中也還有脫難的婦女。聽得兪家姑嫂兩個自賊營得命回來。眞如脫了虎口。都來探問。坐下道。大嫂。你吃了驚。又受了這些日子的苦來了。可憐。可憐。回來了就算天大的造化了。刁氏道。若說受驚。先被他拿去時。恐怕他要殺。還有些怕。過了一兩夜。也就不覺了。要說受苦。阿彌陀佛。不當人子。像這樣的苦。吃一輩子也是願意的。內中有一個老實些的道。我聽得人傳說。流賊搶了婦人去。要傳營的。或五六個男人睡一個婦人。若婦人少了。還有十多個賊共着一個的。所以十個婦人九死一生。大嫂。你還沒有吃虧麼。刁氏道。哎呀。這是那裡話。有那沒廉恥的婦人。到那裡就依從了。嘻嘻哈哈。同那些漢子們頑成一塊。我只是拚死也不依。他拔出刀來嚇我。我就伸着脖子給他殺。他強我不從。也就罷了。只替他們煮煮飯。補補衣服。夜間我把褲帶繫得緊緊的。衣裳總不脫。並沒有同他們沾身。這幾個婦人裡面。有一個姓智的。是個黠滑婦人。暗想道。他明明的被賊不知弄了多少回數。大約肚子裡流賊的種都有了。他還撇這樣淸。等我詐他一詐。便道。大嫂。這是你的造化。我久聽得人說。流賊的㞠子好不怕人。個個都是四方的。又長又大。所以婦人們遇着了他們就死的多。我想天地間的人都是一樣。怎做了賊。連那東西都改變了。刁氏失口道。這都是人胡說的話。那裡有這樣的事。我看也都是圓的。大小長短也不等。誰說都是四方長大的。衆婦人不覺都笑起來。刁氏自知說話露了破綻。臉脖子絳紅。纔不做聲。衆人別去。這兪春姐但愚蠢而已矣。刁氏則可謂愚而詐者也。今日男子中此類亦復不少。閒言不必太煩。且說李自成在潼關住了些日子。等史奇的信。那裡知他全軍覆沒。並無一個報信之人。後來風聞得史奇攻鳳陽不下。又敗了陣。遂傳了衆將到跟前。命他的獻世大將軍澤國公姚澤民道。孤知你謀勇雙全。你可引鐵騎五千。接應了史奇回來。孤先回陝西。等你們到來。再同議大舉。那姚澤民得了令。帶了他大將軍府兩員參謀。一名游夏流。一名勞正。又挑了幾員驍將。領了五千健卒。星夜向鳳陽一路而來。瞎賊也領大隊向陝西而去。專候他們的捷音。以圖後舉。正是。

\begin{quotation}

人心如此如此。天理未然未然。

\end{quotation}

你道姚澤民是朝廷家的一個侯爵了。如何又做了賊的大將軍。他當日奉了天啓的旨意。到廣西省親。路過南京。慕錢貴之名。訪探一遭。未遂其欲。憤然而去。雖接了夏錦兒羅春兒兩個妓女。嫖了兩夜。總不起興。悵悵起身。到了他父親任所。姚華胄已死了三日。他一面報了地方官。交了牌印王命。一面將他父親靈柩裝載回南。到無錫縣本家下了葬。然後進京復命。天啓已崩。崇禎即位。崇禎在藩邸時即耳他父子之名。又是天啓面諭過。後來着他承襲。且他父親又死於王事。就着他襲了侯。到了崇禎五年。李自成在陝西作亂三載。屢次遣將。不能剿滅。漸漸勢大。崇禎知他父子善於談兵。且他父親又平過廣西流寇。他是老將之子。必定有些韜略。特給他平寇將軍的印。叫他往陝西剿賊。他口中雖會說如何排兵。如何禦敵。說得固然好聽。却並不知兵當作何調用\footnote{聖經云。其言之不怍。則爲之也難。千古來。不止一個姚澤民也。世上但會說大話的人。決不能踐言。能幹大事者。決不肯說大話。試看姚澤民如何。}。一路隊伍不成隊伍。軍令也沒有一個。先在腹內地方。還不敢放肆。一過了潼關。便沿途搶劫。比流賊還利害幾分。所以當日有賊梳官箆之謠。他倒不愛金銀。只是兵士們有擄來的好婦女。不許自私。必要送他。爲夜間枕蓆上排兵交鋒之用。如有隱藏者。定按軍法。他帳房中的女子竟有數十。內中有一個是華陰縣擄來的。是南京人。生得甚美。姚澤民甚是愛他。問起來。他姓鍾。是鍾趨之女。因公公勞御史是魏黨正法。同丈夫勞正充發華陰當軍的。姚澤民一來愛他標致。二來是同鄕。就把他立做權夫人\footnote{這權夫人尚不及尖夫人。}。統領衆婦。每日在帳房中痛飮酣歌起來。且說這鍾氏當日嫁了這勞正。他家雖然豪富。那勞正却是一個癆癆怯怯的病夫。勞正因見他是個眞正處女。姿色又好。不在寶姑之下。倒也十分相愛。無奈自己體虛氣弱。腰軟力綿\footnote{昔一大老納一寵。後忽染瘋疾。衆子姪來候安。問夫人道。大人從無此症。如何一旦發此。時寵妾在側。夫人笑指之道。此瘋之始也。勞正得了鍾氏。恐腰體愈軟弱矣。}。錦衾繡榻中的那一番樂境。鍾氏于歸四載。尚未嘗着深趣。後來家貲籍沒。同勞正到了華陰。做了軍妻。衣食皆不能繼。那房幃之樂越發不暇及了。今被姚澤民的步軍擄獲。獻與主帥。姚澤民一見大喜。可居繼母嬌妻之右。不能須臾稍待。忙上前抱住。就要雙飛比翼起來。鍾氏雖到了這個地步。到底是儒門之女。宦室之妻。愧心尚在。左推右拒的不肯。姚澤民的淫興那裡還能止遏得住。以主帥之尊。竟行起強盜之事來。叫了三五個婦女。將他按在床上。剝了衣褲。見他。

\begin{quotation}

肉白如雪。髮黑〈黑〉如墨。面嫩而嬌。體香而怯。指若春葱。足剛一捏。無處不引人魂。更有消魂一穴。

\end{quotation}

姚澤民看到那個去處。\endnotemark[4]想起當年裘氏並家中現存諸美。心中雖有微慘。却又十分興豪。便弄了進去。深深淺淺。徐徐疾疾。緊而慢。慢而緊的抽送起來。他軍中的紀律全然不知。這榻上的兵機頗覺嫺熟\footnote{春燈謎。燕子箋。是阮大鋮之陰符。榻上交鋒。衾中潑戰。是姚澤〔民〕之勇略。也可謂各有一長。}。鍾氏先被他按住強淫。因見他威嚴勢重。口中雖不敢罵。心中着實愧恨。淚流滿面。全是那萬不得已的樣子。弄到後來。漸入佳境。他方知婦人嫁了丈夫。不只但戳戳而已。竟有這許多深微的妙處。眼淚一時也不知往那裡去了。先那一種羞怒之色。變做個笑吟吟的龐兒。見這幾個婦人還按着。他遂說道。你不過是要這樣的罷了。儘着按住我怎麼。姚澤民知他心悅情服了。遣開衆婦。挺矛直搗紅心。那鍾氏也就由不得手之摟之。足之蹺之的起來。姚澤民樂極而洩。各整衣而起。鍾氏見姚澤民正在壯年。較那病夫強多。不但陽物魁偉。且又戰法甚妙。又位高而金多。雖不曾蛇行匍匐。也就樂侍衾裯。姚澤民問他的家世鄕貫。他細述父家夫家的履歷。姚澤民大喜。立他爲權夫人。統衆妾婢。鍾氏也喜出望外。一個軍妻忽得爲將軍之副室。那面上惟見欣欣喜笑之容。全無那憂愁愧赧之色。姚澤民日夜惟與衆婦女鏖戰。那殺賊兩個字全置之腦後。終日在營內盤桓。瞎賊探明了他這些信息。又知他是無紀律之師。便設計誘他。一日。姚澤民在內帳正同衆婦女飮酒作樂。忽轅門傳稟。有幾個流賊來投降。有機密軍情面稟。姚澤民聽說。出來陞了中軍帳。命將降賊傳入。賊進營叩見了。跪稟道。小人們俱是朝廷好百姓。不幸爲賊所擄。無家可歸。只得依附。今聞得將軍領天兵到來。闖賊素知將軍的威名。十分畏怯。手下的衆人越發不消說得。合營惶惶。個個怕死。大家商議了。同心歸順天朝。先差小人來稟上將軍。請將軍今夜去劫大寨。衆人願爲內應。把闖賊獲住。將功贖罪。但求將軍上達朝廷。赦免我們衆人之罪。仍放歸農。感恩不盡。姚澤民聽了。信以爲實。心中大喜。命賞了衆人酒飯。叫他們回去報說。今夜一準進兵。衆人可預備接應。天色傍晚。姚澤民傳令合營人馬全去劫營。不意到了那裡。流賊伏兵四起。他身入重圍。被衆賊殺了個片甲不存。把他生擒了去。他一見了闖賊。便大呼道。臣奉上命而來耳。諒臣豈敢與大王敵。臣非斷頭將軍。情願爲降將軍。賊闖正要〈要〉買人心。命釋其縛。待以上賓之禮。他叩頭謝恩。悅意歸降。復乞恩將他營中婦女給還。李自成傳令在各營查了與他。因賊兵多了。查了數日。方纔查出。一個不少。別的俱無恙。惟這權夫人懨懨一息。到了營中。就吿斃了。這是何故。他劫營被擒之時。闖賊預先分了一枝兵。暗暗襲破了他的大寨。將他所擄的婦女皆爲衆賊所獲。大家分用。獨這鍾氏被一夥賊奪去。在帳房中行樂。十個賊的紫金矛攻他的一個撒毛洞。起先兩三個。他覺比姚澤民的雖長短粗細不一。然各有一種異味。還欣欣得意。到五六個。便覺難當。腹中作脹。痛苦之聲不絕。衆賊愛他生得標致。不忍弄壞了他。將鞋底烤熱。在小腹上揉出積精。餘人又弄。鍾氏雖覺腹脹好些。但他一個嫩蕊柔枝。怎經得這狂風驟雨。雖算衆賊留情。他已肉穿皮塌。哼聲不絕。不能起立。他因犯了冶容誨淫四個字。這些賊那裡輕易搶得這等佳人。爭爭奪奪。遂拿他去傳營。每日輪一架帳房。十名健賊輪戰一個嬌娃。那得不到狼狽的地位。股前那一隻無珠的眼中。日夜精流不絕。額下的那兩隻眼內。昏旦淚滴無休。茶飯都嚥不下。一心想着姚澤民來救他。口中只念白兔記上李三娘那兩句。

道。

\begin{quotation}

你早來三日重相見。遲來三日鬼門關。

\end{quotation}

及至姚澤民求李自成查了回來時。二人只見了一見。鍾氏連話也說不出一句。只落了兩點淚就死了。這是他好父親嫌貧棄婿。把女兒一位命婦弄去。送來做了軍妻。得了這樣個以陽物終於營帳。李自成因姚澤民是侯。今歸順了。要加他一等。瞎賊道。他名字中有個澤字。許後來成了一統。以山西澤州爲他的封邑。先封了他一個澤國公\footnote{賊民者謂之賊。賊民即所以賊國。封他賊國公。是極。}。他欣喜無限。無可報恩。屢屢言及南京華麗富庶。女色又爲天下第一。定要求瞎賊臨幸一番。後來李自成殘殺鳳陽。皆他爲之前驅。史奇爲副。他一路行來。並無一個官軍爲敵。到處得功。瞎賊喜極。說道。若像你這樣行兵。所向直前。天下指日可定。明朝的一個花花世界。算是你獻與我的了。因此又封他做獻世大將軍\footnote{眞是個獻世大將軍。閱此偶憶一故事。昔有一人。門上懸文獻世家四字之匾。有怒其大言不慚者。夜間以紙糊去文字二字。只存獻世二字。其家次日見之怒罵。將紙扯去。是夜。人又將家字糊去。文字上一點亦糊去。只見又獻世三字。次日。其家又大罵扯去。第三夜。人又將文字糊去。家字上糊去一點。只獻世冢三字。姚澤民爲將。眞是現世種也。}。起初姚澤民一降時。李自成知他的夫人被衆兵弄死了。甚不過意。要把兵殺幾個。以安他的心。命牛金星查問。因所淫之人甚多。不得殺這許多。只得罷了。李自成有個堂姐。是李過的親姑娘。他丈夫死了。無子無女。奔了李自成來。他生得倒也不甚醜惡。銀盆的一般大臉。比那大漢子的身軀還粗夯。年已半百。鬢毛也花白了些。性極淫穢。瞎賊並無親人。只此婦是他的親骨肉了。他姐弟二人也有些瞹昧的事。此婦嫌瞎賊不濟。瞎賊一來怕他被窩中的利害。二來又憎他齒邁。所以不甚親厚。就叫此婦隨在姪兒營中。孰意這李過是畜類一樣的人。知道甚麼倫理。他同姑母也就弄起來。李氏見李過常常奉差出外搶劫。他便將營中貌美陽壯的小卒。選了四五個做了親隨。李過雖然知道。他自己也同親姑奸過。如何管得他不收幸童。這李氏合營中都稱他爲郡主。瞎賊也要替他選個郡馬。因部下沒有個大門弟的子孫。今見姚澤〔民〕是個侯子。二來要收買人心。學昭王的故智。欲厚待姚澤民。好招來明朝的將。遂令牛金星宋孩兒做媒。傳諭姚澤民。要招他做郡馬。姚澤民那般歡喜眞說不盡。不想成親之夕。是一位頭毛蒼白五旬外的老佳人。十分掃興。因係瞎賊之姐。不敢薄待。少不得盡力同他如此云云。李氏見他在此道中甚是歷練。較生平所遇之人皆勝。倒也甚是親愛。那姚澤民是強而後可的。毫無留戀。每每討個小差出去。擄些婦人作樂。李氏也不稀罕他一個。仍將舊日心腹傳進去受用。姚澤民聞知。心中反喜。以爲他有了小夫。便自己納些小妻。諒他不好意思吃醋。他二人名雖夫妻\footnote{李自成夫妻都是掛名名色。他的令姐自然是如此。}。日間相會。也還親親熱熱的談笑。一到晚來。便各人去幹各人的正務。間或兩人也還同宿。不過潦草應事而已。李自成素常極愛重他的才能\footnote{妙。姚澤民的才能只好瞎子愛他。}。故此番令他去救史奇。這勞正游夏流如何得跟着他做了參謀。勞正的妻子被官兵擄去。打聽得主帥是姚澤民。在父親官場中他都是知道的。又曉得是同省鄕里。隨後尾了來。要求恩討回。忽聽得姚侯被賊拿了去了。他遂竟入賊境來訪問\footnote{不意他境(竟)是個情種。}。正是姚澤民封侯的時候。他求見了。將始末稟上。姚澤民愀然道。有是有這個人。來時我問是宦門之媳。又且同鄕。我以妹視之。並不曾行苟且之事。後遭了一番搶敵。驚恐致疾。我乞恩尋了來。次日即故。已經數日了。現葬在某處。姚澤民差人領了他到墳前去看了。勞正痛哭了一場。他見了這一座大新墳。不知是爲權夫人而築。只說是姚澤民的厚情。感激不盡。又來叩謝。姚澤民見他習儒。又念鍾氏一脈。就留他在幕下做了一員參謀。這游夏流出家去了。如何也隨着他。天地間的事。是(每)樣罪孽都還可以懺悔。惟獨不忠不孝之罪是通於天。再懺悔不來的\footnote{又是棒喝。}。游夏流自幼不孝父母。後受了惡妻多銀那些凌虐。多銀死後。他自悔往昔之愆。發恨賣了房產。出家當了道士。因想陝西終南山內羽流有道者多。遂來投了\endnotemark[5]一個道觀中。挑水掃地。也苦了幾年。偶然出山閒遊。不意被姚澤民部下遊騎獲住。解到營中。問起也是江南人。游夏流那張嘴是極善說的。一篇奉承。姚澤民恨相見之晚。要留他在幕下。游夏流富貴心一動。情願效勞。又還了俗。姚澤民也放了他做個參謀。待他更厚。這一次帶他們南侵。這是他們惡貫滿了。勞正是他父親不忠之遺孽。游夏流是自己不孝之罪愆。都來享報應了。姚澤民領衆到了鳳陽。寂然無聞。心中甚疑。紮下營寨。差人探視。城中各門緊閉。防守嚴密。他吩咐賊兵四處看有好婦女搶幾個來要緊。再拿幾個人來審問史將軍的下落。衆賊去了一日來繳令。道。地方上聞得兵來。都是驚弓之鳥。早已逃個乾乾淨淨。遍尋婦人。一個沒有\footnote{掃興。}。只有走不動的兩個鄕老兒拿了來等令。姚澤民命帶了進來。問他前番史將軍領兵在這裡。往那裡去了。那鄕老兒戰兢兢的答道。大王爺饒我窮命罷。我鄕下人並不知道甚麼史將軍。只聽見說有一個賊頭逃到長河衛。被官軍殺了。別的小賊頭殺的殺了。降的降了。都帶往南京去獻功去了。澤民大怒。將兩個鄕老兒命帶出去砍了。令兵馬直趨六合。那些惡賊眞正利害。有幾句說他道。

\begin{quotation}

悲風慘慘。殺氣騰騰。劍㦸森森光閃閃。靑天飛雪。旌旗繞繞暗沈沈。白晝如昏。急煎煎星馳電走。慘可可鬼哭神愁。這逆賊。癡癡尚作當年想。謬謬今朝大不然。

\end{quotation}

姚澤民做了這幾年的凶賊。殘破郡縣。戕害生民。因無強將雄兵爲敵。竟忘了自己是小孩頑的皮老鼠。不濟不濟的。公然以爲是大將軍。八面威風。英雄無敵。想道。我旣然到此。可有空回之理。史奇兩次失機喪命。大王又在此敗了兩場。我今日若得勝回來。不但有多少光彩。將來凌煙閣上開國功臣。自然是我第一位了。一個一字並肩王定然有分。豈不又榮似國公。他想了這個利字。把那個害字全然忘却。欣欣得意。傳令火速進兵。數日到了六合。離城尚有二十來里。天色將暮。吩咐安營歇息。明早或打降或攻城。再作計較。正然命人相視地宜。好紮營寨。忽一騎探馬來報道。離此三里外。有一個大堡子。想是聽得大將軍兵到。都聞風逃去。一個人芽兒也沒有。家家都有柴米食物。還有好酒。特來請令。姚澤民聽見有好酒兩個字。不覺涎流。心中大喜。催到這堡中安歇。衆賊聽了。好生快樂。一擁如飛。頃刻便到。有一個小衙署\footnote{明眼人見而即悟此爲鮑信之公署也。非作書人旋謅出爲姚澤民之公館。}。做了寨府。姚澤民前日來時。恐一路擄不出好婦人來。將營中女子扮作男人帶了幾個。又選了兩個少年美賊來做龍陽取樂。此時到了署中。男女混雜一處。歡呼暢飮。那些參謀賊兵將各佔房屋安歇。見果然柴米菜蔬多有。而且家家都放着兩三罎各樣的酒。衆賊造飯吃畢。大家豪飮一番。大醉而臥。他這夥倒運的賊。竟是。

\begin{quotation}

斷頸割頭何足慮。不妨痛醉且高眠。

\end{quotation}

你說這是個甚麼堡子。人都往何處去了。是那裡來的這些酒。原來是尚智林宗(忠)慕義正在南京。史兵部接飛報說流賊不攻鳳陽。竟奔六合。探得只五千人馬。領兵賊帥係當年降賊的姚侯。史公命他三人連夜回去。隨機應戰。他三人到了六合。衆人要聚兵迎敵。尚智道。我們的人纔散去不久。喘息未定。瘡痍未復。又聚了來。未免奔疲勞困。我今不用張弓隻矢。叫他一個難逃。只用我一千人足矣。遂道。可如此如此行之。衆人大喜。遂騰空了智勇堡。人都暫移到縣中。連夜各處運了幾千罎酒。酒中都下了蒙汗藥。專候他們光臨。正是。

\begin{quotation}

準備醇醪擒逆寇。安排香餌殺凶徒。

\end{quotation}

誰知這幾千賊活晦氣。該他們一劫同歸。齊齊入了圈套。姚澤民見有好酒。就先飮了一個酕醄。何況餘賊。不吃到酩酊。一個個盡皆迷倒。半夜裡。尚智衆人探聽明白了。領着一千人。分南北兩門而入。雖有百十個不吃酒的賊還醒着。濟得甚事。一刀一個。倒不如這迷倒的還不知痛楚。他們這是殺現成的。比屠戶殺豬還省事。如砍瓜切菜一般。不到一個更次。五千流賊皆短了幾寸。做了無頭之物。不曾走了半個\footnote{山海經有一國一臂一足者。須兩人相依始能行。如比目魚相似。流賊若只有半個。如何走法。}。把一個智勇堡竟成了個枉死城。衆人到了衙署中。見姚澤民脫得精光醉臥。一邊睡着兩個標致小賊。一邊睡着三四個少年婦女。也一絲不掛。都醉醺醺睡倒。把那男女都殺了。將姚澤民綁縛起。他纔知覺醒轉。尚智素知崇禎切齒姚澤民。故將他生擒。並他的游勞二參謀同衆賊將。都一齊綁起。解到南京。馬匹器械報了數。史公詳細修了報捷的本。敍了衆人的功。打了囚車。將姚賊衆惡解送京師去了。鮑信命衆人在智勇堡外挖了個大坑。將五千賊屍同埋在一處。成了一個大堆\footnote{西廂記惠明云。把五千人做一頓饅頭饀。此則是五千人做了個土饅頭饀。亦甚慘之極矣。}。此時人皆稱爲流賊墳。這智勇堡後來荒蕪了。雖是一片空地。人皆謂之曰血湖。至今尚有遺址。且說崇禎見了史公的本。已將姚澤民等解到。聖心大悅。獻了俘。吿了廟。將姚澤民碎磔於市。衆賊梟首示市。游夏流勞正同着他們。也就短了些。弄做個身首異處了。姚家的世襲。自姚澤民降之時就削奪了。因念他祖父功勞。還不曾難爲他家屬。後因他爲賊的先鋒。誘李自成殘害了祖陵。崇禎恨極。將他妻子桂氏。同姚予民之子姚步武。俱皆正法。當日姚澤民去後。這桂氏只得姚步武盛旺二人夜間做伴。輪流更換。二人中盛旺又力壯陽強。此時家也無多馬。桂氏〔叫〕別的家人餵養。把盛旺擡舉起來做了買辦\footnote{做買辦。有趣。好使他落錢。養息身子。}。暗地吩咐他好生養息身子。果然不半年間。這盛旺手足上的厚碱(繭)。面上的皺紋都脫去。竟光潤了許多。胖胖壯壯一條結實漢子。也不似先那樣粗鹵。在肚皮上也知若許的溫存。竟會挑新取異的弄起來。桂氏心疼他了不得。十分恩愛。他先還不敢放膽。及姚予民死後。就是桂氏一家之長了。姚步武又是姪兒。料他不敢吃醋。竟將盛旺做了總管。把姚澤民的好衣服賞給〔他〕穿。一身紬緞到底。大包的銀子給他用。夜間公然如伉儷一般。姚步武知道。也甚是氣忿。但他自己也同嬸娘有私。怎敢說他。這盛〔旺〕久之也忘了是主母。儼如夫婦。大白日也竟在房中擁着桂氏同素罄香兒靑梅綠蕚五人取樂\footnote{將他衆人總敍。一齊完結了去。好。}。出門騎上大馬。在家公然野主公。出外便是侯府大管家。家中人人側目。抄斬他家之時。盛旺是他家掌事大總管。也株連捱了一刀。這也是惡奴淫主之報。奉旨又將姚華胄剖棺。焚屍抛撒。那時姚予民已故。聖恩念彼愚蠢無知。罪不及孥。將他妻女免死。發往金齒衛充軍去了。連姚廣孝的封贈都奪去。他原配享成祖。把牌位也撤了。此時磔了姚澤民。聖怒未已。傳旨命將姚廣孝掘出戮屍。衆賊(臣)奏道。姚澤民雖萬死不足擢其罪。但伊祖廣孝曾有大功於成祖。況塚中枯骨何知。徒示天恩不廣。崇禎震怒道。成祖當年豈不願克守臣節。爲廣孝〈孝〉所惑。以致起兵奪位。雖爲一時之功首。但彼已封公晉少師。榮寵極矣。今彼之子孫受先帝厚恩。承襲侯爵。反負恩降賊。勸賊殘我祖陵。殺我宗藩。屠我黎庶。毀我城池。何況此禿賊之腐屍乎。若不正其法。何以警戒餘人。且使萬世後譏議成祖爲不忠不孝不仁不慈。皆此禿賊之所使也。豈能免其爲罪之魁乎。當日他姊曾云。做和尚不到頭的。豈是好人。即此一語。彼罪案已定矣。焉可恕之。速速傳旨。衆賊(臣)見聖怒盛。把他的功罪這樣分開了。誰還敢再言。旨下到了無錫縣。地方官也只說二百多年他定成枯骨了。誰知挖了出來。是一副孔雀斑的杉木棺材。完完全全的。打開了。他面貌如生。絲毫未動。衣服見風粉碎。光光的拉了出來。將一個禿腦袋割下。身子借了狗肚子。零碎葬在他腹中了。姚廣孝在生勸燕王造反。殺害了多少忠良。萬惡滔天。他在陽世雖貴極人臣。冥冥之中不知受了多少地獄之苦。今還轉世爲姚澤民。受了一剮。波及戮屍。姚華胄却是他親生之子孫。過了二百餘年。還至於覆絕宗嗣。而況於惡禿之正身乎。爲臣不忠。做人慘刻。其報若此。寧不寒心。崇禎見慕義等屢得大功。嘆道。若有此輩十數人。賊烏足平也。又降旨。慕義林忠尚智各加右軍都督府都督。國守加都督同智(知)。其千把總加都督僉事。給賞幣鈔有差。鮑信着陞北捕廳通判。仍攝三營事務。賈文物有病。雖未到任理事。着陞兵部職方司郞中。史可法樂爲善皆能薦賢爲國。着晉太子太保兼禮部尚書職銜。旨下。衆人謝恩受職。賀喜熱鬧。是不必說。那慕義林忠尚智鮑信同衆千把都不過是一個編氓。雖然是他們忠義之心。謀勇之能。得享天祿。然而也是他們的命運好。因有感。題了四句打油。道。

\begin{quotation}

命蹇若淹留。何須去強求。

一朝時運至。談笑覓封侯。

\end{quotation}

再說李自成全部人馬回到陝西。等了許久。總不見史奇姚澤民的音耗。遣細作到南京來打聽。那細作去了些時。回來報道。他二人已被擒斬。獻俘京師。人馬喪失全盡。李自成聽說。大怒道。我自興兵十有餘年。從未有如此喪師敗衄。即傳牛金星宋獻策並衆將商議。道。我連年失盡威風。此後也不必流往別處。但厲兵秣馬。養成銳氣。直透北京。也行些假仁假義的事。要買人心。攻城掠地。一人不殺。俟到了北京。孤家高登九五之後。再發兵四出。何愁天下不歸我掌握。衆人皆贊揚道。大王神機妙算。豈臣等愚想所及。此後他各營操練兵馬。以俟大舉。要見將來如何。且看後文正傳。

姑妄言第二十二卷終



\endnotetext[1]{「枝」字原置「丈二」二字之下,據文義改。}

\endnotetext[2]{「靖」原作「請」,「功」原作「公」,據《明史》卷二六八《黃得功傳》改;下文或同,不贅。}

\endnotetext[3]{「子」字原無,據上文加。下同。}

\endnotetext[4]{「去處」原作「處去」,據文義改。}

\endnotetext[5]{「了」字原置「來投」二字之上,據文義改。}

\setcounter{footnote}{0}

\theendnotes

\part*{姑妄言第二十三卷}
\addcontentsline{toc}{part}{姑妄言第二十三卷}
\markboth{姑妄言第二十三卷}{姑妄言第二十三卷}

鈍翁曰。寫梅生得中者。彼一生情意兼篤。並無失德。且讀書一場。不博一第。何以榮其身。中而不仕。正是他之廣識高人一頭處。

鍾生梅生賡和詩詞。陶情山水。不過銷磨歲月而已。不然。一部書他兩個係正經脚色。到收場時恐太令(冷)落。未免有強弩之末之誚。

寫賞江梅爲引出郗友之故。引出郗友要明郗夫人之始末。並將充好古楊爲英收拾了去。

鍾生出京。遇榮公於張家灣。郗友進京。遇榮公於臨淸州。前後隱隱相對。郗友途遇榮公。爲他在土山置房地流寓張本。鍾悛之惡。不應有小狗子改過之兒。但鍾悛之惡。自作之孽也。已報其身矣。小狗子之改過。鍾越之遺德所致也。試以古人匹之。許善心爲隋室忠臣。許敬宗爲唐朝賊子。許遠復爲唐忠烈之士。三代忠佞大異。小狗子今日之事。不相類乎。

連寫易于仁牛質家事。一結二人之淫案。次則逐漸結去諸人。寫關爵閻良傅厚一段。不但是爲勸醒炎涼世態中人。更見得世事變遷。小人之心腸眼孔。不可只看目前也。總是作者一筆不肯放鬆。一人不肯漏去。

李賊之死。雖不足盡其罪。亦可稍快人心。

寫弘光馬士英阮大鋮三人。照應第一回內。神謂燕王云。上天已生聖人。神器已有所歸一語。今看他們所做所爲。正可謂爲大淸敺民者。李自成張獻忠羅汝才也。爲大淸敺明者。弘光馬士英阮大鋮也。

鍾生堅辭馬士英之召。又勸賈文物不受職。不但見他有識。足見那時已非世界矣。

\chapter*{姑妄言卷之二十三\\
第二十三回 梅孝廉決意辭名 鍾員外無心逢姪\\
附 易牛兩富翁報應一\endnotemark[1]生淫刻 弘光一庸主斷送半壁金甌}
\addcontentsline{toc}{chapter}{第二十三回 梅孝廉決意辭名 鍾員外無心逢姪}
\markboth{第二十三回 梅孝廉決意辭名 鍾員外無心逢姪}{第二十三回 梅孝廉決意辭名 鍾員外無心逢姪}

話說崇禎壬午之秋。梅生得領鄕薦。鍾生同宦蕚賈文物童自大約同公賀過了。鍾生旣係故交。又是至戚。等他公事畢後。又來私賀。飮酒之間。鍾生道。吾兄高捷。弟喜之欲狂。但喜中又微有些不足之處。梅生道。莫非弟徼幸後有開罪於長兄處麼。鍾生道。非此謂也。弟與兄自幼至壯。無一月不相聚數次。契厚之情。誠所謂異姓骨肉。後因弟戀着雞肋微名。在京數載。雖夢寐之中。未嘗不以故人爲念。諒吾兄自有同心。後被放歸來。復得與吾兄盤桓。方愜愚懷。今兄高中。明歲春闈得意。杏苑看花。游宦都門。又不知幾年分手。始獲再晤。正是古人所謂。

\begin{quotation}

一回相見一回老。能得幾時爲弟兄。

\end{quotation}

況弟與兄俱鮮兄弟。故鄙心未免有戀戀耳。言畢悽然。梅生大笑道。兄以弟明歲還北上麼。鍾生道。吾兄今旣折桂。明歲定赴瓊林。焉有不去之理。梅生道。弟連今歲這一番都是多舉的。弟與兄幼年同筆。觀諸子皆已釋褐。惟有弟這一領靑衿。他戀着我再不肯去。弟前入場時。主意已定。已將酒果祭過他。替他送過行了。倘得徼幸。也與他永別。即落孫山。亦與他永辭了。今幸叨一第。只算把讀書一場的債負結過就罷了。還想甚麼功名富貴不成。兄看今日這局面。尚可求仕麼。國家已如壘卵。若一入仕籍。竟去和光同塵。尸位素餐。又無此千重面甲。要呈身報國。上言得失。兄就是前轍了。設或竟言聽計從。恐大廈將傾。非一木所能支。前日有一敝友自都來。攜得有逆闖檄文。弟不能記憶全抄。內中有數語道。

\begin{quotation}

君非甚暗。孤立而煬蔽恆多。臣盡行私。比黨而公忠絕少。甚至賄通公府。朝廷之威福日移。利入戚紳。閭左之脂膏盡竭。公侯皆食肉。紈褲而倚爲腹心。宦豎悉龁糠。犬豕而借爲耳目。獄囚纍纍。士無報禮之心。征斂重重。民有偕亡之恨。

\end{quotation}

此數語切中時弊。不可因人廢言。吾兄試看今日之域中。恐非明朝之天下矣。尚何仕爲。弟從此與兄徜徉山水。做一對瀟灑閒人。雖不能效唐六如祝枝山二位先生玩世的高致。且免於流俗。脫乎汚世。世間事總不要管他。了此餘生罷了。鍾生大喜。此後果然他二人無三日不相聚。無十日不同遊。城中則冶城。鐘山。獅子山。淸涼寺。黑龍潭。桃葉渡。史家墩。秦淮河。雞鳴寺。朝天宮。紫竹林。虎踞關。鐵塔寺。小桃源。城外則牛首。祖堂寺。獻花巖。天龍寺。雨花臺。長千里。半山園。靈谷寺。棲霞嶺。木末亭。紫金山。凡是有名古蹟。盡去遊賞。流連終日。皆有留題。也不能盡記。他二人遊倦之時。或鍾生到梅生家。或梅生到鍾生家。不過是羹菜壺酒小飮。賡詩圍碁說劍。別有幽趣。不可共俗人言也。也時常與宦蕚賈文物童自大互相往來。與他們相聚。就不是這個措大的雅淡風味了。無非是大饕豪飮。擊鼓催花。豁拳行令。再不然就是梨園搬演。雜耍打跌。乒乒乓乓。一味熱鬧而已\footnote{辱翁曰。黨太尉之羊羔美酒。亦是人生一樂。}。鍾梅二生雖不耐頻劇。然都是至親。不好却得。也只得隨着逢場做戲。一日。梅生到鍾生家來。二人小齋共酌。偶然落下雨來。鍾生道。此所謂下雨天留客了。梅生笑道。但恐天留人不留耳。鍾生也笑道。這兩句俗談。竟有一個念法甚妙。道是。

\begin{quotation}

下雨天。留客天。留人不。留。

\end{quotation}

可新異否。二人撫掌大笑。鍾生道。吾兄今日在此。我二人抵足共榻。淸話一宵罷。梅生道。這是極妙的了。洗盞更酌。銜杯賞雨。鍾生道。我二人何不以雨窗共酌爲題。各賦一律。不拘五言七言。後成者罰一巨觥。兄意何如。梅生道。兄旣有此高興。弟敢不勉強從命。以步後塵。鍾生取過詩彈。遞與梅生。拈了齋儕懷偕四韻。道。用此四韻。不必拘次。任人各用可耳。遂分了筆硯。鍾生想了一想。一揮而就。看梅生時。也作完了。彼此互相請敎。鍾生先看梅生之作。是一首五言律。

\begin{quotation}

淸風來北牖。細雨灑幽齋。

座內惟知己。飮中無俗儕。

豪吟添逸興。看劍壯雄懷。

心地問高士。肥輕非所偕。

\end{quotation}

鍾生看了。道。珠玉在前。全(令)我形穢。小弟罷(罰)一杯。拙作不看罷。梅生〔道。〕弟不過是抛磚引玉。吾兄恐形我之醜。所以不肯賜敎之意耳。鍾生遞過。梅生看道。

\begin{quotation}

閒倚芸窗對舊儕。何求難助隔天涯。

紛紛\endnotemark[2]細雨催詩興。片片飛花壯酒懷。

說劍昻藏低宇宙。談詩密邇小書齋。

高歌暢飮燒銀燭。笑傲王侯非所偕。

\end{quotation}

梅生道。觀兄佳作。弟眞獻醜了。彼此獎遜\endnotemark[3]了一番。重復又飮。鍾生道。弟今日與兄做個竟日之樂。弟方纔想了十二個字。乃人生之所必有者。我與兄各拈六字。每字任意作一小詞。先成者敬一小杯。後成者罰一大杯。何如。梅生道。弟焉能與兄爲敵。若如此。弟就要酩酊了。先後皆用小杯。但分敬罰之名爲優囗罷。鍾生道。就如尊命。遂將

\begin{quotation}

貴。富。壽。衣。食。奢。吝。酒。喜。怒。樂。愁。

\end{quotation}

十二字錄出。搓成團。放在案上。梅生拈得貴富食吝愁樂六字。那六字不消說是鍾生的了。鍾生掭筆拂紙題壽字。道。

\begin{quotation}

一世渾猶春夢。日月如梭飛動。老健幾多時。二豎傍人胡閧。堪痛。堪痛。縱到百齡何用。

\begin{flushright}右調如夢令\end{flushright}

\end{quotation}

梅生題的是貴字。道。

\begin{quotation}

官將相。位侯王。聲勢豪華世罕雙。一旦到頭春夢覺。金章紫綬兩茫茫。

\begin{flushright}右調搗練子\end{flushright}

\end{quotation}

兩人看畢。各飮了一杯。鍾生心有所觸。援筆一揮而成。道。請敎。梅生纔在思索。見他已成了。笑道。弟罰一杯。方接過一看。是一調浣溪紗。題衣字。

\begin{quotation}

羅綺輕裘稱體裁。夏涼冬暖任心懷。是他頑福自應該。

露肘捉襟襤褸態。先賢曾歷不須哀。皆由前定命安排。

\end{quotation}

梅生道。且敬兄一杯。俟弟完了再領罰。鍾生飮酒。梅生捉筆寫了。遞與鍾生。道。小弟是一調憶王孫。題的是富字。鍾生看道。

\begin{quotation}

堆金積玉費辛勤。美酒羊羔日夕親。繡榻羅幃佳麗呈。任強橫。無奈時光不讓人。

\end{quotation}

鍾生道。兄之佳作。可謂後來居上了。敬服。敬服。梅生笑道。謬獎。謬獎。大呼。斟罰酒來。小廝斟了送上。梅生接酒在手。想了想。一飮而盡。擱下杯。即舉筆。頃刻題就。鍾生也連忙寫完了。先看梅生的。是食字的菩薩蠻一調。

\begin{quotation}

食前方丈杯盤列。炰羔膾鼈華筵設。五鼎款嘉賓。大烹皆八珍。恣情貪飽餟。適口誠堪悅。鼠腹易充盈。黃虀亦飽人。

\end{quotation}

梅生看鍾生的。題的是奢字。

\begin{quotation}

揮金似土逞豪強。寶馬盡銀妝。俊僕豪奴羅侍。美艷列成行。衣錦繡。食馨香。臥牙床。百年歲月。三萬時光。瞬息無常。

\begin{flushright}右調訴衷情\end{flushright}

\end{quotation}

梅生道。兄把這奢華中人說得冰冷。弟因此感動這些鄙吝的人。成了一調醜奴兒令。一筆揮完。鍾生道。弟認罰。等我寫了。一齊飮罷。遂題了一調卜算子說酒字。二人分看。梅生的道。

\begin{quotation}

一生貪鄙惟堆積。衣食難週。聚斂持籌。終日營謀只是愁。

任憑笑罵看財鹵。總不知羞。一旦休休。枉爲他人做馬牛。

\end{quotation}

看鍾生的酒字道。

\begin{quotation}

一醉解千愁。妙處無過酒。事大如天醉亦休。不必拘升斗。

稱做釣詩鈎。又調敺愁帚。不飮傍人笑我癡。樂趣君知否。

\end{quotation}

梅生道。兄之尊作。高出弟萬萬。眞令我甘拜下風。兄之敏思。豈遜於弟。有此妙想。故不肯草率下筆耳。斟上二杯。兩人同飮。各有所思。梅生道。我每人只得二題了。完了一齊飮罷。鍾生道。兄言甚妙。梅生題的是愁字。道。

\begin{quotation}

瀟瀟苦雨。旅客無資斧。囊罄黃金遭貧寠。典盡衣衫襤褸。

終年九食三旬。那堪仰面來人。破戶敗廬風雪。孤衾獨對殘燈。

\begin{flushright}右調淸平樂\end{flushright}

\end{quotation}

題樂字的秦樓月一調。道。

\begin{quotation}

交良友。論文斗酒詩千首。詩千首。春風秋月。問花尋柳。靑山流水迎門〖足庸〗(牖)。漁魚載酒耕南畝。耕南畝。高歌〔一〕曲。和聲樵叟。

\end{quotation}

又看鍾生的一調好事近題喜字。道。

\begin{quotation}

堂上老春萱。百歲猶然康健。遭際昇〔平〕時候。得親心欣忻。妻孥賢孝善承歡。兒孫盡良善。但願斑衣戲彩。富貴何須羨。看他怒字的謁金門一調。道。

人情薄。附勢趨炎逢惡。覆雨翻雲隨意作。善良遭侮謔。誤國奸邪凶虐。悍婦強奴如鍔。髮指衝冠牙盡嚼。目光如炬灼。

\end{quotation}

二人看了一遍。互相贊揚。談笑了一回。又飮了數杯。不覺漏下三鼓。也都有了幾分醺意。方同榻而臥。次日。梅生別去。不多時。又是除夕。過了元旦。到初四日。鍾生請了梅生來同飮春酒。鍾生道。新年俗例。彼此都要互相邀請。終日饕酣酒食。未免爲梅花所笑。弟久慕江梅盛蹟。因無伴侶。未得一遊。不知兄可有此高興。我二人去做此一番冷淡生活。暫脫酒肉地獄之厄。兄意何如。梅生道。妙甚。妙甚。弟生於斯。長於斯。癡長四旬。聞江梅之盛久矣。年年想去一遊。未得其便。兄若有此雅興。弟當趨陪。還有一件。我們不必拘拘定要去看江梅。隨處有可遊賞之地。就盤桓一兩日。索性過了元宵回來。便覺淸靜。鍾生大喜。二人坐了兩乘小轎。攜了三四個家僮。叫人擔着行囊食盒。出了儀鳳門。到天妃宮。在大殿上贍禮了聖像。妃姓林。四海總神。沿海諸郡縣咸祀之。靈顯特異。故人多致敬。在大殿看了看永樂時三寶太監鄭和下西洋帶來四個碧玉磉香柱。又看了殿後那塊天然玉磬。晴則燥。陰則滴水。此乃燕王篡位之後。特差鄭和下海。以覓璽爲名。實物色建文。鄭和訪覓無跡。順便帶回者。又到淨海寺。問住持僧要出那一堂白描水陸來看了。眞畫得面目如生。神情似活。其細如髮。竟不知誰人手筆\footnote{此畫十殿閻羅。被人偷去一幅。只九軸矣。俗相沿傳係西佯(洋)之物。亦鄭和帶來者。但西洋不信鬼神。何得有此也。不過妄言耳。}。又到寺後三宿岩小飮了一回。這是宋朝韓蘄王圍困金兀朮在此宿了三夜。有奸民王志敎他掘小河乘小舟遁去。故有此名。二人談論了一會興亡往事。看看日暮。就在寺內住了。次日早飯罷。叫取了幾錢香資送了和尚。起身。將午到了洪濟寺。揀一處僧房作寓。次日方去遊賞。那梅樹是數百年古物。也不知始自何代。大者有數抱。小者也有兩三圍。有亭亭獨立的。有垂偃如蓋的。有斜欹的。有側臥的。有三五株相聚一處的。有一二株獨立稍遠的。正開得爛熳。遠遠望之。竟是數百棵玉樹。香聞數里。遊人如蟻。他二人揀了一叢四五株之下。鋪裀坐飮。香氣馥郁。沁入肺腑。氣爽神淸。樂難言喩。又見那來賞玩的人。也有乘轎來者。也有坐船來者。也有徒步者。都攜着春盛食盒。還有一種攜撂春盛者\footnote{江南閒漢多。旣喜浪遊。而又無資。買些須佐酒之物。以乾荷葉包之。以囗盧瓶貯酒。親手攜來。到彼賞花。飮畢。一撂而回。故美其名曰撂春盛也。}。也有雅俗。也有男女。但這婦女們窮人家如何來得起。都是富貴人家閨秀。他恐男女混雜。也揀那數株梅樹相聚之下。都解下繡裙來。連結了繫拾(於)樹上。做了幃帳。在內中飮酒賞花。還有挾妓來遊的。還有帶着淸唱來的。絲竹管絃。宮商迭奏。又是淸幽中的一番熱鬧。眞是第一賞心的妙境。鍾生道。三十年來聞說江梅之妙。若非今日一遊。幾負梅花。二人賞玩了數日。又遊了遊燕子磯。看了一番江景。正下山來。到關帝廟前。只見一羣人圍着。鍾生同梅生也近前一看。地下跪着兩個花子。一個沒了鼻子。一個瞎了雙眼。一腿臁瘡\footnote{余向在江南內橋遇見兩個乞兒私語。一個算着倒運的帳。臨年逼節。把兩腿的臁瘡又好了。方知臁瘡是花子的本錢。}。有一個人穿得也甚齊整。是個買賣人的氣象。儘着踢打那花子。罵道。你這沒良心的奴才。你做了這樣傷天理的事。只說你長遠躱了。一般的今日遇見了我。你做了這喪良心的事。今日也到了這個樣子。眞是現世現報了。你只把我家的人還我個下落就罷了。一面說着。一面打。那花子只是喊叫。並不說甚麼。那人道。你這奴才。問着你不說。我就罷了不成。我送你到了衙門夾起你來。看你說不說。那花子打急了。說道。是我一時吃了狗屎\footnote{不是吃了狗屎。因楊爲英而賣妻。是吃羊屎。}。做錯了。你如今就把我打死了也沒用。你妹子是我賣到外路去了。那人道。賣與了甚麼人。花子道。賣與江西巡撫榮老爺家了。那人道。我不信。你如何就賣到他家。花子道。現有媒人。這個可是說得謊的。那人忿忿的又打了兩下。道。我不同你講。吿了下來。憑官處治。夾着你這奴才。追着媒人。自有個的實下落。我且尋了地方聰(總)甲來。把你兩個奴才交付明白。我再去呈狀。轉身就走。鍾生聽見話話有因。叫家人攆上那人。請他來說話。那人正走。聽得後面叫道。那位爺站站。我家老爺請你說話。那人聽見。連忙回身問道。是那位老爺。叫我說甚麼。家人指着鍾生。道。我家老爺姓鍾。是刑部員外。那人住在同城。豈不知道。忙走回幾步。到鍾生面前。鍾生與他拱拱手。他不敢回禮。但躬身道。小人不敢。請問老爺呼喚。有何吩咐。鍾生道。兄上姓。那人道。小人賤姓郗。名友。鍾生道。方纔兄打的那人是甚麼人。姓甚麼。郗友道。那個瞎子叫做充好古。當日小人的妹子不幸嫁了他這個下流奴才。一生酷好屁股。把家私花盡。後來厚上了一個兔子。叫做楊爲英。他沒有錢使。小人外邊去做買賣不在家。他竟公然把小的妹子賣掉了。那個臁瘡腿沒鼻子的花子就是他心愛的楊爲英了。小人後來回到家中。聽了這話。要去吿他。他不知如何知覺。把間破房子賣了。兩個就一齊逃了出來。躱了這十多年。不知幾時害天報瘡。弄成這個樣子。小人今日來看看江梅。偶然遇着這兩個奴才。雖然他瞎了眼。聲音舉動還影影認得。他今日到了這個地步。也就算現報在眼了。但不知舍妹下落。所以要呈官追出個底細去處。小人好尋了去看看。以盡兄妹之情\footnote{世間有如此好哥哥耶。我不敢信。果你眞是郗有矣。人有視姐妹如陌路者。見此愧否。}。鍾生聽了這話。方明郗氏到榮公家的緣故。上前一把拉着他的手。笑道。兄不必着急。令妹的始末原由。我盡知道。我曾會見過兩次。我替兄報個喜信罷。不必與那下流奴才較論。也不必驚動官府衙門了。那郗友驚道。老爺貴人。如何得知舍妹下落。鍾生道。這話說起甚長。此處也非說話之所。兄同我到敝寓。細細奉吿。郗友同鍾生梅生步着到洪濟寺來。鍾生向梅生道。這件事弟胸中胡塗了這些年。今聽得郗兄說這些原委。方纔明白。梅生道。從不曾見兄提及此事。鍾生道。連賤內跟前。弟皆不曾說。說話之間。已到了寓處。攜手共入。讓坐。郗友道。小人怎敢坐。鍾生定拉他坐了。道。兄如今是一位夫人的令兄了。郗友笑道。老爺這語甚奇。舍妹焉有這樣的福。鍾生笑着道。兄疑我是說謊麼。我當年做秀才時。在這位梅兄府上會文。回來途間遇雨。天又晚了。只得在一園中棚下暫避。遂將郗氏投水起。怎樣救他。次日送他衣服盤費。後來只說兄八月內回家。令妹就有靠了。接着那時我徼幸得中。忙忙碌碌。所以我就不曾去看。又把出京到了張家灣。如何遇見。如何相待。怎樣承他夫婦二位盛情。如今侍郞夫人難道還是假的。幸虧今日遇我。若到了官。審出根由。再行文到榮公處。說是有夫婦女。令妹一位夫人。豈不削了面皮。況且令甥也生了幾位。郗友聽說。歡喜眞說不盡。忙跪謝鍾生道。眞大恩人。若不虧老爺救拔。舍妹焉有這一步。鍾生忙扶起。大家又談了一會。郗友吿辭。滿臉喜色而回。鍾生送了出來。只見兩個小和尚跑來。道。方纔兩個花子不知爲甚事跳下江去。連泡兒也不見冒一個。就不見了。好些漁船救了一會。總不見影兒。鍾生向郗友道。也就足以洩舍妹之氣了。郗友別去。鍾生與梅生次日到燕子磯山頂上亭中坐下。俯瞰大江。見一羣少年操弧矢。賭飮江岸。內有一生。百發百中。滿座傾倒。忽見一搖船客從而觀之。嘆道。善則善矣。惜乎未盡其神也。那生慍而操弓進曰。請爾試之。搖船客令立十竿於百步外。引彀大呼道。中某節。百矢無一虛謬。諸少年大驚。邀上座。遂取觥自酌。鍾生遙見之。知爲異人。邀之上山同飮。請述姓名。彼大笑道。吾搖船客耳。有何名姓。豪飮了數觥。見鍾生的小童捧着筆硯。他立起取筆在手。蘸得黑濃。向壁上大揮。道。

\begin{quotation}

一叫蒼天一撫膺。可憐功業已無憑。

吞聲泣盡傷心淚。贏得霜毛兩鬢增。

\end{quotation}

其二。

\begin{quotation}

一葉長江萬里浮。塡胸空有半天愁。

癡心想望黃河水。逆向崑崙西北流。

\end{quotation}

其三。

\begin{quotation}

自嗟無地可依棲。只合孤舟東復西。

怪殺傷心堤畔樹。年年春暮子規啼。

\end{quotation}

題罷。擲筆。如飛而去。迨呼不顧。到江畔。跳上小船。放於中流。不知所往。二生不勝嘆異。雖知其爲隱君子。恨不識其姓字。鍾生梅生又遊了兩三日。也興盡而返。不由舊路。就進了觀音門。又看了看陳妙常女貞觀故址。進了神策門內城。又到古寧庵紫竹林二處。遊賞了兩三日。這兩處都修枯禪的眞僧。一個吃酒肉的混帳和尚也沒有。甚是幽雅。正合了古詩兩句。道。

\begin{quotation}

曲徑通幽處。禪房花木深。

\end{quotation}

他二人也合了兩句。道。

\begin{quotation}

因過竹院逢僧話。偷得浮生半日閒\footnote{偶憶一笑談。一大老與友僧相約某日到彼寺閒遊。至日到彼。亦吟此二句。主僧笑道。老先生雖閒了半日。老僧却忙了三日。}。

\end{quotation}

二人途中分路歸家。正値大雪瀰漫。鍾生坐在轎中。賞着那亂瓊碎玉。歸來到家中不遠。見一羣人圍在街上。不知何故。看時。都是左右街坊。忙叫住轎。下了轎。那些街坊上人先不防是他。見他下了轎。都躱避不及。上前道罪。道。不知老爺駕到。失於迴避。多有得罪。老爺貴人。大下着雪。就坐着過去也罷了。鍾生道。列位是甚麼話。都是好街鄰。這可使得\footnote{眞古道君子。使輕薄兒郞愧殺。}。列位。這樣大雪在此有甚麼貴幹。內中一個姓金的。名叫金德性。是鍾生緊鄰\footnote{可記着此人。}。上前答道。不知何處來了一個花子。凍死在這裡。是我們地方上的事。所以同在這裡看看。鍾生忙問道。竟死了麼。衆人道。纔摸他的胸口。還有些溫熱。但誰敢擔這干係。擡了家去救他。只好看着斷了氣。報官去罷了。鍾生聽了。艴然變色。道。豈有此理。救人一命。莫大陰功。況惻隱之心。人皆有之。那裡有個見死不救的理。遂吩咐家人道。你們同轎夫快把這人擡了回去。那家僮上前一看。道。這個樣子是活不得的了。何苦擡個死人到家去惹是非。鍾生喝道。胡說。就是死在我家。衆位高鄰都是證見。難道這樣一個人。還怕人說我圖財害命不成。他就死了。我與他一口棺材埋葬了。也是一點仁心。衆人道。老爺的恩德。這是極好的事。衆街坊巴不得要推乾淨。向轎夫道。你擡着老爺的轎。我們幫着送了這人去。衆人上前擡了那乞兒到鍾生家來。鍾生也不坐轎了。隨衆人踏着雪。步了來家。把他擡到一間小房內。放在一張床上。衆人作別去了。鍾生命家人替他撣淨了雪。叫取了副鋪蓋來與他睡下。燒了些薑湯灌下。睡了好半日。漸漸甦醒過來。鍾生大喜。忙叫取了熱酒來。叫他吃了兩鍾。又煮了稀粥。叫他吃了半碗。鍾生吩咐家人照看着他。然後回到上房去安歇。鍾生見了這乞兒。就像至親骨肉一般。由不得心裡惦着。再睡不着\footnote{但恐近日至親骨肉未必如此。}。天纔微明。就叫人煮粥與他吃。親自又起身去看。見他也動得些了。叫家人取了兩件綿衣。一條綿褲。與他穿上。還叫睡倒。扶養了兩三日。那乞兒已好了。他原沒有病。不過是凍餓壞了的。得了這幾日的飽食暖衣。屋裡大盆火生着。暖氣騰騰的。自然就好了。那日鍾生來看他。他慌忙爬下床來。跪倒叩謝道。小人已是死了的。蒙老爺天恩救拔。殺身也感報不盡。鍾生拉起他來。道。你姓甚麼。是那裡人。爲何就到了這個地步。那人見問。哭着說道。小人姓鍾。就是本京人。原也是好人家兒女。祖上都是詩禮人家。因爲自己不長進。自幼貪賭好吃。纔到了這個地位。也是自作自受。怨不得人的。鍾聲(生)聽得他是同姓。又覺得他彷彿像當日哥哥的形狀。心有\endnotemark[4]所觸。忙問道。你可有父母麼。今在那裡。他聽見問這話。越發大哭起來。答應不出。鍾生道。問你緣何不說。他方道。老爺若問到這上頭。我越發該死了。所以不敢答應。鍾生道。你只管說。他道。我父親原在此處住。後搬到淸江浦去開店。爲了一場人命。把房子也賣了。纔救出命來。小人不成器。賭輸了沒得還人。將父親的幾兩銀子輸了。不敢回家。遂投了一個四川酆都縣姓顧的四衙。跟了去。這些年顧四衙又死了\footnote{酆都縣的故四衙。焉有不死者。}。小人空身出來。幾千里奔到這裡。想到淸江浦去。我又不敢見我父親。在這裡要尋我的一個叔叔。總問不着。年程荒旱。幾個錢用完了。衣服也當賣吃了。後來沒法。只得討飯。誰知連飯也化不出來。所以流落到這個田地。肚裡空着。前日遇那場大雪。故此就凍倒了。要不是老爺的天恩憐救。小人此時也餵了豬狗了。鍾生見他說的與向年嫂子話相近。忙又問他道。你叔叔叫甚名字。他做甚麼事。〔他道。〕我的那叔叔比我只大三四歲。離他時。他纔十來歲。我只七八歲。如今就在眼前也不認得。也不知他做何事業。所以找尋不着。他的名字我常聽見爹媽說。他在城外公家讀書。叫做鍾情。鍾生聽說。知他是小狗子了。却不認得。又眞(問)了一句道。你父親叫甚名字。你母親姓甚麼。他道。我父親叫做鍾悛。我母親姓鄂。我叫做小狗子。鍾生上前一把抱住他。哭道。我的姪兒。我就是你親叔叔鍾情了。小狗子把他看了一看\footnote{看了一看他。妙。猶相逢是夢中也。}。重復跪倒。叩了幾個頭。放聲大哭了一場。鍾生把他拉着到了內裡。指着錢貴。對他道。這是你嬸娘。他也叩了頭。又指着代目。道。這是你小嬸娘。他又要叩頭。鍾生拉住道。作揖。〔他〕把手一揖。又叫了鍾文鍾武來拜見了哥哥。然後叫他坐下。〔問道。〕你父母如今可知道怎麼樣了。他又哭起來了。〔道。〕姪兒不肖。自從出來。如今已十多年了。並不知父母音耗。鍾生也流着淚。將他上京會試時。遇見鄂氏已嫁了何家。並他父親已死了。無力買地水葬的話。\endnotemark[5]對他說了。那小狗子聽了這話。站起來向着牆儘力一頭撞去。血流滿面。倒在地下。鍾生驚得忙抱住。叫道。姪兒。你快醒來。叫了有多聲。只見他喉中聲響。總不做聲。鍾生要熱水。錢貴忙遞過。撬開牙灌了幾口。聽得喉中一聲響。吐出兩口鮮血。大哭道。姪兒此刻就死已是遲了。叔叔不殺我。還救我做甚麼。鍾生哭道。那是你幼年無知。你如今就死也救不轉你父親了。你若能改過自新。你父親也就瞑目了。勸撫了一會。替他把血拭了。包好了頭。扶他起來。叔姪二人悲悲切切。連晚飯都沒吃。過了一夜。次日。叫他洗浴了。鍾生取出自己一身新衣。叫他徹底更換。這日梅生來。聞知他們叔姪相逢。約了宦蕚賈文物童自大公分來賀。鍾生領着小狗子都去回謝。又請酒。也鬧了數日。鍾生每日留心看姪兒可能改過。見他時常提起父母來就暗暗悲啼。鍾生甚慘然。知道他有自悔之意。心中暗喜。又暗地吩咐鍾用。叫誘他外邊去戲耍。他總不聽。後來多次了。他怒起道。我是要該死的人。叔叔把我還當人看。我再有絲毫不成器。不但叔叔殺我。我父親陰靈自然就殺了我了。再要來這樣引誘。我就吿訴叔叔與你了不成。鍾用復了鍾生。鍾生又悲又喜。喜的是姪兒改過。將來可以接續哥哥一脈。悲的是姪兒雖然會着了。但哥哥已沒了。嫂又嫁了人。一家永不能再會了。過了幾日。鍾生替他起了個名字。叫做鍾自新。字又新。又遲了兩個月。鍾生叫媒人替他尋媳婦。他知道了。對鍾生道。姪兒蒙叔父收養。侍奉一生。再不娶媳婦的。鍾生道。這是何故。他又哭起來。道。我父親因我氣死。母親因我父死無依。方纔嫁人。姪兒若是長進。父親未必得死。就是父親病故。有我養活。母親也未必改嫁。想到這裡。恨不得自己拿刀割出心來。姪兒如今死有餘責(辜)。還敢望娶妻生子的受用麼。說着。流淚不止。鍾生也滴了幾點淚。正色道。你說的固是。但你父死者已不能復生。你可知道書上說。不孝有三。無後爲大。你若不娶妻。豈不〈能〉絕了你父親的後嗣。這是因小而廢大了。他又哭着道。叔叔敎導。姪兒焉敢不聽。但是我父親雖不能復生。我母親如今現在人家。不知作何光景。我忍心在這裡快樂麼。鍾生嘆了幾口氣。道。好好。你的意思怎麼樣。他道。姪兒想要去\endnotemark[6]祭奠祭奠父親。看看母親的光景。回來再做商議。鍾生道。這是極好的事。我成你的孝思。遂取出二十兩銀子遞與他。道。你拿去做盤纏。他道。那裡用得這些。四五兩銀子就夠了。鍾生也是試他。看他見了銀子拿他花費不花費的意思。見他說多。也不好收回。便道。你母親嫁的那家也甚貧窮。你用不了的。就與了你母親罷。鍾自新見叔叔說得關切。也就收下。第二日天未亮。他就來辭叔嬸。鍾生又叮囑早回。他起身去了。過了二十多天。鍾生在房中向錢貴道。此處到淸江浦不過有五六天路程。往返半月餘就夠了。他如今去了許久。還不見回來。不知何故。過了幾日。只見鍾自新面帶喜色進來。向叔叔嬸母作揖。鍾生問道。我正在這裡念你。你回來了。你母親可好麼。鍾自新道。母親同姪兒回來了。鍾生驚問道。他在人家。如何得同你來。他道。姪兒到了那裡。找着了母親。那繼父已死了兩三個月。母親正孤身無依靠。姪兒祭了父親。帶的盤纏多了。又替母親旋製了幾件衣裳。所以躭遲了日子。雇了一隻小揚州划子到了儀眞閘上換了滿江紅。同母親來了。現在旱西門外石城橋泊着。鍾生道。旣然來了。你爲何不同他來家。他道。母親說他曾嫁過人家。不知叔叔許回來不許。因此\endnotemark[7]不敢同來。鍾生〔道。〕這是甚麼話。你母親當日也是萬不得已。今日旣來。焉有不來家之理。遂叫家人雇轎夫擡轎。隨姪兒去接。吩咐備下酒飯。不多時鄂氏到了。鍾生率領着錢貴代目兩個兒子都接到廳上。進來哭了一場。然後見了禮。衆人見鄂氏時。兩鬢斑白。已是老媼了。大家訴說幾年的往事。然後安席接風。歡聚飮酒。鍾生收拾了一個獨院三間。原是小廳。間隔了與他母子同住。又與了鄂氏一個小婢。又派了兩個僕婦輪流供送茶飯。梅生知他嫂姪重圓。知會了宦賈童三人。李氏侯氏鐵氏富氏都來看賀。錢貴留下酒飯。鍾生着鍾自新進來謝了四位親家母。李氏因問鄂氏道。令郞可有了親事沒有。錢貴接着道。還不曾有岳家。正叫媒人替他尋着呢。富氏道。我倒看見一家有個好女兒。生得貞靜賢淑。模樣又乾淨。我去提了看。錢貴道。這好極了。但不知是誰家。富氏〔道。〕原是我家門下鮑信之。他如今不做了北捕廳通判了嗎。他的娘子請我。有他一個嫡堂小姑陪我。我說的就是〔他。〕他的親哥哥是個秀才。錢貴忙下來。斟上了一鍾酒敬富氏。又拜了一拜。笑道。全仗鼎力了。富氏回拜。笑着道。事還不知成與不成。我倒先吃了媒酒。錢貴道。親家奶奶去說。再沒有不成的。天晚散去。錢貴對鍾生道。鍾生聞〔之〕甚喜。次日。又親去托賈文物。賈文物也允諾。他夫妻二人商議了一番。去請了含香妯娌來當面講。遂差人去請鮑大奶奶二奶奶二位閒敍。請了來。飮酒中間。富氏提起這門親事。含香滿口應承。貞姑道。回去同丈夫商議回話。晚了辭歸。次午。含香打發一個僕婦來說。親事允了。請鍾老爺這裡着人到二房去求便成。賈文物遣人與鍾生說知。鍾生知道鄔合與鮑信之是故交。請了鄔合來。煩他去求。鮑復之允了。鍾生擇日行聘。又選吉辰娶了進門。果然好個媳婦。他是自幼跟着貞姑陶鎔出來的。知文達禮。十分賢孝。鄂氏得了這樣個好媳婦。喜是不消說。倒像個婆婆一般疼愛他\footnote{此言謬矣。世間媳婦疼愛婆婆者幾人哉。}。就是錢貴代目也着實疼他了不得。一家和美。鍾生敬這鄂氏。還是以長嫂之禮。並不以另嫁過的人待他薄情。愛這鍾自新媳夫(婦)如親兒媳一樣。錢貴與鄂氏妯娌也甚親熱。鍾自新不但能孝順母親。他孝敬叔嬸如同父母一般。疼愛這兩個兄弟無比。眞可謂敗子回頭金不換。鍾生見姪兒如此老成。心中大悅。把家務全交付與他。自己無事只看書或賦詩。高興了約梅生同去陶情山水。俗事總不經心。鍾自新也不負叔叔所托。把家中料理板板策策的。甚有次序。鍾生一日在家。正同姪兒閒話。忽門上傳進有個姓郗的人求見。鍾生知是郗友。叫請了進來。到了廳上。郗友叩謝。鍾生忙還禮不迭。郗友道。怎敢當老爺這樣過謙。定要請起。鍾生決乎不肯。方一齊起來。作揖坐下。郗友道。前幸遇老爺。小人次日就當來叩謝。恐老爺尚未回府。因有些要緊事件。往杭州去了許久。昨晚到舍。今日特來奉叩。鍾生道。豈敢有勞臺駕。我們都係相與間。兄這等稱呼太謙。就不是了。郗友袖中取出個禮單來遞上。道。不堪微物。孝敬老爺賞人罷。鍾生接過一看。都是上樣食物。

\begin{quotation}

金華火腿。紹興笋鮝。松紅糟黃雀。鱸魚。江陰糟鰣 魚。炙鱭。衢州橘子。湖州酒楊梅。臺州天摩笋。蜜 浸雕棗\footnote{天摩嶺。言其極高之意。非天目山也。嶺上有大刹。左右有百餘家。無地可耕。土人皆採笋貨賣。即市上所賣之細綠笋也。以地得名。嶺上產棗極大。皆去核。雕鏤人物花卉。以蜜浸之。本處即賣二分一個。過客買做土物饋人。若食只甜而已矣。全無棗味。嶺上更多紫荆樹。土人掘其根。製香几筆筒匙箸瓶之類貨之。頗有佳致。}。

\end{quotation}

並惠泉酒之類。鍾生道。如何敢當這樣厚愛。決不敢領。郗友道。舍妹蒙老爺再生之恩。萬分不能報一。只不過聊盡鄙心。老爺要不收。使小人愧死了。鍾生推辭不却。然後道謝收了。擡了進去。因問道。兄近來作何貴幹。郗友道。當日原在外邊作些買賣。數年來因湖廣沿江一帶流寇從(縱)橫。所以不敢遠出。只在家株守。不過蘇杭近處走走罷了。鍾生道。兄若無事。何不到都中看看令妹。郗友道。小人也有此想。鍾生道。兄爲何還是如此稱呼。只做朋友相稱纔是。郗友道。承蒙老爺見愛。斗膽了。晚弟倒要去看看。但恐榮公位尊。難得見面。倘或一時不認起來。徒費了往返盤纏。辛苦還是小事。仰攀豪貴親戚。不遇而歸。回來有何顏面以見親友。所以欲前又止。鍾生笑道。兄所慮乃勢利中之常情。但榮公令妹決不是那種人。弟不過些須的微情。令妹夫人尚念念不忘。榮公尚如此相愛。而況兄骨肉之間乎。且令甥今年已十數歲了。焉有不認之理。兄若果然要去。弟有一字問候榮公。內中再致一函候令妹夫人。備言兄去探親的話。兄到那裡。先煩人投入。若令妹見了。自然請會。郗友大喜。稱謝不已。鍾生遂同他到書房坐下。寫了一封候榮公稟啓。併那郗夫人小啓一封。也裝在一處封了。押圖書用了。付與郗友。道。素常山東一帶土賊竊發。行旅甚難。兄不若搭船。自運河而去。庶可放心。郗友道。承老爺盛愛。敢不遵命。辭了回來。過了幾日。收拾齊備。搭了一隻長船行客貨船進京。行將及一月。到了臨淸等閘。船中無事。上岸走走。有兩箭之遙。過了閘口。見數隻大座船也泊在那裡。船頭上豎着兩面奉指(旨)榮歸的金字大牌。吹吹打打。十分熱鬧。郗友正站住了看。聽得傍邊一個人道。這不知是那位大官府榮歸故里。這般體面。又一個道。我纔在閘上聽見閘官齊集人夫伺候。有禮部侍郞榮老爺。是湖廣人。吿病回籍的船要過閘。郗友聽了。心下一驚。道。此人莫非就是我妹子的丈夫。正在躊躇。只見船上搖搖擺擺走下一個體面管家來\footnote{世上偏是大老得用之奴僕。一旦乍富之貧兒。慣會搖搖擺擺。而正經人決無此態。}。郗友上前陪着小心問道。請問大爺。這位老爺榮歸。可是原任做過江西巡撫的。那人道。可正是。你問他怎麼。郗友滿心歡喜。答道。有南京住的原任邢部鍾老爺有書問候老爺。我正要進京。不想在這裡遇見。那人道。旣有鍾老爺的書。拿來。我替你投進去。郗友道。書還在船上。大爺略等片時。我去取來。忙回到船上。換了一身新衣服。取出書子。到船邊遞與那人。他道。這是夫人的坐船。你還遠遠站着。不許你近前。等候回話。郗友便退回些立住。那家人將書拿上船去。到艙門口稟了。僕婦接入。呈與榮公。榮公拆開一看。是一封問安並謝向年厚愛的話。又一個小封寫着夫人稟啓。榮公也拆開看了。上面先是問安。並錢氏戴氏同候致謝。後方說偶然遇見夫人令兄郗友。久想夫人骨肉之情。不敢輕造潭〈前〉府相認。晚生勸其來京。特具函奉達。着其親自上投。榮公見了。忙叫丫鬟在內艙請出夫人來。把字兒念與他聽了。遂問道。這是待你刻薄的令兄麼。郗夫人聽見字兒上說的是郗友。便道。不是。那一個是我叔伯哥哥。這是我同胞的哥哥。我那年到這裡來時。他在外面做生意去了。遂問那家人道。送書子的人在那裡。〔家人道。〕現在岸上站着。郗夫人忙到窗前向外一看。果然是他親兄。忙叫道。快請舅爺上船來〈從〉相會。那家人方知是夫人的親胞兄。忙跑上岸。向郗友垂手躬身道。小人先不認得舅爺。大膽得罪。夫人請上船相會。郗友遂上船來。那家人忙搭扶手\footnote{眞可謂前倨而後恭。}。榮公接出艙門。攜手到了艙中。郗友先與榮〔公〕作揖。然後兄妹兩個人大哭了一場。見禮坐下。郗夫人叫五個外甥兩個外甥女見了娘舅。大兒已十五歲。業经娶過外甥媳婦。也拜了舅公。榮公向郗友道。我五十歲尚還無子。以爲後嗣無望了。自娶了令妹。今十六年中。得五男二女。實出望外。因指着大兒子。道。他名榮錫。第二的名榮杖。三的名榮浩。四的名榮燿。五的名榮臺。郗友道。此皆姑老爺忠君愛民陰德所致。舍妹亦叨福庇。郗夫人兄妹各敍了十數載的想念話。榮公問及鍾生近況。郗友與鍾生原非深交。不知其詳。只約略答了數句。榮公又問他往京可還有別事。郗友道。因別舍妹久了。欲圖一會。並無別事。榮公道。旣如此。我們同回去。吩咐家人隨舅爺去搬了行李來。在頭號客船上安歇。郗友還帶了許多南京食物做土儀的。都搬來送上。郗夫人見哥哥來得這樣體面。着實歡喜。榮公擺酒接風。入席共飮。郗友與榮公對席。夫人打橫。飮了數巡。郗夫人問可曾續絃。娶了嫂子。生了姪兒沒有。郗友道。就是那年我八月盡回家。上冬就娶了邵氏女兒續絃。到如今生了兩個女兒。一個十四。一個十一。一個兒子五歲了\footnote{此一問斷不可少。一則是兄妹敍敍家常。二來是做後娶他女兒做媳婦。若此處不說。後來便是突如其來。}。郗夫人問道。那惡人好古還在麼。郗友會意。答道。他自那年聽得我回家。便逃得不知去向。今年春間無心遇着。我要送他到官。他着了急。同楊爲英俱投江死了。郗夫人嘆了兩聲。復喜笑道。天有眼。天有眼\footnote{於情論之固可恨。於事論之當感之不置。非他一賣。何有今日。}。榮公問道。你說的是誰。郗夫人道。就是我那惡兄了。榮公點頭嘆了兩聲。道。人於骨肉無情者。豈無報應。但遲早耳。又向郗友道。我湖廣故鄕屢遭流寇殘害。似不可歸。愚意要在南京左近村中。有傍山臨水可以陶情的地方。覓一所住宅暫居。不知可有這去處麼。郗友聽說。滿心歡喜。若在南京住下。他兄妹可常相會。十分慫恿。〔道。〕離城只二十來里。乃當日謝安石所居之東山。今名土山。那個地方眞好山水。若要卜居。除非那裡方妙。榮公道。旣然有此妙處。舅兄暫歇數日。煩帶幾個小价先去覓下住宅。預備下一應器皿並動用傢伙要緊。過了幾日。煩郗友同了家人。帶了銀子。雇了快船先去。又復鍾生的回書。並謝他送\endnotemark[8]郗友來相會一事。郗友到了南京。見了鍾生。投了榮公書嗣(翰)。並謝了鍾生的引進。然後說榮公要南京卜居。鍾生也是甚喜。遂着人打聽榮公幾時可到。過了些日子。榮公到了。鍾生接到船上見了。他夫婦送了下程。再三謝向日之情。次日。錢貴戴氏帶了兩個兒子。也來船上謝了郗夫人。鍾生又請榮公與郗夫人接風。榮公辭謝。他一槪總不入城。鍾生夫婦又送了席來。榮公感謝不已。郗夫人又送了許多京中之物。先是那郗友到土山訪買房地。易于仁聞知是侍郞公要買房子田地。他住居隔壁有一所大宅。並數百畝良田。願白白相送。郗友不肯。他竟賤價售與。希圖借光。這是自有生民以來小人之常情。又不足爲笑。郗友來復榮公。已置了房子。是樣俱全備了。榮公闔家搬到土山去住。因易于仁有讓價之情。又係貼鄰。時常請來相敍。鍾生也常來看榮公。偶與易于仁會着。說起牛質是他的親家。牛質的外甥女是鮑復之之妻。鮑復之之妹是鍾生之姪媳。這樣算盤打不淸的親戚。他望着人道。刑部鍾老爺也是我的四門親家\footnote{非四門親家。乃八門親家矣。與其認這等瓜葛。不若道。鍾老爺原是我要招的女婿。還覺親熱。}。榮侍郞老爺又是我親家的好友。勢利場中。依草附木。借人爲榮者不少。惟明眼看之覺可恥可笑耳。後來榮公勸郗友也搬上土山同住。郗夫人見大姪女生得好。娶了做次子的媳婦。親上加親。分外親熱。不在話下。一日。鍾生特到土山來看榮公。榮公喜道。學生僻處鄕隅。此地竟無一可談之人。內兄還略可晤對。他又往浙江去了。承先生不棄。命駕遠臨。鄙意欲奉屈草榻數宵。古人作平原十日之飮。我輩雖非飮客。作十宵淸夜之談。不知臺意如何。鍾生笑道。晚生此軀也是毫無世事的。旣承老先生見愛。敢不遵命。但恐過擾郇廚。大費主人物料。晚生有所不安耳。榮公笑道。先生前雖降臨數次。皆因怱忙。未得深敍。心甚歉仄。今奉屈者。欲談積愫耳。但鄕村間恐無美品以款嘉賓。何敢當個擾字。兩人在書房中促膝而談。無非講些經史。談些詩文。議論些古今興亡得失。或手談數局。或品茶數甌。午後備了一桌極豐盛的酒席相待。鍾生不安道。晚生蒙老先生過愛。當以通家子姪相待便好。如此盛設。反使晚生不安了。榮公笑道。一餐便飯。先生何須過謙稱譽。鍾生道。旣承費事。只晚生一人在此。何不約易親翁同坐坐。老先生尊意可否。榮公道。此翁於世情則圓熟\footnote{果爾天下皆去得。}。恐不足以對高明談吐\footnote{毫不在此。世人但患無勢利耳。若有此二字。雖放屁。無人不贊其香美者。}。鍾生笑道。若請他來論文。或恐強其所難。請來吃酒。大約也還不妨。榮公大笑。吩咐家人去請。那易于仁聽見榮公請他陪鍾生。可有不來的。頃刻而至。相揖罷。榮公道。都係至交。就請坐罷。彼此相遜。鍾生讓易于仁齒長。易于仁讓鍾生是客。決不敢僭。讓了多時。榮公向鍾生道。先生不必謙了。請坐了罷。易親翁與學生比鄰。還算半東。先生遠來是客。倒是托契的好。況又非大席。何必過遜。易于仁道。榮老大人尊言是極。我小弟是決不敢僭老親翁先生的。鍾生只得道了罪。坐了客位。易于仁還要讓榮公對陪。榮公笑道。主人有僭客的禮麼。這不消讓了。他方與鍾生對坐。榮公下陪。須臾。送上菜來。說不盡的美味。雖無鳳髓龍肝。也極盡人間佳品。飮了幾巡。送上飯來。吃畢撤了。與鍾生家人吃。又換上果碟。都是絕精下酒之物。榮公道。我們並無外客。知己相逢。要脫客套纔妙。我學生酒量不堪。與麯蘖無緣。不能奉陪。鍾先生同易公要盡其酒量方妙。鍾生道。承老先生厚愛。但晚生溝渠量耳。數杯之後。即醄然矣。易親翁尊量極宏。請寬飮數觥。以盡老先生雅愛。易于仁雖是個土財主。每常以爲雞魚鵝肉。間或廚子包的酒席有些海參魚翅之數。就是絕妙的了。何嘗見過這樣佳饌異味。俗語說。三代爲宦。纔知穿衣吃飯。雖然不過牲畜治辦的。但烹炰異樣。竟不知是如何整治。他方纔雖吃了那些美味。還有幾品不認得是何物。見別人吃。他也吃。只知美口而已。此時擺列着這些稀奇果品。異樣佳饌。酒又香得噴鼻。要去大飮大嚼。恐人笑他村氣。見鍾生讓他。可還不吃。便放量大饕。榮公是做大官的人。每常宴客。人在他面前踧踧踖踖。做出許多斯文態度來。今見易于仁這樣大啖大嚼。不知他是村俗。不曾見過大老家禮貌。反以爲他老實可喜。叫家人取了個玉杯來。連連送酒。他也杯杯的不辭。飮到掌燈以後。雖未到十分酩酊醉。也有了八分醺意。鍾生也酒夠了。說道。晚生鼠量已盈。夜深了。吿止罷。榮公還要留坐。鍾生苦辭。方纔肯了。易于仁也辭別歸家。榮公要陪鍾生同榻。鍾生再三不肯。他吿了安置。始回上房。一宿晚景不題。鍾生睡到五鼓時醒來。隱隱聽得哭聲。心下動疑。到天明時。又聽得窗外有人。雖是輕輕說話。却內中帶着咨嗟嘆息之聲。覺得有些異樣。叫家人起來去問。榮公管家進來說。易大爺昨晚回去。五鼓時得暴病死了。方纔他兒子到門口來叩頭報喪。鍾生吃了一驚。忙穿衣起來。不多時。榮公出來。也不勝駭異。早點畢。榮公同鍾生到他家去弔唁。問其病故之緣由。只見他的兩個兒子蠢蠢然毫無悲戚之容。答道。我們也不知道是甚麼病。母親說好好的睡覺。半夜裡叫心疼\footnote{何嘗心疼。或倒是耳朶疼。}。五更天就死了。鍾生聽得內中有許多婦人號哭。細聆其聲。不住點頭嗟嘆。二人回來。到書房坐下。榮公道。適纔先生在他家聽得哭聲。有許多疑色。是何緣故\footnote{鍾生好耳。榮公好目。}。鍾生道。此人之死。定有不明。晚生雖不能聆音察意。也還聽得一個大槪。那些婦人無非是他的妻妾婢婦。內中哀而不傷者。此乃衆人不得不哭。不過乾號而已。此無所關心者也。內有數人。哭旣不哀。聲又帶懼。不知何故。只有一個哭得哀慟迫切之至。其中倘有他弊。異日必自此人身上明之。晚生鄙見如此。或他日有驗。亦未可知。榮公點頭嘆了幾聲。鍾生住了數日。辭了回家去了。你道易于仁如何死的。那馬蚤兒水良兒先配了那苗秀谷實。借得了種。這兩個丫頭豈有不貪主人之妾之尊。而肯爲家奴之妻之賤的理。當日原是叫他下去借種。旣已借得。自然要回復主人。況且若生得兒女。將來還想做副主母。就吿訴了易于仁。又叫了兩個人上來。但易于仁的妾婢甚多。他雖好淫。但以一人之身。焉能盡供許多人之樂。這些婦人生於鄕。長於鄕。又遇着這樣個淫主公。可還知有甚羞恥。易于仁他是不論白日黑夜。院內房中。興到即弄的。家人男子雖一個不許上來。但他不過是個土財主。又非仕宦門第。禮樂人家。知道甚麼叫做閨門嚴肅。這些婦女瞞了他的眼。都時常往外邊走動。也都各有私夫。這水良兒馬蚤兒的舊夫苗秀谷實。雖是兩條坌漢。都陽物粗雄。腰間力猛。他二人還常常出去同他敍舊。後來頑得多次。又棄舊取新。二人私想商議互易其夫。那兩男人有何不肯。要是他自己的老婆。恐蒙龜名。或還吝嗇。這牝是主人公之物。何不可公中而用之。兩下就換了。自從換過。就任水馬二婦欲新則新。愛舊則舊。或他兩人中有一個偷空下來。遇着苗谷二人在一處。舊新就同門起來。他的牝戶竟成了田地。苗谷一齊栽種。他四人倒也過得甚是和美。從無爭競之意。已非一年。因衆婦人皆有所私。互相隱瞞。誰肯洩誰的事。內中只有鄒氏。他自從遇着那仙狐。經過他那種交媾。料道世間男子也無出其右者。倒覺淫心消退。後來生了奇姐。大了嫁了人家。易于仁也另眼看他。袁氏日夜惟以貪淫爲事\footnote{這也算得是一件事。奇談。}。家務總置之膜外。鄒氏位居其次。少不得要做了當家婆。越發尊位體重了。再不肯做淫褻的事。他知易勤易壽非夫主之骨血。將來這分家私。他女兒有多半承受。不想女兒又死了。他主持家務。一味從寬。倒也頗得下人們的感戴。易于仁自從收了焦面鬼大娘來家之後。雖喜他善淫。但面目旣已可憎。此物又寡骨精瘦。毫無可取。先還偶爾寄興。後竟不一過而問焉。這焦氏是騷得無對的人。當日名曰守寡。因無垂靑之人。實是死捱。後遇了卜通。痛弄了數年纔罷。騷氣略出了一出。今到了易于仁內邊。見他不時同人高興。恩波總總不能相及。弄得眼飽肚飢。如何過得。雖分了一個角先生。並相與了後院中那幾個老(毛)猴子\footnote{角先生毛猴子倒也堪對。}。安能解得饞吻。想以一杯之水。救車薪之火。如何能夠。後來知道人皆有副夫我獨無。不但他新來乍到。不知誰是誰人的契友。從何處而尋覓。兼之貌又不揚。他間或做些媚態去勾引人。豈但不能邀愛。且失笑者多。贊美者絕無一人。知道這樣美事輪不到他的了。一團興致化爲忿怒。怒而繼之以醋。常出去打聽。要拿別人的錯縫。出他胸中惡氣。那一日。他倒是無心下去看看他的兒子。四處尋覓不見。找到苗秀的窗下。聽得房中聲息大異。在他窗洞中一張。見苗秀同馬蚤兒在地下凳子上弄。谷實同水良兒在床上弄。兩處響聲〈所〉聞於窗外。他見了這樣美事。如何肯走。儘着站住。看了多時。四人又互相另等更換。那水良兒馬蚤兒到了樂境。那嘻笑哼叫之聲。雖不敢不(大)膽歡呼。忍不住時就露出來了。焦氏顧看這個。又顧看那個。看他兩下出出進進。不忍瞬目。眼睛都看花了。下邊的水順着把褲脚褶衣都淌濕了還不知道。見他們事畢要收兵了。苗谷二人拔出陽物。大有可觀。由不得打了一個寒噤。渾身一麻。再要看一會。恐他們出來看見。不但不能分惠稍嘗。還恐要受他輕薄。只得咬着牙。拍了拍胸\footnote{拍了拍胸。妙極。看旣不可。去又不捨。眞難㓦劃。}。兩條腿像癱了一般。酥軟難行。只得慢慢一步步走。掙着走到上邊。倒在床上。惱氣了半日。一日。見袁氏左右沒人。他悄悄吿訴。連他二人的陽物有多粗多大的東西。用手比與袁氏看。說得那弄法津津有味。袁氏聽得他們偷漢。毫不介意。後說到二人有如此之具。倒怒起來。暗想道。如今他年已五旬。精力大非昔比。叫我日夜守着這牛角先生對頭。要想嘗個好肉滋味。比奇珍異寶還難得的。他們有這樣好美物。不送來孝敬我。竟留〇〇起來了。好生可惡。因對焦氏道。你留心打聽。他們再要做此事。你來吿訴我。我去拿住他。定有好處到你。那焦氏合了他的心事。日日留神打聽。一日。易于仁進城看女婿牛耕去了。衆妾婢得了這個閒空。都去各尋對偶。這焦氏留心。見水良兒馬蚤見隱隱藏藏一溜煙也去了。他隨後跟了去。在窗洞一張。他四人正在起手。忙飛走上來。喘吁吁笑嘻嘻向袁氏做個手勢。道。他四個人又穿上了。這樣這樣呢。奶奶快些去看。袁氏同他下來。走到苗秀門口。把門一推。不想門不曾閂好。隨手而開。見他四個好弄。他們見了主母。魂飛魄散。赤條條一齊跪下叩頭。袁氏也不做聲。先向二人腰間一看。果然兩件好東西。濕達達一個紫光頭。直豎豎一撮黑鬍鬚。好生動火。坐在床上。假意怒道。你們後來瞞着我做這樣的事。該當甚麼罪。四個人不敢做聲。只是叩頭。袁氏見了這美具。一來忍不得了。二來怕誤了工夫。笑罵道。你這兩個奴才。有這樣好東西。不來孝敬我。倒孝敬了丫頭。又向水馬二人道。你這兩個淫婦。有他們這等好美物。都不送了上來。許你們私藏着受用麼。他四人聽得這話。都纔放了心。馬蚤兒笑道。久要孝敬奶奶。因不見出奇。怕奶奶不稀罕。故此不敢。奶奶要不嫌棄。叫他用力服事。二人竟站起來。水良兒就替他脫衣服。袁氏道。大白日裡脫甚麼衣裳。馬蚤兒就去褪他的褲子。他借那意兒。就倒在床上。馬蚤兒纔替他脫光。水良兒忙取過一床夾被。疊了墊在屁股底下。向苗秀道。你好好的服事奶奶。那苗秀還疑是夢。笑盈盈忙上去就弄。袁氏見谷實還跪着呢。說道。他兩個每常也弄夠了。你同焦氏弄弄去。谷實雖不愛他。奉主母之命。不敢不遵。也就跳起身來。焦氏忙自己扯去褲子\footnote{慢些。扯破了可惜。}。谷實將他按在一張破椅子上。焦氏兩足大蹺。谷實將他腿夾在肋下。做一齣懶漢推車的故事\footnote{偶憶一笑談。一偷兒入人室。正値夫妻行房。聽得婦問道。這叫個甚麼名色。夫答道。這是懶漢推車。少刻。其妻淫聲浪語。哼哼叫笑。偷兒忍耐不住。急得滿地亂走。其夫聞得。大駭。說道。那裡脚步響。偷兒道。是走路的。其人詫道。你如何在人屋裡來走路。偷兒道。你在床上推得車。難道屋裡走不得路。}。馬蚤兒要奉承主母。爬上床。在苗秀背後。雙手抱着他的屁股。加力狠推。水良兒也看上興來了。向谷實笑〔道。〕我也來幫幫你。便在後面推起。好半日工夫方散去。且說那鄒氏那日在窗內坐着。袁氏下去時不曾看見他。他却瞥見袁氏帶着焦氏出去。多時尚不見回來。也還不在心上。後來。但是易于仁不在家。袁氏便同焦氏出去。半日方回。不知何故。如此者多次。心疑道。他從來不甚往下邊去的。這些時不住往那裡去。這有些古怪。那一日。易于仁又有事他往。又見袁氏同焦氏忙忙的走了出去。鄒氏便自己出去尋探。到了院門外邊。四處望望。房子又多。不知在何處。想道。管他的閒事作甚麼。正想要回來。只見一個十來歲的小丫頭走來。鄒氏問道。你看見奶奶沒有。那丫頭道。我先見奶奶同嬸嬸焦氏到苗大叔家裡去。這一會沒看見。那苗秀的房子那裡。在拐角盡頭。是沒人來往去處的\footnote{此句下得好。不然他們做事。豈無人見聞也。}。鄒氏悄悄走到房前。見門關着。隔門窗聽聽。響聲大怪。又到窗外一張。谷實同袁氏在床上扛着兩條腿。像扯風箱一般。抽得那袁氏上哼下響。馬蚤兒在後推着。苗秀同焦氏在一條凳子上幹。水良兒在傍笑着看。那焦氏雖不敢大呼。那親祖宗親哥哥親爹爹親漢子。無般不叫出來。鄒氏一見。連忙抽身走回房中。坐下想道。這樣的事。丫頭無羞恥也還不該。奶奶一個正主。如何也做這等下賤的勾當。大白晝同丫頭老婆兩個家奴在一處淫蕩。也就〈無〉到無臉面之極了。這事對老爹說旣不好。不說又不好。倘事露。渾的帶淸的。壞的帶好的。這還是小事。設或有意外之變。那時怎了。他們到了這樣地位。人心喪盡。還有甚麼夫妻主婢的情意。還是說的好。使老爹好用心防範他。過了數日。易于仁到鄒氏房中來。鄒氏欲言又忍。吞吞吐吐的樣子。易于仁甚是動疑。再三盤問。鄒氏不得已。把所見的事相吿。又再三叮囑。你千萬不可聲明。只放在心裡。改日若是親眼見了。把兩個奴才驅逐出去。旣不落醜名。又無禍患。一天的事都完了。你此時倘鬧起來。旣無憑據。何以爲信。若說是我說的。豈不叫奶奶同我結下寃仇。就是奶奶娘家同他的親戚知道。又無贓證。不但說你誣賴他。還怪你聽小老婆的話呢。寃害嫡妻。鬧到官。你我都有罪名。那易于仁雖一竅不通。却還懂得人情。聽了鄒氏的話。也深以爲有理。隱忍在心。俗語說。狗肚裡着不得酥油。他雖然忍住。不曾發洩。見了袁氏焦氏馬蚤兒水良兒。就不像當日的面孔。一臉勃勃怒氣。將過了兩三日。那榮公約他去陪鍾生。他那日多了一杯。到家在上房堂屋中。坐一張椅子上。酒湧上來。要茶吃。那焦氏不知機。也不看他的面色。還搶尖希寵。忙篩了一鍾茶。扭扭捏捏送來。他一時觸氣。怒從心起。忍不得了。也不接茶。兜臉一掌。打了一個踉蹌。焦氏手中茶鍾掉在地下。跌得粉碎。他罵道。你這淫婦。把一個精棒棒的漢子生生被你弄死。後來你又私偷着卜先生。先生去後時。沒人愛你。你每日浪聲號哭。我倒好意收你進來。有穿有吃。我也還有些情意到你。你受用得肥瘋了。又做牽頭。同沒廉恥淫婦們養漢。罵上氣來。站起又是幾拳。踢上幾脚。打得那焦氏蹲在地下叫救命。易于仁怒氣越發。一下推倒。將褲子一把扯下。露出那若彼濯濯也的牝物。脫下鞋來。拿鞋底把光屁股並陰門亂打\footnote{打陰門。趣極。但此非受刑之所。辱翁曰。此處是紅棍舂杵之所。非鞋底打嘴巴之所耳。}。焦氏殺豬也似的叫。此時衆妾婢聽見。都來到堂屋裡。各懷鬼胎。那鄒氏只暗暗跌脚叫苦。怕他說出自己。見他醉了。又不敢勸。那馬蚤兒水良兒只知會淫。却是兩個蠢物。也不聽主人公的話頭。倚着他是有兒子的妾。上前來拉他。道。是那裡這樣無風生有的話。我們成日間在一處。那裡這麼便宜的漢就到他養。難道男女的東西都生在額頭上的麼。走到那裡就撞了一下不成。易于仁怒氣越發起來。丢了鞋。夾馬蚤兒\endnotemark[9]劈面一拳。打得跌\endnotemark[10]了幾跌。不曾跌倒。口鼻中鮮血直冒。兩手摀着臉飛跑。易于仁一手採過水良兒頭髮。撂倒在地下。拳脚齊下。脊背上打了幾拳。陰門上踢了幾脚。罵道。你們通同作弊。一同偷漢。還敢來替他分辯。袁氏先見易于仁罵的話頭有因。賊人膽虛。未免自愧。不敢出來衛護。今見打得十分狼狽。未免心疼這三人。在房中走將出來。坐在椅子上說道。哎呀\footnote{哎呀二字用處多矣。此哎呀一聲。如聞淫婦口角。}。一個人活來五十多歲。重新撒起酒瘋來了\footnote{何不自道。哎呀。一個人活了五十多歲。重新養起漢來了。}。養漢那是賴得人的。你親眼看見來麼。肏攮了黃湯酒。這麼個賊樣。無緣無故把幾個人打的恁樣兒。易于仁一跳八丈。罵道。無廉恥的淫婦。還來護衛他們甚麼。\endnotemark[11]虧你有屄臉彈子出來說話。吃魚又嫌腥。養漢又抛淸。就是你了。你沒有同苗秀谷實弄麼。你還同焦氏那淫婦兩個弄。馬蚤兒水良兒兩個淫婦推。你當我不知道麼。袁氏見他說的對住了針眼。無辭可答。又是那愧。只大哭大罵道。沒良心的忘八。我同你夫妻三十多年。你聽那個忘八淫婦調唆呢。賴我養漢。易于仁罵道。臭淫婦。你同奴才肏的不値了。反說我賴你。就要撲上去打。鄒氏見不是勢頭。抵死抱住。他此時的酒越發湧了出來。也受不住了。鄒氏扶他到屋裡袁氏床上睡下。他咬牙切齒罵道。今日晚了。我不同你們講。明日我把苗秀谷實兩個奴腿子擰將起來拷問。看他招不招。等問明。我不碎刴萬段了你這幾個淫婦。不算手段。鄒氏替他脫了上衣。安撫他睡下。他氣忿忿的怒吼了一會。就睡着了。時將三鼓。衆人都歇息。袁氏同焦氏馬蚤兒水良兒。在西間屋裡悄悄的道。這件事他怎得知道得這等詳細。明日果然拿他兩個審問起來。設或招出。我六個人的命都難保。他那惡性子是說得出就做得出的。古人說。先下手者爲強。捨了他一個。救了我們衆人們罷。馬蚤兒道。我們不敢主張。聽憑奶奶的心裡。袁氏又想了一會。就算着未必便得死。從此便斷絕了這條路。再沒得適口的了。發個狠道。罷。罷。不是他死。就是我亡。但我們下手不得。你倆悄悄的開門去叫苗秀谷實來。不多時來了。袁氏把易于仁的話向他說了。道。這是如今不好了。除非是害了他。我們纔得生路。你兩個怎麼說。那苗秀谷實是鄕村中的坌漢。一點世事都不知的\footnote{却會幹事。}。他曉得什麼叫做利害。聽得明日要處治他。不知是如何的刑罰。遂道。奶奶吩咐怎麼的。我們就怎麼的。袁氏道。我想來要勒死摀死。恐人看出形跡。我當日在家做女兒。聽人說古語。說一個女兒謀死丈夫。耳朶裡釘了一根釘子。再看不出。除非是也這個法兒纔妙\footnote{不意袁氏竟善於學古。大約他聽人說古語。未必皆是謀殺丈夫之事。其話必多。而他獨學了此一事來。然不足異也。如聖經賢傳所云忠孝節義之事不少。人皆不學。其奸臣逆子凶惡之事。而人多效之。奸猶袁氏也(之)聽古也。}。但我們下不得手。恐怕他跳起來。拿不住。那益發不好了。故此叫你兩個來。他二人道。這値甚麼。大呆子水牛還容容易易的宰呢。何況一個醉人\footnote{以主公比大水牛。妙譬。然而易于仁也只算得水牛。}。可有釘子尋根來。袁氏道。釘子倒沒有。前日一根斷火筋我撂在簸籮裡。大約也用得。尋了出來。遞與苗秀。苗秀看看道。好得很。比釘子還好。只怕他叫起來。人聽見怎處。向谷實道。你先摀着嘴。等我好釘釘。奶奶同衆人按住他的身子。不要給他動。苗秀要了個棒槌拿着。遂一齊到了東屋。袁氏同三個婆娘將他按住。谷實忙摀着嘴。易于仁醉眠如小死。一毫也不知。苗秀將火筋放入耳中。一棒槌就釘將進去。易于仁連掙也不曾掙一掙。就完帳了\footnote{刻薄一生。苦掙銀錢。臨死還掙些甚麼。}。袁氏恐他耳中流血。用棉花塡入塞緊。一毫不露形跡。悄悄打發二人出去。時已五鼓時分。故做驚慌之狀。大哭道。不好了。老爺說心疼。此時一覺就睡死了\footnote{好睡。世人有愛睡。俱當如此睡法。}。衆人忙起來看時。已冰冷鐵硬。忙替他穿上衣服。拿門板停上。蒙了臉。那易勤易壽畜生一般的人\footnote{禽獸非畜生而何。}。也不知道哭\footnote{此等孝子甚多。又不止此勤壽。}。叫他去報喪。纔去報喪。叫他在屍前守着。他就守着\footnote{父故而遵母命。眞是孝子。}。榮公同鍾生來時。鍾生聽得哭聲內中哭而帶懼者。袁氏四人。他們謀死了夫主。雖無人知。到底心中害怕。所以其聲懼。其餘的妾婢視主人如傳舍。無關痛癢。一味乾嚎而已。只鄒氏見丈夫之死。實由於他言而起。死得又甚可疑。要出頭詰問。又沒第二個幫手。又怕果是暴病睡死的。豈不結怨於袁氏。心下千思百慮。所以哭得甚哀。次日入殮。延請僧道念了幾個經。到了盡七。埋於易老兒之側。袁氏先還假哭了幾場。自棺材出去之後。惟聞得嘻笑之聲。毫無悲慟之意。只鄒氏一個。還時常哭哭。袁氏嫌上邊人多礙眼。把些妾婢都嫁的嫁。賣的賣了\footnote{雖是他嫌礙眼。却積了許多德。}。單留焦氏馬蚤兒水良兒並幾個心腹丫頭。意思要叫鄒氏改嫁。鄒氏道。我雖不曾生兒。也養過女兒嫁了人家。我已四十多歲。活是易家人。死是易家鬼。我往那裡去。一家都去盡了。我還去不着呢\footnote{暗指袁氏諸夫(人)。妙。}。到易于仁靈前痛哭半日。袁氏也不敢強他。過了些時。袁氏獨處了許久。耐不得了。晚間悄悄叫進苗秀谷實來。他同幾個婦人滾做一床。一日。夜闌人靜。鄒氏一覺醒來。忽聞得上房笑聲隱隱。起來向窗外一張。見上房窗子上燈光大亮。他輕輕開門。躡足走到窗下張看。見男女六人都脫得精光。焦氏馬爬在床上。屁股蹶高。袁氏仰臥在他背上。馬蚤兒水良兒每人抱着他一條腿。使牝戶大張。苗秀同袁氏大弄。谷實在後用力揉他。幾人一面弄一面笑。所以聲聞下室。鄒氏心中怒恨至極。却不敢做聲。忙走回來。他幾人淫褻的事甚多。不堪出口。袁氏將銀錢供着二人。華衣美食。大非昔比。也快活了有八九個月\footnote{九者數之奇也。該他們晦氣進宮了。}。且說榮公的一個會場門生姓智。是山西人。乃晉國智伯之裔。他單名一個功字。新點差南京代巡。他居官淸正。眞是鐵面冰心。人都稱他爲龍圖包老的後身。他知榮公寄寓土山。政事稍暇。減去衣從。只坐了轎。帶着十數個人。下鄕來謁見老師。村中人也並不知他是按院。剛到了村外。忽一陣旋風。夾沙撲面。在轎前旋轉不散。智按院心中一動。喝道。若係寃枉魂魄有靈。可領我衙役同往。纔說畢。那風便旋着前去。智按院吩咐兩個衙役道。你兩個快隨了這風去。看到何處止。看眞實了來回話。那兩個衙役如飛般跑着。跟定那旋風去了。他到了榮公門上。閽人傳了進去。請入相會。到廳上拜謁過。師生坐了。敍了些寒溫。獻過了茶。只見兩個衙役上前跪稟道。小的隨了風去到一座墳前。一旋就散了。小的問明附近居人。說那墳是此處財主易家的。纔葬了不到一年。榮公問其緣故。智按院道。門生纔到村外。忽然一陣旋風。聚而不散。門生覺其有異。故差衙役隨去。此事必有奇寃。故鬼魂到門生轎前來顯示。榮公不勝驚訝。道。鍾麗生眞異人也。智按院道。老師聞此而驚詫。必有所聞也。鍾麗生又是何人。乞明以見示。榮公道。內中隱微。我學生不知其詳。遂將鍾生來看他。留宿。約易于仁相陪。掌燈後時散去。次早聞他五鼓暴卒。同鍾生往弔。鍾生回來說。數個哭聲帶懼。一個哭聲甚哀之婦人。此人死必不明。叫學生記着。將來定有驗處。今日賢契遇旋風之異。彼有先知之明。豈非異人乎。智按院忙問道。此鍾麗生何人也。今在何處。榮公笑道。此人賢契豈不聞其名。即向年請罷太監監軍。被放歸來之鍾情也。麗生乃其字耳。智按院道。門生慕其芳名久矣。況他是前輩先生。明日定然去一拜訪。以伸渴仰之私。榮公笑道。他做人孤介得很。從來不會當事的。閉門推病。賢契果要會他。除非帶我一個名帖去。纔可相會。智按院道。門生初進。始歷仕途。雖有爲民伸寃理枉之心。無奈才力不及何。即如易家這一段公案。當何以究之。祈老恩師賜敎。榮公道。賢契少年英雋。何詢及於我老朽。當年鍾麗生在刑曹時。無寃不白。至今爲人稱仰。賢契但訪之與他。定有所益。智按院一恭道。領命了。榮公因他遠來。留飯而別。智按院回衙。次日即往拜鍾生。他的拜帖同榮公的名單一齊傳入。鍾生連忙出迎。一恭道。不知老公祖大人降臨。有失遠迎。得罪了。智按院笑吟吟一恭道。豈敢驚動大駕。爲罪耳。讓到廳上。揖罷坐下。智按院道。弟在都門時。聞老先生大名。渴仰久矣。常以未得識荆爲歉。昨見敝座師。談及起來。故特深誠晉謁。鍾生道。治弟草野放民。不敢干謁當道。所以老公祖大人駕臨此地。也不敢趨叩。反辱先施。獲罪多矣。按院又一恭道。豈敢。茶罷。按院顧左右道。迴避。衆人都退了出去。他將椅子拉近前。與鍾生促膝相對。說道。昨天弟謁敝座師去。方到村外。忽起一陣旋風。盤旋不散。弟覺有異。命衙役隨去。云係易姓之墳。葬未期年。敝座師道老先生向聆哭聲。便覺有寃。有前知之哲。故此弟特來請敎當作何審究。鍾生道。弟向日不過一時臆度。偶爾中耳。治弟孤陋寡聞。何敢多啄(喙)。老公祖大人素有神明之稱。此等事直饒爲之。按院道。一應詞訟。即疑難事。弟或可爲斷理。此陰魂事。現從何處究起。以何爲證據。祈老先生明以敎我。開我茅塞。不但弟感老先生厚愛。即寃死者冥冥之中亦荷大恩矣。鍾生道。老公祖\endnotemark[12]大人旣諄諄下問。敢不獻芻蕘之見。前哭得極悲慟之婦人。必有連心之苦。不能出之於口。故隱痛於心。若得此人詢之。必得其詳。衆婦必俱調來面訴。審其辭語。查其顏色。公堂之上自有鬼神。心虛者必現之於面。只細心詳審。必得其情。較勝用刑多矣。管窺之見如此。老公祖大人自另有高明。非治弟之所能測矣。智按院道。承敎了。又問道。向年同老先生爲事回來的那位關年兄老先生。可知他近況何如。鍾生道。老公祖大人與敝年兄相識麼。按院道。他令先尊與先君同年。向年又同年在翰院。弟與關年伯關年兄相聚數載。情同如骨肉。今別將二十載矣。鍾生道。關年兄貧寒素守。今住在和州孝義鄕。弟曾去看過一次。老公祖大人若按臨其地。還當靑目一二。按院道。這是自然。說罷。遂別了出來。鍾生隨去答拜了。按院次早吩咐四名差役。到土山去。將易家得用的家人訪拿兩個來。不許驚擾地方。差役領命。去到土山。訪問易家的鄰佑道。借問一聲。易家得用的管家是那兩個。那數人問道。你列位打那裡來。問他怎麼。一個差人悄悄的道。我們是上司衙門差了來的。叫他家的兩個管事的去問話。鄰舍們近來見苗秀谷實都穿上了綿紬直裰。腰中銀錢不斷。洋洋自得。儼然〔一〕副財主的身分。目中無人的樣子。有些看不得\footnote{世上此等看不得的人不可勝數。}。又風聞得他夥伴中有氣不忿的傳說。說他二人私通主母的這些醜話。街坊衆人無不痛恨。就指說他兩個的姓名。差人到他門口。恰好二人坐在那裡高談\footnote{借他二人口中。寫盡暴發戶人家子弟。}。苗秀說道。當日鹹菜梗子。或幾個鹹豆。吃酒吃飯一般也罷了。間或得個雞蛋嘗嘗。覺得馨香美味。近來這嘴還是我的。離了好菜就吃不下去。不但聞着雞蛋一股雞屎臭。連葷菜覺得沒味。我想進城去買些好肴來嗒嗒。這鄕村中不過雞肉之類。吃得很厭煩。別無可吃之物了。谷實道。正是呢。當日穿着破衲頭。赤脚穿草鞋。也不覺得。三五年做件粗布直裰穿上。自己覺得十分光彩。我如今這幾件紬衣服鞋襪。略舊了些。穿着就覺不好意思見人。臉上怪掃掃的。我也要進城去買幾個紬子來呢。明日備兩匹驢子。我同你去。苗秀笑道。你好自己低架子。我們如今還騎驢。不怕人笑話麼。叫佃戶擡兩乘轎來。我們去到了城中。在大酒館裡我請你。谷實道。早半日擾你。下半日我還席。苗秀道。我常聽見人說。城裡武定橋那裡有個舊院。全是好婊子。我當東請你去頑頑。谷實道。那使不得。俗說的好。要叫人不知。除非己莫爲\footnote{殺主時。何不作此思。}。一時被上屋裡知道了。就了不成。一惱了他。我們就要弄出當日的原身來了。嫖字趁早收拾起。還是吃的爲高\footnote{這兩句話可做戒嫖論。吃者。是吃下肚去。補益我。嫖者。是洩了出去。補益他。二便孰使益。}。二人正說得高興。兩個差人上前問道。府上有一位姓苗的。一位姓谷的。可在家。他二人看了幾眼。坐着不動。昻昻然道。我二位就是。你有甚麼話說。差人腰間摸出鐵繩套上。他兩個怒道。我又不犯法。你這是做甚麼。你是那裡來的。這樣大膽可惡。差役笑道。你犯法不犯法我們不知道。奉按院老爺的命。差來請你二位去說話。大膽得罪了。你到了衙門。等老爺替你陪罪。他兩人聽得按院兩個字。魂也不知那裡去了。忙向街鄰說道。煩老爹到我們家裡說一聲。不知爲甚事。按院老爺拿我們呢。四個差人不由分說。帶着飛走\footnote{驢也騎不成。轎也沒得坐。}。二十來里。不到兩個時辰。已拿到衙門。傳梆進去。稟稱拿到易家得用的兩個家人苗秀谷實。按院吩咐帶進後堂來。差人帶入。按院見這二人雖係鄕農。却露一臉凶暴之氣。又穿着紬衣。打扮得古裡古怪的樣子。就有幾分動疑\footnote{此所謂服之不裹身之災也。}。問道。你兩個就是易家的家人麼。二人答應道。是呀。老爺\footnote{的是江南鄕下人聲口。}。又問道。你主人是怎麼死的。有人吿你兩個知道詳細。可實說上來。他二人聽見這話。面色頓改。似有驚恐之意。苗秀望着谷實。谷實也望着苗秀。答應不出。按院喝道。問你話。怎麼不答應。苗秀含含糊糊的答道。小人們並不知道。按院道。糊(胡)說。你們旣是他的家人。主人是怎樣死的都推不知。就該打嘴。谷實道。那日小的主人在榮老爺家吃酒回來。醉了睡到五更。就沒有了。小的們是下人。在外邊住着。那知是怎樣死的。又問道。如今你家上邊還有些甚麼人。谷實答道。一個奶奶姓袁。一個生過姑娘的鄒姨娘。兩個生相公的。一個馬姨娘。一個水姨娘。還有一個主人族間的姪兒的媳婦。姓焦的焦大娘。就是他們幾個守寡。還有幾個丫頭。別的姑娘姐姐都嫁了去。按院道。焦氏旣是你主人的姪兒媳婦。怎麼也守起寡來。苗秀道。他也算主人跟前的小了。按院點頭嘆道。此人家門如此。焉得不弄出事來。吩咐且將二人寄監。即出籤差人提袁氏鄒氏馬氏水氏焦氏五名聽審。再說袁氏先聽得家人上來說。按院差人將苗秀谷實拿了去。心下大駭。不知是爲甚事。忙叫家人跟去打聽。回來報說。帶進後堂。不知問些甚事。把兩人收了監。又差人來拿奶奶姨娘同衆姑娘了。袁氏魂不附體。忙着人飛星去煩親家牛質尋情去說。牛質牛耕聽了這話。飛馬到村中來問。正値差役在廳上坐着提人。牛質先安撫了衆人。衆役都知他是尚書之弟。又是財主。自然做些情面。牛質進內去問詳細。袁氏哭道。並不知爲甚麼事。先拿了兩個家人去。又來拿我們。親家若不顧瞻我們。叫我們出乖露醜的。親家的臉面也不好看。如今也說不得了。有情面說得下來的。情願謝他一千兩銀子。牛質叫預備酒飯款待差人。每人送十兩的一個封兒。且緩停半日。留下牛耕陪着差役。他飛馬回家去求族兄牛騂。牛騂聽得有一千兩謝議(儀)。就親去拜按院。智按院本不欲相會。因牛騂做過布政。在山西是舊公祖官。只得延入坐下。牛騂說起易于仁是他的親家。不知何故。今提他家屬。要求情的意思。智按院道。聞得令親死得不明。把前日寃魂顯示的話說了。道。不過提來一問質而已。牛騂再三婉懇徇情。按院作色道。老先生爲朝廷大臣。見小民有寃者。還該除奸剔弊。令親母袁氏同諸婦固當護。而令親易于仁反不當護麼。今提了來。若無他弊。仍安然回去。倘有別故。正令親報寃雪恨之時。老先生亦當相助行之。爲何有要護庇罪人。鄙性執法如山。寧可獲罪於老先生。決不敢遵命。以負亡者。牛騂被他搶白了一場。掃興而回。按院大怒。復差役速拿前差。並立刻提衆婦到案。若稍遲延。定行重處。差役飛奔而去。牛騂復了牛質的話。牛質又到土山說與袁氏。舉家驚慌。又去求榮公。榮公推辭不管。後差又到。把前差都鎖了。牛質知道事下不來了。也不敢多管。後來的差人見按院動怒。可肯拿性命換錢使\footnote{然而世人有衣冠中人。反以爲命換錢者甚多。}。那還顧情面。闖將進去。問明白了籤上人犯。鎖起袁氏五人。哭哭啼啼。叫轎子如飛般擡到衙門。傳稟了。按院即刻陞堂。將先去的差人每人三十大板。一個個打得七死八活。拖了出去。然後叫上衆婦。點了名。就叫袁氏。按院見他滿臉驚懼之色。也還以爲婦女從未見官。故爾如此。遂問道。你丈夫死得不明。端的是怎麼樣死的。可實說上來。袁氏道。日裡在榮老爺家吃酒。一更天回來。好好的睡覺。到五更不醒。看時已經死了。不知是甚麼急病。又不知酒裡有甚麼緣故。按院笑道。據你的意思說。是榮老爺毒害他的了。按院雖問着話。眼中留神看那幾個婦人。見那三個面色赤黃無主。惟有鄒氏兩眉如鎖。悲容滿面。想道。此婦得非鍾先生所云悲而傷之人耶。詢彼自知其詳。命帶過袁氏衆婦遠遠站着。叫那鄒氏上來到公座前。用好言撫諭。道。本院看你滿臉悲氣之色。定然有傷心的事。你夫主之死。你雖未必知其詳細。但他的寃魂前日到我的轎前來顯示。必有奇寃。因此纔提你們衆人來審問。你可把你知道的前後始末之事。細細說上來我聽。本院再爲詳奪。這鄒氏向因易于仁死得不明。已一肚子的疑心說不出來。後來袁氏把衆妾婢都遣去了。又叫他改嫁。又忍了一口氣。見袁氏同焦氏馬蚤兒水良兒做了一路。同苗秀谷實竟公然大做起來。他並非耳聞。竟是眼見。越疑夫主死得有故。今見按院問他。又說夫主顯魂的話。不勝悲慟。嗚嗚咽咽。連話都說不出來。按院也覺慘然。說道。你不必悲慟。且把內中原委說明。待本院詳查。鄒氏因無證據。不敢稟說袁氏衆人的奸情的話。一面哭着。就將易于仁那日榮府吃酒回來。如何打罵焦氏。並踢打馬氏水氏。聲言次日要處治苗秀谷實。又如何同袁氏相鬧。是他勸息了。扶他在床上睡下。看看睡着了。纔各散去。次日五鼓時分。袁氏上邊叫哭說夫主死了。此係前後實話。並無虛謬。至於如何身死。則不知道。說畢。不禁大慟。按院聽了這番口詞。心內了然。叫他下去。叫上袁氏來。按院將驚堂木一拍。大喝道。你丈夫明明是你謀害。你可實供。免受刑罰。袁氏道。他各人暴病死了。與我何干。叫我從那裡說起。按院大怒。命拶起來。他抵死不肯承認。又命敲了三十。仍不肯招。吩咐放了。又叫上焦氏水氏馬氏來。也每人一拶。都不肯招認。按院想了一想。命將衆婦帶了儀門外去。叫監中提出苗秀谷實來。須臾帶到。按院道。你主人是如何死法。快快實說。二人答道。老爺天恩。小的實不知道。按院怒道。袁氏四人已供稱明白。說你二人同謀下手害了主人性命。你還敢強賴。夾起來。左右答應了一聲。揀極短的夾棍套上。收將攏來。二人從來那裡嘗過這種辣味。叫苦連天。按院道。還不實招。夾折你的狗腿。也不饒你。吩咐着實敲。纔敲夾了幾下。有些受不得了。但他兩個當日雖是凶頑下手害主。因貪愛着主母。又是主母的主意。二者怕主人次日追究。希圖脫禍。就依着高興做了。今日受這酷刑。又被按院一詐。說主母已供是他兩人。他到底是鄕民愚蠢。以爲是眞。內中也有神鬼使然。他心中想道。主母做的事。倒推在我兩個身上。何不大家供出來。便叫道。老爺天恩。小人情願實招。按院命鬆了夾棍。他兩人遂將主人如何醉了睡着。如何半夜主母命馬氏水氏叫他二人上去。如何主母主謀。叫他二人用斷火筯釘在耳朶眼內釘死的話說了。又道。這是主母吩咐小的們做的。與小的們無干\footnote{寫蠢人酷肖。殺害主人。猶云與他無干。其心腸眞可笑。}。按院叫錄了口供。又問道。你家中人也多。單叫你去謀殺主人。你兩個定有奸情。再招上來。二人強說沒有。又吩咐夾起來。二人抵死不招。按院叫帶了袁氏衆人上來。按院笑道。袁氏。苗秀谷實已招認明白。謀殺丈夫是你主謀。用火筯在耳中釘死的。你還有何辯。你只將如何通奸。如何起事。快快供招。袁氏聽說。面色如土。望着苗秀谷實。他二人罔知所措。暗暗叫苦。袁氏還不肯招承。按院道。你謀殺夫主。罪案已定。你就招出奸情。也無重罪科的了。本院不過要明始末緣由。以便定案具題耳。又叫鄒氏道。你夫主之死。他們已竟招承。但他們的奸情。你再沒有不知道的。備細說上來。此案就定了。鄒氏聽得丈夫果是他們謀害。一面慟哭。一面將他們如何通奸。是他親眼看見。是他吿訴夫主。叫他小心。自己謹防。恐他們謀害。不想他吃酒回來發作。遂致喪命。哭訴了。又將夫主死後。他衆人如何淫亂。也詳細稟上。按院又問袁氏。他見事已敗露。徒受刑罰。料不能免。都細細招了。又問他下手時如何。袁氏又供谷實摀嘴。苗秀釘耳。他四人壓在身上也說了。錄了口辭。叫他六人都畫了招。鄒氏又將易勤易壽並非夫主之子也稟了。按院叫馬蚤兒水良兒上去問。二人也實招係主人當日叫借種的事上稟。按院笑道。易于仁所爲。已非人類。一死也不爲過。但妻妾家奴非死他之人耳。命將男婦六人押去收監。鄒氏在外邊住着。聽候發落。牛耕也在衙門前聽風聲。見鄒氏出來。把他接到家中去了\footnote{牛耕前〔在〕察院門口接了香姑家去。今在按院門口接了鄒氏家去。前後遙遙一對。}。按院擬衆人的罪。疑(擬)道。袁氏因奸。主謀殺害夫主。苗秀谷實不但烝淫主母。又同謀下手殺害家主。三人皆依律凌㓾。馬蚤兒水良兒雖係同謀。未曾下手。減一等。律斬。焦氏雖未同謀。知情不首。奸因他起。致害多人。律絞。衆犯俱供明白。易于仁免燬屍檢驗。題請了上去。奉旨依議。袁氏苗秀谷實馬蚤兒水良兒焦氏剮的剮。殺的殺。絞的絞。俱正了典刑。按院叫鄒氏去。吩咐易勤易壽係家奴奸生之子。如何承得宗嗣。即行逐出。其易于仁家產。一半入官助餉。一半給付鄒氏養老。着於本族擇親友承繼夫後。發放回家。可笑易于仁半世貪淫。一生刻薄。把妻妾俱化爲淫物。自己死於非命。妻妾惡僕死於國法。雖袁氏衆人之罪。實起於易于仁倡淫之罪也。若非鄒氏化淫爲良。易于仁覆盆之寃。終莫能雪。鄒氏得繼嗣兒。享下半世之福。乃淫而能改之報也。看官須當着眼\footnote{一部書雖皆是警惡勸善。然以淫字爲第一戒。故諄切言之。看者勿以瑣絮謂也。}。易于仁借種生兒。何若繼本宗之子爲妙。愚人之愚。一至於此。貪淫刻薄。橫死絕後。以天理論之。雅當然耳。至於袁氏等之死。果易于仁之寃魂能報之耶。蓋冥冥之中神鬼爲之。不肯容此等淫婦奸夫惡奴濫婢以汚世界耳。按下不題。且說牛質這一年是他的五旬大壽。古人說得好。

\begin{quotation}

貧在鬧市無人問。富在深山有遠親。

\end{quotation}

他不但囊索(橐)中有元寶家兄。且仕路上又有尚書家兄。眞是勢利雙全的時候。這些親戚朋友送錦屛的。送壽帳的。送八仙的。送三星的。豬羊鴨鵝。果酒麪桃。紛紛而來。如蟻聚腥羶一般。眞個是其門如市。他少不得治酒席。叫梨園。懸花結綵。謝友酬親。熱鬧了十多日。纔事畢了。那牛耕自從奇姐死後。他心中自想。天地間那裡再去尋第二個妻名而夫實的女子來續絃。況且他弄婦人的事少。小子們弄他的日多。他愛長則有王彥章。愛粗則有疙瘩頭。儘可供後庭之樂。就是偶然高興。這八個丫頭的牝戶。香的香。緊的緊。高的高。水的水。無所不備。足以盤桓。故此他也不復再娶。他父親生辰。家中忙了多日。這晚無事。他同丫頭們先陰陽交合了一回。然後敎小子們同他以陽攻陽。弄得他前後飽足。方纔睡下。別的小子丫頭各尋對偶。也都狂蕩了半夜。一來連日辛苦。二來這一番豪興。都乏倦了。一齊酣睡。不想他們縱淫的時候。房中燭臺點着通宵大蠟。高罩紗燈。點得如同白晝。照着行樂。一時困倦睡去。就不曾吹滅。也是天厭人惡。不知如何。遺火房中灼將起來。這些人睡得好不受用。及至煙嗆醒時。睜眼一看。滿屋火光飛舞。濃煙迷目。又加心慌。也不知門在何處。惟喊叫救命。闔家的人都是熬乏了的。正在好睡。有睡得醒些的。耳中聽得必必剝剝的火聲。一睜眼。窗外一片通紅。急忙穿衣起來。走到房門外看時。原來是小主人房中回祿\footnote{雖是急忙起來。已是好一會了。}。忙四處跑着。高聲喊叫衆人\footnote{又是好一會。}。一面去報老主。趕着去擡水的。拿鈎的\footnote{又是好一會。衆人因是七手八脚忙亂。已許多工夫矣。}。比及到了跟前要救火時。已燒了個七八。牛質眼見得賢郞乃孫皆成灰燼了。要往火中跳。衆家人拉住了。正在勸時。這時是十一月下旬的天氣。西北風大作。風吹火勢。火趁風威。刮得火星四處亂舞。到處就灼。霎時一片通紅。一所宅子中。前後左右。無處不是火。衆人忙把牛質擡着跑了出去。苟氏自胡旦死後。又接着奇姐死了。他嘔了許多血。一病幾死。後雖好了些。成了一個痼疾。不時舉發。他思念胡旦。但一傷心。便嘔血不〈定〉止。竟以此疾而故。牛質自苟氏死後。也不曾再娶。看女兒香姑的面上。將計氏立了正。此時計氏見火緊。顧命爲上。一絲東西顧不得。單身逃了出來。牛騂並衆親友知他家被火。都率人來救。見火熱猛列(烈)。連大門也進不去。只見廳房樓屋已倒。剩了些大柱子。燒得通紅。如豎着許多大風蠟一般\footnote{奇喩。}。直燒到日午後。方纔火熄。倖虧他家四面都是風火牆。只他一家被難。竟不曾禍延鄰佑。第二日火冷了。牛質進去一看。眞正可惜。

\begin{quotation}

把一座雕梁畫棟繁華宅。化做烏焦巴弓破瓦窰。

\end{quotation}

牛質旣是心疼兒孫。又是心疼財帛。悲慟欲絕。香姑親來。再三勸慰。牛騂見他無歸。接到家中去權住。牛質要揀兒孫的骨殖殯葬。男婦大小燒在一處。知誰是誰。但是白骨都揀了出來。一處裝殮了。埋葬在奇姐一穴。他們這些男婦。眞算生同衾死並骨了。牛質的住宅雖成一片空地。他的佃房甚多。擇了一所寬大的。騰了出來。搬了去住。帶領着家人。在火燒的房基內四處刨挖。那燒燬的散碎金銀。也還獲得數千金。牛質無一日不悲慟。不到數月。就同兒孫一處往幽冥地府相聚去了。計氏將他棺殮。做齋開喪出殯。同苟氏並了骨。葬後總算家產。也還有萬餘金。見丈夫無後。知道紅梅所生之子。雖有胡旦一半工夫。本係丈夫的骨血。遂請了牛騂同衆族間並女兒香姑來家商議。要立他爲嗣。這事衆人都是知道。況這小子形容與牛質無二。也都無異言。計氏將這小子叫上來。改名牛承嗣。以繼牛家宗祧\footnote{辱翁曰。這結局是。}。紅梅也就母以子貴。體面起來。闔家稱爲姨娘。相伴計氏守節。可笑牛質父子妻媳僕婢。正是。

\begin{quotation}

淫到不堪回首處。一齊交付與西風。

\end{quotation}

一陣風助火。弄得如此結局。世上淫之一字。料人人所不能免。却不可淫到沒道理的地位。自然就生出那極慘烈的禍來。可不愼歟。結過不題。再說那關爵自得了鍾生所贈。家中尚有祖遺的薄田數十畝。惟有省儉度日。也還無求於人。他足不履戶。手不釋卷。倒也家門淸靜。人口平安。一日。閻良五十歲。關爵買了一分禮。貧淡家風。不過是雞魚鴨肉壽桃壽麪而已。打發兒子媳婦去拜祝。到了丈人家內。拜了壽坐下。創氏不瞅。半日連茶也沒有一鍾。坐了一會。只見丫頭小子如飛的跑進來。說道。傅姑爺姑奶奶來送禮拜壽來了。閻良創氏慌忙出去迎着。閻良一手拉着女婿。創氏一把攙住女兒。同進房來。正面放了兩張椅子。讓他夫妻坐。那傅金見了關必顯。待理不理的拱了拱手。富姐看見姐姐。只假意讓讓坐。創氏忙道。他們是老女婿女兒了。你二位是嬌客。不消讓得。他夫妻也竟坐了。傅家的禮物擡了進來。紬緞履襪。食物菜品。擺了一堂屋。閻良創氏滿心歡喜。一面叫丫頭僕婦收了。創氏連聲叫茶。頃刻就是茶來。創氏叫先送到傅金〔富〕姐面前。拿下了。纔叫送與關必顯貴姐。那關必顯正在少年。性氣剛傲。茶也不接。忿了一口氣。辭也不辭。徉徜走了出去。閻良創氏只當不曾看見。也並不留不送。貴姐見丈夫去了。心中也想要回去。因是父親整壽。只得耐住。見爹娘奉承妹夫妹子的樣子。心中好惱。坐不住。就走到西屋裡坐着。見爹娘那邊擺果子茶。款待妹夫妹子。竟不請他一聲。又是一口氣咽在心裡。這些下人見主人待姑爺如此。也就放肆起來。這個道。今日老爹一個整壽。你看傅姑奶奶家送的尺頭鞋襪。並許多的吃食。纔像個禮。關姑娘家那樣的東西。虧他家拿得出來。送我還不稀罕呢。又一個道。傅姑娘的是有福的。怪不得老爹奶奶疼他。關姑娘寒呵呵的樣子。不要說老爹奶奶。連我們也看不上眼。這個一嘴。那個一舌。貴姐的肚子幾乎氣脹。又不好發作。少刻。有幾個親戚家的內眷來了。創氏走過來。向貴姐道。今日你爹的好日子。衆親戚奶奶們來。像你妹子那樣體面就罷了。你又沒穿的戴的。怎麼陪人。或是問你妹子借幾件衣服首飾穿戴穿戴。或是你不出去。我叫人送兩碗菜來。你就在這屋裡坐坐罷。貴姐一聽了。由不得那胸頭的氣發將起來。一面哭着嚷道。我不過窮罷了。我難道少個鼻子眼睜(睛)。就陪不得人。我家掉了鍋底了。到娘家來討飯吃的麼。我家雖窮。公公也做過官。蹺起脚來。比那有錢村牛頭還高些。手掌看不見手背。媽也不要太認眞了。我窮的難道只是窮。富的只是富麼。我洗淨了眼睛看着你。創氏道。哎呀\footnote{如聞其聲。}。這扯淡的話打那裡來。你家窮是誰帶累你窮的麼。你罵富呀富的。牽扯着你妹子做甚麼。貴姐道。也罷。媽也你只認得有錢的女兒。我從今日去。我不得好日子過。誓不上爹娘的門。創氏道。哎呀\footnote{先一個哎呀。是衛護小女兒。這一個哎呀。是責備大女兒。神情話(活)跳。}。今日是甚麼日子。你沒得孝敬老子的。你哭哭啼啼來魘樣他麼。你來也罷。不來也罷。我也不借你公公的光來榮耀我家。料道也不求着你\footnote{辱翁曰。少刻就要求了。}。要去就請行。貴姐道。我不去。賴在你這裡麼。賭氣就往外走。閻良在外邊聽着。聲也不嘖。連下人也沒有一個送他。那家中的狗也可笑的很。不知是嫌他窮。又不知因他不上門來眼生。跟着他汪汪亂叫\footnote{諺云。人敬有時的。狗咬穿破衣。可見世上人之勢利者。人與狗同。}。貴姐到家。一把眼淚。一把鼻涕。向公婆丈夫細說。他母子皆有些氣忿。口中牢騷不平。關爵道。你們婦人小孩子見識淺薄。他當日之親厚我者。並非道義。因見我做官故爾。今他見我官壞了。仍如貧士。他自然不能如前。雖然如此。我家也承過他的厚情。但念他當年的好處。把今日的壞處就待諒過去了\footnote{辱翁曰。眞讀書人。此即聖賢絜矩。知道忠恕之心。}。那關必顯夫婦自此總不上丈人家的門。且說那閻良見女婿女兒賭氣竟然回去。他不伏氣來請。旣受了他的禮。又不好意思的。只得叫家人來請關爵。關必顯道。這樣炎涼人家。父親到他家做甚麼去。關爵道。你少年不知事。大丈夫正要在這等處看得破纔好。看了他們的行徑。不強如看戲文麼\footnote{辱翁曰。此則是英雄豪傑之心胸眼界。}。遂到了閻家來。只見那傅厚昻昻然先占了首位。見了關爵。只把手略舉了舉。還有幾個親朋都同關爵作了揖。彼此讓坐。衆人道。太祖儀制。鄕紳在間。非長親父執。不許僭坐。自然是關老爺請坐首位。閻良忙道。雖然如此說。鄕黨莫如齒。況都是至親。傅親家年長些。應坐首席的。關爵笑道。客隨主便。自然是親翁。傅厚也竟不推辭。公然竟坐。關爵又讓衆人道。內中還有齒長的。我如何好僭。衆人決定不肯。關爵坐了二席。衆人按次序坐。閻良只在傅厚面前周旋。關爵同衆人跟前。他淡淡推讓而已。席散後。關爵含笑歸家。此後兩親家竟如陌路。正是。

\begin{quotation}

天倫骨肉貧猶淡。何況婚姻兩姓親。

\end{quotation}

那傅厚一步時運好。歷年來田上大收。家中又放些帳目。積累得將有萬金之富。他一個小人。自不能知富而無驕。但那些無恥的匪類見了他。明知這種看財奴任你怎麼樣奉承他。他還捨得拿出個紙錢來給人的麼。不知是甚緣故。世人見了有錢的。他自己親像出了〖月戎〗的㞠子一般。不覺就軟了。又像個大烏龜把頭縮了進去。只剩兩個肩頭。那一種脅肩諂笑搓卵抱毬的樣子。眞看不得。所以把那有幾文臭錢的人。敬奉得不知如何尊貴。那傅厚父子就以爲是。天下第一個貴的是皇帝。第一個富的就是他了。眞是人罵的王胖子的雞巴。把他看得那多粗多大。他鄕中有一個土棍姓吳。因他生性憊賴。人都順口叫他做吳賴。他也該傅家的幾兩銀子。他原只借了十兩。五分行息。不到二年。便二十利錢。分文俱無。便換二十兩的文書一張。不消十年。滾到一百餘兩。但問他要時。便道。十多年我還欠你一文來麼。利錢年年淸你的。你儘着催甚麼。傅厚却一文不曾見。只不過換借約而已。傅厚依之不得了。叫家人去村着要。那吳賴氣恨恨的揪着那家人到他家來。恰好遇傅金在廳上。吳賴道。我該你家幾兩銀子。有了自然還你。你叫家人村我怎麼。傅金道。殺人償命。欠責(債)還錢。你該我的。怎麼不村。你旣怕村。還了我就罷了。難道我是漢子。你不是漢子麼。吳賴道。我一個雞巴割三截。拿那一截比你。我就安心不還錢。不怕你這財主扛了我去求雨。你拿你財主的勢兒嚇唬我。不要說我脚雞眼不待見你。我連雞巴還不朝你溺尿呢。那傅金是到處人奉承慣了的。誰敢挺撞他。見吳賴說了幾句這無賴的話。那裡還容得。就破口大罵道。肏娘眼的奴才。你敢在我跟前放肆。把你祖奶奶送給叫驢肏。吳賴道。人之父母。己之父母。你恃着有幾個浪錢。你傷我的祖父。你罵我就同罵你的祖宗父母一樣。都着你。都着你。傅金越怒。喝叫那家人打。那吳賴素常也會幾着三脚貓的把式。也就支手舞脚的起來。那家人敵他不住。傅金大怒。四處望了幾望。大廳傍邊豎着一根大門栓。他雙手舉起來。劈頭就打。那吳賴正同他家人相持。見那門栓下來。把頭一側。不想一下正着耳門。一交跌倒。動也不曾一動。就絕了命。家人忙去報與傅厚。他聽了。魂魄皆無。飛跑了來。見那傅金也嚇得面無人色。傅厚恐屍親來難爲兒子。叫他夫妻都躱到隔壁丈人家去。傅厚將相熟的親友請了許多來作衛護。然後去報與屍親人家知道。那吳賴的父母哥嫂兄弟老婆兒子媳婦女兒哭哭啼啼。拿棒槌的。拿短棍的。拿撥火棍的。婦女們拿着馬刷的。就來了一大陣。喊進門來。見他家人多。不敢打人。只將廳上桌椅隔扇打得粉碎。還想打到內裡去。他那內門關得鐵桶一般。衆人打得性癱了。傅家親友出來做攔停。再三再四的講私和。不必到官。將舊次的文書還他。還與他一百銀子。講來講去。說到五百外加五十兩。將屍首他各人擡回。自己發送。吳家是個窮戶。倒也肯了。那總甲里正有同傅厚對不着的。竟先去報了官。這知州姓喜名惠。聽得是財主兒子打死人命。因他老子是監生。不好拘拿。差了四名衙役。立刻拿凶犯。提屍親到案。隨命吏目帶忤作人役相驗屍傷。到了傅家。傅厚都有厚贈。治酒飯款待。一面煩親友尋門路。向知州求情。許送三千金爲壽。懇求免究。屍親底下講和。喜知州先執意不依。定要凶身。後來纔鬆口。要銀一萬。方完此案。不然定拿凶身抵償。傅厚着了急。只這一個獨子。如何捨得。出到五千還不依。講到了六千上仍不准。傅厚的家私連房屋地土不足萬金。這六千兩。連屍親五百五十。並雜項使用。七千出頭。也就算罄家所有了。再要添。加何還來得。眞急的要死。把個閻良創氏也急的恨不得抹脖子上吊。團團亂轉。那幾個差人因提凶犯不到。每日打了屁股。到他家來高坐痛吃。雖然大塊的銀子送了他。嘴裡沒話說。但終非了局。況一個死屍放在家中。着實厭氣。正在爲難。恰好智按院按臨和州。知州因接按臺忙亂。這事且暫擱起。雖得耳邊略靜。若按院去了。又將奈何。此時傅厚也就幾乎要急死了。傅金躱在閻家。總不敢露頭。且說那智按院公事稍暇。就到孝義鄕來拜關爵。把這村中人的屁都驚出來。互相傳說按院都來拜關老爺來。家家關門閉戶。雞犬無踪。按院到了關家。迎入遜坐。他二人是世交。也無套話。只說了些久別渴慕的眞情。問問所處的近況。並將前日聞得鍾生說知他家寒。因屛退左右從人。說道。地方上或有無礙的事。老年兄可尋一兩件來。弟當盡情。稍助老年兄薪水之需。關爵再三致謝。關爵因他遠來。說道。老年兄遠來賜顧。弟備一餐便飯。但鄕村中之物不堪。不敢相待。奈何。智按院道。兄與弟兩輩世交。何尚作此客話耶。一盂脫粟飯。蔬食菜羹。弟可敢不飽。關爵也不過是殺雞爲黍而食之。見其一子焉。關必顯出來拜見了。按院問習何業。關爵道。小人不才。去歲幸得遊庠了。按院甚喜。從人飯畢。然後別去。傅厚見按院來拜關爵。忙來尋閻良。到了房中坐下。道。關親家旣同按抬(臺)相厚。小兒就可得命了。但他向日來家。弟絲毫不曾盡情。待他喬梓太薄。今日不好去奉求。懇親家將前後事細說。我情願將許州尊的六千金送他。只求免提小兒。完結此案。就是造化了。閻良道。親家你待他薄。我待他也沒那些厚呢。我也有些沒面見他了。因抱怨創氏道。他當日回來時。我說或是請請他。替他接接風。或送個下程。人說的。冷竈裡着一把。熱竈裡着一把。那時依了我的話。到今日也好求人。你執定不肯。到這時候。閒時不燒香。忙時抱佛脚。有甚麼臉面去求。創氏道。啐。你一個男子漢。不拿定主意行。誰叫你來問我的。此時倒來抱怨我\footnote{辱翁曰。這却沒話答他。}。閻良道。你可記得那年五十歲。你望着大姐。把話都說絕了。至今幾年。女兒女婿都不上門。古人說。凡事留一線。以後好相見。被你說得盡情盡意。你當日說借不着他公公的光。求不着他家。過頭的飯兒好吃。過頭的話兒少說。你把話都說絕了。叫我如今去見他。只好拿褲子蒙着臉兒去\footnote{炎涼者尚知如此。何臭氏之不堪特甚也。}。那創氏大鬧起來。道。老殺才。臭忘八。不說你沒能幹。倒儘着抱怨我。如今的年程。早起不做官。晚上不唱喏。他倒了運。自然就不理他。他又有了時運。自然又該敬他。這是普天下人情之常。你難道就不曾聽說。

\begin{quotation}

白馬紅纓彩色新。不是親者強來親。

一朝馬死黃金盡。親者如同陌路人。

\end{quotation}

今日他又有了勢。再去陪個小心奉承他何妨。況是爲女婿的事。怕甚麼羞。丢了你甚麼臉面。你是個甚麼大東大西。怕拆了你的架子麼。若惱了我。我把褲帶子一鬆。拿頂綠帽子套在你頭上。那纔眞沒臉面呢。富姐拉着創氏。勸道。媽且不要吵鬧。商議女婿的事要緊。二位爹請去同關家爹講。我到裡邊去求親家娘同姐姐。閻良想了一會。說道。講不得我捨着老臉彈子同親家去走一回。據我的意思。俗說。不見棺材不下淚。竟把銀子擡到他家。他一個窮官。見了這些白晃晃的東西。就不看親戚面上。肯看家兄的面上也不可知\footnote{眞是老於世務者。}。況且栽住了他。他便推辭不得。傅厚道。有理。有理。忙回去拿出預備送知州的那六千金來。裝了六個酒罎。叫家人擡着。同到關家來。關爵聞知。見他數年不上門。今日突如其來。也疑了幾分。是見按君來拜。動了他們勢利的念頭。只得出來接着。到廳上。讓了富姐進去。那閻良同傅厚假做一臉的笑。深深一揖就跪下。慌得關爵忙還禮。道。二位親家。這是爲何。二人道。有事奉求親家。敢不跪懇。關爵道。豈有此理。我們兒女至親。何須如此。有話但請坐下見敎。弟力量可行的。無不效勞。再三讓着。纔起來坐下。二人同聲說道。自從親家回府。弟們因窮忙。總不曾絲毫盡情。着實抱愧。關爵道。我輩至戚。何必作此客話。閻良接着道。傅家女婿因人命一事。州太爺不知聽了甚麼人的謊言。說傅親家是個大財主。定要一萬兩。纔肯完事。已出到六千金。他還不依。傅親家的家私。親家所知\footnote{是同鄕緊鄰的話。}。通共不及一萬。如今連屍親雜項所費已七千有餘。所剩者不道(過)是些房地。難道不留些度日。今破着一總給他。也不能足數。因親家老爺同按臺相厚。特來奉求轉央一個情。倘事完了。六千兩不拘送按臺也可。親家老爺留下也可。只求完事。屍親底下傅親家自去打點。指着罎子道。這是銀子。先送到府上。關爵道。按臺今早遠來賜顧。承他厚情。已過分了。弟一個革職的窮官。那裡有這樣的體面\footnote{語中暗帶敲打。妙。}。況且纔相會。怎麼就好求情。倘說不下來。悟(誤)了二位親家的事。還是另尋的當門路爲妙。閻良傅厚見他推辭。又忙跪下道。若說的當。那裡還有過於親家老爺的。若念舊事。就不得結局了。傅厚落淚道。小兒若不得命。弟並無他男。也就不能活了。關爵跪下扶起。道。再做商議。正說着。裡面一個僕婦來說道。奶奶請老爺呢。關爵別了他兩人進來。關奶奶道。傅姑娘再三求我轉對你說。替他女婿說說罷。你的主意怎麼樣。關爵道。方纔二位親家說了這一會。我不好去說的呢。富姐跪下哭道。親家爹要不救你女婿。你忍心看着他死麼。關爵叫媳婦拉着他。說道。姑娘。你請起來商量。富姐道。沒有甚麼商量的。親家爹要不肯說。我也不敢起來。官(關)奶奶道。也罷。你看他們急的恁個樣子。你替他說說。看按院依不依。再做計較罷了。關爵躊躇了一會。道。也罷。我明日去說了看。你請起來。那富姐叩了個頭。纔起來了。關爵出去。富姐也跟了出來。向閻良傅厚道。二位爹。關親家爹依了。許明日替我們說去呢。二人笑吟吟忙下來作揖道謝。辭了要回去。關爵道。親家把這銀子還擡了去。事體還不知如何。等妥當了再來取。傅厚\endnotemark[13]道。老親家的金面去說。再沒個不完的事情。何必又擡去。只管請收下。二人就走了出去。富姐也同着去了。關爵送到門外回來。叫家人把罎子擡到上房。連罎放着。次日。進城回拜按院。按院留住酒飯。閒話中間。關爵見左右無雜人。說起富(傅)厚是他四門親家。伊子過失傷人。屍親都說明白了。已肯和息。州尊誤聽人言。說舍親是財主。定要伊子到官。昨日承老年兄光降寒廬。舍親托弟轉求。不敢瞞老年兄說。許弟有厚贈。祈鼎言免究。不但舍親父子感恩。弟亦叨受多矣。按院道。這些須小事。明日自當報命。關爵袖中取出個帖兒遞過。按院接過一看。舍親傅金求靑目幾個字。遞與家人接了。關爵吃畢酒飯。辭了回家。傅厚閻良走來討信。關爵把按院的話相吿。他二人喜不自勝。作了十數個揖。謝而又謝。方去了。關爵見事情已妥。把銀子取出。生平來未見這許多。也自歡喜。收入箱中。次日。按院傳了推官進來。說道。傅金過失殺傷人命一案。屍親並無異辭。喜知州無故刁難。顯有情弊。可傳諭他。叫他將此案速速完結。推官出來。向着知州說了。喜知州丢了一主大財還是小事。聽見按院知道索賄。驚得魂不附體。忙差人去傳前差。傅金免提。又差人忙傳諭吳家屍親。作速領屍埋葬。又差人去命傅厚給屍親燒埋銀兩\footnote{一連幾個差人。寫出知州驚得屁滾尿流的樣子。}。即刻將案卷注銷。稟了按臺。按院差人去復關爵。關爵送了他個折酒飯的封兒去了。又親到城中察院去謝。傅厚父子二人同閻良到關家來叩謝。富姐也來拜謝關爵夫婦姐夫姐姐。傅厚把屍親的銀子也給淸了。屍首吳家擡回。一天大事已完。感激關親家不盡。那知州打聽按臺何以得知這事。訪問得傅厚係關翰林的親家。關翰林是按臺的年弟兄。猶恐怕關爵懷恨。忙親到鄕中拜見。陪了許多不是。又送了一分厚禮。尚求在按臺前吹噓。那閻良傅厚見州官如此奉承陪罪。越發敬這親戚如神明一般。閻良備了戲酒。一來算接風\footnote{宦實回家數載。童自大始接風。是吝嗇。關爵回家數載。閻良始接風。是勢利。前後如一。而各是各人心腸。妙極。辱翁云。俗說。有心拜年節。淸明也不遲。}。二人奉謝。親自來請關爵夫婦同女兒女婿。關爵道。你我至親。何必拘此。決不敢奉擾。閻良道。一杯薄酒。原不是敬親家老爺的。不過盡我的窮心。戲都叫了來。老親家若不肯光降。我難道自己家裡吃不成。鄕中親友們看着我連親家都請不去。我就羞死了。還出得府中的門麼。關爵見他如此說。便道。親家旣費了事。我就領情便是。他方纔笑了。見關必顯在傍。說道。姑爺也請到我家坐坐。關必顯道。家父去領就是一樣。小婿是甚麼人。怎敢去擾岳父。不怕岳母見怪麼。閻良紅着臉陪笑道。你還記你丈母娘的餿話麼。那是吃屎的人。你別同他一般見識。你要惱他。我老丈人也替他陪罪。關爵向兒子道。長者命。少者不敢辭。岳父叫你。去就是了。多講甚麼。對閻良道。少刻弟帶小兒一同到府。閻良向關必顯道。今日一個外客也沒有。專請親家老爺親家太太姑爺姑奶奶。約傅親家夫妻你姨夫小姨奉陪。煩姑爺你進去請聲太太同姑奶奶。關必顯去了一會出來。道。家母就去。女兒身上有病。去不得。閻良笑道。我知道。我知道。旣這樣說。我叫你丈母親自來請。辭了回去。向創氏道。親家夫妻二位同女婿請了都來。惟有我家這位姑奶奶不肯。說是有病。大約還是記着你當日的話。我說了等你去請。創氏道。你不濟。等我去。如今時世論甚麼娘母女兒的。他要記恨不來。我就下他一跪。陪個禮。還怕他不肯麼。閻良笑道。我自己覺得我算炎涼得很了。誰知你比我還狠幾分。你有那樣狠嘴。也纔配得這副老花臉。富姐道。我同媽媽去請他。創氏富姐到了關家。逯氏讓了坐下。創氏向貴姐道。親家太太倒肯去了。你是自家女兒。倒重新做起客來推辭。貴姐道。自己爹娘。有甚麼推辭的。一來我身不好。二來恐怕玷辱了爺娘。我所以不敢去。創氏笑嘻嘻的道。罷麼。我的姑奶奶。你還記着我的餿話麼。我是待死的人。你同我一般見識做甚麼。你若惱我。就如同惱那狗的一樣。我正在這裡要借你府上的光。恐(怎)講玷辱的話。你要不去。我就跪着了。看你可過得意。富姐笑道。姐姐。媽這麼說。你再不去。也不好意思的。我跪着罷。纔要跪。貴姐忙拉了起來。關奶奶道。親家奶奶同姑娘這樣說。你還推甚麼。就同我去。貴姐見婆婆允了。又見娘同妹子的樣子又可笑又可憐。答道。奶奶吩咐。我怎敢不去。創氏道。席都齊備了。請親家太太就同去罷。關奶奶見貴姐穿着家常的舊布衣裙。說道。我有年紀的人罷了。你少年人。還換件衣服去呢。貴姐道。就是這樣好。不換罷\footnote{辱翁曰。不換衣。好。}。關奶奶也不強他。他兩家一牆之隔。出了關家的門。就是閻家的門。也不用轎子。就同走了過去。閻良又親自來邀。關爵父子也就同到他家來。傅厚爺兒兩個迎到大門外。深恭大喏。讓到廳上。關爵看時。廳西邊簾子隔了一間。堂客坐外面。兩間待官客。正中放了一席讓關爵坐。傅厚同閻良下陪。關爵道。那裡有這個坐法。傅親家年長。傅厚道。豈敢。今日特爲親家而設。弟是叨陪的。親家自然是這樣坐。讓之再三。關爵要傅厚同他並坐。傅厚決定不肯。閻良道。今日是弟特請親家老爺。傅親家決不肯僭。倒求親家老爺依實些罷。關爵只得坐了。關必顯傅金橫頭安了一席。唱戲飮酒。不必煩說。女眷們在內坐席。那創\endnotemark[14]氏好不肉麻。敬這樣。奉那樣。一會道。親家太太。不堪的東西。你請用些。一會道。姑奶奶。你只怕餓了。將就吃些兒。也盡盡我們的窮心。又說道。我這大姑奶奶此時也不錯。後來有個大造化。小小的年紀就穩重。不像人家輕狂。你看他打扮得模樣實實的。貴姐道。我家貧寒。沒有得好穿戴。裙布荆釵。原是我們窮人的打扮。創氏笑着連聲道。哎㖿。大姑奶奶你玷我麼。我說的是實話。你當我譏誚你麼。我要有這個心。就嚼舌根死了我。我說的是眞心話。奉承得婆媳兩個眞說不出的樣式。那傅奶奶同富姐沒話說強說。不笑強笑。做出那些假親熱來趨奉。當日貶淺貴姐那幾個婢婦。這個拿過酒壺來。道。姑奶奶的酒恐怕寒了。我換換。那個捧鍾茶來。道。姑奶奶。請用一杯茶。叫得那姑奶奶震心。席散了。進去更衣。衆人沒一個不簇擁着貴姐。要勻面。這個忙去捧鏡子。那個就去拿粉盒。要洗手。這個趕忙去掇水。那個慌去拿手巾。十分小心慇懃。都不足爲異。連當日望着他叫的那幾條狗。如今見人奉承他。也跟着前後搖着尾巴亂跳。也似乎來奉承之意\footnote{前後皆夾敍着狗。不過謂炎涼世態中人。皆狗而已矣。}。外面吹打上席。衆堂客也都出來上坐。外邊閻良。內裡創氏。無非一味奉承而已\footnote{此一段。看者謂作者將閻良創氏描寫太過。人之無恥或不至是。余有一相識白姓者。其親姪皆係宦而富。彼稱之曰姑太爺。更有一至親。不必著其姓。彼之二女妻堂兄弟二人。稱其弟曰姑爺姑奶奶。婿之兄弟皆曰姑爺姑奶奶。婿之姪曰相公。其兄家寒。則稱曰女婿女兒。婿之兄弟咸曰相公。此等小人。與閻良創氏又何異哉。}。傅厚兩口子也幫着打撮棒鼓趨奉。到三鼓席散時。傅厚夫妻在席上就面請了關爵父子夫妻。關爵再三辭謝。他更再四敦請。關爵只得允諾。次日。又擾了傅家一日戲酒。此後。閻良傅厚同關家親厚無比。沒三日不接女兒女婿。無十日不請親家夫婦。關爵因見世事不妙。也不叫兒子求名。置了些肥田美產。溫飽以終天年。這便是他的結局。智按院臨行。又來辭關爵。關爵因受了知州之托。向他道。州尊在地方上雖然要幾個錢。也還是他分中當得的。從不酷虐害民。求老年兄垂靑一二。按院首肯。次早。關爵又進城拜送。按院起行之日。知州送到交界。按院道。前日關年兄力薦該州在地方上頗得民心。此後更加淸愼勤。本院自有公道。不須遠送。回去罷。喜知州滿心歡喜。辭了回衙。又到鄕中來拜謝關爵推揚之德。逢時遇節。厚禮相送。不必多說。日月如流。又是崇禎十七年新歲。歲次甲申。鍾生聞得流寇漸逼京師。終日眉頭不展。飮食俱廢。每談及此。即長吁墮淚。錢貴見他如此。勸道。古云。不在其位。不謀其政。如今這些當道大老。受朝廷莫大之恩。將國事盡皆置之膜外。何況君林下小臣。做此杞人之憂何益。鍾生正色道。賢妻是何言也。我雖蒙聖恩放逐歸來。我當日也曾食祿數載。食人之食者。當憂人之憂。豈可以今日不曾做官。把朝廷之事就不經心乎\footnote{君子則謂之忠。小人必笑其迂。}。錢貴見他說得大義凜然。不勝嘆息。又過了些時。聞知李闖三月十九日攻破都城。崇禎皇帝在煤山自縊。已殉社稷。他打聽這信眞了。白衣冠向北拜祭。大哭了一場。要尋自盡。被人知覺了。合家啼哭勸止。他只是哭泣。堅執不聽。中(鍾)自新同着鍾文鍾武日夜守着他。寸步不離。錢貴暗暗着人對梅生說了。請他來勸解。梅生來了。勸道。合城多少鄕宦。未聞以身殉國者。兄何必乃爾。鍾生道。士各有志。古云。主辱臣死。況主已死了。爲臣子者與闖賊誓不俱生。恨我書生力綿。不能殺賊。故欲一死以報君恩耳。尚忍臣賊乎。梅生道。故君雖崩。自有嗣君繼立。尚還仰望殲賊復仇。以雪斯恨。今日徒死奚益。況我們這南京地方。還是明朝地土。並不曾爲賊所有。何得就是賊之臣子。何必預先就死。若此地果爲賊所有。弟雖未仕。亦叨一第。亦當蹈東海而死矣。肯臣賊乎。今日尚早。死非其時。不必着急。鍾生聽他這話。尋思亦似有理。答道。兄言亦是。弟姑俟之\footnote{鍾生後之不死者。非一旦變節。他今之欲死。特不肯臣賊耳。後闖賊已死。又何必死。所謂可以死可以無死是也。}。次日。宦蕚賈文物童自大鄔合鮑信鮑復之聞得了。都來勸解一番。鍾生自此以後。總足不履戶。惟尤(兀)坐小齋。終日書空咄咄。雖於妻妾之前。從不曾見他有一點喜笑之容。如此者將及一載。一日。宦蕚到鍾生家中來。坐下說道。老父聞得長兄自從知先帝升遐之信。與賊誓不俱生。終日贊嘆。方纔得一邸報。知李自成已被天誅。特命弟送來與長兄一看。稍快心胸。鍾生接過看了。以手加額。道。先帝有靈。先帝有靈。復恨道。恨不能以此賊剖心瀝血。肆諸市朝。以祭先帝在天之靈爲快耳。你道這瞎賊是怎麼死的。他自得了北京。親自領兵去攻山海關。到了石河。被大淸兵馬殺得大敗虧輸。亡魂喪膽。跑回北京。也不想做皇帝了。收拾了些子女玉帛。領着賊衆。星夜直奔襄陽。他此時賊兵尚有十數萬。分爲四十八部。在武昌住了五十日。改江夏縣爲瑞符縣。設立爲官。歛各處銅炭。拘匠役鑄永昌錢。李自成一日聚衆將道。湖廣四戰之地。不可久居。須先奪南京。以爲根本。爾等心下何如。衆人公議了一會。主意皆同。遂謀奪船。先取宣歙二處。他復道。西北旣不能定。東南豈可再失。今當星夜速往。擇期將行之日。險(陰)霾四塞。暴風烈雨。旗槍皆折。他於四月二十二日改路。由金牛保安走延寧蒲圻。沿路恣意殺擄。到通城。命四十八部先發。通城有一座九官山。又名羅公山。山上有一所北極元天上帝的廟宇。那日。山下左近百姓聞得流賊到來。聚衆賽會。大家設誓共保鄕里。李自成帶了二十名騎兵。他要到山上去看看。到了山下。命衆人不許跟隨。他單騎登山入廟。見帝像拜謁。若被神擊。伏不能起。衆村人疑是劫盜。取鍬鋤棍棒一齊下手。打得頭顱粉碎。骨肉如泥。見他腰下有一顆金印。內有非常衣服。大驚大駭。皆從山後逃出。那二十騎見他久不下來。上去看時。只見一堆白骨。看看又是一所空廟。驚疑爲神所殺。也就各逃散了。瞎賊凶惡一生。這就是他的結果了。有一首打油道。

\begin{quotation}

百歲人生草上霜。無端妄覬作君王。

龍袍暫褂雖尊貴。山廟生捐亦慘傷。

四水逆流河湧漲。魂靈悲切日無光。

早知黃屋誠非據。何似林泉樂趣長。

\end{quotation}

此時弘光已即位在南京。以鳳陽總督馬士英先陞禮部尚書。即命入閣辦事。馬士英又特薦阮大鋮。奉旨起阮大鋮爲江防兵部尚書。衆人皆仰望太平。不想他君臣如醉生夢死一般。不知所作所爲是些甚事。只有一個史兵部一個樂府尹兩位好官。那史兵部雖也入了閣。又督兵往揚州駐鎭去了。樂府尹雖也陛了吏部尚書。只是一齊人傳之。衆楚人咻之。他也沒法。那一位弘光皇帝自從登極。一絲朝政皆不理。悉委大學士馬士英批發。他在宮中做他的正務。終日服春藥。養大龜。弄得那厥物粗長。如一條驢腎相似。漁獵少童幼女。間或一夜高興。或兩三個弄死了。拉出宮來。後來見婦女都禁受不得。還日逐服春藥。陽物脹得難受。放在草驢牝中。同驢交媾\footnote{千古以來之帝王。以驢爲〈也〉媾者。只他獨異。}。猶令乞兒們都派交大蝦蟆。取蟾酥配春藥。上揷黃旗。大書上用二字。你道可笑不可笑。更有妙處。除夕之夜。弘光臨御興寧宮。百官進朝辭歲。見他兩眉如鎖。低首沈吟。像有萬千心事不能解釋的一般。都以爲他是憂慮國家的大事。這些模樣宰相。伴食中書。家而忘國。私而忘公的臣宰。倒都有些內愧。朝廷有如此隱憂。我輩食祿者。反毫不以國事爲意。也太覺無良。只得上前伏俯奏道。闖逆萬惡。致先帝崩遐。危我社稷。此皆臣子不共戴天之仇。終當盡力撲滅。以雪天人之恨。今日除夕。陛下且稍寬聖心。弘光也不答應。只聽得他口中嘆息道。這怎麼處。如何過得。有司禮監太監韓贊周上前奏道。雖然國事日蹙。如今天下尚有東南半壁。事猶可爲。明歲勅文武諸臣。各盡心力。以抒國患。皇爺且請開懷。弘光聽得有些厭煩了。忽大聲道。你衆臣不能上體朕心。惟以這些瑣事爲言。我所憂者大。又是目前緊急要務。說了。連聲嗟嘆。衆臣不知他慮的甚麼大事。聽見責備。俱免冠謝罪。道。臣等鄙猥小見。蠡測營(管)窺。不識聖意若何。萬乞示知。弘光長嘆了數聲。道。目下新年。戲班中竟沒有一個好女旦。以供娛樂。所以朕心如焚。寢食不安。那裡爲那些國家的小事。衆臣見他降下這樣的綸音。竟無言可對。韓贊周俯伏泣道。以爲今日令節。皇爺或上念先帝。或追思皇考。豈意作此想耶。弘光滿面怒容。方要發作。只見首輔馬士英出班奏道。臣一介庸材。蒙皇上天恩。待罪首揆。諒此等事。臣不能盡力以開聖懷。何顏居百僚之上。臣星夜訪求。必得一色藝雙絕者。以娛聖意。弘光轉怒爲喜。道。足見先生忠君之心。燮理之才。但朕望此不啻飢渴。當速求之。朕自當有以報卿。話猶未了。左班中又有一個大臣。紅袍玉帶。象簡烏紗。履聲橐橐。上前跪奏道。臣在先帝時。被東林諸賊臣誣陷。放棄者十有餘載。今荷蒙聖眷。得掌本兵。夙興夜寐。正無可上報天恩之萬一。況旣逢堯舜之主。安敢不效臯夔稷卨以輔之。臣今當佐輔臣。選擇精同(通)音律美女上獻。稍盡臣報主之忠忱一二。弘光天顏大悅。說道。朕素知卿才兼文武。歌詞一道。甲於元人之上。若得卿盡心爲朕謀得佳人。富貴當與君共之。衆臣看時。原來是阮大鋮。無不匿笑。又只見班部中兩個官兒出來跪下。一個是馬士英的表弟。名喚馮寅。現任都察院僉都御史。一個是阮大鋮的妻弟毛羽健。現任工部左侍郞。馮寅奏道。輔臣與阮尚書雖各具忠臣愛主之心。恐一時難獲其人。新年何以開悅宸衷。臣家有一女子。雖不敢謂色藝雙絕。尚還可以娛目。但不敢上獻耳。弘光喜甚。道。卿有此美意。朕心嘉悅無比。何不敢之有。馮寅道。乞赦臣死罪。方敢上奏。弘光〔道。〕卿如此盡忠於朕。當以百世宥之。尚有何罪。只管奏上。馮寅道。古云。主憂臣辱。臣今見聖容不懌。不得已而爲此。臣妻解氏。小字語花。頗知演劇。雖無十分嬌麗。也還可以寓目。意欲上獻。不識聖意容納否。弘光滿臉是笑。說道。卿愛君如此。肯捐妻以奉朕。朕不但不肯輕待於解氏。必不肯薄報於卿。若果如朕意。當以貴妃封彼。卿亦不須愁不富貴矣。馮寅奏道。聖恩若此。誠臣夫妻之厚幸也。弘光又問毛羽健道。卿又有何妙論。毛羽健頓首道。馮寅旣能獻妻。臣又何難進女。臣有一女。年方十六。頗覺聰慧。雖不能歌舞。乞陞(陛)下留在宮中。使之慢慢學習。或可以供御樂之用。弘光喜笑道。這更妙了。卿愛朕一至此乎。朕亦不惜茅去(土)之封以報爾也。今晚二卿便可送進宮來爲妙。二人謝恩而退。這些文武衆臣。有那無恥的。深恨家中無美妻嬌女可獻。有幾員略有骨氣的。無不暗笑。無不長嘆。冠其冠而走其走。退朝而散。次早元旦。不知甚麼人題了一副對聯在午門外。道。

\begin{quotation}

福人沈醉未醒。全憑馬上胡謅。

幕府凱歌已休。惟聽阮中撥出。

\end{quotation}

所謂福人者。弘光乃福王世子而踐位者也。沈醉未醒。謂他如昏昏醉夢。愚頑毫無所知。全憑馬士英胡謅打混而已。幕府凱歌已休者。阮大鋮爲江防兵部。西北數省盡失。猶終日報捷。愚南京人之耳目。惟聽阮中撥出。阮者。月琴也。暗指阮大鋮之姓。謂何嘗有凱歌。乃阮之撥出耳。我且把弘〔光〕的來歷表白一番。纔見這聖子神孫的妙處。他父親是神宗萬曆皇帝之愛子。名常洵。乃鄭貴妃寵冠後宮。萬曆將他之子冊爲福王。那羣臣擇一富處之地封他。衆臣擬了河南洛陽爲他封建之處。福王就國之日。海內全盛。神宗遣稅使鑛使數十人。月有奉。日有進。廣南明珠。滇點(黔)丹砂。宜靑寶石。豫章磁器。陝西異織文毳。蜀中重錦。並齊楚鑛金鑛銀。搜括嬴羨億萬計。雖名人主私財。都入貴妃掌握。以十分之九與了福王。福王之富厚甲於天下。及流賊逼城。援兵到洛。毫無費資。衆人口語藉藉。詈於道中。道。王府金銀千萬。府中之人粱肉都厭了不吃。以飼犬豕。却叫我們枵腹殺賊。下次有警。我們也不來了。此時文武衆官苦苦勸王給賞。王堅執不聽。後賊復攻城。叛兵內應。及城破之日。賊入王府。珠玉金寶山積。悉爲賊有。王之血肉做了賊之福祿宴。世子逃在外邊。南都聞崇禎殉了社稷。因他是萬曆親孫。是崇禎從兄。故擁立了他。有那樣個昏老子。就生了這個昏兒子。總之。國運與人的家運一般。該興旺。自然生出好兒孫來成家立業。若該敗壞。自然就有不肖子弟輕輕送去。敗國亡家。總是一理。再說馮寅毛羽健二人到家。隨即將妻女送了進宮。原來這解語花是馮寅用千金買來的一個妹女戲子。以他做妾。嫡妻故了。即命他掌管家務。此時假說是他的妻子。獻與弘光。希圖厚賞。弘光一見。果然生得妖嬈。就叫他唱了兩齣。着實風騷可愛。淫心輒起。攜手登榻。交媾起來。弘光把個陽具養得如驢腎一般。他在宮中行樂。無非都是些幼女。即有少年婦人。如何當得。每每不能暢意。這次遇着這個香算\footnote{謂千人日(肏)個個弄也。}。正是勁敵。喜樂無比。次日元旦。即加封貴妃。是日朝畢。合宮妃嬪稱觴上壽。弘光多了幾杯。去行幸毛羽健的女兒。乘着酒興。兩下盡根。只見那女子哎呀了一聲。早已上(尚)饗。弘光疾忙抽出陽具。叫宮女救時。血如泉湧。已是無及。這毛氏是個十六歲的處子。乍經弘光這驢腎大小的厥物。若逡巡畏避。啼哭難禁。弘光就醉到十分。自然還有些憐惜。決不至冒冒失失。忍心弄死了他。內中有個緣故。毛羽健在陝西時。溫氏星夜到他任上。遣去了美妾。發作了幾場。毛羽健再不敢萌一毫妄念。復命之後。十年間。他歷陞到通政司左通〔使。〕一日。溫氏偶染時疫。他夫妻分床而臥。溫氏昏昏沈沈。七八個丫頭日夜服事。都辛苦了。那一夜因溫氏昏睡。衆人也就倫(偷)空去睡。丫頭中有一個叫做夭桃。是在陝西買的。生得頗有幾分可愛。兩隻小脚還不足三寸。毛羽健常垂涎他。因懼溫氏之威。不敢放肆。今得了這個空兒。見衆丫頭都睡着了。悄悄摸到夭桃鋪上。見他睡得如小死。就替他脫光。摸那妙處時。肥滑可愛。用指頭探探。原來不是原來頭(的)了。毛羽健爬上身。送了進去。乾暖有趣。抽拽良久。夭桃方纔醒來。料無他人。知是主公。將錯就錯。也就聳身上迎。毛羽健歡喜非常。竭力弄了一番方歇。也有數次。同伴丫頭有知覺。眼中冒火。心中發醋。過了幾日。溫氏病癒。丫頭們就悄悄的將此事稟知。溫氏大怒。將夭桃上下剝得精光。渾身打有數百。用鞋底將陰戶打得腫有饅頭大\footnote{更妙。}。稀爛烏靑。方纔饒了。把毛羽健的鬍子幾乎撏去。幸得他女兒救護父親。不致狼狽。夭桃遭了這一番荼毒。恨入心髓。暗想道。我當日在舊主家時。蒙主公\endnotemark[15]時常寵幸。主母只做不知\footnote{此等主母豈可常得。}。今日是主公偷我。又非我去撩漢子。就下這樣毒手。你怎麼帶夥着溫世幸來。就不放點鬆給我們。想了一會。道。這妒婦我是沒奈何他的了。我把他女兒撩動春心。弄成個破罐子。等嫁人家時。送了回來。羞辱這惡婦一場\footnote{這倒不妨。他的令姑並不曾送回來。}。也可雪我的愁(仇)恨。原來毛羽健的這女兒已十四歲了。生得一貌如花。不但全不像乃堂之陋。且比乃尊還美麗許多。而且生性聰明。百伶百俐。溫氏上邊家人沒一個敢上來。只溫世幸是乳母之子。又是大管家。溫氏時常叫他上來說話走動。毛羽健可敢攔阻。間或有空。二人便做作一番。一日。溫世幸買了一本春宮圖兒。放在袖中。要送與溫氏賞鑒學樣。不想一時失落。找尋不見。又不敢問人。以爲不知掉在何處。也就罷了。不想掉在堂屋門檻底下。恰被這女兒拾得。他翻開一看。見都是男女如此如此。忙放在袖中。到床上放下帳子推睡。逐張逐張細看。雖見男子的那東西放入婦人此道之內。十分動心。却不知何故。要問人又不好開口。到夜間。用個指頭塞入小牝中試試。有疼無樂。這女子素常極喜夭桃。那日見他被母親打得幾死。悄悄走出看他。私問道。你爲了甚麼事。奶奶這樣打你。夭桃正想要引誘他。便悄應道。奶奶前日害病。老爺同我偷弄了幾回。不知那個賤嘴的淫婦吿訴了他。今日纔把我這樣打。打身上還罷了。把我的下身幾乎打爛了呢。那女子道。爲甚麼把你下身打的這樣利害。夭桃道。說是老爺弄我的這個來。故此纔下死打他。怪是也怪不得奶奶。婦人家把男人這件東西像性命一樣。那裡還捨得讓人。那女子正想要問這內中的妙處。就借他這話。笑問道。這是有甚麼好處。你就說的這樣。夭桃道。姑娘。你後來嫁了人家。嘗着了。纔知道呢。弄憒了。渾身鬆爽。心窩裡那個快活的法。那裡說得出。那女子道。弄的可疼麼。夭桃道。就是頭一回有些疼。下次就不相干了。你不信。先拿一個指頭摳摳看。頭一回有些疼。忍着些。到第二回就好了。摳熟了。用兩個指頭。後來又用三個。你只多用些唾沫潤滑了。一點也沒事。等你挖開了。我尋個好東西送你受用。那女子見說到這裡。袖中摸出那春宮本兒與他看。道。你看男人的可是這個樣兒麼。夭桃看了。道。畫的一絲不錯。你是那裡得來的。那女子道。是我前日在門檻底下拾的。不知是誰掉下來的。夭桃道。我猜得着。但是老爺出門。奶奶就叫溫世幸上來。兩個人關着房門幹事。這定是他掉的。女子道。怪道我說老爺不在家。溫世幸便進來。關着門。我當說甚麼要緊話。原來同我奶奶幹這事。夭桃道。等我好些。弄個好東西送你試試看。定有妙處。那女子滿心歡喜。瞞着母親。叫自己的丫頭日日送湯水給夭桃吃。他夜間果然將個食指潤濕了。忍着疼。將小牝摳挖。一連三四夜。內中竟容下三指尚有餘。雖無大樂。也覺有些意味。他原是十個尖尖嫩指。因指甲戳得疼。剪得光光禿禿。衆人也不知他何故。他一心只望夭桃好了起來。尋假物送他。過了半月有餘。夭桃起得來了。他尋了一根紫竹斷傘把。用刀將竹根刻下有四寸餘長一節來。就那竹根頭做了個龜頭樣子。用磁瓦刮光。宛似一根陽物。他起來到姑娘房內。先謝了照看。見左右沒人。將此物送上。那女子一見。如獲至寶。笑吟吟接過。請敎他的用法。夭桃附耳傳授密訣。叫他仰臥。將腿揸開。多用唾沫。不住抽扯。自有樂處。這女子是伶俐極了的。自然會意。夜間如法作用。果然甚妙。每夜如此。被他將一個嫩而小的牝戶。弄成了鬆而大的個窟窿。到了崇禎十七年。毛羽健見賊情緊急。正月內即將家眷打發回南。他獨自在京。闖賊破城之日。他也隨衆投降。周鍾勸進表內。他與項水心也都附有名字。李自成被大淸兵馬殺敗逃去。他也逃回南京。阮大鋮啓奏弘光。說他不忘故主。冒死逃回。弘光大喜。加授今職。他圖永固富貴。故將女兒進上。弘光乍幸這女子時。以爲他是處子。也有些憐愛。先還款款而入。這女子用那竹郞君弄了二年\footnote{阮寶兒之蠟夫。同此女之竹郞君。俱是異想。可謂難姊難妹。}。陰門已寬闊久了。今忽經弘光這大物送進。覺比竹夫更妙。竟不隄防。弘光見他並無痛楚之態。以爲是大量的女子。可與解氏爭衡。興致大發。憑身用力。兩下盡根。這女子不過陰門寬濶而已。內中開闢的不甚深。況他的身軀小巧。弘光的此物是放樣無當的。妖童艷女不知死了多少在此物上。何況他未曾經風雨的一個柔怯女子。被他先一下。已受不得。只叫了哎呀一聲。急要迴避。已是無及。被他那連珠箭的弄法。第二下直攮到心窩。登時吿斃。此女之死。罪雖起於夭桃。實由於毛羽健獻女求榮。溫氏淫妒之報也。弘光見死了。也着實深悔冒失。以(次)日傳旨。以妃禮殯葬。又勅毛羽健着陞禮部尚書。馮寅着陞都察院左都御史。開印後到任辦事。開印之日。二人到了任。那時禮尚左都現有其人。所以當時有

\begin{quotation}

總憲衙門兩都御。禮部堂上二尚書。

\end{quotation}

之謠。都下傳爲笑談。末世君臣所作所爲。一至於此。無不可笑。毛羽健知女兒被弄死了。心中也甚慘。得陞顯職。也還歡喜。熱鬧了一番。自不必說。一日。正在家閒坐。家人進來稟說。京中劉老爺的奶奶到了。毛羽健同溫氏忙接了出去。只見劉奶奶帶着一個三歲的兒子。一個老僕跟着。蓬頭垢面。衣裳襤褸。不堪之甚。那劉奶奶見了他夫妻二人。痛哭了一場。讓到上房坐下。你道這劉奶奶是誰。就是劉懋的繼配了。劉懋元配已故。這是他在北京繼娶的夫人。姓鈕。纔得二十六七歲。劉懋涖仕數年。點了一次鹽差。又收了兩次稅務。家有十數萬之富。闖賊亂極時。崇禎向百官借餉。知他家富。坐名借一萬。他只獻一千。崇禎不准。他出了三千。苦苦說家私已罄。闖賊陷城後。比較衆官銀兩。他也在內中。幾銅夾棍獻出十萬。賊猶不足。闖賊知向年裁驛是他附和而成。恨入骨髓。直夾死而後已。家私抄擄。將他妻子賞了一個小賊頭。那時李自成本要殺毛羽健。因他迎降勸進。十分諂媚。〈耍〉要買人心。只得容忍過了。到闖賊去後。毛羽健知表兄已死。表嫂已配了賊。也不暇尋問。獨自逃歸。今見他們尋來。又悲又喜。坐下問問別後事情。鈕氏哭訴道。自你表兄被賊夾死之後。屍骸也不知撇在那裡。家私抄沒。家人都逃散了。我帶着這孩子分在一個賊家。那賊幾次要殺這孩子。我苦苦護庇。喜得那賊他心愛我。肯聽我勸。纔替你哥哥留下這個根兒。後來賊兵敗了。星夜逃去。我母子故得留下。賊退後。這個老家人來尋着我母子。幸喜那賊家中還有些銀子。故此纔得同逃了回來。如今夫忘(亡)家破。我寡婦孤兒全仗叔叔嬸嬸照看。毛羽健叫家人將大廳傍邊三間房收拾了。與鈕氏母子住。擺酒接風。做衣裳。縫被褥。俱不必細說。過了幾日。這鈕氏飽食暖衣。把前日來的風霜之色盡皆退了。嫩森森一個白臉。妖嬈嬈一個身材。蹺尖尖一雙小脚。嬌滴滴\endnotemark[16]一口京話。甚是可愛。比溫氏強了百倍。毛羽健就動了偷竊之念。想道。當日我弟兄兩個屁股弄過無數。何況表嫂的此道。且又是經賊弄過的。我也何妨一弄。遂每日到他房中來挨光。說趣話。調風情。那鈕氏配了一番流賊。也不是甚貞節的了。況在少年。也無可無不可的時候。且依附在他家。也想同他摸皮貼肉。方可久遠。但素知溫氏吃醋。始尚在未決。還在兩可之間。毛羽健一日說話間笑問道。嫂子前日嫁了流賊。那樣人可也還知道些溫存情愛麼。鈕氏紅了臉。含愧笑道。那也是沒奈何。有甚麼情愛。毛羽健道。便宜了這賊。享用嫂子這樣美人。我還沒有他的福氣。鈕氏斜溜了一眼。低頭微笑。百媚俱生。毛羽健不覺魂飛。也顧不得他肯與不肯。走上前抱住。就親了一個嘴\footnote{不愧是禮部尚書。這等的大臣。方輔佐得那等天子。}。那鈕氏也不說話。只笑笑將臉扭轉。毛羽健將他抱到床上。就扯褲子。鈕氏道。你快放手。我素常知道嬸子利害。恐他知道。不是兒戲的。你不要闖禍。毛羽健笑道。你放心。我如今不怕他了。兩人褪了褲子。弄將起來。你道毛羽健一時如何就這樣膽大。內中有個原委。溫氏自京中回來之時。他父母雖沒了。他兄嫂在烏程聞得妹子到家。差了一個家人溫世寵來問候。這溫世寵便是溫世幸的親弟。生得標致異常。宛似一個美女。溫氏一見了他。心魂飛越。毛羽健此時尚在京未回。溫氏就強逼着同他私通了。誰知這樣個嬌媚少年。竟有個絕大的陽具。溫氏旣愛其外貌。又喜其內材。便不肯放他回去。寫字差人送與寄(哥)嫂。要了溫世寵留下。他將臥房隔壁一間耳房裱糊潔淨。床帳桌椅收拾得甚是華麗。就在臥房床後開了個便門出入。做間秘室。以爲行幸世寵世幸之所。後來毛羽健來家。以爲是溫氏收拾了爲休息之所。也不敢常到那屋裡去。一日朝罷回來。走到上房堂屋。恰好夭桃在那裡。見左右沒人。儘着用手向房內指。毛羽健也不懂其故。誰知溫氏同他弟兄兩個正在秘室行樂。聽見毛羽健回來了。一時做手脚不迭。溫氏只得走了出來。毛羽健見他鬢鬆面赤。心中大疑。想起夭桃指的必有緣故。遂走到那秘室中一看。只見溫世幸溫世寵面色如土。壁在牆上站立。毛羽健冷笑了兩聲。走了出去\footnote{好大肚皮。}。溫氏雖然淫悍。到此也羞赧無地。毛羽健此後也不睬他。他也不敢再發一語。溫氏這幾日冷眼見毛羽健時常在鈕氏屋裡。他留了一片心。不住叫丫頭竊聽。這日聽得他二人在房中成了好事。溫氏暗喜。忙忙走來。他不曾關門。直走到床前。他二人方纔看見。鈕氏羞得把眼緊閉。溫氏笑對毛羽健道。此後你也管不得我。我也不管你的閒事。大家混着些罷。遂抽身回去。鈕氏問毛羽健溫氏這話的緣故。毛羽健把溫氏的事吿訴了他。鈕氏纔放了心。毛羽健索性不上去。將夭桃叫了來。三人一床滾。溫氏也公然叫了他弟兄二人上去。也是三個一床。他這兩分人家。與禽獸何異。試看如此之君。若是之臣。焉有不敗壞天下者乎。還有那一位賢宰輔馬士英。惟以喜怒用人。賣官鬻爵爲事。醜名四著。鍾生在家。聞知這些言動。時時撫脾長嘆道。天下事囗矣。馬士英獨掌朝權。開納助工列(例)。武英殿中書納銀九百兩。文華殿中書納銀一千五百兩。內閣中書二千兩。待詔三千兩。拔貢一千兩。推知銜一千兩。監紀職方萬千不等。時人爲之語曰。

\begin{quotation}

中書隨地有。都督滿街走。

監紀多似羊。職方賤如狗。

蔭起千年塵。拔貢一呈首。

掃盡江南錢。塡塞馬家口。

\end{quotation}

馬士英也耳有所聞。他一來見自己做得太醜。想圖掩飾。二來也要公用一兩個人。要買人心。他知鍾生年未四十。是個好官。又素有名望。要以翰林院學士起用他。以崇人望。鍾生是生平耿介淸高的人。一則不肯因人而熱。二則見不成局面。決不肯應命。只推有病。苦苦辭了。因長嘆了幾聲。念陳希夷先生的兩句。道。

\begin{quotation}

九重丹詔。休敦(敎)彩鳳啣來。

一片野心。已被白雲留住。

\end{quotation}

後人見鍾生不肯做官。想那時擇闖諸臣是何心哉。有一詞道。

\begin{quotation}

盛世昇平。主聖臣賢樂事頻。祖父皆封贈。妻子蒙恩蔭。哎一旦亂離臨。少忠多佞。背主求榮。反面操戈刃。你看那歷代奸邪豈乏人。

\begin{flushright}右調駐雲飛\end{flushright}

\end{quotation}

忽一日。賈文物來會鍾生。道。阮大司馬向保先岳故交。當日在熹宗時。弟也蒙過他提攜之力。他今要用弟在他幕下爲鳳陽兵備。弟見兄苦苦推辭。官爵不受。必有所爲。弟持疑不決。特來請敎。鍾生道。旣承垂問。況我輩又是多年至契。俱在親誼。敢不傾心吐膽。以至誠相吿。兄看今日之規模。還成一個世界麼。雖在仕途。亦當拔足。避之猶恐不及之時。豈有反往火坑中跳入之理。當日先帝聖恩欽賜的堂堂正郞。尚且不宜受。今日反受幕府私情之一兵備乎。弟鄙見若此。兄或另有高裁。弟亦不敢苦勸。賈文物道。承兄喚醒愚迷。弟佩愛多矣。遂絕意仕進。你道阮大鋮他是魏當(璫)門下漏網的一個餘孽。今日忽然一旦做了大司馬。看他替朝廷幹些甚麼事。並他的結局如何。要知始末。接看後文。

姑妄言第二十三卷終



\endnotetext[1]{「一」原作「平」,據書前目錄改。}

\endnotetext[2]{「紛紛」原作「粉粉」,據文義改。}

\endnotetext[3]{「奬遜」原作「漿遊」,據文義改。}

\endnotetext[4]{「心有」原作「有心」,據文義改。}

\endnotetext[5]{「的話」原作「話說」,據文義改。}

\endnotetext[6]{「要去」原作「去要」,據文義改。}

\endnotetext[7]{「因此」原作「固然」,據文義改。}

\endnotetext[8]{「他送」原作「送他」,據文義改。}

\endnotetext[9]{「兒」字原無,據上文加;下文或同,不贅。}

\endnotetext[10]{「跌」原作「失」,據文義改;下文或同,不贅。}

\endnotetext[11]{「甚麼」原作「麼甚」,據文義改。}

\endnotetext[12]{「公祖」原作「祖公」,據上下文改。}

\endnotetext[13]{「傅厚」原作「厚傅」,據文義改。}

\endnotetext[14]{「創」原作「劉」,據第十六回及上文改;下文或同,不贅。}

\endnotetext[15]{「主公」原作「公主」,據文義改。}

\endnotetext[16]{「滴滴」原作「嫡嫡」,據文義改。}

\setcounter{footnote}{0}

\theendnotes

\part*{姑妄言第二十四卷}
\addcontentsline{toc}{part}{姑妄言第二十四卷}
\markboth{姑妄言第二十四卷}{姑妄言第二十四卷}

鈍翁曰。要寫慕義等辭官。先寫阮大鋮一番貪惡。不然慕義諸人皆一時之傑。豈不識時務。那時局勢尚可戀戀於功名耶。不辭去。則爲不知天時之流。欲\endnotemark[1]辭去。又不忍負崇禎之大恩。史樂二公之知遇。眞難下筆。算出阮大鋮一番索賄。衆人一齊辭退。不但不做負恩人。且不失爲知機之士。後應史公之命者。非寫衆人爲馮婦。所謂士爲知己者死。正是英雄心事耳。豈止衆人去得高。即三千義勇亦囗囗妙。不然。將來這些人何以結局。二來正寫強將之下無弱兵也。

寫鐵化嬴陽之得官。雖是寫竹思寬之詭計。陰氏之舊情。總是要顯出阮大鋮的貪惡來。此一回內極詆毛氏之淫濫者。借其妻以罵其夫耳。雖係曲筆。以阮大鋮立身行己受之。亦不爲屈。

艾鮑艾福弑父之人。而阮大鋮馬士英受其重賄以官之。阮馬二人雖不曾弑君。送去明朝之天下。較弑君之罪等耳。凶惡相遇。自然臭味相同。無足怪也。

竹思寬郝氏初遇。一部書淫事起頭。竹思寬火氏同死。一部書淫案總結。思與絲同音。謂以一絲總貫二十四回大書也。是一部大關鎖。

樂公憂國而卒。高傑爲賊所害。史公與城所(同)碎。國家將止。大家散場而已。令人酸鼻。

鍾生未去之先。旣去之後。連寫許義士輩許多忠義之人者。謂將此等國家之幹。皆屈於草莽。而廟堂之上。專任阮馬宵小之流。焉得不四海分崩。天下盡喪。又見得人者昌。失人者止(亡)之意耳。

此書二十四回中。各色人無一不備。並未極力寫一孝子。雖寫鍾生之孝。亦不過能至乎哀。至於韓無儔之賣子葬親。蔡繹生之刻苦養父。亦不過淡淡寫去。並未寫事以禮。葬以禮。祭以禮之一人也。昔人有云。當今之日。或有忠臣。決無孝子。作者亦是此意。

此一部書中。殘寇惡人甚多。竟無一梁上君子。此何故。要知爲人臣而不忠者。國家之賊。爲人子而不孝者。家庭之賊。讀書而不循道理者。聖門之賊。不悌不信無禮無義者。倫常之賊。蒞仕而虐下者。地方之賊。自暴自棄者。世間之賊。此等賊。書中不可勝數。其穿窬之賊故不足道也。

此部書內。或詩。或詞。或賦。或贊。或四六句。或對偶句。或長短句。或疊字句。或用韻。或不用韻。雖是打油。然而較諸小說中。無一不備。眞可謂善於說鬼話者矣。正與姑妄言名相合。

\chapter*{姑妄言卷之二十四\\
第二十四回 小狗子敗子竟回頭 鍾麗生神龍不見尾\\
附 定國奸謀害勇將 鍾生神膽救仙狐}
\addcontentsline{toc}{chapter}{第二十四回 小狗子敗子竟回頭 鍾麗生神龍不見尾}
\markboth{第二十四回 小狗子敗子竟回頭 鍾麗生神龍不見尾}{第二十四回 小狗子敗子竟回頭 鍾麗生神龍不見尾}

話說這一位阮大司馬。他名大鋮。字圓海。原是魏忠賢門下頭一個心腹用事的走狗。殺害東林諸公。那一本點將錄呈與魏璫。按名殺害。全是他的主意。一生專與正人爲仇。不想他竟得漏網。躱了這些年。他與馬士英自來相厚。臭味同投。所以馬士英一入了閣。就薦他平素知兵。起他做了江防兵部尚書。大學士高弘圖請下九卿會議。馬士英道。若命會議。大鋮決不得用。況魏璫之遂非闖賊可比。給事中羅萬象上言。阮大鋮不知兵。恐燕子箋春燈謎乃彼枕上之陰符。袖中之黃石也。馬士英力違衆議。特疏舉薦。弘光惟以他言是聽。竟准用了。阮大鋮退居了這十數年。今日一旦做了顯官。越發凶鄙不堪。眞是。

\begin{quotation}

一朝權在手。便把令來行。

\end{quotation}

他無錢不受。無惡不作。無醜不備。都還是末事。更有可恨之處。令人髮指。南都擇日祭先帝烈宗之靈。黎明。百官皆衰絰齊。獨阮大鋮一人不到。衆人排班等候。直至已飯時。他纔八輿黃蓋。鳴鑼呼擁而至。衆人看時。他內穿大紅圓領。外罩白袍。進門大號道。先帝呀。因你不曾殺盡東林逆黨。致有今日。臣必殺盡諸人。以爲先帝雪恨。徐汧諸人今皆北走矣。馬士英忙趨過。以手摀他的嘴。道。徐九一現住蘇州。東林尚有多人。先生快不要如此。兩班衆人見他兩個這樣子。也有忿恨的。也有匿笑的。却不敢發語。你道可恨不可恨。他到了江北。慕義林忠尚智國守鮑信同衆千把。少不得都要來呈履歷參見。他見沒有送了禮來。心中大惱。稟過三四次。方許進見。參畢。他滿面怒容。道。你衆人虛報軍功。本部素知。當日何嘗有一個流賊到此。史閣部爲爾等矇蔽欺騙。欺主騙朝廷爵祿。這幾年也受用的夠了。俟本部查訪實確。把你們這些冒功受職的。少不得都要題參問罪。且各回去管事後再定奪。衆人雖滿腔忿忿。却不敢出言。出來聚在一處商議道。我們當日原爲各保身家。大家義舉。原不指望受賞加官。不意蒙史樂二公天恩。提拔我們至此。又蒙先帝天恩。我等一介小民。雖有殺賊微功。叨食皇家二品俸祿。本欲殺身報國。盡我一片忠忱。今看阮家這個賊坏(胚)。是想我們的銀子。我們一腔忠義。惟天可表。除了俸祿之外。別處毫無所取。如今休說無錢。就有錢。也不與這貪汚之徒。若不理他。久之必爲所害。此時若奮義殺了他。不過如捕腐鼠。上可爲朝廷除害。下可爲東林諸公出一口怨氣。但有識我們心事的。謂我們是一口忠義之氣。倘不知道的。說我們背反朝廷。豈不把生平的忠肝義膽都枉費了。爲今之際。我們戀此微名做甚麼。但我們受史老爺莫大之恩。今日一面寫稟帖送到揚州帥府內。一面申文吿病辭了這官職。他豈奈我何。衆人商議停當。鮑信道。諸位旣有同心。我又何戀此微名。如今樂老爺現掌吏部。我也辭了罷。遂一齊吿了病。此時各衙門正要尋事革官。出了缺。好賣銀子。要無辜革退。還恐人含怨。見來辭職。喜得了不得。可肯有不准之理。就都准了下來。他們大家都繳了箚。各自回去了。有四句打油道這阮大鋮的惡處。

\begin{quotation}

北都附逆忠良盡。脫網南逃故土來。

今爲朝廷驅猛士。奸邪貪惡甚於豺。

\end{quotation}

樂公先還不准鮑信辭職。後來見衆武官都辭准了。留他一個文職何用。也就准了。史公見了他衆人稟帖。大驚道。可惜失此沿江保障。差人探聽兵部准與不准。回報都准辭退了。史公跌足嘆息不已。欲上疏保留已無及了。差官去調他們來軍前效用。尚智知機。苦推有病。惟慕義林忠到他幕下。千把總\endnotemark[2]也有一半去的。一半情願退閒。國守先也還有意赴調。他與尚智最相契厚。再三勸他留下了。史公見衆人到來。心中大喜。皆以厚待。以原銜委用。後來揚州城破。史公自刎。慕義林忠也自殺殉難。國夫人正在巷戰。見丈夫自盡。他是婦人家。恐死於道路。屍骸暴露。忙將丈夫的屍首搶回寓處。縱火自焚。他夫妻的忠烈不愧爲英雄。有兩句道。

\begin{quotation}

義烈雙雙同自盡。夫妻千載姓名香。

\end{quotation}

那幾員千把總死的死了。去的去了。此是後話。不題。再說阮大鋮正要尋事害他們。見他們知機辭退。心中暗喜。出了這二十多個缺。正算計要賣一塊好銀子。暗叫一個心腹書辦名叫黃金聚。在外招攬主顧。誰想這些鄕勇見主將辭退了。也大家聚攏。說道。我們又不吃朝廷糧餉。各人自己替朝廷出力。原是大家的義舉。今日衆將主都無故辭了。我們爲甚麼叫別人來管轄。這個事做不成。就是流賊再來。憑他殺了也罷。我們大家也散了罷。只有盔甲器械原是官給的。我們一齊到江防兵部衙門交還了他。各人去安生理。大家約同了。一兩日傳遍了三縣。這三千人齊集了。到了衙門口。大喊道。小人們原是百姓。因怕流賊。故大家出力相保。今日太平了。情願歸農。將當日領的盔甲器械交還老爺。遂一齊堆在衙門前。一閧而散。中軍官忙傳稟了進去。阮大鋮知道了。又羞又氣。氣的是纔來未久。就激散這些義勇。失了沿江保障。氣不氣否。羞的是這些缺。也賣得好些銀子。這一散了。旣無兵可管。還設這官做甚事。豈不白丢了這股財。想要殺幾個出氣。又恐人多激出禍來。只得罷了。他着了急。但是有缺。只要有銀子就賣。雖娼優隸卒總也不管。銀一到就補授。咨送到馬士英跟前來考驗。馬士英因他是久交。況又是他舉薦一場。凡事不好違阻。每每曲從。後來竟連瞎子。瘸子。撆手。並七八十歲的老漢。都放了要緊武職。送來考驗。馬士英太覺不堪。也恐人談笑議論。遂回下一角文書。道。此後送來考驗人員。貴部當稍選略似人形者。方不遭物議。尚恐他來歪纏。出了一張大吿示。內云。

\begin{quotation}

凡來考驗武職。若有疲癃病廢殘疾不似人形者。除革退外。仍重處不貸。

\end{quotation}

這些買官的人見了吿示。恐費了銀子反要獲軍(罪)。不肯買了。纔阻住了他。他見了這些話。恬不知恥。但是馬士英不准也沒法。無奈何。只得又略略稍揀不瘸不瞽之人。眞是自古來亡國之人臣。再沒有個醜似他的。可笑似他的了。阮大鋮在外邊無惡不作。他夫人在家中無樂不爲。向年。阮大鋮差龐周利往京中去探聽逆黨的事體。回來路上遇見了馬氏。到家稟知了阮大鋮。過後有人傳入毛氏耳中。毛氏急於要問狗(苟)雄的信。因阮大鋮在家。不敢叫龐周利來問。一日。阮大鋮往祖堂寺去了。毛氏恐怕上房人多耳衆。就到嬌嬌那房裡去。原來毛氏將此房收拾潔淨。床帳俱有。時常到那裡閒坐。這日到來坐下。叫丫頭叫了龐周利來。問他道。前日我聽得人說你稟老爺。說你在路上看見馬六姨。可是眞麼。龐周利道。小的眞看見來。還同他說了半日的話。毛氏道。他跟着苟雄逃去。你旣看見他。可曾見苟雄。龐周利將苟雄被殺。馬氏爲娼的話。詳細說上。毛氏聽說苟雄死了。心中蹬住了一會。由不得掉下淚來。恐龐周利同丫頭們看見。連忙轉過臉去拭了。只長嘆了幾聲。道。這淫婦倒還在。可惜了個苟雄倒死了。這龐周利自幼生得淸秀。是阮大鋮的龍陽。他奸詐百出。有一段鬼聰明。哄得阮大鋮滴溜溜的轉。故此阮大鋮着實擡舉他。長大了。遂將他做了大管家。他自聽得馬氏說毛氏與苟雄有私。他也就懷着希望之心。非愛主母之色。乃貪主母之財。倘弄厚了。定有重賞。況他又熟知主人的陽物不甚修偉。他腰中的一副本錢可爲苟雄之副。以爲得主母一幸。定然是他的如意君。心雖如此想。却無進身之策。今日恰好毛氏叫他來問話。有此機緣。又見毛氏聽得苟雄死了。這番悲慘嘆息傷心的樣子。知他非悲苟雄之橫死。不過是念苟雄孽具。隨機應〈應〉變。無中生有。謅出一篇話來哄誘毛氏。便說道。馬六姨向小的啼啼哭哭。好生埋怨來。說奶奶坑了他。有好些話叫小的吿訴老爺。小的蒙奶奶這樣恩典。怎敢向老爺說。毛氏道。這淫婦他同苟雄逃走了。自作自受。怎麼埋怨我。又叫你對老爺說甚麼。龐周利道。這話只奶奶聽得。兩位姐姐在這裡。小的怎敢說。毛氏遂叫丫頭們都出去。等我叫再來。兩個丫頭去了。毛氏道。你說罷。龐周利道。奶奶不要怪小的。小的纔敢說。毛氏道。你是過馬家那淫婦的話。我怎麼怪你。龐周利道。馬六姨說他當日好好的在家。偶然一日要對奶奶說話。也是到嬌嬌這屋裡。奶奶正同苟雄做甚麼事。被他撞見了。奶奶同苟雄光着屁股。跪在地下。百般哀求。叫他不要對老爺說。恐他過後嘴不穩。苦苦求他也要同苟雄弄弄纔放心。他見奶奶是這樣小心。心裡軟了。纔同苟雄相好。後來恐怕老爺知道。沒奈何。纔同他逃走。可不是奶奶害他。叫小的細細的回稟老爺。奶奶請想。這個話可是說得的。毛氏聽了。臉脖子通紅。低了頭不做聲。龐周利道。奶奶只管請放心。這話小的爛在心裡。決不肯吿訴人。就是老爺知道些風聲。憑着怎麼盤問小的。小的可有個不衛護奶奶的。決不肯說。又挑一句道。苟雄這沒良心的人。不要被強盜殺了。就剮一萬刀也是該的。不想想我們一個做下人的。蒙主母這樣天恩。把千金身子都賞你受用。就死也値。怎麼就忍心撇了就走。要是小的蒙奶奶這樣恩典。拿刀壓着脖子。還攆我不去呢。毛氏想了一會。見龐周利的漢子也不亞似苟雄。且又少年。模樣還比他強了許多。且他的聲口有幾分訛意。若不給他個甜頭。恐張揚得阮大鋮知道。亦非兒戲。二者自苟雄去後。守了活寡。多時臍下那件作怪的東西不住發癢發燥。也有些忍不住了\footnote{看此偶憶起一個掛眞兒。與毛氏正合。俏寃家不住叮。又不是虼蚤咬。陰天又發癢。晴天又發燥。尋一個棒槌大的好東西。搗上他幾千搗。}。\endnotemark[3]遂道。我當日也是一時錯\footnote{好錯。只恐今日又要錯了。}。失身給這奴才。誰知他這樣沒良心。你剛纔嘴倒說得好。但你男人們的心腸走滾大。那裡拿得定的。龐周利聽毛氏口氣有幾分俯就之意。忙跪下道。小的若蒙奶奶施恩擡舉。敢有一毫負心。天打雷劈。遇強人斫一萬刀。比苟雄死的還利害。毛氏也就笑道。只要你心應口纔好。龐周利見這話明明是肯了。遂叩了個頭。道。日後奶奶纔知道小的的心呢。站起來。就將毛氏抱在榻上睡倒。掀開衣服。替他脫褲。毛氏道。我依了你。你要憐惜我些纔好呢。龐周利見他說得肉麻。不覺暗笑。忙自己也脫了。毛氏偷眼看他的陽物。比苟雄略次。心中私喜。龐周利將他的臀墊起。挺起陽具。直向毛竅中攮了進去。使起蠻力。如搗碓一般。足搗夠有兩頓飯時。還不敢歇。毛氏覺他的陽物堅硬。伶泛過於苟雄。十分歡喜。已丢了數次。說道。你歇了罷。恐丫頭們等得太久了疑心。改日老爺出門。我來這裡。叫人叫你去。龐周利道。奶奶略等一等。小的也快了。說着。他自首至尾狠搗了百餘下。搗得毛氏面赤口張。哼聲震耳的。他方纔洩了。毛氏將他摟住。把舌尖度入他口中咂了一回。龐周利穿了衣褲。喜孜孜出去了。毛氏還〖扌歪〗在椅子上。喘息了一會。纔穿褲起來。慢慢走回上房。心中不勝暗喜。此後但是阮大鋮出門。他二人便在嬌嬌房中行樂。一日。兩人在床上。龐周利抱着毛氏親嘴咂舌。又咂了咂癟乳。說道。小的不知前生怎樣修積。今生有福蒙奶奶這樣擡舉。毛氏裝嬌作媚。偎倚在他懷中。道。我有年紀了。怕你嫌我老。你要始終這樣好。我也不肯忘你。龐周利忙道。小的敢嫌奶奶老。就該萬死了。小的看奶奶的這件寶貝。比少年的還嫩呢。不要說別的。女人的這件東西。小的也見過些。從沒有這麼些好毛。奶奶是貴人的。到底比別人不同。毛氏笑道。這上頭毛多倒好麼。這是你反說。敢自是光的有趣。龐周利道。小的怎敢說謊。奶奶請想。譬如男人四五十歲。嘴上沒有鬍子。像一個甚麼樣子\footnote{會奉承。}。說着。縮下身去。含着花心。咂了一陣。又攄着那毛。贊了一番。然後伏上身。大弄一番。半晌方歇。常常得空便弄。到今阮大鋮常往江北去。毛氏同龐周利纔得任意行事。無三日不弄。龐周利也暗受了毛氏許多賞賜。都不過是阮大鋮刻薄來的餘貲。把毛氏的事且暫擱下。再說阮大鋮的醜不能盡言。姑舉一二以見其餘。他受了鐵化三千金之賄。喜得非常。特題了他長河衛掌印指揮。公然三品武臣。這也還罷了。連嬴陽戲子而兼龜的人。也就放了他浙江湖州府歸安縣守禦所千戶。豈不可笑。你道他是何故。這阮大鋮酷喜塡詞。魏璫正在勢燄之時。他或南來替他採訪害人的事體。或無事之時。在祖堂寺呈劍堂。日間同諸狎客敍飮談笑。夜間便作戲文。作了幾部傳奇。那時嬴陽正在他門下。他夫妻都唱得好。陰氏又風騷可喜。這大鋮除了氣字之外。酒色財三字是無一不愛的。同這陰\endnotemark[4]氏契厚。不過後來嬴陽回去了。每每想念。過了數年。嬴陽因記掛女兒。到南京來看看。此時魏璫已敗。阮大鋮正縮頭藏頭躱在家中。門庭冷落。嬴陽因感念他向年托鐵按院替他報了聶變豹的仇。不能相忘。親自到他家叩謝。又送了些蘇州人事。阮大鋮見他心不忘舊。冷竈添柴。倒也甚是感他。近來嬴陽聞知他陡然做了兵部尚書這樣大官。想來打他抽豐。備了百金一分賀禮。陰氏又悌(梯)己帶了些私房人事送他。嬴陽到了南京投見送上。他心中甚是歡喜。但他要人的銀子。心中尚嫌不足。可肯拿出己囊來贈人。多了捨不得。輕了過不去。無可答情。正値這個缺出。就補放了他。以酬他夫婦之情。那嬴陽來意。不過想他兩百銀子。夢想不到竟得個官做。公然峨冠博帶戴起來。他在戲場上久了。禮貌比別人更熟。來會了女兒女婿。又見外甥十多歲了。甚是淸秀。好生歡喜。他向年來雖已見過。那時鄔繼祖尚少。他只當是女兒親生的。並不知是拾來之物。今見女兒女婿家道更覺從容。也買了房。又有家人使用。外甥又大了。要帶他們同往任所。鄔合此時正替宦蕚管事。他手頭寬裕。又戀土難移。嬴氏難捨丈夫兒子。都不願去。嬴陽不能強他。回到崑山。他丈人丈母早已亡故。只同陰氏郡君四轎而黃蓋。這樣一對好老爺奶奶。竟赴任做官去了。阮大鋮所用之人。大都如是。這算他忠心爲主。薦賢報國了。豈非奇聞。那嬴陽夫婦自到任之後。一日。兩人偶然閒話。嬴陽嘆道。我家世代做戲。少年時遭了多少卑汚苟賤。今日竟得了個些小前程。無非天地鬼神之恩。我們無可報答。只做一個好人。存一點善念。以報上蒼萬一。我想好人也不知是何做起。我又不曾讀過書\footnote{這倒不然。我見讀書者不肯做好人的更多矣。}。不知這些道理。我聽見人說好話。開口就說萬惡淫爲首。況我二人在這個淫字上也領敎過多了。從此把淫心盡息是第一件。二者我現做着個頭目。待這些管下人衆。要着實的恩待他們。你想我們是何等出身。娼優隸卒。良人蹺起脚來。比我們的頭還高。衆人誰不知道我們來歷。自己却不可忘了本。陰氏屢年來淫心也甚淡。頗有良善之心。聽了這些話。大以爲有理。深贊甚是。嬴陽此後待人一味謙和。馭下甚有恩惠。管下的衛丁個個感激他。倒也有個好名。他夫妻年俱半百。嬴陽時常嘆道。我家世代單傳。今到了我。却斷了根了。陰氏道。我是不能生的了。你娶個小。或者還生得出。也不可知。嬴陽笑道。我這樣人雖弄了一頂紗帽在頭上。不過如在戲場上一般。爲人要自己知道出處。我們今日享朝廷一命之榮。已是過分。又想娶小生子。眞是妄想折福了。我有個笑話說給你聽。

\begin{quotation}

當日一個女人嫁了丈夫。總不生育。他一個親戚婦人同他閒話。問道。奶奶。你同你家爺是不的麼。怎再不生產。那女人答道。倒也不不呢。那婦人道。旣不不。你不生是甚緣故。他道。這個道理連我也不明白。若說不生。我在家做女兒時。也生過幾個。要說生。自從嫁到這裡。竟不生一個。

\end{quotation}

即如你若說不生。當初怎麼生皎皎來。雖說是金家的種。到底有我的氣兒。就是你心裡。也未必辨得出是那一個的骨頭。要說你生。這二十多年種也下過幾千次。我的精脈也去了幾盆。總不見個影兒。可見是命中該絕了。命旣如此。就尋個小來。也未必能有。我年力精壯。你還不受胎。今日衰了。越發沒用。何苦白擔誤人家女兒。陰氏笑了一會。道。事情不是這樣論。這叫做撞造化。必定有個可生的東西。你去撞了看。或者撞着了。竟生個兒子。亦未可料。我如今是不能生的了。你就撞塌了頭皮。也是沒用。〔嬴陽道。〕你要想男兒誕子的事。圖僥倖你自己生出個兒子來。婦人家到了五十歲還不能生育。何況於男子。諒越發不能夠。勸你不要癡想\footnote{他夫妻談笑這一段。甚趣。確是他二人的話。移於他人不切。}。夫妻大笑了一會。他衙署隔壁有一個秀才。叫做陳繼常。他妻子東氏。妒惡異常。他家有一個丫頭。叫做海榴\footnote{謂甚多生子意。}。也並非美婢。不過生的黑的是髮。白的是臉。身軀不粗不細。還是個人形。不至於魑魅惡態。東氏疑他丈夫同這丫頭有苟且的事。時常打罵。那陳秀才極其老實。循規蹈矩。那丫頭雖在面前伏侍。他連眼也不敢多看。看的禁不得東氏動了疑。見丫頭上前。說他浪漢。在主公面上討好。及至退後。又說他故意做出嬌態。引誘主公。無日不打。面上掐的瘢痕。身上打的血印。新舊重疊。再不能脫。陳繼常看得甚不過意。想要勸勸。又恐越發疑心起來。倒不是替丫頭求生。反是與他送死了。他夫妻二人同嬴陽兩口子都說得來。頗覺親密。後院僅隔一牆。有個便門可通往來。陳繼常把這事吿訴嬴陽。求陰氏內中解勸解勸。打發掉他。救他一命。只當大積陰隲。嬴陽也嘗向陰氏說過。陰氏近日想替丈夫尋小。每常見這個丫頭也還看得。就想到他身上。也不說破。一日。聽得隔壁東氏打這丫頭。打夠百餘。那丫頭殺豬似的叫。還不肯饒。口中不住大罵。又聽得不明白。陰氏帶了個僕婢。開了後門。就走了過去。東氏見了他。方纔住手。陰氏笑道。奶奶。爲甚事又在這裡生氣。東氏讓了坐下。臉都氣得雪白。戰篤篤的說道。奶奶。說不得天地間那裡有這樣大膽的事。大白日裡。這騷淫婦同那沒廉恥的私偷。剛剛被我撞見。你說氣得過麼。我定要打死這淫婦纔罷。看官。你道這事可是眞麼。原來這早東氏卯飮了幾杯。一時醋興發作。拿這丫頭來消酒。結結實實打了一頓。一時酒湧頭暈。到床上去睡。這丫頭受不得了。趁這空兒。到後面廚房去尋死。却好陳繼常走去看見。再三叫他不可輕生。〔道。〕我已求了隔壁嬴奶奶救你。你權且忍耐。那丫頭聽見有生路。自然就捨不得死了。陳繼常也恐東氏醒來。忙忙走出。恰巧東氏睡醒了。見丈夫匆忙自後出來。心中大疑。忙走到後邊去。看那丫頭還在那裡拭淚。一見了。大發雷霆。說他兩個偷情。定是向主公哭訴他的狠惡。定要打死。陰氏也不知內中眞假。便勸道。奶奶。殺生不如放生。一條人命何苦輕輕斷送。你不如賣放他。眼中何等乾淨。又算行了好事。可不好麼。東氏道。奶奶勸我。我可有不聽的。況我的心比菩薩還軟幾分。別的都待得過。就是這一件。我眼裡心裡都放不下。這一時那裡就有人〔買。〕若要留着他。我那有這些精神去防閒。眞是一刻也留不得的。陰氏笑道。我跟前一個大丫頭配了漢子。近來正沒個丫頭使喚。正要買人。奶奶不若賣給我罷。也不好講價錢。奶奶當日是多少銀子買的。我照原價送來罷。奶奶可肯麼。東氏道。這淫婦原是十二兩銀子買來的。旣是奶奶要。我就奉送也有限。但只是一件。他是引漢子的班頭。恐怕後來同你家嬴爺七個八個的。你不要抱怨我。奶奶。你看我這樣防得緊。他們還偷空弄鬼呢。陰氏笑道。我自然會管他。便叫僕婦回去取了十二兩銀子來。遞與東氏。東氏也將原文書查出給與他。說道。此是海榴丫頭文書。請奶奶收了。陰氏即將文書收了。亦即吿別。就將丫頭帶回。把前話吿訴了嬴陽。笑了一陣。過了幾日。這丫頭脫了棍棒之難。陰氏又着實恩待他。好茶飯給他吃。那臉上身上的痕跡都漸漸退去。陰氏見他好了。叫他洗了個澡。換了一身新衣服。叫他到跟前。向他說要與丈夫做小的話。那丫頭一脫地獄。連登天堂。已感恩不盡。忽然又聽見說要他做小。他雖是下人。十七八歲的丫頭。何嘗不想見見男子的異物。因主母利害。主公畏怯若此。何暇萌及此念。今有這樣美事臨頭。那種歡喜那裡還說得出來。頓時笑容滿面。跪下叩了無數的頭。陰氏叫他起來。請了嬴陽到跟前。笑道。你一番好心。叫我去救了這丫頭來。今日已將息好了。算你救了他的命。他給你做小。報你的恩。也不爲過。你今夜去同他共宿。又悄向他耳邊道。看陳奶奶的話。這丫頭也未必是處女了。只要他有本事養兒子。整破也不必論罷。那嬴陽還要假推幾句。心裡喜得只是笑。連話也說不出。陰氏便叫僕婦送上果酒。他夫妻並坐。就叫那丫頭坐在傍邊\footnote{閱至此。偶憶一奇事。有一相識姓傅。四十餘無子。其妻爲之娶妾。進門之日。三人共坐而飯。至晚。其妻親送二人進新房。次早衆人起時。其妻自縊死矣。此婦心腸豈非奇極。令人不解。若謂如何不〈不〉與之娶。旣爲之娶矣。何又自己吊殺。眞可笑。}。嬴陽細看那丫頭。雖比陰氏少年。而美貌十不及一。但小女子十五至二十五六。十年之中。只要略似人形者。定有幾分丯韻。今日陰氏老了。兩處看着。而竟有可愛之處。嘻笑得意。晚間陰氏叫他二人到西屋去睡。嬴陽乘着酒興。自己脫光了。替那丫頭脫衣褲。每常惡主母拿木棍摸打。還不敢不脫。何況此時善主公要用肉棍具抽。是件有樂無苦的事。可還敢做假。脫得好不快當。嬴陽摸他身上。也還光光滑滑。摸到了那個妙處。沒有這樣大方的處子。少不得佯羞做態。將腿夾緊。用手摀住。嬴陽興發如狂。將他扶正了。跨上身。以爲他是久經風雨的了。向胯中亂戳。戳着了門。努力一下。戳進去了一個頭子。那丫頭先也有些畏怯。見戳得外邊癢癢酥酥。以爲內中也是如此。不防戳了進去。大非前局。嬴陽的厥物又成文。疼得那丫頭把身子忙往後續縮。口中連聲哎呀哎呀不住。嬴陽見他竟是眞處子。更加歡喜憐惜。附在耳上笑道。我當你是破了的。誰知你還是女兒。旣然如此。你家奶奶爲甚麼只管打你。前日又吿訴我家奶奶。說你同陳相公私偷。那丫頭也將主子疑心。不能辯白。那日尋死。被相公看見解勸的〔話。〕細說一番。流淚道。我只說奶奶救了我來。免得終日捱打。就是造化了。那裡想到今日。嬴氏(陽)十分憐愛。款款輕輕做了一度。完事後將帕兒一揩。看了看。猩紅點點。他當初娶陰氏時並未見此。今日五旬的人。初嘗處子的滋味。你道他樂也不樂。喜得他摟緊了。親了好幾個嘴。相抱而睡。半夜又做了一次。那丫頭見不似前番苦辣。欣然承受。嬴陽大展鎗法。戰了一場。興盡而止。次早吿訴了陰氏。說他竟似(是)女身。把前後話細說了一遍。陰氏也好笑了一場。大凡人有一善之念。上蒼決不負人。嬴陽陰氏各存了一點的好心。戒淫行善。定然不致絕嗣。那知嬴陽同這丫頭一夜中風流了兩度。竟得了胎。眞果喜出望外。到分娩之時。竟破了他祖宗單傳之例。生了一個雙胎。得了兩個兒子。喜得他夫妻笑得嘴都合不攏。他此時大小到底是個地方官。賀喜者塡門。雇奶娘。擺酒席。那是不消說得。彌月之後。替丫頭上了頭。家中皆稱姨娘。那東氏知道了這事。心中大惱。怪陰氏爲何把丫頭與丈夫做小。這樣傷風敗俗的事都做了出來。遂同陰氏斷絕往來。這種妒婦吃別人家的醋。眞可笑之極。後來嬴陽這兩個兒子大了。一個叫做嬴紹之。一個叫做嬴續之。也不學戲了。抛去這祖傳衣缽。都敎他們念書。嬴陽做了兩年官。一日。向陰氏道。爲人不可不知足。古人兩句話說得好。

\begin{quotation}

知足知辱。知止不殆。

\end{quotation}

更還有兩句話更說得好。

\begin{quotation}

爲官一身淸。有子萬事足。

\end{quotation}

我僥倖做了這一任官。眞出於意想之外。還圖陞遷到那裡去。況又得了兒子。有了後代了。你我都將望六旬的人了。還不想退步。便是無厭之徒了。我於今辭了回去罷。陰氏也着實贊助。嬴陽便吿老還鄕。在他也就算榮歸故里了。到了家中。自然比當年更熱鬧些。金鑛閔氏更加親熱。後來他兩個〔兒子〕都大了。俱娶妻生子。合家歡樂。他夫妻偕老壽終。可見人能一心向善者。天必賜之以福。嬴〔陽〕陰氏何等之人。當日受閔氏之恩。便念念不忘。吿聶變豹雖是自己報怨。却救拔出閔氏。又全他嫁了金鑛。也算以德報德了。今在任上又存了一番善念。又生了二子。可見人存一番好心。自有一番好報。嬴陽之人猶如此。何況勝於嬴陽者。反不自省。自貽伊戚。豈不惜哉。話不多敍。再說那竹思寬自當年遇了火氏這一位佳人。模樣旣少而美。美而騷。牝戶又小而緊。緊而洩。較之郝氏。不啻有雲霓之隔。且他那一番相愛之情。又深又厚。厚而且濃。眞是一個生死寃家。魂靈兒已死久了在他身上。多年來。二人雖會過十數多次。都是提心吊膽。偷偷摸摸的。不但火氏不得大遂心懷。就是竹思寬也不曾十分的暢快。後來巧兒大了。火氏沒氐。自己要留他做馬泊六。躭誤他到老不嫁人之理。雖欲分惠於他。他那個原封未動的牝戶。可禁得這放樣異常之孽具。沒奈何。只得把他嫁了人去。不像嫁了個丫頭。竟像沒了丈夫。如剮了心頭肉一般。淚流了多日。自從沒了牽頭。有好幾年他二人不曾相會。火氏想另托一個丫鬟。但都是蠢物\footnote{古云。乘駿馬。使癡奴。可見僕婢都是蠢的纔好。}。不足與語的。倘機事不密。走洩了風聲。越發無望。只得待其時而已矣。但他兩地相思。如山高海濶之比。這火氏他旣去了一個知心貼意的丫鬟。又老死了那條解饞殺火之妙狗。眞是愁腸百結。度日如年。竹思寬雖是五十多歲的人。因他陽物放樣。少年不曾作喪。還精精壯壯。像個四旬多的面貌。那郝氏是花甲外的老嫗了。靑年間在色字上掏傷了的。髮白蓬鬆。形容衰朽。況且臍下那件閱歷多人的癟牝。當日被竹思寬揎得甚寬大無比。今日一老了塌下去。竟成了一個大坑。惟有許多縐皮而已。只可相伴。難以行樂爲歡。因此竹思寬時時刻刻把火氏放在心頭。閉上眼似乎他就在眼前。欲會無由。要想設個法兒騙了鐵化遠處去了。好與火氏時常相親。數年來總沒有一個良策。近聞得阮大鋮懸榜賣官。他黃家舅舅的孫子黃金聚。現在他衙門當書辦。替他走線索。因想做財主的人。心中再無不想做官。我如今拿功名二字。或者可以打動他。這日是端陽佳節。他也無心去遊戲。衷心忖道。我到老鐵家去。今日大節下。他必定在家。不但對他可以說話。還可以痛擾一回。戴了一頂馬尾瓦楞帽兒\footnote{一。}。穿了一件新葛布袍兒。濶桶漂白水襪兒\footnote{二。三。}。淺臉黃草鞋兒\footnote{四。}。拿着一把靑陽扇兒\footnote{五。}。拴着一個阿魏扇墜兒\footnote{六。}。一氣走到鐵家。門上並沒一人。原來這年秦淮河龍舟大盛。鐵化被〔邀〕去遊船。家人見主公高興。衆們大家也就行樂去了。竹思寬走到廳上。也沒有人。見書房院子門虛掩。推開走入。跨進書房。一眼看見了五百年風流孽寃。只見火氏靠着一張桌子。手托香腮。口中咬着小指指甲\footnote{活是一幅美人圖。}。面前放着一本如意君傳。看那上面的圖像。見薛敖曹蚓筋兔首的那件陽物。正觸着心事。想起竹思寬來。攻得火上雙腮。正情不能禁。猛聽得脚步響。一擡頭。見了這歡喜寃家。喜極而悲。竟掉下兩點淚來。你道火氏緣何在這裡。這日他知鐵化不在家。吃了幾杯雄黃酒。一時事上心來。無可消遣。也道是大節下。定無人來。故到書房中走走解悶。偶然見架子上有一部書。順手拿過一本。翻開一看。上面都是做這件風流事\footnote{這纔是鐵化架上的書。不然放何書。}。正看得入神。一見了竹思寬。因相思日久。不覺的下淚來。生怕丫頭看見。忙背過臉拭去\footnote{毛氏先滴淚。是悲死苟雄。火氏此時滴淚。是見活思寬。雖是兩樣心腸。却是相思二字。}。竹思寬上前作了個揖。道。我是來尋鐵大爺的。不知奶奶在此。多有得罪。說了。就在窗外站立。火氏故意問丫頭道。這位是誰。丫頭道。就是竹相公。火氏道。原來是你爺的好友\footnote{倒是奶奶的好友。}。大節下。你快燒一壺好茶來\footnote{火熱的天。涼茶正好。燒新鮮茶有好一會躭擱。此淫婦之急計也。}。那丫頭答應去了。竹思寬見他遣開了丫頭。忙去閂了院子門。跑來抱住。不暇開言。親了個嘴。抱到涼床上。就要扯褲子。火氏道。不好。恐一時到了高興的時候。丫頭拿茶來。怎麼處呢。你去關了角門來。竹思寬一面解着衣帶。一面跑去關門。回來時。火氏已經脫得精光。臥在床上。竹思寬連忙脫去衣褲。扒在他肚上。往裡就頂\footnote{兩人都急得有趣。却正是白日畏人來。}。那火氏先看書時。就有許多水出來。滑順至極。兩人都是情急了的。忘了扎根子。被竹思寬猛然一下。比每常多進去了寸餘。那火氏哎喲了一聲。腸肚生疼。眼淚都流出來。揉着肚子。道。哎喲。被你頂斷了腸子了。又是那好笑。有幾〔句〕說他二人。道。

\begin{quotation}

一別多年。相逢半霎。回想昔年滋味。難敎片刻從容。何暇款款爭鋒。急急匆忙對壘。花心雖綻。半入尚可承當。玉莖全投。腹內如何可受。只因久渴。心中愛至。柔腸中損傷。

\end{quotation}

竹思寬見傍邊放着一條縐紬手帕。忙拿過來纏上了。又恐躭誤了工夫。不敢稍停慢弄。用力蠻抽重扯。二人都是相憶久了的。不多時。就一齊大洩。連忙起來穿好衣服。竹思寬久別嬌容。仔細把他一看。雖然年過三旬。丯韻如同昔日。黑油油的頭髮。高高的吊着個桃兒\footnote{一。}。兩邊刷的光蓬蓬的鬢兒\footnote{二。}。挽着個蘇州纂兒\footnote{三。}。揷着兩根金簪兒\footnote{四。}。戴一枝香噴噴的茉莉花兒\footnote{五。}。白白的臉兒\footnote{六。}。紅紅的嘴兒\footnote{七。}。彎彎的眉兒\footnote{八。}。直直的鼻兒\footnote{九。}。水汪汪的眼兒\footnote{十。}。齊斬斬牙兒\footnote{十一。}。金丁香墜兒\footnote{十二。}。外面穿着金壇葛布衫兒\footnote{十三。}。裡面桃紅生紗襯衣兒\footnote{十四。}。下繫着玉色露地紗裙兒\footnote{十五。}。顯着紅通通紗褲兒\footnote{十六。}。一彎小脚兒\footnote{十七。}。嫩尖尖手兒\footnote{十八。}。誠然可愛\footnote{前敍竹思寬打扮只六個兒字。此處敍火氏是十八個兒字。一部書中所無。}。竹思寬每常都是燈下相會。今在白晝。看得分外眞切。愛到百分。摟住又親了幾個嘴。抱他在懷中坐着。各訴相思。竹思寬把他近來想的計策。詳細說了一遍。火氏喜的只是笑。就如頑石聽得生公說法一般。儘着點頭。竹思寬又道。外邊慫恿在我。裡邊攛掇在你了。火氏有利於己。自然虛心承敎。兩人敍到情深之際。竟忘了丫頭拿茶。聽得敲角門響。吃了一驚。火氏道。丫頭拿茶來了。我兩個在這裡好好的閂着門做甚事。這怎樣的。竹思寬道。不妨。我且回去。你去開門。只說我去久了。火氏還有些不捨。竹思寬道。我們若此計成了。相會有日。不在此一時。忙忙開門而去。火氏把院門揷了。將書仍放在架上\footnote{細。}。把那一條幹一塊濕一塊的汗巾。揩不得嘴了。塞在褲帶上。以備他用。走到後邊來開門。道。竹相公早去了。我又怕撞了外人來。故此把前後門都閂了。你跟我回去罷\footnote{此等脫空話。只好哄丫頭。外人自後門而來耶。}。到了房中。他數年所聚的那些慾火。今日忽經了這一番狂弄。雖不能十分大洩。也覺寬舒了好些。心中快爽。上床睡了一覺。過了數日。火氏正想竹思寬所說之話不見動靜。恐計不行。心下憂疑。只見鐵化走了進來坐下。鐵化當日怕他。躱避慣了。或一兩個月進來宿一夜。火氏總不許他沾身。他也無可奈何。自從火氏與竹思寬私通之後。自己良心尚有些過不去。未免內愧。可有個人家的妻子。陰戶外人倒弄得。親男人倒弄不得。焉有此理。後來待鐵化也就寬了幾分了。鐵化見他不開口便罵。動手就打。以爲他年漸日增。故而知事賢慧。也就漸漸來溫存親熱。就是要高興高興。火氏也不那樣拒絕。鐵化覺他的陰戶大的無比。也不疑着他有別的甚事。只說他身上發了福。所以此竅也隨運而寬\footnote{奇想。甚趣。}。還怨自己的東西太小。再不想是竹思寬揎得如此。那火氏見他素常要弄。也便任他弄弄的。也不知癢。也不知麻。似有如無。只知肚子撞肚子。混拱一會而歇。這火氏叫做。

\begin{quotation}

曾觀滄海難爲水。除却巫山不是雲。

\end{quotation}

數年來。一個月中他夫妻竟有十數夜同臥。五七上身。這日鐵化到了房中。說道。我有一件事來同你商議。我是要遠去的。你看可行不可行。火氏道。是甚麼樣事。〔鐵化道。〕如今兵部阮老爺大行賣官。價錢又賤。老竹勸我趁此去求取功名。他的親表姪姓黃。是他母舅的孫子。現當阮老爺的書辦。在外招攬過付。有這個好機會。你道這事該做不該做。火氏知道是竹思寬的計行了。心中大喜。一臉的笑。道。這是上好的事。爲甚麼不做。豈有戀着夫妻的恩愛。連功名都不去求麼。要去。該快些去纔好\footnote{何不云此刻就去呢。}。鐵化見他說得名明正順。疑他想做夫人的心切。那知他是棄小鐵而取大竹。鐵化道。老竹也說事不宜遲。要去早晚就要動身。火氏道。你這樣大家私。你去了。我只照管得內裡。外面的事托誰料理。鐵化道。我去若得了功名。打發老竹回來。托他照看。火氏聽得這話。心中喜極。由不得要笑。板住臉道。老竹做人如何。他可肯替你照看。旣要托他。除非常在家裡住着纔好\footnote{這是第一句要緊的話。}。誰沒家小。恐他未必肯來。鐵化道。老竹做人又老實。又能幹。可以托妻寄子的好朋友\footnote{如今好朋友大槪如是矣。}。我同他商議明白了。包他家中一年需用。他雖不能成年住在我家。就是兩頭來往照看也罷了。火氏道。你到那裡。事體一完。就快快打發他回來纔好。鐵化道。這是自然。不用你說。火氏道。如今你只快些料理外邊的事。裡面事我替你打點。鐵化見他忽然賢慧到這地位。感激不盡。那裡知道火氏巴不得此一刻送他出去。別圖樂境。火氏忙吩咐丫鬟僕婦打點衣裳行李。又把家中有七八個壯僕丁都叫了來。每人賞銀十兩。製辦行裝。跟主公出門。鐵化要留兩個看家。火氏道。你如今要出去謀官。也要個體面。家中有兩個老頭子看門就罷了。要人做甚麼\footnote{人多礙眼。不得不盡行遣去。}。鐵化見他盛情。不好違得。也去打點。一應停當。擇日起身。先一夜少不得要同火氏餞行。枕上又囑了許多看家的話。火氏別無他囑。惟以家下無人。着竹思寬速回要緊。次早分別。火氏同他雖不恩愛。也是許多年的夫妻了。今日雖喜他遠去。心中竟像要永別一般。悽然有戀戀之意。送到了廳上。又看見竹思寬。不覺掉下淚來\footnote{見鐵化去。心只悽然。見竹思寬方掉淚。刻畫淫婦情形。輕重入神。如見淫婦心。}。鐵化見他如此。只當是捨不得他。心中甚是難過。便撫慰了幾句。硬着心腸去了\footnote{盲鰍做夢。}。火氏見他出了門。好事有了八九。專等竹思寬回來。便做圓滿會場了。望了有個來月。不見他來。每日求籤問卜。問行人回來的日期。家中婦女見主人纔去了幾日。主母盼他歸來。暗地好笑。那知他問的是心上情人。有那八句說火氏。道。

\begin{quotation}

天涯海角有窮時。惟有相思無盡期。

殘夢樓頭空自憶。離愁花底問誰知。

雲山深鎖眞堪恨。風雨翻成薄命詞。

幾句鱗鴻占信候。金錢擲破嘆歸遲。

\end{quotation}

一日。童自大有事經他門口過。想道。內兄去了月餘。不知可有家信回來。我何不進去看看。遂走了進來。看門的老僕忙入內報知。火氏請他到上房。笑吟吟的迎着。讓了坐下。問了些家常。火氏忙叫取酒來相待。童自〔大〕道。不消了。我要回去。火氏殷勤相留。童自大見他情意諄切。只得坐下。頃刻。擺下一桌絕精的果肴。火氏斟了一鍾酒。送與童自大。他連忙接下飮過了。然後彼此相讓。各飮了數杯。火氏素常聽見鐵化說童家妹夫會採戰。崔命兒被他弄死。火氏久有心想領他的大敎。此時望竹思寬。正等得心中火發。今見了他。就注意在他身上。火氏是無酒量。頻頻相勸。童自大的酒童(量)自大。本好飮一杯。見他讓得殷勤。也便杯杯不辭。飮到將暮。竟酩酊大醉。就伏在桌上睡着。叫丫頭擡到自己床上。他把四個丫頭每人賞了幾鍾酒。亦都醉了。他到西邊屋裡設了一鋪自睡。不多時。丫頭們都醉得沈沈睡熟。他便走過東屋來。上了床。輕輕替童自大脫了衣服。他自己也脫光了。蓋上被。共枕而臥。伸手去摸他的陽物。雖無竹思寬的長大。較之鐵化更強許多。淫心頓起。那裡還睡得穩。又不好叫他。喜得夏夜甚短。直到五鼓將盡。童自大方纔醒了。見傍邊臥着一個精光的婦人。拿手弄他的陽物。他糊糊塗塗。也忘了是舅子家。只當是家中與妾同臥\footnote{有此一語。以洗童自大的罪名。}。趁着些酒興。就上身高興起來。採了一次。那火氏快樂非常。覺勝竹思寬數倍。淫浪得無比。渾身戰巍巍的。如舞梨花一般。四肢百骸活動異常。童自大覺得他衆妾中無此伎倆。心中疑惑。問道。你是那一個。火氏不好答應。只嘻嘻的笑。不多時。天色微曙。童自大定睛一看。原來是嫡嫡親親的嫂。忙拔出來。道。這是怎麼說。忙忙穿了衣服。回家去了。火氏見他如此。雖然覺得有些不好意思。已得嘗了這美味。心中十分快暢\footnote{此一段極寫火氏之淫濫者。爲死做結。二寫童自大始終不脫一呆字。亦是做結尾耳。}。且說鍾生在家。一日。梅生來相探。說道。弟適間在笪橋市。遇見擁擠着許多人。說是殺流賊的奸細。兩個賊頭。十數個從賊。不知是從何而獲。正說着。宦蕚也來相訪。說起方知其詳。你道殺的這奸細是何處人。是如何擒獲的。他二人是一胞胎生的兄弟兩個。父姓艾名金。妻子能氏。並無子女。在江寧縣牧龍亭居住。家中開着個小客店。在鎭市盡頭安歇過客。這牧龍亭是當年秦檜祖居生身之所。秦檜的墳基尚在此處。這一村姓秦者一多半。皆不認秦檜是一強徒。從無一人在他墳上祭奠。那墳地漸漸平塌。不過有一故址而已。艾金的房子就在他墳前。這艾金臨生之時。他父母夢見秦檜來投胎。因此他的小名叫做檜兒。長大了時。他父母已故。他將父母葬在秦檜墓傍。相離咫尺。他見秦檜之墓竟像他家祖墳一般。年年添土。把一座荒墳壘得老高。節節拜奠。傍人無不含笑驚詫。他夫妻二人一生貪財。見了錢連命都不惜\footnote{何今日愛金夫婦之多也。}。又刻薄不過。見別人的一文錢。他都是心愛的。有那趕集的人在他店中過夜。次早開發店錢分。數足了遞與了他。他接過來數。定要藏起一二文。賴說短數。那人接過來數。果然少了。只說自己數錯。添上給他。那裡疑他開着個店。戴頭識臉的人。肯落一兩文錢的理。孰不知他叫做老臉大發財。那錢竟歸之袖中。諸如此類。他無樣的相應不想出來。到處定要沾人些便宜。是俗語說的。見糞桶的過。也要拿笊籬撈撈的主兒。一日。傾盆大雨。時將下午。他道此時這大雨料也無人來了。出去要關鋪面。只見有兩個人騎着兩頭肥驢如飛而來。竟奔他的店中。他連忙讓進。接了頭口。就去搬行李。覺得內中甚是沈重。送到客屋裡去。關了門進來。忙叫妻子做飯。整治菜蔬。忽聽得外邊客人叫。忙走出來。那客人道。我們因趕路程。不想遇了這樣大雨。渾身上下連被都淋濕了。此時身上有些涼涼的。你把好酒熱得熱熱的兩壺來。那艾金耳朶聽着他說話。眼睛往兩張床上一看。見他的被褥衣裳都打開晾着。一張床上放着一個搭連。每個裡邊約有三四百兩的樣子。心中好生動火。進來燙酒。那能氏正在燒火。那柴被雨淋濕了。吹灼又滅。煼得兩眼眼淚直流。他把火筒一摜。道。受瘟罪的。我看開了這些年的店。也不見積的錢在那裡。煼得七死八活。受這樣的罪到那一日。火還燒不灼。還燙酒呢。艾金把壺就放在鍋裡。就拾起火筒去吹火。一面燒火。一面出神。不住點〔頭〕磕腦的算計。能氏道。你出神想甚麼。艾金道。我纔見這兩個客人竟有八九百銀子。我想我們開着這個店。那一日纔得發財。要得了這項物件。也不枉爲人一世。所以在這裡想昏了。那能〔氏〕更愛錢。更毒。他想了一想。道。我倒有一個主意。可以圖得他的。只怕你不肯。艾金笑道。你的意想是要捨了身子。弄他的銀子麼。他五錢一夜嫖得好不受用。你是個甚麼天上有地下無的奇屄。他兩個就捨得這些銀子送你。若遇着兩個狠手。銀子不能得。皮還弄塌了呢。能氏笑着啐了一口。道。我是正經話。你就胡說白道的。艾金笑道。你有甚麼主意。能氏道。自古說。圖財害命。你肯害了他的命。就可以得了他的財。艾金道。人說婦人家見識短。果然不錯。你也想一想。他是兩個。我是一個。財謀不成。弄的不好。到了官。先要短了半截。就作算謀死了他。放在那裡。鄰舍們知道了。豈是兒戲的事。況且還有兩頭大驢。越發沒處安放。能氏道。你麼空給你一個男子漢做。一點見識都沒有。今日這樣大雨。他兩個進來時。料想街上是一個人也沒有見的。只要有本事弄死了他。我家後園裡頭大靛池那裡。不要說兩個。再有兩個。也放下了。深深的埋上。眞是神不知鬼不覺。兩頭驢殺了醃起來。驢比不得豬。殺時又不會叫。醃成驢巴還夠吃好些日子呢\footnote{眞能。}倒只怕你一個沒本事收拾兩個。還是正經話。說話之間。酒已熱了。拿了兩碟小菜。送了出來。那客人嘗了嘗。說道。你們一個大路口開着這麼個店。怎麼賣這樣薄酒。眞吃不得。換些好的來。艾金道。我們這一鎭的酒並沒有一家的好。要有好的。豈可不打來爺們吃的麼。客人道。旣無好酒。你把黃酒拿回。可買好燒酒來。多買幾斤我們吃罷。艾金只得進來。尋傘找瓶。嘓嘓噥噥道。天下雨。將就吃些也罷了。又叫我去打燒酒來。泥爛路滑的。能氏大喜道。這是龍天保佑。該我們發財了。艾金道。怎麼說。能氏道。東頭米奶奶家今日正淋燒酒。昨日他老人家約我今日去嘗。因下雨。我沒有去。你到那裡。不要說客人要吃\footnote{細心。眞能。}。只說我身上有病。要些乾榨酒泡藥酒吃。寧可多幾個錢一斤。不要攙了水的。那乾酒甜甜的。吃着爽口。一時發作起來。如同小死。若天幸。他兩人醉倒了。那時動手就容易了。這豈不是天賜財緣麼。艾金聽了他賢妻這樣的妙計。歡天喜地而去。也不顧脚下的泥濘。如飛而回。連瓶拎到客屋裡頭。〔道。〕這是五斤好酒。爺們請嘗嘗。他二人嘗了嘗。道。好酒。你連瓶放着。倒是冷吃罷。燙熱了又沖鼻子。又噎喉嚨。這寡酒難吃。你把菜飯都拿來。我們先就着吃酒。艾金進來取菜。只見能氏拿了一把艾金防身的短刀在那裡磨呢\footnote{記着這一把刀。}。艾金說笑道。古人的話。一些也不錯。道是。

\begin{quotation}

靑竹蛇兒口。黃蜂尾上針。

兩般猶似可。最毒婦人心。

\end{quotation}

你就這樣性急。快打發菜。他們要飯菜吃酒呢。能氏便裝了盤子。無非是煎雞子炒韮菜兼蝦米拌木耳腐皮之類。掇了出來。擺在桌上。這兩位客人酒量頗雄。也是該他命盡。一個一鍾好吃。把那酒也就吃了個八分。都有大半酣了。艾金在傍伏事。聽得內邊叫道。來拿了菜去。艾金忙出去接了。攜送到二客面前。笑道。這是今日早起。我買得幾個活鯽魚。做兩碗醒酒湯。敬二位爺。二客正是酒渴。喝了一口。又酸又鮮。連贊道。好東西。肚裡也有些饑了。連魚帶湯全吃了下去。這是能氏想的妙計。恐怕醉不倒他二人。見他吃的是冷酒。做了這兩碗熱湯來。名曰是醉(醒)酒。其實是發酒。一肚子的冷乾燒酒。被這熱湯一沖。就發作起來了。不多時。一個仰着臉頭靠在椅背上睡去。一個伏在桌子上也就去夢黃粱。艾金忙走進去。拿出母夜叉蒙汗藥武松的樣子來。向能氏笑嘻嘻的拍着手。道。倒了。倒了。能氏歡喜得了不得。忙把刀給與他。一同出來。大門閂〔得〕好不結實。進房中來。能氏先指着那仰面睡的脖子。做個殺雞的手勢。叫他動手。艾金貪財心勝。膽大如天。也顧不得天理了\footnote{俗云。色膽如天。此則財膽如天。可見人壞心一起。則不能制服。}。走到跟前。壯着膽子。對準喉管。盡力一勒。那客撥鼓通一聲。跌倒在地。那一個伏着的驚醒了。擡頭看見。叫了一聲哎呀。那艾金着了急。連頭帶腦狠狠的一下。也劈倒在地。蹬了蹬腿亦已嗚呼\footnote{記着他二人是如此死法。與艾金死時對看。此一段雖寫艾金能氏之意。亦是警省在外做夢者。第一要小心。第二勿貪飮酒。愼之。}。夫妻二人見都完賬了。擡到後園。抛在靛池中。那裡還顧得甚麼泥水。忙忙埋好。又來把他二人的行李搬了進去。將兩個搭連向床上一倒。每個裡面八封。兩個十六封。共八百兩。餘外還有幾十兩零碎的。擺了一床。眞是歡心樂極。眉開眼笑。忙騰個竹箱收了。又忙到客屋裡。將血跡都洗淨。收拾得乾乾淨淨。夫妻二人一夜不睡。把兩頭驢也宰了。開剝醃了。眞是人逢喜事精神爽。竟不知困乏。夫妻又商量一會。過了好幾日。將店關了。從新修飾。開了一個雜貨鋪。四路鄕村明知他的東西比城中貴些。因省了往返四五十里路。都在他鋪中來買。總是他這一步時運好。倒也着實大發財。他又買了幾十畝膏腴好地。招人佃種。合村的人都疑他是掘藏。那裡知他是做了這件盛德的好事。發了外財。過了些時。能氏竟懷起孕來。他夫妻大喜。道。我兩人十多年來總不見生育。今做了這樣好事。不但發財。又得了胎。眞是百福駢臻了。\endnotemark[5]到了月分滿足。分娩之期。一胎生下兩個兒子。能氏將四十歲的人纔生頭胎。萬分艱難。昏暈過去幾次。兒子雖然生了。却把兒子的這位成家能氏早已了賬。艾金雖悲哀亡妻。却喜得子。此時他在村中算小財主了。典了村中兩個有奶婦人來做乳母。男人替他家種地。能氏死的那一晚。他父親續娶的後娘亦臨產。他父親夢見能氏復來托生。說道。兒今來托生。將來還嫁艾家。好了結前賬。他父親醒來。雖不懂其中的原委。心中暗暗稱異。少刻。他妻子果然生了女兒。次日。艾金到丈人家報喪。他丈人方知夢幻非虛。就將小女兒叫做再來姐。光陰似箭。日月如梭。有話即長。無話即短。那艾金的兒子大的取名艾鮑。次的名叫艾福。與再來姐同年同月同日所生。但時刻略差。都到了十歲。艾金見再來姐的神情舉動同能氏再生無二。又素常聽見丈人說是他亡妻托生。就向他丈人說要定他續絃。他丈人是個窮莊家漢。見女婿於今是財主了。況女兒托生時原說還嫁艾家的話。那管他年紀大着四十多歲。滿口應承。到了十六歲。娶了過來。此時艾金五十餘了。且說這艾鮑艾福先年小時還好。到了十二三歲時。就是兩條蠢牛。艾金若使喚他兩個。他兩個便橫眉豎眼的道。我們不知道。如再叫狠些。他二人便喃喃嘟嘟的亂罵。艾金又溺愛慣了的。只裝不聽見。如是多次。越無忌憚。艾金或唬嚇要打他。一跑無踪。躱在鄰舍家空園中。艾金怕他逃了遠處去。便各處尋遍。方覓得回來。逢時遇節。叫他二人祭母親墳。他道。我們不知道甚麼叫做母親。我們沒有見過。要上墳你上去。我們不去。艾金強要叫他去。他二人便跑去。不知去向。到晚方回。後來不但性子憊賴。又是吃酒。又賭錢。又行凶。他兄弟二人却甚是和氣。獨同老子是寃家。常在背後嘓噥道。我同你甚麼父子。那一日我還要殺你呢。那艾金明明聽見。自己旣不能管。他又捨不得送官處治\footnote{溺愛不肖之子。必致釀成大害。}。不由得心中竟隱隱有些害怕他二人。他兄弟見老子娶了後娘進門。暗忖道。這個老頭兒作孽。這樣大年紀娶這樣個少年妖精。他同我兩個同年月日。與我們正是對子。今旣在一家。豈可錯過。他兄弟二人商議道。俗語說。月裡嫦娥愛少年。姨娘嫁了這個老頭子。再沒有個不氣的。我們兩個慢慢的齊心調戲他。管他姨娘不姨娘。後娘不後娘。你弄上了手也不要偏我。我弄上了手也不偏你。大家受用。那艾金見兩個兒子十七八歲。長成兩條大漢。他常向人誇道。我行了一輩子的好事。陰隲上積了這一胞胎。生兩個好兒子。外人知他奶(乃)郞的壞處。還只是暗笑。惟有這再來姐獨看上了這兩位賢郞。他心中常想道。我這樣靑春年少。正該同他兄弟兩個相配。怎爹娘把我嫁了這位老姐夫。如今無可奈何了。兩個外甥我雖明嫁不得。暗裡嫁他誰人管得。那尼姑下山的曲子道。男有心。女有心。那怕山高水又深。何況是一家住着。又沒傍人礙眼。他們旣都有了私心。在那言語顧盼之中。也有許多曼倩之態。一日。艾金進城去買貨。艾鮑向兄弟道。我看姨娘近來也像有些愛我們的樣子。今日趁老頭兒不在家。我去硬撞。我若上了手。你就接上。艾福喜諾。艾鮑昻昻進姨娘房中。艾金床頭有防賊的一把短刀\footnote{此刀即前所用之刀也。此處再用一次。是現刀之時也。}。他走去一下拔出來。惡狠狠劃剌一聲。往桌子上一摜。就上前將姨娘抱住。親了個嘴。便伸手去扯褲子。再來姐笑道。短命鬼。你不過是要這樣罷了。冒冒失失。嚇我一跳。艾鮑見他毫無拒意。抱到床上。脫去了下衣。竟弄將起來。再來姐雖嫁了多日。那艾金齒邁力疲的人。怎如這艾鮑少壯雄偉。今嘗此味。心喜非常。做出許多騷樣。艾鮑亦是初嘗滋味。乍親女色。不多時。便洩了下來。艾福在門外張看。見哥哥完事。他忙忙進來。接着就上。再來姐興尚未足。也便笑納。幹訖一度。從此以後。他老子出去了。他二人便來同再來姐作樂。這再來姐得他兄弟兩條健漢。以供胯下之樂。反視艾金如贅疣一般。一日。艾金出〔去。〕他兄弟兩個走來同再來姐大弄。彼上此下。弄個不住。正在興濃。不想艾金撞進來。猛然一見。竟癡呆了。氣得大張着嘴。話都說不出。圓睜大眼。怒狠狠瞪着。再來姐着了急。忙把枕頭下那刀拔出。遞與艾福。道。你不動手等甚麼。艾福接過刀跳下床。艾金見了刀。越走不動。方要叫喊。被艾福舉手劈面一刀砍去。跌倒在地\footnote{即前殺客之刀。}。艾福棄了刀。忙穿衣服。那艾鮑見他老子還在地下蹬腿。拾起刀。向喉下一勒。一個頭伶仃將斷\footnote{艾金製此刀只用一次。他賢郞倒用了三次。}。三人忙穿完了衣服。他兄弟上街買了一口薄皮棺材。將他老子屍首裝好。停在堂屋裡。將血跡洗得乾乾淨淨。然後走去報喪。也不念經。也不開弔。只放了七日。就厝在後園內。再來姐戀着他兄弟二人。不肯改嫁。只說要守貞節\footnote{說要守節者。決不能守節。自古來。口口說忠孝。能盡忠孝者幾人。}。他父母不能相強。誰還來管他家的閒事。況且傍人見他不但是繼母兒子。而更是親姨娘外甥。那裡疑他有禽獸內亂的事。他二人倒像再來姐的一妻一妾一般。夜間三人同榻。好生恩愛。不二三年。他兩個把家私賭得精光。連房子都輸了。算了與人。他三人在後園中搭了一間小房子住着。再來姐一夜夢見艾金渾身是血走到他跟前。道。那兩個奴才殺我。算一報還一報也罷了。我同你兩世夫妻。有何仇恨。你挑唆他殺我。我如何饒得過你。說了。就往他身上一撥。再來姐一驚醒了。却是一夢。心中甚是憂疑。又不好向他兄弟二人說。隔了些時。忽然有孕。他二人着急。恐一時有人知道怎處。便往城中去買打胎的藥。走到半路上。遇着搖鈴的大夫。便問他有打胎的藥沒有。大夫說有。他便買了些回來。與再來姐吃了下去。其應如響。肚子疼得要死。却生不下來。喊叫不止。他二人着了急。艾鮑一把緊緊抱住。艾福一隻手摀住他的嘴。一隻手下力一陣揉。約有兩個時辰。纔把娃娃打下來了。再來姐也就血暈而死\footnote{好藥。此婦兩世遭產難而死。其凶淫之報晈然。}。他\footnote{巧寃巧報。毫髮不紊。}們此時精窮。也無錢買棺材。那能老兒夫婦又死了。他哥哥是個大窮漢。艾鮑向他商議。他竟不管。兩人只得在後園中挖了一個坑掩埋。挖下有三尺餘深。忽見兩副骸骨。他二人慘然道。這不知是甚麼人的屍首\footnote{就是二位。}。不要動他。就將再來姐的屍首並那死娃娃。同那二骸骨合埋了\footnote{昔年能氏云。再有兩個。也放了下去。不想他夫妻轉生下世。亦在此中。是四個。以應先兆。}。\endnotemark[6]他二人毫無所戀。商量道。我們如今無穿少吃。站不住了。常聽得人傳說流賊們着實快活。金帛子女四處搶擄。無窮的受用。我們把這園子賣幾兩銀子做盤纏。去投他們。豈不是下半世快樂。二人主意議定。把園子賣與房主。房主嫌他父柩厝在裡面。不要。他二人將艾金的棺材擡到秦檜墳邊一塊地上放着\footnote{前世愛金朝。今世愛金銀。貪惡之骨。應該葬在一處。}。就算了牛眠吉地。他把園子賣了幾兩銀子。塞在腰中做路費。投流賊去了。艾金的那口薄材風吹日曬。久之朽爛。那骨節也就抛散四處。不消說得。還不如那兩個客人。得個靛坑掩埋。免得暴露。艾金夫婦想做財主。圖得受用。艾金落了這樣個下腸(場)頭。雖不曾遭了國法。這報應也就儘夠了。那能氏更可嘆可笑。設了一番毒計。想做財主婆。剛剛快樂一年而慘死。人算不如天算。信哉。艾鮑艾福眼見得是那二客轉世索債的。再來姐生前挑撥艾金殺二客。今世又挑撥二子殺艾金。旣以身償二子之淫債。又遭產厄。據他之夢。此子又是愛(艾)金來索命。報應分明。毫厘不爽。鬼神在冥冥中。豈有一着放鬆。世人只知任性胡爲。就不回想一想將來的後患。眞是感應篇上說的好。如漏脯救饑。鴆酒止渴。非不暫飽。死亦及之。可不懼哉。艾鮑艾福雖是再生復仇。但今生名分。以子弑父。奚能免得天誅。不死於此。必死於彼。少不得暗暗有一場惡報。他二人奔到陝西。投了李自成。闖賊見他二人數千里遠而來相投。且又生得雄阧。正在少年。心甚歡喜。便留在部下。充了兩名小頭目。後來差了些賊四處攻劫。時常得勝。漸漸得攻(功)。破北京時。每人給了一張僞總兵的箚副。瞎賊被大淸兵殺敗。逃奔湖廣。又想攻取南京。因他兄弟是南京人。又聞得馬士英阮大鋮大賣官職。叫他二人各帶了五七個心腹小賊。馱了兩萬銀子並金珠之類。投托阮大鋮。買兩名京城中管兵的武職。又給與僞箚數十張。以招攬好漢。將來好做內應。他二人歡喜無限。李自成又許他成功之日。俱加封侯爵。他二人更感恩效死以報。旣到了江北。訪着了黃書辦。浼他送了阮大鋮五千金。還有許多珠寶。要求補京營的武員。阮大鋮大喜。就放了他二人兩員京營遊擊。咨送馬士英考驗。艾鮑艾福也送了馬士英一分重禮。馬士英見他二人旣生得魁偉。又且有厚贐。自然依允補授。二人好生榮耀。金乎帶而虎其補。左右跟隨兵丁皆帶刀之士。他二人又將帶來的金珠之類孝敬了馬士英。拜在他名下。馬士英愈喜。待他二人甚厚。時常賜以酒飯。也熱鬧了幾個月。不想他二〔人〕該福盡災生了。一日。樂公下朝。到了私宅門口。只見一個人迎着轎子跪下。道。小的有機密軍情上稟。樂公吩咐帶了進來。問他何事。他道。小人姓蒙名德。係河南人。向年逃難來南。蒙百萬童老爺衆位施恩。救了一家的性命。次年得回故鄕。小的尋親戚。又還來了一次。回去時。不想遇了流賊。將小的一家殺害了。小的就被這艾鮑留了在家下使用。屢要逃出。但賊兵連營百里。再逃不脫的。前日李自成在山海〔關〕兵敗。今往湖廣去了。知道阮老爺賣官。因此打發他兄弟兩個來買兩員京營武職。並招攬人衆。他到湖廣聚兵來攻南京。將來好做內應。小的原係良民。怎肯做賊。向年老爺榮任府尹時。小的曾叨恩典。故此到老爺臺下來出首。樂公驚道。這雖是你的好心。但沒有甚麼憑據。蒙德道。他兩個現帶了李自成的許多箚副來招攬人衆。老爺只一搜獲。便是憑據了。樂公道。果然有此。你的功名也不小。忙差人去請魏國公。此時他正管提督禁軍團營事務。又差人去請都察院正堂。兵刑二部尚書。並錦衣衛指揮。不多時。都到了。樂公叫蒙德過來。將前事又說了一遍。衆官計議了一會。魏國公差人去傳艾家兄弟二人。又叫錦衣衛暗暗領人到他寓處。將他跟隨的人一並拿獲。不可走漏一個。把他行囊盡數拿來。少刻。艾福艾鮑已傳到了。叫了進來伺候着。不多時。錦衣衛官校將他人衆並行囊亦都取到了。樂公命細細搜檢。在一個皮匣內。搜出總兵副參游守僞箚數十張。魏公喝叫將艾家兄弟並手下人盡都拿下。叫蒙德將前事又說了一遍。他二人見活口僞箚俱已當面。無辭可辯。一一招成。樂公同衆官進朝面駕。將前事細奏。艾鮑艾福凌遲處死。從賊斬首示市。弘光准奏。立刻押赴笪橋正法。樂公又奏。阮大鋮身爲朝廷大臣。受賊私賄。題補京營武職。若非蒙德出首。釀成後患。京城內外生靈尚忍言哉。願亟賜斥逐問罪。即閣臣爲朝廷股肱。不察奸細。和光同塵。亦不能辭賊。這阮大鋮馬士英都是弘光的心腹。進美女。獻孌童。合春方。感激他們了不得。焉肯說他的不是。替他辯道。馬先生他不過見阮大鋮送來考驗。他見人品好就准補了。這有何過。就是阮大鋮他也不知他是流賊一黨。他要知道。肯擢用他麼。至於說受賄。那不過是蒙德小人口中的一句話。如何就做得准。便輕易壞一個大臣。樂公再四進言。弘光執意不聽。樂公又奏蒙德有出首之功。當加重賞。弘光因他說阮大鋮的不是。心中暗惱。說道。蒙德從賊已久。今雖出首。原自首免罪例足矣。如何還要賞他。樂公只得同衆官退出。見朝廷功罪不分。還成個甚麼法度。不勝忿怒。遂嘔了兩口血。從此就得了病。將二賊的行囊中尚有萬餘金。賞了蒙德五百兩。餘者咨送戶部。留充兵餉。蒙德身無所歸。情願在樂公家當長隨。樂公也着實優待。後來樂公病故還鄕。他送到了家。然後纔回河南。這是後事。宦蕚同梅生在鍾生家說起殺奸細的話。宦蕚道。方纔有一個舍親在刑部。他纔說起這事。因把蒙德出首。樂公擒賊。並馬士英阮大鋮受賄賣官與賊。弘光堅執不聽的詳細相吿。鍾生惟長嘆數聲。再無他語。且說那火氏自試了童自大一番之後。心中想道。不意世間有此奇物。他若肯與我相交。又還稀罕老竹做甚麼\footnote{有此一念。可見火氏全是貪淫。毫無情意。淫婦之濫如此。}。我看他前日那個樣子。是決不肯再來的了。只好等老竹來家。做個長遠主顧。他一時淫情舉發。那裡還制伏得住。日夜盼竹思寬回來。好做一番繡衾大戰的事。總不見到。又過了幾日。時已初秋。情緒無聊。他到樓上去倚欄盼望。兩眼眞要張穿。見那朱榻依然。那一條妙狗已成朽骨。不由得一陣酸心。口編了個四句半的劈破玉。低聲唱道。

\begin{quotation}

趁此秋光。凭倚南樓。想當初大雁兒飛去。小燕兒飛來。他兩個相遇在途中。他把春秋談論。大雁兒叮嚀小燕兒。囑咐你我兩個。作速分飛。休要耽誤了工夫。他那裡說。你我失却了信行。到如今。你看小燕兒飛去。大雁兒他信信行行又來了。寃家。你可記得孔聖之言。大車無輗。小車無軏。豈可以行之哉。言而無信的寃家。你反不如了個禽鳥。我自眼含着珠淚。哭進了香房。跌綻了金蓮。自嘆了一聲。哎喲。奴家不是悔恨當初錯認了你這人兒。寃家。我似醉如癡方纔醒。好一似吊桶落在他人井。

\end{quotation}

唱罷。不由得香腮上淚下了數點。心似油煎一般。忽門上那老漢進來說。竹相公帶了信回來了。火氏聽見。眞是喜從天降。精神頓起。忙忙下樓回房。便道。快請了進來。少頃。老漢同竹思寬到了堂屋內。火氏出來。竹思寬作了揖。火氏回拜。讓了坐下。竹思寬道。恭喜奶奶。老爺榮任去了。自從到了那裡。送了禮。阮老爺大喜。特放了長河衛掌印指揮。我又同到了任上。那裡沒有文官。老爺上馬管軍。下馬管民。地方又富庶。着實威武。我住了幾日。老爺恐奶奶懸望。着我折身回來。所以遲了這些日子。因鐵化做了官。有幾句說那時的時事。道。

\begin{quotation}

是非倒置太糊塗。此輩如何濫仕途。

只爲錢神能效力。掌餘不復問賢愚。

\end{quotation}

火氏將家中男婦却叫了上來。吩咐道。你主人得了官。上任去了。竹相公在家中照看。竹相公在書房安歇。你們小心伺候。但是竹相公到來。可到上邊來說。老爺不在家。我這裡也無事。僕婦們也不必上來。有事來叫你們。大廳後總門並角門。不到晚丫頭就早早關上。衆人應諾。以爲奶奶這樣貞操持家。誰敢不遵。那知全是詭計。竹思寬起身辭道。我今日到家看看。明日再來。火氏也不留。仍着老漢同他出去了。火氏次日命擡了一罎好酒。自己親手整理了一桌豐盛碟子。下午竹思寬來了。老家人上來說。火氏吩咐廚下備飯與他吃。老早就叫丫頭把大廳後門關上。床上換了一副新被褥。虎皮褥子。虎丘蓆。正是。

\begin{quotation}

安排新衾枕。好接舊情人

\end{quotation}

到晚來。從新梳妝打扮。換了一身新衣。把牝戶用香肥皀搓洗了一番。掌燈時。火氏命丫頭點上了兩枝通宵紅燭。擺上碟子。燙着酒。吩咐丫頭們道。你主子托竹相公看家。我們是主。他是客。豈有個不款待的。請他來坐坐。你們都在跟前伺候。不許躱懶。着兩個丫頭前邊去請。不必走大廳。打角門裡去。丫頭去不多時。同竹思寬來了。讓了對面坐下。竹思寬見他越發風流標致。身上一陣陣的香氣撲鼻。神魂愈覺癡迷。火氏也同他睽違兩個多月。且從不曾來到這個屋裡。也想兩樁舊物試試新房。無奈丫頭在傍。只得勉強假做正色吃着酒。口中雖假說着正經話。兩隻眼却餳瞪的望着他。面上不住微微的笑。竹思寬也心中着急。恨不得同他摟做一處。眉頭一蹙。計上心來\footnote{思寬此計。火氏前在童自大面前已用過。}。望着火氏丢了個眼色。道。難爲這些姐姐們在這裡伏事。我每人敬他一碗。便要了個飯碗來。一人一碗。丫頭們誰有這樣大量。推辭不飮。火氏道。不識擡舉的賤肉。竹相公賞你們。敢不吃麼。幾個丫頭沒奈何。捏着鼻子每人灌了一碗。竹思寬道。敬個雙鍾。丫頭們見竹思寬讓着。主母壓着。諒也不得不吃。又吃了一碗。內中只有一個略好些。那三個跑到西屋裡。連晚飯同酒一齊從嘴裡鼻孔裡都倒出來了。吐得無處不是。倒在地下就睡着了。這一個執着壺。東晃西晃。也站不住。竹思寬道。你把壺且放着。我自己吃。你歇息罷。那丫頭巴不得。一聲把壺放在桌上。也跑過去。倒下頭睡了。火氏忙把房門關上。此時初秋還熱。二人脫得精光。相摟相抱。一口一遞吃了幾杯。火氏又嘴含着度了他幾口。看見竹思寬的陽物直豎。忙拿汗巾裹了根下。火氏上他身來。對面套入。一面吃酒。一面動作。頑了多時。酒性已闌。色性大熾。相攜上床。這一夜。此上彼下。此下彼上。二人做了個通宵的活計。自相交十多年來。這算頭一次放心受用。天色將明。竹思寬穿衣出去。火氏睡了一覺。方纔下床。到西屋裡看丫頭們時。都還醄然未醒。叫了起來。一個個都還暈頭昏腦。收拾了傢伙。隔三四日定請竹思寬一次。幾個丫頭也大醉一次。且說他家這些僕婦。丈夫都跟隨主人去了。主母上邊又不叫他們。每日無事。三個成羣。四個作黨。在一處閒磕牙。偶然一日。三個婦人相聚在一處說家常。正是當日看見竹思寬陽物的那一個。同着火氏在西屋裡說的那兩個。他三人說話中間。一個道。我們男人這一去。不知幾時纔回來。料道貞節牌坊是輪不到我們的。竹相公現在我們家裡。你當年看見他那件寶貝。是個異樣的東西。你何不去試他一試。看是個甚麼滋味呢。這一個道。那東西我是不敢惹。他一時搗斷了腸子。沒處尋這小皮匠到裡面去縫。你當日說吃四兩燒酒還捱得半截。你何不喝四兩。先去擋個頭陣。要不妨事。我們大家也去嘗嘗。那一個又慫恿道。你果有這高興。我去打酒。替你助助興。這個婦人又好酒。酒下一字更好。也說上興來了。便道。從來沒有聽見閻王跟前有肏死的鬼。你果然打了酒來。我吃個半酣。去捱一下子看看。那個婦人果然掏出幾十文錢。到大門口。煩那看門的老兒打了幾斤燒酒來。他接了拿到屋裡。撕了兩碟小菜。三人說說笑笑的共飮。讓那個婦人道。你多喝兩鍾。膽子壯些。那婦人也不辭。飮到掌燈時候。酒已罄了。便道。我們同去。再遲了。恐他睡下。都有了幾分酒意。就到書房裡來。見院子門關着。輕輕敲了幾下。竹思寬正打點要睡。聽得敲門。不知何故。只得走來開門。見是三個婦人。一擁而入。到了房中。竹思寬跟了進來。道。三位大嫂此時到這裡來。有何話說。那一個要擋頭陣的望着他嘻嘻的笑。這一個道。竹相公不認得我了麼。竹思寬道。雖然常在這裡。嫂子們的模樣都認得的。却不知姓甚麼。這個婦人笑道。他的男人叫做高興。竹相公是認得的。我那一年在茅廝上倒馬桶。遇見竹相公在那裡溺尿。我見了你那個稀奇物件。偶然對他說了。他想到如今高興哥跟老爺去了。他見竹相公自己一個在這裡。情願來奉陪。他自己不好說的。煩我兩個來做媒。竹思寬見他來就敎。何嘗不喜。但他三人同來。沒有個取一棄二的。恐怕一時弄上了。夜夜來纏。豈不誤了火氏那裡的事。又怕或遇了丫頭來請。走漏了風聲。假做正色道。這事如何行得。你主人托我看家。我若做了這事。一時人知道了。有何臉面。那婦人一團高興。被他一掃。老羞變怒。喉(猴)急起來。道。我好意來伴你。你這樣掃我。我當眞是求你的文麼。你怕沒臉面。我明日給你個當眞沒臉面。沒人處。我抓破了你的臉。我〖口么〗喝起來。說你調戲我。等主人同我男子漢回來。合你說話。看你有臉面沒臉面。竹思寬暗想。這等婦人。他知甚麼羞恥。倘然眞果做出來。如何了得。要回家避了。一來捨不得火氏。二來受了鐵化之托。突然回去。何以爲辭。心下一轉。道。老住了他。給他個辣手。叫他魂夢也怕。一個吃了虧。那兩個自然不敢再來纏繞。遂作笑容道。我是衛護你的話。我(你)爲何倒着惱。承你這樣好情。我感激了的了不得。我的東西旣是這位嫂子曾看見過。恐怕你受了苦。故此假拿那話回你。是我一團好意。那婦人道。我不信就這樣利害。你家奶奶也不過是一個屄。難道兩三個拼成的不成\footnote{奇想。}。竹思寬道。我先給你看看。你吃了苦。不要抱怨。遂扯開褲子。拿出陽具來。道。你看看。做得做不得憑你。那婦人見他厥物硬幫幫。像一節大熟藕一般。眼中冒火。也顧不得死活。口中道。我不怕。不怕。再大些我還不怕呢。他此時忍不得了。便褪下褲子。在那張醉翁椅上睡倒。兩條腿放在兩邊椅軸上。牝戶大張。竹思〔寬〕也脫了。安心要給他個利害。不但不用一點吐沫。對準了門。憑身盡力往裡一下。竟進去了有一半。只聽得那婦人叫了一聲道。哎呀。我死。竹思寬又往裡送了兩送。婦人眼淚直流。叫道。竹老爺。饒了我的命罷。竹思寬也不理他。又加力狠搗了兩下。進去有多半截。那婦人聲都啞了。渾身亂戰。叫喊哎喲哎喲。那兩個婦人看得毛髮皆豎。也不覺戰起來。竹思寬道。你纔說不怕。你忍一會就好了。一下全拔出來。又往裡一搗。那婦人又哎喲了一聲。戰都都的道。不好了。不好了。可要死了。我的小肚子要通了。竹思寬知他試着了辣味。猛然往外拔出。那婦人又叫了一聲。罷了我了。口中哎喲哎喲的哼。這兩個婦人看他時。臉白唇靑。渾身戰個不住。口中說。不好了。不好了。我的東西兩半邊了。二人再看他的牝戶。果然把後邊裂了開來。與糞門成了一個大窟窿。竹思寬兩隻手拉着兩個婦人。道。他不濟。你兩個來試試看。那兩個婦人用手摀着褲襠。兩腿夾得緊緊的。道。竹老爺。竹祖宗。我們是不敢惹你的。留着肚子吃飯罷。竹思寬笑着放了手。他兩個將那婦人扶了起來。他腰也彎着。直不起來。歇了有一個更次。也不穿褲。這兩個婦人攙扶着他。他一手揉着小肚子。一瘸一跛。嘴裡還哎喲哎喲不住聲而去。這婦人睡了有半個月才起來。腰還彎着有些疼。陰戶不知長嚴了不曾。就不知道。此後三婦再不想了。話休繁敍。光陰迅速。又早寒冬。一日天氣大寒。瑞雪紛紛。下了一日。火氏晚間請竹思寬進來圍爐賞雪。把丫頭們都灌醉了。全躱過去。鑽在被中。冷呵呵的。誰肯走來做甚麼。火氏同竹思寬飮了一會。都有了酒意。火氏道。床上冷。我們在火箱裡睡罷。起來鋪了被褥。放下了枕頭。把桌子擡過。靠了火箱。火盆也擡過來。好燙酒。二人脫了上衣褲子。火氏穿着一件紅綾小襖。竹思寬只着一件藍紬主腰。拿被蓋着下身。坐着吃酒。火氏道。我行一個令。我同他(你)猜枚。你贏了我。你上我身來抽五十下。我吃大大一杯。我贏了你。我到你身上抽五十下。你吃一大杯。可好麼。竹思寬笑道。難爲了我些。也罷。依你就是這樣來。二人猜拳。先是火氏贏了。竹思寬睡倒。火氏上身來套入。竹思寬兩手搊着他屁股。用力蹬坐了五十下。竹思寬吃了一大杯。又猜。這是竹思寬贏了。火氏仰臥。竹思寬扒上身來。火氏兩手扳着他的屁股。也狠狠的搗了五十下\footnote{火氏在上。故竹思寬用搗。竹思寬在上。故火氏用拔。罵出兩人淫像。此書開首。于敷同昌氏猜枚。書已將完。用竹思寬同火氏猜枚作結。前後照應。}。火氏吃了一大杯。上下幾次。竹思寬洩了。火氏正在高興。替他百般搓弄。弄硬了。重新又起。竹思寬連洩了三次。這却起不來了。你道竹思寬爲何就這等不濟。當日守着郝氏。髮蒼陰癟。十日半月不過偶然適興。近來遇了火氏。三四夜就要弄一回。這一夜中。饒不得他。要丢三四度。回家又要同郝氏做作一番。一個望六的人。如何禁得這等作喪。半年來精力衰敗。三洩之後。如一條粗皮條相似。火氏此時酒有十分。淫興也濃到十分。那裡肯放鬆他。替他百般舞弄。竹思寬也醉得很了。見他這樣騷淫。也想大弄一場。無奈陽軟。硬不起來。又生一段慚愧。又是那作急。還儘着呷熱酒。圖酒力來助他的興頭。火氏含了熱酒。在那龜頭上吮咂。又用舌尖在他馬眼又四圍舔那龜頭。竹思寬被他如此掇弄多時。又昻然大舉。二人又一陣翻騰。酒都有了十二分。火氏酒也翻上來了。仰搧着。閉着眼。口中模模糊糊的道。你用些力。狠狠的弄弄睡罷。竹思寬也醉昏了。聽了這話。一進一出的亂搗。火氏心中快活。竟矇矓睡去。竹思寬洩過三次。也容易不得再洩。陽物被酒助動虛火。分外堅硬。形如鐵杆。混舂亂攮。他二人翻騰了一夜。不想束根子的那條汗巾揉撞散了。竹思寬醉昏了的人。忘其所以。覺汗巾拖着礙事。伸手扯去。又平(憑)身向下加力。往下一攮。一下直到了根。只見火氏大叫了一聲。腿蹬了兩下。就不動了。竹思寬連忙一看。面上漸漸變色。覺得陰中一股熱往外冒。便有些心慌。急將陽物拔出。低頭一看。血如泉湧。摸了摸。口中惟有冷氣。竹思寬酒也驚醒了。魂也嚇飛了。忙穿上衣服。開門出來。天已微明。走到書房。開了院門走出來。見大門已開。冒着微雪。迎着北風。一溜煙往家中去了。丫頭們睡到日高三丈。方纔酒醒。睜眼一看。此時雪已住了。日色滿窗。連忙起來。恐主母見怪。慌走過來。床上不見有人。回頭一看。主母光着下身。睡在火箱內。忙近前要替他蓋被。只見面如白紙。兩腿大揸。胯下鮮血淋漓。褥子上流得一窪。牝戶大張。尚津津血出。吃了一驚。推了幾推。不見動轉。伸手在身上一摸。已冰冷鐵硬。做了風流鬼去了\footnote{多銀被驢弄殺。火氏被如驢之具弄殺。蓋淫婦之報也。}。替他把被蓋上。兩三個忙收拾傢伙。一個跑出去說與衆家人。幾個僕婦都跑了上來。看見死得這個樣子。都不解其故。家中沒正經人。叫了個老僕到火家童家去報信。他父母已亡。只他哥哥火大夫婦來了。童自大自從那日在鐵家回去。心中自恨道。只爲貪了一口黃湯。做了這件壞事。宦哥連外人還不肯淫汚。我竟淫了內嫂。心中如何過得。又想道。這不是我去奸他。是他來奸我。我醉後無知。也還無大過。此後再不到他家去。聽得鐵家人來報說火氏死了。還疑是他弄了那一次之後。引動淫心。無處發洩。抑鬱而死。心中倒十分過意不去。那知他是這樣風流死法。同鐵氏到了他家。大家哭了一場。聽說死的這個樣式。都疑是急病暴死。決想不到被人弄殺。回回家屍首不停放的。即日殮了。請了老師傅同幾個滿喇嘛混念了一陣。擡去回回墳埋葬。忙寫信雇人去報鐵化。火大把妹子住的房門封鎖了回去。那竹思寬弄了一夜。洩了三次。也是虛飄飄一個身子了。吃了一夜大空心的酒。眼花頭暈。嚇得戰兢兢。迎風冒雪而回。受了寒氣。染成夾陰傷寒。頭疼肚痛。手足厥冷。遍身火熱。昏迷不醒。郝氏忙叫竹美請了醫生來看。吃了許多肉桂附子之類。總無濟於事。二來也有年紀了。身子又虛弱。又看見火氏死了。是他多年契厚。未免傷心。如何得好。郝氏又聽得有一個專治傷寒門的胡道貴。手段高強。特請了他來醫治。說。寒重了。不得汗。再不得好。藥力不濟。須要滾藥水蒸洗。方得汗出。這郝氏叫作病急亂投醫。便依他。他撮了一大包藥。燒了一鍋滾湯。將竹思寬脫光。拿塊板放在澡盆上。擡他睡在上面。四圍放上火盆烤着。他將滾水倒在盆內。一面蒸。一面用布蘸水。渾身淋水。略溫便換。那竹思寬如死人一般。絲毫不動不知\footnote{竹思寬竟是水火煉度一般。}。掇弄了半日。並無汗出。也不見他動展。再看。已嗚呼了。渾身的肉已燙了個半熟。剛是火氏三日之期。趕到陰司去與他做長久相知去了。鐵化在任所正然興頭。忽接舅子的信。云妻子病故。着實悲悼。要想回來。還捨不得空丢這項銀子。以爲內邊雖無火氏。外邊還有竹思寬可托。過了兩日。又接信。云竹思寬死了。家中要緊。只得吿病回來。丢了幾千兩銀子。只落了個半年的熱鬧。賺了個叫一聲老爺。還有個冠帶崢嶸。到家時。他舅子也來了。交付門上的鑰匙。開門進去。房中無人。想起火氏這幾年來頗有恩情。臨別那一種依依光景。今日歸來。音容已失。不覺痛心。大哭了一場。過了兩日。請了火大夫婦童自大同妹子去上墳。回來家又請了幾個老師傅並許多喇嘛。家中殺牛宰羊煠油香。做哈里哇。念了一日回回經。完了又往竹思寬家去弔孝。送了二十兩奠儀。不在話下。火氏背夫貪淫。即以淫死。理所當然。竹思寬負友奸妻。臨死雖燙得半熟。猶爲正寢。尚屬彼幸。不足盡其辜。鐵化交不擇人。致妻子如此。亦尖酸促恰之報也。人生世上。持身交友。可不愼諸。再說郝氏自從竹思寬死後。他年紀雖老。淫心較少年更勝。竹思寬在日。他那荷包口一般的牝戶。再沒十日半月不叫他揎一揎。今竹思寬死將兩月。不經此道了。心中時刻如有所失。意欲還要相與個老孤老。無奈白髮蒼蒼。皺紋滿面。不但兩手招郞郞不至。就是死命去拉。也未必有這樣高興的人來領敎。況且他的大名口碑載道。誰人還有那賽敖曹的物件來尋他。日間混着還不覺。到了上床之後。長夜迢迢。把那一段人肉放在心頭。時刻不能合眼。要叫竹美去買個角先生來聽用。一來這件事不好叫兒子去當買辦。二來這件東西是他少年間領敎過的。就是頭號巨物。也沒有竹思寬的粗大。料道不足以供行樂。急得那心似滾油澆的一般。那日竹美買了幾段香腸來家。他心中觸動。恍然大悟。就觸類旁通起來。叫竹美買了一根牛大腸並五斤牛肉來。他在房中將牛肉刴爛。把臟頭取了有尺餘長一段。把肉塞上塡緊。約有碗口粗大。用線紮好。他掂了幾掂。道。此時若用。似乎太粗。等風乾了。自然合式。吊在屋後簷下沒日色處。竹美夫妻看見。以爲他放着香豬腸不吃。倒灌了這根牛腸子。不知有何妙處。暗暗失笑。意思等乾好了還要些嘗嘗是甚麼滋味。郝氏每日眼巴巴望那腸子。求他速乾。過了十數日。那腸子漸漸縮小。粗只鍾口。長約一尺。比竹思寬的物件還略肥壯些。郝氏喜道。雖比他的大些。料道也還容得。再要狠乾了。未免太硬。過於小。遂取了下來。晚間到了床上。脫光仰臥。兩足大蹺。就拿那腸子對着陰門往裡搗。那裡進得去。他的牝戶只剩了兩片寬皮。那腸子粗了又乾的。硬幫幫的。連皮塞了進去。如何能入。用了許多吐沫。仍然不能送進。他急了一身臭汗。急出一個妙想來。下床拿脚盆舀了些熱水。將腸子泡濕了。他蹲在盆中。牝戶大張。也用水濕透。然後拿那腸子往內一塞。進去了小半。他就勢往下一坐。全然弄進。心喜非常。忙起來揩了屁股上的水。將那腸子夾在陰中。上床睡下。不住用手一出一進的抽。大遂其意。覺比竹思寬還強。因竹思寬後來有了年紀。雖粗大如故。不比壯年勇猛堅硬。大遜往日的形狀。這牛腸中肉是風乾了的。熱水一燙。漸漸發脹。又比竹思寬的粗長了好些。所以郝氏覺得更美。況且又離了兩月餘。復嘗新美之物。愈覺其樂。不住手搗了一會。內中固然快活。但年老了。膀力有限。酸痛非常。此時渾身已覺暢快。想到。且睡一覺。歇歇力。醒來再弄。恐睡熟掉出。把那腸子反往裡塞了塞。全送入陰門之內。將腿夾緊而睡。他因遍體通快。又費了些力。一覺直睡到五鼓。覺得有個東西在腹中。攻得心窩生疼。驚醒來。忙用手摸那腸子時。已不知何往。伸指頭往陰中去探。只摸得着。却拿不着。心中一急。越覺得往上攻。滿腹作脹。這是他臨睡時全塞了進去。及至睡着了。那氣往上一提。故此那腸子便抽了進去。他先用熱水一泡。後又被陰津一浸。那乾了的腸肉着了潮濕。又發脹如新。他的陰戶雖然出了揎。內中可禁得飯碗粗尺餘長的一件巨物。他此時也着了急。下地蹲在馬桶上。要想他掉出來。坐了許久。那腸子在腹中脹滿。如何得出。漸覺脹得難過。下邊阻住了。氣不得行。便往上攻。臉上如火燒的一般。眼中都冒出火來。急得沒法了。也顧不得羞恥。叫了財香來。吿訴了他。叫他想法取出來。郝氏上床仰臥。將兩手扳住兩足。牝門張得如鍾口一般。財香用指頭探探。也摸得着。但沒處下手。只得走出去向竹美說道。前日媽灌的那根腸子。我們只說他老人家要吃\footnote{是回回家上嘴吃的。不意他下嘴吃。如何能尅化。}。誰知他拿了當㞠子用的。如今塞了進去。攻得心疼。又勾不着。弄不出來。怎麼處。竹美驚道。這却沒有甚麼法兒。想了一想。道。你拿鐵鉗子放在裡面去夾。或者夾得出來。也不可知。忙尋了鉗子遞與財香。他走進來向郝氏說了。郝氏也急得想不出法兒。只得叫他夾。送了進去。腸子又大又滑。鉗子如何夾得住。東一下夾着〔肉。〕西一下也夾着肉。疼得郝氏亂叫。說道。這個法兒不好。你再想個別法。財香拿出鉗子。想了一會。道。我那一回小產。胎不下來。是楊奶奶伸手進去取出來的。我也學他取罷。郝氏此時覺得十分難過。便道。就是這麼。你快些救我的命罷。財香取了一碗油來。把手潤了。向陰中一伸。已進去了。手雖送入。那腸子已滑。手上有油更滑。左找找不着。右攥攥不住。越捏越弄了上去。直送到胸口之上。那郝氏也年老了。氣脈虛弱。看看顏色漸變。口中如牛喘一般。手足癱了下來。財香見局面不好。忙把手縮出。叫竹美進來看時。口中氣已微細。不多時。便入黃泉\footnote{不圖爲樂一至於此。}。他二人也哭了幾聲。忙替他把衣褲穿上。停放好了。竹美跑到鍾家去報了。錢貴聽得。親身來到。大哭了一場。問及是何病症。財香把這個新奇死法細細奉吿。錢貴聽他是這樣壽終。倒滿臉含愧。看着入了殮纔回去。還同鍾生來。上了個祭。送殯安葬。與竹思寬拼了骨。不贅。郝氏騷淫了一生。老年如此死法。雖說自尋的死路。也正是他好淫之報。竹美發送了郝氏。查點他的私囊。竟將二千金之蓄積猶存。滿心歡喜。同財香商議了一夜。次日。拿了三百兩。到江北尋着了黃金聚。要謀幹個小前程。黃書辦道。表叔表嬸去世。連百日還沒有過。你怎麼就想做這事。竹美道。趁着於今阮老爺賣官。有這條門路。若等我服滿。或換了官府。或者老表兄又不在這裡。就無望了。我於今若〔謀〕個官做。父母英靈自然歡喜。決不怪我。黃書辦見他這樣說。笑了笑。將他銀子吃起。向阮大鋮乞恩。說竹美是他的親表弟。求賣個前程。阮大鋮雖捨不得白放人去做官。但靠他拉撁。也掙了許多銀子。後來大事還要靠他。只得忍着心疼。假敍軍功。放了竹美一個錦衣衛百戶。竹美領了箚回家。公然到任。紗帽珏帶。大紅縐紗圓領起來。人人都知他是郝氏之兒。又是兔子出身。編了四句歌兒打趣他。道。

\begin{quotation}

而今兔子大軒昻。只爲襠中穀道香。

義父賭錢猶篾片。母妻俱是女邊娼。

\end{quotation}

竹美聽得。恬不以爲恥。到處以老爺自居。人見他還有幾個錢。無不奉承此老爺矣。國家之事至此。眞笑殺多少識者。嘆壞了多少義士。閒話稍住。且說鍾生在他家聞得樂公同劾阮大鋮。弘光不聽。有年紀的人了。着了氣。嘔了幾口血。又朝夕爲國事憂勞。食少事繁。搆疾而殂。鍾生不應馬士英之辟。杜門不出。不敢往弔。在中途設位祭奠。痛哭了一場。以盡師生之情。宦蕚偕賈文物童自大親到他寓處祭奠。樂公兩袖淸風。毫無宦囊。他三人共送千金薄儀。爲搬家回籍之費。鮑信到靈前大慟。親爲執喪。也送了奠儀一百二十兩。以報知遇之恩。到臨行之日。童自大親自送到浦口。贈銀三千兩與夫人公子爲安家用度。以報當日不聽劉弘之譖。護庇之德。那鍾生在家中終日鬱鬱不樂。對月臨風。惟有長嘆。錢貴代目百般勸解。他只張目不答。聞得人傳說。睢州鎭將許定國將興平伯高傑謀害。已往北走。史閣部在維揚。十分危急。你道許定國是何出身。他如何謀害了高傑。他係太康人氏。也是一員驍將。他初守河南。流賊突至。箭如雨射城中。定國站在敵樓以刀左右亂揮。箭皆兩斷。高與身等。賊射漸緩。他笑向賊將道。你乏了麼。你旣不能射。快去每人取一塊板來。好擋洒家的箭。賊將素知他是神射。果叫賊兵取了板來。賊將躱在板後。看他如何射法。定國以鐵枝箭連發數矢。將賊將釘死在板上。賊皆驚散。他常同衆人聚飮。衆人請道。聞公有神射。已見之矣。但公神勇。願借一觀。他應一聲。忽然躍起。兩手扳住簷椽。全身懸空。走長簷殆遍。色不變。他此時已七十多歲。以總兵赦罪出獄。鎭守睢州。毀家養士。他自以爲功高。不得顯爵。常輕高傑是流賊投降。反得封伯。每次上本。詆之爲賊。高傑後來知道。心中恨甚。常道。我若見彼。必手刃之。這時史閣部欲恢復中原。親自督師。厚撫高傑。命他統領本部將士兵馬爲前部。高傑到睢州。定國迎出數十里。在馬前跪接。高傑見他如此。下馬冷笑扶起。道。你是總兵大將。爲何也行此禮。到了營中坐下。問他道。你豈不知我要殺你。爲何不逃去。敢來見我。徐(許)定國叩首道。定國知公每常動怒。但不知我得何罪。高傑道。你屢屢上疏。稱我爲賊。還不是罪麼。定國道。因此定國不肯去躱。來見公也。定國目不知書。凡上疏皆是書記代寫。定國又一點文墨不知。不懂得疏中是何等話。若以此殺定國。眞是寃枉了。高傑道。你這書記在那裡。定國道。他自知有罪。聽得公來。逃去不知何往。定國不逃躱者。正要向公明此一事。非定國之意也。高傑是個粗直漢子。見他這樣小心屈服。倒反〖忄桀〗(憐)起他來。聽他這話。以爲眞實。定國標下有一員千戶。知道定國要謀害高傑。投上牒文。云定國謀公。高傑要以誠心待定國。將這千戶笞了六十。送與定國殺之。他遂同定國宰牲。約爲兄弟。定國裝飾了一個美女送來與高傑。高傑不受。笑道。軍行用此不着。你但養養。待我成功回來。以娛老景。高傑大營離城二十里。給王命旗一桿。付與定國。命懸在城上。傳令道。我兵非有令。不許擅自進城。違令者斬。定國請高傑進城飮宴。高傑只帶三百名驍騎。到了他署中。定國設宴燒燈。奏樂飮酒。叫他兄弟陪待衆將親兵在別所。婦女賓客皆雜坐。酒半酣。定國之弟動靜失常。高傑部將中有明見的。覺得有異。起身走到席上。附着高傑的耳道。今日之宴。看他兄弟志意非常。恐有詐謀。不可不防。高傑用手推開。道。你去。他如何敢萌此念。但放心痛飮。那員將見主帥如此說。也就不在意下。飮了多時。到了三鼓盡。三百人俱醉。俱就別所休息。高傑臥榻之前。只幾個小兒服侍。夜漏將殘。忽聽得房上歷歷瓦響。高傑心驚。出外看時。壯士踰牆越屋。已進來數十個。高傑急覓鐵棍。已被人偷去。遂奪了一桿鎗。力鬪多時。此時進來的人越發多了。腹背受敵。孤力無援。遂被衆人拿住。從去的三百個驍健盡被所殺。許定國南向坐下。道。三日來受你屈辱也盡了。你今如何。高傑大笑。叫道。我爲豎子所算。死何懼乎。大罵不絕。定國遂將他殺害\footnote{高傑雖死。還是個直腸漢子。不過失於粗鹵耳。如許定國。則不忠不義。大奸大詐之小人。誠所謂老而不死是爲賊。}。知道他大營人馬是邢夫人統領。素常聞名。知他的智勇。恐他來報仇。帶了親丁家屬。連夜潛逃往北去了。睢州一城的人聞知。都逃個乾淨。高傑有一名驍健伏於床下。得脫出城。詳細報與邢夫人知道。帶領衆將士如飛奔來。已是一座空城。邢夫人大怒。連累睢州二百里內居民。悉遭屠戮。史閣部到了徐州。初得這報。還不肯信。後聞果是眞實。痛哭道。中原不可復圖矣。回兵退守揚州。看看勢不能保。鍾生又聞得沿塘飛報。左良玉聞知崇禎太子自海上逃來。馬士英執意不認。誣是王之明假冒。在午門外拶拷。衆人雖知是眞。背地潛泣。俱不敢出一語相救。恐忤了馬士英之意。有人題了一首詩。大書於宮牆之上。內有一聯云。

\begin{quotation}

海上扶蘇原未死。獄中病已又奚猜。

\end{quotation}

合城人聲洶洶。馬士英也恐觸了公怒。暫且監禁。左良玉心中大怒。謂馬士英仇害先帝太子。欲淸君側之惡。率領重兵。自湖廣殺來。聲勢猛甚。士英將沿江一帶兵將。黃得功劉澤淸劉良佐等。悉調去上流迎擋。也有人勸他道。大淸兵馬南來。其勢甚銳。若將兵將全撤去。以堵上流。沿江一帶作何守禦。況左鎭並非背叛朝廷。不過欲救太子耳。馬士英大怒道。我寧爲大淸所殺。不肯爲左良玉所殺。衆人如何敢拗他。遂將各路兵馬盡行調去。一日。不知何人書了一聯在他堂中。云。

\begin{quotation}

闖賊無門。匹馬橫行天下。元凶有耳。〇〇〇〇〇〇。

\end{quotation}

鍾生聽了這些事。知大勢已去。心中朝夕不安。又聞知許義士髯樵叟二雪和尚三人的事。嘆道。髯樵叟無一命之榮。尚有魯仲連義不帝秦之志。許義士豈有官祿之榮哉。猶自國亡身死。何況我食祿數載者耶。我常恨近賊諸臣。若輩熟讀詩書。平居談忠說孝。臨難只圖富貴。我每每切齒。我今旣不能死。以負初心。愧許君髯叟多矣。若再不效二雪。尚戀戀妻子家園。以圖歡聚。不但爲名敎罪人。異日何以見先帝在天之靈同我祖宗父母於地下耶。浙中深山老谷甚多。我何不隻身遠避。做一個世外閒人。庶可以此心稍安。遂拿定了主意要去。且道這許義士髯樵叟二雪和尚是怎麼個始末。聽我一番細說便知。許義士名重〈如〉玉。吳郡長洲縣人。自幼穎異。六歲讀論語。至攻乎異端。問其師道。何謂異端。師云。非聖人之道。楊墨之敎是也。又問道。此方今日孰似。\endnotemark[7]師道。釋道二敎是也。他道。今之害天下者。此輩人耶。從此遂不拜佛。有人問他何故。他道。彼佛乃異端。我何拜爲。他日讀孟子。至能言距\endnotemark[8]楊墨者。聖人之徒也。遂慨然以道自任。深惡緇衣黃冠之流。說道。我異日爲政。必盡除之。以淸吾道。時有一僧。法名宗衡。與他父兄相善。嘗過其家。重玉見必變色。宗衡訝道。貧衲與相公無仇。何爲怒目相待。他道。汝輩聖賢棄倫常甘心異端。以亂吾儒。何謂無仇耶。他此時年僅七歲。宗衡微笑而去。久不至其家。父兄偶然相遇。叩其故。宗衡笑道。君家有聖人。吾輩異端。當自絕。因述其言。聞者大異。十三入庠。於諸生最少。然有老成氣度。同學數十輩。多敬之。弱冠補廩。聲譽益沸。讀書必求精義。不事呫嗶。嘗向人道。學者稽古。當探聖賢心髓。而務身體而力行。以復其天性。否則無益也。父母死。六年之喪。未嘗一日輟哭。亦未嘗入寢內室。思慕久而愈切。聞崇禎駕崩。即遍書崇禎皇帝四字於裡\endnotemark[9]衣衰絰。悲號誓死。家人勸道。君一介書生。非有官守之責。可以死。可以無死。死傷勇。聖賢所不取也。重玉嗔目叱道。君安天下。以生我臣民者也。生我臣民。天下之父母也。焉有父母爲賊所害。而爲子者尚可苟活乎。夷齊餓死首陽。豈有官守穀祿者乎。不過欲全大節於一身。明大義於天下也。況我已食廩。食人之食者。當死人之死。吾志已決。\endnotemark[10]毋煩多喙。乘間投閶江。家人奔救起。乃不食八日而死。\endnotemark[11]髯樵叟失其\endnotemark[12]籍。亦未詳其姓名。因其美鬚髯。旣善樵。而年最高。故人皆呼曰髯樵叟。身長八尺餘。多膂力。每負薪三百斤貨於市上。止索百斤之値。人怪之。問其故。他答道。人之力均負百斤。我能力負三百斤者。天也。我寧敢邀天之功。以爲己力哉。邀天不祥。利己不善。皆惡德也。人生天壤間。不能履德。可蹈惡乎。人皆笑以爲迂。每淸晨必負薪入市。貨薪必沽酒痛飮。放歌以歸。日日如是。午後則採薪洞庭山中。人跡罕到之地乃入。人又怪問之。他道。我力多。合遠採樵。彼等力少者。應讓之近地也。初夏。山中人沸聞得闖賊陷京師。崇禎殉社稷。賊已改元永昌。髯樵叟聞知。搥胸長號。道。我向知天子姓朱。何忽換\endnotemark[13]姓李耶。良久道。賊何可爲我天子乎。遂痛哭三日。投震澤中而死。二雪和尚名行幟。\endnotemark[14]族姓林。其先福建莆田人。\endnotemark[15]始祖遷浙之瑞安。和尚天性至孝。弱冠游庠。萬曆乙卯舉於鄕。崇禎戊辰成進士。與鍾生是同年。初任湖廣蒲圻令。庚午癸酉兩科分房楚闈。俱稱得士。三年循良之聲上達。擢翰林院編修。在朝與黃道周倪元路諸君子最深契。未幾。特遷東宮講讀。時國事日非。言路壅塞。乃進易卦講章。隱爲諷諫。觸當道忌中。以他事降三級。於是公論不平。掌院黃景昉。冡宰李日宣。皆抗疏請復。遂晉\endnotemark[16]侍講經筵。兼起居注。尋轉少詹。他終日勤勤懇懇於章句之間。冀得一格君心。反亂爲治。奈天命已移。闖賊犯闕。國破君亡。惟有仰天長號。搥心泣血而已。闖賊逼他從順。酷刑幾斃。終不肯屈賊。後遁脫難南還。與史可法共圖國事。時馬士英當國。素知其才。數召見。與語多不合。二雪心知必敗。日夜憂之。史閣部薦以禮部起用。二雪識不能容。遂稱有疾。固辭旋里。未幾。又以內閣徵用。二雪知大事已去。乃就呂峰逾尊長老。剃度爲僧。鍾生聞知他三人的事跡。想道。我雖不能效許義士髯樵叟。何不學二雪去逃禪。或儒或道。潛踪遠遁。主意決了。旋製了籜冠布氅。麻履絲縧。一副道裝行頭。打點停當。遂對妻妾姪兒說道。我看這光景。京城不能不(留)矣。我去尋一個避身之地。再來接你們同去。錢貴道。端的往何處去覓地。幾時歸來。鍾生道。我隨步覓去。却定不得地方。歸期也定不得日子。你們但好好在家度日。一有去處。我就歸來。又向鍾自新道。我見你諸事老成。不用我多囑。此時他大兒子鍾文已十六歲。次子鍾武十四歲了。對着他二人道。我像你們這樣大時。久已無父母了。你兩個可聽母親敎導。哥哥管訓。立志上進。勿墮家聲。衆人見他雖說回家。却又都是不回來永別的話。再三哭勸苦留。他那裡肯聽。瞞了衆親友。只帶了一個小童。自己換了一身布衣。命小童着了一襆。悄悄步出了通濟門。家人一個也不許送。他到了城外。雇了兩匹騾子。踽踽而去。宦賈童同衆人得了此信。都來探問。差人四處找尋。並無踪跡\footnote{此處將宦賈童一提。從此接去矣。}。再說那鍾生主僕二人。策蹇到了丹陽。搭船直氐(抵)虎丘。店中住下。他向那小童兒道。我前日出門。一時匆忙。忘帶盤纏。你可回去取來。我就住在此處等你。那小童兒也信以爲實。就搭船去了。到家見了主母。把上項話說了。錢貴〈錢貴〉疑心道。帶了盤纏去的。如何說這話。叫了鍾用。交與他銀子。同小童星夜趕到虎丘。鍾生已不知何往。去問店主時。他道。只住了一夜。次日就不知往那裡去了。鍾用遍尋了幾日。杳無踪跡。只得歸家報信。合家聽了。不知是生是死。痛哭了幾場。鍾自新要去尋叔叔。錢貴不肯。道。你叔叔已是安心避去。必不在塵寰近處。浙江一路深山窮谷甚多。知道往何處去尋。況你兄弟又小。無人照管家務。你如何去得。他見說得有理。只得在家。但時常想起叔叔的恩情。便哭一場。錢貴代目並他二子。不知淌了多少眼淚。過了十多年。鍾家一個鄰舍。叫做金德性\footnote{鍾生救小狗子時即有此人姓名。不過以爲隨手編一姓名。爲小狗子得父母之消息耳。不意伏到此時。謂鍾生一去十多年方得信息。編書原要首尾相照。貫串得宜。閱者方不釋手。}。\endnotemark[17]往浙江臺州府去探親。因慕雁宕\endnotemark[18]之勝。到那裡去遊賞。偶見老僧岩下有一間茅庵。進去歇脚。見一道人在裡面獨坐。見有人來。也就起身讓坐。却不交談。金德性覺這道人好生面善。目不轉睛看了一會。猛然想起。道。這人酷像鍾老爺。他出來了十多年。原來在這裡出家。猶恐怕不是。不住的仔細端詳。那道人道。居士爲何只管看我。金德性聽得聲音更熟。忍不住問道。你可是鍾老爺麼。那道人笑道。旣是鍾老爺。他如何到得這裡。金德性道。鍾老爺雖離家十多年。我是緊鄰。認得很熟。尊面相似得很。只是反豐嫩了些。那道人笑而不答。金德性注視良久。越看越是。暗想道。他形貌雖然略少。而聲音不能改變。定然是他無疑。遂站起說道。老漢同老爺一牆之隔。住了多年。常常相見。豈有不認得之理。老爺何必瞞我。鍾生見他認破。也立起笑道。高鄰。你好眼力。我便是鍾麗生。拉着他的手讓坐下。金德性道。自老爺出來之後。府上奶奶相公至今想念。老爺難道就不憶念家鄕麼。鍾生笑道。我已棄家爲方外野人。復何記念之有。金德性道。老爺這些年在何處居住。今何孤身在此。鍾生知他是個盛德老實人。也將數年所歷之處細細相吿。天色將暮。鍾生道。日已銜山。老丈請回貴寓。此地不堪留宿。明日再來相晤罷。金德性也就辭了回寓。次日早飯後。又到庵中來。只得一間茅屋而已。內中已空空如也。一絲他物皆無。正合了古詩二句。道。

\begin{quotation}

又被世人尋討着。移家不免更深居。

\end{quotation}

那金德性嘆息了一會。也還在左近訪覓了兩日。並無踪影。知他又遠避去了。後來回到南京。把這信詳細說與鍾家。錢貴大家又哭了幾場。鍾文鍾武此時俱已婚娶。定要去找尋父親。鍾自新也要去尋叔叔。錢貴起先不肯。道。你們雖去。決定尋不着。就僥倖尋着了。他也定不肯回來。你父親叔叔的天性。可是肯做馮婦的麼。他弟兄三人見錢貴不允。終日號泣。〔錢〕貴叫他們到跟前。說道。我豈不願你們去見一面。但恐空費跋踄。不能相會。徒勞往返。也就哭起來。道\footnote{妙筆入神。不叫他們去者。是深知鍾生。然而夫妻之情。豈不記憶。焉有不哭者。情節肖然。}。你們旣如此思慕。我安忍阻你們的孝思。鍾武在家罷。你兄弟二人同去。尋得着。尋不着。要早早回來。不要叫你母親同我在家倚門懸望。鍾武道。同是父母遺體。大哥哥是姪兒。倒還去呢。我難道不是兒子。我定要去。鍾用也哭稟要跟了去尋主人。錢貴只得都依了。他們收拾一肩行李。帶些途費。星夜去了。到了雁宕。尋了半月有餘。杳無影響。訪問附近居人。皆云不知。三人恐母在家懸望。號哭而返。到家說了備細。鄂氏錢貴代目合家大小又哭了幾場。你道金德性遇見鍾生。他緣何到了那裡。他當年在虎丘店中哄那小童回去之後。即改了道裝。次日就泛海到了崇明。地僻海陬。住了月餘。來遊江陰。賞澄江風景。見城西白石山幽靜可居。自號白石山樵。復返儒複(服)衣裳。訓徒自食。大淸天兵南下。維揚失守。史閣部自刎。弘光聽知這信。也不與衆臣商議。同了十多個內監。十數個宮嬪。共三十餘騎。半夜開城向采石而遁。數十里外即爲我兵所獲。次早宮門大開。宮娥內豎紛紛逃散。百官進朝。方知聖駕已蒙塵在外了。正是。

\begin{quotation}

九重尚有逃天子。朝內焉無遁大夫。

\end{quotation}

大家一哄而散。先是。韓贊周養子李國輔提督勇衛營。操練禁旅。盡心爲國。馬士英奏弘光。遣彼往浙江開礦。奪其營篆。把他那呆兒子馬台改名馬錫。提督營務。以此呆物綰兵柄。時人無不笑罵。馬士英年前特往貴州。調了數百苗兵來京。充當禁軍。他此時帶領。將他妻子蹇氏假充太后。同着家眷。向浙江逃去。浙人登城詬罵。閉門不納。只得逃往福建。因家貲重了。不能速行。那些五百兩一個的大元寶雖不能帶。尚有數十萬零碎之貲。日行十數里。過了仙霞嶺。那時鄭芝龍正在閩中猖獗。他聽了這信。遣將領兵。中途邀截。馬士英夫婦。同那呆子馬台。假孫馬加盧。皆死於兵刃之下。媳婦香姑同他的妾婢。皆被衆卒搶去。不知所終。一生宦蓄悉爲賊有。那阮大司馬更是在行。纔聽得淸兵一到。即匍匐營門拜降。營內諸公久聞他有燕子箋。雙金榜。獅子賺。春燈謎諸劇。問他能自度曲否。他欣然即起。執板蹬足。唱以侑酒。無恥到這個地步。他更算計的妙。想脚踏兩頭船。做兩朝的功臣。一面投順了我朝。一面着人私通隆武。後大兵追隆武。到贛州擒獲。在文書箱中收得阮大鋮密本。差兵擒拿。他正在中首獻花岩飮酒撥悶。聞得此信。自上投下。頭顱粉碎。骨肉如泥。阮大鋮向日曾以私〖阝少日小〗殺雷縯祚於獄。此日早間忽見寅(縯)祚以斧擊其腦。大鋮頫手道。介公饒我。介公。縯祚之字也。他因心悸。故出外閒遊。是日果碎腦而死。有幾句贈他。道。

\begin{quotation}

上臨之以天鑒。下察之以地祇。

明有王法相繼。暗有鬼神相隨。

行凶畢竟逢凶。恃勢終須失勢。

勸人自警平生。可嘆可驚可畏。

\end{quotation}

他自阮最阮優死後。並無餘子。此時毛氏也花甲初度了。也不想立嗣。着擁重貲。同龐周利朝夕行樂。別的妾見夫人如此。都效顰馬氏當日所爲。都各相厚了個健僕逃去。莫知所往。後因阮姓族間衆口嘵嘵。毛氏無奈。方繼了一子。當日阮大鋮在日。毛氏雖同龐周利常常作樂。還不過是鼠竊狗偷的事。自阮大鋮死後。他無可畏之人。竟大張旗鼓。日夜叫龐周利到上邊。如同伉儷。他愈老愈淫。夜間弄了不算。日裡還要找零。龐周利雖是一個壯年。當日偶然應差還不覺。如今要日夜應付起來。如何有此力量。又恐失了主母之歡。他有同盟的三個家人。一個叫盛〖考口〗。一個叫司敷\footnote{二名前已見過。}。一個叫楊壯\footnote{此係新見。}。都知他是主母的嬖幸。常常求他介紹。龐周利一則不負衆人之托。二則實有些支撑不來。要薦賢自代。一夜正同毛氏幹着。趁毛氏歡喜的時候。說道。小的有一句話要說。奶奶不要見怪。方敢開口。毛氏將他摟住。親了個嘴。道。怪奴才。我同你的恩情像夫妻一樣了。有甚麼話不許你說。還捨得怪你麼\footnote{一部書中。淫婦甚多。有醜如毛氏者乎。恨阮大鋮不知耳。}。龐周利一面抽。一面笑說道。小的蒙奶奶的恩。粉身碎骨也報不盡的了。但小的覺得近來的力量不能如當日了。恐怕服事得奶奶不遂心。小的想要薦舉兩三個人同來服事的意思。不知奶奶的恩典可要麼。毛氏聽了。歡喜得了不得。假說道。我看你的本事還好。況且我同你這樣相厚。怎好又要別人來的。你且說你要薦誰\footnote{語語是不要之要。妙。}。龐周利道。這是小的無可報恩。出自小的的一點孝心\footnote{好義僕。非阮大鋮這樣忠臣家不能有。}。俗語說。船多不礙港。不要說小的薦來服事奶奶。就是奶奶此時要叫人來服事。小的還敢爭說半個字麼。小的薦的是自家家裡的三個。就是盛〖考口〗司敷楊壯。他三個年輕力強。可充此任\footnote{此謂毛氏愛龐周利勝於苟雄。以之爲私夫。爲其陽壯耳。細閱方明。大有趣甚。}。小的看他三個的漢仗力量都好。即下身的東西。只有強似小的的。惟盛〖考口〗的。比當日苟雄的還旺個半寸。不瞞奶奶說。當年小的們大家往䘕衏裡去打釘。都曾比較過。說得毛氏心花都開。摟住他不住親嘴。笑嘻嘻的道。我的身子已是你的了。你說的話。我還有個不依的麼\footnote{眞可謂納諫如流。}。只管叫他們來罷。龐周利道。奶奶這樣施恩。他們感恩不盡了。憑奶奶吩咐。先叫那一個來服事。毛氏道。哎喲。你旣舉薦他們一場。要叫。少不得都一齊叫了來。若分個先後。不要說他們說我的恩偏。還要說你待他們的意有厚薄呢。龐周利道。奶奶恩典。旣這樣說。小的明日晚上同他們一齊來。毛氏聽說他三個人雄壯。盛〖考口〗還勝似苟雄。心中火發。恨不得此時就到跟前。嘗嘗他們的滋味如何。那裡還等得到明晚。忙道。於今老爺已去世了。幾個小老婆都去了。過繼的小相公在外邊。又不上來。只這幾個丫頭。都是我的心腹。又都是你弄過的。還怕甚麼。一家就是我大。誰還管得我。你明日吃過早飯。就帶了他們來。二人幹夠多時歇了。次早。龐周利下去。毛氏叮囑他早些來。龐周利應諾。尋着他三人說了。皆喜不自勝。都打點精神服事主母。毛氏忙忙催飯吃了。坐在一張花梨木八步床上。斜靠着枕頭等候着。龐周利同他三人一齊到房中。他三個忙跪下叩了個頭。起來望着毛氏嘻嘻的笑。毛氏也微微含笑。這日他三人都幸。毛氏試過。興也十分足了。身子也軟癱了。此後或輪流服事。或四個齊來。也弄了幾年。毛氏年將古稀。淫性猶未倦。却也漸漸乾枯。骨瘦如柴。白〔髮〕蓬鬆。渾身如雞皮皺一般。一個牝物越發癟塌不堪了。陰毛比當日更長更多。不黃不白。甚是難看。他四人貪主母之賞。少不得竭力以奉。毛氏一日偶染了病。飮食減少。奄奄一息。日夜還要他四個齊攻。那日大白晝。他四人正輪班同毛氏大弄。弄了數次。只見他哼了兩聲。四肢癱於褥上。雙眼緊閉。龐周利忙摸他嘴鼻時。只有微微冷氣。已吿終了\footnote{毛氏之淫安得治。竹思寬之胡搗鬼。用藥水燙熟而死。始快人心。一部書之淫事以毛氏作結者。極寫其淫態之極。較諸人猶勝耳。}。他四人慌了。忙各穿衣下床。將他的箱櫃偷開。把阮大鋮在生所積的官貲。各捲千金之物。一同逃去。丫頭們過來。見毛氏死了。忙報知他那螟蛉之子。追問毛氏死的原故。丫頭們隱瞞不住。只得細細供出。那螟蛉即尋他四人時。已不知去向。意欲報官。恐拿着了供出前事。醜聲揚播。只得罷了。開喪出殯。將毛氏同阮大鋮合葬了。阮大鋮作孽一生。落得一家如此而已。古語說。世間壞人。遠報兒孫。近報自己。試看阮大鋮馬士英兩家。奸邪誤國。到今日身死嗣絕。貽笑千古。豈不信乎。再說龐周利四人盜了重貲。直逃到江西地方住下。恃着囊有餘物。終日〖貝票〗(嫖)賭。不上數月。空空如也。他們赤手空拳。就入了江洋大盜的夥內。後被官軍擒獲。皆戮於市。亦可謂惡奴之報\footnote{他四人朋淫主母。其罪應磔。因毛氏不成主母。故罪減一等。此書中之報應。皆有輕重之分。}。再說弘光逃後。衆文武官見他一個皇帝。棄天下如敝屣。他們這一頂烏紗能値幾何。各擁着嬌妻美女。白鏹黃金。一閧而散。並無一個死節之人。只有一個乞兒。氣憤不過。題了二十八個大字在文廟照壁之上。投入泮池而死。題道。

\begin{quotation}

三百年來養士朝。如何文武盡皆逃。

忠良留在卑田院。乞丐羞存命一條。

\end{quotation}

鍾生聞知。撫膺嘆道。朝廷高爵厚祿。以養此輩。臨難不如一乞丐。竟做如此散場乎。常常淚下。這白石山中居人。曩不知書。皆業農樵。鍾生居數年之後。樸敎子弟皆嚮學。能文章。後明經者十餘輩。鍾生不愛交游。惟與東山笑和尚相善。往來無間。這笑和尚不知何處人。語似楚音。忽來瓢子崗。寄棲一座破大王廟中。捆履爲食。不乞化一文。人有與之者。笑而弗受。入市賣履。口不二價。他從不肯輕與人言。見人輙笑。人問之。則大笑不止。常山谷獨行。則鼓掌高笑。或臨池獨立。每顧影自笑。捆履之暇。或仰天長笑。或倚風豪笑。虛庭獨立。或啞然冷笑。或莞爾微笑。卒然或壺盧大笑。舉止未嘗輟笑。故鄕村男婦老幼都呼他爲笑和尚。每入市。市中羣小兒因他好笑。皆拍手喧笑。擁繞大叫。笑和尚來了。和尚也喜與羣小兒歡笑。相與大笑不休。常同鍾生危坐空山。終日作耳語。語畢。輙相視大笑而散。和尚有一厚友叫做哭道士。也不知何處人。來江陰席冒山。蓋了一間茅屋獨居。冬夏戴一籜冠。麻履入市求食。人與之。必北面再拜而祭。祭必哭。哭必哀。人問其故。哭而不答。固問之。則放聲大哭。起初人皆怪異。後皆識其誠。每入市。人都道。哭道士來了。爭與之食。食必祭。祭必哭。哭罷。誦黃庭經以報之。笑和尚一日邀鍾生去訪他。到了廬外。道士方陳芋粟在中庭哭祭。哭聲極哀。鍾生和尚聽得傷心。亦欷歔泣下。兩人在扉外佇立。等他事畢。候了許久。他哭愈勁。而聲愈慘。鍾生與和尚也掩面大慟\footnote{笑和尚已哭矣。}。日暮。道士哭休。二人叩門。拭淚入見。道士即獻茶。祭品共食。和尚說起適纔聞他哭時。我二人也不禁傷心悲慟。不想觸動了道士的心。又復呼天號泣。悲慘動地。鍾生和尚亦皆潸潸淚下。相對達旦。於是三人遂成知己。道士善哭。每於風雨臨花。月明繞樹。或雲紉遠嶂。雪滿空山。莫不對景悲哀。椎心泣血。聞者莫不酸鼻。然不知他是爲何故。又年餘。道士辭別鍾生。攜手痛哭。往終南而去。次年。笑和尚也要別去。鍾生挽留不住。乃握手大笑而別。並不知所之。鍾生見他二人去了。無可爲伴。也想他遊。意尚未決。不意城中有許多人紛紛來尋鍾員外。他恐露了形跡。也飄然去了。你道城中人如何知道。內中有個緣故。那時江陰有一個杲頭陀。字劍庵。倒不知他的俗姓。天性端慤。幼孤。事母至孝。身長八尺餘。力能舉鼎。每食。粟一斗。肉十斤。酒一斛。家貧。力作奉養。日以草帶束腰。忍餓以給母。嗜學。晝則耕。夜則讀。每達旦不寐。三十成文章。工書法。下筆數千言立就。補邑博士弟子員。每試輙奪第一。里中弟子皆豐束脯。從學舉子業。於是始獲飽餐。後母亡。遂爲僧。隱居城南陽武墩。參心學。得某知識記莂。然無叢林氣習。風流瀟灑。常芒鞋草笠。獨步山中。拉樵夫牧豎話古今興亡事\footnote{這叫做對牛彈琴。}。樵牧不懂。欲謝去。杲則把其袂。必語竟而後釋\footnote{杲豈不知不可與言而與之言乎。或者謂衣冠中人不足與語。不若向此輩言之。}。初。邑南境地高。不通潮汶。田家必藉山谿暴水始得稔。若經旬雨水流不迭。則苗腐。經旬不雨。土壤燥裂。則苗槁。多歉少稔豐。多貧困。皆鶉衣草食。杲深憐憫。捐貲募工鑿溝。澮濬溪港。建閘啓閉。旱則儲水各渠。潦則注水入江。由是數里瘠壤皆成膏腴之地。常向人道。大丈夫不能致身廊廟。爲國家建不朽之業。居一鄕。則當爲一鄕立奕世利益\footnote{此話只可爲富者道。貧者難與言也。}。若誘愚夫愚婦修齋誦經建廟鑄像爲功德。不特有干名敎。抑且獲罪佛祖\footnote{此語近日和尚見之。不但謂之反敎。且以爲敗類矣。}。大負天地生我之意。故雖受臨濟衣缽。未嘗踞坐說法。操疏募緣。一年。値歲遭饑荒。里中富室每患剽竊。杲一夕獨立要道。候羣盜來。遮謂之曰。我劍庵和尚也。大衆識之乎。大衆不過爲饑寒所逼。聊以自救。所謂夜裡大人是也。赤子之心原未絕滅。何可久迷不悟。今有稍贈君輩。持歸各理生計。毋爲此齷齪事。上辱祖宗。下羞子孫也。羣盜皆棄杖羅拜。道。願奉敎。杲袖中取出白金以贈之\footnote{倒應虧朱提之力。若無此。杲雖千言萬語。終屬徒饒。}。此後衆盜悉改爲良民。那時江邑賦重事煩。歷來令二堂出入。俱以廣福寺鐘鳴爲度。早政聽訟。曉鐘動即出堂。午政催科。暮鐘方息入休。不然。則政多廢墜。寺鐘忽屢日不鳴。令怪之。呼司鐘僧詰問。對道。連夜忽有妖物盤踞鐘樓。僧每登樓。則擲石如雨。不得上。以故失更。實非僧過。縣令怒道。爾等多飮醇酒。沈醉所致。何得以妖物支飾耶。笞而遣之。是夕。鐘仍不鳴。明旦復召僧來詰責。僧泣訴妖狀甚張。令益怒。限今夕不鳴即置爾死\footnote{好胡途(塗)知縣。前笞猶可者。或以爲貪飮失誤。此謂明知是妖矣。不敢奈何妖。而欲處僧。此等官宜爲狐所侮之得耳。}。僧懼歸。泣吿住持。住持道。我聞劍庵大師乃得道者。汝速往求之。或可除也。僧遂走吿。杲道。能擲石拒人者。必狐也。狐性嗜雞。最忌梧子油。可以梧子油炙肥雞置樓下。彼聞香味必來取啖。啖則必大吐。吐則神散力憊。僵臥不能動。乃可縛也。俟其說誓乞命即釋之。萬不可殺。殺則羣狐必來索命。禍難解矣。僧如其言。果獲一狐。黑毛九尾。狐被縛。怒道。吾通神狐也。吾自得道以來。橫行大江南北。無敢攖者。至江靖兩邑城廓間。所懼者惟三人耳。爾等何人。輙敢擒我。衆僧問道。三人爲誰。狐道。東郭村學究單。城南劍庵和尚杲。白石山刑部員外鍾。除此三人外。我皆得而侮之\footnote{不但諸生聞之當愧殺。即縣令聞之亦當愧殺。}。僧道。吾奉杲頭陀命。汝奈何。狐道。若是。我當遠避。毋爲君子棄也。吾誓不禍汝。從此逝矣。衆僧縱之去。同走訪單學究。乃皤然老翁。七十餘矣。將狐言相吿。且詰其生平。學究道。我一生無甚好處。但敎授五十年。未嘗一日稍怠。待生徒。貧富無二心。與人交接。無欺誑之念而已。此時轟傳得合城皆知。有些文人墨士。素聞鍾生之名者。紛紛到白石山來訪鍾員外。四處訪問。並無其人。村中有幾個老誠有識的。疑心道。我們這裡那年來了個先生。不說姓名。自稱白石山樵。想就是甚麼鍾員外埋名隱姓的罷。衆人就到他館中來探問。鍾生問其故。衆人把老狐的話相吿。鍾生道。請問這鍾員外他爲〔何〕到這裡來。今在何處住。衆人道。因爲不知。故此特來奉問先生。鍾生笑道。我一個敎書糊口的人。何以得知。衆人雖散去。都疑心是他。無一日沒人來問。鍾生恐或有人識出。遂辭了衆門徒出來。聞得人說邑中有一個張顚。每日雞鳴而起。即詣山谷痛哭。大呼崇禎皇帝數聲。日出乃返。風雨不輟。往訪之。這張顚名印頂。字大育。幼明辯。博學工詩。善鼓琴。又工擊劍。然不挾劍。每酒酣興發。持雙葦或柳枝狂舞中庭。如梨花亂落。紫電交馳。令人目眩。天性忠義。甲申傳聞李賊弑帝。一慟即成顚疾。常號泣狂走於市。或裸體悲歌於道。人多惡之。乃移家定山雲停里。自署其門道。

\begin{quotation}

山定人隨定。雲停我亦停。

\end{quotation}

鍾生訪着了他。亦實吿其始末。相攜大慟。一見如故。款留數日而別。又問陳顚夫之名。要訪覓一晤。竟不知其所往。這陳顚夫字樂山。名景。性豪俠。倜儻不羈。崇禎末年。中原流寇猖獗。顚夫憤之。\endnotemark[19]盡變家產。渡江募壯士五百人起義。與河南巡撫朱明合軍大破賊於柳園。生擒賊首八斗糟斬之。旣而朱明以讒去。援師不繼。且食盡。遂散壯士歸。乃漆八斗之頭顱爲酒器。大會親朋。酒至客前。必令大罵逆賊者三。然後飮盡。如此者七晝夜。此後或住或去。踪跡莫定。鍾生訪問數日。不得一遇。有人見他行藏異人。知他是個埋名的高士。說道。陳顚一時那裡便覓得着。四明有個萬履庵。也是個義士。他是總不出門的。一去便可相晤。鍾生即往四明去相訪。原來這萬履庵名泰。自幼穎悟絕倫。凡書寓目即成誦。垂髫即有文名。鄕士大夫皆矜詡之。舉諸生。以端方稱。性孝友。內外無閒言。閉戶求天人之學。終日危坐。靜思聖賢克己復禮的工夫。卒悟心性本原。故其詩文多自出性情。不事雕琢。無斧鑿痕。不蹈浮華。絕煙火氣。讀之者蕭蕭然。兩腋若有淸風來。吳越學人一時翕然。多宗之。然尚氣\endnotemark[20]節任俠。無腐頭巾氣。與人以誠。雖田夫牧豎。必推心置腹。里巷有犯之者。多不與校。及一旦有急。已忘其懟。即殫力拯其危。傾囊周其困。性雖耿介。然接人甚和。與之處。油油然如坐春風中。即最猥瑣之夫。一望見其顏色。鄙吝頓消。傲僻全捐矣。思宗崩。即棄家野服。築居水中央。自署其門道。

\begin{quotation}

有天不戴逃方外。無地堪依住水中。

\end{quotation}

鍾生尋到他住處。將來歷向他家小奚說明。履庵自駕小舟迎諸水滸。共載而歸。悲歌十餘日。鍾生辭別。復親自棹送十數里始返。鍾生由浙江出江西饒州府到豫章。偶遇着一個姓蕭的主人。與語投機。定要留鍾生到他東山鄕中。訓他子弟。鍾生此時又改了名姓。姓金。名生。取了姓的半邊。字下的一字。蕭家子弟十數人皆從受學。一日。蕭家有子弟畢婚宴客。那時他村中有一個巫人。善用妖法。里人事之甚謹。稍有忤觸。禍必立至。每宴會。必奉以首席。鍾生此日以師道自居。並不遜讓。竟自坐了。這妖巫心甚怒。數以言語侵犯鍾生。鍾生恚甚。厲聲叱之道。爾何物宵人。敢與正人君子爭坐次耶。那妖巫亦怒。忿然作色。出不遜之語。二人幾次犯言。衆人勸開。皆不歡而散矣。衆弟子輩恐鍾生爲其所害。備述其素常凶惡。今夜妖必致禍。因備籃輿。請鍾生遠避三十里可免。鍾生笑道。妖不勝德。邪不干正。理也。吾雖不德。然自揣生平無自欺者。妖何能爲。弟子堅請。鍾生弗從。弟子知鍾生精於易。固請筮之。得輿尸凶象\footnote{不意此象應在妖巫。}。鍾生道。我姑備之可耳。命諸弟子藏匿他舍。鍾生於齋中用沙畫八卦繞几。秉\endnotemark[21]燭焚香。硏硃點周易以俟。夜闌。忽聽空庭似落葉聲。果有一人乘斑斕大虎從窗櫺中進來。狼首豹眼。披鎖子甲。持方天㦸。忽長一丈。繞卦疾走。鍾生毫無懼。以點易硃筆投之。應手而倒。忽然縮小。鍾生近前拾起一看。乃尺餘長紙剪的形狀。拿來夾在易經中。久之。又聞牖外寒風蕭蕭。一人藍面赤髯披髮。持着斧。跨白象。排闥而入。馳繞卦外。却不能進。鍾生又拈筆擲仆。檢視。也同前番一樣。乃紙所造者。亦夾在易經中。少頃。復有一人。牛頭兩角。騎黃毛獅子。黑盔皀甲。提偃月刀。直入內室。環繞三匝。控\endnotemark[22]勒向鍾生口吐火焰。直逼衣冠。鍾生凝神危坐。端然不動。所乘獅子張牙舞爪。作搏噬狀。四外皆啾啾鬼聲。那妖見鍾生不睬。掄刀作擊刺之勢。鍾生又以筆投之。豁然仆地。作呻吟之聲。半刻乃息。視之。仍紙剪者。拾起同夾在一處。不多時。雞旣鳴矣。東方漸\endnotemark[23]明。衆弟子趨來問候。見戶牖大開。鍾生尚明\endnotemark[24]燭端坐。問道。先生夜來曾見甚妖異否。鍾生詳細吿之。將三個紙剪與他們看了。仍夾於書內。弟子們都吐舌變色。鍾生令掃除屋內。然後上床高臥。不多時。有一老嫗號哭而來。在門外求先生饒命。衆弟子出去問他是何故。老嫗道。我丈夫不道。昨與先生相忤。夜間攝了親子的魂爲魅。來魘先生。不料皆被執下。今收魂不返。三子殆將斃矣。乞轉達\endnotemark[25]還三紙。願送千金爲報。弟子入對鍾生說了。鍾生道。我正欲絕其妖種。以除一方之害。豈敢還彼。衆弟子道。還彼可得千金厚贈。何樂不爲。鍾生笑道。我豈是貪財之鄙夫耶。執意不與。那妖巫的三子即日俱斃。妖巫不數日亦慚忿而死。鍾生復購得其妖書焚之。遂除了一害。人漸聞名。都來拜訪鍾生。鍾生恐被人識破。又辭了主人。復回浙來。要入天臺山覓一隱居之地。那一日到了嵊縣旅店中。遇一老人先在店內。見他鶴髮童顏。虬髯碧眼。鍾生奇其狀。知非庸流。殷勤詢其居址姓名。那老人道。老朽姓胡名佐。字良弼。天臺人也。亦詢鍾生何往。鍾生對以欲往天臺覓一隱地。老人道。天下不若雁宕之可居也。雁宕深邃可隱。君可卜居於彼。但彼處地僻人稀。恐一時口糧不繼。枵腹奈何。老朽有一方。君可依方合之。倘菽水缺乏之時。含一丸於口內。任食百草木葉。可以無饑矣。雖不能辟穀。可免饑餒之患。鍾生大喜道。倘蒙長者見賜仙方。我當傾囊以報。老人道。吾非利徒也。且有求於君。如君首肯。我尚有相報之處。如其不許。命也已夫。鍾生道。長者意若何。請試言之。老人道。祈君今夜活我老朽一命。不知肯垂慈否。倘不見憐。非敢請矣。鍾生道。我平生尚俠。趨義如歸。苟有利於長者。吾何愛於髮膚耶。請具言狀。爲長者謀。若吾力能。當效折枝。老人乃邀鍾生入室。泣吿道。老朽非人也。乃狐也。高曾祖父皆學老莊。俱仙去。吾生於唐貞觀丁亥仲秋月圓之夕。幼讀百家書。旣長。有大志。不屑與羣類爭伎倆。思欲立名節於天壤。値武氏亂唐。海內擾攘。恥無賢主可輔\footnote{可憐彼時諸臣宰尚不及一狐狸耳。}。遂棄家入終南。從南華眞人學道。時門下三百餘輩。眞人皆不許以性命眞傳。惟以老朽器度不凡。密授不死之術。一甲子盡其道。至天寶末年。壽百有二十歲。丹始成。即誓願立三千行八百功。以速沖舉。乃遍遊人間。任俠慷慨。推恩市義。所止待老朽舉火者。恆數百戶。歲饑。即入水求沒金敗票以賑。數百來年。身之所至。得活者不下數千百人。凡有急難相吿。識與不識。莫不周濟\footnote{安得此輩千萬。布滿天下。則窮人甚幸矣。}。至於醫藥棺衾。金錢束帛之惠。歲以萬計。未嘗或倦也。因南宋紹定初。豫章有豪惡殘毒一方。以小忿故殺一家八十餘口。僅漏一子。匍匐赴吏。而吏復受賄。欲戕其子。老朽哀其寃。密具千金貢吏始免。旣而豪惡聞之。又欲謀害老朽。因一時忿發。操刀潛殺其一門。以此獲罪於天。功不准過。遂落殺劫\footnote{此老狐救人有如許之功。且害者又是巨惡。尚落殺劫。如流賊殺人無數。其罪云何。}。前夕正當五百年厄運。天將遣雷擊老朽。命在須臾矣。老朽知君品行高潔。必憐庇老朽。故敢乞命耳。鍾生道。諾。然不知何以救長者。老人道。君頭圓目俊。神爽氣豪。而髮與身齊。必心雄膽大。老朽縮骸伏匿君之髮中。君但正冠危坐。雷一擊不中。即撇然長往矣。老朽得逃此劫。再五百歲。多立功德。以償宿愆。則君於老朽有大恩德。焉敢須臾忘報乎。鍾生道。吾哀長者功將成而欲墜。願引手。焉敢望報乎。遂宿旅店中。乃戒門戶。嚴罅〖阝少日小〗。如其言。散髮委地。老人幻形寸許。伏於髮根。鍾生焚香端坐以候。頃之。風雨驟至。雷電交作。繞屋四境。震得牆垣傾動。已而霹靂大震入室。火光繞體。煙焰塞目。須臾雷去。而門闥如故。罅〖阝少日小〗不裂。不知雷從何入。自何出也。鍾生剔燈照髮。已截去大半。意老人必斃。急揭冠呼之。應聲躍出。再拜謝道。老朽無憂矣。受此大恩。今小有所報。遂密傳了鍾生修養運氣之術。囑道。依此行之不倦。雖不能沖舉。當却病延年。久之而爲地仙矣。又把那藥方寫出。付與鍾生。

\begin{quotation}

黑豆一升。去皮 貫仲。一兩 粉草。一兩 白茯苓。五錢 蒼朮。五錢 砂仁。五錢

用水五碗。文火慢熬。及至水盡。去藥。將豆搗如泥。作芡子大。每嚼一丸。恣食苗葉。

\end{quotation}

鍾生深深致謝。老人道。君之恩不能報萬分之一。後晤有期。當宜自愛。迨曉。老人促裝而去。鍾生修合了丸藥。到了雁宕。你道這雁宕在何地方。自臺州府赴永嘉路。出樂淸縣。則雁宕在道左焉。大荆樂淸戍也。去天臺縣百四十里。初到老僧岩。乃雁門戶也。去大荆五六里。可數千尺。偏眉偏袒。絕似老僧。海氣觸山石。侵曉皆成白雲。或橫亘蕩下。遠望之。儼若趺坐狀。行益近。雲氣稍薄。比至岩下。巍立石耳。一肩一項。乃是兩峰。自此林木蓊翳。岩石削立。徑紆\endnotemark[26]壑邃。漸入佳境矣。至石梁洞。洞可容千人坐。石梁環\endnotemark[27]洞門起。長數十丈。扶留女蘿雜綴其上。略如蒼髯老龍飮澗。作攫拿之勢。亦一奇境也。顧向遊天臺之石梁。蜿蜒跨空。飛泉萬丈出其下。游者目搖心悸。多不能度。彼則石梁高架絕頂。重以瀑布增勝。此獨偃蹇岩下。似稍遜耳。澗下南出百步許。折而西行。有謝公嶺。自嶺以東。皆爲雁宕東外谷。踰謝公嶺而西。山石皆盡立。別有天地矣。嶺下有大澗。度危石過澗。羣峰如劍。如槊。如華表。如靈芝。各種奇幻詭怪。不可殫述。石徑出諸峰下。行里許。得古寺。名靈峰。不虛也。寺傍爲靈峰澗。澗外靑天\endnotemark[28]一片。下廣上銳。空明滴翠。驟張目。絕似大野中望見遠山者。尋入寺。作苾蒭之饌。緩步出舊路。憩菱\endnotemark[29]筍峰下。意謂山水奇境。至此觀止也。西靈峰五里而寺者曰淨名精舍。\endnotemark[30]頗不俗。有老僧居焉。精舍在谷中。數過絕澗。始至門前。有地寬平\endnotemark[31]百畝。果木樹皆成行列。其後軒面石壁。如百尺牆。牆下雜植花竹。條葉鮮麗。長如春時。堦前列藥爐茶臼。架上多名人手蹟。皆題咏甌越諸山者。卷帙各精緻有法。兀坐斗室中檢閱移時。令人有超然之想。僧徐言靈岩佳處。鍾生問。何如靈峰。僧笑道。過之。興致躍躍。別僧去。鍾生暗想道。前老人謂雁宕實勝天臺。初余未到雁宕。不能定其優劣。比之靈岩。歎老人之言不虛。靈岩有寺。廢久矣。而羣峰益刻露呈秀。固知天地自然之奇。非斧鑿所能出。稍一點綴。反掩眞色耳。寺基負石屛峰。峰高揷天。左有峰曰展\endnotemark[32]旗。右有峰曰天柱。高與石屛等。天柱後爲玉女峰。兩峰之間別有小峰二。土人呼爲僧拜石。頗肖。鍾生坐廢寺柱礎上。歷數諸峰。尋由石屛後小嶺上盤折行千步。至龍鼻洞。龍鼻水出焉。洞視石梁更隘。而險倍靈峰。獨秀卓筆兩峰在其下。洞之勝至靈峰而止。峰之勝至\endnotemark[33]靈岩而止。瀑布之勝至\endnotemark[34]大龍湫而止。自大荆凡行四十餘里。日晡至馬鞍嶺。\endnotemark[35]徐行至嶺上。望觀音諸峰。旣度嶺。欲投羅漢寺宿。未至寺六七里。遇寺僧。詢路。僧指路傍谷道。從此而入。爲大龍湫。明日可一往也。鍾生因念明日至龍湫。則當自寺中却行十餘里。往復甚費。\endnotemark[36]遂入谷。緣澗行。\endnotemark[37]水聲潺湲。遙見一峰聳出。㟏\endnotemark[38]岈其端。則是剪刀峰矣。南行又里餘。徑始絕。仰視石岩。高數千丈。下臨絕谷。谷中皆磊砢大石。龍湫水直從岩頂飛下。空中散落如雨。激亂石磳〖石厷〗作聲。初冬久\endnotemark[39]旱。始至時。水勢頗緩。有頃。忽大至。橫流倒瀉。如決潰川。豈山靈有知耶。風聲颼颼。吹雨過隔潭。直至岩下。諦視。則岩端出石脚反數十丈。故水直下如建瓶。\endnotemark[40]立未定。鬚髮已盡濕。不覺大笑。爲水聲所抑。不聞也。谷中多石菖蒲。着水尤鮮潔可愛。詎那庵瑞鹿院皆僅存餘址而已。先是靈岩卓筆峰\endnotemark[41]下。亦有龍湫瀑布。僅長三百餘尺。故有大小之別。坐龍湫上。不覺日晚。自龍湫出里許。谷中有小嶺甚銳。即寺後山也。過此便可直達僧廚下。不必出谷行矣。日暮道遠。鼓餘勇凌轢而上。初不知嶺之銳。至嶺脊俯視。則削如堵。寺中炊\endnotemark[42]煙一縷。從牆脚出。寺後樹高百尺。皆負牆而立。微茫\endnotemark[43]有小徑可下。則松葉塡集不可辨。遙見寺僧直下。如履平地。膽若稍壯。然每一措足。惴然如履春冰。扳藤附葛而下。卒無恙。鍾生喟然嘆道。天下事。每失於不及持。而成於多畏。故馳康莊則馬逸。飽怒帆則舟覆。無所畏也。世路險巇。\endnotemark[44]時時如行此嶺。當無患巓蹶矣。寺之四面皆高山。夜坐望東北上。僅見斗柄。問僧雁宕在何處。不知也。但言相傳靈岩絕頂有大湖。雁過南海。常棲止其中。故名雁宕。水流出谷。爲大龍湫。蓋不可至矣。次日就路。破岩出竹。\endnotemark[45]踏霜葉簌簌有聲。二里許。至能仁寺。亦久廢。有大鑊。容四百斛。置榛莽中。是宋時物也。又西行爲丹芳\endnotemark[46]嶺。甚高峻。凡四十九盤而下。\endnotemark[47]山勢始開拓。大小芙蓉山在焉。自靈岩以東爲雁宕東谷。自靈岩以西爲雁宕西谷。能仁至丹芳則西外谷也。鍾生賞玩了數日。初意欲住深山之中。恐米糧難以措辦。因老僧岩離鄕村路近。於僻處樹了一間茅屋靜養。行那老人傳授的工夫。頗有所得。間或饔飱不繼。試嚼藥丸以啖草果木葉。亦不覺苦澀。住了二三載。以爲此地決無人識。可以久居。不想被金德性識認。恐他次日復來。那晚就不知避到何處去了。自此以後。總不知他下落。眞是見其首而不見其尾。確是英雄作用。但他這樣一個盛德君子。我雖不敢效小說家說他成仙了道的俗套。大約自然也壽享遐齡。做一個出世的高人去了。再說鍾生二子俱已成立。皆能紹讀(續)書香。長子鍾文娶了梅生之女。次子鍾武娶了宦蕚之女。子孫連綿不絕。鍾自新也生了三子。此時有七十餘歲。與到聽同時的人知道鍾生宦蕚賈文物童自大四人夫妻事蹟的。與到聽昔日之言相符。方信向日到聽所說古城隍廟話非謊。後來鄂氏也活到七旬之外。錢貴與代目俱享高壽。見了四代重孫。方纔老故。予固知此事鑿鑿。故著成一帙。以娛觀者之目。但信之者少。非之者衆。故不得不爲之妄言也。予尚有八句。實不成詩。亦名之曰妄言。不過因此一部妄言之後。持續此數句。以證此妄字耳。

\begin{quotation}

爲報諸公識我麼。我心惟祇與天那。

醒觀世俗傷心重。醉著新編入意多。

興到高談劉子論。悶來豪放寗生歌。

妄言一任他人議。且自優游安樂窩。

\end{quotation}



\endnotetext[1]{「欲」字原置「流」字之上,據文義改。}

\endnotetext[2]{「千把總」原作「總千把」,據文義改。}

\endnotetext[3]{批註「憶起」原作「起憶」,據文義改。}

\endnotetext[4]{「陰」原作「殷」,據上文改;下文或同,不贅。}

\endnotetext[5]{此段原有眉批「此謂禍不足報以福報」九字。}

\endnotetext[6]{此段原有眉批「巧寃巧報毫髮不紊」八字。}

\endnotetext[7]{「似」原作「是」,據陳鼎《留溪外傳》卷一《許義士傳》改。}

\endnotetext[8]{「言距」原作一「詎」字,據陳鼎《留溪外傳》卷一《許義士傳》加改。}

\endnotetext[9]{「裡」字原無,據陳鼎《留溪外傳》卷一《許義士傳》加。}

\endnotetext[10]{「決」字原作「絕」,據陳鼎《留溪外傳》卷一《許義士傳》改。}

\endnotetext[11]{「八」原作「死」,「死」原作「入」,據陳鼎《留溪外傳》卷一《許義士傳》改。}

\endnotetext[12]{「其」下原衍一「節」字,據陳鼎《留溪外傳》卷二《髯樵叟傳》刪。}

\endnotetext[13]{「換」原作「喚」,據陳鼎《留溪外傳》卷二《髯樵叟傳》改。}

\endnotetext[14]{「幟」原作「熾」,據陳鼎《留溪外傳》卷十八《二雪和尚傳》改。}

\endnotetext[15]{「人」字原無,據陳鼎《留溪外傳》卷十八《二雪和尚傳》加。}

\endnotetext[16]{「晉」原作「留」,據陳鼎《留溪外傳》卷十八《二雪和尚傳》改。}

\endnotetext[17]{批註「小狗子」原作「小苟子」,據上文改。}

\endnotetext[18]{「宕」原作「岩」,據文義改;下文或同,不贅。}

\endnotetext[19]{「之」原作「義」,據陳鼎《留溪外傳》卷五《高士列傳》改。}

\endnotetext[20]{「氣」原作「義」,據陳鼎《留溪外傳》卷五《萬履庵傳》改。}

\endnotetext[21]{「秉」原作「炳」,據陳鼎《留溪外傳》卷五《孫先生傳》改。}

\endnotetext[22]{「控」原作「扣」,據陳鼎《留溪外傳》卷五《孫先生傳》改。}

\endnotetext[23]{「漸」原作「晰」,據陳鼎《留溪外傳》卷五《孫先生傳》改。}

\endnotetext[24]{「明」原作「炳」,據陳鼎《留溪外傳》卷五《孫先生傳》改。}

\endnotetext[25]{「達」字原無,據陳鼎《留溪外傳》卷五《孫先生傳》加。}

\endnotetext[26]{「紆」原作「行」,據周淸原《遊雁蕩山記》改。}

\endnotetext[27]{「環」原作「還」,據周淸原《遊雁蕩山記》改。}

\endnotetext[28]{「天」原作「石」,據周淸原《遊雁蕩山記》改。}

\endnotetext[29]{「菱」下原衍一「峰」字,據周淸原《遊雁蕩山記》刪。}

\endnotetext[30]{「西」原作「兩」,「者」原作「名」,「名」原作「寺」,據周淸原《遊雁蕩山記》改。}

\endnotetext[31]{「平」原作「半」,據周淸原《遊雁蕩山記》改。}

\endnotetext[32]{「展」原作「辰」,據周淸原《遊雁蕩山記》改。}

\endnotetext[33]{「至」字原無,據周淸原《遊雁蕩山記》加。}

\endnotetext[34]{「瀑布之勝至」五字原無,據周淸原《遊雁蕩山記》加。}

\endnotetext[35]{「嶺」原作「山」,據周淸原《遊雁蕩山記》改。}

\endnotetext[36]{「費」原作「廢」,據周淸原《遊雁蕩山記》改。}

\endnotetext[37]{「行」原作「谷」,據周淸原《遊雁蕩山記》改。}

\endnotetext[38]{「㟏」原作「嵢」,據周淸原《遊雁蕩山記》改。}

\endnotetext[39]{「久」原作「火」,據周淸原《遊雁蕩山記》改。}

\endnotetext[40]{「瓶」原作「平」,據周淸原《遊雁蕩山記》改。}

\endnotetext[41]{「卓筆峰」原作「卓峰筆」,據周淸原《遊雁蕩山記》改。}

\endnotetext[42]{「炊」原作「吹」,據周淸原《遊雁蕩山記》改。}

\endnotetext[43]{「茫」原作「芒」,據周淸原《遊雁蕩山記》改。}

\endnotetext[44]{「巇」原作「戲」,據周淸原《遊雁蕩山記》改。}

\endnotetext[45]{「竹」原作「行」,據周淸原《遊雁蕩山記》改。}

\endnotetext[46]{「丹芳」原作「丹房」,據周淸原《遊雁蕩山記》改,下同。}

\endnotetext[47]{「下」原作「上」,據周淸原《遊雁蕩山記》改。}

\theendnotes

\bookmarksetup{startatroot}

\chapter*{後記 《姑妄言》小說抄本之發現}
\addcontentsline{toc}{chapter}{後記 《姑妄言》小說抄本之發現}
\markboth{後記 《姑妄言》小說抄本之發現}{後記 《姑妄言》小說抄本之發現}

一九六三年起,筆者開始調查蘇聯所藏中國章回小說及俗文學作品版本。這個調查工作在列寧格勒(現名聖彼得堡)開始,第一天發現了東方硏究所所藏《石頭記》八十回抄本,即著名的「列藏本」《石頭記》。此外,筆者還發現了不少俗文學目錄未著錄的俗文學作品(鼓詞、彈詞、子弟書、大鼓書等)以及孫楷第《中國通俗小說書目》未著錄的章回小說的版本。

回到莫斯科,筆者開始調查莫斯科圖書館所藏漢文古籍。一九六四年有一天,在列寧圖書館(蘇聯最大的國家圖書館)抄本部門,又意外發現了K.I. Skachkov(一八二一—一八八三)收藏的《姑妄言》小說抄本。

關於Skachkov及其藏書,有必要先做個介紹。Skachkov一八四四年由黑海Odessa城Ri-chelieu學院的物理數學系畢業,學天文同時也學農業,所以一八四八年俄羅斯派他到北京,任務是在北京東正敎館設天文台。到北京之後,Skachkov馬上開始學中文並搜集中國書籍,有的是他自己買的,有的是中國朋友、學者、官員或駐北京俄羅斯神父送給他的,有些與天文有關的善本甚至是皇帝(疑是道光)叔伯所贈。

Skachkov對書籍的興趣廣泛,他買天文、地理、水理著作,也購買文學、宗敎、歷史、經濟、語言、哲學、民族學等各種的書,也特別注意各種歷史地圖,如宋代畫的西夏圖,或淸代各種地圖,如十八世紀的湖北圖、嘉定府圖、台灣圖及較仔細的早期的台南圖等等。另外,他還購買了一些有名的藏書家的書,如一八四八年去世的徐松藏的書及舊抄本(都有徐松的藏書章),和姚文田、姚元之的舊藏。

Skachkov的中國書,大部分是他一八四八—一八五九年在北京搜集的。回俄國後,外交部又派他到新疆當駐塔爾巴哈台的領使。到新疆時,他特別注意當地的歷史資料,又購買不少書與抄本,他收藏有三十四種新疆歷史抄本(其中多半從未刊行),和十一種未刊行的新疆地圖。

一八六三年,Skachkov又回到俄國。他曾想把他收藏的中文書賣給敎育部,但敎育部不買;問科學院亞洲博物館(當時俄國唯一的硏究東方文化的機構),雖然最有名的漢學家V.P. Vasilev寫信給科學院,證明Skachkov的收藏非常寶貴,但科學院因爲經費不夠,沒有購買。

Skachkov在中國一共搜集了一千五百多種書與抄本,大槪家裡沒有地方放,所以一八六七年他把書籍交給聖彼得堡皇家公共圖書館臨時收藏。過了六年(一八七三年),西伯利亞Irkutsk(伊爾庫次克)城大商人A. Rodionov(與中國做貿易,在漢口買茶葉)表示,如果政府給他一個勳章,他同意付錢購買Skachkov的收藏,贈給莫斯科Rumjantsev博物館——其圖書館卽是列寧圖書館的前身,一九九〇年代又改名爲俄羅斯國家圖書館。

一八七三年Skachkov的中文書正式入藏圖書館以後,好多年都沒有人整編目錄(只有Ska-chkov自己的一些卡片),因此日本漢學家羽田亨博士、法國漢學家P. Pelliot(伯希和)敎授,先後於一九一四年、一九二五年到莫斯科看過Skachkov的藏書,但似乎都不曾注意《姑妄言》這個抄本,他們都是歷史學家,注意Skachkov收藏歷史資料,如羽田亨敎授硏究元代史,利用Skachkov收藏從「永樂大典」抄的資料,Pelliot敎授在荷蘭出版漢學期刊T'oung pao(通報)一九三二年二十九卷發表了一文,專門介紹Skachkov收藏的一些歷史手抄本。

一九三七年,列寧圖書館邀請列寧格勒冬宮博物館的漢學家V.N. Kozin來莫斯科整理Skachkov的收藏,他改正不少Skachkov自己寫的目錄,但因第二次世界戰爭開始,整理工作只好停止,而Kozin在列寧格勒圍城時也犧牲了。又過了二十多年,圖書館邀請老漢學家A.I. Melnalknis先生於業餘之時到館整理Skachkov收藏的舊抄本並編纂書錄(他在東方硏究所許多年,參加編纂四卷本的《華俄大辭典》)。

一九六四年,筆者來到列寧格勒圖書館抄本部門看看那裡藏的中文抄本,Melnalknis先生知道筆者硏究中國文學,從抄本書庫拿出來幾個文學作品抄本,並說他自己不是硏究文學的,不大知道是什麼作品。筆者打開一個較大的紙盒子,裡面放的正是二十四册的《姑妄言》小說抄本。Skachkov大量搜集各種文學作品,小說方面除了著名的四大奇書之外,還有一些較罕見的作品,有的版本在孫楷第《中國通俗小說書目》及大塚秀高編的《增補中國通俗小說書目》未著錄,如三槐堂本《繡像飛龍全傳》、孔耕書屋本《增訂精忠演義》等,或海外較少見的《三分夢全傳》(道光十五年版)、《蓮子瓶全傳》(道光二十二年版)、《海公大紅袍全傳》(道光十三年版)、《娛目醒心編》(咸豐二年刊)等等,可見一八四八年到北京的Skachkov大多買了道光時期小說版本,他企圖較全面的搜集各種小說,所以得到《姑妄言》抄本大槪也不是偶然的(其他小說都是刻本)。

《姑妄言》是章回小說,作者爲三韓曹去晶,有一七三〇年(雍正八年)自序,林鈍翁總評,分二十四卷。筆者當時查孫楷第的《中國通俗小說書目》和其他書,都未見著錄。與孫楷第敎授寫信時,提到這本書,他回答說從未見過,並懷疑它是韓國人用中文寫的作品。其實「三韓」是中國的一個縣名,淸代屬熱河省,《姑妄言》作者定是三韓縣的漢族人。可惜筆者許多年都查不到關於曹去晶和《姑妄言》的材料。

一九六六年,筆者於《亞非民族》發表一篇長文〈中國文學各種目錄補遺〉,補充孫楷第《中國通俗小說書目》及各種俗文學目錄,第一次著錄了在列寧圖書館發現的《姑妄言》手抄本。

過了八年,一九七四年莫斯科東方文學出版社出版A.I. Melnalknis先生編的《Skachkov所藏中國手抄本與地圖書錄》一書,仔細記錄Skachkov收藏的抄本及手繪的地圖、風俗畫三三三種。其中N245著錄《姑妄言》,注明抄本是幾個人抄的,有人寫楷書,有人寫行書。第二卷、第二十一卷有中國收藏家之圖章。用的紙是「仁美和記」和「仁利和記」兩個紙廠的。每册他都數有多少葉,也注明缺哪一葉,如第八册缺十七—十八葉,哪一葉撕掉一塊等等。可惜Melnal-knis先生編的目錄很少人注意,蘇聯用的人很少,國外漢學家及中國學者大槪完全沒有注意。

又過了十年,一九八四年日本大塚秀高敎授編印《中國通俗小說改訂稿》,記錄《姑妄言》只寫?卷?回,周越然舊藏;一九八七年增補時,著錄的仍是周越然舊藏的「素紙精抄本,存第四十至四十二回」。這個殘抄本不知去向,但一九九〇年北京吳曉鈴敎授、法國陳慶浩敎授都告訴我上海優生學會有它的鉛印本;陳慶浩敎授則早已從筆者一九六六年發表的文章,得知筆者在莫斯科發現了《姑妄言》較完整的舊抄本。一九九〇年中國文聯出版公司出版的《中國通俗小說總目提要》據周越然〈孤本小說十種〉著錄了上海優生學會鉛印殘本,但未見該書;一九九三年北京出版的《中國古代小說百科全書》才首度介紹該殘本的內容和居士山人序的大意,疑是明末淸初作品,並說:「淸代禁書諸目及諸家藏書目均未著錄,故無法確考其成書時代及作者。」(中國大百科全書出版社,頁一〇七)他們也聽說「前蘇聯藏此書之全帙,抄本二十四册」,但因未見一九六六年拙著,不知莫斯科所藏抄本有作者曹去晶的名字,也有一七三〇年的作者自序,它肯定不是明末淸初之作,而是雍正時期的小說。

一九八九年至一九九一年間,筆者在北京與劉世德敎授、法國陳慶浩敎授討論過《姑妄言》的影印。一九九二年到台灣敎書,淸華大學王秋桂敎授也提到出版《姑妄言》的問題。一九九三年俄羅斯國家圖書館館長I.S. Filippov敎授到台灣參加中央圖書館館慶,王秋桂敎授和筆者與館長趁此機會,終於談好在《思無邪滙寶》出版《姑妄言》的排印本。

距離俄國Skachkov在北京開始大量搜購中國古書及舊抄本約一百五十年後的現在,他所收藏的《姑妄言》小說抄本才第一次要在中國問世,這也是台灣與俄羅斯第一次合作出一本書。希望將來中國(大陸、台灣)學者及各國漢學家多注意Skachkov的藏書,以及其他俄羅斯所藏中國善本、抄本和稀見的刻本,繼續合作出版。

這次《姑妄言》小說重新問世,特別是陳慶浩、王秋桂、陳益源三位敎授的功勞,以俄羅斯學界之名要感謝他們。

\begin{flushright}

俄羅斯科學院通訊院士・李福淸(B. Riftin)

\end{flushright}
